\sethandouttitle{The Mean Value Theorem}


\begin{enumerate}


	\item 	Verify that the function satisfies the three hypotheses of Rolle's Theorem on the given interval.
		Then find all numbers $c$ that satisfy the conclusion of Rolle's Theorem.
		\begin{enumerate}
			\item	$f(x) = 5 - 12x + 3x^2$ on $[1,3]$.
			\item	$f(x) = \sqrt{x} - \frac{1}{3}x$ on $[0,9]$.
		\end{enumerate}

	\item	Let $f(x) = 1 - x^{2/3}$. Show that $f(-1)=f(1)$ but there is no number $c$ in $(-1,1)$ such that $f'(c)=0$.
		Why does this not contradict Rolle's Theorem?

	\item 	Verify that the function satisfies the hypotheses of the Mean Value Theorem on the given interval.
		Then find all numbers $c$ that satisfy the conclusion of the Mean Value Theorem.
		\begin{enumerate}
			\item	$f(x) = 2x^2 - 3x +1$ on $[0,2]$.
			\item	$f(x) = \ln(x)$ on $[1,4]$.
		\end{enumerate}

	\item	Let $f(x) = 2 - |2x-1|$. Show that there is no value of $c$ such that
		\[
			f(3)-f(0) = f'(c)(3-0).
		\]
		Why does this not contradict the Mean Value Theorem?

	\item	Show that the equation $2x + \cos(x)$ has exactly one real root.

	\item	If $f(1)=10$ and $f'(2) \geq 2$ for $1 \leq x \leq 4$, how small can $f(4)$ possibly be?

	\item	Does there exist a function $f$ such that $f(0) = -1$, $f(2) = 4$ and $f'(x) \leq 2$?


	\item	Prove that
		\[
			\arcsin \left( \frac{x-1}{x+1} \right) = 2 \arctan \sqrt{x} - \frac{\pi}{2}.
		\]



\end{enumerate} 




