\documentclass[10pt,oneside,letterpaper]{article}

\usepackage{studenthandouts}



\title{Sample Student Handouts output}
\author{James Fennell}

\begin{document}


\maketitle


\noindent
This is a sample collection of student worksheets compiled using the Student Handouts package.
	In this sample project there are seven underlying handouts -- 1.1, 1.2, 1.3, 2.1, 2.2, 3.1, and 3.2 -- 
	but not all are compiled initially in order to illustrate the functionality of the handout import management commands.

Some of the handouts contain the output of \verb$\allhandoutinfo$.
	This is to show you the specific handout information that can be used when styling the handouts.
This information can also be used in the handouts themselves, as here.

If you wish to use the package, one option would be to copy the sample code over and use that as the basis for your project.
	However, as you will see from looking at the sample code itself, very little is required to get a handouts project up and running from scratch.

\vspace{\baselineskip}
\noindent
Regarding formatting: the sample handouts here illustrate two possibilities in terms of the function of a handout.
	The handouts in unit 1 are meant to be written on by students, whereas the handouts in units 2 and 3 are not. 
	The unit 1 handouts do this by judicious use of the \verb$\vfill$ command which spreads the questions evenly down the page.
	Inspecting the source will make it clear how this is done.




\tableofcontents



\renewcommand{\thehandoutsdirectory}{worksheets/}
\renewcommand{\thehandoutscredit}{NYU Calculus 1 Summer 2015}

\setunittitle{1}{Introduction to Differentiation}
\setunittitle{3}{Advanced Differentiation}


\importhandout{1}{1}
\importhandout{1}{2}
\importhandout{1}{3}
\importhandout{2}{1}


% Some random handout import management instructions.
% Note that 2.2 will not be imported. Even though 2.2 is in the list of admissible handouts, its unit is not in the list of admissible units.
\importonlyunits{3}
\importonlyhandouts{2.2,3.2}


\importhandout{2}{2}
\importhandout{3}{1}
\importhandout{3}{2}




\end{document}





