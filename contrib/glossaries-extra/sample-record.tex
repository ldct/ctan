% arara: pdflatex
% arara: pdflatex
\documentclass{article}

\usepackage[colorlinks]{hyperref}
\usepackage[record,style=index]{glossaries-extra}


% Use \loadglsentries (or \input) for .tex files:
% \loadglsentries{example-glossaries-brief}

% Use \glsxtrresourcefile or \GlsXtrLoadResources for .glstex files:

%\GlsXtrLoadResources % loads \jobname.glstex
\glsxtrresourcefile{sample-resource}% loads sample-resource.glstex

\GlsXtrRecordCounter{section}

\begin{document}

\section{Sample Section}
\printunsrtglossaryunit{section}
% or:
%\printunsrtglossary*{\printunsrtglossaryunitsetup{section}%
%  \renewcommand*{\glossarysection}[2][]{\subsection*{Summary}}%
%}

\gls{dolor}, \gls{amet}.

See \url{https://github.com/nlct/bib2gls} for converting
.bib to .glstex files.

\section{Another Section}
\printunsrtglossaryunit{section}

\glspl{lorem}, \gls{amet}.

\printunsrtglossary

\end{document}
