% !TEX encoding = UTF-8 Unicode
% !TEX TS-program = LuaLaTeX
%
% toptesi-it.tex
%% Copyright 2013-2016 Claudio Beccari
%
% This work may be distributed and/or modified under the
% conditions of the LaTeX Project Public License, either version 1.3
% of this license or (at your option) any later version.
% The latest version of this license is in
% http://www.latex-project.org/lppl.txt
% and version 1.3 or later is part of all distributions of LaTeX
% version 2003/12/01 or later.
%
% This work has the LPPL maintenance status "author-maintained".
%
%
% Version 1.3 of the LaTeX Project Public License is included in the
% appendix of this documentation, but the primary source remains in
% http://www.latex-project.org/
%
\begin{filecontents*}{\jobname.xmpdata}
\Title{La classe TOPtesi}
\Author{Claudio Beccari}
\Publisher{Claudio Beccari}
\Keywords{Monografia di laurea\sep 
Tesi di laurea\sep 
Tesi di dottorato\sep 
classe LaTeX\sep 
pdfLaTeX\sep 
XeLaTeX\sep 
LuaLaTeX}
\end{filecontents*}
%
\documentclass[12pt,twoside]{toptesi}
\ProvidesFile{toptesi-it.tex}[2016/11/10 v.0.9.15]
\ifPDFTeX
%
  \usepackage[a-1b]{pdfx}
%
    \usepackage[utf8]{inputenc}
    \usepackage[T1]{fontenc}
    \usepackage{newtxtext,newtxmath,textcomp,textalpha}
    \usepackage{amsmath,amssymb}
    \usepackage{mflogo}
    \usepackage{guit}
    \setactivedoublequote
\else
    \ifLuaTeX
      \usepackage[a-1b]{pdfx}
    \fi
    \usepackage{fontspec}
    \usepackage{xcolor}
    \newcommand*\MP{{\setbox0\hbox{M}\relax
    \includegraphics[height=\ht0]{MPlogo}}}
    \newcommand*\GuIT{{\setbox0\hbox{Hg}\relax
    \raisebox{-\dp0}{\includegraphics[height=\dimexpr\ht0+\dp0]{GuITlogo}}}}
    \setmainfont[Ligatures=TeX]{TeX Gyre Termes}
    \setsansfont[Ligatures=TeX, Scale=MatchLowercase]{TeX Gyre Heros}
    \setmonofont{UM Typewriter}
    \setmainlanguage[babelshorthands]{italian}
    \usepackage{amsmath}
    \usepackage{unicode-math}
    \setmathfont{XITS Math}
    \setotherlanguage[variant=ancient]{greek}
    \newfontfamily{\greekfont}{GFS Didot}
\fi
\usepackage{metalogo,longtable,booktabs,array,tabularx,fancyvrb,
enumitem,xspace,fancyvrb,ragged2e,siunitx}
%\usepackage[margin=15pt,font=small,labelfont=bf,labelsep=quad]{caption}
%\captionsetup[table]{position=above}
\usepackage{trace}
\usepackage{imakeidx}
\usepackage{hyperref}
    \hypersetup{%
        pdfpagemode={UseOutlines},
        bookmarksopen,
        pdfstartview={FitH},
        colorlinks,
        linkcolor={blue},
        citecolor={blue},           
        urlcolor={blue}
    }

\makeindex[intoc,columns=2]

\newcommand*\prog[1]{\textsf{#1}\index{programma!#1@\textsf{#1}}}
\newcommand*\pack[1]{\textsf{\slshape#1}\index{pacchetto!#1@\textsf{\slshape#1}}}
\newcommand*\class[1]{\textsf{\slshape#1}\index{classe!#1@\textsf{\slshape#1}}}
\newcommand*\amb[1]{\textsf{\slshape#1}\index{ambiente!#1@\textsf{\slshape#1}}}
\let\env\amb
\newcommand*\meta[1]{$\langle${\normalfont\textit{#1}}$\rangle$}
\newcommand*\file[1]{\texttt{#1}\index{file!#1@\texttt{#1}}}
\providecommand*\cs[1]{\texttt{\char92#1}\index{#1@\texttt{\char92#1}}}
\newcommand*\oarg[1]{\texttt{[}\meta{#1}\texttt{]}}
\newcommand*\marg[1]{\texttt{\{}\meta{#1}\texttt{\}}}
\newcommand*\Arg[1]{\texttt{\{#1\}}}\let\Marg\Arg
\newcommand*\Oarg[1]{\texttt{[#1]}}
\newcommand*\opt[1]{\texttt{#1}\index{opzione!#1@\texttt{#1}}}\let\opz\opt
\newcommand*\Font[1]{\textit{#1}\index{font!#1@\textit{#1}}}
\let\acro\textsc

\providecommand*\pdfLaTeX{}
\renewcommand*\pdfLaTeX{pdf\/\LaTeX\xspace}
\renewcommand*\XeLaTeX{%
\ifPDFTeX X\kern-0.14em\raisebox{0.975ex}{\scalebox{-1}[-1]{E}}\kern-0.075em\LaTeX
\else\Xe\LaTeX\fi\xspace}
\let\originalLuaLaTeX\LuaLaTeX
\renewcommand*\LuaLaTeX{\originalLuaLaTeX\xspace}

\providecommand*\MiKTeX{MiK\TeX\xspace}
\providecommand*\TeXLive{\TeX\,Live\xspace}

\makeatletter
\def\GetFileInfo#1{%
  \def\filename{#1}%
  \edef\@tempa{\csname ver@#1\endcsname}%
  \def\@tempb##1 ##2 ##3\relax##4\relax{%
    \def\filedate{##1}%
    \def\fileversion{##2}%
    \def\fileinfo{##3}}%
  \expandafter\@tempb\@tempa\relax? ? \relax\relax}
  
\def\bottomfraction{1}

\begin{document}
%
\GetFileInfo{toptesi.cls} 
%
 \sedutadilaurea{Classe \textsf{toptesi} versione \fileversion\ del \filedate}


\thispagestyle{empty}
\vspace*{.2\textheight}
\hrule
\begin{center}
\Huge \textsf{La classe TOPtesi}
\end{center}
\hrule

\clearpage
\ifbool{@twoside}{\thispagestyle{empty}\null\clearpage}{}




\frontespizio

\sommario
 Questo testo serve per descrivere come comporre tipograficamente la  tesi di laurea o la monografia o la dissertazione di dottorato  mediante il noto programma di composizione \LaTeX, o meglio, mediante le sue varianti \pdfLaTeX, \XeLaTeX\ o \LuaLaTeX; per produrre con \XeLaTeX\ il file finale in formato PDF archiviabile secondo la norma ISO \mbox{19005-1} bisogna procedere  come descritto nel paragrafo~\ref{sec:XePDFA}.

\english
\sommario
This text describes how to typeset a university master thesis, or the bachelor final report, or the PhD dissertation through the well known typesetting program \LaTeX, or rather through its variants \pdfLaTeX, \XeLaTeX, or \LuaLaTeX; in order to produce the final document in a PDF archivable format according to the ISO regulation \mbox{19005-1} it's necessary to proceed as described in section~\ref{sec:XePDFA}.


\italiano
\ringraziamenti

Ringrazio gli studenti del Politecnico di Torino che mi hanno sollecitato a mettere la mia esperienza a loro disposizione per predisporre e rendere disponibile il software necessario per preparare le loro tesi, monografie o dissertazioni con la qualità che solo \pdfLaTeX, \XeLaTeX\ o \LuaLaTeX\ riescono a produrre\footnote{Questa particolare composizione è stata eseguita con \XeLaTeX. Questo motore di composizione presenta numerosi vantaggi su \prog{pdflatex} per quel che riguarda l'uso dei font, anche per la matematica, ma non è ancora (2015) in grado di produrre l'uscita direttamente nel formato PDF, anche se apparentemente lo fa. In realtà la sua uscita è in un formato intermedio che viene poi trasformato automaticamente in PDF. L'esperienza mi insegna che in realtà le limitazioni di \XeLaTeX\ sono pochissime e, con la classe \pack{toptesi}, quella forse più importante, ma facilmente risolvibile, riguarda il formato PDF archiviabile che non è ottenibile direttamente, tanto che bisogna procedere come indicato nel paragrafo~\ref{sec:XePDFA}.}.

\figurespagetrue
\tablespagetrue
\indici


\chapter{Guida rapida all'uso di questo manuale}
Nella pagina~\pageref{warn:avvertenza} c'è scritto in rosso quanto segue.
\begin{quote}{\color{red}Dopo avere letto un po' di documentazione e aver giocato un poco  con i programmi già predisposti sia dalla distribuzione del sistema \TeX\ sia dai vari editor ASCII, siete in grado di capire come funziona il tutto.

Attenzione: non si minimizzi la frase precedente: la documentazione va letta sempre e capita fino in fondo; se c'è qualcosa che non capite, provate ad esercitarvi con qualche piccolo esercizio, visto che la pratica permette di capire la teoria (e viceversa), ma non andate a cercare in rete aiuto nei vari forum dedicati a \LaTeX\ per farvi dire: ``guarda nella pagina tal-dei-tali della documentazione''; per voi sarebbe una vera umiliazione\dots}
\end{quote}

Detto in altre parole, non cominciate nemmeno a leggere questo manuale e non usate TOPtesi se non avete ancora un minimo di conoscenza di \LaTeX. Non è solo una questione di esperienza pratica; è anche una questione di linguaggio; se non si conosce la terminologia, non si capisce nemmeno quello che si legge. Faccio solo un paio di esempi che mostrano l'ambiguità di certi termini, dei quali bisogna conoscere il significati e bisogna saperli distinguere in base al contesto.
\begin{description}
\item[Pacchetto] In inglese esistono due termini usati nel sistema \TeX: \emph{bundle} e \emph{package}: il primo termine si riferisce ad una collezione di diversi file, per lo più di tipo \emph{package}, ma non solo; i file di tipo \emph{package} sono delle collezioni di definizioni, le cosiddette macro. In italiano i due termini vengono abitualmente tradotti entrambi con il nome \emph{pacchetto}, per cui la parola italiana è ambigua.
\item[Formato] Nel mondo \TeX\ la parola \emph{formato}, in inglese \emph{format}, ha almeno tre significati; i principali significati sono i seguenti.
\begin{description}
\item[Forma del mark up] Con questo significato ci si riferisce al file con estensione \file{.fmt} che contiene la traduzione in linguaggio macchina dell'insieme di macro che definiscono il mark up specifico del linguaggio usato; qui potrebbero interessare i file di formato \file{pdflatex.fmt}, \file{xelatex.fmt}, \file{lualatex,fmt}; ne esistono diversi altri.
\item[Forma della pagina del testo composto] Le varie carte disponibili vengono vendute in diversi formati; per esempio A4 (210\unit{mm} per 297\unit{mm}), B5 (176\unit{mm} per 210\unit{mm}), eccetera. Si può usare questa parola anche per riferirsi alla forma della gabbia del testo  e al layout della pagina composta tipograficamente.
\item[Codifica di registrazione delle immagini] La parola formato viene usata per specificare in realtà il modo di codificare una immagine, e si parla del formato JPEG, del formato PDF, eccetera. Esistono formati vettoriali e formati raster (o a matrici di punti, o bitmapped); i formati raster possono essere \emph{lossless} oppure \emph{lossy}; in questo caso la perdita, da cui il prefisso ``loss'', si riferisce al modo di comprimere l'informazione dell'immagine: la compressione senza perdita permette di recuperare esattamente l'immagine di partenza, mentre la compressione con perdita  permette di comprimere di più ma a spese di una approssimazione nel recupero dell'immagine di partenza; il formato PNG è di tipo lossless; il formato JPG (o JPEG) è lossy.
\end{description}
\end{description}


\section{A cosa serve TOPtesi}

Serve per comporre una tesi, sia essa la \emph{monografia} (detta anche \emph{elaborato finale}) preparata alla fine della laurea triennale, o la \emph{tesi di laurea} predisposta alla fine della laurea magistrale o della laurea a ciclo unico, sia essa la \emph{dissertazione di dottorato}. Sia essa da scrivere in italiano o un un'altra lingua.

La tesi, in realtà, non è altro che il rapporto relativo allo studio, alla ricerca, alle sperimentazioni, al progetto, svolti come lavoro conclusivo di un periodo di studi universitari. Il contenuto della tesi non differisce sostanzialmente da qualunque altro rapporto scritto in una qualsiasi disciplina su cui verta principalmente la tesi.

Questa classe è già stata usata per comporre tesi di vario livello in ingegneria, in matematica, in fisica, in economia, in filologia greca, in filologia copta, in medicina, eccetera. Non è quindi destinata solo ai rapporti finali degli studi di ingegneria.

La ``tesi'' differisce da un generico rapporto, perché ha un valore legale come elaborato da presentare all'esame finale per un ciclo di studi superiori; deve quindi avere certi requisiti che permettano di soddisfare le richieste di tipo burocratico di ogni ateneo.

Questi requisiti riguardano principalmente il frontespizio; quindi in questo manuale si dedicherà molto spazio alla predisposizione del frontespizio da comporre con vari stili, e in diverse lingue.

TOPtesi serve anche per fornire alcune altre semplici estensioni che rendono più agevole la redazione del contenuto della tesi, ma in fondo non è questa la parte più importante di TOPtesi.

Come tutti i pacchetti che estendono la funzionalità di \LaTeX, TOPtesi definisce la geometria della pagina e dispone le informazioni accessorie come le testatine e i piedini. Queste impostazioni non sono modificabili né con TOPtesi né con la maggior parte degli altri pacchetti destinati alle tesi. Vale il concetto ``prendere o lasciare''.

{\tolerance=3000 Prima di usare TOPtesi si esaminino i semplici esempi contenuti nei file accessori di questo pacchetto: \file{toptesi-example.pdf}, \file{toptesi-example-luatex.pdf}, \file{topfront.example.pdf}, \texttt{toptesi-example-con-frontespizio.pdf}\index{file!toptesi-example-con-frontespizio.pdf@\texttt{toptesi\discretionary{-}{}{-}example\discretionary{-}{}{-}con\discretionary{-}{}{-}frontespizio.pdf}}. Si potrà vedere se l'impaginazione aggrada oppure se si desidera un'altra impaginazione; in questo secondo caso ci si rivolga ad altre classi e ad altri pacchetti: in rete non è difficile usare un motore di ricerca per trovare il pacchetto \pack{frontespizio}, le classi \class{sapthesis}, \class {suftesi}, \class{TesiModerna}, \class{TesiClassica} e diverse altre classi prodotte specialmente nel mondo angloamericano e ancor meno adatte ad una personalizzazione multilingue. Alcune di queste classi sono dotate di eccellenti file di documentazione, altre sono usabili grazie a ``template'' o modelli di tesi scritte facendo uso di testo fittizio, ma tali da rendere immediatamente l'idea di come usare quei software.\par}

Volendo, anche i file sorgente di questi esempi d'uso del  pacchetto TOPtesi, elencati sopra con i nomi dei corrispondenti file PDF, possono essere usati come modelli; persino il file sorgente di questo manuale può essere usato come modello; basta cercare nelle cartelle dell'installazione di TOPtesi i file \file{.tex}, copiarseli in una propria cartella personale, cambiare loro il nome (attenzione, questo è importantissimo) e modificarne il contenuto a proprio piacimento. Con un minimo di attenzione si possono eliminare quelle parti che non servono; si possono modificare le opzioni della classe, si possono compilare con diversi motori di tipocomposizione del sistema \TeX. Non è forse fuori luogo sottolineare che l'uso dei modelli può essere molto utile per cominciare, ma fa perdere di vista il fatto che \LaTeX\ può fare molte più cose di quelle che si inseriscono solitamente negli esempi.\enlargethispage{-1\baselineskip}

\section{Cosa leggere}
Un manuale di solito non è da leggere dalla prima pagina all'ultima. Se ne possono saltare diverse sezioni, ma è bene sapere che cosa si salta, quindi una rapida sfogliata delle pagine che si saltano non fa male.

Se avete già il sistema \TeX\ installato nel vostro PC, potete saltare buona parte del capitolo~\ref{cap:introduzione}, ma almeno una volta conviene leggere la parte che precede il primo paragrafo numerato.

Del capitolo~\ref{cap:uso} è opportuno leggere il paragrafo~\ref{sec:configurazione}, perché descrive l'uso di un file di configurazione, che non è obbligatorio usare, ma che risulta molto comodo.

Conviene leggere l'uso dei loghi nel paragrafo~\ref{ssec:loghi}, perché TOPtesi consente di usare diversi loghi nel frontespizio e li può collocare in posti diversi della pagina.

Il paragrafo~\ref{sec:cominciare} espone come impostare inizialmente il documento principale della tesi e di come frazionarne il contenuto in diversi file. Non esiste un unico metodo e i vari metodi presentano vantaggi e svantaggi. Conviene conoscere bene questi metodi e i loro pro e contro, anche perché nell'uso elementare di \LaTeX\ questi concetti non vengono mai affrontati.

Il paragrafo~\ref{sec:codifica} è fondamentale; disporre di un editor che salva i file sorgente con una certa codifica e poi cercare di compilare quei file con \LaTeX\ impostato per una codifica diversa, vuol dire combinare pasticci inenarrabili; in questo manuale si consiglia di usare sia per l'editor si per \LaTeX\ la codifica \opt{utf8} ma è i laureando che deve sceglierla in base alle caratteristiche del suo software e ai programmi che intende usare per la compilazione. Tutti e tre i programmi principali di composizione, \pdfLaTeX, \XeLaTeX, e \LuaLaTeX\ funzionano bene con la codifica \opt{utf8}; \pdfLaTeX\ funziona anche con altre codifiche; gli editor un po' datati non funzionano con la codifica \opt{utf8}, quindi è evidente che le varie situazioni richiedono impostazioni attente e accurate. Se fosse necessario, l'argomento codifiche può essere approfondito anche su altri testi liberi, per esempio la guida tematica che si trova nella sezione documentazione dell'associazione \GuIT, in \url{http://guitex.org/home/images/doc/GuideGuIT/introcodifiche.pdf} dal titolo \emph{Introduzione alle codifiche in entrata e in uscita}.

Il paragrafo~\ref{sec:lingue} descrive come impostare il file sorgente della tesi per comporre il testo in diverse lingue o per impostare una lingua principale diversa dall'italiano. È essenziale per comporre tesi in programmi di doppia laurea. Infatti TOPtesi è stato creato anche per poter soddisfare le esigenze che un numero sempre maggiore di laureandi incontrano quando partecipano a programmi di doppio titolo con i vari programmi europei Erasmus, Life Long Learning, Erasmus Mundus, eccetera; le tesi svolte in questi percorsi solitamente richiedono l'uso di altre lingue oltre o in sostituzione dell'italiano. TOPtesi viene incontro a queste esigenze sia grazie all'uso di \LaTeX\ o \XeLaTeX\ o \LuaLaTeX\ che sono in grado di gestire una ottantina di lingue, sia perché quasi tutti i comandi che devono essere usati per il frontespizio e per molte strutture interne sono completamente configurabili in accordo con le lingue usate. Il paragrafo~\ref{sec:lingue} provvede a spiegare come eseguire queste configurazioni.

Per la questione codifiche e la questione lingue, che sono due aspetti che hanno forti collegamenti, è opportuno tenere presenti le seguenti considerazioni.

TOPtesi non serve solo per comporre tesi di laurea in ingegneria. Sono al corrente che il pacchetto TOPtesi è stato usato per comporre almeno una tesi di filologia greca classica, e almeno una tesi di commento ad un testo copto altomedievale. Di queste sono sicuro, ma ho informazioni indirette che sono state composte altre tesi in lingue moderne e antiche che facevano uso di alfabeti diversi da quello  cosiddetto ``latino''. Al tempo di quelle due tesi sul greco antico e sul copto altomedievale non esistevano ancora i programmi del sistema \TeX, \XeLaTeX\ e \LuaLaTeX; oggi che sono disponibili forse sarebbero state composte più facilmente con uno di questi due programmi; ma con questi è obbligatorio usare la codifica d'entrata \opt{utf8}\footnote{In realtà si potrebbero anche usare le codifiche a 8~bit tipiche di \pdfLaTeX, ma bisogna ricorrere ad artifici che non vale la pena usare quando sono disponibili i font OpenType; sarebbe stato necessario farlo per il copto altomedievale e per il copto liturgico, per i quali ho creato solo font con codifiche a 8~bit.} e font codificati UNICODE, come i font OpenType.

La codifica di entrata \opt{utf8} è consigliabile anche se si usa \pdfLaTeX, ma non è imposta, quindi TOPtesi è ``indifferente"' alla codifica d'entrata purché glielo si dica:  è quindi compito e responsabilità del laureando quello di specificare la codifica d'entrata. Similmente in relazione alla lingua o alle lingue usate è compito del laureando specificare la o le codifiche dei font da usare e di specificare i nomi dei font da usare.

Infine il capitolo~\ref{cap:comandispecifici} contiene tutti i comandi e gli ambienti introdotti da TOPtesi per la composizione del frontespizio, per la personalizzazione del frontespizio, per la composizione di strutture di testo o di figure in estensione a quelle normali di \LaTeX, e via di questo passo.

\section{Errori da evitare}

Tuttavia si abbia anche cura di \textcolor{red}{non copiare nel preambolo di una tesi da comporre con TOPtesi un preambolo recuperato dalla tesi di un amico o, peggio ancora, dalla rete!}. Sarebbe una operazione fonte di molte delusioni, perché da una parte potrebbero venire caricati pacchetti incompatibili con TOPtesi, dall'altro alcuni comandi di TOPtesi ne potrebbero venire modificati con funzionalità diverse da quelle previste per TOPtesi. Potrebbero anche manifestarsi dei conflitti fra pacchetti caricati nel preambolo e quelli già caricati da TOPtesi.

\section{Pacchetti già caricati da TOPtesi}
Vale la pena di elencare i pacchetti già caricati dalla classe \class{toptesi}, al fine di evitare di ricaricarli nel preambolo.
\begin{description}\def\Item[#1]{\item[\normalfont\pack{#1}]}
\Item[toptesi] è il file di macro di TOPtesi; potrebbe anche essere usato con una classe diversa da \class{toptesi}.
%
\Item[graphicx] serve per gestire diverse funzioni grafiche e per l'inclusione di file grafici esterni, come fotografie, disegni, e simili.
%
\Item[etoolbox] è un pacchetto di servizio, le cui funzionalità possono essere usate anche da un laureando molto competente; questo pacchetto agevola moltissimo la gestione dei file di classe e di quelli di estensione. Se il laureando ha sufficiente padronanza di \LaTeX, può definirsi altri comandi specifici per la sua tesi, sfruttando le funzionalità avanzate di questo pacchetto.
%
\Item[topfront] contiene i comandi specifici per comporre il frontespizio; potrebbe venire anche usato da solo con un'altra classe; mediante l'opzione \opt{noTOPfront} per la classe \class{toptesi} se ne può inibire il caricamento, così che l'operatore possa comporre il frontespizio con altri pacchetti esterni, o possa comporlo a modo suo sfruttando come meglio crede l'ambiente \env{titlepage}.
La classe \class{toptesi} carica questo modulo solo durante l'esecuzione del comando \verb|\begin{document}|, quindi nessun comando che abbia a che fare con il frontespizio può essere usato nel preambolo, in quanto \pack{topfront} non è ancora stato letto e quindi i comandi non sono ancora stati definiti.
%
\Item[topcoman] un piccolo pacchetto che contiene alcuni comandi utili in generale, non solo per l'uso con TOPtesi. Non è possibile inibirne il caricamento, ma spero che provveda da solo a non entrare in conflitto con altri pacchetti; finora non ho incontrato conflitti.
%
\Item[ifxetex] serve per distinguere il motore di composizione con cui si compila la tesi.
%
\item[\normalfont\meta{nome del main file}\file{.cfg}] Il file di configurazione specifico per una data tesi viene caricato solo se esiste nella medesima cartella dove risiede il main file della tesi stessa.
%
\Item[babel] viene caricato con le opzioni \opt{english}, \opt{italian}, cosicché l'italiano risulta svolgere le funzioni della lingua principale; questo pacchetto viene caricato solo se si compone la tesi con \pdfLaTeX, e TOPtesi riconosce da solo quale sia il motore di composizione usato.
%
\Item[polyglossia] viene caricato solo se la tesi viene composta con \XeLaTeX\ o \LuaLaTeX. L'italiano viene specificato come lingua principale, e l'inglese come altra lingua.
%
\Item[pdfx] deve venire caricato esplicitamente dall'utente, come spiegato nell'apposito paragrafo; nelle versioni precedenti esso veniva caricato specificando l'opzione \opt{pdfa} per produrre un file possibilmente conforme al formato PDF/A. Ora, per compatibilità con il passato, l'opzione è ancora attiva, ma il suo scopo è solo quello di emettere un avviso di consultare la documentazione.
%
\Item[hyperref] serve per comporre i collegamenti ipertestuali. Non è il caso di preoccuparsi di quando TOPtesi carica \pack{hyperref} perché ci pensa lui a ritardarne la chiamata al momento di iniziare la composizione del documento. Si veda più avanti per sapere come e quando eventualmente configurare \pack{hyperref} con le sue opzioni.
\end{description}

Il laureando è tenuto a documentarsi su ciascuno di quei pacchetti; normalmente egli dispone di tutta la documentazione di cui necessita già nella sua installazione del sistema TeX\ completo e aggiornato; basta che apra un terminale e vi scriva dentro;
\begin{flushleft}
\texttt{texdoc} \meta{nome del pacchetto}
\end{flushleft}
e, dopo aver premuto il tasto \fbox{invio}, sullo schermo del suo PC si apre la finestra che contiene la documentazione.

Conoscere l'elenco di questi pacchetti è importante proprio per non ricaricarli e per evitare conflitti quando si specificano opzioni diverse.

\section[Pacchetti che il laureando deve caricare personalmente]{\fontsize{16.5}{16.5}\selectfont Pacchetti che il laureando deve caricare personalmente}

Si noti: non sono precaricati i pacchetti \pack{inputenc} per definire la codifica d'entrata; \pack{fontenc} per definire la codifica dei font di uscita; e non è preimpostato nessun font particolare da usare per la composizione della tesi.

È voluto: \pack{inputenc} non deve essere caricato se si usano \XeLaTeX\ o \LuaLaTeX, ma in entrambi i casi l'editor che si usa per comporre il file sorgente \emph{deve} essere configurato in modo che salvi i file sorgente con la codifica \opt{utf8}. Con \pdfLaTeX\ ci sarebbe una certa libertà nello scegliere la codifica d'entrata, ma, insisto, sarebbe meglio in ogni caso evitare di usare qualunque altra codifica diversa da \opt{utf8}. 

Per i font di uscita la o le codifiche da specificare dipendono dalle lingue usate; ripeto la raccomandazione di preferire i font espressamente confezionati per l'uso con \pdfLaTeX, quando si compone con questo programma; se si usano i programmi \XeLaTeX\ o \LuaLaTeX\ si abbia l'accortezza di usare il pacchetto \pack{fontspec} specificandogli le opzioni giuste e caricando poi, mediante se sue funzionalità, i font OpenType di cui si sia accertata la presenza sulla propria macchina, e si sia verificato che contengano tutti i glifi che si intendono usare nella tesi. Tanto per citare l'importanza di questa verifica, questi programmi lavorano di default con i font Latin Modern, che, come dice il nome, contengono \emph{solo} i caratteri latini. Se ci fosse bisogno di scrivere in greco o in cirillico, per esempio, allora i font OpenType UCM (distribuiti con il sistema \TeX\ completo e aggiornato) contengono anche questi alfabeti.

Ecco quindi che la specificazione delle codifiche dei font di uscita è importante per \pdfLaTeX\ perché con questo programma i font predefiniti sono i Computer Modern codificati in \opt{OT1}, cioè mancano di qualunque segno accentato e non sono nemmeno completamente compatibili con la codifica \acro{ascii}. Questi font preimpostati potevano (forse) andare bene ai primordi dell'esistenza del sistema \TeX\ ma non vanno bene oggi nemmeno per l'inglese, visto che anche in inglese si fanno citazioni di testi o di nomi di persone in lingue straniere che usano caratteri latini accentati.

Per motivi diversi non si sono caricati i pacchetti per la composizione della matematica estesa, \pack{amsmath}, \pack{amssymb} (che a sua volta carica \pack{amsfonts}); né i pacchetti \pack{amsthm} per la definizione di enunciati come teoremi, lemmi corollari, definizioni e simili, né il pacchetto \pack{bm} per comporre in neretto simboli isolati di una espressione matematica.

Non si sono caricati nemmeno i pacchetti per il disegno programmato come, per esempio, \pack{tikz}; a un laureando in ingegneria certamente servono, a un laureando in letteratura medievale difficilmente potrebbero servire, ma non venga in mente al laureando di caricare quei font che non gli permettono di comporre la matematica secondo le norme ISO-UNI; fra questi ci sono i font \file{euler}, certamente molto belli, ma non consentono di rispettare le norme ISO-UNI, perché il font non è inclinato come invece  deve essere un buon corsivo matematico (\emph{math italics}) e molti caratteri non sono facilmente distinguibili da quelli composti in tondo (\emph{roman}).

\section{Pacchetti da non caricare affatto}

\textcolor{red}{Non si devono assolutamente caricare pacchetti che modifichino l'aspetto della pagina, né caricare pacchetti che modifichino la composizione dei titoli o delle didascalie o delle note}; eccezionalmente il pacchetto \pack{caption} può venire caricato; \class{toptesi} verifica se tale pacchetto è stato effettivamente caricato e, nel caso, non definisce nemmeno la macro per le didascalie esplicitamente creata per la composizione delle tesi. 

Se si vogliono fare questo genere di modifiche, stile della pagina, testatine, piedini, note, \dots\  è meglio rivolgersi ad altri pacchetti diversamente configurabili; cito fra gli altri \class{suftesi}, \class{sapthesis}; ma quando si va a leggerne la documentazione si scopre che tutti vietano la modifica dell'aspetto della pagina o dei titoli o delle didascalie, a meno che quelle stesse classi non dispongano di comandi già predisposti per la personalizzazione di quegli elementi.

Se il laureando va a cercare in rete altri pacchetti per comporre tesi, si trova davanti allo stesso ostacolo. Se la propria università prescrive stili di pagina diversi o mette a disposizione file classe appositi (di solito piuttosto datati e costruiti male),  si renderebbe necessario creare una classe apposita per soddisfare quelle esigenze. Non saprei cosa consigliare, se non ricorrere ad una classe generica come \class{book} e caricare tutti i possibili pacchetti di configurazione che si considerino necessari, per arrivare ad un risultato come quello richiesto dalla propria sede universitaria; i pacchetti preconfezionati come TOPtesi, e gli altri citati sopra, risparmiano questo lavoro, ma sono rigidi. In fondo anche TOPtesi è costruito così: parte dalla classe \class{report} e vi costruisce attorno quello che si è voluto fare in base a prescrizioni valide per il Politecnico di Torino, ma lungamente discusse e concordate con l'ateneo e poi rese compatibili con le prescrizioni di diverse altre università. Non è certo un pacchetto perfetto, ma è un buon compromesso.

\section{Comandi e ambienti di TOPtesi}

Nel capitolo~\ref{cap:comandispecifici} sono descritti i comandi specifici e gli ambienti introdotti da TOPtesi in aggiunta a quelli della classe \class{report} originali o modificati da TOPtesi.

La maggior parte di questi comandi si riferisce alla compilazione del frontespizio, ma è bene che il laureando li abbia sempre a portata di mano per poter eseguire le molteplici personalizzazioni che sono offerte da TOPtesi, in particolare dal suo pacchetto \pack{topfront}. Quel capitolo contiene anche le figure che rappresentano otto tipici frontespizi in italiano, per elaborati finali della laurea triennale, per la laurea magistrale, per la dissertazione dottorale, sia svolte in singoli atenei, sia svolte in atenei associati. Sia con i loghi in testa alla pagina sia con questi loghi nella metà inferiore della pagina. Sono tante varianti che possono soddisfare molte esigenze, ma che evidentemente non le soddisfano tutte. Queste sono quelle previste e il pacchetto \pack{topfront} non è abbastanza elastico per gestirne altre. Se si vogliono stili diversi esiste sempre il pacchetto \pack{frontespizio} creato apposta da un altro docente universitario di un altro ateneo, quindi con un'altra visione d'insieme sull'aspetto dei frontespizi. Infine esiste sempre la possibilità di usare l'ambiente \amb{titlepage} per creare il frontespizio in modo assolutamente libero da ogni vincolo.

In quel capitolo sono anche rappresentati i quattro frontespizi fondamentali nella figura~\ref{fig:frontespizi}; è importante avere queste quattro figurine sotto gli occhi per sapere come comporre il proprio frontespizio in modo che corrisponda alle prescrizioni del proprio ateneo.

Il laureando non dedichi invece troppo tempo alla lettura del paragrafo~\ref{sec:PDFA} e seguenti, perché vi si parla di come comporre la tesi in modo che soddisfi alle prescrizioni di archiviabilità introdotte dalle norme ISO~19005 e successive. Quanto scritto in quei paragrafi serve solo se è richiesta la tesi in versione archiviabile secondo le norme ISO. In ogni caso, se non è richiesto, non è opportuno addentrarsi per questa strada; se è richiesto, le operazioni necessarie si possono eseguire a tesi completata come ultimo tocco finale.

\section{Modelli di tesi e di frontespizi}

Ricordo infine che il pacchetto TOPtesi contiene anche un certo campionario di modelli di tesi e di frontespizi di vario genere, che possono essere composti con diversi motori di composizione. Il laureando può servirsene in modo molto semplice: copia il file di esempio nella propria cartella di lavoro, gli cambia nome e poi modifica il modello commentando o de-commentando alcune righe, togliendo parti che non servono; per esempio  vi ho messo un piccolo esempio di ringraziamenti, ma i ringraziamenti non andrebbero mai usati -- vedi più avanti perché; vi ho messo un piccolo esempio di dedica, ma le dediche sono superflue nel 99\% dei casi; eccetera.

Ripulito il preambolo delle cose che non servono, e scelte le righe ritenute necessarie,  basta cambiare i nomi di fantasia che ho usato e i testi di fantasia che ho inserito per ottenere lo schema della tesi, che va poi riempito col contenuto relativo alla tesi che si vuole effettivamente comporre.



\chapter{Introduzione}\label{cap:introduzione}

Si legge ancora nelle istruzioni per scrivere le tesi di molte
università italiane e  straniere:
\begin{quote}\ttfamily
Comporre la tesi  con interlinea 2 e con righe di 60 battute;~\dots
\end{quote}

Quelle università non si sono ancora accorte che le macchine da
scrivere meccaniche o elettromeccaniche sono rimaste oggetti di
sola curiosità, ammesso che ce ne sia ancora qualcuna disponibile
e le poche superstiti non siano tutte nei musei.

Oggi si scrive con uno dei tanti sistemi di elaborazione di testi,
detti anche ``word processor'', che fanno parte più o meno di
default di ogni dotazione iniziale di qualsiasi PC di qualunque
marca e con qualunque sistema operativo.

Fra i vari programmi disponibili, uno in particolare spicca per la sua particolarità: \LaTeX. Veramente esso non è un word processor, anche se a prima vista lo sembrerebbe; il suo scopo non è finalizzato al testo in quanto tale, bensì alla sua composizione tipografica; esso è un programma di \emph{tipocomposizione}.

\LaTeX\ è nato come
sovrastruttura di un altro programma, \TeX, che continua ad
esserne il motore. La prima versione di \TeX\ è stata creata nel
1978, ma è ancora usata oggi, naturalmente molto aggiornata e ampliata, e
questo fatto è una cosa insolita nel panorama turbolento di novità
dell'informatica.

Secondo me, il suo successo è dovuto a due fatti: (a) esso è stato
progettato e implementato da un matematico per comporre i suoi
stessi libri di informatica matematica; (b)~egli l'ha messo a disposizione di
chiunque, sin dal primo momento, come software libero.

Oggi il software libero è piuttosto diffuso, ma nel 1978 parlare
di software libero era quasi una bestemmia.

D'altra parte Donald E.~Knuth non era soddisfatto della bassa
professionalità che anno dopo anno manifestavano i compositori
delle case editrici, i quali anno dopo anno si abituavano a quanto
i programmi di elaborazione mettevano loro a disposizione, ma
contemporaneamente perdevano le loro conoscenze professionali via
via che si adattavano a quanto quei programmi consentivano loro di
fare.

La cosa era o stava diventando insostenibile durante i vari anni
in cui uscivano i successivi volumi dell'opera di Knuth \emph{The Art
of Computer Programming}, e così Knuth si dedicò alla creazione
della tipografia elettronica realizzando il programma \TeX\ che,
ripeto, è  ancora in ottima salute e molto vivace dopo oltre trent'anni di
onorato servizio.

L'uso di \TeX\ per eseguire direttamente la composizione era piuttosto
difficile, e ogni utente doveva prima o poi imparare a scriversi delle
macroistruzioni che gli consentissero di agevolare il suo lavoro.

Nel 1984 \TeX\ era già così diffuso in tutto il mondo,
specialmente in ambito accademico, che Leslie Lamport decise di
produrre un sistema quasi completo di macro che consentisse agli
utenti di usare \TeX\ lasciandolo dietro le quinte, in modo da
potersi concentrare sul contenuto dei loro scritti e non sulla
forma da dare a questo o a quel dettaglio.

Nel 1990 Knuth pubblicò la versione di \TeX\ che consentiva di
comporre in diverse lingue simultaneamente; nel 1994 molti utenti
di \LaTeX\ costituirono il \LaTeX3 Team al fine di rendere
gestibile la mole enorme delle estensioni di \LaTeX\ che in
10~anni utenti entusiasti avevano messo a disposizione della
comunità degli altri utenti. Insomma è  successo con \TeX\ e
\LaTeX\ quello che succede normalmente con il software libero.

Attenzione: Knuth paga di tasca sua un assegno a chiunque trovi un
errore nel suo software; a tutt'oggi non è  andato in bancarotta,
sia perché  gli errori sono rarissimi, sia perché  quelle poche
persone che hanno segnalato errori veri e hanno ricevuto l'assegno
di Knuth, non l'hanno incassato ma l'hanno incorniciato come 
una reliquia preziosa.

\section{\LaTeX\ e le tesi di laurea}

Ovviamente \LaTeX\ serve per scrivere qualunque cosa; o meglio;
serve per comporre tipograficamente qualunque testo. \LaTeX\ non è
un programma di impaginazione, è un programma di composizione
tipografica. Non aspettatevi quindi di poter fare qualunque
acrobazia con le righe di testo, come per esempio piegarle, deformando i caratteri che vi sono appoggiati sopra,
mettendovi attorno aloni di luce cangiante, sfumature,
ombreggiature, evidenziandone i contorni, eccetera. Queste cose
sono riservate ai creativi che si occupano di pubblicità.

Aspettatevi invece di comporre testi in cui ogni capoverso è ottimizzato per avere il minor numero di parole divise in sillabe in fin di riga, e di avere il minor numero possibile di ``ruscelli'' fra le parole grazie alla uniformità dello spazio interparola; aspettatevi di comporre formule complicatissime con il minimo di sforzo da parte vostra ma con la certezza che esse saranno composte come nessun altro programma riesce a fare. Se state componendo una tesi nel campo delle scienze umane aspettatevi il meglio in assoluto; se poi vi interessate di filologia di lingue antiche, non c'è altro programma che possiate usare a questo scopo. Aspettatevi uno stampato estremamente professionale.

Per questo motivo voi studenti che userete questo pacchetto di
macroistruzioni chiamato \textsf{TOPtesi} dovrete astenervi
dall'introdurre errori compositivi così da vanificare quanto di
bello riesce a produrre \LaTeX.

Inizialmente vi troverete un po' a disagio perché vi siete
abituati anche voi ai programmi commerciali che consentono la
``composizione sincrona''; vi consentono di vedere subito sullo
schermo il frutto del vostro lavoro. Per ottenere questo risultato
questi programmi hanno necessariamente rinunciato a diverse
funzioni, badando invece a  presentare sullo schermo con la massima 
velocità il testo composto.

\LaTeX\ richiede che voi scriviate un testo non formattato in puri
caratteri ASCII\footnote{In realtà potete scrivere con qualunque set di caratteri e con qualunque \emph{codifica}; oggi poi è possibile avere editor testuali che usano la codifica UNICODE, che consente, essendone capaci, di scrivere anche in cinese. Il documento che state leggendo è stato composto usando un file sorgente codificato in UNICODE; non contiene caratteri cinesi, ma avrei potuto farlo se conoscessi il cinese!}, ma marcato con un particolare sistema di
\emph{mark-up} che consente di sapere che cosa sia ogni parte del
vostro scritto: una equazione, una citazione, una enumerazione, una elencazione, una descrizione, una poesia, una bibliografia, una epigrafe, eccetera. Ci pensa poi \LaTeX\ in un secondo tempo a dare forma
al vostro testo e, in particolare, a dare la stessa forma a ogni
elemento del vostro scritto a seconda di come lo abbiate ``marcato'';
elementi marcati nello stesso modo vengono composti nello stesso modo.
Così si evitano quelle disuniformità compositive che si notano assai
spesso quando si usa un word processor comune.

Attenzione: questa guida presuppone che voi abbiate già una conoscenza di base del linguaggio  \LaTeX. Se non l'avete ancora, installate pure il programma, ma non usate questa guida per imparare a comporre testi con \LaTeX. Esistono diversi testi gratuiti in rete, da \emph{\LaTeX\ per l'impaziente} a \emph{L'Arte di scrivere con \LaTeX} a \emph{Introduzione all'arte della composizione tipografica con \LaTeX}; basta cercare questi titoli in rete e si troverà da dove scaricarli.

Ricordatevi però che il vostro primo problema sarà quello di superare lo scoglio psicologico di non vedere subito il frutto delle vostre fatiche; abituati come siete ai word processor che seguono il paradigma \emph{What you see is what you get} (Quello che vedi è quello che ottieni), che più realisticamente andrebbe scritto \emph{What you see is all you can get} (Quello che vedi è tutto ciò che puoi ottenere), dovete passare al paradigma: \emph{What you see is what you mean}, dove l'aspetto grafico di quel che si vede sullo schermo del PC in fase di scrittura iniziale non è importante, ma è importante quello che avete scritto; il suo aspetto grafico gli verrà dato da \LaTeX\ in un secondo tempo e in una maniera estremamente professionale. 

Non pretendo di dire che non si possano scrivere tesi ancora più professionali di come si ottengono con \textsf{TOPtesi}, ma certo il risultato è molto migliore di quello che si può ottenere con qualsiasi word processor.

Vi sconsiglio fortemente di lasciarvi ``attrarre'' dall'usare programmi come LyX o TeXmacs; vi danno l'illusione di comporre come fareste con un comune word processor, anzi TeXmacs usa gli stessi font che userebbe \LaTeX; è tutta una illusione; la comodità del comporre in modo da vedere sullo schermo qualcosa che vorrebbe essere molto simile a quanto \LaTeX\ produrrà, vi distrae dal vostro compito di fare attenzione al significato di quello che scrivete, e vi mettete a ``giocare'' con la sua forma. Non solo, ma quando vorrete estendere le capacità di presentare il testo che volete scrivere alla ``\LaTeX'', scoprirete ben presto che con LyX è molto difficile e con TeXmacs è impossibile.

Procuratevi invece un ottimo \emph{shell editor} predisposto per lavorare con \LaTeX, \XeLaTeX e \LuaLaTeX; supererete ben presto l'imbarazzo di non poter vedere subito il risultato del vostro scrivere, ma sarete abbondantemente ricompensati dalle infinite possibilità compositive dei programmi del sistema \TeX. Più avanti ne parlerò diffusamente, ma qui ho voluto avvisarvi subito di non farvi incantare dal canto delle sirene come LyX e TeXmacs.

\section{Installare \LaTeX}
Per cominciare a lavorare con \LaTeX\ per prima cosa dovete
installarvi il software; ovviamente se vi siete già installati
l'intero sistema \TeX\ ed è già calibrato per comporre
correttamente, potete saltare questo paragrafo.

Ci sono sostanzialmente tre situazioni.
\begin{description}
\item[Windows] Gli utenti delle piattaforme Windows possono collegarsi in rete al sito \url{www.miktex.org} e scaricarvi ed installarsi la distribuzione di \LaTeX, o meglio, del sistema \TeX\ chiamata \MiKTeX. Se si procede per questa via, allora si scarichi l'installazione completa, anche se si potrebbe installare la versione di base \texttt{small-miktex} che, seguendo le istruzioni di installazione, bisogna configurare in modo da consentirle di scaricare dalla rete ogni possibile pacchetto di estensione che possa via via essere necessario.

Ovviamente questo modo di installare e usare il sistema \TeX\ implica una connessione di rete sufficientemente veloce; personalmente, quando usavo una piattaforma Windows, ero solito scaricare la versione completa ed ero molto soddisfatto. 

Oggi, però, anche su una macchina Windows consiglierei di installare la distribuzione \TeXLive che viene gestita nello stesso modo sia sulle piattaforme Windows, Linux e Mac. Il pregio è che \TeXLive viene aggiornata quasi quotidianamente sui server e comunque è la versione dalla quale il curatore di MiKTeX attinge per creare gli aggiornamenti della sua distribuzione; necessariamente, quindi, prima che il curatore sia riuscito a portare MiKTeX allo stesso livello di \TeXLive, passano di solito alcune settimane, a volte anche di più; non voglio togliere niente alla bravura del curatore, ma il progetto MiKTeX è il lavoro di un solo uomo, mentre \TeXLive è il frutto del lavoro di una squadra completa e attivissima.

\item[Linux] Gli utenti di Linux sono particolarmente fortunati perché  il sistema \TeX\ è spesso parte integrante di qualunque variante di Linux, anche se non viene installato di default; basta inserire il disco di installazione o  basta connettersi in rete e invocare uno dei vari programmi come \prog{apt-get}, \prog{rpm}, \prog{yast} o~\dots\ per scaricare tutto quanto serve e per configurare l'installazione. Bisogna ricordarsi che Linux è un po' più ``ruspante'' di Windows, e quindi la configurazione richiede un po' più di attenzione e di ``smanettamento'' con la riga di comando; ma a questo i ``pinguini doc'' ci sono abituati; in compenso si ha il beneficio di avere tutto quanto il software già predisposto fin dalla nascita per macchine UNIX e Linux e quindi si evitano tutti i (pochi) piccoli bug che si incontrano quando le cose sono tradotte per altri sistemi operativi. 

Si faccia solo attenzione alle distribuzioni Debian; sono eccellenti e ne è garantita la compatibilità con ogni sistema operativo conforme ai dettami del consorzio Debian, ma solitamente sono in ritardo di alcuni mesi, in passato anche di un paio di anni, rispetto alle versioni aggiornate pubblicate dal \TeX\ Users Group! Appena si può si installi la distribuzione \TeXLive completa scaricata dal sito \url{http://www.tug.org/ctan.html} ufficiale. Esiste in rete un testo intitolato \emph{texlive-ubuntu.pdf}\footnote{Questo testo è stato scritto nel 2010 per installare \TeXLive su Ubuntu, che è di tipo Debian, ma ci sono anche le istruzioni per Fedora e OpenSuse; con piccole varianti sono istruzioni che vanno bene per qualunque macchina Linux.}, che si trova con qualunque motore di ricerca; specifica  come installare \TeXLive fresco di giornata (e aggiornabile sistematicamente come detto sopra per le macchine Windows) a fianco della distribuzione Debian di \TeXLive; quest'ultima serve per soddisfare le dipendenze di altri programmi Debian; la prima, invece, serve per lavorare davvero.

\item[Mac] Gli utenti delle piattaforme Macintosh e possessori di un portatile o di un desktop che funziona con il sistema operativo Mac~OS~X hanno anche loro a disposizione una distribuzione che si chiama \prog{Mac\TeX}, che viene installata e configurata con un particolare software adatto alla specificità del sistema operativo. Si cerchi il nome MacTeX con un qualunque motore di ricerca che indicherà  un sito dal quale si può scaricare il pacchetto di installazione (piuttosto grosso); alla fine del download  viene chiesto se continuare con l'installazione; rispondendo affermativamente, il software  viene scaricato sul disco che si sarà indicato, ma quel che è più comodo, esso è già completo, come ogni distribuzione e installazione \TeXLive. Il sistema \prog{Mac\TeX} produce il suo output essenzialmente in formato PDF; se ne tenga conto leggendo attentamente la documentazione. Il pacchetto è già dotato dello shell editor \TeX{}Shop che fa ricorso ad un suo previewer interno per il formato PDF; questo previewer consente di eseguire sia l'\textit{inverse search} sia di fare la \textit{forward search}. Questo è estremamente comodo durante la fase di editing del  documento. 
%Potrebbe essere una buona idea scaricare dalla rete anche l'applicativo freeware Adobe Reader che dalla versione 9 in poi è anche in grado di riconoscere se un file è o potrebbe essere conforme al formato PDF/A, anche se non è in grado di validarlo come tale; Adobe Reader X, quando apre un file che potrebbe essere conforme allo standard PDF/A, apre in testa alla schermata una riga di informazione nella quale dice che il file è conforme a questo standard; non è il caso di prendere alla lettera quella affermazione, perché i file che vengono classificati conformi allo standard PDF/A da Adobe Reader X, talvolta non passano il test di Preflight di Adobe Acrobat. Tuttavia consente di estrarre delle informazioni che sono difficili da estrarre con altri visualizzatori dei file in formato PDF e consente di aggiungere annotazioni. Naturalmente esistono altri visualizzatori dei file in formato PDF; sulle macchine Apple il visualizzatore di sistema consente di aprire e di editare parzialmente anche altri formati grafici e consente di eseguire molte operazioni, in particolare sui file PDF, che in generale richiedono altri sofisticati programmi di editing grafico.

%Attenzione, però: i previewer interni di \TeX{}shop e \TeX{works} consentono di eseguire l'\textit{inverse search} e, ovviamente, anche di fare la \textit{forward search}. Questo è molto comodo durante la fase di editing del vostro documento. Gli altri visualizzatori hanno difficoltà se non proprio l'impossibilità di ``accordarsi'' con la parte di editing testuale per consentire la ricerca diretta e inversa. Dal 2012 anche Texmaker e TeXstudio, entrambi multipiattaforma, sono dotati di visualizzatori interni che consentono la ricerca diretta e inversa; TeXstudio dalla versione 2.5 è anche in grado di interpretare le righe di autoconfigurazione scritte con le sintassi di \TeX{}shop e \TeX{}works, e si è aggiunto alla lista degli \emph{shell editor} che sanno interpretare le righe di autoconfigurazione. L'editor \prog{emacs} con il suo plug-in \pack{auctex}, funziona benissimo come shell editor per lavorare con \LaTeX, sembrerebbe che possa venire ``sincronizzato'' con un visualizzatore PDF con il quale poter eseguire la ricerca diretta e inversa; sa riconoscere le righe di autoconfigurazione che però sono scritte con una sintassi diversa da quella degli altri programmi nominati prima. 

\end{description}

Merita di segnalare che per tutte e tre le piattaforme è disponibile e, talvolta, è già installato di default anche l'editor \TeX{}works (multipiattaforma) che, con il suo visualizzatore interno, consente di eseguire  la ricerca diretta e inversa. Ad alcuni, abituati a interfacce grafiche fornite di molte barre cariche di icone per eseguire il possibile e l'impossibile, \TeX{}works piace poco perché la sua schermata è minimale, ma c'è un motivo: nei moderni schermi larghi, con rapporto di forma 16:9, lo schermo contiene accostate e senza sovrapposizioni sia la finestra di editing sia quella del testo composto; questo è molto comodo, più  di quanto si possa immaginare, per ``lavorare'' agevolmente il documento da comporre. Benché anche Texmaker e TeXstudio siano in grado di accostare le due schermate, essi consumano molto spazio per le barre superiori, inferiori e laterali, cosicché le vere aree destinate all'editing o alla visualizzazione ne risultano corrispondentemente ristrette, tanto da far preferire \TeX{}works sui laptop e ancor di più sui netbook.

Poi sono necessari i programmi accessori per visualizzare sullo schermo e/o stampare su carta i prodotti della  composizione. Ognuna delle tre piattaforme tipo può avere già installati sia i visualizzatori dei file in formato \texttt{.dvi}, \texttt{.ps} o \texttt{.pdf}. Il formato \texttt{.dvi} è  il formato nativo del sistema \TeX\ quindi il software per visualizzare e stampare arriva insieme alla distribuzione  già installata, però oggi è un formato che non si usa quasi più. Per il formato \texttt{.ps} bisogna disporre di qualcosa come \prog{ghostscript} e/o \prog{GSView} o altri simili software che con Linux sono solitamente già disponibili insieme al sistema. In ogni caso non è  difficile trovare in rete i luoghi da dove scaricarli. Per il formato \texttt{.pdf} Linux, come al solito è già attrezzato, ma non è male per tutti e tre i tipi di piattaforma il programma \prog{Adobe Reader} che la Adobe mette a disposizione di chiunque gratuitamente e per tutte le possibili piattaforme\footnote{Sembra che con Linux le cose non stiano più così, ma in rete si trovano ancora delle installazioni non recentissime, ma funzionanti, anche per Linux.}. Naturalmente l'\prog{Adobe Reader} è una specie di programma dimostrativo, per altro eccellente; ma credo che la Adobe lo metta a disposizione per far venire l'acquolina in bocca e per invogliare a comperare il prodotto commerciale completo \prog{Adobe Acrobat}; per gli studenti non costa molto in accordo con il programma Education di quell'azienda. Secondo me vale ogni dollaro che costa, ma ovviamente questo giudizio dipende dall'uso che se ne fa.

Oggi, invece, è molto importante disporre di visualizzatori PDF integrati con l'editor, in modo che siano predisposti per lavorare in tandem sia per mostrare costantemente a fianco della finestra di editing la finestra del file composto in formato PDF, sia per fare la ricerca diretta e inversa; oltre ai già citati \TeX{}shop (solo Mac) e \TeX{}works (multipiattaforma) posso citare gli editor TeXstudio e Texmaker (multipiattaforma e molto simili fra loro); altri \emph{shell editor} non dispongono di un visualizzatore integrato, ma possono venire ``sincronizzati'' con visualizzatori esterni; per esempio su piattaforme Windows il visualizzatore sincronizzabile è SumatraPDF; su Linux è Okular; su Mac ne posso citare altri due: TeXnicle (gratuito) e Texpad (commerciale, ma con un costo accessibilissimo) che non solo hanno il visualizzatore integrato, ma le loro ``finestre''  sono in realtà due parti di una stessa finestra; ingrandendola a pieno schermo si ottiene una comodità di composizione difficilmente ottenibile con altri sistemi.

Anche Emacs (multipiattaforma), arricchito del plug-in Auctex, che lo rende particolarmente adatto per gestire i file del sistema \TeX, è sincronizzabile con vari visualizzatori PDF; avendo la pazienza di imparare ad usarlo in modo non superficiale, Emacs assieme ad Auctex rendono il lavoro con il sistema \TeX\ particolarmente comodo. 

%Per le piattaforme Linux esiste uno \emph{shell editor} che si chiama
%\prog{Kile} (KDE Integrated \LaTeX\ Environment) ed è fatto
%apposta per lavorare con \LaTeX. Esso è installabile anche sulle
%piattaforme Windows pur di disporre dell'ambiente \prog{Cygwin}
%(un simulatore di UNIX compatibile con la maggior parte dei
%sistemi operativi per Windows) dentro il quale si sia già
%installato tanto KDE (K Desktop Environment) quanto le sue
%librerie di sviluppo. Di solito \prog{Kile} è obbligato alla dipendenza da \TeXLive/Debian, per cui sarebbe meglio non farne uso.

Per le piattaforme Windows il programma di installazione di \MiKTeX\ offriva la possibilità di installare \prog{TeXnicCenter}, ma lo sconsiglio vivamente, perché non è all'altezza delle distribuzioni moderne di \MiKTeX\ e di \TeXLive; oggi mi pare che \MiKTeX\ venga distribuito con TeXstudio. C'è anche lo \emph{shell editor} WinEdt (shareware), ottimo e dalla versione 8 in poi sembra che sia dotato di un visualizzatore integrato PDF (comunque è sincronizzabile con SumatraPDF).

Per le piattaforme Mac esistono diversi programmi; a me sembra che il migliore
di tutti sia \TeX{}shop, automaticamente installato quando si usa la
distribuzione \prog{Mac\TeX}. ``Migliore'' significa qui il giusto compromesso fra la semplicità e l'efficienza e la validissima integrazione con un suo visualizzatore interno che consente di eseguire con un semplice click di mouse il passaggio da un punto della finestra di composizione del file sorgente al punto corrispondente nella finestra del documento composto in formato PDF, e viceversa. Suo ``figlio'' \TeX{}works (multipiattaforma) sembra avere qualche funzionalità in meno (non è così vero), ma ha una interfaccia comodissima per scoprire la codifica di un file \file{.tex} e per convertire il file in un'altra codifica; \TeX{}shop e \TeX{}works sono auto"configurabili per ciascun file \file{.tex} grazie ad alcune righe di commenti speciali da scrivere in testa al file, cosa che rende il loro uso incredibilmente comodo. TeXstudio dalla versione 2.5 in poi è in grado di interpretare le stesse righe speciali di \TeX{}shop e di autoconfigurarsi di conseguenza (settembre 2012). Emacs con Auctex è in grado di usare righe di autoconfigurazione che però hanno una sintassi diversa da quelle di \TeX{}shop. Questa affermazione vale anche per Aquamacs, che è una applicazione per Mac che integra direttamente emacs e Auctex.\looseness=-1

Per tutte e tre le piattaforme principali \prog{TeXStudio} offre notevoli vantaggi, compresa la visualizzazione di parti del testo da comporre che richiedano più interazione fra il compositore e il software. Permette anche di disporre di una finestra laterale che contiene tutta la struttura ad albero del documento da comporre; cliccando su ogni ramo o rametto di questo albero, il programma sposta la finestra sul punto del file sorgente dove quella sezione comincia. Quasi tutti gli editor citati \emph{non} consentono di fare la ricerca inversa con il formato di uscita PDF, ma solo con il formato DVI. \prog{TeXShop} per Mac, e \prog{TeXworks}, \prog{TeXStudio} e \prog{TeXmaker} per tutte le piattaforme permettono di eseguire nativamente la ricerca inversa anche con il formato PDF, e questa particolarità è estremamente comoda. Per le piattaforme Windows esiste il visualizzatore PDF \prog{SumatraPDF} che può essere configurato per interagire con molti editor per poter essere usati assieme sia con la ricerca diretta sia con quella inversa. Va da sé, che se non ci sono esigenze diverse, il formato di uscita PDF è sicuramente quello da preferire.

Mi sono ripetuto diverse volte nel descrivere gli editor per lavorare con il sistema \TeX? L'ho fatto apposta, a costo di essere noioso. Lavorare con editor efficienti, ben adattati al lavoro che si deve fare e che consentano la ricerca diretta e inversa fra la finestra di editing e quella del file composto in formato PDF è talmente importante, che le ripetizioni non sono mai abbastanza.

\goodpagebreak

\section{Ora siete pronti}
Ora che avete scaricato tutto il software gratuito o commerciale di cui
avete bisogno siete pronti per cominciare.

{\color{red}Dopo avere letto un po' di documentazione e aver giocato un poco  con i programmi già predisposti sia dalla distribuzione del sistema \TeX\ sia dai vari editor descritti sopra, siete in grado di capire come funziona il tutto.\label{warn:avvertenza}

Attenzione: non si minimizzi la frase precedente: la documentazione va letta sempre e va capita fino in fondo; se c'è qualcosa che non si capisce, si provi con  qualche piccolo esercizio, visto che la pratica permette di capire la teoria (e viceversa), ma non andate a cercare in rete aiuto nei vari forum dedicati a \LaTeX\ per farvi dire: ``guarda nella pagina tal-dei-tali della documentazione''; per voi sarebbe una vera umiliazione\dots\ I forum vanno benissimo, ma per rispondere a domande serie, non a cose che si trovano già documentate.}

A qualcuno può venire in mente: ``Ma non sarà mica che ci siano in giro dei programmi che permettono di fare tutto questo in modo WYSIWYG?'' Come noto, WYSIWYG è  l'acronimo che si forma con le iniziali di ``what you see is what you get''. \LaTeX\ dovrebbe essere classificato con l'acronimo WYSIWYM che sta per ``what you see is what you mean''. Certo per ottenere esattamente quello che si vuole comunicare bisogna lavorare (apparentemente) di più; in realtà bisogna usare di più la testa e di meno il mouse.

Tuttavia là fuori nei negozi ci sono diversi prodotti che consentono di usare  \LaTeX\ praticamente in modo WYSIWYG; da Scientific Word a LyX a TeXmacs a Textures ce ne è  per ogni piattaforma; LyX e TeXmacs sono freeware mentre gli altri costano attorno ai 500\textdollar. Poi c'è la soluzione gratuita di OpenOffice e di LibreOffice con l'estensione~1.2 di \prog{Writer2LaTeX}. Io le sconsiglio tutte, come ho già avuto modo di dire, e qui ne ripeto i motivi.

Per poter operare in modo ``sincrono'' così da avere immediatamente sullo schermo una cosa molto simile a quello che si otterrà sulla carta, il programma deve essere velocissimo ad eseguire il rendering grafico di quanto viene via via immesso nel testo; per questo motivo deve rinunciare a non poche funzionalità del sistema \TeX. Però tutti questi software hanno la possibilità di salvare i file in formato \texttt{.tex}, cioè  nel formato sorgente del sistema \TeX, per cui una volta finito l'editing si può eseguire la composizione finale con il programma vero, e non tramite le funzionalità del programma di editing.

Io ho cominciato a lavorare con \LaTeX\ a metà degli anni '80 e non ho mai usato editor sincroni. Negli anni '90 ho esaminato Textures per aiutare un collega statunitense che stava scrivendo un libro con quel software, ma non sapeva come fare per disporre di macro adatte per la composizione della matematica di cui aveva bisogno; ma dopo poco ho lasciato perdere perché dovevo lavorare su una piattaforma Mac altrui e non potevo seccarlo in continuazione per chiedergli come si fa questo, come si fa quello; allora il sistema operativo era molto diverso dall'attuale Mac~OS~X e solo gli addetti ai lavori sapevano come usarlo al meglio. Però non ne ero rimasto particolarmente impressionato, anche perché allora i font vettoriali venivano gestiti in modo molto più complesso di oggi.

\iffalse
\begin{table}[!tb]{\centering
\caption[Le principali differenze fra \textsf{pdflatex} e \textsf{xelatex}]{Le principali differenze fra \pdfLaTeX\ e \protect\XeLaTeX }\label{tab:pdflatex-vs-xelatex}
\begin{tabular}{lp{.3\textwidth}p{.25\textwidth}}
\toprule
					& \centering pdf\LaTeX		& \centering \XeLaTeX	\tabularnewline
\midrule
Formati di uscita	& \centering DVI e PDF		& \centering XDV e PDF	\tabularnewline
Font				& \raggedright Font di 256 caratteri con codifiche OT1, T1, T2, LY1, LGR, eccetera
									& \raggedright Font OpenType con codifica UNICODE\tabularnewline
Alfabeti diversi	& \raggedright Solo mediante pacchetti esterni
									& \centering  Font OpenType \tabularnewline
Lingue retrograde	& \raggedright	Solo mediante pacchetti esterni
									& \centering  Font OpenType \tabularnewline
Ideogrammi			& \raggedright	Solo mediante pacchetti esterni
									& \centering  Font OpenType \tabularnewline
Gestione Lingue		& \centering  circa 80 lingue &\centering  circa 80 lingue \tabularnewline
Formati immagini	& \raggedright  PDF, JPG, PNG,  EPS($\dagger$)	& PDF, JPG, PNG, EPS \tabularnewline
Possibilità di scontornare&\centering SI	& \raggedright Ridotta ma può essere migliorata  attraverso i comandi primitivi del motore \prog{xetex}\tabularnewline
Microgiustificazione& \centering Completa	& \centering Solo protrusione \tabularnewline
Formato archiviabile& \centering PDF/A-1b	& \centering SI, ma\dots\ (*)		 \tabularnewline
\bottomrule
\end{tabular}\par}
\vspace*{\medskipamount}
\small \noindent (*) \XeLaTeX\ con la distribuzione 2016 di \TeXLive, consente la produzione diretta del formato PDF/A, ma richiede delle attenzioni particolari che verranno descritte nel seguito.

\noindent ($\dagger$) Dalla versione del 2010 \pdfLaTeX\ converte automaticamente in formato PDF i file EPS, conservandone quindi il carattere vettoriale; provvede anche a scontornarli.
\end{table}
\fi

\begin{table}[!tb]{\centering
\caption[Le principali differenze fra \textsf{pdflatex} e \textsf{xelatex}]{Le principali differenze fra \pdfLaTeX, \protect\XeLaTeX\ e \protect\LuaLaTeX}\label{tab:pdflatex-vs-xelatex}
\begin{tabular}{p{0.25\textwidth}p{.2\textwidth}p{.2\textwidth}p{0.2\textwidth}}
\toprule
			& \centering pdf\LaTeX	& \centering \XeLaTeX	&\centering \LuaLaTeX	\tabularnewline
\midrule
Formati di uscita	& \centering DVI e PDF	
					& \centering XDV e PDF	
					& \centering PDF	\tabularnewline
Font				& \raggedright Font di 256 caratteri con codifiche 
						OT1, T1, T2, LY1, LGR, eccetera
					& \raggedright Font OpenType con codifica UNICODE	
					& \raggedright Font OpenType con codifica UNICODE 
						\tabularnewline
Alfabeti diversi	& \raggedright Solo mediante pacchetti esterni
					& \centering  Font OpenType 
					& \centering  Font OpenType		\tabularnewline
Lingue retrograde	& \raggedright	Solo mediante pacchetti esterni
					& \centering  Font OpenType 
					& \centering  Font OpenType 	\tabularnewline
Ideogrammi			& \raggedright	Solo mediante pacchetti esterni
					& \centering  Font OpenType 
					& \centering  Font OpenType 	\tabularnewline
Gestione Lingue		& \centering  più di 80 lingue 
					& \centering  più di 80 lingue 
					& \centering  più di 80 lingue 
						\tabularnewline
Formati immagini	& \raggedright  PDF, JPG, PNG,  EPS($\dagger$)	
					& PDF, JPG, PNG, EPS 
					& PDF, JPG, PNG, EPS 
						\tabularnewline
\raggedright Possibilità di scontornare&\centering SÌ	
					& \raggedright Ridotta ma può essere migliorata  
						attraverso i comandi primitivi del motore \prog{xetex}
					& \centering SÌ
						\tabularnewline
Microgiustificazione& \centering Completa	
					& \centering Solo protrusione
					& \centering Completa	 
						\tabularnewline
Formato archiviabile& \centering PDF/A-1b	
					& \centering SÌ, ma\dots\ (*)
					& \centering SÌ	
						 \tabularnewline
\raggedright Integrazione con il linguaggio Lua& \centering NO
					& \centering NO
					& \centering SÌ
						\tabularnewline
\bottomrule
\end{tabular}\par}
\vspace*{\medskipamount}
\small \noindent (*) \XeLaTeX\ con la distribuzione 2016 di \TeXLive, consente la produzione diretta del formato PDF/A, ma richiede delle attenzioni particolari che verranno descritte nel seguito.

\noindent ($\dagger$) Dalla versione del 2010 \pdfLaTeX\ converte automaticamente in formato PDF i file EPS, conservandone quindi il carattere vettoriale; provvede anche a scontornarli.
\end{table}



Disponendo di una piattaforma Linux ho anche verificato il funzionamento sia di LyX sia di TeXmacs, ma alla fine sono tornato a usare i miei semplici editor ASCII con i quali non ho assolutamente nessuna limitazione per quel che riguarda la composizione dei miei testi. Bisogna dire che io faccio un grande uso di macroistruzioni adatte al mio modo di comporre; in generale le macroistruzioni personali non sono interpretabili da quei sistemi che sono un po' chiusi in se stessi, proprio per poter massimizzare la velocità di rendering.

Il difetto maggiore di questi programmi di composizione sincrona è che distraggono l'autore con l'aspetto del testo composto più o meno fedelmente a quello che si potrà ottenere davvero. Scrivere in modo WYSIWYM significa concentrarsi sul messaggio e non sul suo aspetto. Inoltre quando con quei programmi si esporta il documento in formato \LaTeX, il codice generalmente è penoso; dipende dal contenuto, ma generalmente è richiesto un pesante lavoro di pulizia e di riscrittura di alcune parti per renderle veramente scritte e marcate come si deve. Se poi bisogna fare delle modifiche, queste vanno comunque fatte sul file \LaTeX, perché in generale quei programmi non accettano macro personali o pacchetti per i quali non siano già predisposti. In sostanza è una gran perdita di tempo e farne uso non vale assolutamente la pena, nemmeno se si è principianti e si trova comodo ricorrere a qualcosa che ricorda l'uso dei word processor a cui si è già abituati; il principiante che cominci con questi software, non imparerà mai a usare \LaTeX\ come si deve.

\textcolor{red}{Nel seguito partirò dal presupposto che si abbia già una certa
conoscenza di \LaTeX\ e che si conosca la differenza fra \LaTeX,
\pdfLaTeX, \XeLaTeX\ e \LuaLaTeX.}

È possibile che qualcuno non abbia conoscenze sufficientemente approfondite a proposito del programma e mark-up \XeLaTeX; è un tipo di mark-up molto simile a quello di \LaTeX\ e di \pdfLaTeX, ma ha una gestione dei font diversa e può usare anche i font del sistema operativo, senza dover fare nessuna acrobazia per installarli e configurarli. Esso ha bisogno di attenzioni particolari per quel che riguarda la creazione dei file in formato PDF archiviabile, anche se l'uscita finale sia in formato PDF; gli si possono dare in pasto le figure nei formati accettati sia dal programma \prog{latex} (formato EPS), sia dal programma \prog{pdflatex} (formati PDF, PNG, JPG). Per la gestione delle lingue dispone di un suo pacchetto \pack{polyglossia} che è specifico per questo programma di composizione. L'uso di \XeLaTeX\ sta guadagnando terreno specialmente fra i linguisti; certamente è il motore di composizione più adatto alle tesi di carattere letterario, specialmente se contengono estesi brani composti con ``lettere'' non appartenenti all'alfabeto latino. Oggi \LuaLaTeX\ può sostituire completamente \XeLaTeX\ anche se talvolta sembra leggermente più lento nel suo lavoro; in compenso la sua integrazione con il linguaggio di scripting Lua gli permette di fare cose impossibili con gli altri programmi citati.

Merita qui segnalare con una tabellina le principali differenze fra \pdfLaTeX, \XeLaTeX\ e \LuaLaTeX, tabella~\ref{tab:pdflatex-vs-xelatex} nella pagina~\pageref{tab:pdflatex-vs-xelatex}.


%%%%%%%%%%%%%%%%%%%%%%%%%%

\chapter{L'uso di \textsf{TOPtesi}}\label{cap:uso}

La maggior parte delle macro definite nel pacchetto TOPtesi servono per comporre il frontespizio; siccome l'utente potrebbe desiderare di comporre il frontespizio in modo diverso da quello preimpostato in questa classe, prima esporrò come e perché la classe si comporta in un certo modo per comporre questa prima e importante pagina della tesi, poi esporrò che cosa bisogna fare se si usano altri pacchetti o altre tecniche per comporre il frontespizio.

Ci si ricordi comunque che l'aspetto generale dalla pagina, i font usati o gli altri font alternativi che si possono usare, il frontespizio, e altri elementi che costituiscono il ``look' della tesi composta con questa classe, sono abbastanza rigidi; non dico che questa classe sia rigorosamente del tipo ``prendere o lasciare'', ma sicuramente non è una classe generica da poter personalizzare a piacere in ogni dettaglio. Ci sono altre classi disponibili già distribuite con il sistema \TeX\ (aggiornato e completo); ne cito alcune: la classe \class{sapthesis}, impostata sulle prescrizioni dell'università di Roma La~Sapienza, la classe \class{suftesi}, che in realtà non serve solo per comporre tesi, ma serve anche per comporre diversi altri tipi di documenti in diversi formati e con diversi stili. Nessuna di queste classi è  configurabile a piacere, ma solo nei limiti delle personalizzazioni previste. Quella più configurabile mi pare sia la classe \class{suftesi}, ma anche questa, esplicitamente predisposta  per tesi nel campo delle scienze umane, forse è meno flessibile per le scienze sperimentali.



\section{Impostazioni standard di TOPtesi}

L'insieme di macro contenute in \textsf{TOPtesi} realizza esattamente
le specifiche di composizione per le monografie, tesi di laurea e
dissertazioni di dottorato da comporre presso il Politecnico di Torino;
l'acronimo TOP sta per \textcolor{red}{TO}rino \textcolor{red}{P}olitecnico e naturalmente gioca un po' anche sull'altro significato della parola inglese ``top''.

Tuttavia il pacchetto non è  stato creato solo per il Politecnico
di Torino, il cui nome è quello preimpostato per l'Ateneo;  invece  la stringa ``Facoltà di~'' e il nome della facoltà sono vuote. Questo dipende dall'ultima riforma universitaria approvata in Italia, che ha obbligato gli atenei a riformulare gli statuti in modo da eliminare sostanzialmente le facoltà; in alcuni casi strutture didattiche equivalenti alle facoltà sono sopravvissute sotto altro nome, in altri sono sparite definitivamente. In alcuni casi il coordinamento delle attività didattiche è passato ai dipartimenti. Lo studente deve quindi informarsi presso la sua segreteria didattica competente per sapere come debba essere intestato il frontespizio delle monografie, tesi magistrali o dissertazioni di dottorato. I comandi predisposti in questo pacchetto consentono di personalizzare la composizione del proprio lavoro finale per molti atenei e molte strutture didattiche e, oserei dire, in molte lingue.

Presso il Politecnico di Torino, l'ateneo preso come riferimento, le facoltà non esistono più; perciò il valore preimpostato per la stringa ``Facoltà di~'' è nullo; un test per la composizione del frontespizio verifica la presenza di questa stringa nulla e omette completamente di indicare qualsiasi informazione sulla struttura didattica; in altri atenei potrebbe essere utile inserire nel file di configurazione qualcosa come:
\begin{verbatim*}
\StrutturaDidattica{Dipartimento di }
\struttura{Ingegneria Strutturale}
\corsodilaurea{Ingegneria di Ponti e Strade}
\end{verbatim*}
Il nome del corso di laurea potrebbe essere facoltativo se la struttura didattica competente cura la didattica di un solo corso di laurea, ma quasi sicuramente questo corso non ha lo stesso nome della struttura didattica, quindi è opportuno specificarlo anche in questi casi; se invece la segreteria competente non richiede il nome di nessuna struttura didattica, allora il corso degli studi deve essere necessariamente indicato. Si noti: per facilitare l'inserimento di queste informazioni sono stati predisposti i comandi \cs{StrutturaDidattica}, equivalente a \cs{FacoltaDi}, e \cs{struttura}, equivalente a \cs{facolta}.

\iffalse % METACOMMENTO
Tenete conto che \textsf{TOPtesi} è già stato usato anche per
scrivere tesi in filologia classica greca e in filologia copta;
\textsf{TOPtesi} non è limitato solo a scritti delle discipline
cosiddette esatte, quelle che fanno un grande uso del linguaggio
matematico, ma è altrettanto utile per scrivere testi in
discipline umanistiche, perché \pdfLaTeX\ e \XeLaTeX\ sono
altrettanto precisi nel comporre la matematica di quanto lo siano
nel comporre una poesia o un romanzo o un'edizione critica.
\fi % FINE METACOMMANTO

\subsection{Dove sono i file di \textsf{TOPtesi}?}
La distribuzione di \textsf{TOPtesi} contiene molti file; ma ogni distribuzione moderna del sistema \TeX\ li carica tutti senza che dobbiate intervenire a mano. Tuttavia è importante sapere dove trovare questi file.

L'installazione del sistema \TeX\ prevede che i suoi numerosissimi file siano installati in un certo numero di strutture di cartelle, chiamate ``alberi''; ogni albero ha una radice; normalmente la radice della distribuzione si chiama \texttt{texmf-dist}, ma sulla vostra macchina e con il vostro sistema operativo potrebbe avere un altro nome. Esiste anche un albero radicato nella vostra ``home''; nei sistemi di tipo UNIX essa si indica simbolicamente con la tilde e la radice del vostro albero personale potrebbe chiamarsi \texttt{\textasciitilde/texmf}; sulle macchine Windows invece dovete cercare in \texttt{C:\char92Documents and settings} seguito dal vostro \emph{user name}, oppure da \texttt{All users}, oppure da \texttt{Users} seguito dal vostro nome; magari ci sono ancora altri rami di albero da percorrere, ma poi si trova \texttt{texmf} o \texttt{localtexmf}. Nei sistemi Mac la radice del vostro albero personale è sotto la vostra \texttt{\textasciitilde/Library}. 

L'albero personale non è mai inizialmente predisposto con l'installazione del sistema \TeX; ve lo dovete creare voi. In esso creerete una struttura di rami identica a quella degli alberi di sistema, magari sarà un albero più semplice e non così ramificato come gli alberi di sistema, ma i gruppi di cartelle presenti dovranno essere innestati nello stesso modo. Qui metterete i vostri file personali di classe, di stile, di definizioni, eccetera. Quando aggiornate la vostra distribuzione del sistema \TeX\ le cartelle degli alberi di sistema potranno essere completamente riscritte, mentre il vostro albero personale non verrà assolutamente modificato.

Tutti i file del pacchetto \textsf{TOPtesi} vengono caricati nell'albero della distribuzione; i file della documentazione lungo il ramo \texttt{doc}; i file sorgente, lungo il ramo \texttt{source}; i file ``eseguibili'' in una cartella lungo il ramo \texttt{tex}; fra questi file ``eseguibili''\footnote{Chiamare ``eseguibili'' i file che vengono usati da \LaTeX\ è molto improprio: si tratta di file necessari durante l'esecuzione del programma, ma non sono quei file ad essere eseguiti nel senso informatico del termine.} c'è anche il file \file{toptesi.cfg}. Copiate questo file nella vostra cartella di lavoro come spiegato qui di seguito, cambiategli il nome ma non l'estensione, e modificatelo secondo le vostre necessità.


\subsection{Il file di configurazione}\label{sec:configurazione}


È comodo, ma non è obbligatorio, disporre di un file di configurazione. Esso serve essenzialmente per contenere i comandi con i loro argomenti necessari per comporre il frontespizio della tesi. Ma se usate qualche altro pacchetto o qualche altro metodo per creare il frontespizio, il file di configurazione, anche se presente, non viene usato.

Il  pacchetto \textsf{TOPtesi} contiene già alcuni file con configurazioni di default da usare come modello; chiunque se ne può copiare uno in un altro file mantenendo l'estensione \texttt{.cfg} e con il nome identico a quello del file principale della tesi. Questo file, se esiste, verrà letto durante l'esecuzione del programma e verranno eseguiti tutti i comandi che esso contiene; se quindi il laureando ne fa uso, deve servirsi del modello per cambiarne i dati e metterci le informazioni che ritiene utili per la sua tesi. Se non vuole servirsi del file di configurazione deve solo inserire nel suo file principale tutte le informazioni necessarie al frontespizio (e al retrofrontespizio, se lo vuole usare) dopo l'istruzione \verb|\begin{document}| ma prima di specificare l'ambiente \amb{frontespizio} o il comando \cs{frontespizio}.

{\tolerance=3000 Il laureando perciò può usare un diverso file di configurazione, sempre chiamato  \meta{mainfile}\texttt{.cfg}, collocato nella stessa cartella dove risiede il materiale da comporre. Per esempio, potrebbe comporre la monografia di laurea creando da qualche parte la cartella \texttt{/monografia} e in questa cartella si crea \meta{mainfile}\texttt{.cfg} nel quale scrive quello che gli è necessario; in questa cartella sistema anche i file per comporre la monografia. Quando due anni dopo compone la sua tesi di laurea magistrale, crea da qualche parte la cartella \texttt{/tesi} e ci mette dentro un altro  file \meta{mainfile}\texttt{.cfg} con la sua configurazione adattata alla tesi magistrale; in questa stessa cartella mette i file relativi alla sua tesi magistrale. Se dopo tre anni prende il dottorato e si scrive la sua dissertazione dottorale, si crea da qualche parte la cartella \texttt{/dissertazione} e vi mette dentro un altro file \meta{mainfile}\texttt{.cfg} che configura per la sua dissertazione; sempre in questa stessa cartella mette i file necessari per comporre la sua dissertazione. Ovviamente nei tre casi \meta{mainfile} è il nome del file principale che, altrettanto ovviamente, sarà  diverso per ciascuna delle tre tesi.\par}

{\tolerance=3000 Nota bene: il modulo \pack{topfront} che contiene i comandi per la composizione del frontespizio oltre al comando per caricare l'eventuale file di configurazione, chiama quest'ultimo file col nome \verb|\jobname.cfg|. La macro \verb|\jobname| è la stessa usata da \LaTeX\ per conservare il nome senza estensione del main file del documento che si sta componendo; quindi se il main file si chiamasse \texttt{GiorgioRossi"TesiMagistrale.tex} il file di configurazione associato a questo main file si deve chiamare obbligatoriamente \texttt{GiorgioRossiTesiMagistrale.cfg}. Siccome i sistemi operativi di tipo UNIX distinguono le lettere maiuscole dalla minuscole nei nomi dei file, se il file di configurazione si chiamasse \texttt{giorgiorossitesimagistrale."cfg} non verrebbe letto da Linux o da Mac~OS~X perché il nome proprio del file è scritto in tutte lettere minuscole, e non sarebbe uguale al nome proprio del main file che contiene anche delle lettere maiuscole. Per evitare problemi si consiglia di rispettare le maiuscole e le minuscole anche agli utenti dei sistemi operativi Windows, anche se questi sistemi operativi non distinguono i nomi dei file in base al fatto che contengano lettere maiuscole o minuscole.\par}


\section{Pronti? Via!}

Ora siete pronti per comporre la vostra tesi o monografia o
dissertazione. Ricordate solo di non giocare con quei pochi
comandi di \LaTeX\ che permettono di fare pasticcetti alla
WYSIWYG; \LaTeX\ compone da solo benissimo; al massimo, alla fine,
quando tutto sarà finito e non sarà più necessario apportare
correzioni, potrete anche inserire qualche spazio fine positivo o
negativo per fare degli aggiustamenti di seconda o terza
approssimazione. Ricordatevi però che, anche se non c'è limite
al meglio (ma nemmeno al peggio!), \LaTeX\ lavora benissimo da
solo purché non lo si disturbi con interventi non professionali; se non siete un tipografo professionista, lasciate perdere ogni modifica tipografica; \LaTeX, lo ripeto, lavora generalmente meglio di molti tipografi; è proprio il motivo per il quale Knuth ha realizzato il sistema \TeX.

\section{I file accessori}

Qui verranno descritti alcuni usi del modulo \pack{topfront} e del modulo \pack{topcoman}. Il primo serve per comporre il frontespizio; il secondo mette a disposizione alcuni comandi utili. Né l'uno né l'atro sono indispensabili, ma sono utili. Questi due pacchetti sono sempre caricati da TOPtesi, tranne in alcuni casi specificati meglio più avanti. Ma il fato di venire o non venire caricati dipende dall'utente, cioè dalle opzioni che egli specifica alla classe \class{toptesi} e dai pacchetti aggiuntivi che egli vuole usare; in altre parole l'utente non ha mai bisogno di caricare esplicitamente questi due pacchetti ma può solo esplicitamente specificare azioni che ne impediscono il caricamento.

\goodpagebreak

\subsection{Comporre il frontespizio con topfront}\label{sec:frontespizi}

Il pacchetto TOPtesi contiene fra le sue parti il file
\pack{topfront} che serve solo per comporre il 
frontespizio; può essere usato per comporre il solo frontespizio separatamente dalla tesi, ma in questo caso, talvolta, può essere meglio ricorrere a quanto esposto nel paragrafo~\ref{sec:solofrontespizio}. 

In questo paragrafo, comunque spiego come comporre il frontespizio indipendentemente dal fatto che rappresenti il frontespizio isolato della tesi o sia quello della tesi intera.
Questo pacchetto in effetti, oltre a leggere l'eventuale file di
configurazione, contiene solo i comandi per definire gli elementi
del frontespizio e per comporlo. 

Comporre il frontespizio isolatamente può tornare utile in
diverse circostanze, per esempio quando si deve ancora modificare
la tesi quasi ultimata, ma è necessario cercare il o i relatori per far
loro firmare alcune copie del solo frontespizio.

Basta predisporre un piccolo file come questo, salvandolo con un nome a
piacere, per esempio, con grande fantasia, \file{myfile.tex}:
\begin{flushleft}\ttfamily\obeylines
\% !TEX TS-program = XeLaTeX
\% !TEX encoding = UTF-8 Unicode

\verb|\documentclass[12pt]{toptesi}|
\verb|%%%%%%%%%%%%%%%%%%%%%%%%%%%%%%%%%%%%%%%%%%%%%%%%%%%%%%%%%%|
\% Impostazioni per comporre con XeLaTeX
\verb|\setmainfont[Ligatures=TeX]{TeX Gyre Termes}% o altro font|
\verb|\setotherlanguage{french}% o altra lingua|
\verb|%%%%%%%%%%%%%%%%%%%%%%%%%%%%%%%%%%%%%%%%%%%%%%%%%%%%%%%%%%|
~
\meta{il resto del preambolo}
~
\verb|\begin{document}|
~
\verb|\begin{frontespizio*}|
\verb|\ateneo{Università di Marconia}|
\verb|\logosede{logouno}|
\verb|%%%%%%%%%%%%%%%%%%%%%%%%%%%%%%%%%%%% Tesi magistrale|
\verb|\corsodilaurea{delle Telecomunicazioni}|
\verb|\titolo{Titolo della tesi\\ di laurea magistrale}|
\verb|\sottotitolo{Sottotitolo della tesi di laurea magistrale}|
\verb|\relatore{prof.\ Enrico Rosa}|
\verb|\sedutadilaurea{Dicembre 2114}|
\verb|\candidata{Susanna Rossi}|
\verb|\secondacandidata{Laura Bruni}|
\verb|%\retrofrontespizio{...}|
\verb|\end{frontespizio*}|
~
\meta{il resto della tesi}
~
\verb|\end{document}|
\end{flushleft}

La parte che ho indicato con \meta{il resto del preambolo} e \meta{il resto della tesi} sono facoltative; certo sono necessarie quando si compone l'intera tesi; per comporre il solo frontespizio si possono omettere.

Perciò bisogna fare attenzione: se si compone il frontespizio separatamente dalla tesi, la riga \cs{documentclass} deve essere identica alla corrispondente riga del vostro main file della tesi; siccome \pack{toptesi} può usare diversi formati di carta, è necessario che il frontespizio sia composto su carta dello stesso formato. Non è invece necessario usare come font normale un font dello stesso corpo della tesi; con carte particolarmente piccole potrebbe essere più opportuno usare il corpo di 10\,pt invece che quello di 12\,pt. Se si deve scrivere in un'altra lingua diversa dall'italiano e dall'inglese lo si specifica con \cs{setotherlanguage} con le modalità che verranno esposte più avanti. Si noti che l'esempio precedente è impostato per comporre il frontespizio con \XeLaTeX, ma se si lavora con \pdfLaTeX, allora vanno specificati i font da usare, e le codifiche di entrata e di uscita; bisogna anche impostare la lingua nella maniera specifica di \pack{babel}. Non scendo nei dettagli, perché fra i vari modelli forniti con il pacchetto TOPtesi, ce ne sono alcuni predisposti per lavorare con \pdfLaTeX\ e altri con \XeLaTeX\ o \LuaLaTeX.

Comunque, se si vuole usare \prog{pdflatex}, la parte fra le prime due righe \verb|%%%|
va sostituita,  per esempio, con:
\begin{verbatim}
\usepackage[utf8]{inputenc}
\usepackage[T1]{fontenc}
\usepackage{newtxtext}% o altro font
\end{verbatim}
Ma se si deve scrivere il frontespizio in un'altra lingua bisogna agire come spiegato poco più avanti.

In realtà il pacchetto \pack{topfront} lavora bene in italiano e distingue i singolari dai plurali, i maschili dai femminili; non fa nulla di particolare se la lingua di composizione del frontespizio è l'inglese o un'altra lingua; ricordate che l'inglese è già impostato ma non attivato; lo si attiva nel main document con il comando \cs{english}. Se si vuole  comporre il frontespizio in un'altra lingua, allora, usando \prog{xelatex}, la si specifica mediante il comando  \cs{setotherlanguage}; invece usando \pdfLaTeX\ la si indica fra le opzioni di \cs{documentclass} e poi, prima di usare l'ambiente \amb{frontespizio} o \cs{frontespizio*} o i comandi corrispondenti, la si imposta con \cs{selectlanguage}. Poi si inseriscono i seguenti comandi, modificando gli argomenti in accordo con le restrizioni burocratiche per i frontespizi delle tesi nella vostra università:
\begin{Verbatim}[fontsize=\small]
\retrofrontespizio{This work is subject to the Creative Commons Licence}
\DottoratoIn{PhD Course in\space}
\NomeMonografia{Bachelor Degree Thesis}
\TesiDiLaurea{Master Degree Thesis}
\NomeDissertazione{PhD Dissertation}
\InName{in}
\CandidateName{Candidate}% or Candidates
\AdvisorName{Supervisor}% or Supervisors
\TutorName{Tutor}
\NomeTutoreAziendale{Internship Tutor}
\CycleName{cycle}
\NomePrimoTomo{First volume}
\NomeSecondoTomo{Second Volume}
\NomeTerzoTomo{Third Volume}
\NomeQuartoTomo{Fourth Volume}
\logosede{logouno,logodue}% one logo or a comma separated list of logos
\end{Verbatim}

Passando uno di questi  file a \prog{xelatex} o a \prog{pdflatex}, si ottiene
un documento con il frontespizio e, facoltativamente, il retrofrontespizio, che si possono stampare in varie copie per gli scopi detti sopra. Ricordate che \class{toptesi} precarica \pack{babel} o \pack{polyglossia} a seconda che usiate \prog{pdflatex} oppure \prog{xelatex} o \prog{lualatex}; non avete quindi l'obbligo di invocare di nuovo questi pacchetti se componete il solo frontespizio usando \class{toptesi}, ma dovete specificare la lingua che volete usare nei modi descritti sopra.

Il retrofrontespizio non è necessario in senso assoluto quando volete produrre il frontespizio isolato. Non è nemmeno necessario quando volete comporre la tesi. Tuttavia il comando è disponibile e permette di aggiungere la dichiarazione legale in merito alla licenza; il retrofrontespizio di questa documentazione è stato composto con i comandi seguenti\footnote{Questo codice contiene dei comandi \cs{ref} perché è servito per comporre l'intero documento  dove le etichette invocate sono definite. Se si vuole usare questo codice per comporre il retrofrontespizio assieme al frontespizio, ma isolatamente dalla tesi (lo sconsiglio), bisogna sostituire quei comandi e i loro argomenti con i valori reali.}:
\begin{verbatim}
\retrofrontespizio{Questo testo è libero secondo le condizioni 
stabilite dalla \LaTeX\ Project Public Licence (LPPL) riportata 
nell'appendice~\ref{ch:LPPL} alla pagina~\pageref{ch:LPPL}.
 
\bigskip
 
\noindent Composto con \XeLaTeX\ il \today
\vspace*{5\baselineskip}}
\end{verbatim}


Non dovete dimenticare che, se volete usare \prog{xelatex} o \prog{lualatex}, l'editor con cui scrivete i vostri file deve essere configurato per registrarli sul disco fisso con la codifica \texttt{UTF-8}.

Nell'esempio di frontespizio presentato sopra si è usato l'ambiente \amb{frontespizio*} che è disponibile a partire dalla versione del pacchetto TOPtesi 5.85. I comandi \cs{frontespizio} e \cs{frontespizio*} funzionano ancora, ma suggerisco di usare gli ambienti.

Vediamo le differenze: nella figura~\ref{fig:4frontespizi} si vedono sempre i loghi all'inizio della pagina, non ci sono filetti di nessun genere, il nome dell'università ci può essere o può mancare, le diverse tipologie di tesi hanno informazioni diverse. I loghi sono tutti in testa alla pagina ma potrebbero essere anche nella parte inferiore; la differenza dipende dall'usare l'ambiente \amb{frontespizio} che mette i loghi in testa, o l'ambiente \amb{frontespizio*} che mette i loghi nella parte inferiore della pagina; la figura~\ref{fig:frontespizi} nella pagina~\pageref{fig:frontespizi}, infatti, mostra i frontespizi di sinistra che sono stati composti con l'ambiente \amb{frontespizio} e quelli di destra composti con l'ambiente \amb{frontespizio*}. Il nome dell'ateneo appare o non appare a seconda che il comando \cs{ateneo} contenga un argomento vuoto o un argomento con un nome esplicito. Le altre informazioni dipendono da quali comandi si sono usati per caratterizzare la tesi triennale, magistrale o dottorale. Questi comandi verrano descritti dettagliatamente più avanti.

\begin{figure}[p]\centering
\includegraphics[height=0.875\textheight]{FrontespiziAssemblati1}
\caption[Frontespizi composti con lo stile standard]{Frontespizi composti con lo stile standard. In questo stile mancano i filetti e i loghi delle università possono essere messi in posti diversi della pagina; il nome dell'università può essere presente esplicitamente oppure può essere compreso nel logo.}\label{fig:4frontespizi}
\end{figure}


Nella figura~\ref{fig:altri4frontespizi} sono mostrati invece i quattro frontespizi che si ottengono quando alla classe (o anche al solo modulo \pack{topfront} quando lo si usa con una classe diversa da \class{toptesi}) viene specificata l'opzione \opt{classica}. Attenzione: le opzioni \opt{classica}, e quelle che da essa dipendono, come l'opzione \opt{oldstyle} e l'opzione \opt{autoretitolo}, possono venire specificate al pacchetto, quando questo viene usato con una classe diversa da \class{toptesi}; alcune non hanno molto a che vedere con il frontespizio, sebbene possano venire usate anche all'interno del pacchetto \pack{topfront}. 

In particolare, se per esempio, si volesse comporre il frontespizio con lo stile \opz{classica} usando la classe \class{book}, ma con il frontespizio composto con \pack{topfront}, basterebbe impostare nel preambolo:
\begin{flushleft}\ttfamily\obeylines
\char92documentclass\oarg{opzioni}\Arg{book}
...
\char92usepackage\Oarg{classica}\Arg{topfront}
...
\char92begin\Arg{document}
\end{flushleft}

Nello stile \opz{classica} si nota che il nome dell'ateneo è separato dal resto della pagina da un filetto orizzontale; analogamente l'anno accademico in calce alla pagina è separato da un filetto orizzontale. I candidati sono chiamati ``Laureandi''. Il blocco contenente i nomi dei relatori e correlatori e quello contenente i nomi dei laureandi sono allineati superiormente e non sono sfasati come nello stile standard. La seduta di esame è indicata con la dicitura ``Anno accademico'' in maiuscoletto e l'anno, o l'intervallo di anni è indicato con le cifre arabe minuscole (old style). Con questo stile più classico, il logo o i loghi sono collocati fra i titolo e i blocchi dei relatori e dei laureandi.

Si noti infine che in entrambi gli stili esistono esempi con due loghi, per i quali i nomi dei due atenei vanno scritti in forma un poco ellittica, ma piuttosto antiestetica; in generale potrebbero formare una riga così lunga da non entrare nella pagina fisica. Si veda però più avanti nel paragrafo~\ref{sec:casi-particolari}.

Lo si potrebbe considerare una ``feature'', una particolarità del pacchetto TOPtesi. In realtà non conosco altre classi o moduli di estensione dove sia possibile fare riferimento a diversi loghi e a diversi nomi.

Quello che consiglierei in questi casi sarebbe di aggiungere il nome dell'ateneo al logo, se già non lo contenesse, e lo farei un con un font di contrasto, per esempio con un font senza grazie maiuscolo o maiuscoletto con iniziali maiuscole, magari su due righe collocato sotto il logo vero che non contenga a sua volta già il nome dell'ateneo; come per esempio nel logo modificato mostrato nella figura~\ref{fig:logomodificato}.

\begin{figure}
\centering
\begin{minipage}[t]{30mm}\centering
\includegraphics[width=30mm]{logodue}\\[3pt]
\fontsize{17}{16}\sffamily U\simulatedSC{NIVERSITÀ}\\
\simulatedSC{DI} M\simulatedSC{ARCONIA}
\end{minipage}
\caption{Un logo modificato con l'aggiunta del nome dell'ateneo}\label{fig:logomodificato}
\end{figure}

Fatta questa modifica per tutti i loghi privi del nome dell'università che è necessario usare, si compone il frontespizio con lo stile standard  usando semplicemente un nome vuoto come argomento di \cs{ateneo}. Si veda poco più avanti come usare una pluralità di loghi.


Vale la pena di ricordare che la retrocompatibilità con i comandi \cs{frontespizio} e \cs{frontespizio*} è conservata. 

Perciò merita ricordare la molteplicità di configurazioni offerta dal modulo \pack{topfront} mediante la tabella comparativa~\ref{tab:modalitaperfrontespizi}.

\begin{table}\caption{Varie modalità di composizione del frontespizio}\label{tab:modalitaperfrontespizi}
\centering
\begin{tabularx}{\textwidth}{*3{>\raggedright X}}\toprule
Comando o ambiente	& Stile standard			& Stile classica 	\tabularnewline
\midrule
\amb{frontespizio}	& loghi in testa			& loghi in basso	\tabularnewline
\cs{ateneo} vuoto 	& senza nome dell'ateneo	& messaggio d'errore\tabularnewline
\cs{ateneo} non vuoto& Nome dell'ateneo in testa & nome dell'ateneo in testa
\tabularnewline
filetti				& assenti					& presenti			\tabularnewline
\midrule
\amb{frontespizio*}	& loghi in basso			& loghi in basso	\tabularnewline
\cs{ateneo} vuoto	& senza nome dell'ateneo	& messaggio d'errore\tabularnewline
\cs{ateneo} non vuoto& nome dell'ateneo in testa	& nome dell'ateneo in testa	\tabularnewline
filetti				& assenti					& presenti			\tabularnewline
\midrule
\cs{frontespizio}	& come \amb{frontespizio}	&come \amb{frontespizio}
																	\tabularnewline
\midrule
\cs{frontespizio*}	& come \amb{frontespizio*}	& come \amb{frontespizio*}																					\tabularnewline
\bottomrule
\end{tabularx}
\end{table}




\begin{figure}[p]\centering
\includegraphics[height=0.875\textheight]{FrontespiziAssemblati2}
\caption[Frontespizi composti con lo stile \texttt{classica}]{Frontespizi composti con lo stile \texttt{classica}. Con questo stile si ha una disposizione diversa delle varie informazioni e i loghi sono sempre posti nella parte inferiore della pagina. Il nome dell'università è sempre presente.}\label{fig:altri4frontespizi}
\end{figure}


\subsection{I comandi accessori di topcoman}
L'altro file accessorio  \pack{topcoman} contiene diversi utili comandi, alcuni già incorporati nella definizione della lingua italiana del sistema \TeX\ (solo se si usa \pack{babel}). Tuttavia se dovete scrivere in una lingua diversa dall'italiano quei comandi non sono disponibili e qui viene utile avere quei comandi in un pacchetto distinto da \pack{babel} e non soggetto a test sulla lingua in uso. Alcuni di questi comandi, non definiti da \pack{poliglossia} quando si scrive in italiano, si rendono particolarmente utili quando si compone con \prog{xelatex} o \prog{lualatex}.

Il file contiene anche un comando per stampare in modo verbatim il contenuto di un file esterno, per esempio il listato di un programma; ci sono altri pacchetti \LaTeX\ che consentono di stampare il listato di un programma in modo anche migliore, per esempio il pacchetto \pack{listings}, ma, visto che può servire per la tesi e, visto che una volta che lo si conosce
potrebbe piacere e poi si vorrebbe continuare ad usarlo, è meglio
disporne come pacchetto separato. Non sarebbe difficile per
esempio stamparsi in pulito il listato del proprio programma contenuto nel file 
\texttt{myprogram.c}: si scrive semplicemente un file del tipo
\begin{verbatim}
\documentclass{report}
\usepackage{topcoman}
\begin{document}
\pagestyle{plain}
\chapter*{Il programma myprogram.c}
\listing{myprogram.c}
\end{document}
\end{verbatim}
e lo si passa a \prog{pdflatex} per ottenere poi una stampa
ben composta da conservare per documentazione; per evitare che le
righe  fuoriescano dalla larghezza normale del testo, o, peggio
ancora, fuori dalla carta, sarebbe opportuno che le righe del
programma non fossero più lunghe di una ottantina di caratteri 
per la carta A4, proporzionalmente di meno per formati di carta più piccoli.

\section{La composizione del frontespizio}
Non è generalmente necessario inserire nel frontespizio il logo dell'ateneo; controllate presso la vostra segreteria competente le regole fissate per la composizione del frontespizio. Se il logo è richiesto, allora la segreteria vi dice anche quale logo usare e da dove scaricarne il file che ne contiene la descrizione grafica; \textcolor{red}{siate precisi nel richiedere le limitazioni legali d'uso del logo dell'ateneo, perché il logo è proprietà esclusiva dell'ateneo e l'uso indebito potrebbe essere fonte di problemi legali non indifferenti.} Siccome tali limitazioni esistono, il pacchetto TOPtesi non viene più distribuito con i file dei loghi di diversi atenei, come veniva fatto nelle prime edizioni.

Controllate se si tratta di un file vettoriale (estensione \file{.pdf} oppure \file{.eps}) oppure bitmapped (estensione \file{.jpg}, \file{.png}, \file{.bmp}, \file{.wmf}, \file{.tiff},\dots). Ricordate che \prog{pdflatex}, \prog{xelatex} e \prog{lualatex} possono incorporare solo i formati \file{.pdf}, \file{.eps}\footnote{Dall'edizione di \TeXLive 2010 anche \prog{pdflatex} è in grado di incorporare file in formato \file{.eps}, mentre prima era necessario convertirli preventivamente in formato \file{.pdf}.}, \file{.jpg}, \file{.png} o \file{.mps}\footnote{I file vettoriali in formato \file{.mps} sono ottenuti con il programma \MP; i file delle figure create con questo programma hanno di default una estensione numerica; si può impostare il programma per assegnare l''estensione \file{.mps} oppure si può cambiare il nome e l'estensione.}. Se il file bitmapped viene fornito in un formato diverso da questi cinque, è necessario convertirlo in uno dei formati accettati. 

Vale la pena di ricordare che il formato \file{.jpg} va bene per immagini a colori sfumati come le fotografie, mentre il formato \file{.png} è più adatto per disegni al tratto con righe fortemente contrastate rispetto allo sfondo. In generale, quindi, sarebbe meglio convertire nel formato \file{.png} un logo bitmapped fornito in un formato diverso da quelli elencati sopra, ma il formato \file{.png} potrebbe contenere delle ``trasparenze'' che non sono accettabili con le prescrizioni PDF/A dei documenti archiviabili e in questo caso bisogna ricorrere al formato \file{.jpg}. Potendo disporne, sarebbe sempre meglio usare un disegno veramente vettoriale del logo.

Merita segnalare che le tesi svolte nell'ambito di collaborazioni fra atenei possono richiedere i loghi dei diversi atenei partecipanti all'accordo; segnalo la Scuola Interpolitecnica di Dottorato condotta nell'ambito di un accordo fra i politecnici di Bari, Milano e Torino. Ricordo i programmi di doppia laurea condotti nell'ambito di accordi fra l'ateneo di appartenenza e un altro ateneo, generalmente europeo. Ricordo il programma europeo Erasmus Mundus dove il titolo di studio viene rilasciato da consorzi di almeno quattro atenei di nazioni europee diverse. In queste circostanze è spesso necessario inserire nel frontespizio i loghi di tutti gli atenei partecipanti alla convenzione.

\subsection{Il logo dell'ateneo per l'uso con TOPtesi e TOPfront}\label{ssec:loghi}

Il pacchetto \pack{topfront} contiene un comando molto potente, \cs{logosede}, che accetta la sintassi seguente:
\begin{flushleft}\obeylines
\cs{logosede}\oarg{altezza}\marg{file del logo}
\medskip
oppure
\medskip
\cs{logosede}\oarg{altezza}\marg{lista di nomi di file di loghi separati da virgole}
\end{flushleft}
dove il valore prefissato per l'\meta{altezza} del logo è 25\,mm per la carta A4, ed è scalato proporzionalmente per carte di formato diverso; con l'argomento facoltativo si può specificare un'altezza leggermente maggiore o minore del valore predefinito. L'argomento \meta{file del logo} o \meta{lista di nomi di file di loghi separati da virgole} è costituito da un solo nome di file o da una lista di nomi di file separati da virgole, corrispondenti ciascuno al logo di un diverso ateneo. Questi file devono avere i formati corrispondenti a una delle cinque estensioni consentite, ma non è necessario specificare il nome del file completo di estensione; questa specificazione potrebbe essere necessaria solo nel caso in cui si disponga del logo di uno stesso ateneo in diversi formati grafici (o, cosa da non fare mai, immagini o loghi diversi con il nome del loro file uguale e l'estensione diversa; questo genere di pasticci è fonte di numerosi errori, facilmente evitabili se si mantiene in ordine la scelta dei nomi e delle estensioni).

Il programma codificato nel pacchetto \pack{toptesi} provvede a scalare tutti i loghi all'\meta{altezza} specificata e li compone affiancati su una sola riga distanziati mediante uno spazio predefinito di 3\,em; lo si può impostare ad un valore leggermente maggiore o minore tramite il comando \cs{setlogodistance}, per esempio:
\begin{verbatim}
\setlogodistance{2em}
\end{verbatim}

Se la riga di loghi, eventualmente scalati in altezza e distanziati dello spazio predefinito o esplicitamente specificato, risultasse più larga della giustezza della pagina del titolo, il programma provvede ad uno scalamento dell'intera riga di loghi in modo che non superi la giustezza della pagina. Un solo logo viene composto centrato; due o più loghi vengono ugualmente composti centrati ma distanziati come si è detto. Seguono alcuni esempi, ma si tenga presente che i loghi usati per questi esempi sono solo dei disegni creati in modo tale da avere la forma di un logo; loghi veri di diverse università non sono distribuiti insieme a TOPtesi, a causa delle restrizioni sulla loro proprietà legale; solo il vero logo del Politecnico di Torino appare nel frontespizio, perché questa è una pubblicazione ufficiale di quell'ateneo. Un solo logo appare come nel frontespizio di questo manuale; due loghi appaiono così:\\*[\medskipamount]
\makebox[\textwidth]{\logosede{logouno,logodue}\printloghi}

\goodpagebreak[10]

\noindent Tre loghi appaiono così:\\*[\medskipamount]
\makebox[\textwidth]{\logosede{logodue,logotre,logoquattro}
\printloghi}\medskip


\noindent Quattro loghi, nonostante sia stata conservata  l'\meta{altezza} predefinita, producono una riga di loghi di altezza ridotta per consentire che appaiano tutti e quattro nella stessa riga:\\*[\medskipamount]
\makebox[\textwidth]{\logosede{logoquattro,logouno,logotre,logodue}
\printloghi}\medskip

Si noti che il logo in testa alla pagina del frontespizio di questo documento è una esplicita richiesta espressa dal Politecnico di Torino; molte altre sedi come la quasi totalità delle case editrici che mettono il loro logo nel frontespizio, lo mettono nella metà inferiore della pagina; ma la tesi di laurea o di dottorato è una documento particolare, il cui frontespizio deve soddisfare ad alcune esigenze burocratiche, e non può, quindi, soddisfare a criteri estetici molto stringenti. Per questo motivo, lo si vedrà meglio avanti, il modulo \pack{topfront} consente una variante del comando per creare i frontespizi: essa intesta la pagina col nome dell'ateneo e mette i loghi nella metà inferiore della pagina. Se, come per il Politecnico di Torino, il nome fa parte del logo\footnote{Il logo di questo Politecnico è fatto di due parti: quella tonda di sinistra prende il nome di ``marchio'', mentre quella letterale a destra prende il nome di ``logotipo''; l'Ufficio Comunicazione e Immagine che si occupa di questi aspetti formali ha emesso una normativa specifica sui colori e le forme sia del marchio sia dei font usati per il logotipo.} non è il caso di ripetere il nome dell'ateneo nella parte alta della pagina perché vi appare già nel logotipo.
Comunque, volendo, si può inserire in testa alla pagina sia il logo, sia il nome generico dell'ateneo sia il nome proprio dell'ateneo; come fare lo sì è già descritto in precedenza e lo si è riassunto nella tabella~\ref{tab:modalitaperfrontespizi} nella pagina~\pageref{tab:modalitaperfrontespizi}.

\section{Comporre il frontespizio senza ricorrere a TOPfront}\label{sec:solofrontespizio}
Potrebbe essere necessario comporre il frontespizio in modo diverso da come viene composto tramite i comandi contenuti nel pacchetto \pack{topfront}, sia pure ricorrendo alle due varianti dell'ambiente \amb{frontespizio} e \amb{frontespizio*}.

Esistono alcune strade: una è quella di usare l'ambiente \amb{titlepage} e scriverci dentro quello che si vuole in accordo con le prescrizioni del proprio ateneo; l'altra è quella di usare altri pacchetti espressamente creati per comporre frontespizi, in particolare il pacchetto \pack{frontespizio}.

Ci sono diversi motivi per i quali un laureando o dottorando vorrebbe comporre un frontespizio diverso da quello predefinito da TOPtesi con il suo modulo \pack{topfront}.
\begin{enumerate}[noitemsep]
\item Le specifiche per le tesi di una data università sono inconciliabili con quelle preimpostate da TOPtesi.
\item Le informazioni da inserire nel frontespizio sono diverse da quelle pre"impostate da TOPtesi.
\item Il laureando non deve soddisfare a specifiche particolari ma vuole comporre un frontespizio personalizzato a suo piacimento.
\end{enumerate}

A seconda dei motivi per i quali il laureando non vuole o non può usare il layout e la strutturazione del frontespizio predisposto da TOPtesi, le due strade offrono alcuni vantaggi e svantaggi.

La prima via, quella di usare l'ambiente \env{titlepage}, permette il massimo della libertà e consente di comporre il frontespizio come il laureando lo vuole, ma ha lo svantaggio che il laureando dentro quell'ambiente deve specificare ogni dettaglio di composizione. Se conosce bene \LaTeX\ e ha buone conoscenze tipografiche può ottenere un risultato gradevole, ma può anche cadere negli errori tipografici più comuni e produrre un frontespizio impresentabile.

La seconda via può dare risultati buoni, grazie alla possibilità di personalizzazione che offre, in particolare con il pacchetto \pack{frontespizio}. Ma non si può fare sempre tutto; anche questo pacchetto è preimpostato per produrre  il frontespizio  nel formato di carta A4 e con due stili, ``standard'' ed  ``elements'' specifico per l'uso con la classe \class{suftesi}. Più avanti indicherò come produrre frontespizi con altri formati di carta.

\subsection{Comporre il frontespizio con l'ambiente titlepage}
Per seguire questa strada, qui posso dare solo delle indicazioni generiche.
\begin{enumerate}[noitemsep]
\item Aperto l'ambiente \env{titlepage} sarebbe opportuno cambiare localmente la geometria del layout della pagina, specificandolo con
\begin{flushleft}
\cs{thispagestyle}\marg{stile}
\end{flushleft}
dove \meta{stile} è stato precedentemente definito usando il pacchetto \pack{geometry} o altri pacchetti. Sarebbe opportuno ridefinire i due margini interno ed esterno in modo che siano uguali. Qualcuno desidera apportare  un piccolo spostamento verso l'esterno di non più di 5\,mm per compensare la curvatura della prima pagina; questa curvatura dipenderebbe dal modo di confezionare/legare la tesi; di solito  con una buona brossura ben eseguita o una legatura a filo, non è necessario apportare questa correzione, con una tesi spillata con punti metallici, probabilmente una piccola correzione non guasta; questa correzione è concepita per le pagine interne ma, siccome il frontespizio è la prima pagina della tesi, la curvatura della pagina è irrilevante e questo piccolo spostamento è del tutto superfluo.

\item Non è necessario che i margini laterali  siano complessivamente equivalenti ai margini del corpo della tesi, ma non guasta se lo sono.

\item Per la disposizione verticale del testo, sarebbe preferibile che i margini superiore e inferiore siano otticamente equivalenti; questo dipende dal fatto che il primo elemento in alto sia, per esempio, il nome dell'ateneo scritto con caratteri relativamente alti e neri, oppure sia il logo dell'università, che può essere alto, ma di solito non è tanto largo. Inoltre lo stile della pagina non deve presentare né testatina con il suo spazio di separazione del testo, né piedino, con il suo spazio di separazione dal testo, e non deve contenere il numero della pagina.

\item Il laureando deve provvedere autonomamente alle spaziature verticali fra un elemento e l'altro, deve decidere se certe parti siano da comporre in modo centrato, oppure in bandiera col palo a sinistra o a destra, se certe parti debbano essere affiancate e come debbano essere distanziate. 

\item Il laureando deve naturalmente scegliere collezione, codifica, famiglia, serie, forma, corpo e scartamento dei font che usa per le varie parti del testo. Una possibile scelta per un font senza grazie da usare per il nome dell'ateneo, potrebbe essere quella di riferirsi alla collezione Iwona, scegliendo la famiglia \Font{iwona} con codifica \opt{T1} nella serie \texttt{m} con la forma \texttt{n}, nel corpo di \texttt{15}\,pt, e con lo scartamento poco interlineato di soli \texttt{17}\,pt. Dovrebbe quindi chiamare il pacchetto \pack{iwona} nel preambolo e per usare il font dovrebbe usare la serie di comandi
\begin{flushleft}
\cs{fontsize}\Arg{15}\Arg{17} \cs{usefont}\Arg{T1}\Arg{iwona}\Arg{m}\Arg{n} 
\end{flushleft}
quando compone quell'informazione; dovrebbe curare di fare questa operazione dentro un gruppo al fine di mantenere queste impostazioni solo per la scrittura di quella informazione. Sarebbe meglio definirsi un comando personale per evitare di dover ogni volta introdurre quella serie di istruzioni, con il rischio di commettere anche il più piccolo errore di battitura e quindi di generare fastidiosi messaggi d'errore.
\end{enumerate}

Non scendo in ulteriori dettagli; l'utente che volesse comporre un frontespizio molto personalizzato per questa via, deve sapere quello che sta facendo e deve farlo con la dovuta competenza tipografica e del linguaggio \LaTeX. I risultati estetici dipendono fortemente da queste sue capacità.

Un esempio di un frontespizio costruito per questa via è quello che un utente ha voluto  creare con la pagina in \opt{landscape}; la sua tesi era veramente originale nel formato ``panoramico'' dovuta al fatto che conteneva molti disegni. Egli ha poi scritto questo codice:\label{proc:titlepage}
\begin{verbatim}
\documentclass [titlepage]{article}
\usepackage[a4paper,left=3cm,bottom=1.5cm,right=3cm,top=1.5cm,%
             landscape]{geometry}
\usepackage{graphicx}
\usepackage[T1]{fontenc}
\usepackage[utf8]{inputenc}   
\usepackage{iwona}
\usepackage[italian]{babel}             
\nofiles 

\newcommand {\institutionfont}{\fontsize {22}{30}\scshape}
\newcommand {\divisionfont}{\fontsize {16}{20}\rmfamily}
\newcommand {\pretitlefont}{\fontsize {16}{16}\rmfamily}
\newcommand {\titlefont}{\fontsize {18}{22}\usefont{T1}%
                {iwona}{bx}{n}}
\newcommand {\fixednamesfont}{\fontsize {14}{17}\mdseries}
\newcommand {\namesfont}{\fontsize {15}{18}\bfseries}
\newcommand {\footfont}{\fontsize {15}{18}\rmfamily}

\begin{document}

\begin{titlepage}

  \centering
    
    \begin{minipage}[c]{.475\textwidth}\centering
    \includegraphics[width=30mm]{logouno}\\[\baselineskip]
    {\institutionfont UNIVERSITÀ DI PAPEROPOLI\par}
    \end{minipage}
    \hfill
    \begin{minipage}[c]{0.475\textwidth}\centering
    {\divisionfont FACOLTÀ DI PENNUTOLOGIA\\[1ex]
     Corso di~laurea in~ingegneria del~volo~battente\par}
    \end{minipage}\vspace{\stretch{1}}
    
    {\pretitlefont Tesi di laurea\par}\vspace{\baselineskip}
    {\titlefont STUDIO ANALITICO DELLE PROPRIETÀ DELLE PENNE,
    INCLUSE QUELLE DEGLI UCCELLI ACQUATICI, IN PARTICOLARE QUELLE
    DEI~CIGNI NERI AUSTRALIANI\par}
    \vspace{\stretch{0.7}}
    
    \makebox[\textwidth]{\null\hfill\def\arraystretch{1.5}%
    \begin{minipage}[t]{.375\textwidth}\raggedright  
    \begin{tabular}[t]{@{}l@{}}
    \fixednamesfont Relatore\\
    \namesfont Prof. ing. Mario Rossi\\[1ex]
    \fixednamesfont Correlatore\\
    \namesfont Dott. ing. Giuseppe Scapigliati
    \end{tabular}
    \end{minipage}
    \hfill
    \begin{minipage}[t]{.375\textwidth}\raggedleft
    \begin{tabular}[t]{@{}l@{}}
    \fixednamesfont Candidato\\
    \namesfont Alfredo Bianchi
    \end{tabular}
    \end{minipage}\hfill\null}
    
    \vspace{\stretch{1}}
    
    {\footfont Anno accademico 2111--2112\par}
    
\end{titlepage}
\end{document}
\end{verbatim}

Nel codice precedente appaiono alcuni comandi per i font i cui nomi per le corrispondenti dichiarazioni sono ispirati a quelli contenuti nel pacchetto \pack{frontespizio}, ma i nomi sono irrilevanti, nel senso che qualunque altro nome sarebbe stato altrettanto valido; è invece utile che i nomi dei font siano scelti in modo che ricordino la funzione per la quale sono stati definiti.

La pagina che l'utente ha ottenuto è contenuta nella figura~\ref{fig:frontespizio-landscape}. Per inserirla come frontespizio della tesi egli ha poi usato il pacchetto \pack{pdfpages} che permette di inserire pagine scelte di un file PDF in un documento da comporre con \LaTeX. Se ne veda la documentazione aprendo un terminale o un prompt dei comandi, scrivendoci dentro \verb|texdoc pdfpages| e poi premendo il tasto \fbox{invio}.

\begin{figure}[t]\frame{\includegraphics[width=\textwidth]{FrontespizioLandscape}}
\caption{Frontespizio con l'orientamento \opt{landscape} ottenuto con l'ambiente \env{titlepage}}\label{fig:frontespizio-landscape}
\end{figure}


\subsection{Comporre il frontespizio con il pacchetto \texorpdfstring{\pack}{}{frontespizio}}

Esistono altri pacchetti, oltre a \pack{frontespizio}, per comporre pagine di titoli; ma per le tesi questo pacchetto sembra il più indicato. Definisce comandi per inserire virtualmente qualsiasi informazione necessaria per un frontespizio di testi di laurea triennale, magistrale o dottorale; per impostazione predefinita tutte le informazioni fisse sono in italiano, ma ciascuna può essere modificata come si vuole in qualsiasi lingua.

Fra le opzioni della classe \class{toptesi} si deve inserire l'opzione \opt{noTOPfront} per evitare conflitti fra i comandi del modulo \pack{topfront} e quelli del pacchetto \pack{frontespizio}. \textcolor{red}{Dimenticarsi di specificare questa opzione vuol dire andare incontro a messaggi d'errore causati dai conflitti fra i due pacchetti}.

Questo pacchetto \pack{frontespizio} si avvale di un file ausiliario \file{-frn.tex} per comporre il frontespizio vero e proprio; questo implica che per disporne come frontespizio del documento finito, bisogna comporre anche questo file ausiliario.

Precisamente tutte le informazioni riguardanti il frontespizio devono essere messe dentro un ambiente \amb{frontespizio} collocato dopo l'istruzione \cs{begin}\Arg{document}. Qui si metteranno i comandi per definire il nome del o dei candidati, il nome del o dei relatori e degli eventuali correlatori, il nome dell'università, il nome del corso di laurea, il nome del tipo di tesi, il titolo e il sottotitolo della tesi, eccetera.

Eseguendo \prog{pdflatex} oppure \prog{xelatex} o \prog{lualatex} viene scritto il file ausiliario; se la tesi ha il main file che si chiama \texttt{MarioRossiTesiMagistrale.tex}, dopo averlo compilato, nella cartella di lavoro ci sarà anche il file ausiliario \texttt{MarioRossiTesiMagistrale-frn.tex}; bisognerà dunque compilare questo file ausiliario, e poi ricompilare il documento che \emph{riscriverà il file ausiliario} con i nuovi eventuali dati modificati, ma intanto ingloberà il file PDF appena prodotto. La sequenza delle compilazioni sarà dunque:
\begin{verbatim}
pdflatex MarioRossiTesiMagistrale
pdflatex MarioRossiTesiMagistrale-frn
pdflatex MarioRossiTesiMagistrale
\end{verbatim}

Se  il frontespizio deve essere composto con font diversi da quelli di default, con codifiche particolari, con altre impostazioni particolari è meglio che il file ausiliario sia dotato di un preambolo adeguato; quindi il file principale avrà la struttura seguente;
\begin{flushleft}\obeylines
\cs{documentclass}\oarg{opzioni-classe}\Arg{toptesi}
\texttt{\% Preambolo del main file}
...
\cs{usepackage}\oarg{codifche}\Arg{inputenc}
...
\cs{usepackage}\oarg{opzioni-frontespizio}\Arg{frontespizio}
...
\cs{begin}\Arg{document}
\texttt{\% Per il preambolo del file ausiliario}
\quad\cs{begin}\Arg{Preambolo*}
\qquad\cs{usepackage}\marg{opzioni}\marg{pacchetto}
\qquad\cs{newcommand}...
\qquad\cs{renewcommand}...
\qquad...
\qquad\cs{renewcommand}...
\quad\cs{end}\Arg{Preambolo*}
\texttt{\% Informazioni per il frontespizio}
\quad\cs{begin}\Arg{frontespizio}
\qquad\cs{Universita}\marg{nome corto dell'università}
\qquad\cs{Logo}\oarg{altezza-logo}\marg{logo}
\qquad\cs{facolta}\marg{nome della facoltà}
\qquad...
\qquad\cs{NCorrelatore}\marg{singolare}\marg{plurale}
\qquad\cs{Correlatore}\marg{Nome-Cognome di un correlatore}
\quad\cs{end}\Arg{frontespizio}
\texttt{\% Corpo del main file}
\qquad...
\cs{end}\Arg{document}
\end{flushleft}

Il contenuto dell'ambiente \amb{Preambolo*} può essere molto vario e dipende dal programma usato per la compilazione del frontespizio, da scelte particolari dei font, da comandi da ridefinire, eccetera. La documentazione di \pack{frontespizio} espone diversi esempi che danno un'ottima visione d'insieme per tutte le operazioni che si possono fare e dei risultati che si possono ottenere.

Però, come già detto, questo pacchetto compone di default il frontespizio in una pagina  A4, e in due stili, quello ``standard'' e quello per lo stile ``elements'' della classe \class{suftesi}.

Se si volesse comporre la tesi in formato B5, per esempio, bisogna sfruttare adeguatamente il contenuto dell'ambiente \amb{Preambolo*}.

La procedura da seguire, dunque dovrebbe essere modificata come segue. Assumiamo che non si debbano apportare modifiche al contenuto del frontespizio; in caso contrario bisogna ripetere la procedura dopo ogni modifica del suo contenuto all'interno dell'ambiente \amb{frontespizio}.

\begin{Verbatim}[fontsize=\small]
\documentclass [11pt,b5paper]{book}
% Preambolo del documento
\usepackage[norules,swapnames]{frontespizio}
\usepackage[T1]{fontenc}
\usepackage{iwona}
\usepackage[italian]{babel}             
\usepackage[utf8]{inputenc}   
\nofiles 
\begin{document}
% Per il preambolo del file ausiliario
    \begin{Preambolo*}
        \usepackage[11pt,b5paper,top=25mm, bottom=25mm,
                    left=15mm,right=15mm]{geometry}
    \end{Preambolo*}
    
    \begin{frontespizio}
        \Istituzione{...}
        ...
        \Correlatore{...}
    \end{frontespizio}
\end{document}
\end{Verbatim}

Si noti che nel corpo dell'ambiente \amb{Preambolo*} si può inserire un intero preambolo, non semplicemente la chiamata al pacchetto \pack{geometry} con le sue opzioni;  questo ambiente può contenere qualsiasi cosa sia ragionevole in un preambolo; l'importante è che l'opzione relativa al formato della carta sia identica sia fra le opzioni della classe, sia fra quelle di \pack{geometry} nell'ambiente \amb{Preambolo*}.

Compilando questo documento si ottiene un'unica pagina contenente il frontespizio composto, che può poi venire incorporato nella tesi mediante l'uso del pacchetto \pack{pdfpages} che, come si è già detto, consente di inserire pagine scelte di un file PDF dentro un documento da comporre con \pdfLaTeX. In questo modo la composizione del frontespizio e quello della tesi procedono separatamente e non c'è bisogno, ogni volta che si cambino i dati per il frontespizio, di eseguire le tre compilazioni.

\subsection{Casi particolari}\label{sec:casi-particolari}
Esistono delle università con nomi lunghissimi, in particolare nomi bi lingui o trilingui. Ne cito due, entrambi italiani:
\begin{itemize}[noitemsep]
\item Università della valle d'Aosta -- Universitè de la Vallée d'Aoste
\item Freie Universität Bozen -- Libera Università di Bolzano -- Università Liedia de Bulsan
\end{itemize}
In questi casi se il nome dell'ateneo va in testa al frontespizio in corpo \verb|\Large| o \verb|\LARGE| non solo uscirebbe dalla giustezza della pagina, ma la parte iniziale del nome e quella finale potrebbero finire fuori del foglio. Il pacchetto \pack{topfront} dispone di un meccanismo automatico con il quale riduce in scala il nome dell'ateneo in modo da ridimensionarlo alla giustezza.

Questa operazione funziona automaticamente in modo  decisamente accettabile se il nome dell'ateneo eccede la giustezza di 10 o 15 punti percentuali, ma con nomi così lunghi come quelli elencati sopra, il meccanismo funziona ma il risultato lascia molto a desiderare perché i font con cui è composto il nome dell'ateneo, ridotti di una grossa percentuale, diventano più chiari e inaccettabilmente piccoli per la funzione che il nome dell'ateneo svolge nel frontespizio di una tesi.

In questi casi sarebbe più consono usare una immagine, come quella del logotipo dell'Università di Bolzano dove, accanto al ``timbro'' dell'università, i tre nomi sono elencati in colonna invece che su un'unica riga. Sono scritti con un carattere piccolo, ma affiancando il ``timbro'' l'insieme è giusto. Ovviamente, usando il logotipo, non è poi necessario usarlo di nuovo come simbolo dell'ateneo a metà della pagina.

Con i nomi in due sole lingue, come per l'università valdostana, si potrebbe anche pensare ad una soluzione di questo tipo:
\begin{verbatim}
\DeclareRobustCommand{\uniVdA}{\smash{%
   \parbox[b]{\textwidth}{\centering \Large
   \MakeUppercase{Università della Valle d'Aosta}\\
   \MakeUppercase{Universitè de la Vallée d'Aoste}}}}
\end{verbatim}
e usando il comando \verb|\uniVdA| come argomento per il comando \verb|\ateneo|.
Probabilmente si può usare anche con i nomi in tre lingue, con un corpo leggermente minore, definendo per esempio:
\begin{verbatim}
\DeclareRobustCommand{\uniBZ}{\smash{%
   \parbox[b]{\textwidth}{\centering\large
   \MakeUppercase{Freie Universität Bozen}\\
   \MakeUppercase{Libera Università di Bolzano}\\
   \MakeUppercase{Università Leidia de Bulsan}}}}
\end{verbatim}

Questa soluzione si può usare anche con nomi in una sola lingua ma molto lunghi, per esempio:
\begin{verbatim}
\DeclareRobustCommand{\uniMRE}{\smash{%
   \parbox[b]{\textwidth}{\centering\Large
   \MakeUppercase{Università degli Studi}\\
   \MakeUppercase{di Modena e Reggio Emilia}}}}
   \end{verbatim}

Queste tecniche si possono usare con creatività anche per ottenere effetti diversi; per esempio, se il nome dell'ateneo è poco più lungo della giustezza, si potrebbe pensare di lasciarlo uscire dai margini ma non lo si può scrivere direttamente come argomento del comando \cs{ateneo} perché scatta l'automatismo che lo ridimensiona alla giustezza. Però si può definire un comando, per esempio così:
\begin{verbatim}
\DeclareRobustCommand{\uniModenaReggio}{\smash{%
   \parbox[b]{\textwidth}{\makebox[\textwidth]{\Large
   \MakeUppercase{Università degli Studi di Modena
   e Reggio Emilia}}}}}
\end{verbatim}
usando poi \cs{uniModenaReggio} come argomento del comando \cs{ateneo};
il nome è inserito in  una scatola \LaTeX\ di larghezza pari alla giustezza, per cui la sua larghezza vera viene mascherata da quella della scatola e il meccanismo di ridimensionamento non entra in funzione.
 

\section{Come si comincia}\label{sec:cominciare}
Non vi dico niente di come si scriva una tesi; so per
esperienza che all'inizio ci si sente sperduti davanti allo
schermo vuoto. 
Ma i laureandi possono scaricare dal portale della didattica del sito
del Politecnico \url{http://didattica.polito.it} l'opuscoletto 
\emph{Saper Comunicare --- Cenni di scrittura tecnico-scientifica}
dove c'è scritto più o meno tutto quel che bisogna sapere per
impostare e scrivere un rapporto tecnico-scientifico qual è la
tesi, la monografia o la dissertazione.

Qui dirò solo come si comincia a scrivere il file sorgente della tesi. Ci sono essenzialmente due vie, ognuna delle quali offre vantaggi e svantaggi, quindi non si può dire a priori quale via sia la più indicata.
\begin{enumerate}
\item Si compone un unico file come esemplificato con i file \file{toptesi-example.tex}, \file{toptesi-example-xetex.tex} e \texttt{toptesi-example-con-frontespizio.tex}\index{file!toptesi-example-con-frontespizio.tex@\texttt{toptesi\discretionary{-}{}{-}example\discretionary{-}{}{-}con\discretionary{-}{}{-}frontespizio.tex}} allegati alla documentazione di TOPtesi. Questi file permettono di sperimentare diverse cose semplicemente mettendo o togliendo dei segni di commento all'inizio delle righe del preambolo che contengono alcuni comandi particolari. Essi sono completi di bibliografia e quell'unico file permette di comporre una tesi completa; ovviamente può essere usato come modello per una tesi; basta cambiare il contenuto dei comandi che contengono i nomi dei candidati, dei relatori, del titolo, dell'ateneo, della eventuale facoltà, eccetera. Ovviamente bisogna cambiare anche il contenuto dei capitoli e della bibliografia.

\item Si spezza il file sorgente in un numero adeguato di file parziali da gestire come specificato qui di seguito; ogni file parziale conterrà solo una parte della tesi, per esempio un solo capitolo; siccome il tutto è gestito dal master file, ciascun file parziale non dovrà più contenere la dichiarazione della classe né il preambolo né le righe \cs{begin}\Arg{document} e \cs{end}\Arg{document}, ma solo il materiale relativo ad uno specifico capitolo cominciando appunto con la riga che ne specifica il titolo:
\cs{chapter}\Arg{Il titolo di questo capitolo}.

Il file che contiene il capitolo può venire chiuso con il comando \cs{endinput}, dopo il quale il file può contenere qualunque cosa che non verrà mai letta dal programma di tipocomposizione, ma per il laureando potrebbe essere molto utile per scrivere alcune  annotazioni personali.
\end{enumerate}

Avendo già predisposto una scaletta da seguire per comporre la
tesi e avendo dato un nome a ciascun capitolo (nome un po' più
espressivo di quelli che scriverò io qui a titolo di esempio),
si predispone un master file con un nome ``diverso'' da
\file{tesi.tex}; bisogna sbizzarrire la fantasia, ma che il nome sia
mnemonico e ricordi subito a che cosa ci si riferisce. Questo
master file sarà dunque composto, per esempio, così:
\begin{verbatim}
\documentclass[b5paper,11pt]{toptesi}
 \includeonly{%
    preliminari,%
    capitolo1,%
    capitolo2,%
    capitolo3,%
    appendici,%
    bibliografia%
 }
\begin{document}
\begin{frontespizio}
 \ateneo{Università di Bengodi}
 \nomeateneo{La Saggezza}
 \titolo{Studi di tricotetratomia applicata}
 \sottotitolo{Cosa succede quando si spacca un capello in quattro}
 \candidata{Maria Chiomafolta}
 \relatore{prof.\ Piero Scapigliati}
 \sedutadilaurea{Maggio 2125}
 \end{frontespizio
%
 \include{preliminari}
 \include{capitolo1}
 \include{capitolo2}
 \include{capitolo3}
 \appendix
 \include{appendici}
 \include{bibliografia}
\end{document}
\end{verbatim}

È da osservare in particolare che l'aver scritto l'argomento di
\cs{includeonly} nel modo esemplificato sopra, permette di
inserire dei segni di commento all'inizio di alcune delle righe in
modo da eseguire, per esempio, la compilazione selettiva di un
solo capitolo conservando tutte le informazioni incrociate
relative agli altri capitoli già compilati. Con i PC di oggi la
compilazione è rapidissima, quindi lo scopo di risparmiare tempo 
di compilazione non è il motivo per il quale raccomando di seguire 
questa procedura; faccio invece
notare che tenere separate le varie parti della tesi
permette di lavorare con più ordine e di risparmiare non poco
tempo per la gestione dei vari file sorgente e per le loro
correzioni.

Non sarebbe male nemmeno tenere sgombra la cartella di lavoro, 
cercando di confinare tutto il materiale ausiliario (per esempio 
disegni, fotografie, brani di codice, eccetera) in sottocartelle 
della cartella di lavoro; ma queste raccomandazioni sono quelle 
che valgono per ogni lavoro da eseguire con \LaTeX\ e il lettore 
con un minimo di esperienza sa già come gestire. All'occorrenza 
vale la pena di ripassare l'argomento nella guida \emph{L'Arte 
di scrivere con LaTeX} da scaricare dalla sezione Documentazione 
del sito \url{http://www.guitex.org}.

Il file \file{preliminari.tex} conterrà presumibilmente il sommario,
molto facoltativamente i ringraziamenti\footnote{I ringraziamenti 
in una tesi non sono quasi mai necessari; lo sono quando bisogna
ringraziare una istituzione o una o più persone \emph{esterne}
alla propria università che hanno messo a disposizione le loro
strutture o le loro competenze per lo svolgimento della tesi. Non
si ringraziano genitori, fidanzati/e, amici, e tutti i parenti di
ogni grado, il quali saranno più che contenti di sapervi laureati,
piuttosto che vedere il loro nome scritto sulla tesi\dots.
Inoltre rappresenta un atto di piaggeria ringraziare il o i
propri relatori; essi hanno svolto un loro compito istituzionale,
non vi hanno fatto un piacere. Naturalmente questo non vieta che
la consuetudine del lavoro assieme non abbia dato origine a
sentimenti di reciproca stima e anche di amicizia, ma questa è
un'altra faccenda.} e gli indici.

Il file \file{capitolo1.tex} conterrà l'introduzione e in
pratica esporrà lo stato dell'arte oltre allo scopo del lavoro e
ai risultati che si vorrebbero ottenere.

I file \file{capitolo2.tex}, \file{capitolo3.tex}, eccetera,
conterranno la descrizione del lavoro svolto e l'analisi critica
dei risultati via via ottenuti; l'ultimo file prima delle
appendici conterrà i commenti finali, quali ad esempio un sunto dei risultati raggiunti, una breve esposizione degli ulteriori sviluppi, possibili approfondimenti, eccetera.

Nelle varie appendici verranno raccolte le misure eseguite, gli
sviluppi matematici, i listati dei programmi, le note critiche, le
lunghe citazioni e tutte le altre cose che rovinerebbero la
fluidità del testo, se fossero inserite direttamente nel corpo
della tesi.

La bibliografia conclude la tesi. A seconda del campo disciplinare in cui si svolge la tesi può essere importante scegliere un metodo di citazione piuttosto di un altro; la collezione di pacchetti del sistema \TeX\ vi consente un'ampia scelta di personalizzazioni per gli stili di composizione e di citazione. Da alcuni anni, poi, è disponibile il pacchetto di estensione \pack{biblatex} che consente una gestione diretta della bibliografia mediante comandi \LaTeX; in ogni caso per queste personalizzazioni è necessario creare un database bibliografico (con estensione \file{.bib}) in un formato particolare, e poi è necessario elaborare questo database mediante il programma \prog{bibtex} o, ancor meglio, con il programma \prog{biber} (predefinito con \pack{biblatex} dal 2012), sempre facente parte della distribuzione del sistema \TeX. I risultati che si possono ottenere sono estremamente professionali.


\subsection{La scelta della codifica per il file sorgente}\label{sec:codifica}

Si noti che per comporre i file del corpo della tesi si possono
usare direttamente le lettere accentate della tastiera italiana o
quelle che si possono ottenere con combinazioni di tasti sia sotto
Windows sia sotto Linux o Mac~OS~X; per default \textsf{TOPtesi}
non specifica nessuna particolare codifica per l'input; ma perché
il programma di tipocomposizione elabori correttamente i file sorgente, o si usa \prog{xelatex}  o \prog{lualatex} ricordando di salvare i file con la codifica UNICODE o UTF-8, oppure non si specifica nessuna particolare codifica ma si usano solo i caratteri ASCII a 7~bit; oppure ancora si usano i caratteri nazionali direttamente da tastiera ma si specifica la codifica che l'editor ha
impiegato per salvare nei file i codici dei caratteri non ASCII,
per esempio le lettere accentate; a meno che non usiate \prog{xelatex} o \prog{lualatex}, vi suggerisco di specificare esplicitamente la codifica 
mediante il comando
\begin{verbatim}
\usepackage[utf8]{inputenc}
\end{verbatim}
che va bene sia per le macchine Windows, sia per quelle basate su UNIX.

Gli editor \TeX{}Shop, \TeX{}works, TeXstudio, Texmaker, e qualcun altro, consentono di usare la codifica \opt{utf8} che permette di inserire qualunque segno fra le decine o centinaia di migliaia che la codifica UNICODE consente -- se usate \prog{xelatex} o \prog{lualatex} questa codifica è necessaria ma non bisogna esplicitarla nel file sorgente. Il problema, in effetti, non è la codifica, ma è la tastiera. Con sistemi operativi diversi si sono sviluppati programmi di vario genere per scrivere in alfabeti diversi da quello latino esteso, qual è quello invocato, per esempio, con l'opzione \opt{utf8}. Ma se dovete scrivere la vostra tesi con brani in greco o in russo o in arabo o in lingue con segni sillabici o ideogrammi, questo problema si presenterebbe qualunque sia l'editor e il programma di composizione che usate, ed è un problema che dovete risolvere da soli.

Bisogna notare che in italiano vanno accentate anche le lettere maiuscole; siccome la tastiera italiana non le prevede, consiglio di inserire nel preambolo la definizione del comando \cs{E} per scrivere `È'. Credo che questa sia la maiuscola accentata più frequente in italiano e le altre non capitano mai all'inizio di un periodo; possono capitare nei titoli e allora bisogna arrangiarsi con codici numerici o con tabelle di caratteri. Il comando:
\begin{verbatim}
\usepackage{xspace} 
\newcommand*\E{\`E\xspace}
\end{verbatim}
inserisce direttamente una È all'inizio di un periodo e inserisce direttamente lo spazio successivo, per cui, senza inserire esplicitamente nessun comando di spaziatura dopo il comando \cs{E}, si può scrivere semplicemente:
\begin{verbatim}
\E conveniente rilevare che\dots
\end{verbatim}
per ottenere:
\begin{quote}
È conveniente rilevare che\dots
\end{quote}
L'uso del pacchetto \pack{xspace} fa sì che esso, tramite il comando \cs{xspace} inserisca uno spazio, a meno che il comando \cs{E} non sia seguito da qualunque cosa che non sia uno spazio, per esempio un segno di interpunzione, o una tilde di legatura; anche il comando esplicito di spazio \verb*|\ | impedisce l'inserimento di un ulteriore spazio da parte di \cs{xspace}.

Se il vostro editor vi consente di selezionare le maiuscole accentate da una tabella oppure vi consente di usare combinazioni di tasti per inserire direttamente le maiuscole accentate nel file sorgente, fatene uso, perché la lettura del file sorgente ne guadagna moltissimo.

Se volete o dovete usare \prog{pdflatex}, dovete caricare nel vostro preambolo tutto quello che è necessario per gestire quelle lingue scritte con alfabeti diversi, ma non chiedetemi come si fa, perché so dirvelo solo per il greco. Per questa lingua, se usate l'ortografia monotonica, basta inserire l'opzione della lingua \opt{greek} fra le opzioni \emph{della classe}; se usate l'ortografia politonica\footnote{Il greco moderno può essere politonico o monotonico; quello antico è sempre e solo politonico; però le regole di sillabazione sono diverse; con \pack{polyglossia} non ci sono problemi; con \pack{babel} a tutt'oggi (ottobre 2015) esiste l'attributo \opt{ancient} per selezionare la sillabazione antica oltre alla grafia politonica, ma non è ancora possibile disporre effettivamente della sillabazione antica completa.}, allora dovete richiedere per \pack{babel} lo specifico attributo seguendo questo procedimento (per \pack{polyglossia} si veda più avanti):
\begin{verbatim}
\documentclass[...,greek.ancient,...]{toptesi}
...
\begin{document}
...
\end{document}
\end{verbatim}
oppure
\begin{verbatim}
\documentclass[...,greek,...]{toptesi}
...
\languageattribute{greek}{ancient}
\begin{document}
...
\end{document}
\end{verbatim}
Al posto di \opt{ancient} si può specificare \opt{polutoniko}; \opt{polutoniko} si riferisce al greco moderno scritto con tutta la varietà di accenti e diviso in sillabe con i criteri adatti alla lingua moderna; \opt{ancient} si riferisce al greco classico, che oltre ad usare la solita varietà di accenti, usa altre regole per la sillabazione di questa varietà di lingua.

Per una tesi che tratti il greco antico in forma filologica, consiglio l'uso del pacchetto \pack{teubner}; visto che l'ho scritto io, un po' di réclame non guasta~\dots\ Bisogna però leggerne la documentazione con attenzione. In particolare conviene ricordare che \pack{teubner} non viene caricato se non è stato precedentemente specificata la lingua greca fra le opzioni, ma poi provvede da solo ad impostare l'attributo \opt{ancient} e tutte le altre caratteristiche necessarie per comporre in greco con il \emph{mark-up} filologico. A tutt'oggi (2016) il pacchetto \pack{teubner} non è stato adeguatamente controllato sotto \XeLaTeX\ e \LuaLaTeX. Probabilmente non è completamente compatibile, ma non sono ancora in grado di dire niente né in positivo né in negativo.

%%% MODIFICARE LA FRASE PRECEDENTE QUANDO TEUBNER SARÀ A POSTO!!!!

Immagino che per usare le lingue che si scrivono in cirillico, basti specificare  l'opzione della lingua, come si fa per il greco. Per alcune altre lingue bisogna invece caricare l'apposito pacchetto, il quale provvede a tutte le impostazioni necessarie; in questi casi bisogna leggere con attenzione la documentazione di quei pacchetti.

Suggerisco di riferirsi al documento \emph{L'Arte di scrivere in diverse lingue con (Xe)LaTeX}, liberamente scaricabile dalla sezione Documentazione del sito \url{www.guitex.org}. Questa guida tratta molti argomenti utili per comporre in lingue straniere comprese alcune di quelle che si scrivono con insiemi di caratteri diversi da quelli latini; descrive anche molte caratteristiche delle tradizioni tipografiche in uso in nazioni diverse dall'Italia oltre, ovviamente, le tradizioni italiane.

Ma, tornando alla codifica UTF-8, il problema non è tanto l'editor che consenta o non consenta di usare quella codifica; quasi tutti i sistemi operativi consentono di installare diversi driver per la tastiera del PC, cosicché con un semplice click del mouse si può passare da un alfabeto all'altro; il problema sono i tasti della tastiera sulla cui faccia è riportato il segno che si ottiene premendo quel tasto quando è attivo il driver corrispondente a quella specifica tastiera. Non è ovviamente possibile cambiare il disegno che appare sui tasti, semplicemente cliccando per cambiare driver; forse lo si può fare con le tastiere virtuali dei PC \emph{touch screen}, ma con i PC normali questo non è possibile. Ecco quindi che la difficoltà di scrivere con un alfabeto diverso richiede una memoria ferrea per ricordare a quale segno dell'altro alfabeto corrisponda il tasto della nostra tastiera. 

Per il greco moderno e classico, se si usano i font di default con la codifica LGR (Local GReek), che vengono installati praticamente con ogni installazione  del sistema \TeX\ aggiornata e, soprattutto, completa, le corrispondenze con la tastiera latina sono studiate in modo da semplificare molto questo processo di ``memorizzazione''. Tuttavia ad una persona abituata a leggere il greco scritto (giustamente) in greco, vederlo scritto con caratteri latini fa un certo effetto; la documentazione di \textsf{babel} e quella di \textsf{teubner} ricordano questa corrispondenza fra i tasti latini e i segni greci (lettere, accenti, spiriti, segni speciali, ecc.). La corrispondenza fra greco e latino è abbastanza semplice perché il greco ha 24 maiuscole e 25 minuscole, mentre il latino ne ha 26 di ciascun tipo. Non so che cosa succeda con il cirillico che ha un alfabeto composto di più di 30 segni, per cui la corrispondenza diretta con l'alfabeto latino è impossibile.

Con le macchine Macintosh il sistema operativo Mac OS X mette a disposizione una specie di tastiera virtuale sui tasti della quale si può cliccare con il mouse; si possono installare diverse tastiere che possono venire scambiate con un semplice click sulla icona disegnata nella barra superiore; per cui scrivere in greco politonico o in russo o in ebraico, o in\dots, non è un problema. Temo che lo sia con altre macchine e altri sistemi operativi.

Mi dicono che sulle macchine Linux l'editor \prog{Vim} e, su tutte le macchine, l'editor \prog{emacs} prevedano delle combinazioni di tasti che consentono di scrivere direttamente in greco sullo schermo, nella finestra del file \texttt{.tex}. Ho letto qualcosa in merito, ma non sono in grado di confermare per esperienza diretta.

\goodpagebreak

\subsection{La scelta della codifica per il file di uscita}

Solo per scrivere in inglese non è necessario specificare nessuna codifica per i font da usare per comporre la tesi; questo è vero in quanto l'inglese, fra le lingue più comuni nel mondo occidentale, è l'unica che non fa uso di accenti o altri segni diacritici. Perciò la codifica di default \opt{OT1} per i font di uscita adatti a scrivere in inglese va benissimo e non è necessario specificare nulla. Ma attenzione: come ho già detto, se si deve inserire anche una sola parola diversa dall'inglese e che contenga un accento, o se si deve scrivere in ``middle English'', che usava gli accenti, questa vecchia codifica non va assolutamente bene per produrre file PDF/A compatibili. Quindi anche per l'inglese sarebbe opportuno usare una codifica più avanzata.

Inoltre con questa codifica \opt{OT1} la presenza di accenti impedisce la divisione in sillabe o ne riduce molto l'efficacia; per l'italiano il problema è relativamente modesto perché normalmente gli accenti compaiono solo sull'ultima vocale delle parole \emph{tronche}. Quindi con questa codifica la parola \emph{qualità} viene divisa in ``qua-lità'' invece che in ``qua-li-tà''. Invece in francese una parola come \emph{électricité} non viene divisa per niente a causa dell'accento sulla prima vocale.

Allora per tutte le lingue diverse dall'inglese, ma anche per l'inglese, è quanto mai opportuno specificare per i font di uscita la codifica \opt{T1}. C'è però un problemino, facilmente risolubile, ma bisogna pensarci: fino ad oggi (2015) i font di default usati da \pdfLaTeX\ sono i font della collezione Computer Modern con codifica \opt{OT1}; e ogni installazione del sistema \TeX\ li può usare sia nella versione ``bitmapped'' sia in quella PostScript Type~1. Quest'ultimo formato dei font dovrebbe essere quello da usare sempre con  qualunque font per il formato PDF del file di uscita, perché questo formato è portabile da una macchina ad un'altra, indipendentemente dal sistema operativo usato. Se poi il file deve essere archiviato, esso deve essere in formato PDF, quindi, di fatto non ci sono scelte.

I file della collezione Computer Modern con codifica \opt{T1}, conosciuti anche come font EC, sono invece distribuiti solo nella forma bitmapped, la quale, oltre ad essere vietata nei file PDF archiviabili, si legge molto male sullo schermo, se il programma di visualizzazione non è molto ben adattato ai font bitmapped. Esiste la collezione di font  \textsf{cm-super} per ovviare a questo inconveniente (questa collezione è installata di default in ogni installazione completa del sistema \TeX), ma l'uso di questi font implica un aumento non trascurabile delle dimensioni del file di uscita. Per aggirare questi inconvenienti è meglio usare i font della collezione Latin Modern, per cui nel preambolo del file da gestire con \prog{pdflatex} si specificherà:
\begin{verbatim}
\documentclass[...]{toptesi}
...
\usepackage[T1]{fontenc}
\usepackage{lmodern}
...
\begin{document}
...
\end{document}
\end{verbatim}

Se il file viene gestito con \prog{xelatex} o \prog{lualatex}, oltre alla codifica \opt{utf8} di default \emph{che non va esplicitata}, basta richiedere l'uso del pacchetto \pack{fontspec} senza ulteriori opzioni e viene automaticamente caricata ed usata la collezione dei font Latin Modern nella loro versione OpenType; a questo provvede direttamente la classe \class{toptesi}.  Usando \prog{xelatex} o \prog{lualatex}, gli appositi comandi per la selezione dei font sia testuali sia per la matematica può essere eseguita nel preambolo senza bisogno di ricaricare il pacchetto \pack{fontenc}. Volendo, invece di usare i font Latin Modern, si possono impostare facilmente altre famiglie di font con comodi comandi che fanno riferimento ai font OpenType installati sulla propria macchina.

Naturalmente per gli utenti esperti di \LaTeX\ ci sono anche altri pacchetti usabili per impiegare altri font diversi da quelli della collezione Latin Modern, che comunque, secondo me, sono i migliori per la loro completezza; grazie al diverso disegno dei corpi minori la lettura dei pedici e degli apici ne risulta agevolata.

Qui, per la composizione della tesi con \prog{pdflatex} segnalo solo la collezione dei font Times eXtended, richiamabili con i pacchetti \pack{newtxtext} e \pack{newtxmath}, e la collezione dei font Palatino eXtended, richiamabili con i pacchetti \pack{newpxtext} e \pack{newpxmath}. I Times sono font più stretti di quelli della collezione Latin Modern e sono forse indicati per rendere più compatta una tesi un po' voluminosa. I Palatino sono un po' più larghi dei font della collezione Latin Modern e allungano la tesi di qualche pagina. I nuovi pacchetti \pack{newtx...} e \pack{newpx...}  sono configurabili con diverse opzioni; li consiglio a coloro che vogliono usare i font Times o Palatino, ma se devo dare un consiglio assoluto, direi che sia sempre preferibile usare i font Latin Modern dalle prestazioni molto migliori grazie ai corpi ottici; volendo si può usare il pacchetto \pack{cfr-lm} che consente di personalizzare fortemente le proprietà dei font Latin Modern da usare, per esempio permettendo l'uso delle cifre minuscole \ifPDFTeX\oldstylenums{1234567890}\else {\addfontfeature{Numbers=OldStyle}{1234567890}}\fi\ nel testo, e le cifre maiuscole 1234567890 nella composizione della matematica.

Tanto per confronto, questo testo di documentazione su \textsf{TOPtesi} è composto con il motore di composizione \prog{lualatex} usando i font OpenType TeX Gyre Termes con codifica UNICODE presenti nella distribuzione del sistema \TeX; come si vede la divisione delle parole in fin di riga  non presenta inconvenienti particolari; forse con un testo di altro genere, con composizione in colonne strette o all'interno di tabelle, si potrebbe verificare la presenza di righe molto lasche, dove è stato allargato troppo lo spazio interparola. Per questo, quando si usa il motore di composizione \prog{pdflatex}, si consiglia sempre di usare la codifica \opt{T1}; non si sa mai, ma certamente con questa codifica si hanno meno inconvenienti e le lettere accentate sono disegnate meglio. Se si usa come motore \prog{pdflatex}, è conveniente usare anche il pacchetto di estensione \pack{microtype} che consente di comporre ancora meglio le righe del testo; con \prog{xelatex}, invece, le funzionalità di questo pacchetto sono solo parzialmente usabili; comunque si può usare \pack{microtype} anche compilando con \prog{xelatex} sebbene i miglioramenti siano meno ``vistosi'' che con \prog{pdflatex}; con \prog{lualatex} il pacchetto \pack{microtype} produce gli stessi benefici che si ottengono con \prog{pdflatex}, senza i piccoli inconvenienti che si manifestano con \prog{xelatex}.


\section{Come stampare la tesi}

Il pacchetto \textsf{TOPtesi} contiene anche l'opzione di composizione
\emph{in bianca e volta}, ma di default compone su una facciata sola
della pagina.

Qual è la differenza? È quella che si osserva nei libri composti
meglio, cioè  nei libri con i fogli stampati davanti e di dietro:
quando il libro è aperto la striscia bianca formata dai due
margini interni delle pagine affacciate è circa uguale a ciascuno
dei margini esterni. Invece quando si stampa su una facciata sola
il corpo del testo è centrato e i margini laterali sono
uguali. Inoltre nelle testatine delle pagine stampate solo in
bianca compare sempre e solo il titolo del capitolo corrente,
mentre nelle testatine delle pagine \emph{pari} stampate in bianca
e volta compare il titolo del capitolo, mentre nelle pagine
\emph{dispari} compare il titolo del paragrafo con cui comincia la
pagina.

Molti laureandi ritengono di fare una cosa utile stampando fronte
e retro, o, come dicono i tipografi, in bianca e volta. Certamente
è una buona idea, sia perché il fascicolo della tesi viene ad
assumere un aspetto più professionale, sia perché, ecologicamente
parlando, si consuma meno carta.

Per questo ho cambiato le impostazioni di default della classe
\class{report}, sulla quale è  basata la classe \class{toptesi},
per far sì che comunque la stampa della pagina, sia essa solo in
bianca o in bianca e volta, appaia sempre con i margini laterali
uguali.

Ho deciso per questa soluzione perché quando si rilega la tesi,
necessariamente dal lato della cucitura o dell'incollatura le
pagine sono piegate e l'insieme dei due margini interni delle
pagine affacciate appare visivamente più piccolo di ciascuno dei
margini esterni. Se si stampa solo in bianca invece sarebbe il
caso di aumentare il margine sinistro; tuttavia non ho realizzato
questo spostamento non solo perché il margine destro resta di una
certa ampiezza e può accogliere le note
marginali\marginpar{\footnotesize\sffamily\raggedright Questa è  una nota
marginale}, se si decide di farne uso, ma anche perché, almeno al
Politecnico di Torino, la copia da depositare nella segreteria 
didattica non deve essere realizzata su carta, ma deve essere
registrata su un CD-ROM in formato \texttt{.pdf} e chi legge la
tesi sullo schermo preferisce senz'altro avere il testo centrato
piuttosto che spostato alternativamente da un lato o dall'altro
dello schermo.

Però questa soluzione potrebbe non piacere ad alcuni laureandi,
specialmente se compongono tesi piuttosto voluminose e/o se le
rilegano con una costola piuttosto rigida. In questo caso la
classe \class{toptesi} accetta l'opzione \opt{cucitura}
mediante la quale le facciate di sinistra e di destra vengono
spostate all'esterno di 7\,mm, per compensare la rigidezza 
della cucitura e la forte curvatura delle pagine causate dallo 
spessore e/o dal tipo di rilegatura. Questo implica che la 
copia da memorizzare sul CD-ROM dovrà essere compilata senza 
specificare questa opzione, che invece sarà specificata per 
ricomporre il file prima della stampa su carta.

\goodpagebreak

\section{Comporre la tesi in diverse lingue}\label{sec:lingue}
Il pacchetto \textsf{TOPtesi} usa di default il pacchetto \pack{babel} o il pacchetto \pack{polyglossia}  per gestire la composizione in diverse lingue.
Tuttavia la composizione in lingue diverse dall'inglese (che è la
lingua di default per una installazione ``vergine'' del sistema
\TeX) richiede che il sistema sia configurato per gestire diverse
lingue; perciò le operazioni da fare sono essenzialmente due.
\begin{enumerate}
\item Bisogna configurare il sistema \TeX\ per gestire diverse lingue.
\item Bisogna specificare nel comando di apertura del master file di quali altre
lingue ci si vuole servire \textcolor{red}{oltre all'italiano e all'inglese che sono già automaticamente invocati da \textsf{TOPtesi}}. Si ricorda che non si può più invocare nel preambolo il pacchetto \textsf{babel} una seconda volta, quando esso è già stato invocato una prima volta nel corpo della classe. Si veda nel seguito come bisogna specificare le ulteriori lingue oltre all'italiano e all'inglese.
\end{enumerate}


\subsection{Configurazione iniziale}

Per la prima parte, cioè  per configurare il sistema \TeX\ per
gestire diverse lingue, subito dopo l'installazione lanciate
\LaTeX\ su un file di prova qualunque, per esempio
\file{sample.tex} (facente parte della distribuzione del sistema \TeX), poi apritene il file \texttt{.log},
nell'esempio \texttt{sample.log}; nelle prime righe di questo file
troverete le lingue per le quali il vostro sistema è già
configurato.

Per esempio, componendo con \prog{pdflatex} questo file di documentazione, il file
\texttt{toptesi\discretionary{}{-}{-}it\discretionary{}{-}{-}xetex.log} nel 2012 conteneva nelle prime righe quanto segue:
\begin{verbatim}
This is pdfTeX, Version 3.1415926-2.4-0.9998 (TeX Live 2012)
    (format=xelatex 2012.7.10)  9 SEP 2012 16:04
entering extended mode
 \write18 enabled.
 file:line:error style messages enabled.
 %&-line parsing enabled.
**toptesi-it-xetex.tex
(./toptesi-it-xetex.tex
LaTeX2e <2011/06/27>
Babel <v3.8m> and hyphenation patterns for english, dumylang, 
nohyphenation, german-x-2012-05-30, ngerman-x-2012-05-30, afrikaans, 
ancientgreek, ibycus, arabic, armenian, basque, bulgarian, catalan, 
pinyin, coptic, croatian, czech, danish, dutch, ukenglish, 
usenglishmax, esperanto, estonian, ethiopic, farsi, finnish, french, 
friulan, galician, german, ngerman, swissgerman, monogreek, greek, 
hungarian, icelandic, assamese, bengali, gujarati, hindi, kannada, 
malayalam, marathi, oriya, panjabi, tamil, telugu, indonesian, 
interlingua, irish, italian, kurmanji, latin, latvian, lithuanian, 
mongolian, mongolianlmc, bokmal, nynorsk, polish, portuguese, 
romanian, romansh, russian, sanskrit, serbian, serbianc, slovak, 
slovenian, spanish, swedish, turkish, turkmen, ukrainian, 
uppersorbian, welsh, loaded.
\end{verbatim}

Dal 2014, purtroppo l'elenco delle lingue è sostituito dalla seguente laconica frase:
\begin{verbatim}
Babel <3.9k> and hyphenation patterns for 81 languages loaded.
\end{verbatim}
Non c'è bisogno di preoccuparsi; nella mia macchina sono installate alcune lingue in più di quelle di default, ma se il numero è così alto, vuol dire che sono installate tutte le lingue che i programmi di composizione del sistema \TeX\ sono capaci di gestire; fra queste c'è l'italiano, tre versioni di greco, tre versioni di latino, eccetera. L'elenco completo può venire letto nel file \file{xelatex.log} (o, componendo con \prog{pdflatex}, nel file \file{pdflatex.log}) che si trovano in uno degli alberi del sistema \TeX. La lettura è un po' criptica, anche perché il file è piuttosto lungo, ma basta cercare la stringa \texttt{l@english}, e di lì in avanti è registrato tutto quello che serve conoscere per l'elenco delle lingue installate. \LuaLaTeX, diversamente da \pdfLaTeX\ e da \XeLaTeX, carica le impostazioni, sillabazione compresa, per le lingue del documento solo al momento di compilare quel dato documento; per questo compilatore solo l'inglese è precaricato, ma al momento opportuno esso si carica tutto ciò che è necessario per comporre in ciascuna dell'ottantina di lingue che riesce a gestire, o meglio in ciascuna delle lingua specificate per la composizione di ogni particolare documento, massimo un'ottantina (bastano?\dots).

Quel listato o quella riga è quanto appare sempre quando si usa la distribuzione \TeXLive completa o quando si usa la distribuzione \MiKTeX~2.9 (o successiva) completa; serve per dare un'idea di quello che contengono le prime righe del file in questione. Se, come è possibile, dopo la prima installazione il vostro primo file \texttt{.log} non contiene fra le lingue elencate anche l'italiano (oppure sono elencate solo una mezza dozzina di lingue, oppure si dice che le lingue installate sono in un numero molto inferiore a 80), allora questa è la cosa più urgente da fare; non vorrete mica che la vostra tesi, scritta in italiano, abbia le parole divise in sillabe in fin di riga con le regole angloamericane? Se poi siete su un programma di doppia laurea e dovete scrivere parte della tesi nella lingua dell'università straniera di cui prendete l'altro titolo, allora è necessario avere disponibile almeno anche quella lingua. Questo inghippo avviene quando si scarica una versione ridotta delle distribuzioni del sistema \TeX, oppure quando si usa la distribuzione \TeXLive/Debian (sempre sconsigliabile per vari motivi, non escluso il fatto che l'installazione di default prevede solo la sillabazione per l'inglese).

Le regole per inizializzare le lingue gestibili differiscono da
distribuzione a distribuzione del sistema \TeX. Bisogna quindi leggere la
documentazione che accompagna la vostra distribuzione; qui darò
alcuni cenni relativi alle distribuzioni che io conosco.

\subsubsection{\MiKTeX}
Questa distribuzione è forse quella che più sovente viene installata nella forma ``basic'' (errore! mai installare la versione di base, perché è tutt'altro che completa) e perciò è fra quelle che hanno bisogno di essere gestite. Si apre il wizard da \texttt{Avvio\discretionary{}{|}{|}Programmi\discretionary{}{|}{|}MiKTeX\discretionary{}{|}{|}MiKTeX\textvisiblespace Settings}. Esso apre una finestra di dialogo con diverse linguette; si clicca sulla linguetta \textsf{Languages} e viene aperta un'altra finestra con l'elenco di tutte le lingue gestibili, alcune già con il segno di spunta alla loro sinistra nell'apposito quadratino, altre senza il segno di spunta. Potete cliccare per togliere il segno di spunta sulle lingue già spuntate ma che non userete mai, ma ve lo sconsiglio, e potete aggiungere il segno di spunta cliccando nel quadratino corrispondente alle lingue desiderate, ma vi consiglio di spuntarle tutte.

Poi uscite dal dialogo per la scelta delle lingue cliccando \texttt{OK} e tornate  sulla linguetta \textsf{General}; qui trovate due bottoni il primo dei quali serve per aggiornare il database dei nomi dei file, mentre il secondo serve per re-inizializzare i file di formato, cioè  quei file che contengono già la traduzione in linguaggio macchina di tutte le operazioni che la vostra distribuzione del sistema \TeX\ è capace di compiere con i suoi vari applicativi. La divisione in sillabe è una di quelle operazioni che deve essere inizializzata, perché richiede strutture dati particolari che sarebbe troppo lungo generare di volta in volta (\LuaLaTeX, però ci riesce). Cliccate sul bottone per re-inizializzare i file di formato; alla fine di questa operazione potete chiudere il wizard e controllare, dopo aver ricomposto, per esempio, \file{sample.tex} che le prime righe del file \texttt{.log} contengano tutte le lingue che volete usare.


\subsubsection{\TeXLive}
La distribuzione \TeXLive completa distribuita sul circuito \url{ctan} nasce già configurata per tutte le lingue che \LaTeX\ è capace di gestire, quindi anche l'italiano; bisogna ricorrere ai suoi comandi di configurazione solo se si vuole aggiungere una lingua la cui sillabazione non sia distribuita insieme a \TeX\ Live. Questo evento è talmente raro che non vale la pena di insistervi sopra. Tuttavia un laureando che svolga la sua tesi sull'ostrogoto altomedievale e disponesse solo della sillabazione per l'ostrogoto classico e per l'ostrogoto contemporaneo, si troverebbe in difficoltà; infatti, prima  ancora di inserire le regole per l'ostrogoto altomedievale, dovrebbe scriversi le regole di sillabazione e codificarle nel linguaggio specifico richiesto dal sistema \TeX; questo è tutt'altro che facile ed è riservato a specialisti molto avanzati.


\subsubsection{Mac\TeX}
Dal 2007 la distribuzione di \prog{Mac\TeX} sostanzialmente coincide con \TeX-live, salvo che contiene anche programmi accessori specifici per le macchine Macintosh;  è pre"configurato per gestire tutte le lingue di cui il sistema è capace, compreso l'italiano. Ma per l'ostrogoto altomedievale\dots


\subsubsection{Distribuzioni commerciali}
Le distribuzioni commerciali differiscono da quelle gratuite essenzialmente per i programmi accessori che accompagnano la distribuzione del sistema \TeX\ che, di sua natura, è  gratuito; il costo delle distribuzioni commerciali corrisponde al valore delle parti accessorie, all'assistenza per i clienti, eccetera.

Fra le parti accessorie in generale ci sono anche i programmi realizzati attraverso comode interfacce grafiche per la inizializzazione e l'aggiornamento del software.

Perciò con queste distribuzioni bisogna leggere il manuale di istruzioni e provvedere corrispondentemente.

\subsection{Le lingue della tesi}

Di default \textsf{TOPtesi} compone la tesi in italiano e per scrivere la tesi in italiano non occorre altro.

Se si devono inserire nella tesi brani di testo in lingua straniera, ma lasciando la struttura della tesi in italiano, se si usa \prog{pdflatex} basta elencare fra le altre opzioni nel comando di apertura del master file i nomi (inglesi) delle lingue da usare; per esempio, per inserire brani in francese si scriverà:
\begin{verbatim}
\documentclass[12pt,twoside,french]{toptesi}
\end{verbatim}

Se invece si vuole scrivere la tesi in inglese, lingua già inserita di default in \textsf{TOPtesi}, ma non attivata, bisogna dare l'indicazione esplicita dopo l'inizio del documento mediante il comando \cs{english}; così:
\begin{verbatim}
\documentclass[11pt,cucitura]{toptesi}
...
\begin{document}
\english
...
\end{verbatim}

\begin{sloppypar}
Per commutare dall'italiano all'inglese e viceversa basta alternare le dichiarazioni \cs{italiano} e \cs{english}. Attenzione: queste dichiarazioni alterano anche le parole come Chapter o Capitolo, Table o Tabella, quindi per inserire brani nell'altra lingua è  opportuno  servirsi dei comandi di \pack{babel}, in particolare l'ambiente \amb{otherlanguage}, per esempio:
\begin{verbatim}
... disse:
``\begin{otherlanguage}{english}
    Mr Livingstone, I suppose\dots
\end{otherlanguage}''
e si strinsero la mano.
\end{verbatim}
\end{sloppypar}

L'ambiente \amb{otherlanguage} è  adatto per citazioni relativamente lunghe; per citazioni brevi, come quella dell'esempio, sarebbe meglio usare così il comando 
\cs{foreignlanguage}:
\begin{verbatim}
disse: ``\foreignlanguage{english}{Mr Livingstone, I suppose\dots}'' 
e si strinsero la mano.
\end{verbatim}

Con i programmi \prog{xelatex} e \prog{lualatex} le lingue ausiliarie, oltre l'italiano e l'inglese già preinstallate in \pack{toptesi}, basta specificare nel preambolo, per esempio:
\begin{verbatim}
\setotherlanguages{french,spanish}
\end{verbatim}
Oppure, se la lingua richiede un alfabeto speciale, se ne può specificare il font specifico e prescrivere, per esempio:
\begin{verbatim}
\setotherlanguage[variant=ancient]{greek}
\newfontfamily{\greekfont}{GFS Didot}
\end{verbatim}
Ciò fatto si dispone per ogni lingua, tranne quella principale, di un ambiente dal nome uguale alla lingua, che all'occorrenza seleziona anche il font specifico, la cui famiglia abbia un nome che comincia con lo stesso nome della lingua. Per esempio:
\begin{verbatim}
\begin{greek}
Οἰόνται τινές, βασιλεῦ Γέλων, τοῦ ψάμμον τὸν ἀριθμὸν ἄπειρον εἶμεν
τῷ πλήθει· λέγω δὲ οὐ μόνον τοῦ περὶ Συρακούσας τε καὶ τὰν ἄλλαν
Σικελίαν ὑπάρχοντος, ἀλλὰ καὶ τοῦ κατὰ πᾶσαν χώραν τάν τε οἰκημέναν
καὶ τὰν ἀοὶκητον.
\end{greek}
\end{verbatim}
\unless\ifPDFTeX la cui composizione risulta la seguente.\medskip

\begin{greek}\noindent
Οἰόνται τινές, βασιλεῦ Γέλων, τοῦ ψάμμον τὸν ἀριθμὸν ἄπειρον εἶμεν
τῷ πλήθει· λέγω δὲ οὐ μόνον τοῦ περὶ Συρακούσας τε καὶ τὰν ἄλλαν
Σικελίαν ὑπάρχοντος, ἀλλὰ καὶ τοῦ κατὰ πᾶσαν χώραν τάν τε οἰκημέναν
καὶ τὰν ἀοὶκητον.
\end{greek}
\fi

Bisogna definire i comandi \cs{sommario} e \cs{ringraziamenti} nella stessa maniera illustrata qui di seguito per comporre con il programma \prog{pdflatex}  anche quando si usano \prog{xelatex} o \prog{lualatex} .

Invece per comporre la tesi in una lingua diversa dall'italiano e dall'inglese bisogna lavorare un pochino di più perché  bisogna ridefinire alcune cose; per esempio, per scrivere la tesi in spagnolo bisogna agire così:
\begin{verbatim}
\documentclass[12pt,spanish]{toptesi}% <--- la lingua come opzione
                                     %      della classe!
\ExtendCaptions{spanish}{Resumen}{Agradecimientos}
\newcommand*{\spagnolo}{\selectlanguage{spanish}}%
...
\begin{document}
\spagnolo
...
\end{document}
\end{verbatim}

Ovviamente le parole ``Resumen'' (Sommario) e ``Agradecimientos'' (Ringraziamenti) andranno scelte accuratamente; io ho indicato solo ciò che ho trovato sul vocabolario.

È anche possibile servirsi del file di configurazione; per esempio le definizioni dei comandi suddetti possono venire scritte nel file di configurazione e alternando i comandi \cs{italiano}, \cs{english} e \cs{spagnolo} si possono alternare le regole di sillabazione oltre a cambiare le parole \emph{infix}, come Chapter, Capitolo, Capítulo, eccetera.

Se si usa \prog{xelatex} o \prog{lualatex} ci si ricordi che il pacchetto \pack{polyglossia} è già caricato dalla classe \class{toptesi}, che provvede già a dichiarare l'italiano come lingua principale e l'inglese come lingua accessoria; si possono nominare nel preambolo della tesi diverse altre lingue accessorie, per esempio:
\begin{verbatim}
\setotherlanguages{french, spanish}
\end{verbatim}
e per le lingue che richiedono un trattamento particolare si usa un comando simile (al singolare), per esempio:
\begin{verbatim}
\setotherlanguage[variant=ancient]{greek}
\end{verbatim}

Per scrivere in lingue che implicano alfabeti diversi da quello latino, ovviamente, bisogna aver curato di disporre di una configurazione del sistema \TeX\ completa anche dei font che servono. Non dovrebbero esserci problemi con il cirillico e con il greco; per altri alfabeti e per le lingue che si scrivono da destra a sinistra bisogna ovviamente disporre dei pacchetti necessari. Si tenga presente che le versioni moderne del sistema \TeX\ \emph{non} usano come interprete il programma originario di Knuth, ma la sua estensione che in origine si chiamava $\epsilon$-\TeX; la variante che produce il file del documento composto in formato \texttt{.pdf} si chiama \prog{pdftex}, ma questa versione comprende tutte le estensioni di $\epsilon$-\TeX. Bene, queste estensioni servono anche per gestire  le lingue con scrittura retrograda pur di disporre dei pacchetti relativi a queste lingue.

Anche \prog{xelatex} e \prog{lualatex} sono in grado di gestire le lingue retrograde e sono capaci di comporre in verticale (cinese, giapponese,\dots) con modesti adattamenti di alcuni comandi.

\section{Il formato PDF/A}\label{sec:PDFA}

Il Politecnico di Torino, come anche molte università italiane e straniere, intende  ``smaterializzare'' le copie di archivio delle tesi di laurea. Questo intendimento è cominciato nel 2008 ma non sono al corrente dell'esito di questa operazione a cui ho collaborato parzialmente prima di andare in pensione.

Il problema dell'archiviazione elettronica è che il materiale archiviato deve essere reperibile, leggibile e stampabile per un tempo indefinito! Per questo è necessario che esso sia archiviato in un formato standard e che nel futuro continuino ad esistere i programmi per la visualizzazione e la stampa di questo formato, non necessariamente quelli che esistono oggi.

L'International Standards Organization (ISO) ha pubblicato nel 2005 uno standard per l'archiviazione dei file con la norma ISO 19005-1:2005 (nel 2011 e poi ancora nel 2012 sono stati approvati due aggiornamenti dello standard, ma per quello che interessa qui non è necessario vederne in dettaglio le modifiche). Secondo questa norma i file archiviabili devono avere il formato PDF della versione 1.4 e devono soddisfare ad altri requisiti, oltre a contenere un certo numero di \emph{metadati} specificati dalla norma stessa. 

Esistono due sotto"formati: il PDF/A-1a e il PDF/A-1b. Il formato PDF/A-1a deve rispondere ai requisiti del formato PDF/A-1b oltre ad altri requisiti specifici, in particolare che i font siano tutti inclusi nel file ed abbiano codifica UNICODE, che il file debba contenere le informazioni relative alla sua struttura logica, e che queste possano essere esaminate con i motori di ricerca. Viceversa il formato PDF/A-1b, con meno pretese, richiede che i font siano tutti inclusi nel file, anche se non rispettano la codifica UNICODE, e che il file sia riproducibile a schermo e sia stampabile esattamente nello stesso modo di quando il file è stato archiviato. 

Per il Politecnico sarebbe sufficiente il sotto"formato PDF/A-1b. Mi risulta che sia allo studio da parte del Politecnico la possibilità di richiedere il rispetto di una norma molto più restrittiva, che garantisca l'accessibilità e la fruibilità dei documenti archiviati anche a coloro che hanno alcune forme di disabilità; tuttavia a tutt'oggi (2016) l'ISO non ha ancora messo a punto tale norma, quindi l'apposita commissione del Politecnico sta giustamente precorrendo i tempi perché l'Ateneo sia pronto ad attuare la futura norma. Nel frattempo l'apposito gruppo di lavoro del TUG sta cercando di estendere i suoi pacchetti per l'archiviabilità anche per la norma PDF/A-1a, che richiede di disporre di file PDF classificati come \emph{taggged PDF}.

Le istruzioni che seguono per la produzione di un file conforme alla normativa PDF/A sono applicabili se si compone la tesi sia con il programma \prog{pdflatex} sia con \prog{xelatex} sia con \prog{lualatex}. Si notate che se si è predisposto il preambolo del file sorgente per la tesi in modo da poterlo compilare con \prog{pdflatex}, si può ugualmente compilare il  documento anche con \prog{xelatex} o \prog{lualatex}, perché TOPtesi è perfettamente conscio del motore di composizione che si sta usando e sa che il file sorgente parte dal presupposto che si volevano creare gli hyperlink. In altre parole, non c'è bisogno di modificare sostanzialmente il file sorgente per passare da \prog{pdflatex} a \prog{xelatex} o \prog{lualatex}, se non eventualmente usando solo il ramo \emph{vero} oppure \emph{falso} del test indicato qui sotto. Però il file prodotto con \prog{xelatex} necessita di ulteriore elaborazione per soddisfare le specifiche richieste dalla norma ISO relativa al formato archiviabile, mentre con \prog{pdflatex} e \prog{lualatex} non sono necessarie operazioni di post"processing.

Si può comunque fare riferimento alla parte iniziale del preambolo del file usato per comporre questa stessa documentazione, che qui si riporta nuovamente:\label{pag:preambolo-pdfa}
\begin{Verbatim}[fontsize=\small,numbers=left,numbersep=3pt]
% !TEX encoding = UTF-8 Unicode
% !TEX TS-program = luaLaTeX
\begin{filecontents*}{\jobname.xmpdata}
\Title{La classe TOPtesi}
\Author{Claudio Beccari}
\Publisher{Claudio Beccari}
\Keywords{Monografia di laurea\sep 
Tesi di laurea\sep 
Tesi di dottorato\sep 
classe LaTeX\sep 
pdfLaTeX\sep 
XeLaTeX\sep 
LuaLaTeX}
\end{filecontents*}
%
\documentclass[12pt,twoside]{toptesi}
\ProvidesFile{toptesi-it.tex}[2016/11/10 v.0.9.15]
\ifPDFTeX
%
  \usepackage[a-1b]{pdfx}
%
    \usepackage[utf8]{inputenc}
    \usepackage[T1]{fontenc}
    \usepackage{newtxtext,newtxmath,textcomp,textalpha}
    \usepackage{amsmath,amssymb}
    \usepackage{mflogo}
    \usepackage{guit}
    \setactivedoublequote
\else
    \ifLuaTeX
      \usepackage[a-1b]{pdfx}
    \fi
    \usepackage{fontspec}
    \usepackage{xcolor}
    \newcommand*\MP{{\setbox0\hbox{M}\relax
    \includegraphics[height=\ht0]{MPlogo}}}
    \newcommand*\GuIT{{\setbox0\hbox{Hg}\relax
    \raisebox{-\dp0}{\includegraphics[height=\dimexpr\ht0+\dp0]{GuITlogo}}}}
    \setmainfont[Ligatures=TeX]{TeX Gyre Termes}
    \setsansfont[Ligatures=TeX, Scale=MatchLowercase]{TeX Gyre Heros}
    \setmonofont{UM Typewriter}
    \setmainlanguage[babelshorthands]{italian}
    \usepackage{amsmath}
    \usepackage{unicode-math}
    \setmathfont{XITS Math}
    \setotherlanguage[variant=ancient]{greek}
    \newfontfamily{\greekfont}{GFS Didot}
\fi
\usepackage{metalogo,longtable,booktabs,array,tabularx,fancyvrb}
%
\usepackage{imakeidx}
\usepackage{hyperref}
    \hypersetup{%
        pdfpagemode={UseOutlines},
        bookmarksopen,
        pdfstartview={FitH},
        colorlinks,
        linkcolor={blue},
        citecolor={blue},           
        urlcolor={blue}
    }
\end{Verbatim}

Un commento va fatto subito: il lettore non si spaventi di una prima parte di preambolo così complessa, e non ha nessun bisogno di copiarla integralmente. Per questo documento ho predisposto un test per verificare con quale programma di composizione compilo il documento e come si vede imposto i font e le lingue come specificato sopra, ma nel ramo \emph{falso} (cioè non si sta compilando con \prog{pdflatex} ma con \prog{lualatex} oppure \prog{xelatex}; tuttavia se il programma con cui si compila è \prog{xelatex} si evita di caricare il pacchetto speciale \pack{pdfx}, necessario per produrre file PDF/A compatibili. Il lettore sa perfettamente con quale programma vuole compilare la sua tesi e può tranquillamente mettere nel suo preambolo solo il ramo \emph{vero} del test se compila con \prog{pdflatex} o solo il ramo \emph{falso} se compila con \prog{lualatex} o con \prog{xelatex}. Se vuol produrre un file PDF/A compatibile con \prog{xelatex} ha due vie:
\begin{itemize}
\item lascia perdere \prog{xelatex} e compila con \prog{lualatex}; il frutto della compilazione è persino migliore, visto che \prog{lualatex} usa le funzionalità del pacchetto \pack{microtype} in modo più avanzato rispetto a \prog{xelatex};
\item ma se proprio non può fare a meno di usare \prog{xelatex},  usa solo il ramo \emph{falso} del test \verb|\ifPDFTeX| e toglie il test \verb|\ifLuaTeX|, coservando solo l'istruzione \cs{usepackage} per caricare \pack{pdfx}; tuttavia ciò non basta perché deve poi procedere ad un certo postprocessing come specificato nel paragrafo~\ref{sec:XePDFA}.
\end{itemize}

Nel codice precedente le prime due righe sono commenti per il programma di compilazione, ma per il programma di editing sono comandi di autoconfigurazione; essi dicono al programma di editing di salvare il documento dopo la prima creazione e dopo ogni modifica con la codifica \opz{utf8}; e la seconda dice che quando si preme il bottone per la compilazione deve dare al sistema operativo di lanciare il programma \prog{lualatex}. Ricordo che questo tipo di commenti di autoconfigurazione sono decifrati correttamente dagli editor TeXShop, TeXworks e TeXstudio (forse anche da altri editor, ma questi tre sono quelli che io consiglieri per ogni utente).

Le righe dalla 3 alla 14 contengono l'ambiente \amb{filecontents*} che provvede a generare la lista dei metadati che ho usato per questo documento.

La riga 16 contiene la dichiarazione della classe; come si vede specifico solo il corpo normale dei font al valore di 12\,pt, e seleziono la composizione in bianca e volta, successivamente non specifico nessuna correzione per la legatura.
Importante per questo documento è la riga 17, dove specifico il nome del file, ma ne specifico anche la data e la versione; così sono sempre sicuro di quale versione mi sto servendo e i lettori che trovano questo file fra la documentazione di \class{toptesi} sanno subito se si tratta di documentazione aggiornata o se non sia il caso di aggiornare la propria installazione.

Con la riga 18 comincia il test importante per configurare il preambolo correttamente secondo le necessità di compilazione; dalla riga 19 alla riga 28 si ha il ramo \emph{vero} da usarsi quando si compila con \prog{pdflatex}; invece dalla riga 30 alla riga 45 si ha il ramo \emph{falso} da usarsi quando si compila con \prog{lualatex}; per compilare con \prog{xelatex} ho già detto che, oltre a togliere le righe 30 e 32 (ma non la riga 31), bisogna poi procedere come specificato nel paragrafo~\ref{sec:XePDFA}.

Le righe successive dalla 47 alla 59 servono per mostrare non solo i pacchetti che ho caricato, ma anche come ho configurato le opzioni per \pack{hyperref}; nessuna opzione nella chiamata del pacchetto, ma tutte le impostazioni eseguite mediate l'argomento del comando \verb|\hypersetup|.

Vorrei sottolineare l'opportunità di specificare come indicato nel codice precedente i metadati richiesti per la compilazione del documento in modo che sia conforme con lo standard PDF/A. Non è obbligatorio ma in questo modo si è sicuri che il file dei metadati abbia l'estensione giusta e il nome assolutamente coincidente con il nome del main file del documento, cioè della tesi.

L'altro punto importante è non solo caricare il pacchetto \pack{pdfx}, ma è necessario caricarlo nell'ordine indicato, cioè come primo pacchetto.

Ho collaudato questo preambolo compilando sia  con \prog{pdflatex} sia con \prog{lualatex}; volendo, avrei potuto seguire le raccomandazioni che scrivo nel paragrafo~\ref{sec:XePDFA} ma non l'ho fatto; visto che ottengo lo stesso risultato, anzi un risultato migliore, con \prog{lualatex} mi è parso un collaudo superfluo. Con questo non voglio dire che \prog{xelatex} sia diventato inutile; esso come \prog{pdflatex} ha un file di formato che contiene tutte le impostazioni per tutte le lingue che il sistema \TeX\ può gestire, mentre \prog{lualatex} in questo momento di ogni lingua può gestire una sola impostazione per la sua sillabazione; per esempio il greco ha tre varianti e il file di formato di \prog{xelatex} le potrebbe gestire tutte e tre come ci riesce \prog{pdflatex}; lo stesso vale per il latino e per diverse altre lingue; \prog{lualatex} al momento può gestire una sola variante di ogni lingua. Perciò in questo momento \prog{xelatex} potrebbe essere preferibile a \prog{lualatex} per le tesi di carattere linguistico, filologico, umanistico in generale. Personalmente ora (giugno 2016) dispongo di impostazioni mie personali per gestire le tre varianti del greco e le tre varianti del latino, ma esse non sono ancora state approvate dal gruppo di lavoro \textsf{tex-hyphen} del \TeX\ Users Group (TUG) e non le posso distribuire senza la loro approvazione; quando saranno approvate, saranno direttamente disponibili con gli aggiornamenti di \TeXLive.


Va inoltre notato che i due rami del test contengono le istruzioni per caricare pacchetti diversi e alcune definizioni, quelle dei loghi di \GuIT\ e di \MP, che ricorrono a font vettoriali, ma non OpenType, per cui potrebbero dare luogo a problemi per la conformità alle norme PDF/A quando si usa \XeLaTeX. Si spiegherà meglio nell'apposito paragrafo.


%\goodpagebreak


\subsection{\texorpdfstring{\prog}{}{pdflatex} e il formato PDF/A}

Usando \prog{pdflatex} è necessario usare il pacchetto specifico \pack{pdfx}  richiamandolo  con l'opzione giusta; precisamente il preambolo deve contenere la richiesta:
\begin{verbatim}
\usepackage[a-1b]{pdfx}
\end{verbatim}
come è stato mostrato precedentemente nella pagina~\pageref{pag:preambolo-pdfa}; quando si usa \prog{pdflatex} non è molto importante l'ordine con cui si caricano i vari pacchetti, ma la documentazione di \pack{pdfx} raccomanda di caricarlo il più presto possibile. \pack{pdfx} carica già \pack{hyperref} per configurarne i collegamenti ipertestuali in modo conforme alle specifiche della norma PDF/A. Questo implica, però, che il pacchetto \pack{pdfx} venga caricato prima che nel preambolo venga caricato \pack{hyperref}; anzi, avendo caricato \pack{pdfx} non è necessario caricare \pack{hyperref}, ma basta configurarlo con \cs{hypersetup} per le esigenze specifiche del documento. Non è vietato caricarlo di nuovo, purché nel preambolo, dopo aver caricato \pack{pdfx}, \pack{hyperref} sia caricato senza opzioni proprio per evitare l'errore ``option clash''. Questa possibilità di tenere separato il caricamento di \pack{hperref} dalla sua configurazione mediante \cs{hypersetup}, è comoda perché consente di aggiungere la chiamata di \pack{pdfx} (prima di \pack{hyperref}) in un secondo tempo, quando si vuole aggiungere la compatibilità con le norme PDF/A.

Il pacchetto \pack{pdfx} è già presente con ogni installazione completa del sistema \TeX; se si disponesse di una installazione di base o, comunque, parziale e il pacchetto non fosse già installato, bisogna provvedere con i metodi specifici di ciascuna installazione. Se  consiglia la lettura della sua documentazione, perché consente una certa dose di personalizzazione.

Il pacchetto \pack{pdfx} con l'opzione \opz{a-1b} provvede a quasi tutto il necessario, per esempio a definire un profilo di colore, ma non provvede ai metadati specifici di ogni particolare documento, di cui invece parlerò più avanti.

I {metadati} di carattere generale sono inseriti nel file di uscita dall'azione diretta del file \file{pdfx.sty}; ma per i {metadati} specifici del documento che si sta componendo è necessario  predisporre nella cartella dove risiede il file principale della tesi (quello sul quale opera l'eseguibile \prog{pdflatex}) un file contenente i {metadati}. Il metodo migliore per farlo è mostrato nel codice della pagina~\pageref{pag:preambolo-pdfa}: si tratta di inserire i metadati all'interno dell'ambiente \amb{filecontents*} prima ancora della dichiarazione della classe; questo ambiente \LaTeX\ è l'unica struttura del linguaggio di mark-up che può essere inserita prima di \cs{documentclass}.

Ci si ricordi solo che il pacchetto \pack{toptesi} controlla solo se si sta usando \prog{pdflatex} e assume che se non si sta componendo con \prog{pdflatex} allora si sta usando \prog{xelatex} oppure \prog{lualatex}. In particolare si ottiene un errore se si compila con \prog{latex}, il programma ancora esistente ma che io considero obsoleto, con il quale si produce un file in formato DVI, ormai quasi estinto. Si noti che, in effetti, sebbene il comando \prog{latex} esista ancora, il motore di composizione di \prog{latex} e di \prog{pdflatex} è lo stesso, ma ne differiscono le impostazioni; il test \cs{ifPDFTeX} che usa TOPtesi per conoscere se si sta usando \prog{pdftex} per produrre un file in formato PDF piuttosto che in formato DVI, non è detto che riesca a distinguere questi due casi. Non ho mai verificato, perché usando un calcolatore Apple, il mio formato di default è sempre PDF; per ottenere il formato DVI devo dare comandi specifici mediante il terminale, cosa che evidentemente non faccio mai perché, oltretutto, non dispongo di un visualizzatore per il formato DVI.

Se esistono validi motivi per non usare \prog{pdflatex}, allora bisogna usare il semplice \prog{latex}, e poi convertire il file DVI ottenuto in un file PS da trasformare i un file PDF con gli strumenti del sistema \TeX, e successivamente procedere la conversione nel formato PDF/A con il metodo esposto nel paragrafo~\ref{sec:pdfPDFA}. È meglio evitare di usare quei pacchetti che richiedono  comandi PostScript, come per esempio \pack{PSTricks} o \pack{XYpic}, ma è preferibile usare i pacchetti di grafica, come \pack{pgf} con il suo modulo \pack{tikz}, che consentono di fare quasi tutto quello che si può fare con gli altri pacchetti che richiedono il linguaggio PostScript. Inoltre esistono pacchetti da usare con \prog{pdflatex} che permettono di usare (almeno) il potentissimo PSTricks anche con \prog{pdflatex}; questi pacchetti, lavorando in background senza che l'utente debba metterci mano,  estraggono il codice di PSTricks dal file sorgente, lo salvano in un nuovo file \file{.tex} di servizio che compilano con \prog{latex} e ne trasformano il file di uscita in formato PDF per poi importarlo già scontornato nel file che interessa all'utente; se fra una compilazione e l'altra non sono state eseguite modifiche al codice PSTricks, questa operazione in background viene tranquillamente saltata, quindi non si nota nemmeno nessun rallentamento nell'elaborazione della tesi. Vale comunque il consiglio di non usare questi moduli che richiedono l'uso del linguaggio PostScript, perché i moduli che lavorano direttamente con il linguaggio PDF lavorano altrettanto bene nella stragrande maggioranza dei casi.

Bisogna ricordare che produrre un file PDF/A non è semplice, ma se si usa \prog{pdflatex} con i font di default, se si incorporano solo figure PNG (senza trasparenze) e JPEG con profilo di colore RGB (red, green, blue), se ogni file PDF da incorporare contiene al suo interno anche i font che eventualmente sono necessari per comporre il suo testo, non si dovrebbero incontrare problemi di certificazione della natura PDF/A del file della tesi. 

Più avanti c'è un paragrafo dedicato al caso che un file così prodotto non risulti conforme alla norma PDF/A.

\goodpagebreak

\subsection{\texorpdfstring{\prog}{}{lualatex} e il formato PDF/A}
A partire dalla distribuzione del sistema \TeXLive del 2016 anche il programma \prog{lualatex} può produrre direttamente file PDF/A conformi.

Con \prog{lualatex} è importante specificare il preambolo come mostrato nella pagina\pageref{pag:preambolo-pdfa} sia per quel che riguarda i metadati sia per quel che riguarda la posizione della chiamata del pacchetto \pack{pdfx}. 

Il programma \prog{lualatex} è meno elastico rispetto al programma \prog{pdflatex} per quel che riguarda la posizione della chiamata del pacchetto \pack{pdfx} che deve essere assolutamente il primo dopo \cs{documentclass}.

Per il resto valgono le stesse indicazioni scritte per l'uso del programma \prog{pdflatex}. Sottolineo ancora che il file sorgente della tesi deve essere assolutamente salvato con la codifica \opz{utf8} e che i metadati devono essere scritti e salvati nel loro file \file{.xmpdata} con la stessa codifica.

Se i metadati devono essere usati da programmi di biblioteconomia per la ricerca dei contenuti, delle parole chiave, e simili, suggerirei di scrivere i metadati in inglese sia perché probabilmente la ricerca delle informazioni biblioteconomiche risulta facilitata anche all'estero, sia perché si evitano i caratteri accentati, almeno in tutti i campi in cui l'italiano non è essenziale, come il titolo (se è in italiano) e i nomi propri.

Come si legge nella parte di preambolo che appare nella pagina~\pageref{pag:preambolo-pdfa}, questo stesso file è stato composto con \prog{lualatex} e la verifica della conformità ha dato esito positivo.



\subsection{\texorpdfstring{\prog}{}{xelatex} e il formato PDF/A}\label{sec:XePDFA}

Con \prog{xelatex} le cose sono un po' più complesse che con \prog{lualatex}, ed è per questo motivo che consiglio di usare direttamente \prog{lualatex}.
Il motivo è dovuto al fatto che \prog{xelatex} non produce direttamente la sua uscita nel formato PDF, ma in un formato DVI esteso che viene trasformato in PDF dal programma \prog{xdvipdfmx}, che di per sé non sa nulla a proposito del fatto che il file da trasformare contenga o non contenga dei metadati.

Per questo motivo si deve compilare ``a mano'', dando il comando di compilazione dal terminale nel modo seguente:
\begin{flushleft}\ttfamily
xelatex -shell-escape -output-driver="xdvipdfmx -z 0" <filename>.tex
\end{flushleft}
Come si vede \prog{xelatex} deve lavorare con l'opzione \opz{-shell-escape} che consente di lanciare un programma esterno; in questo caso l'uscita di \prog{xelatex} viene inviata al driver di uscita \prog{xdvipdfmx} (e fin qui non ci sono sorprese rispetto a quello che \prog{xelatex} fa di solito), ma la differenza consiste nell'opzione passata a \prog{xdvipdfmx}: \opz{-z 0}.  Questa opzione dice al programma di non comprimere il suo file di uscita; questo fatto produce due effetti: quello buono è che non comprime i metadati; quello ``cattivo'' è che il file di uscita è enormemente grande, anche 10 volte più grande rispetto a quando fosse compresso.

Per il resto, avendo predisposto il main file della tesi come se si dovesse lavorare con \prog{lualatex} e avendo osservato le stesse cautele per i font, i colori, eccetera, il file ottenuto, nonostante le sue dimensioni, può venire sottoposto a verifica attraverso il modulo Preflight di Adobe Acrobat Pro, come si può fare anche quando il file PDF fosse ottenuto con \prog{pdflatex} o \prog{lualatex}; ma il buono, questa volta, è che, se il file ottenuto passa la verifica di conformità, Adobe Acrobat è anche in grado di comprimere il file, lasciando inalterati (non compressi) i metadati.

Vista questa trafila, si capisce anche perché sconsiglio di comporre la tesi con \prog{xelatex} ma consiglio di usare direttamente \prog{lualatex}.

\subsection{Uso di \texorpdfstring{\prog}{}{ghostscript} o di \texorpdfstring{\pack}{}{pdfpages}}

Esistono altri metodi per ottenere la produzione di file conformi alla norma PDF/A. Uno è quello di ottenere la trasformazione mediante il programma \prog{ghostscript} e l'altro  consiste nel produrre un secondo file importando le pagine del primo mediante il pacchetto \pack{pdfpages}. Nel seguito si illustrerà solo il secondo metodo.

Infatti è meglio creare un piccolo file presumibilmente PDF/A-compatibile e verificarne la compatibilità: se è compatibile, non ci sono problemi ad importarlo nella tesi; se è incompatibile prima si esaminano le cause dell'incompatibilità,  prima lo si può correggere per renderlo compatibile o si possono prendere decisioni valide su cosa farne per importarlo nella tesi.

In versioni precedenti di questa guida, si era illustrato il metodo per trasformare un file PDF in un file PDF/A usando \prog{ghostscript}; in realtà \prog{ghostscript} non offre vantaggi particolari, se non quello di poter trasformare un file PS direttamente in un file PDF/A. Il vantaggio è molto modesto visto che il sistema \TeX\ dispone autonomamente dei mezzi per trasformare un file PS in un file PDF. Perciò in questa versione della guida non si parla più di come usare \prog{ghostscript} per produrre file PDF/A-compatibili. 

\iffalse
\subsubsection{Uso di \texorpdfstring{\prog}{}{ghostscript}}

Dopo aver ottenuto un file PDF  della  tesi e ci si ritrova con un file che non è ancora conforme allo standard PDF/A, si può provvedere con altri programmi. Qui indicherò due o tre metodi che possono avere vantaggi o svantaggi; nessuno è perfetto, ma funzionano.

Si può usare \prog{ghostscript} per trasformare un file PDF in un altro PDF/A. Bisogna predisporre nella stessa cartella dove si trova il main file del documento un file che contenga i {metadati}; per comporre questo documento composto con \prog{xelatex}, il file adatto per \prog{ghostscript} è il seguente:
\begin{Verbatim}[fontsize=\small]
%!
% This is a sample prefix file for creating a PDF/A document.
% Feel free to modify entries marked with "Customize".
% This assumes an ICC profile to reside in the file (srgb.icc),
% unless the user modifies the corresponding line below.

% Define entries in the document Info dictionary :
/ICCProfile (~/icc/srgb.icc) def            % Customize
def

[  /Title (La classe TOPtesi)               % Customize.
   /Author (Claudio Beccari)                % Customize.
   /Subject (File PDF ottenuto con XeLaTeX) % Customize.
   /DOCINFO  pdfmark

% Define an ICC profile :

[/_objdef {icc_PDFA} /type /stream /OBJ pdfmark
[{icc_PDFA}
<<
  /N currentpagedevice /ProcessColorModel known {
    currentpagedevice /ProcessColorModel get dup /DeviceGray eq
    {pop 1} {
      /DeviceRGB eq
      {3}{4} ifelse
    } ifelse
  } {
    (ERROR, unable to determine ProcessColorModel) == flush
  } ifelse
>> /PUT pdfmark
[{icc_PDFA} ICCProfile (r) file /PUT pdfmark

% Define the output intent dictionary :

[/_objdef {OutputIntent_PDFA} /type /dict /OBJ pdfmark
[{OutputIntent_PDFA} <<
  /Type /OutputIntent                  % Must be so (the standard requires).
  /S /GTS_PDFA1                        % Must be so (the standard requires).
  /DestOutputProfile {icc_PDFA}        % Must be so (see above).
  /OutputConditionIdentifier (sRGB)    % Customize
>> /PUT pdfmark
[{Catalog} <</OutputIntents [ {OutputIntent_PDFA} ]>> /PUT pdfmark
\end{Verbatim}

Questo file è formato dal nome del file \texttt{.tex.}, (per esempio \texttt{toptesi-it} per questo file), agglutinato alla ``desinenza'' \texttt{-def.ps}, per un nome complessivo, per esempio, \texttt{toptesi-it-def.ps}. Ripeto: deve essere memorizzato nella stessa cartella dove si trova il main file del documento. Voglio far notare che ho usato questo file di metadati per alcuni anni; oggi la libreria di \prog{ghostscript}, versione 9.16, indica di modificare un file \verb|PDFA_def.ps| diverso da quel modello, ma, devo ammetterlo, la documentazione di che cosa indicare nelle nuove parti configurabili è tutt'altro che chiara e, pur essendo possibile farlo, non ci sono riuscito.

Si noti ancora che i file \file{.icc} necessari per specificare i profili di colore devono risiedere in una cartella accessibile attraverso uno dei percorsi indicati nella variabile di sistema \verb|PATH|; io per la mia macchina ho creato una cartella chiamata \verb|icc|, inserita nella mia \verb|$HOME|; nei sistemi UNIX il simbolo \verb|~| indica questa `casa' delle cartelle; nei sistemi Windows recenti, la \verb|$HOME| non è rappresentata da un simbolo, ma corrisponde al nome dell'utente nella cartella \verb|C:\Users\|; quindi la \verb|$HOME| dell'utente Mario, sarà indicata con \verb|C:\Users\Mario\|. Nella cartella \verb|icc| (o comunque la si voglia chiamare) bisogna copiare i vari file \file{.icc} distribuiti insieme all'installazione di \prog{ghostscript}; nelle macchine UNIX, questi file si trovano in una sottocartella di sistema che si chiama \verb|/usr/local/share/ghostscript/|\meta{versione}\verb|/iccprofiles/|. Con la versione che sto usando \meta{versione} vale 9.16 e contiene una decina di file che descrivono i profili di colore, compreso il file \file{srgb.icc} indicato nel precedente listato di codice PostScript (che quindi, se va bene per lo scopo, non necessita di nessuna personalizzazione).

Comunque per eseguire la conversione conviene predisporsi una procedura .bat (per Windows) o bash {per UNIX} del tipo seguente (mostro la procedura bash che con minime modifiche si trasforma in una procedura bat) dal nome \texttt{pdf2pdfa}:
\begin{verbatim}
#!/bin/bash
file1=$1.pdf
file2=$1-a.pdf
file3=$1-def.ps
echo 'Running gs to pdfa'
# QUANTO SEGUE DEVE ESSERE SCRITTO SU UNA SOLA RIGA
gs -dPDFACompatibilityPolicy=1 -dPDFA=1 -dBATCH -dNOPAUSE 
-sColorConversionStrategy=/RGB -sDEVICE=pdfwrite 
-sOutputFile=$file2 $file3 $file1
\end{verbatim}

Salvato il file in un ramo del \texttt{PATH} dove il sistema operativo può trovare i file (per esempio in \texttt{~/bin/}) e assicuratici che sulle macchine UNIX il file abbia impostato il bit di eseguibilità, basta dare il comando nella finestra comandi:
\begin{flushleft}
\texttt{pdf2pdfa} \meta{Nome del main file senza estensione}
\end{flushleft}
e \prog{ghostscript} si mette a lavorare finché produce in uscita un file PDF con il nome del file originario allungato con \texttt{-a}. A questo punto si può procedere con il software di verifica della natura PDF/A del file ottenuto.
Se bisogna trasformare un file PS, basta cambiare la seconda riga \verb|file1=$1.pdf| in \verb|file1=$1.ps| senza la necessità di modificare altro.

Le variazioni per Windows consistono nel fatto che l'eseguibile di \prog{ghostscript} si chiama \prog{gswin32c} invece che \prog{gs} e che i parametri di una macro si indicano con \verb|%1|, \verb|%2|, 
eccetera, mentre quelli per UNIX si indicano con \verb|$1|, \verb|$2|, eccetera; in bash i commenti iniziano con il segno \texttt{\#} mentre in bat essi cominciano con \texttt{REM}.

Nessuno vieta di arricchire la definizione delle due versioni della procedura per verificare l'esistenza del file PDF o PS da convertire, o l'esistenza del file dei {metadati}, con la possibilità di emettere dei messaggi cosicché l'utente possa regolarsi nel caso qualcosa risulti mancante. Tuttavia l'essenza dello script bash è quella indicata.

Va notato che i collegamenti ipertestuali all'interno del file convertito con \prog{ghostscript} in un file conforme alla norma PDF/A vengono tutti disattivati; secondo me non è una grave perdita per un file da archiviare, ma certamente lo è per un documento da consultare a schermo.
\fi

\subsubsection{Uso di \texorpdfstring{\pack}{}{pdfpages}}\label{sec:pdfPDFA}

Un metodo con il quale ho avuto risultati positivi ricorre al pacchetto \pack{pdfpages}, ma ha il difetto di perdere gli eventuali collegamenti ipertestuali interni oltre a quelli esterni.

Supponiamo quindi di disporre di un file PDF composto secondo le raccomandazioni indicate nei paragrafi precedenti. Tanto per mettere i puntini sulle `i', il lettore attento ha notato che le due parti del test indicato nella pagina~\pageref{pag:preambolo-pdfa} sono leggermente diverse, infatti per i loghi \GuIT\ e \MP\ ho creato le definizioni per produrre questi loghi importando e scalando due file PDF appositamente predisposti (con \prog{pdflatex}), conformi alla norma PDF/A e opportunamente scontornati. Li ho predisposti con i loghi molto grandi, ma pur sempre vettoriali, in modo che possano venire ingranditi o rimpiccioliti a piacere. Entrambe le macro  \cs{GuIT} \cs{MP}, infatti, scalano le loro immagini PDF in modo che siano grandi come il testo circostante e abbiano la linea di base esattamente allineata con quella del testo. In quanto file PDF/A, essi contengono tutte le informazioni che devono contenere, metadati e font, esattamente come la norma richiede; presi a sé, essi passano le verifiche eseguite con Adobe Acrobat Pro~XI. 

{\tolerance=3000 Si prepara ora un altro semplicissimo file che viene chiamato, per esempio, \file{toptesi-it-pages.tex} con il seguente contenuto:
\begin{verbatim}
% !TEX encoding = UTF-8 Unicode
% !TEX TS-program = pdflatex
\documentclass[a4paper]{report}
\usepackage[a-1b]{pdfx}
\usepackage{pdfpages}
\nofiles
\begin{document}\pagestyle{empty}
\includepdf[pages=-]{toptesi-it.pdf}
\end{document}
\end{verbatim}
e lo si affianca con il corrispondente file \file{toptesi-it-pages.xmpdata} il cui contenuto è identico al file \file{toptesi-it.xmpdata} che si sarebbe preparato per comporre l'analogo file se lo si fosse potuto comporre direttamente con \prog{pdflatex}.\par}

Il piccolo file precedente è in grado di produrre un file PDF/A perché il suo preambolo è correttamente predisposto per questo scopo; carica il pacchetto \pack{pdfpages} che serve per importare tutte  o una selezione di pagine di un file PDF comunque composto, dentro il file che si ottiene elaborandolo con \textcolor{red}{\prog{pdflatex.}} Se il file \file{toptesi-it.pdf} non contiene file importati con caratteristiche incompatibili, il file PDF ottenuto risulta conforme alla norma PDF/A, ma se il file di partenza conteneva collegamenti ipertestuali, il file finalmente ottenuto è privo di collegamenti ipertestuali.

Ho usato come esempio il nome di questo stesso file di documentazione; il lettore capisce che si tratta solo di un esempio, perché questo documento si compila perfettamente in formato PDF/A senza ricorrere a questo procedimento; ma, se per esempio, si fosse composta la tesi con un word processor e la si fosse esportata in formato PDF, questo procedimento potrebbe crearne la versione PDF/A compatibile senza sforzo e con la sola spesa di perdere i collegamenti ipertestuali interni.

Più comodamente il procedimento descritto in questo paragrafo può servire per aggiungere quanto manca, per esempio, ad una figura PDF per verificarne la conformità oppure per scoprirne le cause di non conformità al fine, eventualmente, modificarla in modo che sia compatibile tanto da poterla includere nella tesi vera e propria.

\subsubsection[Non usare \texorpdfstring{\pack}{}{pax} e \texorpdfstring{\pack}{}{pdfpages} assieme]{Non usare i due pacchetti \pack{pax} e \pack{pdfpages} assieme allo scopo di conservare i collegamenti ipertestuali}
Il pacchetto \pack{pax} è stato costruito per ridare funzionalità ai collegamenti ipertestuali contenuti in file PDF immessi completamente o in parte dentro altri file PDF. Funziona perfettamente, ma il pacchetto \pack{pdfx} non riesce a gestire le operazioni del pacchetto \pack{pax}, per cui tutte le informazioni relative ai collegamenti vengono ripristinate, ma il file complessivo non rispetta le specifiche della norma PDF/A e gli errori segnalati dalla funzione Preflight di Adobe Acrobat Pro~XI riguardano solo gli hyperlink, quando invece lo stesso file PDF ricomposto solo con l'uso di \pack{pdfx} e \pack{pdfpages} è perfettamente conforme con le norme PDF/A, anche se i suoi link non sono attivi.


\section{Verifica della conformità}

Bisogna innanzi tutto disporre degli strumenti per verificare se un file è conforme alla norma PDF/A. 

In rete ci sono siti dove viene eseguita la verifica della conformità di un file PDF con la norma del sotto"formato PDF/A-1b, ma di solito l'esito della verifica è negativo, per lo meno a me non è mai capitato che un file conforme venisse riconosciuto tale mediante le verifiche in rete.

Per verificare in modo serio la conformità ci sono due strade che non si escludono a vicenda.
\begin{enumerate}
\item Ci si procura la versione sperimentale (2016) del programma \prog{verPDF}; questo software è ancora in fase di sviluppo e per questo è considerato sperimentale. L'azienda che se ne occupa è sostenuta dalla Unione Europea proprio perché la versione definitiva si aperta e libera. Non è ancora affidabile al 100\%, ma funziona benissimo e i suoi verdetti, secondo l'esperienza già fatta, sono praticamente sempre concordi con quelli del modulo Preflight di Adobe Acrobat Pro~XI. Questo programma, disponibile per le tre piattaforme Windows, Mac, Linux, è ancora un po' ruspante da installare, e si può sperare che la versione definitiva, attesa epr il 2017, sia più facile da installare.
\item Si cerca in dipartimento una stazione di lavoro dove un computer sia dotato del programma commerciale {Adobe Acrobat Pro XI}; spesso i dipartimenti dispongono di licenze multiple, oppure è possibile che un ricercatore dia una mano lasciando usare un suo calcolatore sul quale è montato il software indicato. Si procede poi come indicato qui di seguito.
\item La Adobe, come molte altre imprese produttrici di software, dispone di un programma \emph{Education} che consente agli studenti e ai docenti di università e scuole secondarie superiori di acquistare i loro software a prezzi molto, molto vantaggiosi. Io come privato cittadino non mi sarei mai comprato l'Adobe Acrobat Pro~XI se non avessi avuto la possibilità di avvantaggiarmi di questa offerta vantaggiosissima, ma come professore ho potuto farlo senza bisogno di ricorrere ad una versione fornitami dal Politecnico sotto una sua licenza multipla.
\end{enumerate}

\subsection{Verifica con veraPDF}
Nell'installare \prog{veraPDF} si sarà avuta attenzione di creare anche un link simbolico verso la componente \prog{verapdf-gui} che è l'applicazione da usare; il link simbolico si troverà in una cartella che sia sul percorso di ricerca dei programmi da eseguire, secondo le impostazioni della propria macchina; comunque tutti i sistemi operativi consentono di impostare il percorso di ricerca attraverso i loro propri comandi o attraverso interfacce grafiche.

Fatto questo, si apre un terminale e si dà il comando \texttt{verapdf-gui}; si pare una finestra come quella che appare nella figura~\ref{fig:veraPDFinitial}.

\begin{figure}[!htb]\centering
\includegraphics[width=0.7\textwidth]{VeraPDFinitial}
\caption{Finestra iniziale di veraPDF}\label{fig:veraPDFinitial}
\end{figure}

In questa interfaccia grafica si clicca il bottone \fbox{Choose PDF} e nella finestra che si apre si cerca la cartella che lo contiene e il file da controllare. Avendo selezionato il file da controllare si ``accende'' il tasto \fbox{Execute}, cliccato il quale parte l'azione di verifica, che dura diversi secondi; finita la verifica, a seconda dell'esito, nella riga allo stesso livello del tasto \fbox{Execute} appare una scritta verde che segnala la conformità del file, oppure na scritta rossa che ne segnala la non conformità, come appare nelle due schermate della figura~\ref{fig:veraPDFesito}.

\begin{figure}[!htb]
\centering
$\vcenter{\hsize=0.7\textwidth\includegraphics[width=0.7\textwidth]{veraPDFconformance}}$\qquad\setbox0\hbox{\((a)\) esito positivo}\parbox{\wd0}{\box0}
\bigskip

$\vcenter{\hsize=0.7\textwidth\includegraphics[width=0.7\textwidth]{veraPDFnoconformance}}$\qquad\setbox0\hbox{\((b)\) esito negativo}\parbox{\wd0}{\box0}
\caption[Verdetto dell'analisi di veraPDF su due file]{Verdetto dell'analisi di veraPDF su due file}\label{fig:veraPDFesito}
\end{figure}

Qualunque sia l'esito conviene visualizzare ed eventualmente salvare il verdetto cliccando sui tasti \fbox{View XML} e/o \fbox{Save XML}, oppure sui corrispondenti tasti~HTML. Se il verdetto è positivo, quanto si stampa può valere come certificazione di conformità; se l'esito è negativo il rapporto contiene  una succinta spiegazione dei motivi per i quali il file non è conforme.

\subsubsection{Verifica con Adobe Acrobat Pro XI}
Disponendo  di Adobe Acrobat Pro XI, si apre il file PDF, si va nella voce di menù \textsf{Edit} e si sceglie l'opzione \textsf{Preflight}; nella finestrina che si apre si sceglie la voce \textsf{PDF/A compliance} e sotto questa la voce \textsf{Verify compliance with PDF/A-1b}; poi si clicca sul bottone \textsf{Analyse}; dopo pochi secondi si apre una seconda finestrina dove appare il risultato dell'analisi. Se il risultato è positivo il file è definitivamente pronto come conforme alla norma PDF/A-1b. 

Se invece l'analisi indicasse errori, bisogna provvedere a correggerli; è vero che nella stessa finestrina c'è anche il bottone \textsf{Analyse and fix}, ma l'operazione di ``aggiustamento'' di solito riesce a correggere errori molto veniali, quindi con i programmi di composizione del sistema \TeX\ (col quale non si compiono errori veniali) questo aggiustamento è raro che abbia successo.

\subsubsection{Correzione degli errori}

Bisogna avere pazienza e cercare di leggere con attenzione i messaggi di diagnostica che \prog{veraPDF} o \textsf{Preflight} producono. Se mancano i metadati e/o il profilo di colore, significa che il file non è stato prodotto con le procedure descritte nei paragrafi precedenti.

Gli errori che restano sono di solito dei tipi seguenti.
\begin{enumerate}
\item Se è stato specificato un profilo di colore CMYK (celeste, lilla, giallo, nero) si è fatto un errore nello specificare il profilo di colore oppure si è inserita un'immagine a colori che è colorata con lo schema CMYK. Nel primo caso è bene verificare di avere specificato il profilo giusto. Nel secondo caso bisogna convertire il profilo di colore mediante qualsiasi programma di fotoritocco; segnalo il programma aperto e libero GIMP (GNU Image Management Program), disponibile per le tre piattaforme più comuni e i sistemi operativi Mac, Windows e Linux. Esso è in grado di aprire immagini in ``qualsiasi'' formato bitmapped o vettoriale, di elaborarlo secondo i desideri dell'utente, per poi salvarlo in qualsiasi formato bitmapped, ma non nei formati vettoriali.
%
\item Alcuni font hanno glifi di larghezza nulla, fra questi ci sono almeno due glifi della famiglia matematica \texttt{cmsym}; \pack{toptesi} provvede già da solo a sostituire questi glifi con altri simili o con disegni, in modo da correggere l'errore; tuttavia questi glifi potrebbero essere presenti nei file importati.
%
\item Alcune figure sia raster (a cui siano state sovrapposte legende testuali con caratteri vettoriali) sia vettoriali (che contengono legende testuali) possono mancare dell'inclusione dei font; per cui quando queste immagini vengono immesse nel file PDF che si vorrebbe conforme con la norma, questa conformità viene a mancare. Se si sono prodotti personalmente tali file da includere, bisogna avere cura di far sì che tutti i glifi dei font usati nel file da includere siano essi stessi inclusi direttamente; esattamente com ho fatto con i loghi \GuIT\ e \MP\ di cui ho detto prima. Purtroppo questo è un caso molto frequente; quindi capita sovente di dover procedere alla correzione di questi file.

Per le figure con legende testuali, basta convertire l'intera figura nel formato \file{.png} (se contiene disegni al tratto e purché non contenga trasparenze) oppure \file{.jpg} se contiene colori sfumati) e il gioco è fatto. 

Ma se si vuole mantenere la natura vettoriale si potrebbe usare il programma gratuito \prog{Inkscape} che è in grado di aprire file vettoriali (PDF, EPS, PS) e di trasformarli in file con le estensioni vettoriali, avendo trasformato nel contempo i caratteri nelle loro outline, cioè nei disegni dei contorni dei glifi, riempiti dello stesso colore che avevano quei glifi nella figura originale. Spesso questa trasformazione va a buon fine. 

Come esperienza personale ritengo che sia opportuno convertire ogni figura vettoriale con i glifi sostituiti dai loro contorni, solo nel formato EPS, che non gestisce le trasparenze, gestite, invece, dal formato PDF. In questo modo i tre programmi che ci interessano, \prog{pdflatex}, \prog{xelatex} e \prog{lualatex}, sono tutti e tre in grado di ricevere figure in formato \file{.eps}, e di incorporarle nel file PDF in uscita. In questo modo si è sicuri di poter evitare le trasparenze, che sono vietate con il formato PDF/A-1b.
\end{enumerate}

Come si vede, dunque, la produzione di un file conforme alla norma PDF/A non è una operazione semplice e senza disporre di \prog{veraPDF} o di Adobe Acrobat Pro XI si consuma molto tempo a ricorrere a postazioni di lavoro esterne, dovendo chiedere favori a laboratori o ad altre persone. Però se si osservano le indicazioni di questo paragrafo, è possibile che il file della tesi sia conforme alla norma fin dal primo momento. Questa stessa guida è stata prodotta con questo tipo di correzioni, ma io sono avvantaggiato perché dispongo personalmente del software necessario alla verifica.



\chapter{I comandi specifici introdotti da \textsf{TOPtesi}}\label{cap:comandispecifici}

\section{Introduzione}

I comandi specifici introdotti da \textsf{TOPtesi} si aggiungono a
tutti quelli definiti da \LaTeX\ e dalla sua classe standard
\class{report}; mentre questi sono tutti in inglese o sono
abbreviazioni inglesi, i comandi introdotti da \textsf{TOPtesi} sono
prevalentemente in italiano o sono abbreviazioni italiane.

Questi comandi sono di diverse categorie; alcuni si possono usare solo
in modo matematico altri solo in modo testo; alcuni solo nel preambolo,
alcuni hanno senso solo durante la composizione del testo; alcuni
servono solo per il frontespizio. Essi saranno descritti nei paragrafi
seguenti.

\section{Le opzioni}
La classe \class{toptesi} accetta diverse opzioni nel
comando di dichiarazione della classe; negli esempi acclusi al pacchetto ne sono
state usate diverse, ma qui forse vale la pena di elencarle tutte
in ordine, senza ripetere le opzioni già definite per
la classe \class{report}.

\begin{description}\def\Item[#1]{\item[\normalfont\opt{#1}]}
\Item[chapterbib] serve per specificare che si desidera la composizione della bibliografia alla fine di ogni capitolo; la bibliografia va composta a mano; se si desidera comporla con \textsc{Bib\TeX} si invochi invece il pacchetto \texttt{chapterbib.sty} con il solito comando \cs{usepackage}. Si abbia cura di leggere attentamente la documentazione di quel pacchetto. Alternativamente si può usare il pacchetto \pack{biblatex} (con la bibliografia da elaborare con il programma esterno \prog{biber}); bisogna specificargli opportune opzioni e si possono comporre direttamente bibliografie distinte per capitoli; si legga con attenzione la documentazione di \pack{biblatex}.
%
\Item[classica] Serve per usare delle denominazioni un po' diverse
dei comandi e per dare una forma diversa al loro contenuto; il
frontespizio ne viene un poco modificato con un look più adatto
alle tesi in discipline classiche. Si veda la tabella~\ref{tab:front5} 
per i comandi disponibili quando questa opzione è attivata.
%
\Item[cucitura] Serve per spostare il blocco del testo verso l'esterno
quando si teme che la piegatura delle pagine verso il centro del
fascicolo rilegato possa impedire la lettura agevole delle parole
vicino al margine interno. Con legature eseguite bene, questa correzione non è necessaria; si ritiene che possa essere utile quando la tesi cartacea viene spillata.
%
\Item[14pt] Aumenta la possibilità di scelta del corpo normale per
la composizione della tesi; può contribuire a migliorare la
lettura quando il testo è fitto di notazioni molto articolate, ma in
generale il corpo così grande è un trucco per rimpolpare una
tesi di modeste dimensioni; ne sconsiglio l'uso per questo secondo
scopo. Avverto che con un corpo così grande è facile che certe
righe risultino troppo lunghe e vadano a capo in punti non
adeguati e/o producano righe non giustificate con gli avvisi
``Overfull hbox''; questo è  un altro motivo per il quale
sconsiglio di usare questa opzione, se non quando è strettamente
necessario. Se la si usa, si verifichi che il titolo, i titoli dei
capitoli, quelli dei paragrafi, eccetera siano composti
correttamente. In particolare il titolo della tesi e i titoli dei
capitoli possono andare a capo solo prima di articoli,
preposizioni semplici o articolate, congiunzioni e brevi avverbi
come ``non''; per evitare di andare a capo dopo queste parti del
titolo si abbia l'accortezza di inserire il segno di legatura
\verb|~| fra queste brevi parole e le parole seguenti; per
esempio, il titolo della tesi fornita come modello va scritto
nella forma \verb|La~pressione barometrica di~Giove|.
%
\Item[autoretitolo] {\leavevmode\tolerance=9999 
Questa opzione funziona solo se viene
specificata anche l'opzione \texttt{classica}; se la si inserisce
senza specificare \texttt{classica}, non succede nulla di male,
semplicemente la classe informa di aver trovato delle opzioni che
né lei né altri pacchetti hanno usato. Serve per comporre la
testatina di sinistra sulle pagine pari con l'indicazione del
candidato e del titolo della tesi. È ovvio che il titolo della
tesi con questa opzione deve essere molto breve, ed è per questo
che è  stato messo a disposizione dello studente l'argomento
facoltativo del comando \cs{titolo} che consente di specificare
un titolo di tre o quattro parole (brevi) ma di senso compiuto,
che possa sostituire il titolo normale, specialmente se questo è
un po' lungo.\par}
%
\Item[oldstyle] Anche questa opzione funziona solamente insieme
all'opzione \texttt{classica}; serve per scrivere i numeri delle
pagine con le cifre minuscole o all'antica, cioè  con segni di altezze e
profondità diverse; si confronti \oldstylenums{1234567890} con
1234567890.
%
\Item[pdfa] Prima della distribuzione di \TeXLive 2016 il pacchetto \pack{pdfx} doveva essere caricato dalla classe, non nel preambolo del documento come si può fare ora con la distribuzione aggiornata di \TeXLive. Per compatibilità con il passato, l'opzione esiste ancora, ma produce solo un avviso e non carica \pack{pdfx}.
%
\Item[noTOPfront] e \opt{usefrontespizio} sono opzioni equivalenti; servono per evitare il caricamento del pacchetto nativo di TOPtesi, \pack{topfront}, in modo da poter comporre il frontespizio con pacchetti esterni, evitando conflitti con quei pacchetti.
\end{description}


\section{Comandi di tipo generale}
I comandi di tipo generale si possono usare in ogni contesto, in
particolare alcuni sono fatti per essere usati sia in modo testo sia in
modo matematico. Essi sono raccolti nella tabella~\ref{tab:generale}.

\begin{table}\def\V{\rule{0pt}{2.5ex}}
\caption{Comandi di tipo generale}\label{tab:generale}
\centering \footnotesize
\begin{tabular}{llp{.26\textwidth}p{.30\textwidth}}
 \toprule
 Comando\V  & Default   & Scopo     & Esempio d'uso  \\[.5ex]
 \midrule
 \cs{interlinea}\Arg{...} &
                    1.0    & Modifica l'argomento di
                             \cs{linespread}\newline \textcolor{red}{NON usare se non
                              costretti con la forza!}\newline Modo testo  
                                 &\cs{interlinea}\Arg{1.05} \newline
                                   oppure \newline
                                   \cs{begin}\Arg{interlinea}\Arg{1.05}\newline
                                   \dots\newline
                                   \cs{end}\Arg{interlinea}\newline
                                  Vedi annotazioni sull'\hskip0pt interlinea nel testo \\
\cs{ohm}         &         & Omega ``diritto''\newline
                             Modi testo e matematico    &  \texttt{45\string\ohm}\\
\cs{ped}\Arg{...}&
                   nessuno & Pedice in tondo\newline
                             Modi testo e matematico   
                             & \texttt{\string$V\string\ped\string{eff\string}\string$} \\
\cs{unit}\Arg{...}&
                   nessuno & Unità di misura in
                             tondo unite al numero\newline
                             Modi testo e matematico    
                             & \texttt{15\string\unit\Arg{k\string\ohm}}  \\
\cs{gei}       &           & Unità immaginaria in
                             tondo \newline
                             Solo modo matematico       
                             & \texttt{\string$\string\eu\string^\string{\string\gei
                             \string\omega\ t\string}\string$} \\
\cs{eu}        &           & Numero ``e'' in tondo\newline Solo modo matematico  
                           & \texttt{\string$\string\eu\string^\Arg{\string\gei
                           \string\omega\ t}\string$} \\
\cs{gradi}     &           & circoletto alzato\newline  Modo testo e
                             matematico  & \texttt{27\string\unit
                             \Arg{\string\gradi\ C}} \\
\cs{listing}\Arg{...}&
                 nessuno    & Listato di un programma in caratteri typewriter     
                 & \texttt{\string\listing\string{toptesi.tex\string}}\\
\cs{blankpagestyle}\Arg{...}&plain	& Impostazione dello stile della pagina eventualmente
                             emessa da \cs{cleardoublepage} quando deve passare ad una 
                             pagina dispari in composizione fronte retro
                                                     	& \cs{blankpagestyle}\Arg{empty}\\
\cs{ifTOPfront}	& true		& Viene impostato a false quando viene specificata 
                              l'opzione \opt{noTOPfront};\ può essere usato per 
                              specificare azioni diverse a seconda del tipo di 
                              frontespizio da comporre&\\	
\cs{goodpagebreak}\Oarg{...}
               & 4          & Inserisce un fine pagina condizionale; l'argomento facoltativo serve per specificare il numero di righe necessarie prima della fine pagina condizionale 
                                                       &\cs{goodpagebreak}\Oarg{5}\\[.5ex]
\bottomrule
\end{tabular}
\end{table}

{\tolerance=6000 Vale la pena di commentare sull'uso dell'ambiente \amb{interlinea} e del comando \cs{interlinea}.
Il primo confina il suo effetto all'interno dell'ambiente da lui
stesso formato; il secondo agisce come una dichiarazione che resta
in vigore finché  una dichiarazione contraria non ne modifichi il
valore.\par}

\textcolor{red}{Tuttavia sia l'ambiente sia il comando non dovrebbero essere mai usati!}
La composizione tipografica non ha nulla a che vedere con la
composizione  dattilografica.
Quest'ultima si faceva con mezzi avanzati per l'epoca, ma oggi quei
mezzi sono del tutto obsoleti; le poche macchine da scrivere
meccaniche o elettromeccaniche che esistono ancora fuori da qualche
museo, vengono usate per riempire formulari o compilare le informazioni
sui documenti cartacei che sopravvivono alla invasione delle carte
plastificate;  ma tolti gli usi burocratici non mi viene in mente nessun
altro uso degno di nota.

La composizione tipografica esige un perfetto equilibrio fra il corpo del font usato e la distanza fra due righe successive, distanza che prende il nome tecnico di \emph{scartamento} o \emph{avanzamento di riga}, ma spesso viene chiamato impropriamente \emph{interlinea}; questo scartamento a seconda del font in uso può essere dal 10\% al 20\% maggiore del corpo del font usato.

Questa documentazione è scritta in corpo 12\,pt e lo scartamento è di 14,5\,pt; si dice che questo testo è composto in corpo 12/14,5.

\begin{interlinea}{1.05}
L'interlinea, come suggerisce il nome, era originariamente lo spazio aggiuntivo da inserire fra una linea e l'altra; quando la composizione tipografica era eseguita con font ricavati da punzoni metallici, l'interlinea era la striscia di metallo che veniva interposta fra una riga di caratteri metallici e la successiva. Il comando \cs{interlinea} e l'ambiente corrispondente hanno pertanto dei nomi che si rifanno alla tipografia tradizionale, ma vengono usati come in dattilografia. In effetti l'argomento del comando e dell'ambiente serve solo come fattore moltiplicativo dello \emph{scartamento}; porre questo fattore al valore \texttt{1.05} vuol dire moltiplicare lo scartamento per 1,05 portandolo quindi al valore di 15,5225\,pt. Questo capoverso è  composto con questo fattore impostato con l'ambiente \amb{interlinea} e, nonostante si tratti solo di un aumento del 5\% dello scartamento, l'occhio lo percepisce in modo più grande di quanto non faccia pensare il suo piccolo valore.
\end{interlinea}

L'avanzamento di default scelto per i caratteri in uso è
l'avanzamento otticamente ottimale; se si desidera usare un font
diverso da quello di default, si potrebbe, per esempio, invocare il
pacchetto \Font{newpxtext} per usare il Palatino esteso come font
di testo. Siccome questo font a pari corpo ha le minuscole
più grandi di quelle dei font di default, potrebbe
essere una idea sensata quello di sperimentare con diversi valori
del fattore di \amb{interlinea}, ma poi si scoprirebbe che
questo fattore differirebbe di pochi centesimi dall'unità e quindi
ci sarebbe da domandarsi se ne valga la pena.

Se proprio si vuole stampare su carta delle bozze scritte abbastanza
larghe per potervi inserire le correzioni e le annotazioni a mano,
allora si imposti il fattore di \amb{interlinea} al massimo a
\texttt{1.5}, ma quando stampate la bella copia, la versione finale,
ricordatevi di re-impostare per \amb{interlinea} il valore unitario
di default.

Tra l'altro non si vuole mica usare l'espediente di un grande fattore
di interlinea solo per rimpolpare una tesi dal volume modesto? Esso
sarebbe un espediente talmente puerile che sarebbe scoperto al primo
sguardo. Ricordate che alcune tesi svolte all'inizio degli anni 20 e non superavano le 30~pagine dattiloscritte o scritte a mano (!) e sulle quali 
si sta studiando ancora oggi dopo più di 80~anni!

Vale la pena di commentare il misterioso comando \cs{blankpagestyle}. Quando si compone fronte e retro senza usare l'opzione di classe \opt{openright}, i capitoli vengono sempre aperti nelle pagine di destra, cioè nelle pagine dispari. Per fare questo il comando \cs{chapter} agisce eseguendo subito il comando \cs{cleardoublepage} che, fra le altre cose, controlla se la nuova pagina su cui scrivere il titolo del capitolo sia dispari. Se non lo fosse provocherebbe la stampa di una pagina ``bianca''; bianca nel senso che non contiene testo ma contiene la testatina e il piedino.

Fino alla versione 0.62 di TOPtesi, questa eventuale pagina bianca prima dell'apertura di un nuovo capitolo conteneva la testatina e il piedino; d'accordo nel piedino c'è solo il numero della pagina, e ci può stare, ma nella testatina rimaneva il titolo del capitolo precedente: tipograficamente molto antiestetico ed errato. Ora di default questa pagina bianca viene composta solo con il piedino che, ricordiamo, contiene solo il numero della pagina. Alcuni preferiscono che questa pagina bianca sia completamente bianca e non contenga nemmeno il piedino; ecco, in questo caso basta specificare:
\begin{flushleft}\ttfamily
\cs{blankpagestyle}\{empty\}
\end{flushleft}
e il problema è risolto. Volendo si potrebbe specificare il nome di qualunque altro stile di pagina già definito ma, ad essere franchi, non vedo alternative fra i due stili \texttt{plain} e \texttt{empty} rispetto agli altri stili che contengono sempre le testatine, eventualmente ridotte al solo loro filetto. 

Se lo si vuole usare, consiglio di farlo subito dopo \cs{begin}\Arg{document}.

Se invece si desidera lasciare il valore di default a \texttt{plain} e usare saltuariamente un altro stile, si usi esplicitamente il comando \cs{cleradoublepage} specificando lo stile desiderato come argomento facoltativo, per esempio:
\begin{verbatim}
\cleardoublepage[empty]
\end{verbatim}

Riassumendo: per una pagina veramente vuota sempre e in ogni caso; si specifichi
\begin{verbatim}
\blankpagestayle{empty}
\end{verbatim}
subito dopo \cs{begin}\Arg{document}. 

Se si desidera una pagina vuota saltuariamente si espliciti \cs{cleardoublepage} con l'argomento facoltativo \texttt{empty}. Se si desidera modificare lo stile della pagina bianca quando il comando \cs{cleardoublepage} è emesso da uno specifico comando \cs{part} o uno specifico comando \cs{chapter} si usi \cs{cleardoublepage}\texttt{[empty]} \emph{subito prima} di quel comando di sezionamento. 

Qualche annotazione va fatta per il comando \cs{goodpagebreak}. Il nucleo di \LaTeX\ contiene il comando \cs{goodbreak} che dovrebbe impostare un consenso di eseguire un fine pagina impostando una penalità molto piccola negativa, più o meno simile a quello che fa il comando \cs{pagebreak}\Oarg{1}. In realtà questo comando nativo di Plain \TeX, conservato in \LaTeX, di fatto non funziona come ci si aspetterebbe. Il comando di questa classe, \cs{goodpagebreak} procede in un altro modo, anche perché si comporta in modo diverso se vene specificato in modo verticale o in modo orizzontale. 

Infatti in modo orizzontale inserisce incondizionatamente un comando \cs{vadjust} il cui argomento vale esattamente \cs{newpage} incondizionato quando la \emph{riga} che lo contiene viene scritta nel file di uscita; si dovrà porre un minimo di attenzione a fare sì che lo spazio alla fine della macro non elimini spazi interparola, né introduca spazi spuri; quando lo si usa, quindi, sarà opportuno lasciare uno spazio prima e dopo la macro, per esempio \begin{flushleft}\verb*|...prima \goodpagebreak di...|\end{flushleft} Vale la pena di usare questo comando in certi casi in cui la bozza mostra un salto pagina prima dell'ultima riga di un capoverso, cioè quando viene prodotta una riga vedova; quando si corregge la bozza si cerca il testo delala riga prima di quella vedova e si inserisce la macro fra due parole qualsiasi di quel testo. È evidente che se si modifica il capoverso e non c'è più bisogno del salto pagina incondizionato, bisogna cancellare la macro da quella posizione.

In modo verticale, invece, viene calcolata la differenza fra le lunghezze \cs{pagegoal} e \cs{pagetotal}; la prima corrisponde più o meno a \cs{textheight} diminuita dello spazio riservato alle note in calce e del loro spazio di separazione dal testo; è insomma l'obbiettivo per raggiungere la giusta altezza del testo; la seconda lunghezza è la lunghezza netta del testo raggiunta fino a quel momento; se questa differenza è \emph{non maggiore} (che per la sintassi nativa di \TeX\ vuol dire \emph{minore o uguale} al numero di righe specificate di default (4 righe) o specificate mediante l'argomento facoltativo, allora viene inserito un salto pagina incondizionato. Vale la pena servirsi di questo comando in modo verticale quando un oggetto relativamente grande e indivisibile viene rinviato alla pagina successiva, così da obbligare \LaTeX\ a stiracchiare lo spazio elastico verticale al fine di giustificare la pagina che precede il salto pagina. 

Questa spiegazione contorta descrive una situazione piuttosto frequente che si manifesta prima dei titoli di sezioni, sottosezioni, e simili. Infatti ogni titolo è separato da un certo spazio dal testo che inizia la sezione e siccome le righe orfane sono tipograficamente molto sgradevoli (meno delle righe vedove, ma comunque sgradevoli), il titolo della sezione deve essere seguito da almeno due righe di testo. Quindi il titolo della sezione, lo spazio verticale di separazione e le due righe di testo formano un blocco di almeno 4~righe; se non ci stanno in fondo alla pagina, \LaTeX\ manda tutto questo blocco alla pagina successiva, stiracchiando gli spazi verticali della pagina corrente. Mettendo la macro \cs{goodpagebreak} prima del titolo di quella sezione che viene rinviata alla pagina successiva, si produce una pagina mozza, ma non troppo, e senza spazi stiracchiati, seguita da una pagina che inizia con il titolo della nuova sezione; ecco come fare.
\begin{verbatim}
... fine dell'ultimo capoverso.

\goodpagebreak

\section{...}
\end{verbatim}

L'algoritmo non è infallibile, ma funziona quasi sempre; può talvolta succedere che il fine pagina produca una pagina quasi vuota e la nuova sezione cominci dopo questa pagina; questo inconveniente è causato dal modo asincrono con cui vengono composte le pagine del file di uscita rispetto alla sequenza di capoversi che creano la bozza non impaginata; in sostanza il funzionamento asincrono fra la composizione del testo, il \emph{paragraph builder} e il \emph{page builder}. La cosa è difficile da spiegare in questa documentazione e la si rimanda alla guida tematica \emph{Il \LaTeX\ Reference Manual commentato} liberamente scaricabile dal sito \url{www.guitex.org}. Ma in questi rari casi la soluzione consiste nello spostare \cs{goodpagebreak} in modo orizzontale fra le parole della penultima riga prima di questo salto pagina mal gestito. 
Il lettore non si scoraggi: questi sono ritocchi destinati a rifinire la tesi quando la si è sostanzialmente completata. In questa documentazione non ho mai usato \cs{goodpagebreak} in modo orizzontale e l'ho usato in modo verticale solo due tre volte.


\section{Comandi per il frontespizio}
\textcolor{red}{Questo paragrafo descrive i comandi per comporre il frontespizio con il pacchetto \pack{top"front}, nativo della classe \class{toptesi}. Se si vuole comporre il frontespizio con un altro pacchetto, non si usino questi comandi ma quelli descritti nella documentazione del pacchetto alternativo. Naturalmente si sarà specificata l'opzione \opt{noTOPfront} fra le opzioni della classe, al fine di evitare conflitti con i pacchetti esterni.}

{\tolerance=3000\textcolor{red}{I comandi per la composizione del frontespizio \emph{non possono} essere inseriti nel preambolo, cioè  prima di \cs{begin}\Arg{document}, qualunque sia il programma di compilazione usato, perché nessuno di quei comandi è definito prima di che il modulo \pack{topfront} sia stato eventualmente caricato, il che avviene solo al momento di eseguire il comando \cs{begin}\Arg{document}; se si specifica l'opzione \opt{noTOPfront} il modulo \pack{topfront} non viene nemmeno caricato.}\par}

È per questo che si consiglia di usare facoltativamente il file di configurazione e comunque tutti i comandi necessari per il frontespizio, non presenti nel file di configurazione, all'interno dell'ambiente \amb{frontespizio} o \amb{frontespizio*}. In questo modo non si corre il rischio di inserirli nel preambolo. Se poi si usa un pacchetto esterno per comporre il frontespizio, a maggior ragione bisogna specificare i comandi necessari come specificato nella documentazione di quel pacchetto esterno.

Con le variazioni di TOPtesi a partire dalla versione 5.85, lo si ripete, la cosa migliore da fare è quella di usare questi comandi dentro uno degli ambienti \amb{frontespizio} o \amb{frontespizio*}; si pigliano due piccioni con una fava; siccome questi ambienti compongono direttamente il frontespizio, essi devono trovarsi dopo \cs{begin}\Arg{document} e questo va bene sia per \pdfLaTeX\ sia per \LuaLaTeX.

I comandi di questa sezione possono essere introdotti in un ordine qualunque, ma è  più chiaro se sono introdotti nell'ordine in cui sono elencati nelle tabelle~\ref{tab:front1}-\ref{tab:front4} riportate dalla pagina~\pageref{tab:front1} alla pagina~\pageref{tab:front4}.

Essi possono anche essere scritti nel file di configurazione
\meta{mainfile}\texttt{.cfg}, anzi, direi che è meglio che siano inseriti
in quel file, lasciando nel preambolo solo quelli che cambiano i
valori di default o che inseriscono informazioni non presenti di
default.

Alcuni comandi sono generali; altri si riferiscono specificatamente
alla monografia di laurea, o alla tesi laurea, o alla dissertazione di
dottorato.

La differenza è che generalmente la monografia di laurea non ha
un relatore; se nella vostra facoltà anche la monografia di laurea
ha un relatore comportatevi come per la tesi di laurea. Se nel
vostro ateneo o nella vostra facoltà succede così, allora inserite
nel file di configurazione il seguente comando
\begin{verbatim}
\TesidiLaurea{Tesi di Laurea}
\end{verbatim}
Per la tesi  del secondo ciclo aggiungete dopo la parola ``laurea'' l'indicazione ``magistrale''  (oppure ``specialistica'', se così è l'uso nel vostro ateneo). 

Similmente cambiate le stringhe inserite dai vari comandi facendo ampio uso
di quelli esposti nelle tabelle~\ref{tab:front1}--\ref{tab:front4}; in
particolare consiglierei di inserirli nel file di configurazione, specialmente se devono 
rappresentare delle modifiche ``permanenti''. \textcolor{red}{Usateli
anche per predisporre quelle stringhe in una lingua diversa
dall'italiano}.

Per comporre questo testo si è usata la classe \class{toptesi},
e si è supposto di scrivere una monografia; tuttavia alcune
delle stringhe di default non sono adeguate a questo uso, quindi il
preambolo di questo documento contiene le seguenti specificazioni
che illustrano l'impiego di diversi fra i comandi descritti nelle
tabelle citate\footnote{In verità ho imbrogliato un poco. I comandi qui elencati producono un frontespizio molto simile a quello effettivamente presente in questo documento, ma non esattamente identico; in fondo questo documento non è una vera monografia di laurea, quindi mi sono permesso qualche lieve ``abbellimento''.}:
\begin{verbatim}\begin{frontespizio}
 \NomeMonografia{Manuale d'uso}
 \monografia{La classe \textsf{TOPtesi}}
 \sottotitolo{Per comporre le tesi al Poli\\
                    e in molte altre università}
 \candidato{Claudio Beccari}
 \sedutadilaurea{Versione \fileversion}
 \ateneo{}% senza il nome che è già contenuto nel logo
 \logosede{logo_blu}
 \end{frontespizio}
\end{verbatim}

\begin{figure}[p]\centering
\includegraphics[height=.9\textheight]{FrontespiziAssemblati1}
\caption{I quattro frontespizi fondamentali}\label{fig:frontespizi}
\end{figure}

Vale la pena di fare le seguenti osservazioni.
\begin {enumerate}
\item Se si usa l'ambiente \amb{frontespizio} o il comando \cs{frontespizio} il logo viene messo in testa; se si usa la forma asteriscatadel comando o  l'ambiente \amb{frontespizio*} il logo viene messo nella metà inferiore della pagina del titolo. Se il logo dell'ateneo contiene per disteso il logotipo, cioè il nome per disteso, allora l'uso del comando \cs{ateneo} contenente il nome generico dell'ateneo implica una ridondanza di informazioni che potrebbe anche essere antiestetica; quindi
se si usa il comando \cs{ateneo} sarebbe meglio usare l'ambiente \amb{frontespizio*}, e viceversa. Ma questa possibilità di scelta per la posizione del logo e la presenza del nome dell'ateneo consente di rispettare le diverse specifiche indicate dalle segreterie didattiche degli atenei; si veda la tabella riassuntiva~\ref{tab:modalitaperfrontespizi} nella pagina~\pageref{tab:modalitaperfrontespizi}.
%
\item \cs{titolo} non deve
essere usato per la monografia (ma \cs{sottotitolo} si
può usare, come si è  fatto per il frontespizio di questo
manuale); per la monografia il comando \cs{monografia}
imposta sia il titolo sia le altre informazioni che distinguono la
monografia da una tesi di laurea magistrale.
%
\item Il comando \cs{titolo} accetta un argomento facoltativo, la versione breve del titolo del frontespizio; serve con l'opzione \opt{classica} e l'opzione \opt{autoretitolo}, dove  il titolo  breve va  nella testatina delle pagine; se il titolo del frontespizio fosse troppo lungo la testatina verrebbe molto male!
%
\item Non dare i comandi \cs{relatore}, \cs{secondorelatore}
e \cs{terzorelatore} né per la dissertazione di dottorato né per la monografia.
%
\item L'indicazione della materia su cui si svolge la tesi di laurea o di dottorato non viene normalmente indicata se non, talvolta, nelle facoltà umanistiche.
%
\item I secondi e terzi candidati non hanno senso né per le tesi di dottorato né per le monografie. Per tutti i candidati il comando per inserirne il nominativo accetta due forme:
\begin{flushleft}\obeylines
\cs{candidato}\marg{Nome Cognome}
oppure
\cs{candidato}\marg{Nome Cognome \textnormal{\cs{IDN}} numero di matricola}
\end{flushleft}
Il \meta{numero di matricola} non è quasi mai necessario, quindi la sequenza \cs{IDN} viene inserita solo se si vuole indicare anche il numero di matricola. La macro \cs{IDN} è predefinita per contenere la stringa \cs{\char92}\cs{quad} matricola:\cs{space}; può venire ridefinita per contenere il nome dell'\emph{Identification number} oppure semplicemente \emph{ID}, e i due punti prima del comando \cs{space} potrebbero essere sostituiti con qualche altro segno o semplicemente eliminati. Ovviamente questo è possibile anche per il comando \cs{candidata}. 
%
\item Per la monografia l'informazione della data può essere omessa se non c'è una data per la presentazione.
%
\item Il pacchetto TOPtesi inizialmente veniva distribuito con un certo numero di loghi di università italiane e straniere. La politica di \TeXLive è quella di distribuire solo materiale con licenza libera e incondizionata; i loghi delle varie università certamente non lo sono, quindi i loghi non vengono più distribuiti con \class{toptesi}. Ogni laureando è quindi tenuto a chiedere il logo alla sua università accettandone tutte le limitazioni d'uso che l'università potrebbe imporgli.
\end{enumerate}

Per quanto riguarda i loghi, la loro inserzione implica l'uso del pacchetto \pack{graphicx}, che \class{toptesi} carica di default; specificare nuovamente quel pacchetto non produce nessun danno, perché \LaTeX\ controlla da solo se lo deve caricare o non lo deve ricaricare. Però, evidentemente, lo si è già detto all'inizio, non è bene caricare una seconda volta i pacchetti che sono già stati caricati

\textcolor{red}{Per i formati dei file grafici da includere, le versioni recenti  di \pdfLaTeX, di \XeLaTeX\ e di \LuaLaTeX\ accettano i formati PDF, PNG, JPG, MPS,  EPS.  La cosa è  scritta a chiare lettere nella guida \file{grfguide.pdf} (già presente nella propria installazione del sistema \TeX\ aggiornata e completa), ma è  un fatto che viene dimenticato troppo spesso e che obbliga a cercare aiuto da chi ne sa di più, compagni o professori; ma che anche loro talvolta dimenticano.}

\textcolor{red}{Esistono programmi per passare da un formato all'altro, ma anche disponendo di quei programmi e conoscendo i vantaggi e gli svantaggi di un formato rispetto ad un altro, questa operazione viene dimenticata troppo spesso, perché  troppo spesso ci si dimentica di queste limitazioni sui formati.}

Per questo io preferisco usare sempre e solamente i compilatori
\pdfLaTeX,  \XeLaTeX\ o \LuaLaTeX. L'unico inconveniente è che non posso usare direttamente il pacchetto \pack{PSTricks}; ma finora me la sono cavata molto
bene anche senza le prestazioni di questo bellissimo pacchetto. D'altra parte esistono pacchetti che consentono di usare \pack{PSTricks} anche con \pdfLaTeX e gli altri programmi; l'operazione diventa un po' più complicata ma se ne occupa direttamente il programma di tipocomposizione usato; l'utente deve solo caricare uno di quei pacchetti e seguirne le indicazioni specificate nella sua documentazione (vedi per esempio il pacchetto \pack{pdftricks} già presente in ogni installazione aggiornata e completa di \TeXLive; documentazione: \texttt{texdoc pdftricks}).


\section{Altri comandi}
La classe \class{toptesi} richiede l'uso di altri comandi il più notevole dei quali è senza dubbio \cs{frontespizio} e la sua variante \cs{frontespizio*} von i rispettivi ambienti; si veda la tabella riassuntiva~\ref{tab:modalitaperfrontespizi} nella pagina~\pageref{tab:modalitaperfrontespizi}). È un comando che non vuole argomenti e che va dato all'inizio, subito dopo l'apertura del documento o all'inizio del primo file incluso dal master file. Si veda comunque la tabella~\ref{tab:comandi}. Ricordo che i comandi \cs{frontespizio} e \cs{frontespizio*} sono definiti dal pacchetto \pack{topfront}; se si vuole comporre il frontespizio con un pacchetto esterno, si è specificata l'opzione \opt{noTOPfront}, che inibisce il caricamento del pacchetto, perciò questi comandi non sono più disponibili.

La differenza fra i due comandi consiste nel fatto che \cs{frontespizio} compone il frontespizio della tesi come richiesto dal Politecnico di Torino, con il logo dell'ateneo in testa alla pagina. Invece la variante asteriscata compone il frontespizio scrivendo in testa il nome dell'ateneo e mettendo il logo nella metà inferiore della gabbia di stampa. Il risultato si vede chiaramente nella figura~\ref{fig:frontespizi}.

Merita sottolineare che il formato richiesto dal Politecnico di Torino va bene nel senso che il logo di quell'ateneo contiene già il suo logotipo\footnote{Il logo del Politecnico di Torino è formato da due parti accostate con un piccolo spazio e un filetto verticale  fra le due; nella parte di sinistra compare il marchio dell'ateneo, il solito simbolo tondo, mentre la parte di destra contiene il logotipo, cioè il nome \textsf{POLITECNICO DI TORINO} su due righe scritto con certi font e con un colore particolare; font e colore non possono venire modificati, pena la violazione d'uso del logo.}. Sarebbe quanto mai inestetico se in testa al frontespizio comparisse due volte il nome del Politecnico; quindi non è strano che l'intestazione della tesi contenga formalmente solo il logo. Lo stesso potrebbe essere fatto, per esempio, con i loghi di diverse università, per esempio quello dell'Università di Torino, di Ca' Foscari a Venezia, di Bologna, dell'Università Cattolica, di Padova, oltre al logo del Politecnico di Milano, giusto per citarne alcuni, perché contengono il marchio e il logotipo.

Per le università che hanno un nome proprio, come per esempio \emph{Tor Vergata} per la seconda università di Roma ma, che io sappia, non hanno il logotipo nel loro logo, sarebbe quanto mai inestetico usare il logo sopra, in testa, poi il nome proprio, poi tutto il resto; io non consiglierei mai di usare \cs{frontespizio} in questi casi, ma userei solo \cs{frontespizio*} o l'ambiente corrispondente.

Nella figura~\ref{fig:frontespizi} nella pagina~\pageref{fig:frontespizi} sono riportati quattro esempi di frontespizi relativi ai tipi di tesi seguenti:
\begin{itemize}[noitemsep]
\item tesi di laurea specialistica o magistrale,
\item dissertazione di dottorato
\item tesi di laurea del Vecchio Ordinamento,
\item monografia di laurea,
\end{itemize}
Con questi modelli lo studente che compila la sua tesi usando \textsf{TOPtesi} può scegliere la versione che fa al caso suo e sa anche che cosa deve configurare per eseguire alcuni cambiamenti.


%\clearpage

\begin{table}[hb]\caption{Comandi ulteriori di \class{toptesi}}\label{tab:comandi}
\def\V{\rule{0pt}{2.5ex}}
\def\d{\discretionary{\%}{\rule{1em}{0pt}}{}}\let\t\ttfamily
\makebox[\textwidth]{\footnotesize%
\begin {tabular}{p{.25\textwidth}p{.25\textwidth}p{.4\textwidth}}
\hline
Comando o ambiente\V           & Default       &  Scopo              \\[.5ex]
\hline
\cs{figurespagetrue}&\cs{figurespagefalse}&\V Fa o non fa stampare l'indice delle
									 figure\\
\cs{tablespagetrue}&\cs{tablespagefalse}&Fa o non fa stampare l'indice delle tabelle\\
\amb{frontespizio}  &              & Contiene i dati per il frontespizio e l'eventuale
									 retrofrontespizio e lo compone 
									con i loghi nella testatina\\
\amb{frontespizio*} &              & Contiene i dati per il frontespizio e l'eventuale
									 retrofrontespizio e lo compone
									 con il loghi nella parte bassa della pagina\\
\cs{frontespizio}  & nessuno       & Fa stampare il frontespizio ed eventualmente il
									 retrofrontespizio; il logo dell'ateneo appare in
									  testa alla pagina \\
\cs{frontespizio*}	& nessuno	   & Fa stampare il frontespizio ed eventualmente il
									 retrofrontespizio; il logo dell'ateneo appare in
									  basso nella pagina; vedi ta tabella 
									  riassuntiva~\ref{tab:modalitaperfrontespizi}\\
\cs{sommario}      &nessuno        & Inizia un capitolo non numerato che ha per
                                    intestazione la parola SOMMARIO (anche in lingua)\\
\cs{ringraziamenti}&nessuno        & Inizia un capitolo non numerato
                                     che ha per intestazione la parola
                                     RINGRAZIAMENTI (anche in lingua)\\
\cs{indici}        &nessuno        & Fa stampare l'indice generale, e,
                                    se sono stati dati i comandi
                                    \texttt{\char92figures\-page\-true} e
                                    \texttt{\char92tables\-page\-true},
                                    anche gli indici delle figure e delle
                                    tabelle.\\
\cs{paginavuota}   &nessuno        & Emette nel file di uscita una pagina
                                     totalmente bianca, senza nemmeno il numero 
                                     della pagina. \\
\cs{NoteWhiteLine} &nessuno        & Anche questo comando richiestomi a gran
                                    voce e serve per mettere in nota una riga 
                                    bianca.\\[.5ex]
\hline
\end{tabular}}
\end{table}


\begin{table}[p]
\def\V{\rule{0pt}{2.5ex}}
\def\d{\discretionary{\%}{\rule{1em}{0pt}}{}}\let\t\ttfamily\let\s\string
\caption{Comandi ulteriori per il frontespizio e per il corpo
della tesi definiti con l'opzione \opt{classica}}\label{tab:front5}
\makebox[\textwidth]{%
\begin {tabular}{p{.22\textwidth}p{.68\textwidth}}
\toprule
Comando\V           &  Scopo              \\[.5ex]
\midrule
\cs{candidato}\V   &Si usa come di solito ma con l'opzione \texttt{classica}
                    produce la scrittura ``Laureando'' invece che ``Candidato''.\\
\cs{candidata}     & Come sopra al femminile. Sia questo comando, sia il
                     precedente accettano che l'argomento sia scritto nella 
                     forma \meta{Nome Cognome{\normalfont\cs{IDN}} matricola} 
                     in alternativa a \meta{Nome Cognome}.\\
\cs{tomo}          & Esegue i frontespizi successivi di una tesi divisa in tomi
                    scrivendovi Tomo primo, Tomo secondo, eccetera, a seconda 
                    del numero progressivo dei volumi in cui è suddivisa la tesi 
                    (massimo quattro). Questo comando viene usato \emph{al posto} 
                    di \cs{frontespizio}\\
\cs{annoaccademico} & Il suo argomento può essere un anno solare o
                    due anni separati da una lineetta.
                    Viene scritto nel frontespizio della tesi o del singolo tomo
                    con la specificazione che si tratta dell'anno accademico e non 
                    della data della presentazione della tesi.\\
\cs{EnDash}        & Produce una lineetta lunga come ---, ma ribassata in modo
                    che stia bene fra numeri in cifre minuscole.\\
\cs{nota}\t\d[...] & Serve per comporre una nota senza ricorrere al contatore 
                    numerico di default. Il simbolo con cui viene richiamata di 
                    default è  l'asterisco, ma si può mettere qualunque segno 
                    matematico senza esplicitare i segni di dollaro, per esempio 
                    si può scrivere {\cs{nota}\t[\s\dagger]\{Questa nota ...\}}\\
\amb{dedica}       & È un ambiente con cui si può stampare una
                    pagina di dedica; generalmente questa pagina viene dopo il
                    frontespizio.\\
\amb{citazioni}    & È un ambiente che consente di scrivere una pagina con frasi 
                    argute. L'arguzia dipende dall'autore; spesso nei libri, 
                    raramente nelle tesi, l'autore cita frasi celebri o che in 
                    qualche modo hanno a che fare con il contenuto del testo.\\
\bottomrule
\end{tabular}}
\end{table}

\clearpage
{\def\V{\rule{0pt}{2.5ex}}
\let\s\string\let\t\ttfamily
\def\d{\discretionary{\%}{\rule{1em}{0pt}}{}}
\tolerance=9999\finalhyphendemerits=0\relax
\setlength{\LTcapwidth}{\dimexpr\textwidth-4em}\tabcolsep=3pt\footnotesize
\begin{longtable}{p{.3\textwidth}>{\RaggedRight}p{.18\textwidth}>{\RaggedRight}p{.2\textwidth}p{.225\textwidth}}
    \caption[Comandi per il frontespizio della monografia di laurea, della
    tesi di laurea e della dissertazione di dottorato]{Comandi per il 
    frontespizio della monografia di laurea, della tesi di laurea e della 
    dissertazione di dottorato}\label{tab:front1}\\
    \toprule
    Comando             & Default  &  Scopo               &   Esempio d'uso   \\
    \midrule
\endfirsthead
    (\emph{continua})\\
    \toprule
    Comando             & Default  &  Scopo               &   Esempio d'uso   \\
    \midrule
\endhead
    \bottomrule
    &&&(\emph{continua})
\endfoot
    \bottomrule
\endlastfoot
\cs{ateneo}\t\s{...\s}\V&POLITECNICO DI TORINO& Definisce il nome generico dell'Ateneo &
                                    \cs{ateneo}\Arg{II Università di Roma} \\
\cs{nomeateneo}\t\s{...\s}& nessuno&   Definisce il nome proprio dell'Ateneo &
                                    \cs{nomeateneo}\Arg{Tor Vergata}           \\
\cs{facolta}\t[...]\s{...\s}& nessuno & Definisce il nome della
                            Facoltà opzionalmente con l'indicazione dell'ordinale&
                                    \cs{facolta}\t[III]\d\Arg{Scienze}         \\
\cs{struttura}				&& sinonimo del comando \cs{facolta}&\\
\cs{corsodilaurea}\t\s{...\s}& nessuno& Definire il nome del corso di laurea &
                                    \cs{corsodilaurea}\d\Arg{Fisica}         \\
\cs{corsodidottorato}\t\d\{...\}& nessuno&Definisce il nome del corso di dottorato&
                                    \cs{corsodidottorato}\d\Arg{Fisica}        \\
\cs{monografia}\t\{...\}&nessuno&Definisce il titolo della monografia e imposta lo stile del frontespizio&
                                    \cs{monografia}\d\Arg{Il~teorema di Eulero}\\
\cs{titolo}\t[...]\{...\}&nessuno&Definisce il titolo  della tesi o della dissertazione&
                                    \cs{titolo}\t\d[La~pressione]\d\Arg{La~pressione di Giove}\\
\cs{sottotitolo}\t\{...\}           &nessuno
                                    &Definisce il sottotitolo della tesi o della
                                    dissertazione
                                    & \cs{sottotitolo}\Arg{Metodo Barometrico} \\
\cs{Materia}\t\{...\}&nessuno&Definisce la materia su cui verte la tesi &
                                    \cs{Materia}\Arg{Remote Sensing}         \\
\cs{materia}\t\{...\}&nessuno&sinonimo del comando \cs{Materia} &
                                    \cs{materia}\d\Arg{Letteratura ostrogota}  \\
\cs{relatore}\t\{...\}\V&nessuno& Definisce il nome del relatore&
                                  \cs{relatore}\d\Arg{prof.\s~Albert Einstein} \\
\cs{secondorelatore}\t\{...\}&nessuno&Se c'è, definisce il nome del secondo relatore&
                                 \cs{secondorelatore}\d\Arg{dott.\s~Grazia Deledda}\\
\cs{terzorelatore}\t\{...\}&nessuno&Se c' è, definisce il nome del
terzo relatore&
                                    \cs{terzorelatore}\d\Arg{ing.\s~Thomas A.\s~Edison}\\
\cs{direttore}\t\{...\}&nessuno&Definisce il nome del direttore del ciclo di dottorato&
                                    \cs{direttore}\d\Arg{prof.\s~Albert Enstein}\\
\cs{coordinatore}\t\{...\}&nessuno&Definisce il nome del coordinatore del ciclo di dottorato&
                                    \cs{coordinatore}\d\Arg{prof.\s~Albert Einstein}\\
\small\cs{QualificaDirettore}\t\d\{...\}& Direttore o Coordinatore& Definisce il titolo da inserire
                                 prima del nome del direttore della scuola di dottorato&
                                 \cs{QualificaDirettore}\d\Arg{PhD~Director}\\
\cs{tutore}\t\{...\}&nessuno&Definisce il nome del tutore&
                                    \cs{tutore}\d\Arg{prof.\s~Karl Von Braun}     \\
\cs{tutoreaziendale}\t\{...\} & nessuno& Definisce la qualifica del tutore aziendale & 
                                     \cs{tutoreaziendale}\d\Arg{Supervisore}\\
\cs{candidato}\t\{...\}&nessuno&Definisce il nome del candidato&
                                    \cs{candidato}\Arg{Galileo Galilei}          \\
\cs{candidata}\t\{...\}&nessuno&Definisce il nome della candidata&
                                    \cs{candidata}\Arg{Maria Curie}              \\
\cs{secondocandidato}\t\d\{...\}&nessuno&Se c'è , definisce il nome
del secondo candidato&
                                    \cs{secondocandidato}\d\Arg{Evangelista Torricelli}\\
\cs{secondacandidata}\t\d\{...\}&nessuno&Se c'è , definisce il nome
della seconda candidata&
                               \cs{secondacandidata}\d\Arg{Rita Levi Montalcini}\\
\cs{terzocandidato}\t\{...\} &nessuno & Se c'è, definisce il nome del
                                        terzo candidato
                             & \cs{terzocandidato}\t\d\Arg{Alessandro~Volta} \\
\cs{terzacandidata}\t\{...\}&nessuno&Se c'è, definisce il nome della terza candidata&
                                    \cs{terzacandidata}\t\d\Arg{Eleonora~Duse}\\\cs{sedutadilaurea}\t\{...\}\V&data corrente &Definisce il mese e l'anno (volendo il giorno) della seduta di laurea&
                                    \cs{sedutadilaurea}\t\d\Arg{Dicembre~2025}\\
\cs{esamedidottorato}\t\d\{...\}&data corrente&Definisce il mese e l'anno (volendo il giorno) della seduta di discussione&
                                    \cs{esamedidottorato}\t\d\Arg{Febbraio~2013}\\
\small\cs{scuoladidottorato}\t\d\{...\}& SCUOLA DI DOTTORATO&Definisce il nome ufficiale
                                     della scuola di dottorato & 
								     \cs{scuoladidottorato}\t\d\Arg{SCUDO}\\
\cs{ciclodidottorato}\t\d\{...\}&nessuno&Definisce il numero ordinale del ciclo di dottorato&
                                   \cs{ciclodidottorato}\t\d\Arg{XV~ciclo}      \\
\cs{logosede}\t[...]\{...\}&nessuno&\raggedright Inserisce nel frontespizio il logo
									 dell'ateneo, specificandone opzionalmente l'altezza 
									 (default: 3\,cm)& 
                                    \cs{logosede}\t\d[2cm]\Arg{logouno,\d logodue}\\
\cs{setlogodistance}\t\d\{...\}& 3\,em	& imposta la distanza fra i loghi&
                                    \cs{setlogodistance}\t\d\Arg{25mm}\\
\midrule\midrule
\cs{retrofrontespizio}\t\d\{...\}& nessuno& \multicolumn2{p{.475\textwidth}}{Se specificato, serve 
                                solitamente per scrivere nel verso
                                della pagina del frontespizio alcune dichiarazioni 
                                di carattere legale; accetta un argomento composto 
                                anche da diversi capoversi}\\
\end{longtable}}


\begin{table}[p]\tabcolsep=3pt
\def\V{\rule{0pt}{2.5ex}}
\def\d{\discretionary{\%}{\rule{1em}{0pt}}{}}
\def\serve{definisce una stringa equivalente a }
 \caption{Comandi per modificare le parole e/o le brevi frasi scritte
nel frontespizio o per cambiare le parole italiane di default in altre
parole diverse, o per esprimere gli stessi concetti in un'altra
lingua}\label{tab:front4}
\makebox[\textwidth]{\small%
\begin{tabular}{@{}p{.35\textwidth}p{.63\textwidth}@{}}
\toprule
\cs{FacoltaDi}\t\{...\}\V        & definisce una stringa con il nome generico della
                                struttura didattica, per esempio ``Scuola di~'', 
                                prima del nome della struttura; di default è 
                                impostata una stringa nulla, cosicché non viene 
                                scritto nulla nel frontespizio in merito alla struttura
                                didattica competente\\
\cs{StrutturaDidattica}\t\{...\}	 & sinonimo del comando  \cs{FacoltaDi}\\
\cs{DottoratoIn}\t\{...\}        & \serve ``Dottorato in'' prima del nome del dottorato\\
\cs{CorsoDiLaureaIn}\t\{...\}    & \serve ``Corso di Laurea in'' prima del nome del corso di laurea\\
\cs{TesiDiLaurea}\t\{...\}       & \serve ``Tesi di Laurea''\\
\cs{NomeMonografia}\t\{...\}     & \serve ``Monografia di Laurea''\\
\cs{NomeDissertazione}\t\{...\}  & \serve ``Dissertazione di Dottorato''\\
\cs{InName}\t\{...\}             & \serve ``in''; in tedesco potrebbe essere ``auf'', 
                                  in francese ``en'', ecc.\\
\cs{CandidateName}\t\{...\}      & \serve ``Candidato''; per questo e i due comandi
                                successivi il programma riesce
                                a scegliere la stringa giusta adattata in numero genere;
                                nel cambiare queste stringhe
                                il compositore ha una sola possibilità e deve scegliere
                                direttamente il genere e il numero.\\
\cs{AdvisorName}\t\{...\}        & \serve ``Relatore''\\
\cs{CoAdvisorName}\t\{...\}      & \serve ``Correlatore''; questo comando si può usare
                                sempre, ma il suo contenuto viene effettivamente usato 
                                solo se si specifica l'opzione \texttt{classica}; se in 
                                italiano non piace ``Correlatore'' ma si preferisce 
                                ``Corelatore'' o ``Co-relatore'', sempre con l'opzione 
                                \texttt{classica}, si può correggere la versione di 
                                default.\\
\cs{TutorName}\t\{...\}          & \serve ``Tutore''\\
\cs{NomeTutoreAziendale}\t\{...\}& \serve ``Supervisore aziendale''; usando un argomento
                                 che contenga anche
                                indicazioni di ``a capo'', nella seconda riga si può
                                scrivere il nome dell'azienda.\\
\cs{CycleName}\t\{...\}          & \serve ``ciclo''; serve essenzialmente per indicare il 
                                ciclo di dottorato\\
\cs{NomePrimoTomo}\t\{...\}      & \serve ``Tomo primo''\\
\cs{NomeSecondoTomo}\t\{...\}    & \serve ``Tomo secondo''\\
\cs{NomeTerzoTomo}\t\{...\}      & \serve ``Tomo terzo''\\
\cs{NomeQuartoTomo}\t\{...\}     & \serve ``Tomo quarto''; tutte e quattro queste
                                stringhe 
                                dipendono dalla lingua usata e dall'ordine che si vuole o 
                                si deve dare alle due parole.\\
\bottomrule
\end{tabular}}
\end{table}



\chapter*{Conclusioni}
\addcontentsline{toc}{chapter}{Conclusioni}

Il pacchetto \textsf{TOPtesi} fa quasi tutto quello che è necessario per comporre uno qualunque di quegli scritti che vengono chiamati \emph{monografia di laurea}, \emph{elaborato finale di laurea},  \emph{tesi di laurea}, \emph{tesi di laurea specialistica}, \emph{tesi di laurea magistrale}, \emph{dissertazione di dottorato}.

Non fa tutto, ci mancherebbe altro, visto che nessuna persona è in
grado di prevedere tutte le necessità delle altre persone.

Da un lato alcuni devono adattarsi a quello che è  disponibile,
magari dandosi da fare per creare qualche cosa di nuovo e utile per se
stessi e forse anche per gli altri. Qualcun altro deve sapersi
immedesimare nelle necessità altrui e magari deve darsi da fare
per creare qualcosa di nuovo e utile agli altri e forse anche a se stesso.

Una cosa certamente che rimane da fare è un procedimento più
semplice per la composizione del frontespizio in una lingua
diversa dall'italiano. Attualmente il procedimento si rifà all'uso
delle macro che servono per definire le stringhe in lingua; il
sistema dovrebbe essere completo ma richiede un po' di attenzione
da parte dell'utente.

\iffalse % INIZIO METACOMMENTO
\begin {sloppypar}
Un'altra possibilità sarebbe quella di predisporre altri file con
i nomi, per esempio, \texttt{topfrnten.sty},
\texttt{topfrntfr.sty}, \texttt{topfrntsp.sty}, eccetera, che
siano cloni del file \texttt{topfront.sty} e che si possono
richiamare come pacchetti aggiuntivi. Ognuno definisce il comando
\texttt{\string\frontespizioinglese},
\texttt{\string\frontespiziofrancese},
\texttt{\string\frontespiziospagnolo}, eccetera, e questo comando
può quindi essere usato in sostituzione del comando per il
frontespizio italiano. Questi file differirebbero da quello per il
frontespizio in italiano perché  vi sarebbero ridefinite tutte le
definizioni di default delle stringhe in lingua. Ovviamente
bisogna che chi clona e modifica il file \texttt{topfront.sty} e
lo modifica per adattarlo ad una determinata lingua, deve
conoscere quella lingua da maestro. Suggerirei però di non
modificare le strutture con cui è composto il frontespizio,
perché  effettivamente contiene tutti gli
elementi richiesti ``per legge'' ed è tipograficamente abbastanza
equilibrato; il fatto che blocco dei relatori e di quello dei
candidati siano sfalsati lascia abbastanza spazio per le firme di legge, anche se
introduce un elemento di asimmetria; tuttavia~\dots
\end{sloppypar}

\fi % FINE METAMMENTO

Il pacchetto \textsf{TOPtesi} carica di default sia \texttt{babel} o 
\texttt{polyglossia} sia \texttt{graphicx}; certo potrebbe caricare di 
default anche \texttt{amsmath} e i suoi compagni, ma non tutti hanno 
bisogno di scrivere e comporre matematica avanzata e se non vengono 
usati essi comportano uno spreco di memoria e un sia pur piccolo 
rallentamento nell'esecuzione. Ritengo quindi che sia meglio che ognuno 
si carichi i pacchetti che intende davvero usare.

È possibile usare il semplice \texttt{latex}; ma questo 
serve solo per produrre un file in formato DVI, che può servire solo per 
una stampa della tesi su carta; può venire trasformato nel formato PS 
tramite l'applicativo \texttt{dvips}, ma questo formato non stamperebbe 
meglio di quel che si potrebbe ottenere dai ``previewer'' recenti per 
il formato DVI che siano anche in grado di inviare i file DVI in stampa. 
Viceversa è più interessante disporre della versione finale in formato PDF, 
cosa che si può ottenere mediante il programma \texttt{ps2pdf} il quale, 
con opportuni artifici, permette anche di produrre un file in formato PDF/A. 
Ma allora perché non ottenere questo risultato in un passaggio solo? Certo, 
potrebbe esserci l'esigenza di usare estensioni che fanno uso del linguaggio 
PostScript, ma, solitamente, agendo direttamente con \texttt{pdflatex} si 
possono usare altri pacchetti di estensione che consentono di evitare del 
tutto il linguaggio PostScript. 

Non consiglierei, invece, l'uso generalizzato 
di \XeLaTeX, ma lo consiglierei solo per quelle tesi, specialmente di 
carattere umanistico, che hanno bisogno di maneggiare agevolmente font di 
diversi tipi per scrivere con alfabeti o sillabari o sistemi di ideogrammi 
per i quali \pdfLaTeX\ è, sì, attrezzato, ma rende molto più faticosa 
l'intera operazione. Molto interessante è anche il programma di 
composizione \LuaLaTeX, che incorpora una buona parte del linguaggio di 
scripting Lua; questo a sua volta permette di compiere diverse azioni 
esternamente al programma di tipocomposizione, e in certi casi esso si 
può considerare una estensione molto utile del programma \pdfLaTeX.  
Questo programma può usare la microtipografia come \pdfLaTeX\ e i font 
OpenType come \XeLaTeX, quindi unisce i vantaggio dell'uno e dell'altro; consente anche di usare il linguaggio Lua, che consente di eseguire operazioni difficili o impossibili da fare con gli altri due; certo non è obbligatorio usare Lua, se si usa \LuaLaTeX, ma volendo lo si può fare. 

So per certo che le tesi di filologia classica possono essere
composte molto bene con \pdfLaTeX\ e \textsf{TOPtesi}; se bisogna
usare una buona dose di lingua greca classica  è conveniente
caricare il pacchetto \pack{teubner}\footnote{A tutt'oggi (2016) il 
pacchetto \pack{teubner} funziona correttamente solo con \pdfLaTeX.}; 
gli intenditori sanno
perfettamente che i font della casa editrice Teubner di Lipsia
sono fra i più gradevoli che esistano; il pacchetto
\texttt{teubner} fa uso di un'ottima imitazione di quei font e
mette a disposizione una miriade di comandi per comporre quei
segni ``strani'' che i filologi usano per scrivere le loro opere.

Per le edizioni critiche esistono i pacchetti \texttt{eledmac} e \texttt{reledmac}; io non li ho mai usati, ma sembra che siano di 
grande aiuto per gli umanisti.

Per gli scienziati e i tecnologi esistono troppi pacchetti
specializzati e sarebbe impossibile, e forse inutile, elencarli
tutti o anche elencare solo i più importanti.

Non resta che augurare:
\begin {center}\Large\bfseries
Buona composizione con \pdfLaTeX\ o \XeLaTeX\ o \LuaLaTeX!
\end{center}
\appendix
\input{LPPL}

\printindex

 \end{document}


