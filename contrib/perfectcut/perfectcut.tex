%%% perfectcut.sty  documentation
%%%%%%%%%%%%%%%%%%%%%%%%%%%%%%%%%%
%%%
%%% Author: Guillaume Munch-Maccagnoni
%%% https://github.com/gadmm/perfectcut
%%%
%%% This work may be distributed and/or modified under the conditions of
%%% the LaTeX Project Public License, either version 1.3 of this license
%%% or (at your option) any later version. Refer to the README file.
%%%
\documentclass[12pt,a4paper,british]{scrartcl}
\usepackage[T1]{fontenc}
\usepackage[latin9]{inputenc}
\setcounter{secnumdepth}{1}
\setcounter{tocdepth}{1}
\usepackage{babel}
\usepackage{array}
\usepackage{booktabs}
\usepackage{calc}
\usepackage{amsmath}
\usepackage[unicode=true]{hyperref}
\usepackage{perfectcut}
\usepackage{stix}
\renewcommand*\ttdefault{txtt}
\usepackage[oldstyle,lining,scale=0.97]{sourcesanspro}
\usepackage[protrusion=true,expansion=true,tracking=false,kerning=true,spacing=true]{microtype}

\begin{document}
\global\long\def\cut#1#2{\perfectcut{#1}{#2}}

\global\long\def\mt{\bar{\mu}}

\title{\texttt{perfectcut.sty} documentation}

\author{\noindent {\large{}Guillaume Munch-Maccagnoni}\thanks{\protect\href{https://github.com/gadmm/perfectcut}{https://github.com/gadmm/perfectcut}}}

\date{23rd April 2017}

\maketitle

\tableofcontents{}

\section{Use}

\paragraph{Original use}

This package supplies the following commands:

\begin{center}
\begin{tabular}{ll}
\toprule 
Command & Produces\tabularnewline
\midrule
\texttt{\textbackslash perfectcut\{\#1\}\{\#2\}} & $\perfectcut{\#1}{\#2}$\tabularnewline
\texttt{\textbackslash perfectbra\{\#1\}} & $\perfectbra{\#1}$\tabularnewline
\texttt{\textbackslash perfectket\{\#1\}} & $\perfectket{\#1}$\tabularnewline
\bottomrule
\end{tabular}\quad{}%
\begin{tabular}{ll}
\toprule 
Command & Produces\tabularnewline
\midrule
\texttt{\textbackslash perfectcase\{\#1\}} & $\perfectcase{\#1}{\#2}$\tabularnewline
\texttt{\textbackslash perfectbrackets\{\#1\}} & $\perfectbrackets{\#1}$\tabularnewline
\texttt{\textbackslash perfectparens\{\#1\}} & $\perfectparens{\#1}$\tabularnewline
\bottomrule
\end{tabular}
\par\end{center}

The effect of these commands is to let the delimiters grow according
to the number of nested \texttt{\textbackslash perfect}\texttt{\emph{command}}s
(regardless of the size of contents). The package is originally intended
for solving a notational issue regarding the representation of abstract-machine-like
calculi in proof theory and computer science.

\paragraph{General use}

The package also defines \texttt{\textbackslash perfectunary} and
\texttt{\textbackslash perfectbinary} for defining custom delimiters
that behave similarly to the above ones. These commands should be
considered when facing the need of nested delimiters that consistently
grow regardless of the contents. (See ``Advanced Use''.)

\section{Example}


\subsection{Using \texttt{perfectcut.sty}}

\global\long\def\cut#1#2{\cutprimitive{#1}{#2}}

\noindent \texttt{\footnotesize{}\textbackslash usepackage\{perfectcut\}}~\\
\texttt{\footnotesize{}\textbackslash let\textbackslash cut\textbackslash perfectcut}

\begin{center}
\fbox{\begin{minipage}[t]{0.8\columnwidth}%
The following states the idempotency of an adjunction: 
\[
\cut t{\mt x.\cut{\mu\alpha.\cut ue}{e'}}=\cut{\mu\alpha.\cut t{\mt x.\cut ue}}{e'}
\]

The following states the commutativity of a strong monad:
\[
\cut t{\mt x.\cut u{\mt y.\cut ve}}=\cut u{\mt y.\cut t{\mt x.\cut ve}}
\]
Using \texttt{\textbackslash underline} to mark redexes:
\begin{align*}
 & \delta(V,x.y,x.y)\\
 & =\mu{\star}.\cut V{\perfectcase{\mt x.\underline{\cut y{\star}}}{\mt x.\underline{\cut y{\star}}}}\\
 & =\mu{\star}.\cut V{\underline{\perfectcase{\mt x.\cut{\iota_{1}(x)}{\mt z.\cut y{\star}}}{\mt x.\cut{\iota_{2}(x)}{\mt z.\cut y{\star}}}}}\\
 & =\mu{\star}.\cut V{\underline{\mt z.}\cut y{\star}}\\
 & =\mu{\star}.\cut y{\star}=y
\end{align*}
 %
\end{minipage}}
\par\end{center}

\subsection{Using \texttt{\textbackslash left}, \texttt{\textbackslash middle}
and \texttt{\textbackslash right} instead}

\global\long\def\fakecut#1#2{\left\langle #1\middle|\mkern-2mu \middle|#2\right\rangle }

\noindent \texttt{\footnotesize{}\textbackslash renewcommand\{\textbackslash cut\}{[}2{]}\{\textbackslash left\textbackslash langle
\#1\textbackslash middle|\textbackslash mkern-2mu\textbackslash middle|\#2\textbackslash right\textbackslash rangle\}}

\begin{center}
\fbox{\parbox[t]{0.8\columnwidth}{%
The following states the idempotency of an adjunction: 
\[
\fakecut t{\mt x.\fakecut{\mu\alpha.\fakecut ue}{e'}}=\fakecut{\mu\alpha.\fakecut t{\mt x.\fakecut ue}}{e'}
\]

The following states the commutativity of a strong monad:
\[
\fakecut t{\mt x.\fakecut u{\mt y.\fakecut ve}}=\fakecut u{\mt y.\fakecut t{\mt x.\fakecut ve}}
\]

Using \texttt{\textbackslash underline} to mark redexes:
\begin{align*}
 & \delta(V,x.y,x.y)\\
 & =\mu{\star}.\fakecut V{\left[\mt x.\underline{\fakecut y{\star}}\middle|\mt x.\underline{\fakecut y{\star}}\right]}\\
 & =\mu{\star}.\fakecut V{\underline{\left[\mt x.\fakecut{\iota_{1}(x)}{\mt z.\fakecut y{\star}}\middle|\mt x.\fakecut{\iota_{2}(x)}{\mt z.\fakecut y{\star}}\right]}}\\
 & =\mu{\star}.\fakecut V{\underline{\mt z.}\fakecut y{\star}}\\
 & =\mu{\star}.\fakecut y{\star}=y
\end{align*}
%
}}
\par\end{center}

As we can see, the legibility of the above rendering is hampered by
multiple issues: the delimiters grow inconsistently, vertical bars
have the wrong size, accents or underlines uselessly make the delimiters
grow, and the spacing could be improved. The package is designed to
fix these issues.

\section{Advanced use}

The package lets you define your own growing delimiters. Let us first
stress that the size of these delimiters is entirely determined by
the number of nestings and is insensitive to the size of the contents.
If you need the size of the contents to be taken into account then
it is probably sufficient to use \texttt{\textbackslash left} and
\texttt{\textbackslash right} while tweaking \texttt{\textbackslash delimitershortfall}
and \texttt{\textbackslash delimiterfactor}.

\subsection{Example}

The following displays a set $\{\#1\mid\#2\}$ with delimiters appropriately
sized if there are other \texttt{\textbackslash perfectcommands}
inside \texttt{\#1} and \texttt{\#2}.\medskip{}

\texttt{\footnotesize{}\textbackslash def\textbackslash Set\#1\#2\{\textbackslash perfectbinary\{IncreaseHeight\}\textbackslash\{|\textbackslash\}\{\#1\textbackslash mathrel\{\}\}\{\textbackslash mathrel\{\}\#2\}\}}{\footnotesize \par}

\texttt{\footnotesize{}\textbackslash{[}\textbackslash Set\{\textbackslash perfectparens\{a\}\}\{\textbackslash perfectparens\{b\}\}\textbackslash{]}}{\footnotesize \par}

\def\Set#1#2{\perfectbinary{IncreaseHeight}\{|\}{#1\mathrel{}}{\mathrel{}#2}}

\[
\fbox{\ensuremath{\Set{\perfectparens{a}}{\perfectparens{b}}}}
\]

\subsection{Custom delimiters}
\begin{description}
\item [{\texttt{\textbackslash perfectunary\#1\#2\#3\#4}}] Displays \texttt{\#2}
\texttt{\#4} \texttt{\#3} where \texttt{\#2} and \texttt{\#3} are
delimiters. The delimiters grow according to \texttt{\#1} which must
be one of \texttt{IncreaseHeight}, \texttt{CurrentHeight}, or \texttt{CurrentHeightPlusOne}.
\item [{\texttt{\textbackslash perfectbinary\#1\#2\#3\#4\#5\#6}}] Displays
\texttt{\#2} \texttt{\#5} \texttt{\#3} \texttt{\#6} \texttt{\#4} where
\texttt{\#2}, \texttt{\#3} and \texttt{\#4} are delimiters. The delimiters
grow according to \texttt{\#1} which must be one of \texttt{IncreaseHeight},
\texttt{CurrentHeight}, or \texttt{CurrentHeightPlusOne}.
\end{description}

\subsection{Stock delimiters}

The stock commands behave as follow:

\begin{center}
\begin{tabular}{>{\raggedright}p{11em}lll}
\toprule 
Command & Produces & Growth & Inserts skips\tabularnewline
\midrule
\texttt{\textbackslash perfectcut\{\#1\}\{\#2\}} & $\perfectcut{\#1}{\#2}$ & \texttt{IncreaseHeight} & Yes\tabularnewline
\texttt{\textbackslash perfectbra\{\#1\}} & $\perfectbra{\#1}$ & \texttt{IncreaseHeight} & Yes\tabularnewline
\texttt{\textbackslash perfectket\{\#1\}} & $\perfectket{\#1}$ & \texttt{IncreaseHeight} & Yes\tabularnewline
\texttt{\textbackslash perfectcase\{\#1\}} & $\perfectcase{\#1}{\#2}$ & \texttt{CurrentHeightPlusOne} & Yes\tabularnewline
\texttt{\textbackslash perfectbrackets\{\#1\}} & $\perfectbrackets{\#1}$ & \texttt{CurrentHeightPlusOne} & Only inside\tabularnewline
\texttt{\textbackslash perfectparens\{\#1\}} & $\perfectparens{\#1}$ & \texttt{CurrentHeight} & Only inside\tabularnewline
\texttt{\textbackslash perfectunary\{\#1\}}~\\
\texttt{~\{\#2\}\{\#3\}\{\#4\}} & $\#2\,\#4\,\#3$ & \#1 & No\tabularnewline
\texttt{\textbackslash perfectbinary\{\#1\}}~\\
\texttt{~\{\#2\}\{\#3\}\{\#4\}\{\#5\}\{\#6\}} & $\#2\,\#5\,\#3\,\#6\,\#4$ & \#1 & No\tabularnewline
\bottomrule
\end{tabular}
\par\end{center}

\section{Options}

\subsection{Option \texttt{mathstyle}}

With this option, the command \texttt{\textbackslash currentmathstyle}
from the package \texttt{mathstyle} is used instead of the command
\texttt{\textbackslash ThisStyle} from the \texttt{scalerel} package.
The latter (default) uses \texttt{\textbackslash mathchoice} which
has exponential time complexity in the number of nestings. The former
does not use \texttt{\textbackslash mathchoice} but redefines many
primitives and is therefore a source of incompatibilities.

\subsection{Option \texttt{realVert}}

With the option \texttt{realVert}, the double bars are obtained with
the \texttt{\textbackslash Vert} command. But without it, two \texttt{\textbackslash vert}
symbols are used and their spacing is controlled with  \texttt{\textbackslash cutinterbarskip}.
Also, if \texttt{realVert} is not activated, then a penalty (\texttt{binoppenalty})
is added, such that $\cut{\mu\alpha.\cut ab}{\mt x.\cut cd}$ splits
across lines.

\subsection{Option \texttt{fixxits}}

For some reason that the author was unable to identify, the vertical
alignment is wrong with the Opentype XITS math font with XeTeX. The
option \texttt{fixxits} fixes this behaviour.

\subsection{Customisation}

The following mu-skips can be redefined in your preamble:

\begin{center}
\begin{tabular}{ll}
\toprule 
Command & Defines the spacing...\tabularnewline
\midrule
\texttt{\textbackslash cutbarskip=5.0mu plus 8mu minus 2.0mu} & around bars\tabularnewline
\texttt{\textbackslash cutangleskip=0.0mu plus 8mu minus 1.0mu} & around delimiters (inside)\tabularnewline
\texttt{\textbackslash cutangleouterskip=0.0mu plus 8mu minus 0mu} & around delimiters (outside)\tabularnewline
\texttt{\textbackslash cutinterbarskip=0.8mu plus 0mu minus 0mu} & between bars (excl. \texttt{realVert})\tabularnewline
\bottomrule
\end{tabular}
\par\end{center}

\noindent (1 mu equals $1/18$ of an em in the current math font.)

\section{Reimplementation of fixed-size delimiters}

In addition, I provide the following corrections and generalisations
of  \texttt{\textbackslash big},\texttt{\textbackslash bigg}, etc.
Why not using the latter? Because both the plain \TeX{} and the \texttt{amsmath}
versions can be incorrect when changing the math font, the font size,
the math style or \texttt{\textbackslash delimitershortfall}. Moreover,
Opentype math fonts sometimes offer more than five sizes. For this
package we need a robust solution.

\begin{center}
\begin{tabular}{lll}
\toprule 
Command & Example & \tabularnewline
\midrule 
\multicolumn{3}{l}{\emph{\#1-th size of delimiter \#2}}\tabularnewline
\texttt{\textbackslash nthleft\{\#1\}\{\#2\} } & \texttt{\textbackslash nthleft\{2\}(} & $\nthleft{2}($\tabularnewline\addlinespace[0.1em]
\texttt{\textbackslash nthmiddle\{\#1\}\{\#2\}} & \texttt{\textbackslash nthmiddle\{2\}\textbackslash Vert} & $\nthmiddle{2}\Vert$\tabularnewline\addlinespace[0.1em]
\texttt{\textbackslash nthright\{\#1\}\{\#2\}} & \texttt{\textbackslash nthright\{2\})} & $\nthright{2})$\tabularnewline\addlinespace[0.1em]
\multicolumn{1}{l}{\emph{delimiter \#2 of height at least \#1}} &  & \tabularnewline\addlinespace[0.1em]
\texttt{\textbackslash lenleft\{\#1\}\{\#2\}}  & \texttt{\textbackslash lenleft\{3.2mm\}{[}} & $\lenleft{3mm}[$\tabularnewline\addlinespace[0.1em]
\texttt{\textbackslash lenmiddle\{\#1\}\{\#2\}} & \texttt{\textbackslash lenmiddle\{3.2mm\}|} & $\lenmiddle{3mm}|$\tabularnewline\addlinespace[0.1em]
\texttt{\textbackslash lenright\{\#1\}\{\#2\}} & \texttt{\textbackslash lenright\{3.2mm\}{]}} & $\lenright{3mm}]$\tabularnewline\addlinespace[0.1em]
\multicolumn{3}{l}{\emph{delimiter \#2 of height exactly \#1 obtained by scaling the
above one}}\tabularnewline\addlinespace[0.1em]
\texttt{\textbackslash reallenleft\{\#1\}\{\#2\}}  & \texttt{\textbackslash reallenleft\{3.2mm\}{[}} & $\reallenleft{3mm}[$\tabularnewline\addlinespace[0.1em]
\texttt{\textbackslash reallenmiddle\{\#1\}\{\#2\}} & \texttt{\textbackslash reallenmiddle\{3.2mm\}|} & $\reallenmiddle{3mm}|$\tabularnewline\addlinespace[0.1em]
\texttt{\textbackslash reallenright\{\#1\}\{\#2\}} & \texttt{\textbackslash reallenright\{3.2mm\}{]}} & $\reallenright{3mm}]$\tabularnewline
\bottomrule
\end{tabular}  
\par\end{center}

\subsection{Example with \texttt{\textbackslash nthleft}}

\texttt{\footnotesize{}\textbackslash nrthleft0(\textbackslash nthleft1(\textbackslash nthleft2(\textbackslash nthleft3(\textbackslash nthleft4(\textbackslash nthleft5(\textbackslash nthleft6(}{\footnotesize \par}

\[
\nthleft0(\nthleft1(\nthleft2(\nthleft3(\nthleft4(\nthleft5(\nthleft6(
\]

\subsection{Example with \texttt{\textbackslash big},\texttt{\textbackslash Big},\texttt{\textbackslash bigg},\texttt{\textbackslash Bigg}}

\texttt{\footnotesize{}(\textbackslash big(\textbackslash Big(\textbackslash bigg(\textbackslash Bigg(}{\footnotesize \par}

\[
(\big(\Big(\bigg(\Bigg(
\]
The above uses the \texttt{\textbackslash big} commands from the
\texttt{amsmath} package. The \texttt{amsmath} package corrects issues
with the original \TeX{} commands, but I could still notice inconsistencies,
such as \texttt{\textbackslash big} starting at size 2, under some
font combinations. \texttt{\textbackslash nthleft}, \texttt{\textbackslash nthright}
and \texttt{\textbackslash nthmiddle} are implemented in a more robust
way.

\section{License}

This work may be distributed and/or modified under the conditions
of the \LaTeX{} Project Public License, either version 1.3 of this
license or (at your option) any later version. Refer to the \texttt{README}
file.
\end{document}
