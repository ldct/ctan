\documentclass{examdesign}
\Fullpages
\NumberOfVersions{3}

\begin{document}

\begin{examtop}
  \noindent Name:\rule{4in}{.4pt}
  \begin{center}
    \textbf{Philosophy 29---Critical Reasoning} \\
    \textbf{Midterm Exam \Alph{version}} \\
    \textbf{November 5, 1997}
  \end{center}
\end{examtop}

\begin{truefalse}[title={True/False (2 pts. each)},suppressprefix]
Print ``T'' if the statement is true, otherwise print ``F''
\begin{question}
  \answer{True} For purposes of evaluating the credibility of a source, a
  statement is less credible if the statement maker knows that his or her
  reputation is at risk.
\end{question}

\begin{question}
  \answer{True} A deductively valid argument can have true premises and a true
  conclusion.
\end{question}

\begin{question}
  \answer{True} A proposition and its double negation are logically equivalent.
\end{question}

\begin{question}
  \answer{False} If an argument is bad, the conclusion must be wrong.
\end{question}

\begin{question}
  \answer{True} Observation reports by the observer are generally to be
  preferred over others' reports of these observation reports.
\end{question}

\begin{question}
  \answer{True} A deductively valid argument can have false premises and a true conclusion.
\end{question}

\begin{question}
  \answer{False} The word ``because'' usually indicates that the next sentence
  is the conclusion of an argument.
\end{question}

\begin{question}
  \answer{False} Observations and the conclusions inferred from them are
  usually equally reliable.
\end{question}

\begin{question}
  \answer{True} A conclusion can also be a reason for another conclusion.
\end{question}
\end{truefalse}

\begin{fillin}[title={Fill in the blanks (3 pts. each)}]
\begin{question}
  An argument is deductively invalid if \hrulefill
\end{question}

\begin{question}
  What fallacy labels best describe the following?
    \begin{enumerate}
      \item You are giving reasons why the referee's decision was a bad one
      because you are for the Lakers. So I don't accept your conclusion. \hrulefill
      \item \word{{Carl Sagan} {Albert Einstein}} says, ``\word{{Asics} {Reeboks}}
      are the best shoes to wear when thinking about \word{{astronomy}
      {physics}}.'' \hrulefill
    \end{enumerate}
\end{question}

\begin{question}
  If an argument has a conditional proposition as a premise and the affirmation
  of the antecedent of the conditional as a premise, then it is deductively
  valid to conclude \hrulefill
\end{question}

\begin{question}
  Write whether each of the following is an observation or a conclusion:
    \begin{enumerate}
      \item The physicist says, ``This is the particle track of an electron.''
      \hrulefill
      \item The physicist says, ``The particle track on this film curves to the
      right.'' \hrulefill
    \end{enumerate}
\end{question}

\begin{question}
  When examining an argument, there are three things you must do. These are:
    \begin{enumerate}
      \item \hrulefill
      \item \hrulefill
      \item \hrulefill
    \end{enumerate}
\end{question}

\begin{question}
  The \emph{then} part of a conditional is called the \hrulefill.
\end{question}

\begin{question}
  Write the contrapositive of the proposition: If John is tall, then John is
  qualified for the basketball team. \hrulefill
\end{question}
\end{fillin}

\end{document}
