\documentclass[a4paper]{article}
\usepackage{polyglossia,fontspec}
\usepackage[style=verbose-trad2]{biblatex}
\bibliography{example}
\setmainlanguage{french}
\usepackage[dir=txt,sectioning,numbering]{handout}
\newcommand{\beforehandout}{%
	\renewenvironment{quotation}{}{}%
}

\newcommand{\citehandout}{%
	\AtNextCitekey{%
		\forhandout{%
			\beforehandoutref%
		}%
		\forhandout{%
			\cite[\strfield{postnote}]{\strfield{entrykey}}%
			}%
        		\forhandout{%
			\afterhandoutref%
		}%
	}%
}
\newcommand{\beforehandoutref}{\par\noindent\hspace{-2\parindent}}
\newcommand{\afterhandoutref}{\par\vskip0.25\baselineskip}
\begin{document}

\section{Au temps des guerre de religions}
L'historien catholique Gabriel du Preau ce sert de l'épisode lorsqu'il est question de Matthieu\citehandout\footcite[33]{Preau1583}:
\handout{Preau1583-not-and-only.tex}
Des commentaires sur ce passage

Mais notre auteur dispose aussi de l'œuvre de Nicéphore Calliste Xanthopoulos. C'est pourquoi il duplique notre évangile retrouvé\citehandout\footcite[194]{Preau1583}:
\handout{Preau1583b}
Des commentaires sur ce passage
\section{Les conflits autour de l'œuvre de Richard~Simon}
Dans son \emph{Histoire critique du texte du Nouveau Testament}, Richard~Simon écrit\citehandout\footcite[45]{Simon1689}:
\handout{Richard_Simon_NT.tex}
Des commentaires sur ce passage.
\end{document}


