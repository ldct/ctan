% Copyright (C) 1994, Andrew J Harding.  All rights reserved.
\documentstyle[lexitex,11pt]{article}
\begin{document}
\bibliographystyle{lexibib}
\bibliography{spore}
\title{Law as Social Engineering in Singapore: `Smart' Laws in the
Intelligent Island}
\author{Andrew Harding\thanks{Senior Lecturer School of Oriental and
African Studies, London University; presented in the Asian studies
seminar series State and Law in Asia, Asian Studies Centre, St
Anthony's College, Oxford, 19 October 1993.}}
\maketitle

\begin{abstract}
  The development of Singapore law has been an outcome of its
  peculiar history, geography and politics. The ideology of
  social discipline has profoundly affected this development: law
  has been seen primarily as an instrument of social engineering
  rather than as the expression of a particular balance of
  principles defined politically or culturally and regarded as
  the embodiment of justice. The development of the region's
  legal systems along the lines of Singapore's is unlikely
  because of the growth of democracy movements.
\end{abstract}

\section{Introduction}
Ever since Sir Stamford Raffles in 1819 alighted at the point on
the Singapore River where his statue now gazes benevolently down
on the backs of public buildings on the waterfront, Singapore has
been a byword for firm government. In his brief sojourns in
Singapore Raffles laid down many of the principles by which
Singapore is now governed: an economically ambitious policy of
free trade, in particular a free port;\footnote{ It is
  interesting to note that the BBC's excellent documentary
  \lexicite{intelligent-island}, focused on the operation of the
  port of Singapore as the most obvious example of the
  ``on-line'' society. Whatever else is argued in this paper, it
  may well be that in the field of information technology the
  Asia-Pacific region's future will resemble Singapore's
  present.} the recruitment of Singapore's many ethnic
communities behind Government policies; law and order;
cleanliness; purposeful administration; centralization of
political power. Raffles was motivated not only by
utilitarianism, but also by humanitarianism.  This latter aspect
of his policy has also been fulfilled, though only partially.
Raffles would no doubt be pleased with Singapore's prosperity and
environment, the expansion and efficacy of its education, health
care, public services and social institutions, and its prominence
in international counsels. He would, I think, be disappointed
that Singapore has not become the cultural focal point of
maritime South East Asia,\footnote{ Or the Malay world, as he
  would have called it.} and that it has moved away from those
principles of the enlightenment which inspired its creator.

I mention Raffles rather than Lee Kuan Yew in this introduction
because it is easy to lose sight of the fact that Singapore's
history, policy and legal system have been determined to a a
large extent by geopolitics. It was chosen by Raffles because of
its natural harbour, situated so as to serve India and Europe in
the West, China in the East, and South East Asia all around it.
It was from the beginning a commercial colony rather than a
strategic necessity.

The Singapore of what we might call ``the Lee Kuan Yew era''
(1959 to the present)\footnote{ In 1990 Lee stepped down as Prime
  Minister after 29 years, but since then, under the leadership
  of Goh Chok Tong as Prime Minister, Singapore clearly stills
  follows all the principle points of Lee's policy, even if some
  slight differences in style of Government can be detected. As
  Senior Minister in the Prime Minister's Department, Lee still
  clearly exercises great influence, and his position in no way
  corresponds to that of Margaret Thatcher over a similar period.
  A common joke in Singapore is that Goh is now ``the Prime
  Minister in the Senior Minister's Department''. Lee himself has
  described his position as that of a goalkeeper rather than a
  centre forward. Recently Lee has finally relinquished the
  important post of Secretary-General of the People's Action
  Party [PAP] to Goh.} has been to a large extent preordained by
an accident of history which resulted in the failure of its
federation within Malaysia (1963-5). Much of what follows in this
paper would have been true even if federation had succeeded, but
the casting out of Singapore into the unpredictable political
environment of 1960s South East Asia has resulted in the
recognition of its leaders that Singapore is a potentially
vulnerable city-state with a racial make-up which differs greatly
from all the surrounding countries. Indeed it was this racial
difference, as well as economic and political factors, which
hastened Singapore's departure from Malaysia in 1965. The
ideology put forward by Lee and other leaders since then has been
that Singapore has only the intelligence and discipline of its
workforce, and no hinterland of rice-padi and rich natural
resources to fall back on, as Malaysia, China and Indonesia have.
Its only route to survival, let alone economic propserity, has
therefore been to take advantage of its position and
infrastructure to provide goods and services to others, and to be
a prime location for multi-national corporations. It is too small
and vulnerable to withstand the shock-waves of a genuinely open
society, and must maintain a rigid policy of social discipline
and clearly defined, forcefully implemented, social objectives.
This siege mentality is reinforced by, for example, the
continuation of national service and reservist training, even
though there is no military threat to Singapore. Attempts have
also been made to enlist confucianism as a guiding philosophy,
but this has had limited success.

The ideology of social discipline has profoundly affected the
development of law in Singapore. Law has been seen primarily as
an instrument of social engineering rather than as the expression
of a particular balance of principles defined politically or
culturally and regarded as the embodiment of justice. This is a
proposition which would probably find few dissenters, but what I
think is interesting is to speculate on the nature and extent of
this analysis, and to see to what extent it represents a model
for other societies to follow. Is law as social engineering in
Singapore purely an outcome of its situation, or is it indeed a
glimpse of the legal future of the 21st century, not just perhaps
in Asia, but over the rest of the planet? To this question I will
return.


\section{Legal Development in Singapore: Common Law and Statute Law}

After the establishment of Raffles' colony, commerce brought with
it Chinese, Indian and other immigrants from SE Asia and beyond.
Commerce brings not only new ideas and values, carried by people
freed from the traditional constraints of their own
cultures,\footnote{ This was written as referring to the Chinese,
  but applies in some ways to the British colonialists too, who
  were constantly at odds with their overlords.} but also a
motive for legal development: a degree of social stability and
law and order is required; guarantees of private property and the
honouring of promises; the legitimation and bolstering of
institutions. The increase in population too, which in
Singapore's case was an essential ingredient of prosperity,
requires all these things.

Imperial policy required the introduction of the common law,
achieved formally by Charter in
1826,\footnote{\lexicite{bartholomew-englaw}.} and then by the
progressive development of legal institutions---courts, judges,
lawyers, local legislation, police, and eventually a bureaucracy,
taxation, elections to a representative legislature,
constitutional government, and political
independence.\footnote{\lexicite{tan-short}.}


The most notable feature of Singapore's legal development during
the Lee Kuan Yew era has been the growth of statute law. Of
course this is probably true of every country in the world, but
in Singapore it has taken a particular form. Statutes have on the
whole conferred administrative powers going far beyond what is
regarded in most common-law countries as appropriate or
necessary, and to the extent that Singapore has developed an
indigenous legal system with its own peculiar features, these
features are almost exclusively uncommon in the extent to which
they regulate social behaviour. The legal system has become, in
short, a regulatory system. In this one can contrast the emerging
legal systems of other developing countries,\footnote{Singapore
  is of course no longer properly described as a developing
  country, and I am speaking historically here.} which, although
occasionally embodying laws comparable with Singapore's, have
been essentially pluralistic in nature, and attempt to establish
a balance of interests, assuming a diverse rather than a
monolithic society. It is this divergence of statute law from the
standard model one generally finds in common-law countries which
marks the autochthony of Singapore's legal system.

The common law, as is forcefully argued by Andrew Phang in a
recent and very impressive monograph,\footnote{
  \lexicite{phang-development}.} has been characterized by its
lack of development in Singapore.  Taking the example of contract
law, Phang shows how the judges failed to take a Singaporean view
of the subject, simply applying English precedents mechanically,
even where the needs of society demanded a different
result.\footnote{\lexicite{phang-development}, chapter 3.} He
refers to the ``emaciation of custom'' and the lack of
development of alternative forms of dispute resolution. Much the
same can be said of tort law and many other areas of Singapore
law.  The common law is characterized by its failure to achieve
autochthony,\footnote{ This is not of course true of most other
  states which have received the common law, even those in the
  developing areas of Africa and Asia. These states have made the
  common law their own, and local precedents are argued alongside
  English and other cases.  An interesting example of the lack of
  development of common law was the insistance on the rule
  against perpetuities, which directly contradicted Chinese
  customary law in preventing the tying up of property for
  ancestor worship. This approach was typical of the colonial
  judges, but has been continued by the Singaporean judiciary.
  See, further, \lexicite{phang-development}+{55, n.~8}.} and, I
would argue because of this, there is, in parallel, an atrophy of
judicial power.\footnote{Phang goes on to discuss criminal law,
  family law, labour law, and public housing law, finding that
  the innovative legislation in these areas has been successfully
  based on Singapore's particular social and economic
  circumstances.}


On this basis I want to take a brief look, by way of example, at
some particular areas of public law in Singapore by way of
amplification of the thesis of this paper.


\section{Constitutional Development}

Singapore inherited a Westminster-style Constitution from its
colonial past. After independence in 1965 a new Constitution was
promised, but in fact Singapore's constitutional development has
proceeded by a series of amendments over the span of the Lee Kuan
Yew era. Far from failing, like the common law, to achieve
autochthony, Singapore's constitutional development has seen a
series of experiments, and has probably now finally worn into its
shoes with the election in 1993 of Singapore's first elected
President, Ong Teng Cheong, under constitutional amendments
passed in 1991.\footnote{See \lexicite{m-const-amend-1991}, and a
  note on this by Kevin Tan at \lexicite{tan-constamend-note}.}

Developments have centred around three issues, which are linked:
race, opposition, and PAP succession. The objectives have been to
recruit the support of the non-Chinese communities while
suppressing communalism; to provide avenues for the expression of
views opposed to those of the Government without undermining the
dominant-party system; and to ensure that the main tenets of Lee
Kuan Yew's policy will be continued by his successors, and not be
replaced by ``welfarism'', which is regarded as the antithesis of
PAP ideology, now that communism is no longer seen as a threat to
Singapore.

\subsection{Race}
The racial tensions and riots of the 1950s and early 1960s made
race an important issue after independence. A Constitutional
Commission under the Chief Justice, reporting in 1966,\footnote{
  \lexicite{m-const-commn-rept}.} was asked to explore ways of
securing the confidence of the non-Chinese communities in their
future as Singaporeans. The result was the Presidential Council
for Minority Rights, set up in 1970,\footnote{ See
  \lexicite{constamend-n19}.} whose function was to scrutinize
legislation to see if it discriminated against any racial or
religious community. The experiment, promising in its original
conception, foundered because the Government insisted that
members of political parties be allowed to sit on the Council;
the result was that the Council was packed with senior members
and former senior members of the Government, including Lee
himself as Chairman. Naturally the Council has never submitted an
adverse report on any legislation; it quickly became an
irrelevance.\footnote{ See \lexicite{harding-const-proc}.}


The eventual resolution of the problem of ethnic minorities was
the creation of the Group Representation Constituencies (GRCs) in
1988.\footnote{ \lexicite{s-constamend-n9}.} The Constitution now
requires that certain constituencies, which supply one half of
the total number of MPs, be represented by a team of three MPs
elected as a ``slate'' by the voters in three former
constituencies now grouped together; one member of each slate
must be a member of an ethnic minority, ie usually a Singaporean
of Malay/ Muslim or Indian (South Asian) descent. Thus the voters
may choose between a PAP slate and an opposition slate, but are
bound to elect at least one non-Chinese MP, whichever way they
vote.

The ostensible objective of the reform was to ensure that ethnic
minorities were represented in Parliament. In fact the objectives
were probably (i) to ensure that the PAP vote remained stable
without resorting to the laying off of non-Chinese MPs, which
would give the lie to the concept of a multi-racial
Singapore;\footnote{ Non-Chinese PAP MPs generally garnered a
  smaller proportion of the vote than their Chinese counterparts;
  this phenomenon was likely to be accentuated by (i) the
  increasingly mathematical distribution of races into new
  housing estates (itself an important aspect of social
  engineering); and (ii) the adoption of an increasingly
  ``Chinese'' policy by the PAP (Lee has been frank about the
  perceived lack of ``loyalty'' among the Malays, and has said
  that Singapore would do better if its population was racially
  monolithic like that of Japan).} and (ii) to make it more
difficult for the opposition to secure an electoral victory in
particular areas.\footnote{ It is far more difficult for the
  opposition parties to win in the equivalent of three adjoining
  constituencies than to pick off the odd constituency here and
  there.}
 

This would suggest that voters had not returned non-Chinese MPs
in the past. In fact both the PAP and the opposition had included
non-Chinese MPs, and some disquiet was occasioned by this reform,
as it implied that non-Chinese candidates were unelectable.  It
is not insignificant that the visit of Israel's President to
Singapore in 1986, which provoked an outraged response from
Indonesia and Malaysia, was perceived to have provided evidence
of disloyalty among Singapore Malay servicemen.\footnote{
  \lexicite{harding-const-proc}.}


\subsection{Opposition}
The Singapore Government has always taken the question of
opposition seriously, even though it ruled in a one-party
Parliament from 1965 to 1980, and since then has been troubled by
only one, then two, then four opposition members in a 81-member
chamber. The reason for this is that, unlike most other
countries, Singapore, as a city-state, has more or less identical
constituencies;\footnote{ Indeed it is part of PAP policy that
  this should be so.} it is thus possible for almost total PAP
domination of Parliament to be suddenly reversed (following
perhaps some serious economic setback), into almost total defeat.
Proportional representaiton was expressly rejected in
1966.\footnote{ See, further, \lexicite{tan-parliament}.}


One solution, the creation of a one-party state, is closed off.
Although the PAP has defined itself as a national movement
(1983), it was forced by adverse reaction to concede that this
was not a step towards elevation of the PAP to the ``leading
role'' given to communist parties in pre-1989 Europe. A severe
reduction in the PAP vote at the ensuing election (1984)
emphasized that Singapore, with a fairly solid 30--40\%
opposition vote,\footnote{ The PAP share of the vote has dwindled
  from around 80\% in the late 1960s to 63\% in 1991 (voting is
  compulsory in Singapore). This perhaps shows that even
  Singapore's apparently total success in implementing its social
  engineering policy must be qualified by the need to defer to
  some extent to public opinion: see, further,
  \lexicite{phang-development}+{357ff}.} could not go down that
road. As a result, the PAP has had to countenance the legitimacy
of parliamentary opposition, and has sought instead to control
it.

The second option, the creation of Non-constituency MPs, designed
to give a seat in Parliament to the most successful of the
unelected opposition candidates, an apparently generous gesture,
did not solve the problem, as it proved unpopular among the
opposition parties, who prefer to win their seats rather than
rely on government charity; and the provisions, applying only
where there are less than two opposition MPs, have been overtaken
by events.

     More recently, the creation of Nominated MPs (the third
option), currently four in number, has met with greater success.
These ``NMPs'' can participate in debates and vote.\footnote{ Except on money bills, supply bills, constitutional
amendments, and confidence motions.}
 

In case of PAP members deciding to cross the floor, an amendment
introduced to deal with the politics of the tubulent 1960s
ensures that they will not be able to do so without forfeiting
their seats in Parliament.\footnote{ \lexicite{s-const}+{46(2)}.}
Indeed PAP MPs who vote against the Government, or even abstain,
are threatened with expulsion from the party.

Any notion that legal development has embraced political
opposition, is however, quickly contradicted by the constant
legal harrassment of opposition MPs and the tough action taken
against those who express opinions outside the arena of party
politics. Not only the Singapore Law Society, but also NGOs and
individual critics, have been targeted, especially in ``Operation
Spectrum'' in 1987, in which 29 people, mainly Church workers and
social activists, were detained without trial under the Internal
Security Act, accused of having mounted a Marxist conspiracy to
overthrow the Government.\footnote{ See \lexicite{ricjs-1987}.}
This action outraged international opinion. At present, however,
there are no ostensibly political detainees in Singapore.

(c) Succession.
     The notion of an elected presidency to replace the nomination
of the President by Parliament was conceived as a means of
buttressing PAP rule, or at least the main tenets of PAP rule, and
in particular as a means of preventing the dissipation of
Singapore's substantial reserves. Although it was thought by most
that this post was one into which Lee Kuan Yew would ease himself
as he talked more and more of giving up executive power, in fact he
remains as Senior Minister, and a former Deputy Prime Minister, Ong
Teng Cheong, has been elected. His single opponent in the 1993
election was a little-known former civil servant.

The main problem with such an elected President is of course how
his powers relate to those of the Government itself. The
structure created in Singapore is unique. The President is
endowed not only with an electoral mandate, but with a formidable
array of powers. He can withhold assent to certain Bills, veto
Government loans, senior appointments and budgets of statutory
boards and Government companies, and exercise various other
powers. In short, by use of his purely negative powers, he can
bring government grinding to a standstill at any time.

Another problem with this reform, from the PAP point of view, was
always that the presidential election might become a hustings for
opposition candidates. This possibility has been preempted by
imposing onerous requirements on presidential candidates, so that
opposition politicans of the present echelon at least, are unable
to stand. The rules are designed so that only members of the
PAP-led political, administrative and business elite, can stand
for election.

Concluding this part of the paper, I would observe that law as
social engineering in the field of constitutional law in
Singapore has been largely a success, judged in terms of the
objectives of reforms.\footnote{ It must be conceded that the
  theme of this paper raises some interesting theoretical issues
  about the nature of law which there has been no space to go
  into. I have deliberately adopted a positivist, Austinian
  approach, because that seems to me appropriate to the
  subject-matter. However, the ``social engineering'' approach to
  the analysis of legal systems does, in general, have to be
  handled with care. For some of the difficulties involved, see
  \lexicite{woodman-allotrev}, and Allott's reply, which follows,
  \lexicite{woodman-allottrev-reply}.} However, there is a kind
of ``smartness'' about these laws which could lead to their
removal at some time in the future. By ``smartness'', I mean that
they appear to be programmed to produce not just in general, but
rather too precisely, the result desired by their creators. This
is a characteristic of many of Singapore's social engineering
laws. They are the kind of laws which in the short term seek out
their targets with relentless accuracy, negotiating every
obstacle placed to thwart their efficacy; but in the long term
they may be shorn of legitimacy by their very smartness---they
are too smart for their own good. Perhaps they will disappear to
the same part of the legal underworld which is reserved for Henry
VIII's Star Chamber and the laws of the communist dictatorships
of Eastern Europe. They do not provide a framework within which
any future Government, even perhaps a PAP Government, would feel
happy.


\section{Administrative Law and the Administrative State}\footnote{See \lexicite{tynne-admin-state}.}

Administrative law has seen burgeoning growth in developing
countries over the last decade or so, and we are now getting used
to examples of judicial independence and administrative-law
reform cropping up in unlikely places, such as Indonesia and
China.

As a generalization about administrative law in Singapore, I
would say that it has displayed great activity but quite
remarkable lack of development. The courts have proved very
unwilling to question administrative decisions in most areas:
compulsory purchase, taxation, citizenship, immigration, control
of the press, to name but a few.\footnote{ See, eg,
  \lexicite{re-dow-jones-asia}.} By way of contrast, they have
been willing to intervene with the decisions of tribunals and
disciplinary bodies where natural justice has not been observed.
But the cases have been few and rather insignificant.

As against the atrophy of judicial review and rule-of-law
principles, administrative law in the narrower sense of
regulatory statute law and meticulous enforcement has been
developing rapidly.  Hardly anything regarded by the Government
as a mischief has been left without drastic regulation: adverse
comment by NGOs, lawyers, church leaders or foreign
journalists;\footnote{ Dealt with by the
  \lexicite{maint-of-relig-harmony-act}++{{8}\dash{19}};
  \lexicite{internal-security-act}++{{8}\dash{19}};
  \lexicite{newsp-printing-presses-amd-act}. For the last see
  \lexicite{batterman-sing-news}.  } blocking of refuse chutes
in apartment blocks; failure by the elite to perpetuate their
genes; smoking in public; drug-trafficking; firearms; strikes;
silent defendants; traffic jams in the city centre; crooked
lawyers; traditional Malay villages (kampongs); official
corruption; litter; chewing gum; and even unflushed public
toilets.\footnote{ The tropical mosquito did not stand a chance,
  and has been dismissed by the irresistible advance of concrete
  modernity.}


In many of these things the Singapore Government's actions have
been amply and loudly justified. However, the smartness of the
legal mechanisms used does not extend to preserving from
collateral damage a large number of fundamental liberties. The
statutes are not sufficiently smart to be programmed to
distinguish between activities which are simply anti-social, and
activities which may have a combination of desirable and
undesirable aspects, or which may be wholly desirable. For
example, restricting traffic in the centre at peak hours and the
size of the car population in general seems sensible (to this
observer at least), the infringement of personal liberty being
marginal. The restrictions on criticism by the press, the
churches, and NGOs, however, the sign of a healthy, democratic
society, may (for all I know) marginally increase foreign
investment, in the sense that Singapore will be perceived as a
stable business environment, but the cost in terms of freedom of
thought and expression, which any entrepreneurial society, let
alone an open society, needs, is very great,\footnote{ Having
  taught at the National University for several years, I can
  vouch for the effect of discouragement of criticism on the
  mentality of a generation of Singaporeans.} unless one sees the
population simply as an unintelligent resource, obedient
automata, or ``digits'', to use the word often used by Lee
himself and other PAP leaders.

\subsection{Habeas corpus}
The development of habeas corpus is a good example of the
smartness of Singapore laws.  In \footnote{% to use -t- here
                                           % later
  \lexicite{chngsuantzevminister}; for comment see Sin Boon
  Ann, `Judges and Executive Discretion---a Look at
  \lexicite{sin-boon-ann-chingvminister};
  \lexicite{harding-singapore-prevent}. See also
  \lexicite{teo-soh-lung-v-min}; \lexicite{vincent-cheng-v-min}.}
the Court of Appeal had to address what is probably the single
most important question of administrative law: to what extent can
the courts review the exercise of a subjectively-framed
discretion (in this case to detain persons under the Internal
Security Act as a threat to security or public order)?

The Court commented adversely on a previous decision\footnote{
  \lexicite{lee-mau-seng-v-min}.} denying the possibility of such
review in security cases, and opined that the appropriate test
was an objective one: the satisfaction of the President (acting
on ministerial advice) had to be objectively reasonable, and it
was not sufficient that the minister genuinely believed himself
satisfied. However, the actual ratio of the case was that there
was no actual evidence of presidential satisfaction. As a result,
habeas corpus was granted and the detainees released. However,
quite cynically, the Government arranged for their rearrest
outside the detention centre on freshly prepared documentation
which complied with the Court's ruling. If the court had framed
the ratio of the case more generally, this would not have been
possible.

Legislation was then passed restoring the previous law, excluding
the relevance of the case-law of any other
jurisdiction,\footnote{ Cases from other Commonwealth countries,
  including Namibia and Zimbabwe were cited in Chng's case.} and
denying an appeal to the Privy Council in security
cases.\footnote{ \lexicite{intl-sec-amend-1989}. Consequential
  constitutional amendments were effected by the
  \lexicite{constamend-1989}.}


The smartness of these laws goes even further. The appeal to the
Privy Council could of course be abolished at any time, but has
been retained because it encourages inward investment. However,
the appeal requires agreement between the parties at any time
before the case goes to the Singapore Court of Appeal, and is not
allowed in security cases and cases involving professional
discipline. Thus important commercial cases can still go the
Privy Council and be decided by English judges in London, but
cases involving the Government can be filtered out simply by the
expedient of the Government refusing to agree to the appeal, thus
allowing the case to be determined finally by the Singaporean
judiciary. The competence and independence of the judges need not
be in issue: if their decisions are not sufficiently smart, they
can be reversed by exercise of legislative power, and if this
requires a constitutional amendment, the Constitution also
ensures that the two-thirds requisite majority is always
forthcoming.

\subsection{Eugenics}
Another example which has become famous is the so-called
`Graduate Mothers Scheme', under which university-educated
mothers were allowed certain privileges with regard to choice of
primary school for their children. The object of this scheme was
to encourage the reproduction of the elite, which was reckoned to
have fallen behind that of less qualified parents. This
represented the reversal of a family-planning policy which had
been rigorously enforced by a series of carrot-and-stick methods
over a period of several years, the object of which was to
prevent a population explosion in a small island with little land
to spare. The scheme failed, as very few mothers took up their
rights under the scheme.\footnote{ Tax incentives still remain,
  however.} After a vigorous defence of its purpose, the
Government quietly dropped it the year following its
introduction.  What is interesting is that, although there was no
constitutional challenge to the scheme, it was clearly perceived
by a significant number of people as an illegitimate use of
administrative power. Smartness, even in Singapore, has to extend
to smartness about public opinion, even though the Singapore
Government has proved adept at opinion-formation, and has
sometimes succeeded in altering what Governments elsewhere might
regard as an intractable environment of public opinion. The
extremes to which Government goes to alter the environmental of
opinion makes the nature of Singaporean laws highly instrumental
and regulatory, when taken in conjunction with administrative
measures and campaigns.\footnote{ Here I part company with Phang
  (eg, \lexicite{phang-development}++{{274}\dash{75}}), who regards
  these laws as proceeding from a favourable environment of
  public opinion. It is of course true that there are
  countervailing values such as ``westernization'', which have
  made the Singapore Government's task much more difficult; this
  development has resulted in a more materialistic society, but
  not in a greater emphasis on individual rights.}


Laws on voluntary sterilization and abortion have also played a
large part in the eugenics policy; these have had the effect of
encouraging and liberalizing access to sterilization and
abortion, thereby restricting population growth.

To conclude this section, administrative law in Singapore has
become law for administrators, not in my view a balance between
the rights of citizens and the practical attainment of collective
goals. There have been some very desirable consequences apart
from the erosion of basic liberties: the virtual abolition of
corruption, the provision of public housing and health care, and
the reduction in environmental pollution, for example. None of
these achievements can really be attributable to the denial of
fundamental rights as such. Labour laws, on the other hand, have
severely restricted rights of freedom of expression, assembly and
association.

\section{Conclusions}

Let me now return to the question posed near the beginning of
this paper: is law as social engineering in Singapore purely an
outcome of its situation, or is it indeed a glimpse of the legal
future of the 21st century in the Asia-Pacific region?

To answer this question one must look more widely at events in
East and South East Asia, and look carefully at the crystal ball
(or at the yam sticks!).

I hope to have shown how the development of Singapore law has
been an outcome of its peculiar history, geography and politics.
It would be an easily achieved answer to say that the case of
Singapore affords us no general propositions about the future of
state and law in the region or the world: it is a one-off case,
albeit a remarkable one.

This would be superficial reductionism. There is much in the
Singapore experience which matches that of the premier-league
players, Japan, Hong Kong, Taiwan, and South Korea, and now also
to some extent, in the first division, Malaysia, Thailand, China,
and possibly also, looking to those aspirants for promotion,
Indonesia, the Philippines, and Vietnam.

     There is also no doubt that Lee Kuan Yew and other Singapore
leaders see Singapore as a model for others to follow. This
attitude is clearly shared by the leaders of some other countries.
China has asked Singapore to create a Singapore clone at the city
of Suzhou, near Shanghai. Thailand, Indonesia and Malaysia are
cooperating with Singapore in the creation of special economic
zones. Singaporean enterprises are investing in Vietnam and China;
the latter has now become the largest recipient of Singaporean
outward investment.

There are of course important differences. These countries all
have autochthonous legal systems dating from pre-industrial
times.  Some are still communist states. Although they display
different degrees of openness in their political systems, all
have large and growing democracy movements, spawned by
educational advances, the rise of a prosperous and ambitious
middle class, and the influence of NGO movements and other
international movements. In December 1991 President Ramos of the
Philippines politely rebuked Lee Kuan Yew for suggesting a
Singapore-model approach in that country, reminding him that his
country had already tried an authoritarian approach without much
success.\footnote{ \lexicite{disc-democracy}.} It is a grave
error, in my view, to think that state structures, political
cultures and legal systems in Asia are similar or are converging.

The question therefore arises whether these emerging NICs, as
they are often called, or newly democratizing countries (NDCs, as
I would prefer) can pursue a Singapore-style legal system. I
think the answer to this is that although, to an extent, they
have already done so, further development of the region's legal
systems along the lines of Singapore's is unlikely. Although it
has been affected by the tail-end of the region's democratic
reform-oriented movement, snuffed out in the 1987 detentions,
Singapore has only marginally conceded a point or two to this
movement, putting forward an alternative ``style'' of PAP
government under Prime Minister Goh Chok Tong since
1990.\footnote{ By representing the two most popular approaches
  to Government in Singapore, the PAP has cleverly, to some
  extent, succeeded in garnering the radical (``consensual
  Government'') vote as well as the conservative (``authoritarian
  Government'') vote.}


The Singaporean educated middle-classes are a much more malleable
entity than their equivalents elsewhere in the region, where
important concessions have been made to ``multi-party
democracy''. Even in Japan, the model which others in the region
try to emulate, a coalition of non-LDP parties has taken power
after two generations of LDP rule, an event unthinkable until
recently.  The effect of such events on Singapore has, so far,
been slight.  Even the present ``consensual style'' of Government
has been able to embrace the dismissal of an opposition
politician from his university post on the flimsiest of charges,
and the hounding of Workers' Party leader JB Jeyaratnam resulted
in some very adverse comments from the Privy Council.\footnote{
  \lexicite{jeyaretnam-v-law-soc-sing}.} There is little sign
that the new style is anything more than the old policy in new
clothing. The necessities of Singapore's situation and the
economic success achieved over the past two decades have enabled
the PAP to immunize itself effectively against the democracy
movement, albeit with some international disapproval.  The size
of Singapore has enabled a growth and effectiveness of the organs
of state to an extent which even Japan, with all its social
cohesion, has not been able to achieve. The severe
anti-corruption laws protect the Government from the most telling
charge which is brought against its peers elsewhere in the
region, and which fuels the demand for legal reform.

My conclusion is therefore that the monsoon-winds of change now
sweeping the region cannot be broken by the wide espousal of a
Singapore-type legal system. On the contrary, I think the
question is whether Singapore itself will be swept along by these
winds. My guess is that the legal system is now firmly
entrenched, and that it will ride the storm. I envisage that much
of the region will become, superficially, more like Singapore,
but Singapore itself will have to make some concessions to the
growing desire for rule-of-law institutions, or a rechtstaat,
and the paraphernalia of constitutional democracy. 

\end{document}








