


\documentclass[12pt,a4paper]{article}

% needed packages
\usepackage{calc}
\usepackage{ifthen}
\usepackage{graphics}
\usepackage{epsfig}
\usepackage{miniplot}

% supplementary packages
\usepackage{wrapfig}
\usepackage{subfigure}
\usepackage{rotfloat}

%\usepackage{showkeys}
\usepackage{flafter}

% personal adjustments
\newcommand{\comm}[1]{\texttt{#1}}
\renewcommand{\floatpagefraction}{0.8}    % default = 0.5, see "The LaTeX Companion, page 142
\setcounter{plotFigures}{1}


\begin{document}

%\pagestyle{myheadings}
%\markright{Section \thesection}

\title{MiniPlot -- A Package For Easy Figure Arrangements\footnote{Version 0.3}}
\author{Tobias Wahl\footnote{I'd like to thank all the people helping me in \comm{comp.text.tex}
for answering my dull questions.}}
\date{1st July 2001}
\maketitle


\begin{abstract}
MiniPlot is a package to help \LaTeX's user to typeset eps figures
using an easy-to-use command-interface.

Figures can easily be arranged as one-figure-only or as a collection of
figure in columns and rows which can contain itself sub-figures in
columns and rows. Also wrapped and sideways figures are supported.

This package provides commands to display a framebox instead of
the figure as the graphics package does already but additionally
it writes useful information such as the label and scaling factor
into these boxes. The content of these boxes can be customized.
\end{abstract}

\tableofcontents


%=================================================================
\section{Motivation}

This section is to show what MiniPlot is about. MiniPlot is a
strong tool for plotting multiple figures and sub-figure as the 
following figures show. It's easy to use interface allows easy and 
fast access to professional looking figure presentation.


\begin{arrangedFigure}{2}{2}{arrangedO521}{left sub-figure}
    \subFig[]{figures/1}%
    \subFig[]{figures/2}%
    \subFig[This is a very long sub-sub-caption]{figures/3}%
    \subFig[]{figures/4}%
\newSubFig{arrangedO522}{right sub-figure}
    \subFig[]{figures/1}%
    \subFig[]{figures/2}%
    \subFig[]{figures/3}%
    \subFig[]{figures/4}%
\setlength{\subFigureAboveCaptionSpace}{7mm} 
\end{arrangedFigure} 
\setlength{\subFigureAboveCaptionSpace}{0mm} 


\begin{arrangedFigure}{1}{4}{arrangedO1521}{top sub-figure}
    \subFig[]{figures/1}%
    \subFig[]{figures/2}%
    \subFig[]{figures/3}%
    \subFig[]{figures/4}%
\newSubFig{arrangedO1522}{bottom sub-figure}
    \subFig[]{figures/1}%
    \subFig[]{figures/2}%
    \subFig[]{figures/3}%
    \subFig[]{figures/4}%
\end{arrangedFigure}


\begin{arrangedFigure}{2}{3}{arrangedO31}{top left}
    \subFig{figures/1}%
    \subFig{figures/2}%
    \subFig{figures/3}%
    \subFig{figures/4}%
    \subFig{figures/1}%
    \subFig{figures/2}%
\newSubFig{arrangedO32}{top right}
    \subFig{figures/1}%
    \subFig{figures/2}%
    \subFig{figures/3}%
    \subFig{figures/4}%
    \subFig{figures/1}%
    \subFig{figures/2}%
\newSubFig{arrangedO33}{bottom}
    \subFig{figures/1}%
    \subFig{figures/2}%
    \subFig{figures/3}%
    \subFig{figures/4}%
    \subFig{figures/1}%
    \subFig{figures/2}%
\end{arrangedFigure}



\clearpage

%================================================================= 



%---------------------------------------------------------------- 
\section{Loading the packages}

In order to run MiniPlot the following lines are needed in the \LaTeX
file


\begin{verbatim}
% needed packages
\usepackage{calc}
\usepackage{ifthen}
\usepackage{graphics}
\usepackage{epsfig}
\usepackage{miniplot}

% supplementary packages
\usepackage{wrapfig}
\usepackage{subfigure}
\usepackage{rotfloat}

% personal adjustments
%\setcounter{plotFigures}{0}
\renewcommand{\floatpagefraction}{0.8}    
\end{verbatim}

The wrapfigure package is only needed when one uses the wrapfigure
features of MiniPlot. The subfigure package is certainly something
one might include since this gives access to the \comm{arrangedFigure}
environment, the most powerful feature of this package.
Rotfloat, and rotating which is loaded by rotfloat, is needed when one 
likes to use the side-way-figure capabilities of MiniPlot.
I recommend to install all packages.

Putting the line \comm{setcounter\{plotFigures\}\{0\}} is not necessary but is
recomended for editing a document with loads of figures or when one likes
to obtain the label of figures for refences. Enable this line when you learned 
how to deal with this package.

The last line is very much my own personal style and not needed for using 
MiniPlot.



%---------------------------------------------------------------- 
\section{Including single figures}

The problem is how to arrange your eps figures efficiently in you
document. The answer is given by MiniPlot. This section is
included to provide an overview over the capabilities of MiniPlot.

We start with a simple implementation of a single eps figure as
shown in Figure \ref{fig:singleSimple}, the simple implementation
is given by

\begin{verbatim} \includeEps{figures/1}{Simple figure included 
with scaling factor equals one.}{singleSimple}{1} \end{verbatim}

\includeEps{figures/1}{Simple figure included with scaling factor 
equals one.}{singleSimple}{1}

Note that the label is given without \comm{fig:}. The general 
parameters are

\begin{verbatim}
\includeEps[htpb]{path}{label}{caption}{scaling factor}
\end{verbatim}

Now imagine one has one type of figure which occurs very often in 
the document. Then the user could use the following command

\begin{verbatim} \includeStandardGraph[htpb]{path}{label}{caption} 
\end{verbatim}

where the standard scaling factor can be set generally for all 
the figures included by this command. The default is 

\begin{verbatim} \renewcommand{\standardGraphScale}{1} 
\end{verbatim}

\includeStandardGraph{figures/1}{Simple figure included with 
scaling factor calculated to scale figure to meet specified 
width.}{singleStandard}

However, by default this value is not used but the figure is 
scaled to achieve the width equal to the length of the following 
variable.

\begin{verbatim} \setlength{\myStandardFigureWidth}{\linewidth} 
\end{verbatim}

If the figures are scaled to the width above or if a standard 
scaling factor is assumed depends on the following counter

\begin{verbatim} \setcounter{scaleToStandardFigureWidth}{1} 
\end{verbatim}

Set \comm{1} to to scale to standard-figure-width or \comm{0} to 
use standard-graph-scale. 

Additionally one can set if each figure is to be scaled 
individually or if the scaling factor is to be calculated at the 
first time the include-standard-graph command is stated

\begin{verbatim} \setcounter{scaleFigureIndividual}{1} 
\end{verbatim} 

Set \comm{1} to to scale individually or \comm{0} to use scaling 
factor calculated when inserting the first standard-graph. 

When we now include a figure using the command for standard 
figures 

\begin{verbatim} \includeStandardGraph{figures/1}{Simple figure 
included with scaling factor calculated to scale figure to meet 
specified width.}{singleStandard} 
\end{verbatim}

we obtain a figure scaled to the specified width in Figure 
\ref{fig:singleStandard}.

To show the use of the counter \comm{plotFigures} we set

\begin{verbatim} 
\setcounter{plotFigures}{0} 

\includeStandardGraph{figures/1}{Simple figure included with 
scaling factor calculated to scale figure to meet specified 
width.}{singleStandard2} 

\setcounter{plotFigures}{1} 
\end{verbatim}

to obtain Figure \ref{fig:singleStandard2}. We can see that the 
scaling factor is no more equals one. Other usefull information 
is given by the label. 

\setcounter{plotFigures}{0} 
\includeStandardGraph{figures/1}{Simple figure included with 
scaling factor calculated to scale figure to meet specified 
width.}{singleStandard2} 
\setcounter{plotFigures}{1} 

Of course, there is also a side-ways version of a figure, see 
Figure \ref{fig:singleStandard3}. This figure is implemented using
the include-standard-side-ways-graph command. 

\begin{verbatim} 
\setcounter{plotFigures}{0} 

\includeStandardSideWaysGraph{figures/1}{Simple figure included with 
scaling factor calculated to scale figure to meet specified in width 
in side-ways-mode.}{singleStandard3} 

\setcounter{plotFigures}{1} 
\end{verbatim}

\setcounter{plotFigures}{0} 
\includeStandardSideWaysGraph{figures/1}{Simple figure included with 
scaling factor calculated to scale figure to meet specified 
width in side-ways-mode.}{singleStandard3} 
\setcounter{plotFigures}{1} 

Note that the width of the side-ways-figure is now determined by

\begin{verbatim} 
\setlength{\myStandardSideWaysFigureWidth}{\textheight}
\end{verbatim}

Note that there is also an include-side-ways-eps command where 
you have to set a scaling factor.

The user can customize the width of the caption relative to the figure's width
simply by changing the folowing length.

\begin{verbatim}
\setlength{\captionWidthPenalty}{0mm}
\end{verbatim}

Setting this length has only an effect to the figures which are introduced as a single
figure

\begin{itemize}
  \item \verb+\includeEps+
  \item \verb+\includeEpsNow+
  \item \verb+\includeSideWaysEps+
  \item \verb+\includeSideWaysEpsNow+
  \item \verb+\includeStandardGraph+
  \item \verb+\includeStandardGraphNow+
  \item \verb+\includeStandardSideWaysGraph+
  \item \verb+\includeStandardSideWaysGraphNow+
\end{itemize}

It has no effect to the wrapped figure environments and the arranged figure environments.

Normally the captions of the listed figure commands and environments have the 
width of the figure. In case the figure's width is smaller than a specified value 
then it will be set to the defined caption minimum width 

\begin{verbatim}
\setlength{\minimumCaptionWidth}{\linewidth * 4 / 6}
\setlength{\minimumSideWaysCaptionWidth}{\textheight * 4 / 6}
\end{verbatim}



\clearpage








%---------------------------------------------------------------- 
\section{Including arranged sub-figures and sub-sub-figures}


The content of this subsection is the generation of figures which can
consist of several columns and rows of figures with their own caption 
(Figure ...), I will call these figures sub-figures.

Inside these sub-figures can now be arranged several columns and rows of 
sub-sub-figures. These are placed using the subfigure package.

A simple example might be given by Figure \ref{fig:arranged11} to \ref{fig:arranged12}

\begin{verbatim}
\begin{arrangedFigure}{2}{2}{arranged11}{This is the left 
                        sub-figure of an arranged figure.}
    \subFig{figures/1}%
    \subFig{figures/2}%
    \subFig{figures/3}%
    \subFig{figures/4}%
\newSubFig{arranged12}{This is the right sub-figure of an 
                                        arranged figure.}
    \subFig{figures/1}%
    \subFig{figures/2}%
    \subFig{figures/3}%
    \subFig{figures/4}%
\end{arrangedFigure}
\end{verbatim}

\begin{arrangedFigure}{2}{2}{arranged11}{This is the left sub-figure of an arranged figure.}
    \subFig{figures/1}%
    \subFig{figures/2}%
    \subFig{figures/3}%
    \subFig{figures/4}%
\newSubFig{arranged12}{This is the right sub-figure of an arranged figure.}
    \subFig{figures/1}%
    \subFig{figures/2}%
    \subFig{figures/3}%
    \subFig{figures/4}%
\end{arrangedFigure}

The first parameter to the \comm{arrangedFigure} environment is the number of 
columns of sub-figures, the second one specifies the number of columns of 
sub-sub-figures. The next parameters is the label and then the caption to the immediate
following sub-figures.

The command \comm{subFig} should actually be called \comm{subSubFig} since it inserts the 
sub-sub-figure specified by the path of the figure to be included. \emph{It is important to 
enter the \%-signs after the \comm{subFig} command in order to avoid introducing 
undesired horizontal spaces between the sub-sub-figures.}

The command \comm{newSubFig} opens the next que to fill up a sub-figure with sub-sub-figures.
The parameter is the label and the caption for the immediately following sub-figure.

Another example could be given by Figure \ref{fig:arranged2}

\begin{verbatim}
\begin{arrangedFigure}{1}{4}{arranged2}{This is the 
                only sub-figure of an arranged figure.}
    \subFig[]{figures/1}%
    \subFig[]{figures/2}%
    \subFig[]{figures/3}%
    \subFig[]{figures/4}%
\end{arrangedFigure}
\end{verbatim}


\begin{arrangedFigure}{1}{4}{arranged2}{This is the only sub-figure of an arranged figure.}
    \subFig[]{figures/1}%
    \subFig[]{figures/2}%
    \subFig[]{figures/3}%
    \subFig[]{figures/4}%
\end{arrangedFigure}

Note that the number of columns of sub-figures changed to one and that we arrange four 
sub-sub-figures in the width of one sub-figure column.

Additionally, the \comm{subFig} command got now an empty optional command leading to the indexing
of the sub-sub-figures from a to d. This behavior is identically with the one of the 
subfigure package\footnote{The only exception is that if one enters \comm{[noParMiniPlot]} it would be
as if no optional parameter is given. This shouldn't be a major problem since there might be little
desire to have a caption like \comm{(a)noParMiniPlot} .}.

To complete a list of examples a last one is given by Figure \ref{fig:arranged31} to
\ref{fig:arranged33}



\begin{verbatim}
\renewcommand{\subfigbottomskip}{0pt}%
\renewcommand{\subfigcapskip}{0pt}%
\renewcommand{\subcapsize}{\scriptsize}%

\begin{arrangedFigure}{2}{4}{arranged31}{top left}
    \subFig[]{figures/1}%
    \subFig[]{figures/2}%
    \subFig[]{figures/3}%
    \subFig[]{figures/4}%
    \subFig[]{figures/1}%
    \subFig[]{figures/2}%
\newSubFig{arranged32}{top right}%
    \subFig[]{figures/1}%
    \subFig[]{figures/2}%
    \subFig[]{figures/3}%
    \subFig[]{figures/4}%
    \subFig[]{figures/1}%
    \subFig[]{figures/2}%
\newSubFig{arranged33}{bottom}%
    \subFig[]{figures/1}%
    \subFig[]{figures/2}%
    \subFig[]{figures/3}%
    \subFig[]{figures/4}%
    \subFig[]{figures/1}%
    \subFig[]{figures/2}%
    \subFig[]{figures/3}%
    \subFig[]{figures/4}%
\end{arrangedFigure}

\renewcommand{\subfigbottomskip}{10pt}%
\renewcommand{\subfigcapskip}{10pt}%
\renewcommand{\subcapsize}{\footnotesize}%
\end{verbatim}

The redefinition of commands introduced by the subfigure package makes the figure 
look nicer but are not necessary to run the code. After the redefinition they are reset to 
their default values.

In this figure we have an odd ratio of sub-figures to sub-figure-columns
and we see that the remaining sub-figure \ref{fig:arranged33} is shifted 
to the right.

The same happens to the sub-sub-figures included in the bottom row of 
sub-figures \ref{fig:arranged31} and \ref{fig:arranged32}.


    \renewcommand{\subfigbottomskip}{0pt}%
    \renewcommand{\subfigcapskip}{0pt}%
    \renewcommand{\subcapsize}{\scriptsize}%

\begin{arrangedFigure}{2}{4}{arranged31}{top left}
    \subFig[]{figures/1}%
    \subFig[]{figures/2}%
    \subFig[]{figures/3}%
    \subFig[]{figures/4}%
    \subFig[]{figures/1}%
    \subFig[]{figures/2}%
\newSubFig{arranged32}{top right}
    \subFig[]{figures/1}%
    \subFig[]{figures/2}%
    \subFig[]{figures/3}%
    \subFig[]{figures/4}%
    \subFig[]{figures/1}%
    \subFig[]{figures/2}%
\newSubFig{arranged33}{bottom}
    \subFig[]{figures/1}%
    \subFig[]{figures/2}%
    \subFig[]{figures/3}%
    \subFig[]{figures/4}%
    \subFig[]{figures/1}%
    \subFig[]{figures/2}%
    \subFig[]{figures/3}%
    \subFig[]{figures/4}%
\end{arrangedFigure}

    \renewcommand{\subfigbottomskip}{10pt}%
    \renewcommand{\subfigcapskip}{10pt}%
    \renewcommand{\subcapsize}{\footnotesize}%


The reader certainly notices that it takes longer to load the page when a lot
of sub-sub-figures are arranged, Figure \ref{fig:arranged41} to \ref{fig:arranged42}. 
This is an important reason to set the counter \comm{plotFigures} to zero to save 
time when browsing through a document on the computer.

As another example an arranged-side-ways-figure is shown, Figure \ref{fig:arranged41}. 

\begin{verbatim}
\setcounter{plotFigures}{0}

\begin{arrangedSideWaysFigure}{2}{2}{arranged41}{left sub-figure}
    \subFig[]{figures/1}%
    \subFig[]{figures/2}%
    \subFig[]{figures/3}%
    \subFig[]{figures/4}%
\newSubFig{arranged42}{right sub-figure}
    \subFig[]{figures/1}%
    \subFig[]{figures/2}%
    \subFig[]{figures/3}%
    \subFig[]{figures/4}%
\end{arrangedSideWaysFigure}

\setcounter{plotFigures}{1}
\end{verbatim}



\setcounter{plotFigures}{0}
\begin{arrangedSideWaysFigure}{2}{2}{arranged41}{left sub-figure}
    \subFig[]{figures/1}%
    \subFig[]{figures/2}%
    \subFig[]{figures/3}%
    \subFig[]{figures/4}%
\newSubFig{arranged42}{right sub-figure}
    \subFig[]{figures/1}%
    \subFig[]{figures/2}%
    \subFig[]{figures/3}%
    \subFig[]{figures/4}%
\end{arrangedSideWaysFigure}
\setcounter{plotFigures}{1}


That this page loads faster is not the only difference one can notice. The label for 
reference is now added below the caption in order to be easier accessible. A disadvantage 
is that an extra space might be included and that might influence the document layout. In order to
compensate for that effect the following length has been introduced

\begin{verbatim}
\setlength{\labelPenalty}{-1.5mm}
\end{verbatim}

which is valid if the following command is design to produce a single line in small font.

\begin{verbatim}
\newcommand{\printLabel}[1]{\centerline{\small{\emph{Label}: \texttt{fig:#1}}}}
\end{verbatim}

If one redefines \verb+\printLabel+ one has to figure out what value might be appropriate
for \verb+\labelPenalty+.

Furthermore, only the most basic information is included in the frameboxes. This is 
because the arranged sub-sub-figures can become quiet small. First the basic filename 
without path is contained. In the next row the scaling factor is shown, than the suffix 
to the label shown below the sub-fig caption is included.

So far referencing to sub-sub-figures does not work but as soon as someone tells me how 
to extract \comm{a} fromm \comm{(a)}, which comes from the \verb+\thesubfigure+ command, I 
would like to implement that. :)


\clearpage








%---------------------------------------------------------------- 
\section{Including wrapped figures}


The use of the include-Eps-Wrap command is similar to the 
include-Eps command. By writing

\setWrapFigOverhang[2cm]
\includeEpsWrap[20]{l}{figures/wrapMe}{Example of an wrapped figure}{wrappedFigure1}{1}
\setWrapFigOverhang

\begin{verbatim} 
\setWrapFigOverhang[2cm]
\includeEpsWrap[20]{l}{figures/wrapMe}{Example of an 
wrapped figure}{wrappedFigure1}{1} 
\setWrapFigOverhang
\end{verbatim}

we obtain Figure \ref{fig:wrappedFigure1}.

The general form is given by

\begin{verbatim} 
\includeEpsWrap[lines]{side}{path}
{caption}{label}{scaling factor} 
\end{verbatim}

where \comm{lines} stands for the number of narrow lines and \comm{side} stands 
for left (\comm{l}) or right (\comm{r}) as specified in the documentation of the 
wrapfig package.

The command \verb+setWrapFigOverhang+ sets the wrap figures overhang to the given optinal 
parameter. Calling this command without a parameter resets the overhang to its default value

\begin{verbatim} 
\setlength{\wrapFigOverhangDefault}{0mm}
\end{verbatim}

Of course, if the counter \comm{plotFigures} equals 0 then this figure is 
replaced by a framebox containing some useful information:

\begin{verbatim}
\setcounter{plotFigures}{0}
\includeEpsWrap[20]{l}{figures/wrapMe}
{Example of an wrapped figure}
{wrappedFigure2}{1}
\setcounter{plotFigures}{1}
\end{verbatim}

The counter \comm{plotFigures} is inteded to be \comm{0} in 
draft mode to improve browsing speed through the document and
to provide information for referencing. However, when a figure 
is to be implemented the first time one likes to see the result. 
This can be done without changing the counter \comm{plotFigures}.
Simply one can add the term \comm{Now} to any command provided 
to include figures. Therefore the following command would show the 
figure and not the framebox:
%
\begin{arrangedWrapFigure}[15]{l}{3cm}{1}{2}{arranged511w}{top sub-figure}
    \subFig[]{figures/1}%
    \subFig[]{figures/2}%
    \subFig[]{figures/3}%
    \subFig[]{figures/4}%
\newSubFig{arranged512w}{bottom sub-figure}
    \subFig[]{figures/1}%
    \subFig[]{figures/2}%
    \subFig[]{figures/3}%
    \subFig[]{figures/4}%
\end{arrangedWrapFigure}

\begin{verbatim}
\setcounter{plotFigures}{0}
\includeEpsWrapNow{l}{figures/wrapMe}
{Example of an wrapped figure}
{wrappedFigure3}{1}
\setcounter{plotFigures}{1}
\end{verbatim}

To customize the figure there is the width which is added to the wrapped figure
in oder to have some space between figure and text

\begin{verbatim}
\setlength{\extraWrapWidth}{2mm}
\end{verbatim}

The space above the wrapped figure is set to be

\begin{verbatim}
\setlength{\aboveWrapFigureSpace}{0mm}
\end{verbatim}

The space below the wrapped figure can be set, obviously, by setting
the number of \comm{linnes} in the optional parameter.








%---------------------------------------------------------------- 
\section{Including wrapped-arranged figures}


To include arranged figures in an wrapped figure environment one has to use the 
following environment, \comm{arrangedWrapFigure}

\begin{verbatim}
\begin{arrangedWrapFigure}[15]{l}{3cm}{1}{2}
                {arranged511w}{top sub-figure}
    \subFig[]{figures/1}%
    \subFig[]{figures/2}%
    \subFig[]{figures/3}%
    \subFig[]{figures/4}%
\newSubFig{arranged512w}{bottom sub-figure}
    \subFig[]{figures/1}%
    \subFig[]{figures/2}%
    \subFig[]{figures/3}%
    \subFig[]{figures/4}%
\end{arrangedWrapFigure}
\end{verbatim}

The general form is given by

\begin{verbatim}
\begin{arrangedWrapFigure}[lines]{position}{width}
                        {sub-columns}{sub-sub-columns}
                        {label}{caption}
\end{arrangedWrapFigure}
\end{verbatim}









\clearpage









%---------------------------------------------------------------- 
\section{Customizing arranged figures}

As for the single figures the widths of the arranged-figure and
arranged-side-ways-figure environments are given by

\begin{verbatim} 
\setlength{\myStandardFigureWidth}{\linewidth}
\setlength{\myStandardSideWaysFigureWidth}{\textheight}
\end{verbatim}

For cutomizing purposes one should know that the space between 
each sub-figure is specified by

\begin{verbatim}
\setlength{\interSubFigSpace}{8mm}
\end{verbatim}

whereas the space between each sub-sub-figure is specified by

\begin{verbatim}
\setlength{\interSubSubFigSpace}{2.5mm}
\end{verbatim}

The horizontal positioning of the numbers of sub-sub-figures, 
\emph{(a)-(..)}, can be changed by changing the following length 
which default value is

\begin{verbatim}
\setlength{\subCaptionPenalty}{0mm}
\end{verbatim}

This has only an effect if the \comm{subFig} command is issued with 
an empty optional parameter. Otherwise the subfigure package's default
value will be assumed: \verb+\renewcommand{\subfigcapmargin}{10pt}+.

If a sub-sub-caption is given as in the following example

\begin{verbatim}
\begin{arrangedFigure}{2}{2}{arranged511}{left sub-figure}
    \subFig[]{figures/1}%
    \subFig[]{figures/2}%
    \subFig[This is a very long sub-sub-caption]{figures/3}%
    \subFig[]{figures/4}%
\newSubFig{arranged512}{right sub-figure}
    \subFig[]{figures/1}%
    \subFig[]{figures/2}%
    \subFig[]{figures/3}%
    \subFig[]{figures/4}%
\end{arrangedFigure}
\end{verbatim}

\begin{arrangedFigure}{2}{2}{arranged511}{left sub-figure}
    \subFig[]{figures/1}%
    \subFig[]{figures/2}%
    \subFig[This is a very long sub-sub-caption]{figures/3}%
    \subFig[]{figures/4}%
\newSubFig{arranged512}{right sub-figure}
    \subFig[]{figures/1}%
    \subFig[]{figures/2}%
    \subFig[]{figures/3}%
    \subFig[]{figures/4}%
\newSubFig{arranged521}{left sub-figure}
    \subFig[]{figures/1}%
    \subFig[]{figures/2}%
    \subFig[This is a very long sub-sub-caption]{figures/3}%
    \subFig[]{figures/4}%
\newSubFig{arranged522}{right sub-figure}
    \subFig[]{figures/1}%
    \subFig[]{figures/2}%
    \subFig[]{figures/3}%
    \subFig[]{figures/4}%
\setlength{\subFigureAboveCaptionSpace}{7mm}
\end{arrangedFigure}
\setlength{\subFigureAboveCaptionSpace}{0mm}


then the following length can be set in order to compensate for the extra space

\begin{verbatim}
\begin{arrangedFigure}{2}{2}{arranged51}{left sub-figure}
    \subFig[]{figures/1}%
    \subFig[]{figures/2}%
    \subFig[This is a very long sub-sub-caption]{figures/3}%
    \subFig[]{figures/4}%
\newSubFig{arranged52}{right sub-figure}
    \subFig[]{figures/1}%
    \subFig[]{figures/2}%
    \subFig[]{figures/3}%
    \subFig[]{figures/4}%
\setlength{\subFigureAboveCaptionSpace}{7mm}        % watch this
\end{arrangedFigure}
\setlength{\subFigureAboveCaptionSpace}{0mm}        % watch this
\end{verbatim}


with a resulting Figure \ref{fig:arranged521} and \ref{fig:arranged522}.

If the figure looks like Figure \ref{fig:arranged5311} and \ref{fig:arranged5312} 
then one could set

\begin{verbatim}
\begin{arrangedFigure}{2}{2}{arranged531}{left sub-figure}
    \subFig[]{figures/1}%
    \subFig[This is a very long sub-sub-caption]{figures/2}%
    \subFig[]{figures/3}%
    \subFig[]{figures/4}%
\newSubFig{arranged532}{right sub-figure}
    \setlength{\horizontalSubSubFigSpace}{4mm}      % watch this
    \subFig[]{figures/1}%
    \subFig[]{figures/2}%
    \setlength{\horizontalSubSubFigSpace}{-1mm}      % watch this
    \subFig[]{figures/3}%
    \subFig[]{figures/4}%
\end{arrangedFigure}
\end{verbatim}

to achieve Figure \ref{fig:arranged531} and \ref{fig:arranged532}.



\begin{arrangedFigure}{2}{2}{arranged5311}{left sub-figure}
    \subFig[]{figures/1}%
    \subFig[This is a very long sub-sub-caption]{figures/2}%
    \subFig[]{figures/3}%
    \subFig[]{figures/4}%
\newSubFig{arranged5312}{right sub-figure}
    \subFig[]{figures/1}%
    \subFig[]{figures/2}%
    \subFig[]{figures/3}%
    \subFig[]{figures/4}%
\newSubFig{arranged531}{left sub-figure}
    \subFig[]{figures/1}%
    \subFig[This is a very long sub-sub-caption]{figures/2}%
    \subFig[]{figures/3}%
    \subFig[]{figures/4}%
\newSubFig{arranged532}{right sub-figure}
    \setlength{\horizontalSubSubFigSpace}{4mm}
    \subFig[]{figures/1}%
    \subFig[]{figures/2}%
    \setlength{\horizontalSubSubFigSpace}{-1mm}
    \subFig[]{figures/3}%
    \subFig[]{figures/4}%
\end{arrangedFigure}



Another parameter for modifying the look is given by

\begin{verbatim}
\setlength{\subFigureBelowCaptionSpace}{5mm}
\end{verbatim}

which sets the space below the main sub-figure caption and the next 
main sub-figure below it.

After all, it might be difficult to reset every variable on its 
own and to remenber their default values. Therefore, the 
command \comm{resetMiniPlot} has been introduced. This 
sets variables and commands to their default values. Note 
that the counter \comm{plotFigures} is not reset. A listing
of the definition is given below.


\begin{verbatim}
\newcommand{\resetMiniPlot}%
    {%
    \setcounter{printCaption}{1}%
    \setcounter{scaleToStandardFigureWidth}{1}%
    \setcounter{isFirstStandardFigure}{1}%
    \setcounter{scaleFigureIndividual}{1}%
    \setlength{\extraWrapWidth}{2mm}%    
    \setlength{\aboveWrapFigureSpace}{0mm}%
    \setlength{\frameBoxPenalty}{2.4mm}%
    \setlength{\myStandardFigureWidth}{\linewidth}%  
    \setlength{\myStandardSideWaysFigureWidth}{\textheight}%
    \setlength{\subFigPenalty}{0mm}%            
    \setlength{\subSubFigPenalty}{0.001mm}%
    \setlength{\interSubFigSpace}{8mm}%
    \setlength{\interSubSubFigSpace}{2.5mm}%
    \setlength{\subFigureAboveCaptionSpaceDefaultOffset}{0mm}%
    \setlength{\subFigureAboveCaptionSpaceDefault}{0mm}%
    \setlength{\subFigureAboveCaptionSpace}
                    {\subFigureAboveCaptionSpaceDefault
                    + \subFigureAboveCaptionSpaceDefaultOffset}%
    \setlength{\horizontalSubSubFigSpace}{0mm}%
    \setlength{\subFigureBelowCaptionSpace}{3mm}%
    \setlength{\subCaptionPenalty}{0mm}%
    \setlength{\minimumCaptionWidth}{\linewidth * 4 / 6}%
    \setlength{\minimumSideWaysCaptionWidth}{\textheight * 4 / 6}%
    \renewcommand{\standardGraphScale}{1}% 
    \renewcommand{\isFirstSubFig}{false}%
    }%
\end{verbatim}

With this one could define a function to set one's own default values, e.g.

\begin{verbatim}
\necommand{\myResetMiniPlot}%
    {%
    \resetMiniPlot%
    %
    % personal settings:
    \setlength{\myStandardFigureWidth}{\linewidth - 20mm}%  
    }
\end{verbatim}

\clearpage






%---------------------------------------------------------------- 
\section{Debugging arranged figures}

Normally it shouldn't be necessary to do any debugging but in case there is
I'd like to mention some tweaks here.

If unluky one might find something as shown in Figure \ref{fig:arranged51d} and
\ref{fig:arranged52d}.

\setlength{\subSubFigPenalty}{-1mm} 

\begin{verbatim}
\begin{arrangedFigure}{2}{2}{arranged51d}{left sub-figure}
    \subFig[]{figures/1}%
    \subFig[]{figures/2}%
    \subFig[]{figures/3}%
    \subFig[]{figures/4}%
\newSubFig{arranged52d}{right sub-figure}
    \subFig[]{figures/1}%
    \subFig[]{figures/2}%
    \subFig[]{figures/3}%
    \subFig[]{figures/4}%
\end{arrangedFigure}
\end{verbatim}

\begin{arrangedFigure}{2}{2}{arranged51d}{left sub-figure}
    \subFig[]{figures/1}%
    \subFig[]{figures/2}%
    \subFig[]{figures/3}%
    \subFig[]{figures/4}%
\newSubFig{arranged52d}{right sub-figure}
    \subFig[]{figures/1}%
    \subFig[]{figures/2}%
    \subFig[]{figures/3}%
    \subFig[]{figures/4}%
\end{arrangedFigure}

From the coding shown above one would expect to yield two columns of sub-figures and
inside those again two columns of sub-sub-figures. However, Figure
\ref{fig:arranged51} to \ref{fig:arranged51} have only one column of
sub-sub-figures. This is because the sub-sub-figures plus the space
between them is too wide to fit into the width of the minipage environment
they are in. So, the figures are shifted down. This effect can be removed
by changing the following length which default value is

\begin{verbatim}
\setlength{\subSubFigPenalty}{0.001mm}
\end{verbatim}

\setlength{\subSubFigPenalty}{0.001mm}

Theoretical this value should be zero but due to numerical inaccuracies the 
sub-sub-figures have to be made a little bit narrower. The above length is to 
compensate for the numerical error.

By increasing the above length one can yield a satisfactory figure layout as 
can be seen in Figure \ref{fig:arranged51dd} and \ref{fig:arranged52dd}.

\begin{arrangedFigure}{2}{2}{arranged51dd}{left sub-figure}
    \subFig[]{figures/1}%
    \subFig[]{figures/2}%
    \subFig[]{figures/3}%
    \subFig[]{figures/4}%
\newSubFig{arranged52dd}{right sub-figure}
    \subFig[]{figures/1}%
    \subFig[]{figures/2}%
    \subFig[]{figures/3}%
    \subFig[]{figures/4}%
\end{arrangedFigure}


Another case which is unlikely but not impossible to occur is that one has 
the displacment of the sub-sub-figures as described above but 
only when \comm{plotFigures} is equals zeros. That means the actual figures
are alright but the frameboxes are shifted down. Then it is that 
the following length needs to be increased from its default value

\begin{verbatim}
\setlength{\frameBoxPenalty}{2.4mm}
\end{verbatim}

The reason is that the framebox is drawn around the actual size of the 
eps-figure, this space needs to be substracted again in order to obtain the
actual figure size.


\clearpage







%---------------------------------------------------------------- 
\section{Customizing content of substituted framed boxes}

If the counter \comm{plotFigures} is set to be equals \comm{0} then a framed 
box with some information is shown. The commands to show information, however can 
be redefined in order to be more customizable for the user.

Three commands are used to disply the content for the different environments:

\begin{itemize}
  \item \verb+\epsContent[placement]{path}{caption}{label}{scaling}+\par\verb+{figure width}{figure height}+
  \item \verb+\arrangedFigContent[sub-sub-caption]{path}{figure width}+\par\verb+{figure height}+
  \item \verb+\wrappedEpsContent[lines]{position}{path}{caption}+\par\verb+{label}{scaling}{figure width}{figure height}+
\end{itemize}

The first command prints the content to the single eps figures. First the \verb+\frameBoxPenaly+ is substracted
from the figures height and width. Then a picture environment is created within a \comm{framebox}. The information 
is contained within a table. More information could be added or the whole structure completely renewed.

\begin{itemize}
  \item file name with path and extension
  \item scaling factor for this eps figure
  \item label of that figure
  \item position in terms of \comm{htbp}
\end{itemize}

\begin{verbatim}
\newcommand{\epsContent}[7][htpb]%
    {%
    \setlength{\figWidth}{#6 - \frameBoxPenalty}%
    \setlength{\figHeight}{#7 - \frameBoxPenalty}%
    %
    \setlength{\figTabWidth}{\figWidth - 31mm}%
    \setlength{\figTabMiddle}{\figHeight / 2}%
    %
    \centerline{\framebox{%
        %
        \setlength{\unitlength}{1sp}%%
        \begin{picture}(\number\figWidth,\number\figHeight)%%
            \put(655360,\number\figTabMiddle)%%
                {%
                \begin{tabular}{lp{\the\figTabWidth}}%
                    \emph{Source:} & #2.eps \\%%
                    \emph{Scaling:} & #5 \\%%
                    \emph{Label:} & \labelPrefix#4 \\%%
                    \emph{Position:} & #1%%
                \end{tabular}%
                }%    
        \end{picture}%%
        }}%
    }%
\end{verbatim}

The second command is to include the content for arranged figures. The structure is
the same as in the above command. Because there is less space for arranged figures the 
content had to be shortend. The filename is shown without path and extension. Furthermore,
there is nor description what the information is. Therefore this is listed here

\begin{itemize}
  \item base name of file without path and extension
  \item scaling factor for this eps figure
  \item suffix to be attached to the base label of that figure to refence this sub-sub-figure
\end{itemize}

\begin{verbatim}
\newcommand{\arrangedFigContent}[4][noParMiniPlot]%
    {%
    \setlength{\figWidth}{#3 - \frameBoxPenalty}%
    \setlength{\figHeight}{#4 - \frameBoxPenalty}%
    %
    \setlength{\figTabMiddle}{\figHeight / 2}%
    %
    \framebox{%
        \setlength{\unitlength}{1sp}%
        \tiny%
        \begin{picture}(\number\figWidth,\number\figHeight)%
        \put(0,\number\figTabMiddle)%%
            {%
            \begin{tabular}{p{\the\figWidth}}%
                \extractminiplotpath{#2} \\%%
                \subFigScale \\%%
                \subLabelSuffix%%
            \end{tabular}%
            }%
        \end{picture}%    
        }%
    }%
\end{verbatim}

The third and last command to determine the content of substituted figures is the one which can 
be seen in single-eps wrapped figures. Since I expect a wrapped figure rather high than wide I
rotate the shown information by \comm{90} degrees. The information is 

\begin{itemize}
  \item base name of file without path but with extension
  \item scaling factor for this eps figure
  \item label to this figure
\end{itemize}

\begin{verbatim}
\newcommand{\wrappedEpsContent}[8][noPar]%
    {%
    \setlength{\figWidth}{#7 - \frameBoxPenalty}%
    \setlength{\figHeight}{#8 - \frameBoxPenalty}%
    %
    \setlength{\figTabMiddle}{\figWidth / 2 - 6.6mm}%
    %
    \centerline{\framebox{%
        %
        \setlength{\unitlength}{1sp}%%
        \begin{picture}(\number\figWidth,\number\figHeight)%%
            \scriptsize%
            \put(\number\figTabMiddle,65536)%%
                {%
                \rotatebox{90}{%
                    \begin{tabular}{p{\the\figHeight}}%
                        \extractminiplotpath{#3}.eps \\%%
                        #6 (scale) \\%%
                        \labelPrefix#5 \\%%
                    \end{tabular}%
                    }%
                }%
        \end{picture}%%
        }}%
    }%
\end{verbatim}







%---------------------------------------------------------------- 
\section{Referencing to figures}

When passing on a lebal to a MiniPlot-function or environment it is not
necessary to write \comm{fig:} or something else. When citing the reference,
however, one needs to write that prefix. The prefix is defined by

\begin{verbatim}
\newcommand{\labelPrefix}{fig:}
\end{verbatim}

Suppose the label given to a figure is \comm{arranged51dd} then the actual label
assign is \comm{fig:arranged51dd} and that is what needs to be referenced to

\begin{verbatim}
... see Figure \ref{fig:arranged51dd} ...
\end{verbatim}

... see Figure \ref{fig:arranged51dd}...\smallskip

Referencing to sub-sub-figures is implemented. The subfigures are labeled by, first the
label prefix, second the base label given by the user and third the position in that specific
sub-figure given by \comm{-a}, \comm{-b}, ... .

\begin{verbatim}
\label{\labelPrefix\myLabel-\alph{subfigure}}
\end{verbatim}

The reference can be called by simply typing

\begin{verbatim}
... see Figure \ref{fig:arranged51dd-c} ...
\end{verbatim}

... see Figure \ref{fig:arranged51dd-c}...\smallskip

The format of the above reference can be changed by redefining the subfigure-packages definition.









%---------------------------------------------------------------- 
\section{Recommendations for working with MiniPlot}

The only recommendation there is is only to set

\begin{verbatim}
\usepackage{miniplot}
\setcounter{plotFigures}{0}
\end{verbatim}

This will display only the frameboxes of the figures and this will
increase browsing speed through the document significantly.

When implementing a figure for the first time one should use the
\comm{\~Now}-version of the implementation command, for example

\begin{verbatim}
\includeEpsNow{...
\end{verbatim}

After the figure is set and everything works fine one should remove the 
\comm{Now} to end up with

\begin{verbatim}
\includeEps{...
\end{verbatim}

When the whole document is finished one simply sets

\begin{verbatim}
\setcounter{plotFigures}{1}
\end{verbatim}

and all the figures will be shown.

Another subject regarding the subfig package is that I would suggest to use
the following settings when implementing loads of figures in an arranged figure
environment like Figure \ref{fig:arranged31} to \ref{fig:arranged33}.

\begin{verbatim}
\renewcommand{\subfigbottomskip}{0pt}%
\renewcommand{\subfigcapskip}{0pt}%
\renewcommand{\subcapsize}{\scriptsize}%
\end{verbatim}

These settings are not used in this manual.







%---------------------------------------------------------------- 
\section{The code}

I'd like to excuse that the code is not pasted in and not explained as 
proper package writers tend to do. I just hope that the comments in 
\comm{miniplot.sty} might give some clue when trying to understand 
pieces of it.




%---------------------------------------------------------------- 
\section{Restrictions when using MiniPlot}

One restriction is that the underscore \_ is not desired in the
labels of the figures. 

This is due to to the fact that MiniPlot displays
the labels and needs to find a proper way of doing this. This is done by the following 
command.

\begin{verbatim}
\newcommand{\printLabel}[1]{\centerline{
                        \small{\emph{Label}: $\mathtt{fig:#1}$}}}
\end{verbatim}


In mathmode the underscore has a different function and is not displayed as an underscore. 
I could imagine there is a solution like using the \verb+\let+ command
but this goes beyond my \LaTeX \hspace{0mm} capabilities.

Another restriction is that the path to an arranged figure cannot contain the \verb+\+-sign. 
This might not be a major restriction. If it is let me know.

Finally, when including a \verb+\subFig[sub-sub-caption]{...+ the sub-sub-caption cannot 
be equals \comm{noParMiniPlot}. This shouldn't be too much of a problem.

The number of sub-sub-figures per sub-figure is restriced by the subfigure package to maximum 
26 sub-sub-figures. If one needs to have more sub-sub-figures one can redefine

\begin{verbatim}
\def\thesubfigure{(\arabic{subfigure})}
\renewcommand{\subLabelSuffix}{-\arabic{subfigure}}
\end{verbatim}

Then the sub-sub-figure caption and the sub-label is no more \comm{(a)}, \comm{(b)}, ... but \comm{(1)}, \comm{(2)}, ... .
This allows for more than 26 sub-sub-figures. If one doesn't like the arabic numbering maybe
someone can come up with an algorithm which counts up to \comm{(z)} and then starts with \comm{(aa)} and so on.



%---------------------------------------------------------------- 
\section{Known bugs}

There is actually just one real bug known. That is that an overhang for an
wrapped-arranged-figure cannot be applied properly.

Generally bugs can please be reported to \emph{enxtw1@nottingham.ac.uk} or 
\emph{tobias\_wahl@gmx.de}.

One thing is that one gets a lot of warnings about overfull hboxes. I am 
personally not really bothered but if somebody might come up with an idea
I would be happy to implement it.

There are just few awkward things about the code and maybe somebody has an idea
how solve things. One thing is that in the following 
function the first line seemed\footnote{I have to note that this behavior happened at one point and 
does not occur annymore. So that \comm{subFigureAboveCaptionSpaceDefaultOffset}
is set to zero. However, I decided to leave this in the code in case it might 
occur again??} to introduce a vertical length even if one
simply writes \comm{vspace\{0pt\}\%}. Due to this strange behavior I had to 
introduce the length \comm{subFigureAboveCaptionSpaceDefaultOffset} equals
3mm.


\begin{verbatim}
\newcommand{\newSubFig}[3][\subFigureAboveCaptionSpaceDefault]%
    {%
    \vspace{\subFigureAboveCaptionSpace}%
    \ifthenelse{\value{plotFigures}>0}%
        {\caption{\myCaptionText}}%
        {\caption{\myCaptionText\printLabel{\myLabel}}}%
    \label{\myLabel}%
    \end{minipage}%
    \renewcommand{\myCaptionText}{#3}%                                                   
    \renewcommand{\myLabel}{#2}%                                                                
    \vspace{\subFigureBelowCaptionSpace}%
    \hspace{\interSubFigSpace}%
    \begin{minipage}{\myStandardSubFigureWidth}%
    \aboveCaptionSpace{#1}%
    }%
\end{verbatim}

Furthermore, in the above command the following command is called

\begin{verbatim}
\newcommand{\aboveCaptionSpace}[1]{\setlength{\subFigureAboveCaptionSpace}
                            {#1 + \subFigureAboveCaptionSpaceDefaultOffset}}
\end{verbatim}

For some strange reason the length \comm{subFigureAboveCaptionSpace} does not
keep its value. Maybe I need to have a look at this again. 

The command \comm{aboveCaptionSpace} was initially thought to be used instead of setting the
length \comm{subFigureAboveCaptionSpace} directly.

In the \verb+\subFig+ command above it was thought to be a replacement of

\begin{verbatim}
    .
    \setlength{\subFigureAboveCaptionSpace}{2.5mm}
\newSubFig{arranged553}{bottom}
    \setlength{\subFigureAboveCaptionSpace}{0mm}
    .
\end{verbatim}

by typing

\begin{verbatim}
    .
\newSubFig[2.5mm]{arranged553}{bottom}
    .
\end{verbatim}

However, maybe it will work one time. Until then one shouldn't use this 
optional parameter.


\clearpage









%---------------------------------------------------------------- 
\section{Recources}

One can obtain this and the required package from the following ftp-sites

\begin{itemize}
  \item ftp://ftp.dante.de/tex-archive/macros/latex/contrib/supported/miniplot
  \item ftp://ftp.dante.de/tex-archive/macros/latex/contrib/supported/subfigure
  \item ftp://ftp.dante.de/tex-archive/macros/latex/contrib/other/misc/wrapfig.sty
  \item ftp://ftp.dante.de/tex-archive/macros/latex/contrib/supported/rotating
  \item ftp://ftp.dante.de/tex-archive/macros/latex/contrib/supported/rotfloat
\end{itemize}

or if only the basic sty-files are needed one can obtain them from 

\begin{itemize}
  \item ftp://miniplot@128.243.122.176
\end{itemize}

The required username and password is \comm{miniplot}.







%---------------------------------------------------------------- 
\section{List of variables, commands and environments}


Note that only those variables, commands and environments are cited which
are useful for the user.


\bigskip\bigskip

\underline{Counters:}

\begin{verbatim}
\setcounter{scaleToStandardFigureWidth}{1}
\setcounter{isFirstStandardFigure}{1}
\setcounter{scaleFigureIndividual}{1}
\setcounter{plotFigures}{1}
\end{verbatim}



\bigskip\bigskip

\underline{Lengths:}

\begin{verbatim}
\setlength{\myStandardFigureWidth}{\linewidth}
\setlength{\myStandardSideWaysFigureWidth}{\textheight}

\setlength{\captionWidthPenalty}{0mm}
\setlength{\minimumCaptionWidth}{\linewidth * 4 / 6}
\setlength{\minimumSideWaysCaptionWidth}{\textheight * 4 / 6}

\setlength{\aboveWrapFigureSpace}{0mm}

\setlength{\interSubFigSpace}{8mm}
\setlength{\interSubSubFigSpace}{2.5mm}

\setlength{\horizontalSubSubFigSpace}{-1mm}
\setlength{\subFigureAboveCaptionSpaceDefault}{0mm}
\setlength{\subFigureAboveCaptionSpace}{0mm}
\setlength{\subFigureBelowCaptionSpace}{5mm}

\setlength{\subFigPenalty}{0mm}
\setlength{\subSubFigPenalty}{0.001mm}
\setlength{\subCaptionPenalty}{0mm}
\setlength{\frameBoxPenalty}{2.4mm}
\setlength{\labelPenalty}{-1.5mm}
\end{verbatim}



\bigskip\bigskip

\underline{Commands:}

\begin{verbatim}
\newcommand{\labelPrefix}{fig:}
\newcommand{\subLabelSuffix}{-\alph{subfigure}}

\newcommand{\isFirstSubFig}{false}
\newcommand{\standardGraphScale}{1}

\includeEps[htpb]{path}{caption}{label}{scaling}
\includeEpsNow[htpb]{path}{caption}{label}{scaling}
\includeSideWaysEps[htpb]{path}{caption}{label}{scaling}
\includeSideWaysEpsNow[htpb]{path}{caption}{label}{scaling}

\includeStandardGraph[htpb]{path}{caption}{label}
\includeStandardGraphNow[htpb]{path}{caption}{label}
\includeStandardSideWaysGraph[htpb]{path}{caption}{label}
\includeStandardSideWaysGraphNow[htpb]{path}{caption}{label}

\includeEpsWrap[narrow lines]{position}{path}{caption}{label}{scaling}
\includeEpsWrapNow[narrow lines]{position}{path}{caption}{label}{scaling}

\epsContent[placement]{path}{caption}{label}{scaling}{figure width}
                                                          {figure height}
\arrangedFigContent[sub-sub-caption]{path}{figure width}{figure height}
\wrappedEpsContent[lines]{position}{path}{caption}{label}{scaling}
                                            {figure width}{figure height}

\newcommand{\printLabel}[1]{\centerline{
                            \small{\emph{Label}: $\mathtt{fig:#1}$}}}
\newSubFig[\subFigureAboveCaptionSpace]{label}{caption}
\subFig[sub-sub-caption]{path}%

\resetMiniPlot

\epsContent[placement]{path}{caption}{label}{scaling}{figure width}
                                                            {figure height}
\arrangedFigContent[sub-sub-caption]{path}{figure width}{figure height}
\wrappedEpsContent[lines]{position}{path}{caption}{label}{scaling}
                                                {figure width}{figure height}
\end{verbatim}


\bigskip\bigskip

\underline{Environments:}

\begin{verbatim}
arrangedFigure[htpb]{sub-fig columns}{sub-sub-fig columns}{label}{caption}
arrangedFigureNow[htpb]{sub-fig columns}{sub-sub-fig columns}{label}{caption}

arrangedSideWaysFigure[htpb]{sub-fig columns}{sub-sub-fig columns}
                                                            {label}{caption}
arrangedSideWaysFigureNow[htpb]{sub-fig columns}{sub-sub-fig columns}
                                                            {label}{caption}
\end{verbatim}                                                          


\end{document}
