%% This is file cjetemplate.tex
%% 2017/04/17 v1.01
%% =========
%%
%% COPYRIGHT (c) 2017 Canadian Economics Association
%% 
%% LICENSE
%% =======
%%
%% This work can be distributed and/or modified under the conditions of the LaTeX Project Public License, 
%% either version 1.3 of this license or any later version. The latest version of this license is available at 
%% latex-project.org/lppl.txt. Version 1.3, or later, is part of all distributions of LaTeX version 2005/12/01 or later.


\NeedsTeXFormat{LaTeX2e}

  \documentclass{cje}          % for authors
% \documentclass[review]{cje}  % for reviewers 
%                              % double spaced with line numbers
% \documentclass[proof]{cje}   % for typesetters 
%                              % with line numbers and typesetting details
%   typesetters to add the following data for [proof] version (these are the defaults):
%   \pubyear{20XX} \volume{00} \issueno{0}                         %(line1)
%   \monthyearen{January 20XX} \monthyearfr{Janvier 20XX}          %(line2)
%   \issn{0000-0000}           % note - page range generated from labels: 
%                              \label{firstpage}--\label{lastpage} %(line3)
%   \articleref{00000} \dispatch{01.01.20XX} \noofpages{00}        %(line4)

  \usepackage{cjenatbib}
  \usepackage{url}
   
% check if we are compiling under latex or pdflatex
  \ifx\pdftexversion\undefined
    \usepackage[dvips]{graphicx}
  \else
    \usepackage[pdftex]{graphicx}
    \usepackage{epstopdf}
    \epstopdfsetup{suffix=}
  \fi

  \usepackage{tabularx}
  \usepackage[figuresright]{rotating}
  \usepackage{floatpag}
    \rotfloatpagestyle{empty}
  \usepackage{amsmath}
  \usepackage{amsthm}
  \theoremstyle{plain}% default
    \newtheorem{theorem}{Theorem}
    \newtheorem{lemma}{Lemma}
    \newtheorem{proposition}{Proposition}
    \newtheorem*{corollary}{Corollary}
  \theoremstyle{definition}
    \newtheorem{definition}{Definition}
    \newtheorem{example}{Example}
  \theoremstyle{remark}
    \newtheorem*{remark}{Remark}
    \newtheorem*{case}{Case}

  \usepackage{txfonts}% times font - you can use any Times font 
  \usepackage{upquote}
 \usepackage{hyperref}
\def\citeapos#1{\citeauthor{#1}'s (\citeyear{#1})}
  \hypersetup{%
    pdftitle = {Guide to Canadian Journal of Economics \LaTeXe\ class file (v. 1.01, 2017)},
    pdfauthor = {Author One, Author Two, Author Three, 
      Author Four and Author Five},
    citecolor=blue,            
    urlcolor=blue,
    colorlinks = true,
  }  	  

  \bibpunct{(}{)}{,}{a}{}{;}

\begin{document}
\label{firstpage}

% shortened version of title [in square brackets] for running head
\title[Guide to \texttt{cje} \LaTeXe\ class]{Guide to Canadian Journal of Economics \LaTeXe\ class file (v. 1.01, 2017)}

% summary of authors for running head
\authors{A. One, A. Two, A. Three, A. Four and A. Five}

% authors and affiliations
\authorone{Author One}{Federal Reserve Bank of Boston}
\authortwo{Author Two}{Division of Economics, Western 
  Technological University}
\authorthree{Author Three and Author Four}{Department 
  of Economics, Peterson University}
\authorfour{Author Five}{Boston School of Economics and 
  Political Science}

\abstract{The \LaTeXe\ class file for the \emph{Canadian Journal of Economics} is derived from \texttt{article.cls}. This guide provides you with all the information you need to use \texttt{cje.cls}. Before submitting your manuscript, please be sure your source files compile without any errors. The journal uses Times New Roman and is printed in black and white, so be sure any colour graphics will appear correctly in black and white or also provide black and white versions (colour graphics will appear in colour in the online version). Include an English and, if available, a French abstract of no more than 160 words, if possible. If you don't have a French version of your abstract, the English abstract will be translated by Gilles Paquet, Professor Emeritus at the University of Ottawa. Our thanks to Professor Paquet for his ongoing devotion to this task. If you have any questions or would like to signal any corrections to the class file or this guide, please contact Kim Nesbitt, CJE copyeditor, at \texttt{kimnesbitt@videotron.ca}.} 

\resume{The English abstract will be translated into French and inserted here.}

\JEL{H77, Q56, H41}

\acknowledgements{The 
  Canadian Economics Association wishes to thank Patrick W. Daly 
  for making the development of \texttt{cjebibstyle.bst} possible and Ali Woollatt for her work on creating this package and documentation. Thank you to Karl Berry, James S. Hefferon and the friendly and helpful members of the \TeX\ Users Group (TUG) for their technical assistance, guidance and advice.\\
  Corresponding author: Kim Nesbitt, kimnesbitt@videotron.ca}
  
\maketitle

\section{Using the \texttt{cje} class file}
Download the latest version from \href{http://economics.ca/cje/en/LaTeX.php}{economics.ca/cje/en/LaTeX.php} or \href{http://ctan.org/}{ctan.org}. This guide has examples of most environments you're likely to need. 

\subsection{\texttt{cje.cls} package files}
The distribution package contains the following nine files.\\[0.5\baselineskip]
\begin{tabular}{@{}ll}
\texttt{readme.txt}        & readme file\\
\texttt{cjetemplate.tex}      & Instructions for authors, written in CJE style\\
\texttt{cjetemplate.pdf}      & PDF of Instructions for authors, written in CJE style\\
\texttt{canadian-flag.eps} & figures~1 and 2, required for cjetemplate.tex\\
\texttt{cje.cls}           & CJE class file\\
\texttt{cjebibstyle.bst}           & CJE bibliography style file\\
\texttt{cjenatbib.sty}     & CJE style file for citations\\
\texttt{cjeupmath.sty}     & CJE style file for non-italic Greek characters\\
\texttt{ageingbib.bib}     & sample \textsc{Bib}\TeX\ database\\ 
\mbox{}\\
\end{tabular}

\subsection{Additional required files}
The following standard files are required for this package. They can be obtained from CTAN if you don't already have them installed.\\[0.5\baselineskip]
\begin{tabular}{@{}ll}
\texttt{amsbsy.sty}        & style file called in by \texttt{cjeupmath.sty}\\
\texttt{amsgen.sty}        & style file called in by \texttt{cjeupmath.sty}\\
\texttt{amssymb.sty}       & accesses AMS fonts \texttt{msam} and \texttt{msbm}\\
\texttt{amsfonts.sty}      & style file called in by \texttt{amssymb.sty}\\
\texttt{amsthm.sty}        & for typesetting theorems, proofs, etc.\\
\texttt{amsthdoc.pdf}      & documentation for \texttt{amsthm.sty}\\
\texttt{lineno.sty}        & style file required for \verb"[review]" and \verb"[proof]" options\\
\texttt{ednmath0.sty}      & style file called in by \texttt{lineno.sty}\\
\texttt{edtable.sty}       & style file called in by \texttt{lineno.sty}\\
\texttt{vplref.sty}        & style file called in by \texttt{lineno.sty}\\
\texttt{graphicx.sty}      & graphics style file\\
\texttt{hyperref.sty}      & to generate hypertext links\\
\texttt{upquote.sty}       & to generate upright quote marks in verbatim\\
\texttt{tabularx.sty}      & style file to manipulate columns in tables\\
\texttt{url.sty}           & for formatting hypertext links, etc.
\end{tabular}

\subsection{Typesetting the title page}

CJE uses shortened versions of the title and author names in the running heads. Typeset the shortened version of the title in square brackets immediately after the \verb"\title" command. Use the \verb"\authors" command  to define the shortened version of author names (see below for both). 

Typeset authors and affiliations using \verb"\authorone", \verb"\authortwo", \verb"\authorthree", etc.,  to \verb"\authorten" for authors one to 10. Please keep the title, author(s), abstract, r\'{e}sum\'{e}, JEL classification and acknowledgement sections in the same order as shown below, with \verb"\maketitle" being the last command. Two other class file options are also available: the review option, which can be used to create your initial manuscript for submission, and the proof option, which is used by the typesetters to create the article proof you will subsequently be asked to review and approve. This guide was set using the following code:
\begin{smallverbatim}
  \documentclass{cje}          % for authors
% \documentclass[review]{cje}  % for reviewers 
%                              % double spaced with line numbers
% \documentclass[proof]{cje}   % for typesetters 
%                              % with line numbers and typesetting details
%   typesetters to add the following data for [proof] version (these are the defaults):
%   \pubyear{20XX} \volume{00} \issueno{0}                         %(line1)
%   \monthyearen{January 20XX} \monthyearfr{Janvier 20XX}          %(line2)
%   \issn{0000-0000}           % note - page range generated from labels: 
%                              \label{firstpage}--\label{lastpage} %(line3)
%   \articleref{00000} \dispatch{01.01.20XX} \noofpages{00}        %(line4)

  \usepackage{cjenatbib}
  \usepackage{url}
  \usepackage{hyperref}

  \hypersetup{%
    pdftitle = {Guide to Canadian Journal of Economics \LaTeXe\ class file (v. 1.01, 2017)},
    pdfauthor = {Author One, Author Two, Author Three, 
      Author Four and Author Five},
    citecolor=blue,            
    urlcolor=blue,
    colorlinks = true,
  }  	  
  
% check if we are compiling under latex or pdflatex
  \ifx\pdftexversion\undefined
    \usepackage[dvips]{graphicx}
  \else
    \usepackage[pdftex]{graphicx}
    \usepackage{epstopdf}
    \epstopdfsetup{suffix=}
  \fi

  \usepackage{tabularx}
  \usepackage[figuresright]{rotating}
  \usepackage{floatpag}
    \rotfloatpagestyle{empty}
  \usepackage{amsmath}
  \usepackage{amsthm}
  \theoremstyle{plain}% default
    \newtheorem{theorem}{Theorem}
    \newtheorem{lemma}{Lemma}
    \newtheorem{proposition}{Proposition}
    \newtheorem*{corollary}{Corollary}
  \theoremstyle{definition}
    \newtheorem{definition}{Definition}
    \newtheorem{example}{Example}
  \theoremstyle{remark}
    \newtheorem*{remark}{Remark}
    \newtheorem*{case}{Case}

  \usepackage{txfonts}% times font - you may use any Times font 
  \usepackage{upquote}

  \bibpunct{(}{)}{;}{a}{}{;}

\begin{document}
\label{firstpage}

% shortened version of title [in square brackets] for running head
\title[Guide to \texttt{cje} \LaTeXe\ class file]{Guide to \texttt{cje} \LaTeXe\ class file}

% summary of authors for running head
\authors{Author One, Author Two, Author Three, 
      Author Four and Author Five}

% authors and affiliations
\authorone{Author One}{Federal Reserve Bank of Boston}
\authortwo{Author Two}{Division of Economics, Western 
  Technological University}
\authorthree{Author Three and Author Four}{Department 
  of Economics, Peterson University}
\authorfour{Autor Five}{Boston School of Economics and 
  Political Science}

\abstract{The \LaTeXe\ class file for the \emph{Canadian Journal of Economics}...
  If you have any questions or would like to signal any corrections to the 
  class file or this guide, please contact Kim Nesbitt, CJE copyeditor, at
   \texttt{kimnesbitt@videotron.ca}.}

\resume{The English abstract will be translated into French and inserted here.}

\JEL{H77, Q56, H41}

\acknowledgements{The 
  Canadian Economics Association wishes to thank Patrick W. Daly 
  for making the development of \texttt{cjebibstyle.bst} possible and Ali Woollatt for her work on creating this package and documentation. Thank you to Karl Berry, James S. Hefferon and the friendly and helpful members of the \TeX\ Users Group (TUG) for their technical assistance, guidance and advice.\\
 Corresponding author: Kim Nesbitt, kimnesbitt@videotron.ca}
  
\maketitle

\section{Using the \texttt{cje} class file}
\end{smallverbatim}

\subsection{User-defined macros}
If you define your own macros, ensure the names you give them do not conflict with any existing macros in plain \TeX\ or \LaTeXe. You can check if the macro name is already used by typing \verb"\show\<macro_name>". Place your macros in the preamble, i.e., between \verb"\documentclass" and \verb"\begin{document}".\footnote{Please be sure to remove (comment out) any macros you've created but didn't use in your article.}

\subsection{Lists}
The \texttt{cje} class file provides for numbered (\verb"enumerate") and unnumbered (\verb"itemize") lists. Use a numbered list in hierarchical or chronological lists, such as lists with elements of increasing or decreasing importance or where the sequence is important. Otherwise, use an itemized list. Be sure the structure of each element in the list is parallel (e.g., all begin with nouns or with verbs). The default numbering system for \verb"\begin{enumerate}..." \verb"\end{enumerate}" is Arabic numbers, beginning with ``1.'' Here is an example of a simple enumerate environment:
\begin{enumerate}
\item Here is the first item of the list.
\item Here is the second item of the list.
  \begin{enumerate}[(b)]
    \item If you require a sublist (like this one), you can create it with just a little extra coding. Determine which character, (a), (b), (c), etc., is the widest in the list. Then add this as an argument in square braces. See the verbatim text below for the coding.
    \item Here is the second sublist item.
  \end{enumerate}
\end{enumerate}
It was set using the following:
\begin{smallverbatim}
\begin{enumerate}
\item Here is the first item of the list.
\item Here is the second item of the list.
  \begin{enumerate}[(b)]
    \item If you require a sublist (like this one), you can create it with just....
    \item Here is the second sublist item.
  \end{enumerate}
\end{enumerate}
\end{smallverbatim}
%
You can change the default numbering to, for example, alpha characters for the top level items and Roman numerals for the sublist items. In the example below, (b) is the widest item label in the list and (ii) is the widest item label in the sublist, so they're placed in square braces following the \verb"\begin{enumerate}" command: 
\begin{enumerate}[(b)]
\usealpha
\item Here is the first item of the list.
\item Here is the second item of the list.
  \begin{enumerate}[(ii)]
  \useroman
    \item If you require a sublist (like this one), you can create it with just.... 
    \item Here is the second sublist item.
  \end{enumerate}
\end{enumerate}
The above list was set using the following code:
\begin{smallverbatim}
\begin{enumerate}[(b)]
\usealpha
  \item Here is the first item of the list.
  \item Here is the second item of the list.
    \begin{enumerate}[(ii)]
    \useroman
      \item If you require a sublist (like this one), you can create it with just.... 
      \item Here is the second sublist item.
    \end{enumerate}
\end{enumerate}
\end{smallverbatim}

Itemized lists are set off with bullet points. Sublists are set off with en dashes:
\begin{itemize}
\item Here is the first item of the list.
  \item Here is the second item of the list.
  \begin{itemize}
  \item Here is a sublist item.
  \end{itemize}
\end{itemize}


\subsection{Extracts and epigraphs}
Extracts such as this:
\begin{extract}
Lorem ipsum dolor sit amet, consectetur adipiscing elit, sed do eiusmod tempor incididunt ut labore et dolore magna aliqua. Ut enim ad minim veniam, quis nostrud exercitation ullamco laboris nisi ut aliquip ex ea commodo consequat. 
\end{extract}
are typeset using the following:
\begin{smallverbatim}
\begin{extract}
Lorem ipsum dolor sit amet, consectetur adipiscing elit, sed do eiusmod tempor incididunt ut labore.... 
\end{extract}
\end{smallverbatim}

Epigraphs, such as the one below, are slightly different:
\begin{epigraph}
The time to begin writing an article is when you have finished it to your satisfaction. By that time you begin to clearly and logically perceive what it is that you really want to say.
\epigraphauthor{\textit{Mark Twain's Notebook}, 1902--1903}
\end{epigraph}
and are typeset using:
\begin{smallverbatim}
\begin{epigraph}
The time to begin writing an article is when...
  ...what it is that you really want to say.
\epigraphauthor{\textit{Mark Twain's Notebook}, 1902--1903}
\end{epigraph}
\end{smallverbatim}

\subsection{Margin notes}

The \texttt{cje} class file redefines the \LaTeX\ command \verb"\marginpar". If you want to add a margin note such as the one alongside this text,\marginpar{Copyeditor: Use 2016 figure} type: \verb"\marginpar{Use 2016 figures}". The copyeditor or typesetters will remove all margin notes from the final proof.

\subsection{Tables}

If you use labels to refer to the tables, \verb"\caption" must precede \verb"\label". The CJE does not use vertical rules in tables. See section~\ref{landtables} for information on typesetting landscape tables. If your table has footnotes, you must use the \texttt{minipage} environment for them to be output in the correct position (see table~\ref{forecast}). The source code for table~\ref{forecast} is shown immediately below the table, reproduced courtesy of David Amdur and Eylem Ersal Kiziler. 

\begin{table}%table1
\caption{Forecast error variance decomposition for the US current-account-to-GDP 
  ratio (in percent)}
\label{forecast}
\begin{minipage}{28pc}% you need this line only if your table has notes and/or footnotes 
                      % (28pc is the text width)
\begin{tabular}{@{}lrrrr@{}}\hline
  Horizon/Shock\tablenote{NOTES: Do not include closing punctuation in table 
  titles. If a table has notes and/or footnotes, you must create the table 
  inside a \texttt{minipage} environment.} & 
  \multicolumn{1}{c}{$\epsilon_g^H$} & 
  \multicolumn{1}{c}{$\epsilon_g^F$} & 
  \multicolumn{1}{c}{$\epsilon_z^H$} & 
  \multicolumn{1}{c}{$\epsilon_z^F$}\\ \hline
  1 quarter   & 45.2 & 52.2 & 0.8 & 1.9\\
  4 quarters  & 41.4 & 54.9 & 0.8 & 2.8\\
  8 quarters  & 39.5 & 55.9 & 0.8 & 3.8\\
  16 quarters & 38.9 & 56.0 & 0.8 & 4.6\footnote{An example of a table footnote.}\\
\finalhline
\end{tabular}
\end{minipage}% you need this line only if your table has notes and/or footnotes
\vspace\baselineskip\hrule % to separate verbatim from table
\vspace\baselineskip
%
\begin{verbatim}
\begin{table}%table1
\caption{Forecast error variance decomposition for the US current-account-to-GDP 
  ratio (in percent)}
\label{forecast}
\begin{minipage}{28pc}% you need this line only if your table has notes and/or footnotes 
                      % (28pc is the text width)
\begin{tabular}{@{}lrrrr@{}}\hline
  Horizon/Shock\tablenote{NOTES: Do not include closing punctuation in table 
  titles. If a table has notes and/or footnotes, you must create the table 
  inside a \texttt{minipage} environment.} & 
  \multicolumn{1}{c}{$\epsilon_g^H$} & 
  \multicolumn{1}{c}{$\epsilon_g^F$} & 
  \multicolumn{1}{c}{$\epsilon_z^H$} & 
  \multicolumn{1}{c}{$\epsilon_z^F$}\\ \hline
  1 quarter   & 45.2 & 52.2 & 0.8 & 1.9\\
  4 quarters  & 41.4 & 54.9 & 0.8 & 2.8\\
  8 quarters  & 39.5 & 55.9 & 0.8 & 3.8\\
  16 quarters & 38.9 & 56.0 & 0.8 & 4.6\footnote{An example of a table footnote.}\\
\finalhline
\end{tabular}
\end{minipage}% you need this line only if your table has notes and/or footnotes
\end{table}
\end{verbatim}
%
\vspace\baselineskip\hrule % to separate verbatim from text
\end{table}

\subsection{Figures}

If you use labels to refer to the figures, \verb"\caption" must precede \verb"\label". See section~\ref{landfigures} for information on typesetting landscape figures. If possible, provide figures as .eps or .pdf files. Use strong black lines of at least 0.75pt at final printed size. Avoid differentiating or identifying with shading but use another method instead, such as bars, dots, diamonds). Beware that colour figures will be converted to monochrome for the print  version. Use standard fonts that are easily read and of an adequate size. We recommend using Times, Times New Roman, Arial or Helvetica. Do not set figures in boxes. Full guidelines for electronic artwork are available at \href{http://authorservices.wiley.com/author-resources/Journal-Authors/Prepare/manuscript-preparation-guidelines.html}{authorservices.wiley.com/bauthor/illustration.asp}. The source code for figure~\ref{flag} is shown immediately below the figure.

\begin{figure}%fig1
\includegraphics[width=0.6\textwidth]{canadian-flag.eps}
\caption{The National Flag of Canada 
  \figurenote NOTE: Canada's flag was approved by resolution of the House of Commons on December 15, 1964, followed by the Senate on December 17, 1964. It was proclaimed by Her Majesty Queen Elizabeth II, Queen of Canada, to take effect on February 15, 1965. The official ceremony inaugurating the new Canadian flag was held on Parliament Hill that same day.}
\label{flag}
\vspace\baselineskip\hrule % to separate figure from verbatim
\vspace\baselineskip
\begin{smallverbatim}
\begin{figure}%fig1
\includegraphics[width=0.6\textwidth]{canadian-flag.eps}
\caption{The National Flag of Canada
  \figurenote NOTE: Canada's flag was approved by resolution of....}
\label{flag}
\end{figure}
\end{smallverbatim}
\vspace\baselineskip\hrule % to separate verbatim from text
\end{figure}

\subsection{Landscape tables and figures (using \texttt{rotating.sty})}

You can typeset your tables and figures (floats) to be landscape using the \texttt{rotating.sty} package. This package uses rotation facilities from the \texttt{graphicx} package. The bottom of landscape tables and figures should always be on the right-hand side of the page (this is taken care of by using the \verb"[figuresright]" option). Ideally, none of the wording should be upside down when the journal is upright, although this may be unavoidable with large graphs.

In addition to using \texttt{rotating.sty}, include \texttt{floatpag.sty} and the command \verb"\rotfloatpagestyle{empty}". This combination ensures that headers and footers don't appear on float pages:
\begin{smallverbatim}
\usepackage[figuresright]{rotating}
\usepackage{floatpag}
  \rotfloatpagestyle{empty}
\end{smallverbatim}

\subsubsection{Coding for landscape tables}
\label{landtables}

Table~\ref{sideways} has been reproduced courtesy of Federico J. D\'iez and Alan C. Spearot. The following coding was used to create table~\ref{sideways}:
%
\begin{smallverbatim}
\newcolumntype{Y}{>{\centering\arraybackslash}X}
%
\begin{sidewaystable}%table2
  \caption{Host and source market potential -- Within country pair -- OLS}
  \label{sideways}
  \begin{tabularx}{\textwidth}{@{}lYYYYYYYYYY@{}}\hline
    & (1) & (2) & (3) & (4) & (5) & (6) & (7) & (8) & (9) & (10) \\\hline
    log(\textit{Host GDPPC})& 0.187\rlap{$^{***}$} & 0.230\rlap{$^{***}$} 
      & 0.190\rlap{$^{***}$}& 0.225\rlap{$^{***}$}\\
      & (0.067) & (0.072) & (0.069) & (0.074)\\[3pt]
    log(\textit{Source GDPPC}) &&& $\llap{$-$}$0.016 & 0.049\\
                               &&&(0.081) & (0.087)\\[3pt]
    $\Delta$ log(\textit{Host GDPPC}) &&&&&0.085 & 0.109\rlap{$^{*}$}\\
                                      &&&&&(0.054) & (0.062)\\[3pt]
    log(\textit{Host GDP}) &&&&&&& 0.144\rlap{$^{**}$} & 0.176\rlap{$^{***}$}\\
                           &&&&&&& (0.062) & (0.066)\\[3pt]
    log(\textit{Source GDP}) &&&&&&& $\llap{$-$}$0.059 & $\llap{$-$}$0.007\\
                             &&&&&&&(0.070) & (0.076)\\
    $\Delta$ log(\textit{Host GDPPC}) &&&&&&&&& 0.095\rlap{$^{*}$} 
                                              & 0.116\rlap{$^{*}$}\\
                                      &&&&&&&&&(0.053)  & (0.062)\\[3pt]
   Observations & 6,583 & 5,935 & 6,520 & 5,872 & 6,497 & 5,857 & 6,520 
                                                & 5,872 & 6,497 & 5,857\\[3pt]
   $R^2$        & 0.560 & 0.567 & 0.556 & 0.563 & 0.559 & 0.565 & 0.556 
                                                & 0.563 & 0.559 & 0.565\\[3pt]
   Cross-border only & No & Yes & No & Yes & No & Yes & No & Yes & No 
     & Yes\footnotetext{NOTES: Dependent variable is the share of full 
     acquisitions. Estimation technique is OLS. Unit of observation is 
     host nation-source nation-time. Host-source and year fixed effects. 
     Robust standard errors in parentheses. $^{***}p<0.001$, $^{**}p<0.05$, 
     $^{*}p<0.1$.}\\
  \finalhline
  \end{tabularx}
\end{sidewaystable}
\end{smallverbatim}

\newcolumntype{Y}{>{\centering\arraybackslash}X}
%
\begin{sidewaystable}%table2
  \caption{Host and source market potential -- Within country pair -- OLS}
  \label{sideways}
  \begin{tabularx}{\textwidth}{@{}lYYYYYYYYYY@{}}\hline
    & (1) & (2) & (3) & (4) & (5) & (6) & (7) & (8) & (9) & (10) \\\hline
    log(\textit{Host GDPPC})& 0.187\rlap{$^{***}$} & 0.230\rlap{$^{***}$} 
      & 0.190\rlap{$^{***}$}& 0.225\rlap{$^{***}$}\\
      & (0.067) & (0.072) & (0.069) & (0.074)\\[3pt]
    log(\textit{Source GDPPC}) &&& $\llap{$-$}$0.016 & 0.049\\
                               &&&(0.081) & (0.087)\\[3pt]
    $\Delta$ log(\textit{Host GDPPC}) &&&&&0.085 & 0.109\rlap{$^{*}$}\\
                                      &&&&&(0.054) & (0.062)\\[3pt]
    log(\textit{Host GDP}) &&&&&&& 0.144\rlap{$^{**}$} & 0.176\rlap{$^{***}$}\\
                           &&&&&&& (0.062) & (0.066)\\[3pt]
    log(\textit{Source GDP}) &&&&&&& $\llap{$-$}$0.059 & $\llap{$-$}$0.007\\
                             &&&&&&&(0.070) & (0.076)\\
    $\Delta$ log(\textit{Host GDPPC}) &&&&&&&&& 0.095\rlap{$^{*}$} 
                                              & 0.116\rlap{$^{*}$}\\
                                      &&&&&&&&&(0.053)  & (0.062)\\[3pt]
   Observations & 6,583 & 5,935 & 6,520 & 5,872 & 6,497 & 5,857 & 6,520 
                                                & 5,872 & 6,497 & 5,857\\[3pt]
   $R^2$        & 0.560 & 0.567 & 0.556 & 0.563 & 0.559 & 0.565 & 0.556 
                                                & 0.563 & 0.559 & 0.565\\[3pt]
   Cross-border only & No & Yes & No & Yes & No & Yes & No & Yes & No 
     & Yes\footnotetext{NOTES: Dependent variable is the share of full 
     acquisitions. Estimation technique is OLS. Unit of observation is 
     host nation-source nation-time. Host-source and year fixed effects. 
     Robust standard errors in parentheses. $^{***}p<0.001$, $^{**}p<0.05$, 
     $^{*}p<0.1$.}\\
  \finalhline
  \end{tabularx}
\end{sidewaystable}

\subsubsection{Coding for landscape figures}
\label{landfigures}

The landscape figure (figure~\ref{sidewaysflag}) was typeset using this coding:
\begin{smallverbatim}
\begin{sidewaysfigure}%fig2
\includegraphics[width=0.6\textwidth]{canadian-flag.eps}
  \caption{The National Flag of Canada}
\label{sidewaysflag}
\end{sidewaysfigure}
\end{smallverbatim}
\begin{sidewaysfigure}%fig2
\includegraphics[width=0.6\textwidth]{canadian-flag.eps}
  \caption{Good artwork can make or break an article}
\label{sidewaysflag}
\end{sidewaysfigure}

%________________________________________________________________________


\section{Mathematics}

\subsection{Equations}
Equations are indented 12 points:
\begin{equation}
\label{eqnexample}
  (\delta + k)e_2 + \delta 
  \left[\mu e_{1H} + (1-\mu)e_{1L}\right]=k.
\end{equation}
Equation~\eqref{eqnexample} was set using the following code:
\begin{smallverbatim}
\begin{equation}
\label{eqnexample}
  (\delta + k)e_2 + \delta 
  \left[\mu e_{1H} + (1-\mu)e_{1L}\right]=k.
\end{equation}
\end{smallverbatim}
Align equations on equals signs whenever possible. If you need to split an equation, because it will extend beyond the right page margin for example, try to break it before an operator. Ensure that pairs of opening and closing parentheses appearing on separate lines are the same size (see equation~\ref{alignexample}). Also note that the extra space around alignments normally found in \LaTeX\ class files has been removed. For example: 
\begin{eqnarray}
\label{alignexample}
\tilde t_a^B &=& \frac{\alpha}{\tilde d_a}\left[
  \left(s_a^4 - 4s_a^3 + 3s_a^2 + 6s_a - 12\right)
  \left(\lambda_a^B\right)^2 \right.\nonumber\\
    && + \left(-2s_a^3 + 9s_a^2 - 4s_a - 25\right)\lambda_a^B 
       + 4\left(s_a^2 - 3s_a + 1\right)\!\bigg].
\end{eqnarray}
Equation~\eqref{alignexample} was set using the following code:
\begin{smallverbatim}
\begin{eqnarray}
\label{alignexample}
\tilde t_a^B &=& \frac{\alpha}{\tilde d_a}\left[
  \left(s_a^4 - 4s_a^3 + 3s_a^2 + 6s_a - 12\right)
  \left(\lambda_a^B\right)^2 \right.\nonumber\\
    && + \left(-2s_a^3 + 9s_a^2 - 4s_a - 25\right)\lambda_a^B 
       + 4\left(s_a^2 - 3s_a + 1\right)\!\bigg].
\end{eqnarray}
\end{smallverbatim}
In equations that extend over more than one line, the equation number will appear on the last line.

\subsection{Typesetting non-italic (upright) Greek characters}
\label{Greekchar}

The \texttt{cjeupmath} package provides macros for upright lowercase Greek (\verb"\ualpha"--\verb"\uxi") characters and upright bold lowercase Greek (\verb"\ubalpha"--\verb"\ubxi") characters. However, use \verb"\uboldeta" for the bold upright symbol \verb"\eta". 

In order to use the \texttt{cjeupmath} package, you must have the AMS \verb"eurm/b" fonts installed.

The AMS packages are supplied with the AMS\,\LaTeX\ distribution. If you have the AMS\,\LaTeX\ distribution installed, you will not need the \texttt{ams*.sty} files supplied in the \texttt{cje} distribution package (it's a good idea to check whether there's a more recent AMS distribution).

For upright characters, add the prefix \verb"u"; for upright bold characters, add the prefix \verb"ub":\\[0.5\baselineskip]
\begin{tabular}{@{}p{6pt}p{30pt}@{\hspace{45pt}}p{6pt}l}
$\ualpha$ & \verb"$\ualpha$"  & $\ubalpha$ & \verb"$\ubalpha$"\\
$\ubeta$  & \verb"$\ubeta$"   & $\ubbeta$  & \verb"$\ubbeta$"\\
$\ugamma$ & \verb"$\ugamma$"  & $\ubgamma$ & \verb"$\ubgamma$"\\
$\udelta$ & \verb"$\udelta$"  & $\ubdelta$ & \verb"$\ubdelta$"
\end{tabular}\\[0.5\baselineskip]
If you don't have the AMS fonts installed, you should still use the above commands. The characters will be substituted by the typesetter.

\subsection{Typesetting the partial symbol}

The \texttt{cjeupmath} package also provides \verb"\upartial" and \verb"\ubpartial".

If you have the AMS fonts installed, you can use the style file \texttt{cjeupmath.sty} to typeset the partial symbol:\\[0.5\baselineskip]
\begin{tabular}{@{}p{6pt}p{30pt}@{\hspace{45pt}}p{6pt}l}
$\upartial$ & \verb"$\upartial$" & $\ubpartial$ & \verb"$\ubpartial$"\\
\end{tabular}\\[0.5\baselineskip]
As mentioned in section~\ref{Greekchar}, if you don't have the AMS fonts installed, you should still use the above commands.


\section{Theorems, definitions, remarks, proofs, etc.}

Many authors use \texttt{amsthm.sty} for typesetting these elements, so it has been included in the \texttt{cje} distribution. Therefore, include the following lines in the preamble:
\begin{smallverbatim}
\documentclass{cje}
\usepackage{amsmath}
\usepackage{amsthm}
\end{smallverbatim}
If you don't have any theorems, proofs, etc., you won't need \texttt{amsthm.sty}. 
%but you do need to include it to run this guide through \LaTeX. 
Note that if you're also using  \texttt{amsmath.sty}, you \emph{must} place it before \texttt{amsmath.sty}.

The instructions for \texttt{amsmath.sty} are given in \texttt{amsthdoc.pdf}. They can be dowloaded from  \href{http://ctan.org/}{ctan.org}. The following subsections discuss the basic features along with some additional ones.

If you don't include the \verb"\theoremstyle" command, the \texttt{plain} style will be used. To specify different styles, divide your \verb"\newtheorem" commands into groups and preface each group with the appropriate \verb"\theoremstyle".

\subsection{amsthm ``plain'' style}

Use the \verb"plain" style for theorems, lemmas, corollaries, propositions, conjectures, criterion and algorithms. You can turn off numbering by using the starred version of  \verb"newtheorem" (see the corollary, below):
\begin{smallverbatim}
\theoremstyle{plain}% default
  \newtheorem{theorem}{Theorem}
  \newtheorem{lemma}{Lemma}
  \newtheorem{proposition}{Proposition}
  \newtheorem*{corollary}{Corollary}
    
\begin{theorem}
  The first fundamental theorem of welfare economics states\ldots
\end{theorem}
\begin{proposition}
  The effects of the target's market potential\ldots
\end{proposition}
\begin{lemma}
\label{Lucas}
  If a separating equilibrium exists, in such an equilibrium\ldots
\end{lemma}
\begin{corollary}
  Within target market $j$, domestic acquisitions\ldots
\end{corollary}
\end{smallverbatim}
produces the following output:
\begin{theorem}
  The first fundamental theorem of welfare economics states\ldots
\end{theorem}
\begin{proposition}
  The effects of the target's market potential\ldots
\end{proposition}
\begin{lemma}
\label{Lucas}
  If a separating equilibrium exists, in such an equilibrium\ldots
\end{lemma}
\begin{corollary}
  Within target market $j$, domestic acquisitions\ldots
\end{corollary}

\subsection{amsthm ``definition'' style}
\label{amsdefn}

The \texttt{definition} style is typically used for definitions, conditions, problems, examples, as follows:
\begin{smallverbatim}
\theoremstyle{definition}
  \newtheorem{definition}{Definition}
  \newtheorem{example}{Example}

\begin{definition}
  The separating equilibrium distinguishes the two types of\ldots
\end{definition}
\begin{definition}
  Under state $H$, the separating equilibrium degenerates to\ldots 
\end{definition}
\begin{example}
  Consider the above analysis of\ldots
\end{example}
\end{smallverbatim}
and produces the following output:
\begin{definition}
  The separating equilibrium distinguishes the two types of\ldots
\end{definition}
\begin{definition}
  Under state $H$, the separating equilibrium degenerates to\ldots
\end{definition}
\begin{example}
  Consider the above analysis of\ldots
\end{example}

\subsection{amsthm ``remark'' style}
The \texttt{remark} style is typically used for remarks, notes, notations, claims, summaries, acknowledgements, cases and conclusions:
\begin{smallverbatim}
\theoremstyle{remark}
  \newtheorem*{remark}{Remark}
  \newtheorem*{case}{Case}

\begin{remark}
  In the pooling equilibria, the expected total loss\ldots
\end{remark}
\begin{case}
  We now move back to the stage in which\ldots
\end{case}
\end{smallverbatim}
produces the following output:
\begin{remark}
  In the pooling equilibria, the expected total loss\ldots
\end{remark}
\begin{case}
  We now move back to the stage in which\ldots
\end{case}

\subsection{Proofs}
\label{proofs}

The \verb"proof" environment is part of the \texttt{amsthm} package and provides a consistent format for proofs.
 For example:
\begin{smallverbatim}
\begin{proof}
  For $i=H,L$, the first order condition~(6) implies....
\end{proof}
\end{smallverbatim}
produces the following output:
\begin{proof}
  For $i=H,L$, the first order condition~(6) implies....
\end{proof}

\subsubsection{Changing the word ``Proof''}

You can substitute the word ``\textit{Proof}'' with another word or phrase using an optional argument. In the above example, ``\textit{Proof}'' becomes ``\textit{Proof of lemma~\ref{Lucas}}.''  
\begin{smallverbatim}
\begin{proof}[Proof of lemma~\ref{Lucas}]
   For $i=H,L$, the first order condition~(6) implies....
\end{proof}
\end{smallverbatim}
produces the following output:
\begin{proof}[Proof of lemma~\ref{Lucas}]
   For $i=H,L$, the first order condition~(6) implies....
\end{proof}


\subsubsection{Typesetting a proof without \qedsymbol}

This is not part of the \texttt{amsthm} package. To accomplish this, use the \verb"proof*" version. For example:
\begin{smallverbatim}
\begin{proof*}
  We prove this part of the proposition in two steps....
\end{proof*}
\end{smallverbatim}
produces the following output:
\begin{proof*}
  We prove this part of the proposition in two steps....
\end{proof*}

\subsubsection{Placing the \qedsymbol\ after a displayed equation}

To avoid the \qedsymbol\ dropping onto the following line at the end of a proof:
\begin{smallverbatim}
\begin{proof*}
  Energy equals mass times the speed of light squared:
  \[
     \equationqed{E=mc^2.}
  \]
\end{proof*}
\end{smallverbatim}
This produces the following output:
\begin{proof*}
   Energy equals mass times the speed of light squared:
  \[
     \equationqed{E=mc^2.}
  \]
\end{proof*}

\subsubsection{Placing the \qedsymbol\ after a displayed eqnarray}

This is not part of the amsthm package. To accomplish this,  use the \verb"proof*" version and add \verb"\arrayqed" and \verb"\arrayqedhere", as shown in this example:
\begin{smallverbatim}
\begin{proof*}
  The following equations prove the theorem:
  \arrayqed
  \begin{eqnarray}
    F_G^{k,j} &=& \psi_G\cdot d_{k,j},\nonumber\\[3pt]
    F_A^{k,j} &=& \tilde F_A + \psi_A \cdot d_{k,j}.
  \arrayqedhere
  \end{eqnarray}
\end{proof*}
\end{smallverbatim}
produces the following output:
\begin{proof*}
  The following equations prove the theorem:
  \arrayqed
  \begin{eqnarray}
    F_G^{k,j} &=& \psi_G\cdot d_{k,j},\nonumber\\[3pt]
    F_A^{k,j} &=& \tilde F_A + \psi_A \cdot d_{k,j}.
  \arrayqedhere
  \end{eqnarray}
\end{proof*}

\section{References}

All in-text citations must include the author name(s) and year of publication (e.g., ``In \cite{Lucas90}\ldots'',  ``According to \cite{Van-Zon}\ldots'', ``\cite{Mas-Colell} and \cite{Glomm92} have shown\ldots''). Before submitting your final files, please ensure there is an entry in the reference list for each source cited in the paper and vice versa. The best way to accomplish this is by using \textsc{Bib}\upshape{\TeX} and a bib database.

\subsection{Creating your reference list using \textsc{Bib}\upshape{\TeX}}

If you're creating your reference list using \textsc{Bib}\upshape{\TeX}, please be sure to include your .bib file with your .tex file. Use the \verb"\bibliography" and \verb"\bibliographystyle" commands to automatically produce the reference list and do not paste the contents of your .bbl file into the main .tex file.

 To generate automatic references from a bib database, place the following two commands where the reference list is to appear:
\begin{smallverbatim}
\bibliography{ageingbib}% to read ageingbib.bib
\bibliographystyle{cjebibstyle}% to impose CJE bibliography style on output
\end{smallverbatim}
%
Next, run your article through \LaTeX\ twice, run \textsc{Bib}\TeX and then run your article through \LaTeX\ once more. This series of runs will generate a .bbl file and include the .bbl file in your .tex file.

For example, if you've cited only these 10 sources from your bib database: 
\cite{Glomm92},
\cite{Mas-Colell},
\cite{Atkinson},
\cite{Van-Zon},
\cite{vanderPolGairns2000},
\cite{Lucas90},
\cite{Mendoza1991}, 
\cite{Glomm97},
\cite{Aisa04} and
\cite{LiSK12}, 
these 10 references, and only these 10, will appear in CJE style at the end of the article as the reference list. See the last page of this guide for the reference list output.

\subsubsection{Citations using \texttt{natbib} commands}
The standard \texttt{natbib} style file has been modified to produce the CJE style. This modified style file is called \texttt{cjenatbib.sty} (included in the \texttt{cje} distribution) and works exactly the same as \texttt{natbib.sty}. Here are some of the \verb"\cite" commands  available (please refer to the \texttt{natbib} documentation \href{https://www.ctan.org/pkg/natbib?lang=en}{ctan.org/pkg/natbib?lang=en} for more information):\\*[0.5\baselineskip]
\begin{tabular}{@{}ll}
\cite{Lucas90}
    & \indexsize\verb"\cite{Lucas90}"\\
\cite{Lucas90, Mas-Colell}
    & \indexsize\verb"\cite{Lucas90, Mas-Colell}"\\
\citep{Lucas90}
   & \indexsize\verb"\citep{Lucas90}"\\
\citep[see][p. $\,$34]{Lucas90}
   & \indexsize\verb"\citep[see][p. $\,$34]{Lucas90}"\\
\citep[e.g.,][]{Lucas90}
   & \indexsize\verb"\citep[e.g.,][]{Lucas90}"\\
\citep[section 2.3]{Lucas90}
   & \indexsize\verb"\citep[section 2.3]{Lucas90}"\\
\citep{Lucas90, Mas-Colell}
   & \indexsize\verb"\citep{Lucas90, Mas-Colell}"\\
\citeyearpar{Lucas90}
   & \indexsize\verb"\citeyearpar{Lucas90}"\\*[0.5\baselineskip]
\end{tabular}

\noindent See the last page of this guide for the reference list output.

Another helpful \verb"\cite" command is \verb"\citeapos{key}". For example, \verb"\citeapos{Lucas90}" produces  ``\citeapos{Lucas90}....'' In order to use this command, however, you must define \verb"\citeapos" in your document using \verb"\def\citeapos#1{\citeauthor{#1}'s (\citeyear{#1})}".   

\subsection{Creating your reference list manually}
If you aren't using the \texttt{cjebibstyle.bst} bibliography style file, you can either create a reference list in plain text or produce the same output as shown at the end of this guide by typing the references in the format shown below. \\


\noindent \textbf{NOTE:} If you produce your reference list manually, you will be asked at the proof stage to verify that each in-text citation appears in the reference list and vice versa.
\\

In article titles, capitalize only the first word of the title, any proper nouns and the first word of a subtitle (word after a colon). In book title, capitalize all words exception those of three or fewer letters. For page ranges, use an en dash (\verb"--") not a hyphen (-). Include the entire second number of the range for pages 1 to 99 but only the last two digits of the second number above 100 (e.g., 226--54, 890--99) unless more are needed for clarity (e.g., 190--208, 998--1004).

\begin{smallverbatim}
\begin{thebibliography}{10}
\newcommand{\enquote}[1]{``#1''}

\bibitem[{Aisa and Pueyo(2004)}]{Aisa04}
Aisa, R., and F. Pueyo (2004) \enquote{Endogenous longevity, health and
  economic growth: A slow growth for a longer life?,} \emph{Economics 
  Bulletin} 9, 1--10

\bibitem[{Atkinson and Stiglitz(1980)}]{Atkinson}
Atkinson, A., and J. Stiglitz (1980) \emph{Lectures on Public Economics}, 
  New York: McGraw-Hill

\bibitem[{Glomm and Ravikumar(1992)}]{Glomm92}
Glomm, G., and B. Ravikumar (1992) \enquote{Public versus private 
  investment in human capital endogenous growth and income inequality,} 
  \emph{Journal of Political Economy} 100, 813--34

\bibitem[{Glomm and Ravikumar(1997)}]{Glomm97}
------ (1997) \enquote{Productive government expenditures and long-run 
  growth,} \emph{Journal of Economic Dynamics and Control} 21, 183--204

\bibitem[{Li et al.(2012)Li, Shrivastava, and K{\"o}nig}]{LiSK12}
Li, P., A. Shrivastava, and A. C. K{\"o}nig (2012) \enquote{{GPU}-based 
  minwise hashing,} in \emph{Proceedings of the 21st World Wide Web 
  Conference (WWW 2012) (Companion Volume)}, pp. 565--66

\bibitem[{Lucas(1990)}]{Lucas90}
Lucas, R. (1990) \enquote{Supply-side economics: An analytical review,}
  \emph{Oxford Economic Papers} 42, 293--316

\bibitem[{Mas-Colell et al.(1995)Mas-Colell, Whinston, and Green}]{Mas-Colell}
Mas-Colell, A., M. D. Whinston, and J. R. Green (1995) \emph{Microeconomic
  Theory}, Oxford: Oxford University Press

\bibitem[{Mendoza(1991)}]{Mendoza1991}
Mendoza, E. G. (1991) \enquote{Real business cycles in a small open economy,}
  \emph{American Economic Review} 81, 717--818

\bibitem[{van der Pol and Cairns(2000)}]{vanderPolGairns2000}
van der Pol, M. M., and J. A. Cairns (2000) \enquote{Zero and negative time
  preference for health,} \emph{Health Economics} 9, 171--75

\bibitem[{Van-Zon and Muysken(2001)}]{Van-Zon}
Van-Zon, A., and J. Muysken (2001) \enquote{Health and endogenous growth,}
  \emph{Journal of Health Economics} 20, 169--85

\end{thebibliography}
\end{smallverbatim}

%\oneappendix
\appendix

\section{Typesetting a single appendix}
\label{single}

Appendices appear before the reference section. To create a single appendix (it won't be numbered):
\begin{smallverbatim}
\oneappendix
\section{Typesetting a single appendix}
  :
\end{smallverbatim}
Numbering of appendix tables, figures, equations, etc., begins at A1:
\begin{eqnarray}
\label{appeqnone}
 c_t(x;\Xi_t,\Theta) &=& \alpha[(1-\tau_0x)w_th_t - \tau_1x - h_t^\sigma],\nonumber\\[3pt]
 d_t(x;\Xi_t,\Theta) &=& \frac{\beta w_t h_t + 2\varepsilon r_t(x))}{r_t(x)(bw_t h_t + \varepsilon r_t(x))} 
 \left[(1-\tau_0 x)w_t h_t - \tau_1 x -h_t^\sigma\right]
\end{eqnarray}
Single appendix headings are numbered as follows:
\begin{itemize}
  \item \textbf{Appendix:\enskip $\langle$section heading$\rangle$}
  \item {\bfseries\textit{A1. $\langle$subsection heading$\rangle$}} 
  \item \textit{A1.1. $\langle$subsubsection heading$\rangle$}
\end{itemize}

\section{Typesetting two or more appendices}
\label{multiple}

Multiple appendices are numbered A1, A2, A3, A4, etc.
The two appendices in this guide were typeset using:
\begin{smallverbatim}
\appendix
\section{Typesetting a single appendix}
 :
\section{Typesetting two or more appendices}
 :
\end{smallverbatim}

The sections in appendix~\ref{single} are:
  \begin{itemize}
    \item \textbf{Appendix A1:\enskip $\langle$section heading$\rangle$}
    \item {\bfseries\textit{A1. $\langle$subsection heading$\rangle$}} 
    \item \textit{A1.1. $\langle$subsubsection heading$\rangle$}
  \end{itemize} 
The sections in appendix~\ref{multiple} are:
  \begin{itemize}
    \item \textbf{Appendix A2:\enskip $\langle$section heading$\rangle$}
    \item {\bfseries\textit{A2. $\langle$subsection heading$\rangle$}} 
    \item \textit{A2.1. $\langle$subsubsection heading$\rangle$}
  \end{itemize} 

% authors generating their own bbl file would uncomment the following two lines and comment out/delete the references below:
% \bibliography{ageingbib}% to read ageingbib.bib
% \bibliographystyle{cjebibstyle} % to impose CJE bibliography style on output

% however, we are going to include cjetemplate.bbl here:
\begin{thebibliography}{10}
\newcommand{\enquote}[1]{``#1''}
\providecommand{\natexlab}[1]{#1}

\bibitem[{Aisa and Pueyo(2004)}]{Aisa04}
Aisa, R., and F. Pueyo (2004) \enquote{Endogenous longevity, health and
  economic growth: A slow growth for a longer life?,} \emph{Economics Bulletin}
  9, 1--10

\bibitem[{Atkinson and Stiglitz(1980)}]{Atkinson}
Atkinson, A., and J. Stiglitz (1980) \emph{Lectures on Public Economics}, New
  York: McGraw-Hill

\bibitem[{Glomm and Ravikumar(1992)}]{Glomm92}
Glomm, G., and B. Ravikumar (1992) \enquote{Public versus private investment in
  human capital endogenous growth and income inequality,} \emph{Journal of
  Political Economy} 100, 813--34

\bibitem[{Glomm and Ravikumar(1997)}]{Glomm97}
------ (1997) \enquote{Productive government expenditures and long-run
  growth,} \emph{Journal of Economic Dynamics and Control} 21, 183--204

\bibitem[{Li et al.(2012)Li, Shrivastava, and K{\"o}nig}]{LiSK12}
Li, P., A. Shrivastava, and A. C. K{\"o}nig (2012) \enquote{{GPU}-based minwise
  hashing,} in \emph{Proceedings of the 21st World Wide Web Conference (WWW
  2012) (Companion Volume)}, pp. 565--66

\bibitem[{Lucas(1990)}]{Lucas90}
Lucas, R. (1990) \enquote{Supply-side economics: An analytical review,}
  \emph{Oxford Economic Papers} 42, 293--316

\bibitem[{Mas-Colell et al.(1995)Mas-Colell, Whinston, and Green}]{Mas-Colell}
Mas-Colell, A., M. D. Whinston, and J. R. Green (1995) \emph{Microeconomic
  Theory}, Oxford: Oxford University Press

\bibitem[{Mendoza(1991)}]{Mendoza1991}
Mendoza, E. G. (1991) \enquote{Real business cycles in a small open economy,}
  \emph{American Economic Review} 81, 717--818

\bibitem[{van der Pol and Cairns(2000)}]{vanderPolGairns2000}
van der Pol, M. M., and J. A. Cairns (2000) \enquote{Zero and negative time
  preference for health,} \emph{Health Economics} 9, 171--75

\bibitem[{Van-Zon and Muysken(2001)}]{Van-Zon}
Van-Zon, A., and J. Muysken (2001) \enquote{Health and endogenous growth,}
  \emph{Journal of Health Economics} 20, 169--85

\end{thebibliography}


\label{lastpage}
\end{document}
