%%%%%%%%%%%%%%%%%%%%%%%%%%%%%%%%%%%%%%%%%%%%%%%%%%%%%%%%%%%%%%%%%%%%%%
% small sample file
%%%%%%%%%%%%%%%%%%%%%%%%%%%%%%%%%%%%%%%%%%%%%%%%%%%%%%%%%%%%%%%%%%%%%%

\documentclass{article}
%
% a brief demonstration of the use of fncylab
%
\usepackage{fncylab,enumerate}
\begin{document}
%
% outer level lists (such as enumerate) use counter enumi for their
% list tags.
\begin{enumerate}[(i)]
\item first item, defining a non-fancy label \label{first}
%
% that label was defined to look otherwise than it was printed for the
% item:
\item second item (see also item~\ref{first})
%
% redefine label formats for this list's labels:
\labelformat{enumi}{(#1)}
%
% and label the another item in a different way:
\item third item, defining a fancy label \label{third}
%
% redefine label formats for this list's labels again:
\labelformat{enumi}{item~(#1)}
%
% we now see that the label has for item three has been defined
% differently, and that for item four differently again.  note the
% reference starting a new sentence.
\item fourth item, defining a fancy label in a different format (see
  also item~\ref{third}).  \Ref{fourth} shows how a sentence may start
  with reference to a fancy label; note that it's the \emph{label}
  that's fancy, so the second change to the label format only affects
  the labels defined after it.  \label{fourth}
\end{enumerate}
\end{document}
%%%%%%%%%%%%%%%%%%%%%%%%%%%%%%%%%%%%%%%%%%%%%%%%%%%%%%%%%%%%%%%%%%%%%%
% small sample file
%%%%%%%%%%%%%%%%%%%%%%%%%%%%%%%%%%%%%%%%%%%%%%%%%%%%%%%%%%%%%%%%%%%%%%

\documentclass{article}
%
% a brief demonstration of the use of fncylab
%
\usepackage{fncylab,enumerate}
\begin{document}
%
% outer level lists (such as enumerate) use counter enumi for their
% list tags.
\begin{enumerate}[(i)]
\item first item, defining a non-fancy label \label{first}
%
% that label was defined to look otherwise than it was printed for the
% item:
\item second item (see also item~\ref{first})
%
% redefine label formats for this list's labels:
\labelformat{enumi}{(#1)}
%
% and label the another item in a different way:
\item third item, defining a fancy label \label{third}
%
% redefine label formats for this list's labels again:
\labelformat{enumi}{item~(#1)}
%
% we now see that the label has for item three has been defined
% differently, and that for item four differently again.  note the
% reference starting a new sentence.
\item fourth item, defining a fancy label in a different format (see
  also item~\ref{third}).  \Ref{fourth} shows how a sentence may start
  with reference to a fancy label; note that it's the \emph{label}
  that's fancy, so the second change to the label format only affects
  the labels defined after it.  \label{fourth}
\end{enumerate}
\end{document}
