\part{User Documentation}


\section{Introduction}

The \wsemclassic\ document class is designed to either conform with the
recommendations of the Bavarian Kultusministerium for typesetting
w-seminar papers (\hyperlink{udoc.opt.strict}{|strict|} mode) or to use another
style which should look better. It is based on the \report\ class which
comes with the standard \hologo{LaTeX} distribution.

If you have any wishes or find bugs, please send an email to the author or
\href{https://github.com/jorsn/wsemclassic/issues}{create an issue at GitHub}.


\section{Usage}

To use \wsemclassic\ for your W-Seminar paper, simply insert the following into your
\hologo{LaTeX} preamble (before |\begin{document}|):

\begin{verbatim}
\documentclass[bibfile=<bibtex database name>]{wsemclassic}

\author{<Your Name>}
\title{<Paper Title>}
\date{Abiturjahrgang~<year>}
\subject{<Seminar Title/Subject>}
\school{<School Name>}
\major{major}{<Seminar Major Subject (Leitfach)>}
\teacher{<Your Teacher>}
\place{<The place where you live/write your paper>}
\end{verbatim}


\section{License}

\wsemclassic\ is distributed under a BSD License

\hypertarget{sec.options}{\section{Options}}

Like many other \hologo{LaTeX} document classes, \wsemclassic\ accepts options in
the well known |key=value| syntax. In the following, you will find a
description of all |keys| and their possible |values|
(`|true|' may be omitted; `|nofoo|' may be used instead of `|foo=false|' multiple
values, where allowed, must be enclosed in braces).

\noindent\\ Option descriptions are in the following format:\\

\describeoption{\meta{option}}{\meta{opt type}}{\meta{default value}}
\meta{describing paragraph: This is an option description.
	This option does this and that and you can
change many things by specifying it.} \\


\noindent
Since \wsemclassic\ is based on \report\ it accepts all of its options, but
\hyperlink{subsubsec.unrecomm}{some of them should not be used}.\\

\noindent
All Options not specified in \wsemclassic\ are passed to \report.

\hypertarget{subsubsec.unrecomm}{\subsection{Unrecommended \report\ options}}

\DescribeOption{\meta{foo}paper} Use \hyperlink{udoc.opt.paper}{|paper|=\meta{foo}} instead. \\
\DescribeOption{\meta{foo}pt} Use \hyperlink{udoc.opt.fontsize}{|fontsize|=\meta{foo}} instead. \\
\DescribeOption{\meta{language}} Use \hyperlink{udoc.opt.lang}{|lang|=\meta{foo}} instead.


\subsection{Strictness}

\describeoption{strict}{boolean}{false}
Use exactly the format recommended by the Bavarian Kultusministerium. \\
This option sets \hyperlink{udoc.opt.stricttitle}{|stricttitle|} and
\hyperlink{udoc.opt.frenchspacing}{|frenchspacing|} to \textit{true}. \\
It also sets the \hyperlink{udoc.opt.fontsize}{|fontsize|} to |12|, the
\hyperlink{udoc.opt.paper}{|paper|} to |a4| and the
\hyperlink{udoc.opt.lang}{|lang|} to |german|.

\describeoption{stricttitle}{boolean}{false}
Typeset "Seminararbeit" uppercase and not in small capitals as recommended
by the Bavarian Kultusministerium.

\describeoption{frenchspacing}{boolean}{false}
Make the spaces after words and sentences equal.



\subsection{Format and Language}

\describeoption{fontsize}{number}{12}
Fontsize in pt. \\

\describeoption{paper}{text}{a4}
Paper format as used as \report\ option |\meta{format}paper|. \\

\describeoption{lang}{text}{german}
Language. \\ If |lang|=|german|, the babel language is |ngerman|.

\describeoption{plxtex}{boolean}{true}
Specifies whether one of \hologo{pdfTeX}, \hologo{LuaTeX} or \hologo{XeTeX} is used. \\
\emph{Set to false if you don't use one of these engines!}



\subsection{Bibliography}

\describeoption{bib}{boolean}{true}
Specifies whether to use a bibliography (requires \hologo{BibTeX}) or not. \\

\describeoption{bibstyle}{text}{natdin}
Specifies the bibliographystlye for \hologo{BibTeX}. \\

\describeoption{bibfile}{text}{\char92{}jobname}
Specifies the filename of the main \hologo{BibTeX} database (|*.bib|). \\
\emph{|.bib| can be omitted.} \\
All etries are included in the document.



\subsection{Options Related to Used Packages}

\DescribeOption[noindex=true]{\meta{package name}args \rmfamily\itshape option list}
For most of the packages used by \wsemclassic, options can be specified in
the format \meta{package name}|args|=\marg{option list}.

These packages are \textsf{fontspec (|*quiet|), hyperref (|*unicode|),
	microtype (|*babel|), amsmath, titlesec (|*small|), geometry, fancyhdr,
tocbibind (|*nottoc|)} and \textsf{natbib (|*round|)}. \\

\noindent
For some packages there are additional or other options available:

\subsubsection{\textsf{fontspec}}

\describeoption{defaultfontfeatures}{\\ \texttt{type:} key value}{\\ Ligatures=\{TeX,\\ Common\}, Fractions=On}
Specifies the \textsf{fontspec} |\defaultfontfeatures|. \\~\\~\\

\describeoption{mainfont}{text}{CMU Serif}
These options specifie the fonts used as main (normally serif),

\describeoption{sansfont}{text}{CMU Sans Serif}
sans serif

\describeoption{monofont}{text}{CMU Typewriter Text}
and monospaced font. \\ \vskip 0.4em ~\\


\subsubsection{\textsf{hyperref} and \textsf{natbib}}

\describeoption{hyperref}{boolean}{true}
Turn \textsf{hyperref} or

\describeoption{natbib}{boolean}{true}
\textsf{natbib} on/off. \\~ \vskip 0.7em ~\\

%\iffalse vim: ft=tex
%\fi
