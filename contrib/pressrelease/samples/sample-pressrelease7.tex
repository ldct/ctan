% arara: pdflatex
% arara: pdflatex
\documentclass[a4paper,11pt]{pressrelease}

\usepackage[british]{babel}
\usepackage{graphicx}
\usepackage[hidelinks]{hyperref}

\usepackage{pressrelease-symbols}

\PRset{releasealign=right,ruled=false,logo=below right,smashlogo}

\renewcommand*{\PRemailformat}[1]{\href{mailto:#1}{\texttt{#1}}}

\PRlogo{\includegraphics[height=2cm]{example-image}}

\PRcompany{Some Company}
\PRdepartment{Some Department}
\PRcontact{Ann Other}
\PRaddress{1 The Street\\The Town\\AB1 2YZ}
\PRphone{01234 56789}
\PRmobile{07123456789}
\PRfax{01234 56788}
\PRurl{http://www.some-company.com/~abc}
\PRemail{ann.other@some-company.com}
\PRhours{9:00--17:30 Mon--Fri}
\PRencl{some stuff}

\date{Some City, \today}

\PRheadline{Some amazing news}
\PRsubheadline{subheading}

\begin{document}

\begin{pressrelease}

This is an example press release. Keep it short and use the third
person. Avoid the use of exclamation marks and all-caps. Put all the
pertinent details in the first paragraph. Answer who, what, when,
where and why.

Use short paragraphs. Try not to exceed 500 words. Keep to the point
and avoid jargon. This example has loaded the
\textsf{pressrelease-symbols} package to use symbols (from the
\textsf{marvosym} package and one designed using \textsf{tikz}) instead of text.
The image is from the \textsf{mwe} package. The header layout has
been changed using:
\begin{verbatim}
\PRset{releasealign=right,ruled=false,logo=below right,smashlogo}
\end{verbatim}

\begin{about}
Some Company was set up in imaginary year $i$.
\end{about}

\end{pressrelease}

\end{document}
