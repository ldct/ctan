% arara: pdflatex
% arara: pdflatex
\documentclass[a4paper,11pt]{pressrelease}

\usepackage[british]{babel}
\usepackage{graphicx}

% Remove textual tags from top info block
\renewcommand*{\PRinfotopline}[2]{%
  \PRinfoentry{#2}\tabularnewline
}

\PRlogo{\includegraphics[height=2cm]{example-image}}

\PRcompany{Some Company}
\PRcontact{Ann Other}
\date{Some City, \today}
\PRaddress{1 The Street\\The Town\\AB1 2YZ}
\PRphone{01234 56789}
\PRmobile{07123456789}
\PRfax{01234 56788}
\PRurl{http://www.some-company.com/}
\PRemail{ann.other@some-company.com}
\PRhours{9:00--17:30 Mon--Fri}

\PRheadline{Some amazing news}
\PRsubheadline{subheading}

\begin{document}

\begin{pressrelease}

This is an example press release. Keep it short and use the third
person. Avoid the use of exclamation marks and all-caps. Put all the
pertinent details in the first paragraph. Answer who, what, when,
where and why.

Use short paragraphs. Try not to exceed 500 words. Keep to the point
and avoid jargon. This layout has removed the textual tags in the
top info block by redefining: 
\begin{verbatim}
\renewcommand*{\PRinfotopline}[2]{%
  \PRinfoentry{#2}\tabularnewline
}
\end{verbatim}
The image is from the
\textsf{mwe} package.

\begin{about}
Some Company was set up in imaginary year $i$.
\end{about}

\end{pressrelease}

\end{document}
