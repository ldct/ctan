\documentclass[10pt,english]{hletter}
%\documentclass[11pt,german,bruni]{hletter}

\begin{document}

\signature{Dr.~C. Featherstonehaugh & Dr.~A. Beauchamp \\ CEO & CIO}

\reference{Impressions of Lausanne}

\date{Lausanne, le 15 septembre 2008}

\begin{letter}{Sir F. Treves, Bart.,\\
               \textbf{Vevey.}\\
               Switzerland}

% if you want the top-right text to be in [..]s use {..} as well:
\opening[\textsc{[draft]}]{Sir,}

%The value of \verb+\chaptername+ is \chaptername.

Lausanne, the capital of the Canton of Vaud, is the smaller of the two
cities of the Lake. It stands on a green slope which glides, in
leisurely fashion, from the wood which crowns its summit to the beach
at its foot. The town is far up on this slope, being about a mile and a
quarter above the port of Ouchy.

Seen from the Lake, it is so discursive a city that no one could
venture to define its outlines. Its houses are scattered in all
directions, among trees and lawns, gardens and green fields. It is as
if a drop of stone- coloured paint, falling from a height, had been
spattered over a green cloth. It seems to be composed entirely of
suburbs. If Chislehurst were given a cathedral and transferred to a
lake-side it might pass muster for Lausanne, since there is nothing to
suggest that this city of Vaud is so serious as it is or that it
possesses 67,000 inhabitants.

Lausanne when seen from a distance and Lausanne when viewed from within
are two towns which are totally unlike. A more deceptive place does not
exist. From afar Lausanne seems to occupy a hill-side as smooth as a
cushion. There is nothing to suggest that it contains streets, much
less railway stations, tramlines and shops. When, on the other hand,
the place is entered it is found to be as irregular and tumbled a town
as could be imagined, a place built in detachments without a plan, a
labyrinth of streets, of green terraces and gardens, of slums and
many-arched bridges all joined up with a \emph{central square} which is
neither square nor central. To this very disorder the town owes much of
its attraction, for the agreeable medley is due to the fact that
Lausanne is located, not on an even slope, but on three abrupt hills
separated by deep valleys. Were it not that these valleys are crossed
by a series of bridges, life in Lausanne would consist in climbing up
hill and in walking down again. Moreover, Alfred de Bougy, writing in
1846, says that owing to the hills and the villainous paving, Lausanne
was, in his day, practically inaccessible to carriages.\footnote{``Le
Tour du L\'{e}man,'' par A. de Bougy, Paris, 1846.}

The hills are round and are disposed in a triangle, like the balls of a
pawnbroker's sign. They are the Cit\'{e}, the Bourg and St.~Laurent. On the
Cit\'{e}, or predominant hill, are the castle and the cathedral. This mound
is, and always has been, the high place of the town and the stronghold
of its government. The Bourg was possessed by the nobles, by the
merchants and by the great inns. St.~Laurent was a suburb occupied by a
church and certain defence works. The poorer folk lived in the gutters
between the hills. In one of these flowed the Flon and in the other the
Louve. Their channels met at the Grand Pont ; but, within the actual
compass of the town, both streams have now disappeared from view.

From ancient prints\footnote{``Itinera Alpina,'' par J. J. Scheuchzeri,
Ludg.~Bat., 1723. Tome ii.} it can be seen that old Lausanne was a very
romantic looking town. Its three hills were crowned with castle and
spire, with turrets and high- soaring roofs; while around it ran a
zigzag wall pierced by gates and surmounted by many towers. The
dwellings that made up the mass of the city were of dark wood with
lofty gables. They huddled in the valleys like a drift of autumn leaves
in a gully. Of the fortifications, no trace remains with the exception
of one tower, the Tour de l'Ale, which stands near the Place du
Chauderon on the St.~Laurent hill. It is a high round tower of the days
of the musketeers, which finds itself now very inappropriately placed
in a modest street of private houses.

\closingtwo{Yours Faithfully,}

\vspace{2cm}
\cc{All Smiths in London\\ Mademoiselle S. Curchod}

\encl{Tourist guide to Switzerland.\\ Plan of Cully.}
\vfill

\end{letter}

\end{document}
