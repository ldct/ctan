%%
%% This is file `sample.tex',
%% generated with the docstrip utility.
%%
%% The original source files were:
%%
%% bibleref.dtx  (with options: `sample.tex,package')
%% 
%%  bibleref.dtx
%%  Copyright 2011 Nicola Talbot
%% 
%%  This work may be distributed and/or modified under the
%%  conditions of the LaTeX Project Public License, either version 1.3
%%  of this license of (at your option) any later version.
%%  The latest version of this license is in
%%    http://www.latex-project.org/lppl.txt
%%  and version 1.3 or later is part of all distributions of LaTeX
%%  version 2005/12/01 or later.
%% 
%%  This work has the LPPL maintenance status `maintained'.
%% 
%%  The Current Maintainer of this work is Nicola Talbot.
%% 
%%  This work consists of the files bibleref.dtx and bibleref.ins and the derived files bibleref-xidx.sty, bibleref.sty, sample-categories.tex, sample-indextools.tex, sample-xidx.tex, sample.tex, sample.ist.
%% 
%% \CharacterTable
%%  {Upper-case    \A\B\C\D\E\F\G\H\I\J\K\L\M\N\O\P\Q\R\S\T\U\V\W\X\Y\Z
%%   Lower-case    \a\b\c\d\e\f\g\h\i\j\k\l\m\n\o\p\q\r\s\t\u\v\w\x\y\z
%%   Digits        \0\1\2\3\4\5\6\7\8\9
%%   Exclamation   \!     Double quote  \"     Hash (number) \#
%%   Dollar        \$     Percent       \%     Ampersand     \&
%%   Acute accent  \'     Left paren    \(     Right paren   \)
%%   Asterisk      \*     Plus          \+     Comma         \,
%%   Minus         \-     Point         \.     Solidus       \/
%%   Colon         \:     Semicolon     \;     Less than     \<
%%   Equals        \=     Greater than  \>     Question mark \?
%%   Commercial at \@     Left bracket  \[     Backslash     \\
%%   Right bracket \]     Circumflex    \^     Underscore    \_
%%   Grave accent  \`     Left brace    \{     Vertical bar  \|
%%   Right brace   \}     Tilde         \~}
\documentclass{article}

\usepackage{bibleref}
\usepackage{makeidx}
\makeindex

\renewcommand{\bvidxpgformat}{textit}

\begin{document}

\title{Sample Document}
\author{Nicola Talbot}
\maketitle

Long citation in text:
\biblerefstyle{text}%
\bibleverse{IICor}(12:15,18,21-33) and
\bibleverse{Jeremiah}.
Short citation in footnote\footnote{%
\biblerefstyle{chicago}\bibleverse{IICor}(12:15,18,21-33) and
\bibleverse{Jeremiah}}.

\section{Default Style}

\biblerefstyle{default}
\noindent
\begin{tabular}{ll}
\verb|\bibleverse{Ex}| & \bibleverse{Ex}\\
\verb|\bibleverse{Exodus}(20:)| & \bibleverse{Exodus}(20:)\\
\verb|\bibleverse{Exod}(20:17)| & \bibleverse{Exod}(20:17)\\
\verb|\bibleverse{IICo}(12:21)| & \bibleverse{IICo}(12:21)\\
\verb|\bibleverse{IICor}(12:21-32)| & \bibleverse{IICor}(12:21-32)\\
\verb|\bibleverse{Ex}(20:17)(21:3)| & \bibleverse{Ex}(20:17)(21:3)\\
\verb|\bibleverse{Ex}(15:)(17:)(20:)| & \bibleverse{Ex}(15:)(17:)(20:)\\
\verb|\bibleverse{Rev}(1:2,5,7-9,11)| & \bibleverse{Rev}(1:2,5,7-9,11)\\
\verb|\bibleverse{IChronicles}(1:3)-(2:7)| &
\bibleverse{IChronicles}(1:3)-(2:7)
\end{tabular}

\section{Jerusalem Style}
This is the style used in the Jerusalem bible.
\biblerefstyle{jerusalem}

\noindent
\begin{tabular}{ll}
\verb|\bibleverse{Ex}| & \bibleverse{Ex}\\
\verb|\bibleverse{Exodus}(20:)| & \bibleverse{Exodus}(20:)\\
\verb|\bibleverse{Exod}(20:17)| & \bibleverse{Exod}(20:17)\\
\verb|\bibleverse{IICo}(12:21)| & \bibleverse{IICo}(12:21)\\
\verb|\bibleverse{IICor}(12:21-32)| & \bibleverse{IICor}(12:21-32)\\
\verb|\bibleverse{Ex}(20:17)(21:3)| & \bibleverse{Ex}(20:17)(21:3)\\
\verb|\bibleverse{Ex}(15:)(17:)(20:)| & \bibleverse{Ex}(15:)(17:)(20:)\\
\verb|\bibleverse{Rev}(1:2,5,7-9,11)| & \bibleverse{Rev}(1:2,5,7-9,11)\\
\verb|\bibleverse{IChronicles}(1:3)-(2:7)| &
\bibleverse{IChronicles}(1:3)-(2:7)
\end{tabular}

\section{Anglo-Saxon Style}

\biblerefstyle{anglosaxon}
\noindent
\begin{tabular}{ll}
\verb|\bibleverse{Ex}| & \bibleverse{Ex}\\
\verb|\bibleverse{Exodus}(20:)| & \bibleverse{Exodus}(20:)\\
\verb|\bibleverse{Exod}(20:17)| & \bibleverse{Exod}(20:17)\\
\verb|\bibleverse{IICo}(12:21)| & \bibleverse{IICo}(12:21)\\
\verb|\bibleverse{IICor}(12:21-32)| & \bibleverse{IICor}(12:21-32)\\
\verb|\bibleverse{Ex}(20:17)(21:3)| & \bibleverse{Ex}(20:17)(21:3)\\
\verb|\bibleverse{Ex}(15:)(17:)(20:)| & \bibleverse{Ex}(15:)(17:)(20:)\\
\verb|\bibleverse{Rev}(1:2,5,7-9,11)| & \bibleverse{Rev}(1:2,5,7-9,11)\\
\verb|\bibleverse{IChronicles}(1:3)-(2:7)| &
\bibleverse{IChronicles}(1:3)-(2:7)
\end{tabular}

\section{Journal of Ecclesiastical History}

\biblerefstyle{JEH}
\noindent
\begin{tabular}{ll}
\verb|\bibleverse{Ex}| & \bibleverse{Ex}\\
\verb|\bibleverse{Exodus}(20:)| & \bibleverse{Exodus}(20:)\\
\verb|\bibleverse{Exod}(20:17)| & \bibleverse{Exod}(20:17)\\
\verb|\bibleverse{IICo}(12:21)| & \bibleverse{IICo}(12:21)\\
\verb|\bibleverse{IICor}(12:21-32)| & \bibleverse{IICor}(12:21-32)\\
\verb|\bibleverse{Ex}(20:17)(21:3)| & \bibleverse{Ex}(20:17)(21:3)\\
\verb|\bibleverse{Ex}(15:)(17:)(20:)| & \bibleverse{Ex}(15:)(17:)(20:)\\
\verb|\bibleverse{Rev}(1:2,5,7-9,11)| & \bibleverse{Rev}(1:2,5,7-9,11)\\
\verb|\bibleverse{IChronicles}(1:3)-(2:7)| &
\bibleverse{IChronicles}(1:3)-(2:7)
\end{tabular}

\section{Modern Humanities Research Association}

\biblerefstyle{MHRA}
\noindent
\begin{tabular}{ll}
\verb|\bibleverse{Ex}| & \bibleverse{Ex}\\
\verb|\bibleverse{Exodus}(20:)| & \bibleverse{Exodus}(20:)\\
\verb|\bibleverse{Exod}(20:17)| & \bibleverse{Exod}(20:17)\\
\verb|\bibleverse{IICo}(12:21)| & \bibleverse{IICo}(12:21)\\
\verb|\bibleverse{IICor}(12:21-32)| & \bibleverse{IICor}(12:21-32)\\
\verb|\bibleverse{Ex}(20:17)(21:3)| & \bibleverse{Ex}(20:17)(21:3)\\
\verb|\bibleverse{Ex}(15:)(17:)(20:)| & \bibleverse{Ex}(15:)(17:)(20:)\\
\verb|\bibleverse{Rev}(1:2,5,7-9,11)| & \bibleverse{Rev}(1:2,5,7-9,11)\\
\verb|\bibleverse{IChronicles}(1:3)-(2:7)| &
\bibleverse{IChronicles}(1:3)-(2:7)
\end{tabular}

\section{Novum Testamentum Graece (Nestle-Aland)}

\biblerefstyle{NTG}
\noindent
\begin{tabular}{ll}
\verb|\bibleverse{Ex}| & \bibleverse{Ex}\\
\verb|\bibleverse{Exodus}(20:)| & \bibleverse{Exodus}(20:)\\
\verb|\bibleverse{Exod}(20:17)| & \bibleverse{Exod}(20:17)\\
\verb|\bibleverse{IICo}(12:21)| & \bibleverse{IICo}(12:21)\\
\verb|\bibleverse{IICor}(12:21-32)| & \bibleverse{IICor}(12:21-32)\\
\verb|\bibleverse{Ex}(20:17)(21:3)| & \bibleverse{Ex}(20:17)(21:3)\\
\verb|\bibleverse{Ex}(15:)(17:)(20:)| & \bibleverse{Ex}(15:)(17:)(20:)\\
\verb|\bibleverse{Rev}(1:2,5,7-9,11)| & \bibleverse{Rev}(1:2,5,7-9,11)\\
\verb|\bibleverse{IChronicles}(1:3)-(2:7)| &
\bibleverse{IChronicles}(1:3)-(2:7)
\end{tabular}

\section{MLA Style}

\biblerefstyle{MLA}
\noindent
\begin{tabular}{ll}
\verb|\bibleverse{Ex}| & \bibleverse{Ex}\\
\verb|\bibleverse{Exodus}(20:)| & \bibleverse{Exodus}(20:)\\
\verb|\bibleverse{Exod}(20:17)| & \bibleverse{Exod}(20:17)\\
\verb|\bibleverse{IICo}(12:21)| & \bibleverse{IICo}(12:21)\\
\verb|\bibleverse{IICor}(12:21-32)| & \bibleverse{IICor}(12:21-32)\\
\verb|\bibleverse{Ex}(20:17)(21:3)| & \bibleverse{Ex}(20:17)(21:3)\\
\verb|\bibleverse{Ex}(15:)(17:)(20:)| & \bibleverse{Ex}(15:)(17:)(20:)\\
\verb|\bibleverse{Rev}(1:2,5,7-9,11)| & \bibleverse{Rev}(1:2,5,7-9,11)\\
\verb|\bibleverse{IChronicles}(1:3)-(2:7)| &
\bibleverse{IChronicles}(1:3)-(2:7)
\end{tabular}

\section{Chicago Style}

\biblerefstyle{chicago}
\noindent
\begin{tabular}{ll}
\verb|\bibleverse{Ex}| & \bibleverse{Ex}\\
\verb|\bibleverse{Exodus}(20:)| & \bibleverse{Exodus}(20:)\\
\verb|\bibleverse{Exod}(20:17)| & \bibleverse{Exod}(20:17)\\
\verb|\bibleverse{IICo}(12:21)| & \bibleverse{IICo}(12:21)\\
\verb|\bibleverse{IICor}(12:21-32)| & \bibleverse{IICor}(12:21-32)\\
\verb|\bibleverse{Ex}(20:17)(21:3)| & \bibleverse{Ex}(20:17)(21:3)\\
\verb|\bibleverse{Ex}(15:)(17:)(20:)| & \bibleverse{Ex}(15:)(17:)(20:)\\
\verb|\bibleverse{Rev}(1:2,5,7-9,11)| & \bibleverse{Rev}(1:2,5,7-9,11)\\
\verb|\bibleverse{IChronicles}(1:3)-(2:7)| &
\bibleverse{IChronicles}(1:3)-(2:7)
\end{tabular}

\section{Text Style}
This style prints the citation out in full. It's based
on `default', and uses the fmtcount package to convert
the numbers into words.

\biblerefstyle{text}
\raggedright

\begin{itemize}
\item \verb|\bibleverse{Ex}| \bibleverse{Ex}
\item \verb|\bibleverse{Exodus}(20:)| \bibleverse{Exodus}(20:)
\item \verb|\bibleverse{Exod}(20:17)| \bibleverse{Exod}(20:17)
\item \verb|\bibleverse{IICo}(12:21)| \bibleverse{IICo}(12:21)
\item \verb|\bibleverse{IICor}(12:21-32)| \bibleverse{IICor}(12:21-32)
\item \verb|\bibleverse{Ex}(20:17)(21:3)| \bibleverse{Ex}(20:17)(21:3)
\item \verb|\bibleverse{Ex}(15:)(17:)(20:)| \bibleverse{Ex}(15:)(17:)(20:)
\item \verb|\bibleverse{Rev}(1:2,5,7-9,11)| \bibleverse{Rev}(1:2,5,7-9,11)
\item \verb|\bibleverse{IChronicles}(1:3)-(2:7)|
\bibleverse{IChronicles}(1:3)-(2:7)
\end{itemize}

\section{MHRA Style---Indexed}

\biblerefstyle{MHRA}
\noindent
\begin{tabular}{ll}
\verb|\ibibleverse{Ex}| & \ibibleverse{Ex}\\
\verb|\ibibleverse{Exodus}(9:)| & \ibibleverse{Exodus}(9:)\\
\verb|\ibibleverse{Exod}(20:17)| & \ibibleverse{Exod}(20:17)\\
\verb|\ibibleverse{Exod}(20:)| & \ibibleverse{Exod}(20:)\\
\verb|\ibibleverse{IICo}(12:21)| & \ibibleverse{IICo}(12:21)\\
\verb|\ibibleverse{IICor}(12:21-32)| & \ibibleverse{IICor}(12:21-32)\\
\verb|\ibibleverse{Ex}(20:17)(21:3)| & \ibibleverse{Ex}(20:17)(21:3)\\
\verb|\ibibleverse{Ex}(15:)(17:)(20:)| & \ibibleverse{Ex}(15:)(17:)(20:)\\
\verb|\ibibleverse{Rev}(1:2,5,7-9,11)| & \ibibleverse{Rev}(1:2,5,7-9,11)\\
\verb|\ibibleverse{IChronicles}(1:3)-(2:7)| &
\ibibleverse{IChronicles}(1:3)-(2:7)\\
\verb|\ibibleverse{IIPeter}(3:8-15a)| &
\ibibleverse{IIPeter}(3:8-15a)
\end{tabular}

\section{New Style}

This new style is based on the `default' style, but
has verses in lower case Roman numerals, and redefines
``Revelation'' as ``Apocalypse''.

\begin{verbatim}
\newbiblerefstyle{sample}{%
\biblerefstyle{default}%
\renewcommand{\BRversestyle}[1]{\romannumeral##1}%
\setbooktitle{Revelation}{Apocalyse}%
}
\end{verbatim}
\newbiblerefstyle{sample}{%
\biblerefstyle{default}%
\renewcommand{\BRversestyle}[1]{\romannumeral##1}%
\setbooktitle{Revelation}{Apocalyse}%
}

\biblerefstyle{sample}
\noindent
\begin{tabular}{ll}
\verb|\bibleverse{Ex}| & \bibleverse{Ex}\\
\verb|\bibleverse{Exodus}(20:)| & \bibleverse{Exodus}(20:)\\
\verb|\bibleverse{Exod}(20:17)| & \bibleverse{Exod}(20:17)\\
\verb|\bibleverse{IICo}(12:21)| & \bibleverse{IICo}(12:21)\\
\verb|\bibleverse{IICor}(12:21-32)| & \bibleverse{IICor}(12:21-32)\\
\verb|\bibleverse{Ex}(20:17)(21:3)| & \bibleverse{Ex}(20:17)(21:3)\\
\verb|\bibleverse{Ex}(15:)(17:)(20:)| & \bibleverse{Ex}(15:)(17:)(20:)\\
\verb|\bibleverse{Rev}(1:2,5,7-9,11)| & \bibleverse{Rev}(1:2,5,7-9,11)\\
\verb|\bibleverse{IChronicles}(1:3)-(2:7)| &
\bibleverse{IChronicles}(1:3)-(2:7)
\end{tabular}

\section{Partial References}

Only display verse numbers, but index under book and chapter:
\ibiblevs{Gen}(19:3-4).

Display only chapter and verses, but index under book:
\ibiblechvs{Gen}(4:1-5).

Complete indexed reference: \ibibleverse{Gen}(20:1-4).

Display chapter range with no verses: \ibibleverse{Gen}(1,4-6,8-9,11:).

Display chapter range with no verses (no index): \bibleverse{John}(1,3,4-6,8:).

\printindex

\end{document}
\endinput
%%
%% End of file `sample.tex'.
