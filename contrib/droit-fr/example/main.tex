\documentclass[a4paper,12pt,french,twoside,footnotereset=true,versetitle=true]{droit-fr}

\usepackage{ifluatex}
\ifluatex % compilation via Lua(La)TeX
	\usepackage{fontspec}
	\setmainfont{FreeSerif} % police proche de Times New Roman. A modifier le cas échéant.
	\setsansfont{FreeSans}
	\setmonofont{FreeMono}
\else % compilation via pdf(La)TeX
	\usepackage[utf8]{inputenc}
	\usepackage{times} % police proche de Times New Roman. A modifier le cas échéant.
	\usepackage[T1]{fontenc}
\fi
\usepackage{microtype} % amélioration du gris typographique
\usepackage{hyperref} % hyperliens PDF
\usepackage{bookmark} % signets PDF
\usepackage[nonumberlist,toc,noredefwarn]{glossaries} % glossaire
\usepackage{glossary-mcols} % glossaire sur deux colonnes
\usepackage[style=droit-fr,backend=biber,indexing=cite]{biblatex} % paramètres de bibliographie
\usepackage{lipsum} % génération de texte automatique

% paramètres des hyperliens PDF
\hypersetup{%
  pdftitle={Mon titre de thèse},
  pdfauthor={Prénom Nom}
}

% paramètres des signets PDF
\bookmarksetup{numbered=true,depth=4}

% création du glossaire et des indexs
\makeglossaries % glossaire, fichier généré: .gls
\makeindexv % index de base par versets, fichié généré: index.idx
\makeindexa % index d'auteurs par versets, fichié généré: auteurs.idx

% index des auteurs séparé de l'index de base
\DeclareIndexNameFormat{default}{%
  \usebibmacro{index:name}{\indexa}{#1}{#3}{#5}{#7}}

% chargement des fichiers de sources bibliographiques
\addbibresource{journaux.bib}
\addbibresource{bibliographie.bib}

% marges
\settrimmedsize{297mm}{210mm}{*}
\setlength{\trimtop}{0pt}
\setlength{\trimedge}{\stockwidth}
\addtolength{\trimedge}{-\paperwidth}
\settypeblocksize{634pt}{448.13pt}{*}
\setulmargins{4cm}{*}{*}
\setlrmargins{*}{*}{1.5}
\setmarginnotes{17pt}{51pt}{\onelineskip}
\setheadfoot{\onelineskip}{2\onelineskip}
\setheaderspaces{*}{2\onelineskip}{*}
\checkandfixthelayout

\listfiles

\renewcommand*{\glsgroupskip}{}


\begin{document}

\frontmatter % pages en chiffres romains, sections non numérotées
\pagestyle{plain} % en-tetes vides
\author{Prénom}{Nom}
\title{Mon titre de thèse}

\university{Nom de l'université}
\school{Droit international, droit européen, relations internationales et droit comparé}
\speciality{Droit privé}
\approvaldate{2 Janvier 2012}

\director{Monsieur}{Amstram}{Gram}{Professeur à l'Université Rébenthine}
\reportera{Monsieur}{Bidule}{Trucmuche}{Professeur à l'Université Lévision}
\reporterb{Monsieur}{Blabla}{Chose}{Professeur à l'Université Tanisé}
\membera{Madame}{Machin}{Chouette}{Professeur à l'Université Lorisation}
\memberb{Monsieur}{Pif}{Pouf}{Professeur à l'Université Alamenthe}

\maketitlepage

 % page de titre
\section{Avertissement}

La Faculté n’entend donner aucune approbation ni improbation aux opinions émises dans cette thèse; ces opinions doivent être considérées comme propres à leur auteur.

\cleardoublepage

\section{Remerciements}


\cleardoublepage

 % avertissement, remerciements, résumé
% sommaire
\renewcommand*{\contentsname}{Sommaire}
\let\oldchangetocdepth\changetocdepth
\renewcommand*{\changetocdepth}[1]{}
\let\oldcftchapterfillnum\cftchapterfillnum
\setcounter{tocdepth}{0}% Parties / Titres / Chapitres seulement

%\cftpagenumbersoff{book}
%\cftpagenumbersoff{part}
%\cftpagenumbersoff{chapter}

\renewcommand{\tocheadstart}{}

\renewcommand{\cftbeforebookskip}{1em}
%\renewcommand{\cftbookfont}{}
\renewcommand{\cftbookindent}{0em}
%\renewcommand{\cftbooknumwidth}{}
\renewcommand{\cftbookpagefont}{\normalfont\bfseries\large}
%\renewcommand{\cftafterbookskip}{}
%\renewcommand{\cftbookleader}{\cftdotfill{\cftdotsep}}

\renewcommand{\cftbeforepartskip}{0.5em}
\renewcommand{\cftpartfont}{\normalfont}
\renewcommand{\cftpartindent}{0.5em}
%\renewcommand{\cftpartnumwidth}{}
\renewcommand{\cftpartpagefont}{\normalfont\scshape}
%\renewcommand{\cftpartleader}{\cftdotfill{\cftdotsep}}

\renewcommand{\cftbeforechapterskip}{0em}
\renewcommand{\cftchapterfont}{\normalfont}
\renewcommand{\cftchaptername}{Chapitre\space}
\renewcommand{\cftchapterindent}{1em}
%\renewcommand{\cftchapternumwidth}{}
\renewcommand{\cftchapterpagefont}{\normalfont}
%\renewcommand{\cftchapterleader}{\cftdotfill{\cftdotsep}}

\clearpage
\tableofcontents

\newacronym{ibid}{ibid.}{ibidem}
\newacronym{dalloz}{D.}{Recueil Dalloz}
\newacronym{cedh}{CEDH}{Cour Européenne des Droits de l'Homme}
\newacronym{rtdciv}{RTD civ.}{Revue Trimestrielle de droit civil}
\newacronym{jcpg}{JCP G.}{Jurisclasseur Périodique, édition Générale}
\newacronym{defrenois}{Defrénois}{Répertoire Defrénois}
\newacronym{alinea}{al.}{alinéa}
\newacronym{assplen}{Ass. plén.}{Assemblée plénière}
\newacronym{bullciv}{Bull. civ.}{Bulletin civil des arrêts de la Cour de cassation}
\newacronym{appel}{CA}{Cour d'appel}
\newacronym{cassass}{Cass.}{Cour de cassation}
\newacronym{cciv}{C. civ.}{Code civil}
\newacronym{chreunies}{Ch. réun.}{Chambres réunies}
\newacronym{dretpatr}{Dr. et patr.}{Droit et Patrimoine}
\newacronym{fasc}{Fasc.}{Fascicule}
\newacronym{gazpal}{Gaz. Pal.}{Gazette du Palais}
\newacronym{jcpn}{JCP N.}{Jurisclasseur Périodique, édition Notariale}
\newacronym{jo}{JO}{Journal Officiel}
\newacronym{prec}{préc.}{précité}
\newacronym{puf}{PUF}{Presse Universitaire de France}
\newacronym{ridc}{RIDC}{Revue internationale de droit comparé}
\newacronym{rjpf}{RJPF}{Revue juridique personnes et famille}
\newacronym{supra}{supra}{ci-dessus}
\newacronym{sq}{sq.}{et suivants}
\newacronym{t}{t.}{tome}
\newacronym{infra}{infra}{ci-dessous}
\newacronym{IR}{IR}{Informations Rapides du recueil Dalloz}
\newacronym{ajfam}{AJ Famille}{Actualité Juridique Famille}
\newacronym{an}{AN}{Assemblée Nationale}
\newacronym{bullcrim}{Bull. crim.}{Bulletin des arrêts de la Cour de cassation (chambre criminelle)}
\newacronym{cah}{Cah.}{Cahiers}
\newacronym{ce}{CE}{Conseil d'État}
\newacronym{chap}{chap.}{chapitre}
\newacronym{v}{v.}{voir}
\newacronym{circmin}{Circ. Min.}{Circulaire Ministérielle}
\newacronym{constit}{Cons. Const.}{Conseil Constitutionnel}
\newacronym{etal}{et al.}{et autres}
\newacronym{gajciv}{GAJ civ.}{Grands Arrêts Jurisprudence civile}
\newacronym{jaf}{JAF}{Juge aux affaires familiales}
\newacronym{loi}{L.}{Loi}
\newacronym{lpa}{LPA}{Les Petites Affiches}
\newacronym{cpc}{CPC}{Code de procédure civile}
\newacronym{obs}{obs.}{observations}
\newacronym{comm}{comm.}{commentaire}
\newacronym{opcit}{op. cit.}{opere citato}
\newacronym{prop}{prop.}{proposition}
\newacronym{rapp}{rapp.}{rapporteur}
\newacronym{somm}{somm.}{sommaire}
\newacronym{vol}{vol.}{volume}
\newacronym{concl}{concl.}{conclusion}

\glsaddall
\printglossary[style=mcolindex,title=Glossaire]



\mainmatter % pages en chiffres arabes, sections numérotées
\pagestyle{corpus} % en-tetes/pied-de-pages en style "corpus"
\chapitre*{Introduction}
\markright{Introduction}
\addcontentsline{toc}{chapter}{Introduction}

\verset{Liens internes!}
\label{liensinternes}
Un exemple de note en bas de page\footnote{Une note de bas de page.}. 
Des exemples de liens divers:
\begin{itemize}
\item \autoref{panom} d'intitulé \titleref{panom}, commençant à la page \pageref{panom} 
\item \autoref{tinom} d'intitulé \titleref{tinom}, commençant à la page \pageref{tinom}
\item \autoref{chnom} d'intitulé \titleref{chnom}, commençant à la page \pageref{chnom}
\item \ref{secnom} d'intitulé \titleref{secnom}, commençant à la page \pageref{secnom}
\item \ref{pgnom} d'intitulé \titleref{pgnom}, commençant à la page \pageref{pgnom}
\item \ref{spgnom} d'intitulé \titleref{spgnom}, commençant à la page \pageref{spgnom}
\item \ref{alnom} d'intitulé \titleref{alnom}, commençant à la page \pageref{alnom}
\item \ref{salnom} d'intitulé \titleref{salnom}, commençant à la page \pageref{salnom}
\item \ref{ptnom} d'intitulé \titleref{ptnom}, commençant à la page \pageref{ptnom}
\item \no{\ref{monverset}} d'intitulé \titleref{monverset}, commençant à la page \pageref{monverset}\fvref{monverset}
\end{itemize}

\verset{Citations d'ouvrages bibliographiques\ldots}
Première\cite{cass_ass_19910531}, seconde\cite[35]{malaurie_obligations} et troisième citation\cite[39]{malaurie_obligations} d'ouvrages.

Une citation d'un article du code civil\cite[1642-1]{cciv} suivie d'une autre\cite[1642-2]{cciv}. 

D'autres citations\cites{saintpern_latexdroitfr}{alland_dicoculturejur}.

Les citations fonctionnent également lorsqu'invoquées depuis une note de base de page \footnote{Comme on peut le voir ici: \cites{egea_fonctionjuger}[35]{malaurie_obligations}, ou bien là: \cite{saintpern_latexdroitfr}. Dans ce cas, il convient de noter qu'aucun point n'est automatiquement ajouté en fin de citation; il faut donc l'ajouter manuellement.}

\verset{Guillemets ?}
Voici un exemple d'utilisation des \enquote{guillemets français de premier et \enquote{second} niveau}.

\verset{Index}
Je cite ici certains termes qui seront ajoutés à l'index: épigénétique, empreinte génétique, filiation. C'est le numéro du présent verset qui sera indiqué dans l'index.
\indexv{epigenetique@épigénétique}
\indexv{empreinte génétique}
\indexv{filiation|(}

\verset{Index, suite}
Je cite ici certains termes qui seront ajoutés à l'index: présomption simple, épigénétique.
\indexv{présomption!simple}
\indexv{epigenetique@épigénétique}

\verset{Index, \emph{suite} et fin}
Je cite ici certains termes qui seront ajoutés à l'index: présomption irréfragable, expertise biologique, filiation.
\indexv{présomption!irréfragable}
\indexv{expertise biologique|seealso{empreinte génétique}}
\indexv{filiation|)}
\newpage
\verset{Renvois}
\label{monverset}
Je renvoie ici à un numéro de verset\footnote{Cf. \vref{liensinternes}.}.

\verset{Renvois}
\label{monverset2}
Je renvoie ici à un numéro de verset précédent\footnote{Cf. \vref{monverset}.}.




\partie{Un nom de partie}
\label{panom}

\verset{}
\lipsum % texte de remplissage
\footnote{Note de remplissage.}

\titre{Un nom de titre}
\label{tinom}

\verset{}
\lipsum % texte de remplissage
\footnote{Note de remplissage.}

\chapitre{Un nom de chapitre}
\label{chnom}

\verset{}
\lipsum % texte de remplissage
\footnote{Note de remplissage.}

\section{Un nom de section}
\label{secnom}

\verset{}
\lipsum % texte de remplissage
\footnote{Note de remplissage.}

\paragraphe{Un \emph{nom} de paragraphe}
\label{pgnom}

\verset{}
\lipsum % texte de remplissage
\footnote{Note de remplissage.}

\souspara{Un nom de sous-paragraphe}
\label{spgnom}

\verset{}
\lipsum % texte de remplissage
\footnote{Note de remplissage.}

\alinea{Un nom de alinéa}
\label{alnom}

\verset{}
\lipsum % texte de remplissage
\footnote{Note de remplissage.}

\sousalinea{Un nom de sous-alinéa}
\label{salnom}

\verset{}
\lipsum % texte de remplissage
\footnote{Note de remplissage.}

\point{Un nom de point}
\label{ptnom}

\verset{}
\lipsum % texte de remplissage
\footnote{Note de remplissage.}

\souspoint{Un nom de sous-point}
\label{sptnom}

\verset{}
\lipsum % texte de remplissage
\footnote{Note de remplissage.}
 % première partie
\partie{Un autre nom de partie}

\verset{}
\lipsum % texte de remplissage

\titre{Un autre nom de titre}

\verset{}
\lipsum % texte de remplissage

\chapitre{Un autre nom de chapitre}

\verset{}
\lipsum % texte de remplissage

\section{Un autre nom de section}

\verset{}
\lipsum % texte de remplissage
 % deuxième partie

\backmatter % pages en chiffres arabes, sections non numérotées
\bookmarksetup{startatroot} % RAZ du niveau des signets PDF
\section{Conclusion and future work}
\label{sec:conclusion}

Current rule-based query optimizers do not provide a very intuitive and
conceptually streamlined framework to define rules and actions.  Our
experiences with the Volcano optimizer generator suggest that its model
of rules and the expression of these rules is much more complicated and
too low-level than it needs to be.  As a consequence, rule sets in
Volcano are fragile, hard to write, and debug.  Similar problems may
exist in other contemporary rule-based query optimizers.

We believe that rule-based query optimizers will be standard tools
of future database systems.  The pragmatic difficulties of using
existing rule-based optimizers led us to develop Prairie, an
extensible and structured algebraic framework for specifying rules.
Prairie is similar to existing optimizers in that it supports both
transformation rules and implementation rules.  However, Prairie
makes several improvements:
\begin{enumerate}
\item it offers a conceptually more streamlined model for rule specification;
\item rules are encapsulated, there are no ``hidden'' operators or
      ``hidden'' algorithms;
\item implementation hints (\eg enforcers) are deduced automatically;
\item and it has efficient implementations.
\end{enumerate}

We have explained how the first three points are important for
simplifying rule specifications and making rule sets less brittle for
extensibility.  A consequence is that Prairie rules are simpler and
more robust than rules of existing optimizers (\eg Volcano).  We
addressed the fourth point by building a P2V pre-processor which uses
sophisticated algorithms to compose and compact a Prairie rule set into
a Volcano rule set.  To demonstrate the scalability of our approach, we
rewrote the TI Open OODB rule set as a Prairie rule set, generated its
Volcano counterpart, and showed that the performance of the synthesized
Volcano rule set closely matches the hand-crafted Volcano rule set.

Our future work will concentrate on developing higher-level
abstractions using Prairie, including automatically generating Prairie
rule sets, and combining multiple Prairie rule sets to automatically
generate efficient optimizers.

\section*{Acknowledgments}
\label{sec:acknowledgments}

We wish to thank Texas Instruments, Inc.\ for making the Open OODB
source code available to us.  Comments by Jos\'e Blakeley, Anne Ngu,
Vivek Singhal, Thomas Woo and the anonymous referees greatly improved
the quality of the paper.

\chapitre{Annexes}

\verset{}
\lipsum % texte de remplissage


%%% paramètres de la bibliographie %%%

% forcer l'impression de toutes les références non citées dans le texte
\nocite{*}

% titres des sections de la bibliographie
\defbibheading{france}{\section{Droit français}}
\defbibheading{europe}{\section{Droit européen}}

% titres de types de sources
\defbibheading{lois}{\paragraphe{Lois}}
\defbibheading{rapports}{\paragraphe{Rapports officiels}}
\defbibheading{jurisprudence}{\paragraphe{Jurisprudence}}
\defbibheading{generaux}{\paragraphe{Ouvrages généraux}}
\defbibheading{speciaux}{\paragraphe{Ouvrages speciaux}}
\defbibheading{theses}{\paragraphe{Thèses}}
\defbibheading{speciaux}{\paragraphe{Ouvrages spéciaux}}
\defbibheading{collectifs}{\paragraphe{Ouvrages collectifs}}
\defbibheading{articles}{\paragraphe{Articles}}

% titres pour la jurisprudence
\defbibheading{juris_ccel}{\souspara{Conseil constitutionnel}}
\defbibheading{juris_ce}{\souspara{Conseil d'État}}
\defbibheading{juris_cass}{\souspara{Cour de cassation}}
	\defbibheading{juris_cass_ass}{\alinea{Assemblée plénière}}
	\defbibheading{juris_cass_1civ}{\alinea{1\iere{} chambre civile}}
	\defbibheading{juris_cass_2civ}{\alinea{2\ieme{} chambre civile}}
	\defbibheading{juris_cass_3civ}{\alinea{3\ieme{} chambre civile}}
	\defbibheading{juris_cass_com}{\alinea{Chambre commerciale}}
	\defbibheading{juris_cass_soc}{\alinea{Chambre sociale}}
	\defbibheading{juris_cass_crim}{\alinea{Chambre criminelle}}
\defbibheading{juris_ca}{\souspara{Cours d'appel}}
\defbibheading{juris_tgi}{\souspara{Tribunaux de grande instance}}
\defbibheading{juris_ti}{\souspara{Tribunaux d'instance}}

% filtres de sélection
\defbibfilter{gen}{\(\type{book} \or \type{inbook}\) \and \not \keyword{special}}
\defbibfilter{spec}{\(\type{book} \or \type{inbook}\) \and \keyword{special}}
\defbibfilter{col}{\type{collection} \or \type{proceedings}}
\defbibfilter{art}{\type{incollection} \or \type{inproceedings} \or \type{article}}

%%% impression des différentes bibliographies %%%

\chapitre{Bibliographie}

\printbibheading[heading=france]

\printbibliography[heading=lois,type=legislation,keyword=french]
\printbibliography[heading=rapports,type=report,keyword=french]
\printbibheading[heading=jurisprudence]
	\newrefcontext[sorting=iymd]
	\printbibliography[heading=juris_ccel, type=jurisdiction, keyword=ccel]
	\printbibliography[heading=juris_ce, type=jurisdiction, keyword=ce]
	\printbibheading[heading=juris_cass]
		\printbibliography[heading=juris_cass_ass, type=jurisdiction, keyword=cassass]
		\printbibliography[heading=juris_cass_1civ, type=jurisdiction, keyword=cass1civ]
		\printbibliography[heading=juris_cass_2civ, type=jurisdiction, keyword=cass2civ]
		\printbibliography[heading=juris_cass_3civ, type=jurisdiction, keyword=cass3civ]
		\printbibliography[heading=juris_cass_com, type=jurisdiction, keyword=casscom]
		\printbibliography[heading=juris_cass_soc, type=jurisdiction, keyword=casssoc]
		\printbibliography[heading=juris_cass_crim, type=jurisdiction, keyword=casscrim]
	\printbibliography[heading=juris_ca, type=jurisdiction, keyword=ca]
	\printbibliography[heading=juris_tgi, type=jurisdiction, keyword=tgi]
	\printbibliography[heading=juris_ti, type=jurisdiction, keyword=ti]
	\endrefcontext % sorting=iymd
\printbibliography[heading=generaux, filter=gen, keyword=french]
\printbibliography[heading=theses, type=thesis, keyword=french]
\printbibliography[heading=speciaux, filter=spec, keyword=french]
\printbibliography[heading=collectifs, type=collection, keyword=french]
\printbibliography[heading=articles, filter=art, keyword=french]

\printbibheading[heading=europe]

\printbibliography[heading=lois, type=legislation, keyword=ue]
\printbibliography[heading=rapports, type=report, keyword=ue]
\newrefcontext[sorting=tymdi]
\printbibliography[heading=jurisprudence, type=jurisdiction, keyword=ue]
\endrefcontext % sorting=tymdi
\printbibliography[heading=generaux, filter=gen, keyword=ue]
\printbibliography[heading=speciaux, filter=spec, keyword=ue]
\printbibliography[heading=collectifs, filter=col, keyword=ue]
\printbibliography[heading=theses, type=thesis, keyword=ue]
\printbibliography[heading=articles, filter=art, keyword=ue]


\pagestyle{plain} % en-tetes vides
%
% bgteubner class bundle
%
% index.tex
% Copyright 2003--2012 Harald Harders
%
% This program may be distributed and/or modified under the
% conditions of the LaTeX Project Public License, either version 1.3
% of this license or (at your opinion) any later version.
% The latest version of this license is in
%    http://www.latex-project.org/lppl.txt
% and version 1.3 or later is part of all distributions of LaTeX
% version 1999/12/01 or later.
%
% This program consists of all files listed in manifest.txt.
% ===================================================================
\index{Buchteil|see{Teil}}%
\index{Verzeichnis!Literatur-|see{Literaturverzeichnis}}%
\index{Verzeichnis!Formel-|see{Formelverzeichnis}}%
\index{Verzeichnis!Abk�rzungs-|see{Abk�rzungsverzeichnis}}%
\index{Verzeichnis!Abbildungs-|see{Abbildungsverzeichnis}}%
\index{Verzeichnis!Tabellen-|see{Tabellenverzeichnis}}%
\index{Verzeichnis!Beispiele|see{Verzeichnis der Beispiele}}%
\index{Verzeichnis!Aufgaben-|see{Aufgabenverzeichnis}}%
\index{Verzeichnis!Stichwort-|see{Stichwortverzeichnis}}%
\index{Schriftgr��e|see{Schriftgrad}}%
\index{Fotografie|see{Bild}}%
\index{Zeichnung|see{Bild}}%
\index{Diagramm|see{Bild}}%
\index{Teilbild|see{Bild\subind Teilbild}}%
\index{Tabellenunterschrift|see{Tabellen�berschrift}}%
\index{Textelement!Beispiel|see{Beispiel}}%
\index{Textelement!Aufgabe|see{Aufgabe}}%
\index{Textelement!L�sung|see{L�sung}}%
\index{Textelement!Anmerkung|see{Anmerkung}}%
\index{Textelement!Aufz�hlung|see{Aufz�hlung}}%
\index{Teilaufgabe|see{Aufgabe\subind Teil-}}%
\index{Index|seealso{Stichwortverzeichnis}}%
\index{Minuskelziffer|see{Medi�valziffer}}%
\index{Majuskelziffer|see{Versalziffer}}%
\index{Kapitel|seealso{Abschnitt}}%
\index{Abschnitt|seealso{Kapitel}}%
\index{Schriftgr��e|see{Schriftgrad}}%
\index{Ziffer!Medi�val-|see{Medi�valziffer}}%
\index{Ziffer!Versal-|see{Versalziffer}}%
\index{Auszeichnung|see{Hervorhebung}}%

% ===================================================================

%%% tabe of contents
%%% ---------------------------------------------------------------------------
\phantomsection
\pdfbookmark[0]{\contentsname}{pdf:toc}
%\settowidth{\cftpartnumwidth}{\cftpartfont III}
%\addtolength{\cftpartnumwidth}{0.5ex}
%\setlength{\cftchapnumwidth}{\cftpartnumwidth}
\tableofcontents
\clearpage % table des matières
\bookmarksetup{startatroot} % RAZ du niveau des signets PDF
\cleartoverso % saut vers une page verso
\pagestyle{empty}
\footnotesize

\paragraphe{Résumé}


\paragraphe{Abstract}




\end{document}
