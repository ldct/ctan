\chapitre*{Introduction}
\markright{Introduction}
\addcontentsline{toc}{chapter}{Introduction}

\verset{Liens internes!}
\label{liensinternes}
Un exemple de note en bas de page\footnote{Une note de bas de page.}. 
Des exemples de liens divers:
\begin{itemize}
\item \autoref{panom} d'intitulé \titleref{panom}, commençant à la page \pageref{panom} 
\item \autoref{tinom} d'intitulé \titleref{tinom}, commençant à la page \pageref{tinom}
\item \autoref{chnom} d'intitulé \titleref{chnom}, commençant à la page \pageref{chnom}
\item \ref{secnom} d'intitulé \titleref{secnom}, commençant à la page \pageref{secnom}
\item \ref{pgnom} d'intitulé \titleref{pgnom}, commençant à la page \pageref{pgnom}
\item \ref{spgnom} d'intitulé \titleref{spgnom}, commençant à la page \pageref{spgnom}
\item \ref{alnom} d'intitulé \titleref{alnom}, commençant à la page \pageref{alnom}
\item \ref{salnom} d'intitulé \titleref{salnom}, commençant à la page \pageref{salnom}
\item \ref{ptnom} d'intitulé \titleref{ptnom}, commençant à la page \pageref{ptnom}
\item \no{\ref{monverset}} d'intitulé \titleref{monverset}, commençant à la page \pageref{monverset}\fvref{monverset}
\end{itemize}

\verset{Citations d'ouvrages bibliographiques\ldots}
Première\cite{cass_ass_19910531}, seconde\cite[35]{malaurie_obligations} et troisième citation\cite[39]{malaurie_obligations} d'ouvrages.

Une citation d'un article du code civil\cite[1642-1]{cciv} suivie d'une autre\cite[1642-2]{cciv}. 

D'autres citations\cites{saintpern_latexdroitfr}{alland_dicoculturejur}.

Les citations fonctionnent également lorsqu'invoquées depuis une note de base de page \footnote{Comme on peut le voir ici: \cites{egea_fonctionjuger}[35]{malaurie_obligations}, ou bien là: \cite{saintpern_latexdroitfr}. Dans ce cas, il convient de noter qu'aucun point n'est automatiquement ajouté en fin de citation; il faut donc l'ajouter manuellement.}

\verset{Guillemets ?}
Voici un exemple d'utilisation des \enquote{guillemets français de premier et \enquote{second} niveau}.

\verset{Index}
Je cite ici certains termes qui seront ajoutés à l'index: épigénétique, empreinte génétique, filiation. C'est le numéro du présent verset qui sera indiqué dans l'index.
\indexv{epigenetique@épigénétique}
\indexv{empreinte génétique}
\indexv{filiation|(}

\verset{Index, suite}
Je cite ici certains termes qui seront ajoutés à l'index: présomption simple, épigénétique.
\indexv{présomption!simple}
\indexv{epigenetique@épigénétique}

\verset{Index, \emph{suite} et fin}
Je cite ici certains termes qui seront ajoutés à l'index: présomption irréfragable, expertise biologique, filiation.
\indexv{présomption!irréfragable}
\indexv{expertise biologique|seealso{empreinte génétique}}
\indexv{filiation|)}
\newpage
\verset{Renvois}
\label{monverset}
Je renvoie ici à un numéro de verset\footnote{Cf. \vref{liensinternes}.}.

\verset{Renvois}
\label{monverset2}
Je renvoie ici à un numéro de verset précédent\footnote{Cf. \vref{monverset}.}.



