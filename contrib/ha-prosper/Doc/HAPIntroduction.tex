\documentclass[pdf]{prosper}

% Introduction to the `HA-Prosper' package.
% Created by: Hendri Adriaens
%             http://center.uvt.nl/phd_stud/adriaens
%             Center for Economic Research
%             Tilburg University, the Netherlands

%================================================
% Please also read the manual of HA-prosper and
% of the specific style you are using since some
% features of this example might not be supported
% by the style you use.
%================================================

\usepackage[toc,highlight,HA]{HA-prosper}

\HAPsetup{%
trans=Wipe,
tsnav=FullScreen,
nsnav=ShowBookmarks,
lf={\href{http://center.uvt.nl/phd_stud/adriaens}{Hendri Adriaens}, \today},
rf={Dualslide example for the HA-prosper package},
iacolor=gray,
stype=1
}

\title{Introduction to the HA-prosper package}
\subtitle{A package for use with prosper}
\author{Hendri Adriaens\\
\institution{CentER}\\
\institution{\href{http://center.uvt.nl/phd_stud/adriaens}{http://center.uvt.nl/phd\string_stud/adriaens}}}


\begin{document}

% ==================================================================================
% Slide 1
\maketitle
% ==================================================================================


% ==================================================================================
% Slide 2
\tsectionandpart{Introduction}
% ==================================================================================


% ==================================================================================
% Slide 3
\overlays{2}{
\begin{slide}{Welcome}
\begin{itemstep}
  \xitem Welcome to the introduction of the HA-prosper package.
  \xitem Features of HA-prosper are:
  \begin{itemize}
    \xitem table of contents;
    \xitem portrait slides support;
    \xitem notes;
    \xitem dual slides;
    \xitem prosper bug solutions;
    \xitem and lots more.
  \end{itemize}
\end{itemstep}
\end{slide}
}
% ==================================================================================


% ==================================================================================
% Slide 4
\overlays{2}{
\begin{slide}{Styles}
\begin{itemstep}
  \xitem The HA-prosper packages adds functionality to prosper, but\dots
  \xitem The available styles show how to extend these possibilities
   even further and for instance:
  \begin{itemize}
    \xitem embed additional navigational elements on slides;
    \xitem use multiple slide layouts in the same presentation;
    \xitem etcetera.
  \end{itemize}
\end{itemstep}
\end{slide}
}
% ==================================================================================


% ==================================================================================
% Slide 5
\tsectionandpart{Features}
% ==================================================================================


% ==================================================================================
% Slide 6
\overlays{4}{
\begin{slide}{Table of contents}
\begin{itemstep}
  \xitem A table of contents can be put on every slide;
  \xitem A table of contents entry is created from the slide title
   or from text that is given as an optional argument;
  \xitem The table of contents has the following features:
  \begin{itemize}
    \xitem Highlighting of the current slide or section is possible;
    \xitem Items can be omitted;
    \xitem Parts of the table of contents can be hidden when these are
     unnecessary.
  \end{itemize}
  \xitem The style that you use should of course support the inclusion
   of the table of contents.
\end{itemstep}
\end{slide}
}
% ==================================================================================


% ==================================================================================
% Slide 7
\overlays{6}{
\begin{slide}{More features}
HA-prosper contains more features which are fully described in the manual.
\begin{itemstep}[sstart=2]
  \xitem Portrait slides;
  \xitem Notes;
  \xitem Dual slides;
  \xitem Blackslide;
  \xitem and a lot more\dots
\end{itemstep}
\end{slide}
}
% ==================================================================================


% ==================================================================================
% Slide 8
\overlays{5}{
\begin{slide}{Prosper bug solutions}
\begin{itemstep}
  \xitem Numbering of equations, tables and figures on overlays is supported be default;
  \xitem Custom counters can be protected throughout the presentation;
  \xitem The `\textbackslash and' command for authors is supported;
  \xitem Improved placement of left and right footers;
  \xitem and more.
\end{itemstep}
\end{slide}
}
% ==================================================================================


% ==================================================================================
% Slide 9
\tsectionandpart{Contributions and questions}
% ==================================================================================


% ==================================================================================
% Slide 10
\overlays{3}{
\begin{slide}{Contributions}
\begin{itemstep}
  \xitem Contributions are always welcome.
  \xitem In case you want to contribute a style, which offers undocumented
   features, please insert both
  \begin{itemize}
    \xitem documentation;
    \xitem an example that demonstrates all the features that your style provides.
  \end{itemize}
  \xitem You can contact
    \href{http://stuwww.uvt.nl/~hendri/Personal/contact.html}{\underline{Hendri Adriaens}}
    (click the link) in case you have comments or questions about
    developing a new template or if want to submit your style.
\end{itemstep}
\end{slide}
}
% ==================================================================================


% ==================================================================================
% Slide 11
\overlays{2}{
\begin{slide}{Questions}
\begin{itemstep}
  \xitem If you have questions, please first consult the documentation
   of the following packages (depending on your problem):
  \begin{itemize}
    \xitem HA-prosper;
    \xitem Prosper;
    \xitem Hyperref;
    \xitem PSTricks.
  \end{itemize}
  \xitem In case you cannot find the answer and your question is related
   to HA-prosper, you can post a message to the HA-prosper mailinglist:\par
  \href{http://listserv.surfnet.nl/archives/ha-prosper.html}{http://listserv.surfnet.nl/archives/ha-prosper.html}.
\end{itemstep}
\end{slide}
}
% ==================================================================================


\end{document}
