\documentclass[a4paper]{scrartcl}
\usepackage[utf8]{inputenc}
\usepackage[ngerman]{babel}
%\usepackage[usenames,dvipsnames,svgnames,table]{xcolor}
\definecolor{blau}{rgb}{0,0,0.75}         
\definecolor{orange}{rgb}{0.8,0.3,0}      
\usepackage{hyperref}
\hypersetup{
 pdftitle = {\LaTeX-Klassen und Pakete für den Einsatz im Bereich der Schule},
 pdfsubject = {},
 pdfauthor = {Johannes Pieper, Johannes Kuhaupt, Ludger Humbert, Andr\'e Hilbig},
 colorlinks = true,
 hypertexnames = true,
 linkcolor=blau, %
 filecolor=orange, %
 citecolor=blau,
 menucolor=orange, %
 urlcolor=orange,
 breaklinks=true
}
\usepackage{caption,xparse,xargs}
\usepackage[german=guillemets]{csquotes}
\usepackage{schule,syntaxdi,schulinf,schulphy}
\usepackage{placeins,float,prettyref}
\usepackage{newfloat}

\lstset{  %
	language=[LaTeX]TeX,                 % the language of the code
	basicstyle=\small,            % the size of the fonts that are used for the code
	numbers=left,                    % where to put the line-numbers
	numberstyle=\footnotesize,           % the size of the fonts that are used for the line-numbers
	stepnumber=2,                    % the step between two line-numbers. If it's 1, each line will be numbered
	numbersep=5pt,                   % how far the line-numbers are from the code
	backgroundcolor=\color{white},       % choose the background color. You must add \usepackage{color}
	showspaces=false,                % show spaces adding particular underscores
	showstringspaces=false,          % underline spaces within strings
	showtabs=false,                  % show tabs within strings adding particular underscores
	frame=false,                    % adds a frame around the code
	tabsize=2,                       % sets default tabsize to 2 spaces
	resetmargins=true,
	captionpos=b,                    % sets the caption-position to bottom
	title=,                    % show the filename of files included with \lstinputlisting;
	breaklines=true,
	breakautoindent=true,
	prebreak=\mbox{ $\curvearrowright$},
	postbreak=\mbox{$\rightsquigarrow$ },
	linewidth=\columnwidth,
	breakatwhitespace=true,         % sets if automatic breaks should only happen at whitespace 
	numberstyle=\tiny\color{gray},         % line number style
	keywordstyle=\color{blue},           % keyword style
	commentstyle=\color{OliveGreen},        % comment style
	stringstyle=\color{mauve},          % string literal style
	morekeywords={
	zeitpunkt, punkteitem, scaleSequenzdiagramm, newthread, newthreadtwo, 
	newinst, node, chainin, draw, to, dokName, jahrgang, minisec, subsection, 
	glqq, grqq, euro
}                % if you want to add more keywords to the set
}
\newcommand{\materialsammlung}{\url{http://ddi.uni-wuppertal.de/material/materialsammlung/index.html}}

\begin{document}
 \section*{Aufgabenumgebung -- u.\,a. automatische Zuordnung der Punkte}
 \begin{lstlisting}[gobble=1,caption={}]
 \begin{aufgaben}
    \item Erstellen Sie aus dem obigen Text mit Hilfe der
       Methode nach Abbott ein Objektdiagramm. Berücksichtigen 
       Sie dabei auch die Bezugsobjekte. Verwenden Sie  
       nur Bezeichner gemäß der Vorgaben aus dem Unterricht
    \punkteitem{2} Geben Sie eine allgemeingültige und
       fachlich korrekte Definition eines Informatiksystems an.
    \punkteitem{10} Nennen Sie die Fachgebiete der 
       Fachwissenschaft Informatik und geben Sie pro Fachgebiet 
       ein Anwendungsbeispiel an.
 \end{aufgaben}
 \end{lstlisting}

 \vspace{0.6cm}
 \hrule width \textwidth
 \vspace{1cm}

 \begin{aufgaben}
    \item Erstellen Sie aus dem obigen Text mit Hilfe der Methode 
       nach Abbott ein Objektdiagramm. Berücksichtigen Sie dabei 
       auch die Bezugsobjekte. Verwenden Sie  nur Bezeichner 
       gemäß der Vorgaben aus dem Unterricht.
    \punkteitem{2} Geben Sie eine allgemeingültige und fachlich 
       korrekte Definition eines Informatiksystems	an.
    \punkteitem{10} Nennen Sie die Fachgebiete der Fachwissenschaft 
       Informatik und geben Sie pro Fachgebiet ein Anwendungsbeispiel 
       an.
 \end{aufgaben}
\end{document}
