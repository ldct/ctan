\documentclass[11pt,a4paper]{article}

\setlength{\parskip}{0.6ex}
\setlength{\parindent}{0mm}
\usepackage{mathptmx,courier}
\usepackage[scaled=0.92]{helvet}
%\usepackage{german}
\usepackage[latin1]{inputenc}
\usepackage{textcomp}
%% \usepackage[a4paper,hmargin=2.4cm]{geometry}
\usepackage[a4paper,hmargin=2.4cm,bottom=3.2cm]{geometry}
\newcommand{\hoch}[1]{\raisebox{0.6ex}{#1}}
\newcommand{\tief}[1]{\raisebox{-0.6ex}{#1}}
\usepackage{url}

\title{\textsf{Elpres}: electronic presentations with (PDF)\LaTeX}
\author{Volker Kiefel\thanks{volker dot kiefel at freenet dot de}}
\date{v0.3}
\begin{document}
\maketitle
\thispagestyle{empty}

{
\setlength{\parskip}{0ex}
\tableofcontents
}

\section{Introduction}

\textsf{Elpres} is a simple class for presentations to be shown on
screen or beamer. It is derived from \LaTeX's \texttt{article} class.
\textsf{Elpres} is primarily intended to be used with PDF\LaTeX{} or with \LaTeX,
\texttt{dvips} and Ghostview/Ghostscript. The ``virtual paper
size'' of documents produced by this class: width=128mm, height=96mm.
\textsf{Elpres} requires that the \texttt{fancyhdr} and \texttt{geometry}
packages are available on the system. Enhancements to the elpres class are
easily made available by other packages, these include hypertext elements
(\texttt{hyperref} package) and slides with a background from a bitmap
(\texttt{wallpaper}, \texttt{eso-pic} packages).


\section{Installation}

Copy \texttt{elpres.cls} into  a directory, where your \LaTeX-system can find
it.

\section{Usage}

The class is used with 
\begin{verbatim}
   \documentclass[options]{elpres}
\end{verbatim}

\textbf{Options of the \texttt{article} class} are also available to elpres,
e.\,g.  \texttt{10pt}, \texttt{11pt}, \texttt{12pt} for selection of font
size.  \textbf{Elpres-specific options} allow selection of the font:
\texttt{tmrfont} (Times Roman), \texttt{helvetfont} (Helevetica),
\texttt{cmfont} (Computer Modern), \texttt{sansfont} (Sans Serif: default).
However, not all options of the \texttt{article} class will be appropriate for
a presentation class, e.\,g.  \texttt{twocolumn}.

A simple example document:


\begin{verbatim}
   \documentclass[12pt,pdftex,helvetfont]{elpres}
   \usepackage[latin1]{inputenc}
   \usepackage{color}
   \usepackage[document]{ragged2e}
   \RaggedRight

   \begin{document}
   \begin{titlepage}
     \centering
     \distance{1}
     {
     \Huge \bfseries \textcolor{blue}{Title of the presentation} \par
     }
     \vspace{1.3ex} \large
     Author\\[2ex]Institution
     \distance{2}
   \end{titlepage}

   \begin{psli}[Title of Page]
   The first page
   \end{psli}

   \begin{rsli}
   The second page
   \end{rsli}
   \end{document}
\end{verbatim}

%% Options to be used are \texttt{10pt}, \texttt{11pt}, \texttt{12pt} (font size),
%% \texttt{pdftex}, \texttt{dvips}
%% (if you use \texttt{graphicx} and/or \texttt{color} packages).

The title page can be created within the \texttt{titlepage} environment, the 
\verb+\maketitle+ command is not available.
Slides may be created with the
\texttt{psli}-environment\footnote{\texttt{psli}: plain slide}, you may add
the title of the slide with the optional parameter. 
The contents of the slide are centered vertically.

Another environment generating a slide is
\texttt{rsli}\footnote{\texttt{rsli}: raw slide}: slides are written 
without title,
contents are not vertically centered. 

The \verb+\distance{number}+ command allows to introduce vertical space into
slides constructed with the \texttt{rsli} and \texttt{titlepage} environments.
You should use pairs of \verb+\distance{}+ commands with numbers indicating
the relative height of empty space, see the titlepage in the example above.

The package provides a ``vertically compressed'' \texttt{itemize}-environment:

\begin{verbatim}
   \begin{citemize}
     \item  one
     \item  two
   \end{citemize}
\end{verbatim}

Similarly, a \texttt{cenumerate} and a \texttt{cdescription} environment may
be used.

Pictures can be included with the \texttt{includegraphics}-command of the
\texttt{graphicx}-package. Please be aware that the dimensions of the pages
are 128mm \texttimes{} 96mm and therefore included graphics are scaled
appropriately.

\section{Enhancements to elpres}

\label{secEnhancements}

\subsection{Include graphics files}

\label{secInclGraphicsFiles}

Graphics files may be included with the \texttt{includegraphics} command of
the \texttt{graphicx} package. If you create pdf-files with \texttt{pdflatex},
both .pdf and .png files can be included, if you create pdf files with
\LaTeX{} and \texttt{dvipdfm} or if you create ps files with \LaTeX{} and
\texttt{dvips} you may include .eps files:

\begin{verbatim}
   \usepackage{graphicx} % (in preamble) 
   ...
   \includegraphics[width=0.9\textwidth]{graphics-file.png} 
                                                            
\end{verbatim}

\subsection{Arrange text and pictures in two (or more) columns}

\label{secTwoColumns}

Text and graphics may be arranged in two or more columns with
\texttt{minipage} environments:

\begin{verbatim}
   \begin{minipage}[b][0.8\textheight][t]{0.5\textwidth}
     \colorbox{white}{%
       \includegraphics[width=0.9\textwidth]{graphics-file.png}}
   \end{minipage}
   \begin{minipage}[b][0.8\textheight][t]{0.48\textwidth}
   \footnotesize
   \begin{citemize}
      \item ...
      \item ...
      ...
   \end{citemize}
   \end{minipage}
\end{verbatim}

Details on the minipage environment may be found in the \LaTeX{}
documentation.

\subsection{Create a ``handout'' from a presentation}

\label{secHandout}

If you wish to create a handout from your presentation, you should create a
PostScipt version of your presentation and convert ot with the
\texttt{psnup}-tool. This is possible, if you create your presentation as
\texttt{.ps} file from a \texttt{.dvi}-file with \texttt{dvips}. If your
primary version is a pdf-file, e.\,g. created by pdf\LaTeX, either

\begin{enumerate}
  \item convert the pdf-file with the command \texttt{pdf2ps} (part of the
  ghostscript package)
  \item or convert it with Acrobat.
\end{enumerate}


The following command creates a handout with four slides on one page with the
\texttt{psnup}-command line program%
\footnote{A win32-version of this tool can be obtained from the website
\url{http://people.freenet.de/vkiefel/compiled-SW.html}. \texttt{Psnup} may
be bundled to your \TeX/\LaTeX{} distribution}.

\begin{verbatim}
   psnup -4 -H96mm -W128mm -m15mm -b6mm old.ps new.ps
\end{verbatim}

Details of the command line options can
be found in the short documentation of \texttt{psnup}. You may print
\texttt{new.ps} with \texttt{ghostview} or \texttt{gsview32/ghostview}.

\subsection{Create presentations with hypertext elements}

\label{secHypertext}

You may use the hyperref package. As you normally will not insert
\verb+\section{}+-like commands, it is easier to define links with

\begin{verbatim}
   \hypertarget{target-name}{text}
\end{verbatim}

which can be addressed by

\begin{verbatim}
   \hyperlink{target-name}{text}
\end{verbatim}

The hyperref package will produce a warning message, if you use the
titlepage-environment (this is inherited from the article class). To avoid
the warning you can use the \texttt{rsli}-environment for the titlepage and use
\verb+\thispagestyle{empty}+ to suppress the page number on the title.

\subsection{Fill background of a presentation with bitmaps}


\label{secBackgroundWallpaper}

\subsubsection{\texttt{Wallpaper} package}

To create a background with color gradient, with pictures or with a ``tiled''
background using bitmaps you may use the \texttt{wallpaper}
package\footnote{written by Michael H.F. Wilkinson and available on CTAN}. 
Load the \texttt{wallpaper} package with

\begin{verbatim}
   \usepackage{wallpaper}
\end{verbatim}

\begin{sloppypar}
in the preamble. In order to generate a background gradient on the basis of the bitmap
file \texttt{gradient2.png}\footnote{included in this 
\texttt{elpres}-distribution} enter 
\end{sloppypar}

\begin{verbatim}
   \CenterWallPaper{1}{gradient2}
\end{verbatim}

before the contents of the presentation\footnote{i.\,e. following
\texttt{\textbackslash begin\{document\}}}. 
This works best with bitmaps with a width:size ratio of 4:3, the included
bitmap files have a size of 640:480 pixel.
Similarly bitmap files may be
used as tiles as described in the \texttt{wallpaper} documentation like

\begin{verbatim}
   \TileSquareWallPaper{4}{TGTamber}
\end{verbatim}

More details on this topic may be found in the \texttt{wallpaper}
documentation.

\subsubsection{\texttt{Eso-pic} package}
\label{secBackgroundEsopic}

Another package which allows you to paint the background with a picture is
\texttt{eso-pic}\footnote{written by Rolf Niepraschk and available on CTAN}:

\begin{verbatim}
   \usepackage{eso-pic}
   
   ...

   \AddToShipoutPicture{
   \includegraphics[height=\paperheight]{gradient2.png}
   }
\end{verbatim}

\verb+\AddToShipoutPicture{}+ puts the picture on every page, 
\verb+\AddToShipoutPicture*{}+ puts it on to the current page,
\verb+\ClearShipoutPicture+ clears the background beginning with the current
page.
Details of \texttt{eso-pic}'s commands can be found in the documentation.


\section{License}

This class is distributed under the \textsl{\LaTeX{} Project Public License}
(LPPL) which may be downloaded from
\url{http://www.latex-project.org/lppl.txt}. No warranty is provided for this
work (as stated in the LPPL).

\section{Versions}

\textbf{v0.1} (19.6.2004):  initial version.
\textbf{v0.2} (1.9.2004): page numbers now changed to footnotesize, left and
right margins slightly changed, `cenumerate' and `cdescription' environments
added.
\textbf{v0.2a} (19.9.2004): Section ``License'' added to the documentation.
\textbf{v0.2b} (17.10.2004): Documentation completed: description of the
\verb+\distance{}+ command included.
\textbf{v0.2c} (28.11.2004): Documentation completed (section
\ref{secHypertext} 
added).
\textbf{v0.2d} (25.12.2004): Documentation completed (section
\ref{secBackgroundWallpaper} added).
\textbf{v0.2e} (15.04.2005): Documentation completed (sections
\ref{secBackgroundEsopic} and \ref{secHandout} added).
\textbf{v0.3} (12.08.2005): new  (class) options for font selection:
\texttt{tmrfont} (Times Roman), \texttt{helvetfont} (Helevetica),
\texttt{cmfont} (Computer Modern), \texttt{sansfont} (Sans Serif: default).
Documentation updated, sections \ref{secInclGraphicsFiles} and
\ref{secTwoColumns} added.
\end{document}
