\documentclass[pagesize=auto, fontsize=12pt, DIV=10, parskip=full, american, french]{scrartcl}

\usepackage{fixltx2e}
\usepackage{etex}
\usepackage{lmodern}
\usepackage[T1]{fontenc}
\usepackage{textcomp}
\usepackage{babel}
\usepackage{csquotes}
\usepackage{microtype}
\usepackage{hyperref}

\newcommand*{\mail}[1]{\href{mailto:#1}{\texttt{#1}}}
\newcommand*{\pkg}[1]{\textsf{#1}}
\newcommand*{\cs}[1]{\texttt{\textbackslash#1}}
\makeatletter
\newcommand*{\cmd}[1]{\cs{\expandafter\@gobble\string#1}}
\makeatother
\newcommand*{\meta}[1]{\textlangle\textsl{#1}\textrangle}
\newcommand*{\marg}[1]{\texttt{\{}\meta{#1}\texttt{\}}}

\addtokomafont{title}{\rmfamily}

\title{Le package \pkg{smalltableof}}
\author{Boretti Mathieu }
\date{2003}


\begin{document}

\maketitle

Ce fichier descrit le package \enquote{\pkg{smalltableof}} :

Ce package permet de gerer des tables de figure, de tables, qui ne sont pas des chapitres, ainsi, il est possible d'avoir toutes les tables/listes de ... dans le meme chapitre :\\
Les commandes fournies sont :\\
\cmd{\chapterNoNumber}\marg{nom du chapitre}\\
Qui cree un chapitre, sans numerotation, mais qui est dans la table des matieres.

\cmd{\sectionNoNumber}\marg{nom de la section}\\
Qui cree une section, sans numerotation, sans en-tete, mais qui est dans la table des matieres.

\cmd{\toc}\marg{extension de la table}\\
Qui insere la table/liste dans le document (par exemple : \verb|\toc{lof}| insere la liste des figures dans le document.

\cmd{\sectiontable}\marg{Titre}\marg{extension de la table}

\cmd{\sectiontableoffigure}\\
Qui insere la liste des figures sous la forme d'une section

\cmd{\sectiontableoftable}\\
Qui insere la liste des tables sous la forme d'une section

\cmd{\tablechapter}\\
Qui cree un chapitre nomme par \cmd{\tablesname}, sans numerotation, pour mettre les tables.

\begin{sloppypar}
  \cmd{\stdtables}\\
  Qui cree un chapitre tables avec les listes de figures et de tables. On peut toujours ajouter d'autre listes avec \verb|\sectiontable{...}{...}|, par exemple, \verb|\sectiontable{Algorithme}{loa}| qui cree une liste des algorithmes sous la forme d'une section.
\end{sloppypar}

\cmd{\mabibliographie}\marg{style}\marg{fichier}\\
Qui insere une bibliographie, qui apparait dans la table des matieres

Ce package a ete cree par Mathieu Boretti\\
@Boretti Mathieu 2003

\selectlanguage{american}
This material is subject to the \LaTeX\ Project Public License. See \url{http://www.ctan.org/tex-archive/help/Catalogue/licenses.lppl.html} for the details of that license

\end{document}
