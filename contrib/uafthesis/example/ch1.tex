\chapter{What's All This, Then?}

\section{Introduction}

This is the project for the \texttt{uafthesis} \LaTeX document class, the
official unofficial volunteer-driven document class for theses written for the
University of Alaska Fairbanks, in \LaTeX.

\section{Prerequisites}

Before you decide to write your thesis in \LaTeX, you should already know a
little \LaTeX and feel comfortable writing and compiling a simple document,
especially one with figures and tables.

Additionally, you will have to use \texttt{bibtex} (or some alternative) to keep
track of references. It's not particularly difficult, but chances are you will
have to either learn it or give yourself a refresher.

One thing you will see in this example that you may not have encountered before
is the use of \textbackslash input and \textbackslash include to split the project into multiple, smaller
files. The most important difference is that \textbackslash include will wrap the included
file in page-breaks, while \textbackslash input may as well be a copy-paste job.

Another issue that may come up is having to run \texttt{pdflatex} and
\texttt{bibtex} multiple times. In order to do a full compile without anything
wrong, you will have to run something like the following commands:

\begin{verbatim}
pdflatex example
pdflatex example
pdflatex example
bibtex example
pdflatex example
pdflatex example
\end{verbatim}

That's right, \texttt{pdflatex} gets ran five times. A similar situation arises
from the use of vanilla \texttt{latex}. This is because of how \LaTeX generates
files while compiling that it uses to fill in data during subsequent 
run-throughs.

This can be mitigated somewhat by using some sort of build system. For example,
Ryan Woodard advocated using a makefile to ease the pain. Another option may
be Rubber, or even the full set of commands in a shell script. There are many
techniques, some more appropriate than others. \cite{buildsystems}

It is also worth your time to read other theses, to get an idea of how they
should be written. This may seem obvious, but I will admit that I did not, and
I ended up going through many a rewrite. Also obvious: This document is
\emph{not} written like a thesis. Duh.

Finally, \textbf{at least skim the UAF Thesis Handbook}. \cite{handbook} It's
not hard, and it will give you an idea of what to expect in terms of formatting.
In particular, as \texttt{uafthesis} is a volunteer effort, \emph{there is no
guarantee that the graduate school's formatting guidelines are met by this
document class.} Moreover, some things (such as proper initial capitals in title
headings) are on you, and not something \texttt{uafthesis} does for you.


\section{Installation}

Like any \LaTeX files, there are basically two ways:

\begin{itemize}
\item Copy \texttt{uafthesis.cls} into the same folder as your project. This is
probably the easiest way.
\item Set yourself up with a properly indexed \texttt{~/texmf/latex} folder,
create a new folder called ``uafthesis,'' and put \texttt{uafthesis.cls} into
that folder. This involves some initial effort, but if you use \LaTeX regularly
it's worthwhile for holding all sorts of packages. In fact, if you are a
regular \LaTeX user, you may have already done this.
\end{itemize}

Which method you choose is up to you.
