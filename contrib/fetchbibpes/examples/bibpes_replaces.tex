\documentclass{article}
% useverses=none avoids a warning in the log when there is not value for
% userverses. In this document we define our verses 'by hand' using the
% declareBVs environment.
\usepackage[ignorecfg,useverses=none]{fetchbibpes}[2016/09/21]
\usepackage{fancyvrb}

\providecommand\cs[1]{\texttt{\char`\\#1}}
\let\pkg\texttt
\def\ameta#1{$\langle\textit{\texttt{#1}}\rangle$}


\addtoBibles{NKJV}
\defaultBible{NKJV}


\begin{declareBVs}
\BV(Gal 1:15 NKJV) But when it pleased God, who separated me from my mother's womb and called me through His grace,\null
\BV(Gal 1:16 NKJV) to reveal His Son in me, that I might preach Him among the Gentiles, I did not immediately confer with flesh and blood,\null
\end{declareBVs}

\begin{document}

\section{Testing replace with \cs{fetchverses}}

This demonstration file features the \texttt{replace} key within the
optional argument of \cs{fetchverses}. We use as our test passes \textsf{Gal
1:15-16}.

\medskip\noindent\textbf{Unmodified passage}
\begin{quote}
\eSQ\fetchverses*{Gal 1:15-16}
\end{quote}

\medskip\noindent
We declare the following verses:
\begin{Verbatim}[xleftmargin=\leftmargini,fontsize=\small]
\fetchverses[%
    replace={But when}{\leavevmode\ldots when}
        {blood,}{blood, \ldots (\textit{continues with verse 17})}
]{Gal 1:15-16}
\end{Verbatim}
which displays as,
\begin{quote}
\fetchverses[%
    replace={But when}{\leavevmode\ldots when}
        {blood,}{blood, \ldots (\textit{continues with verse 17})}
]{Gal 1:15-16}
\end{quote}

\medskip\noindent
A more extensive example of \texttt{replace} is the following one, which requires
special techniques.
\begin{Verbatim}[xleftmargin=\leftmargini,fontsize=\small]
\gobbleto{and}{\gobbletoand}%
\fetchverses*[replace={But when}{\leavevmode\ldots when}
    {who separated}{who\gobbletoand}
%   {who separated me from my mother's womb and}{who}
    {, that I might preach Him among the Gentiles,
        I did not immediately confer with flesh and blood,}
    { \ldots}
]{Gal 1:15-16}
\end{Verbatim}
\begin{quote}
\gobbleto{and}{\gobbletoand}%
\fetchverses*[replace={But when}{\leavevmode\ldots when}
     {who separated}{who\gobbletoand}
%    {who separated me from my mother's womb and}{\textbf{who}}
    {, that I might preach Him among the Gentiles,
        I did not immediately confer with flesh and blood,}{ \ldots}
]{Gal 1:15-16}
\end{quote}
The original \ameta{find} text is `\texttt{who separated me from my mother's
womb and}'. The fact that this \ameta{find} text contains an apostrophe is a
problem. Apostrophises (right single quotes), left single quotes, and double
quotes are \emph{active characters}  (within the Bible passages);
consequently, they prevent a correct match. To circumvent this problem, we
gobble all text past any active characters (an apostrophe, in this example).
This is where \cs{gobbleto} comes in. Declare
\verb|\gobbleto{and}{\gobbletoand}|, this defines a macro \cs{gobbletoand}
that gobbles all text from where it is placed to the next occurrence of the
word `and', in this example. Now, instead of using a \ameta{find} text
containing an apostrophe, we use `\texttt{who separated}' as the \ameta{find}
text, and for the \ameta{replace} text, we use `\texttt{who\cs{gobbletoand}},
which, incredibly, works!

\medskip
The use of the gobble trick is not restricted to passages containing active
characters. For example, the third \ameta{find} text is rather long. We can do the
following:
\begin{Verbatim}[xleftmargin=\leftmargini]
\gobbleto{and}{\gobbletoand}%
\gobbleto{blood,}{\gobbletoblood}%
\fetchverses*[replace=
    {But when}{\leavevmode\ldots when}
    {who separated me}{who\gobbletoand}
    {, that I might}{\ldots{}.\gobbletoblood}
]{Gal 1:15-16}
\end{Verbatim}
\begin{quote}\eSQ
\gobbleto{and}{\gobbletoand}\gobbleto{blood,}{\gobbletoblood}
\fetchverses*[replace=
    {But when}{\leavevmode\ldots when}
    {who separated me}{who\gobbletoand}
    {, that I might}{ \ldots\gobbletoblood}
]{Gal 1:15-16}
\end{quote}

To make the `\cs{gobbleto}' definitions local, put them in a group, as in the \texttt{quote} environment as above, or
they can be passed into a group in which \cs{fetchverse} and \cs{fetchverses} operate using the \texttt{localdefs} key.
\begin{Verbatim}[xleftmargin=\leftmargini,fontsize=\small]
\fetchverses*[%
    localdefs={\gobbleto{and}{\gobbletoand}%
        \gobbleto{blood,}{\gobbletoblood}},
    replace={But when}{\leavevmode\ldots when}
        {who separated me}{who\gobbletoand}
        {, that I might}{\ldots{}.\gobbletoblood}
]{Gal 1:15-16}
\end{Verbatim}
\eSQ
\fetchverses*[%
    localdefs={\gobbleto{and}{\gobbletoand}%
        \gobbleto{blood,}{\gobbletoblood}},
    replace={But when}{\leavevmode\ldots when}
    {who separated me}{who\gobbletoand}
    {, that I might}{ \ldots\gobbletoblood}
]{Gal 1:15-16}

\end{document}
