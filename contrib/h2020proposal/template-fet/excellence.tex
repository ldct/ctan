%%% Important. To have correct table numberings
\renewcommand{\thetable}{\thesection\alph{table}}

\chapter[Excellence]{S\&T Excellence}
\label{cha:excellence}
\instructions{
Your proposal must address a work programme topic for this call for proposals. \\
\textit{This section of your proposal will be assessed only to the extent that it is relevant to that topic.}\\
}

\section{Targeted breakthrough, Long term vision and Objectives}
\label{sec:objectives}
\instructions{
\begin{itemize}
\item Describe the targeted scientific breakthrough of the project.
\item Describe how the targeted breakthrough of the project contributes to a long-term vision for new technologies.
\item Describe the specific objectives for the project, which should be clear, measurable, realistic and achievable within the duration of the project.
\end{itemize}
}

\section{Relation to the work programme}
\label{sec:relation-to-work-programme}
\instructions{
\begin{itemize}
\item Indicate the work programme topic to which your proposal relates, and explain how your proposal addresses the specific challenge and scope of that topic, as set out in the work programme.
\end{itemize}
}


\section{Novelty, level of ambition and foundational character}
\label{sec:ambition}
\instructions{
\begin{itemize}
\item Describe the advance your proposal would provide beyond the state-of-the-art, and to what extent the proposed work is ambitious, novel and of a foundational nature. Your answer could refer to the ground-breaking nature of the objectives, concepts involved, issues and problems to be addressed, and approaches and methods to be used. 
\end{itemize}
}

\section{Research methods}
\label{sec:methods}
\instructions{
\begin{itemize}
\item Describe the overall research approach, the methodology and explain its relevance to the objectives.  
\item Where relevant, describe how sex and/or gender analysis is taken into account in the project’s content.
\end{itemize}
\textit{Sex and gender refer to biological characteristics and social/cultural factors respectively. For guidance on methods of sex/gender analysis and the issues to be taken into account, please refer to \url{http://ec.europa.eu/research/science-society/gendered-innovations/index_en.cfm}}
}

\section{Interdisciplinary nature}
\label{sec:interdisciplinary}
\instructions{
\begin{itemize}
\item Describe the research disciplines involved and the added value of the inter-disciplinarity.
\end{itemize}
}
