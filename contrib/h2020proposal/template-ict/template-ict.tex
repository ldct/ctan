%% EU ICT Proposal LaTeX template
%% V1.0
%% Based on the h2020proposal.cls LaTeX class for writing EU H2020 RIA proposals.
%% 
%% Copyright (c) 2010, Giacomo Indiveri
%%
%%  This latex class is free software: you can redistribute it and/or modify
%%  it under the terms of the GNU General Public License as published by
%%  the Free Software Foundation, either version 3 of the License, or
%%  (at your option) any later version.
%%
%%  h2020proposal.cls is distributed in the hope that it will be useful,
%%  but WITHOUT ANY WARRANTY; without even the implied warranty of
%%  MERCHANTABILITY or FITNESS FOR A PARTICULAR PURPOSE.  See the
%%  GNU General Public License for more details.
%%
%%  You should have received a copy of the GNU General Public License
%%  along with h2020proposal.cls.  If not, see <http://www.gnu.org/licenses/>.
%%
%% Contributors: Elisabetta Chicca
%%
%% Disclaimer: The template is based on the document provided by the EU Participants Portal 
%% h2020-call-pt-ria-ia_en, Version 1.4, 21 May 201
%%
%% Use the original source and the http://ec.europa.eu/ documentation for reference. We make no
%% representations or warranties of any kind, express or implied, about the completeness, accuracy,
%% reliability, suitability or availability with respect to the original template.
%% In no event will we be liable for any loss or damage including without limitation, indirect or
%% consequential loss or damage, or any loss or damage whatsoever arising out of, or in connection
%% with, the use of this template and/or class.
%%
%% Makes use of the memoir class. Read the optimum memman documentation for
%% info on how to customize your proposal.


%\documentclass[]{h2020proposal}     % Remove 'draft' option for final version
\documentclass[draft]{h2020proposal} % Use 'draft' option to show comments and labels

% For in-line comments use:
% \marginpar{comment text}

%% Extra Packages
%% ========
%\usepackage{fontspec}% Latin Modern by default with xelatex

%% LaTeX Font encoding -- DO NOT CHANGE
\usepackage[OT1]{fontenc}

%% Input encoding 'utf8'. In some cases you might need 'utf8x' for
%% extra symbols. Not all editors, especially on Windows, are UTF-8
%% capable, so you may want to use 'latin1' instead.
%\usepackage[utf8,latin1]{inputenc}

%% Babel provides support for languages.  'english' uses British
%% English hyphenation and text snippets like "Figure" and
%% "Theorem". Use the option 'ngerman' if your document is in German.
%% Use 'american' for American English.  Note that if you change this,
%% the next LaTeX run may show spurious errors.  Simply run it again.
%% If they persist, remove the .aux file and try again.
\usepackage[english]{babel}

%% For underlined wrapped text.
\usepackage{soul}

%% This changes default fonts for both text and math mode to use Herman Zapfs
%% excellent Palatino font.  Do not change this.
\usepackage[sc]{mathpazo} % Not needed with xelatex

%% The AMS-LaTeX extensions for mathematical typesetting.  Do not
%% remove.
\usepackage{amsmath,amssymb,amsfonts,mathrsfs}

%% Gantt Charts in LaTeX
\usepackage{pgfgantt}

%% LaTeX' own graphics handling
\usepackage{graphicx}

%% Fancy character protrusion.  Must be loaded after all fonts.
\usepackage[activate]{pdfcprot}

%% Nicer tables.  Read the excellent documentation.
\usepackage{booktabs}

% Compressed itemized lists (with a * at the end)
\usepackage{mdwlist}

%% Nicer URLs.  
\usepackage{url}

%% Configure citation styles
\usepackage[numbers,sort&compress,square]{natbib}
\def\bibfont{\footnotesize}     %for smaller fonts in the biblio section

%% Hyper Ref package. In order to operate correctly, it must be the last package declared
\usepackage[colorlinks,pagebackref,breaklinks]{hyperref} 

%% Extra package options
\hypersetup{
  hypertexnames=true, linkcolor=blue, anchorcolor=black,
  citecolor=blue, urlcolor=blue  
}

\urlstyle{rm} %so it doesn't use a typewriter font for urls.
\DeclareGraphicsExtensions{.jpg,.pdf,.mps,.png} % for pdflatex
\graphicspath{{img/} {./}} %put all figures in these dirs

\newcommand{\alert}[1]{{\color{red}\textbf{#1}}}

%%%%%%%%%%%%%%%%%%%%%%%%%%%%%%%%%%%%%%%%%%%%%%%%%%%%%%%%%%%%%%%%%%%%%%

%% ========================================================
%% IMPORTANT store proposal information in global variables
%% ========================================================
\title{Full Title of the Proposal ($<$200 chars)}
\shortname{ACRONYM} 
\titlelogo{}{0.25} % file name and scale
\fundingscheme{Research and Innovation Action}
\topic{Work Programme topic addressed}
\coordinator{Name of coordinator}{email}{fax}
\participant{University of Coordinator}{UoC}{Country1} % First participant is the coordinator
\participant{University of partner 2}{UoP2}{Country2} % as example...
\participant{University of partner 3}{UoP3}{Country3} % as example...
% etc.

% Page Headers
%\makeoddhead{proposal}{\disptoken{@acronym}}{}{\rightmark}
%\makeevenhead{proposal}{\leftmark}{}{\disptoken{@acronym}}

%Page Footers
%\makeevenfoot{proposal}{ \thepage }{ \date{\today} }{ \disptoken{@acronym} }
%\makeoddfoot{proposal}{  }{ \date{\today} }{ \thepage }

%Page Style
\pagestyle{proposal} %use \pagestyle{showlocs} for debugging

%Heading style 
\makeheadstyles{default}{%
\renewcommand*{\chapnamefont}{\normalfont\bfseries}
\renewcommand*{\chapnumfont}{\normalfont\bfseries}
\renewcommand*{\chaptitlefont}{\normalfont\bfseries}
\renewcommand*{\secheadstyle}{\normalfont\bfseries}
}%
\headstyles{default}

%Chapter Style
\chapterstyle{section} %Avoid writing the word "Chapter" at the beginning of each proposal section
% other possible valid styles:
% article, bringhurst, crosshead, culver, dash, demo2, ell, southall, tandh, verville, wilsondob
\renewcommand*{\chaptitlefont}{\normalfont\Large\bfseries}
\renewcommand*{\chapnumfont}{\normalfont\Large\bfseries}



\begin{document}

\instructions{\centerline{\textbf{Proposal template}}}
\instructions{\centerline{\textbf{(Technical annex)}}}
\vskip0.5cm
\instructions{\centerline{\textit{\textbf{Research and Innovation actions}}}}
\vskip0.5cm
\instructions{Please follow the structure of this template when preparing your proposal. It has been designed to ensure that the important aspects of your planned work are presented in a way that will enable the experts to make an effective assessment against the evaluation criteria. Sections 1, 2 and 3 each correspond to an evaluation criterion for a full proposal.}
\vskip0.5cm
\instructions{Please be aware that proposals will be evaluated as they were submitted, rather than on their potential if certain changes were to be made. This means that only proposals that successfully address all the required aspects will have a chance of being funded. There will be no possibility for significant changes to content, budget and consortium composition during grant preparation.}
vskip0.5cm
\instructions{\textbf{First stage proposals:} In two-stage submission schemes, at the first stage you only need to complete the parts indicated by a bracket (i.e. \} ) . These are in the cover page, and sections 1 and 2.}
\vskip0.5cm
\instructions{\textbf{Page limit:} For full proposals, the cover page, and sections 1, 2 and 3, together should not be longer than 70 pages. All tables in these sections must be included within this limit. The minimum font size allowed is 11 points. The page size is A4, and all margins (top, bottom, left, right) should be at least 15 mm (not including any footers or headers).}
\vskip0.5cm
\instructions{The page limit for a first stage proposal is 15 pages.}
\vskip0.5cm
\instructions{If you attempt to upload a proposal longer than the specified limit, before the deadline you will receive an automatic warning, and will be advised to shorten and re-upload the proposal. After the deadline, any excess pages will be overprinted with a ‘watermark’, indicating to evaluators that these pages must be disregarded.}
\vskip0.5cm
\instructions{Please do not consider the page limit as a target! It is in your interest to keep your text as concise as possible, since experts rarely view unnecessarily long proposals in a positive light.}













%% TITLE
\maketitle
\instructions{
Maximum length for Sections 1,2,3: 70 pages including all tables.\\ 

Please use the same participant numbering as that used in the administrative proposal forms. (Not applicable in the case of stage-1 proposals in two stage schemes.)\\
For first stage proposals, please note that this table will be used to check whether or not you comply with any minimum requirements linked to participation as set out in the eligibility criteria of the relevant work programme.
}

\vspace{-1em}
\renewcommand\contentsname{Table of Contents}
\setlength{\cftbeforechapterskip}{1.0em plus 0.3em minus 0.1em}
\renewcommand{\cftchapterbreak}{\addpenalty{-4000}}
\makeparticipantstable
\tableofcontents*              
%%%% abstract
%%% ---------------------------------------------------------------------------
\begingroup

%%% english
%%% ...........................................................................
\addchap{Abstract}

\blindtext


%%% german
%%% ...........................................................................
\otherlanguage{ngerman}
\addchap{Kurzfassung}

\blindtext

\endgroup              
\pagebreak



%% Main proposal

%% Fixed proposal structure - Do not change
%%% Important. To have correct table numberings
\renewcommand{\thetable}{\thesection\alph{table}}

\chapter[Excellence]{S\&T Excellence}
\label{cha:excellence}
\instructions{
Your proposal must address a work programme topic for this call for proposals. \\
\textit{This section of your proposal will be assessed only to the extent that it is relevant to that topic.}\\
}

\section{Targeted breakthrough, Long term vision and Objectives}
\label{sec:objectives}
\instructions{
\begin{itemize}
\item Describe the targeted scientific breakthrough of the project.
\item Describe how the targeted breakthrough of the project contributes to a long-term vision for new technologies.
\item Describe the specific objectives for the project, which should be clear, measurable, realistic and achievable within the duration of the project.
\end{itemize}
}

\section{Relation to the work programme}
\label{sec:relation-to-work-programme}
\instructions{
\begin{itemize}
\item Indicate the work programme topic to which your proposal relates, and explain how your proposal addresses the specific challenge and scope of that topic, as set out in the work programme.
\end{itemize}
}


\section{Novelty, level of ambition and foundational character}
\label{sec:ambition}
\instructions{
\begin{itemize}
\item Describe the advance your proposal would provide beyond the state-of-the-art, and to what extent the proposed work is ambitious, novel and of a foundational nature. Your answer could refer to the ground-breaking nature of the objectives, concepts involved, issues and problems to be addressed, and approaches and methods to be used. 
\end{itemize}
}

\section{Research methods}
\label{sec:methods}
\instructions{
\begin{itemize}
\item Describe the overall research approach, the methodology and explain its relevance to the objectives.  
\item Where relevant, describe how sex and/or gender analysis is taken into account in the project’s content.
\end{itemize}
\textit{Sex and gender refer to biological characteristics and social/cultural factors respectively. For guidance on methods of sex/gender analysis and the issues to be taken into account, please refer to \url{http://ec.europa.eu/research/science-society/gendered-innovations/index_en.cfm}}
}

\section{Interdisciplinary nature}
\label{sec:interdisciplinary}
\instructions{
\begin{itemize}
\item Describe the research disciplines involved and the added value of the inter-disciplinarity.
\end{itemize}
}
 % Section I
\chapter{Impact}
\label{cha:impact}


\section{Expected impact} 
\label{sec:expected-impact}
\instructions{
\textit{Please be specific, and provide only information that applies to the proposal and its objectives. Wherever possible, use quantified indicators and targets.}\\
\begin{itemize}
\item Describe how your project will contribute to:
\begin{itemize}
\item the expected impacts set out in the work programme under the relevant topic. 
\item improving innovation capacity and the integration of new knowledge (strengthening the competitiveness and growth of companies by developing innovations meeting the needs of European and global markets; and, where relevant, by delivering such innovations to the markets;\\
\item any other environmental and socially important impacts (if not already covered above).
\end{itemize}
\item Describe any barriers/obstacles, and any framework conditions (such as regulation and standards), that may determine whether and to what extent the expected impacts will be achieved. (This should not include any risk factors concerning implementation, as covered in section 3.2.) 
\end{itemize}
}

\section{Measures to maximize impact} 
\label{sec:maximize-impact}

\subsection{Dissemination and exploitation of results}
\label{sec:dissemination-exploitation}
\instructions{
\begin{itemize}
\item Provide a draft ``plan for the dissemination and exploitation of the project's results'' (unless the work programme topic explicitly states that such a plan is not required). For innovation actions describe a credible path to deliver the innovations to the market. The plan, which should be proportionate to the scale of the project, should contain measures to be implemented both during and after the project.\\ 
\emph{Dissemination and exploitation measures should address the full range of potential users and uses including research, commercial, investment, social, environmental, policy making, setting standards, skills and educational training.\\
The approach to innovation should be as comprehensive as possible, and must be tailored to the specific technical, market and organisational issues to be addressed.}\\
\item Explain how the proposed measures will help to achieve the expected impact of the project. Include a business plan where relevant.\\
\item Where relevant, include information on how the participants will manage the research data generated and/or collected during the project, in particular addressing the following issues:\footnote{For further guidance on research data management, please refer to the H2020 Online Manual on the Participant Portal.}
\begin{itemize}
\item What types of data will the project generate/collect?
\item What standards will be used?
\item How will this data be exploited and/or shared/made accessible for verification and re-use? If data cannot be made available, explain why.
\item How will this data be curated and preserved?
\end{itemize}
\emph{You will need an appropriate consortium agreement to manage (amongst other things) the ownership and access to key knowledge (IPR, data etc.). Where relevant, these will allow you, collectively and individually, to pursue market opportunities arising from the project's results.} \\
\emph{The appropriate structure of the consortium to support exploitation is addressed in section 3.3.}\\
\item Outline the strategy for knowledge management and protection. Include measures to provide open access (free on-line access, such as the ``green'' or ``gold'' model) to peer-reviewed scientific publications which might result from the project\footnote{Open access must be granted to all scientific publications resulting from Horizon 2020 actions. Further guidance on open access is available in the H2020 Online Manual on the Participant Portal.}.\\
\emph{Open access publishing (also called 'gold' open access) means that an article is immediately provided in open access mode by the scientific publisher. The associated costs are usually shifted away from readers, and instead (for example) to the university or research institute to which the researcher is affiliated, or to the funding agency supporting the research.}\\
\emph{Self-archiving (also called 'green' open access) means that the published article or the final peer-reviewed manuscript is archived by the researcher - or a representative - in an online repository before, after or alongside its publication. Access to this article is often - but not necessarily - delayed (``embargo period''), as some scientific publishers may wish to recoup their investment by selling subscriptions and charging pay-per-download/view fees during an exclusivity period.}
\end{itemize}
}

\subsection{Communication activities}
\label{sec:communication}
\instructions{
\begin{itemize}
\item Describe the proposed communication measures for promoting the project and its findings during the period of the grant. Measures should be proportionate to the scale of the project, with clear objectives. They should be tailored to the needs of various audiences, including groups beyond the project's own community. Where relevant, include measures for public/societal engagement on issues related to the project. 
\end{itemize}
} % Section II
\chapter{Implementation}
\label{cha:implementation}

\section{Project work plan}
\label{sec:work-plan}
\instructions{
Please provide the following:
\begin{itemize}
\item brief presentation of the overall structure of the work plan;
\item timing of the different work packages and their components (Gantt chart or similar);
\item detailed work description, i.e.:
\begin{itemize}
\item a description of each work package (table 3.1a);
\item a list of work packages (table 3.1b);
\item a list of major deliverables (table 3.1c);
\end{itemize}
\item graphical presentation of the components showing how they inter-relate (Pert chart or similar).
\end{itemize}
\vskip0.2cm
\emph{\indent Give full details. Base your account on the logical structure of the project and the stages in which it is to be carried out. Include details of the resources to be allocated to each work package. The number of work packages should be proportionate to the scale and complexity of the project.}
\vskip0.2cm
\emph{You should give enough detail in each work package to justify the proposed resources to be allocated and also quantified information so that progress can be monitored, including by the Commission.}
\vskip0.2cm
\emph{You are advised to include a distinct work package on ``management'' (see section 3.2) and to give due visibility in the work plan to ``dissemination and exploitation'' and ``communication activities'', either with distinct tasks or distinct work packages.}
\vskip0.2cm
\emph{You will be required to include an updated (or confirmed) ``plan for the dissemination and exploitation of results'' in both the periodic and final reports. (This does not apply to topics where a draft plan was not required.) This should include a record of activities related to dissemination and exploitation that have been undertaken and those still planned. A report of completed and planned communication activities will also be required.}
\vskip0.2cm
\emph{If your project is taking part in the Pilot on Open Research Data\footnote{Certain actions under Horizon 2020 participate in the ‘Pilot on Open Research Data in Horizon 2020’. All other actions can participate on a voluntary basis to this pilot.  Further guidance is available in the H2020 Online Manual on the Participant Portal.}, you must include a 'data management plan' as a distinct deliverable within the first 6 months of the project. A template for such a plan is given in the guidelines on data management in the H2020 Online Manual. This deliverable will evolve during the lifetime of the project in order to present the status of the project's reflections on data management.}
\vskip0.2cm
\emph{\noindent \textbf{Definitions:}}
\vskip0.2cm
\emph{\ul{``Work package''} means a major sub-division of the proposed project.}
\vskip0.2cm
\emph{\ul{``Deliverable''} means a distinct output of the project, meaningful in terms of the project's overall objectives and constituted by a report, a document, a technical diagram, a software etc.}
\vskip0.2cm
\emph{\ul{``Milestones''} means control points in the project that help to chart progress. Milestones may correspond to the completion of a key deliverable, allowing the next phase of the work to begin. They may also be needed at intermediary points so that, if problems have arisen, corrective measures can be taken. A milestone may be a critical decision point in the project where, for example, the consortium must decide which of several technologies to adopt for further development.}
\vskip0.2cm
\emph{\noindent Report on work progress is done primarily through the periodic and final reports. Deliverables should complement these reports and should be kept to the minimum necessary.}
}

% Gantt chart in latex (requires pfggantt.sty file)
%%project Gantt chart


\begin{figure}
  \centering

  \begin{ganttchart}%
    [
    x unit = 0.25cm,
    y unit title=0.4cm,
    y unit chart=0.4cm,
    vgrid,
    title/.style={draw=black!50, fill=green!50!black},
    title label font=\sffamily\bfseries\color{white},
    title label anchor/.style={below=-1.6ex},
    title left shift=.05,
    title right shift=-.05,
    title height=1,
    bar/.style={draw=none, fill=blue!75},
    bar height=.6,
    bar label font=\small\color{black!50},
    milestone label font=\small\color{red!50},
    group right shift=0,
    group top shift=.6,
    group height=.3,
    group peaks={}{}{.2},
    incomplete/.style={fill=red}]{60}
    
    \gantttitle{ACRONYM}{60} \\
    \gantttitle{Year 1}{12} 
    \gantttitle{Year 2}{12} 
    \gantttitle{Year 3}{12}  
    \gantttitle{Year 4}{12}  
    \gantttitle{Year 5}{12} \\ 
    
    \ganttchartdata % data generated by the ICTProposal.cls
    
  \end{ganttchart}

  \caption[Gantt chart]{Project Gantt chart.}
  \label{fig:gantt}
\end{figure}


\subsection{Work package description}
\label{sec:wps}

%Include work-packages as separate files
%%%%%%%%%%%%%%%%%%%%%%%%%%%%%%
%  Work Package Description  %
%%%%%%%%%%%%%%%%%%%%%%%%%%%%%%

\begin{workpackage}{MANAGEMENT  WORK PACKAGE}
  \label{wp:management} %change and use appropriate description

  %%%%%%%%%%%%%%%%%% TOP TABLE %%%%%%%%%%%%%%%%%%%%%%%%%%%%%
  % Data for the top table
  \wpstart{1} %Starting Month
  \wpend{36} %End Month
  \wptype{Activity type} %RTD, DEM, MGT, or OTHER

  % Person Months per participant (required, max 7, * for leader)  
  % syntax: \personmonths{Participant number}{value}    (not wp leader)
  %     or  \personmonths{Participant short name}{value} (not wp leader)
  %         \personmonths*{Participant number}{value}    (wp leader)
  % for example:
  \personmonths*{UoC}{12}
  \personmonths{UoP2}{3}
  \personmonths{UoP3}{2}
  % etc.

  \makewptable % Work package summary table
    
  % Work Package Objectives
  \begin{wpobjectives}
    This work package has the following objectives:
    \begin{enumerate}
    \item To develop ....
    \item To apply this ....
    \item etc.
    \end{enumerate}
  \end{wpobjectives}
  
  % Work Package Description
  \begin{wpdescription}
    % Divide work package into multiple tasks.
    % Use \wptask command
    % syntax: \wptask{leader}{contributors}{start-m}{end-m}{title}{description}   
 
    Description of work carried out in WP, broken down into tasks, and
    with role of partners list. Use the \texttt{\textbackslash wptask} command.

    \wptask{UoC}{UoC}{1}{12}{Test}{
      \label{task:wp1test}
      Here we will test the WP Task code. 
    }
    \wptask{UoC}{UoC}{6}{9}{Integrate}{
      \label{task:wp1integrate}
      In this task UZH will integrate the work done in ~\ref{task:wp1test}.
    }    
    \wptask{UoP3}{All other}{9}{12}{Apply}{
      Here all the WP participants will apply the results to...
    }
    
    \paragraph{Role of partners}
    \begin{description}
    \item[Participant short name] will lead Task~\ref{task:wp1integrate}.
    \item[UoC] will..
    \end{description}
  \end{wpdescription}
  
  % Work Package Deliverable
  \begin{wpdeliverables}
    % Data for the deliverables and milestones  tables
    % syntax: \deliverable[delivery date]{nature}{dissemination
    % level}{description} 
    %
    % nature: R = Report, P = Prototype, D = Demonstrator, O = Other
    % dissemination level: PU = Public, PP = Restricted to other
    % programme participants (including the Commission Services), RE =
    % Restricted to a group specified by the consortium (including the
    % Commission Services), CO = Confidential, only for members of the
    % consortium (including the Commission Services).
    % 
    % \wpdeliverable[date]{R}{PU}{A report on \ldots}

    \wpdeliverable[36]{UoC}{R}{PU}{Report on the definition of the model
      specifications.}\label{dev:wp1specs}
    
    \wpdeliverable[12]{UoP3}{R}{PU}{Report on Feasibility study for the model
      implementation.}\label{dev:wp1implementation}

    \wpdeliverable[24]{UoP2}{R}{PU}{Prototype of model
      implementation.}\label{dev:wp1prototype}

  \end{wpdeliverables}

\end{workpackage}


%%% Local Variables:
%%% mode: latex
%%% TeX-master: "proposal-main"
%%% End:
             %Use \input for first WP
\include{wp-develop}
\include{wp-test}

\subsection{List of work packages}
\label{sec:wplist}
\makewplist

\subsection{List of deliverables}\footnote{If your action taking part in the Pilot on Open Research Data, you must include a data management plan as a distinct deliverable within the first 6 months of the project.  This deliverable will evolve during the lifetime of the project in order to present the status of the project's reflections on data management. A template for such a plan is available on the Participant Portal (Guide on Data Management).}
\label{sec:deliverables}
\instructions{
\textbf{KEY}\\
\emph{Deliverable numbers in order of delivery dates. Please use the numbering convention $<$WP number$>.<$number of deliverable within that WP$>$.}
\vskip0.2cm
\noindent\emph{For example, deliverable 4.2 would be the second deliverable from work package 4.}
\vskip0.2cm
\noindent\textbf{Type:}\\
\emph{Use one of the following codes:}\\
\indent R: Document, report (excluding the periodic and final reports)\\
\indent DEM: Demonstrator, pilot, prototype, plan designs\\
\indent DEC: Websites, patents filing, press \& media actions, videos, etc.\\
\indent OTHER: Software, technical diagram, etc.
\vskip0.2cm
\noindent\textbf{Dissemination level}:\\
\emph{Use one of the following codes:}\\
\indent PU = Public, fully open, e.g. web\\
\indent CO = Confidential, restricted under conditions set out in Model Grant Agreement\\
\indent CI = Classified, information as referred to in Commission Decision 2001/844/EC.\\
\vskip0.2cm
\noindent\textbf{Delivery date}:\\
Measured in months from the project start date (month 1).
}
\makedeliverablelist


\section{Management and risk assessment}
\label{sec:management}
\setcounter{table}{0}
\instructions{
\begin{itemize}
\item Describe the organisational structure and the decision-making ( including a list of
milestones (table 3.2a))
\item Describe any critical risks, relating to project implementation, that the stated project's objectives may not be achieved. Detail any risk mitigation measures. Please provide a table with critical risks identified and mitigating actions (table 3.2b)
\end{itemize}
}



\subsection{List of milestones}
\label{sec:milestones}
\instructions{
\vskip0.2cm
\noindent\textbf{KEY}\\
\textbf{Estimated date}\\
\emph{Measured in months from the project start date (month 1)}
\vskip0.2cm
\noindent\textbf{Means of verification}\\
\emph{Show how you will confirm that the milestone has been attained. Refer to indicators if appropriate. For example: a laboratory prototype that is ‘up and running’; software released and validated by a user group; field survey complete and data quality validated.}}

\milestone[24]{Completed simulator development}{Software released and
  validated}{1}
\milestone[36]{Final demonstration}{Application of results}{WP\,\ref{wp:test}}

\makemilestoneslist

\subsection{Critical risks for implementation}
\label{sec:risks}

\criticalrisk{The dedicated chip sent to fabrication is not functional.}{WP\,\ref{wp:test}}{Resort to Software simulations}

\makerisklist

\section{Consortium as a whole} 
\label{sec:consortium}
\instructions{
\emph{The individual members of the consortium are described in a separate section 4. There is no need to repeat that information here.}
\begin{itemize}
\item Describe the consortium. How will it match the project’s objectives? How do the members complement one another (and cover the value chain, where appropriate)? In what way does each of them contribute to the project? How will they be able to work effectively together? 
\item If applicable, describe how the project benefits from any industrial/SME involvement.
\item \textbf{Other countries:} If one or more of the participants requesting EU funding is based in a country that is not automatically eligible for such funding (entities from Member States of the EU, from Associated Countries and from one of the countries in the exhaustive list included in General Annex A of the work programme are automatically eligible for EU funding), explain why the participation of the entity in question is essential to carrying out the project. 
\end{itemize}
}

\section{Resources to be committed} 
\label{sec:resources}
\setcounter{table}{0}
\instructions{
\emph{Please make sure the information in this section matches the costs as stated in the budget table in section 3 of the administrative proposal forms, and the number of person/months, shown in the detailed work package descriptions.}
\vskip0.2cm
Please provide the following:\\
\begin{itemize}
\item a table showing number of person/months required (table 3.4a)
\item a table showing ``other direct costs'' (table 3.4b) for participants where those costs exceed 15\% of the personnel costs (according to the budget table in section 3 of the administrative proposal forms)  
\end{itemize}
}

\subsection{Summary of staff efforts}
\instructions{Table 3.4a: \emph{Please indicate the number of person/months over the whole duration of the planned work, for each work package, for each participant. Identify the work-package leader for each WP by showing the relevant person-month figure in bold.}}
\makesummaryofefforttable

\subsection{‘Other direct cost’ items (travel, equipment, other goods and services, large research infrastructure)}
\instructions{
Please provide a table of summary of costs for each participant, if the sum of the costs for ``travel'', ``equipment'', and ``goods and services'' exceeds 15\% of the personnel costs for that participant (according to the budget table in section 3 of the proposal administrative forms).
}

\costsTravel{UoC}{2500}{3 pairwise meetings for 2 people, 2 conferences for 3 people, 3 internal project meetings for 3 people}
\costsEquipment{UoC}{3000}{CAD workstation for chip design}
\costsOther{UoC}{60000}{Fabrication of 2 VLSI chips}
\costsOther{UoP2}{40000}{Fabrication of prototype PCBs}

\makecoststable


\instructions{
Please complete the table below for all participants that would like to declare costs of large research infrastructure under Article 6.2 of the General Model Agreement, irrespective of the percentage of personnel costs. Please indicate (in the justification) if the beneficiary's methodology for declaring the costs for large research infrastructure has already been positively assessed by the Commission.\\
Note: Large research infrastructure means research infrastructure of a total value of at least EUR 20 million, for a beneficiary. More information and further guidance on the direct costing for the large research infrastructure is available in the H2020 Online Manual on the Participant Portal.
}

\costslri{UoP3}{400000}{Synchrotron}
\costslri{UoC}{400000}{Synchrotron}

\makelritable
 % Section III


\clearpage

\chapter{Members of the consortium}
\label{cha:members}

\instructions{
\textit{This section is not covered by the page limit.}
\vskip0.2cm
\textit{The information provided here will be used to judge the operational capacity.}
}

\section{Participants (applicants)}
\label{sec:participants}

\instructions{
Please provide, for each participant, the following (if available):\\
\begin{itemize}
\item a description of the legal entity and its main tasks, with an explanation of how its profile matches the tasks in the proposal;
\item a curriculum vitae or description of the profile of the persons, including their gender, who will be primarily responsible for carrying out the proposed research and/or innovation activities;
\item a list of up to 5 relevant publications, and/or products, services (including widely-used datasets or software), or other achievements relevant to the call content;
\item a list of up to 5 relevant previous projects or activities, connected to the subject of this proposal;
\item a description of any significant infrastructure and/or any major items of technical equipment, relevant to the proposed work;
\item any other supporting documents specified in the work programme for this call.
\end{itemize}
}

\section{Third parties involved in the project (third party resources)}
\label{sec:third-parties}

\instructions{
\textit{Please complete, for each participant, the following table (or simply state "No third parties involved", if applicable).} \\
If yes in first row, please describe and justify the tasks to be subcontracted. If yes in second row, please describe the third party, the link of the participant to the third party, and describe and justify the foreseen tasks to be performed by the third party\footnote{A third party that is an affiliated entity or has a legal link to a participant implying a collaboration not limited to the action. (Article 14 of the Model Grant Agreement).}. If yes in third row, please describe the third party and their contributions.}

\begin{tabular}{|p{.85\textwidth}|p{.05\textwidth}|}
  \hline  
  \multicolumn{2}{|l|}{\cellcolor[gray]{0.8}\textbf{UoC}}\\
  \hline
  Does the participant plan to subcontract certain tasks (please note that core tasks of the project should not be sub-contracted) &
  \textbf{Y/N} \\
  \hline
Does the participant envisage that part of its work is performed by linked
third parties &
  \textbf{Y/N} \\
  \hline
  Does the participant envisage the use of contributions in kind provided by
third parties (Articles 11 and 12 of the General Model Grant Agreement) &
  \textbf{Y/N}\\
  \hline
\end{tabular}

\begin{tabular}{|p{.85\textwidth}|p{.05\textwidth}|}
  \hline  
  \multicolumn{2}{|l|}{\cellcolor[gray]{0.8}\textbf{UoP1}}\\
  \hline
  Does the participant plan to subcontract certain tasks (please note that core tasks of the project should not be sub-contracted) &
  \textbf{Y/N} \\
  \hline
Does the participant envisage that part of its work is performed by linked
third parties &
  \textbf{Y/N} \\
  \hline
  Does the participant envisage the use of contributions in kind provided by
third parties (Articles 11 and 12 of the General Model Grant Agreement) &
  \textbf{Y/N}\\
  \hline
\end{tabular}


\begin{tabular}{|p{.85\textwidth}|p{.05\textwidth}|}
  \hline  
  \multicolumn{2}{|l|}{\cellcolor[gray]{0.8}\textbf{UoP2}}\\
  \hline
  Does the participant plan to subcontract certain tasks (please note that core tasks of the project should not be sub-contracted) &
  \textbf{Y/N} \\
  \hline
Does the participant envisage that part of its work is performed by linked
third parties &
  \textbf{Y/N} \\
  \hline
  Does the participant envisage the use of contributions in kind provided by
third parties (Articles 11 and 12 of the General Model Grant Agreement) &
  \textbf{Y/N}\\
  \hline
\end{tabular}
 % Section IV
\chapter{Ethics and Security}
\label{cha:ethics}
\instructions{
\textit{This section is not covered by the page limit.}
}

\section{Ethics}
\label{sec:ethics}
\instructions{
If you have entered any ethics issues in the ethical issue table in the administrative proposal forms, you must:
\begin{itemize}
\item submit an ethics self-assessment, which:
\begin{itemize}
\item describes how the proposal meets the national legal and ethical requirements of the country or countries where the tasks raising ethical issues are to be carried out; 
\item explains in detail how you intend to address the issues in the ethical issues table, in particular as regard:
\begin{itemize}
\item research objectives (e.g. study of vulnerable populations, dual use, etc.)
\item research methodology (e.g. clinical trials, involvement of children and related consent procedures, protection of any data collected, etc.) 
\item the potential impact of the research (e.g. dual use issues, environmental damage, stigmatisation of particular social groups, political or financial retaliation, benefit-sharing,  malevolent use, etc.).
\end{itemize}
\end{itemize}
\item provide the documents that you need under national law(if you already have them), e.g.:
\begin{itemize}
\item an ethics committee opinion;
\item the document notifying activities raising ethical issues or authorising such activities;
\end{itemize}
\end{itemize}
\textit{\indent If these documents are not in English, you must also submit an English summary of them (containing, if available, the conclusions of the committee or authority concerned).}
\vskip0.2cm
\textit{If you plan to request these documents specifically for the project you are proposing, your request must contain an explicit reference to the project title.}
}

\section{Security}\footnote{Article 37.1 of the Model Grant Agreement: Before disclosing results of activities raising security issues to a third party (including affiliated entities), a beneficiary must inform the coordinator -- which must request written approval from the Commission/Agency. Article 37.2: Activities related to ``classified deliverables'' must comply with the ``security requirements'' until they are declassified. Action tasks related to classified deliverables may not be subcontracted without prior explicit written approval from the Commission/Agency. The beneficiaries must inform the coordinator -- which must immediately inform the Commission/Agency -- of
any changes in the security context and --if necessary -- request for Annex 1 to be amended (see Article 55).
}
\label{sec:security}
\instructions{
Please indicate if your project will involve:
\begin{itemize}
\item activities or results raising security issues: (YES/NO)
\item ``EU-classified information'' as background or results: (YES/NO)
\end{itemize}
}
 % Section V

\appendix

% master: main
% format: latex

%%--------------------------------------------------------------------------
\begin{appendices}
%%--------------------------------------------------------------------------

%%--------------------------------------------------------------------------
\chapter{FIRST APPENDIX}
%%--------------------------------------------------------------------------

This is the 1st appendix.  Now alphabetic numbering starts for Appedices.
\ifAMS
\begin{equation}
    A=
    \begin{pmatrix}
	a_{11}&a_{12}&\ldots&a_{1n}\\
	a_{21}&a_{22}&\ldots&a_{2n}\\
	\vdots&\vdots&\ddots&\vdots\\
	a_{m1}&a_{m2}&\ldots&a_{mn}
    \end{pmatrix}
\end{equation}
\else
\begin{equation}
    A=
    \left(
    \begin{array}{cccc}
	a_{11}&a_{12}&\ldots&a_{1n}\\
	a_{21}&a_{22}&\ldots&a_{2n}\\
	\vdots&\vdots&\ddots&\vdots\\
	a_{m1}&a_{m2}&\ldots&a_{mn}
    \end{array}
    \right)
\end{equation}
\fi

%%--------------------------------------------------------------------------
\chapter{MATHEMATICAL SYMBOLS}
%%--------------------------------------------------------------------------
\label{sec:mathsym}

Here is the second appendix.  See how equations, figures, and tables in
appendices are numbered.
\ifAMS
\begin{equation}
    A=
    \begin{pmatrix}
	a_{11}&a_{12}&\ldots&a_{1n}\\
	a_{21}&a_{22}&\ldots&a_{2n}\\
	\vdots&\vdots&\ddots&\vdots\\
	a_{m1}&a_{m2}&\ldots&a_{mn}
    \end{pmatrix}
\end{equation}
\else
\begin{equation}
    A=
    \left(
    \begin{array}{cccc}
	a_{11}&a_{12}&\ldots&a_{1n}\\
	a_{21}&a_{22}&\ldots&a_{2n}\\
	\vdots&\vdots&\ddots&\vdots\\
	a_{m1}&a_{m2}&\ldots&a_{mn}
    \end{array}
    \right)
\end{equation}
\fi
%
\begin{eqnarray}
 \left(\int_{-\infty}^\infty e^{-x^2}\,dx\right)^2
 & =& \int_{-\infty}^\infty\int_{-\infty}^\infty
   e^{-(x^2+y^2)}\,dx\,dy \nonumber \\
 & =& \int_0^{2\pi}\int_0^\infty e^{-r^2}r\,dr\,d\theta \nonumber \\
 & =& \int_0^{2\pi}\left(\left. -\frac{e^{-r^2}}{2}
   \right|_{r=0}^{\infty}\,\right)\,d\theta \nonumber \\
 & =& \pi
\end{eqnarray}

otherwise

\begin{eqnarray}
\textstyle\sin18^\circ={\frac{1}{4}}(\sqrt5-1)\\
\ifAMS
x \in \mathbb{R} \\
\fi
k=1.38\times10^{-23}\rm\,J/^\circ K.
\end{eqnarray}

% Math-mode symbol & verbatim
\def\W#1#2{$#1{#2}$ &\ttfamily\string#1\string{#2\string}}
\def\X#1{$#1$ &\ttfamily\string#1}
\def\Y#1{$\big#1$ &\ttfamily\string#1}
\def\Z#1{\ttfamily\string#1}

%
\begin{table}
\caption{Greek Letters}\label{tab:greek}
\vspace{1ex}
\begin{tabular}{*8l}
\X\alpha	&\X\theta	&\X o		&\X\tau 	\\
\X\beta 	&\X\vartheta	&\X\pi		&\X\upsilon	\\
\X\gamma	&\X\iota	&\X\varpi	&\X\phi 	\\
\X\delta	&\X\kappa	&\X\rho 	&\X\varphi	\\
\X\epsilon	&\X\lambda	&\X\varrho	&\X\chi 	\\
\X\varepsilon	&\X\mu		&\X\sigma	&\X\psi 	\\
\X\zeta 	&\X\nu		&\X\varsigma	&\X\omega	\\
\X\eta		&\X\xi						\\
								\\
\X\Gamma	&\X\Lambda	&\X\Sigma	&\X\Psi 	\\
\X\Delta	&\X\Xi		&\X\Upsilon	&\X\Omega	\\
\X\Theta	&\X\Pi		&\X\Phi
\end{tabular}
\end{table}

\begin{table}
\caption{Binary Operation Symbols}\label{tab:bin}
\vspace{1ex}
\begin{tabular}{*8l}
\X\pm		&\X\cap 	&\X\diamond		&\X\oplus     \\
\X\mp		&\X\cup 	&\X\bigtriangleup	&\X\ominus    \\
\X\times	&\X\uplus	&\X\bigtriangledown	&\X\otimes    \\
\X\div		&\X\sqcap	&\X\triangleleft	&\X\oslash    \\
\X\ast		&\X\sqcup	&\X\triangleright	&\X\odot      \\
\X\star 	&\X\vee 	&\X\lhd$^*$		&\X\bigcirc   \\
\X\circ 	&\X\wedge	&\X\rhd$^*$		&\X\dagger    \\
\X\bullet	&\X\setminus	&\X\unlhd$^*$		&\X\ddagger   \\
\X\cdot 	&\X\wr		&\X\unrhd$^*$		&\X\amalg     \\
\X+		&\X-
\end{tabular}

$^*$ Not predefined in \LaTeXe.
     Use one of the packages  \textsf{latexsym}, \textsf{amsfonts} or
     \textsf{amssymb}.

\end{table}


\begin{table}
\caption{Relation Symbols}\label{tab:rel}
\vspace{1ex}
\begin{tabular}{*8l}
\X\leq		&\X\geq 	&\X\equiv	&\X\models	\\
\X\prec 	&\X\succ	&\X\sim 	&\X\perp	\\
\X\preceq	&\X\succeq	&\X\simeq	&\X\mid 	\\
\X\ll		&\X\gg		&\X\asymp	&\X\parallel	\\
\X\subset	&\X\supset	&\X\approx	&\X\bowtie	\\
\X\subseteq	&\X\supseteq	&\X\cong	&\X\Join$^*$	\\
\X\sqsubset$^*$ &\X\sqsupset$^*$&\X\neq 	&\X\smile	\\
\X\sqsubseteq	&\X\sqsupseteq	&\X\doteq	&\X\frown	\\
\X\in		&\X\ni		&\X\propto	&\X=		\\
\X\vdash	&\X\dashv	&\X<		&\X>		\\
\X:
\end{tabular}

$^*$ Not predefined in \LaTeXe.
     Use one of the packages  \textsf{latexsym}, \textsf{amsfonts} or
     \textsf{amssymb}.

\end{table}


\begin{table}
\caption{Punctuation Symbols}\label{tab:punct}
\vspace{1ex}
\begin{tabular}{*{5}{lp{3.2em}}}
\X,	&\X;	&\X\colon	&\X\ldotp	&\X\cdotp
\end{tabular}
\end{table}

\begin{table}
\caption{Arrow Symbols}\label{tab:arrow}
\vspace{1ex}
\begin{tabular}{*6l}
\X\leftarrow		&\X\longleftarrow	&\X\uparrow	\\
\X\Leftarrow		&\X\Longleftarrow	&\X\Uparrow	\\
\X\rightarrow		&\X\longrightarrow	&\X\downarrow	\\
\X\Rightarrow		&\X\Longrightarrow	&\X\Downarrow	\\
\X\leftrightarrow	&\X\longleftrightarrow	&\X\updownarrow \\
\X\Leftrightarrow	&\X\Longleftrightarrow	&\X\Updownarrow \\
\X\mapsto		&\X\longmapsto		&\X\nearrow	\\
\X\hookleftarrow	&\X\hookrightarrow	&\X\searrow	\\
\X\leftharpoonup	&\X\rightharpoonup	&\X\swarrow	\\
\X\leftharpoondown	&\X\rightharpoondown	&\X\nwarrow	\\
\X\rightleftharpoons	&\X\leadsto$^*$
\end{tabular}

$^*$ Not predefined in \LaTeXe.
     Use one of the packages  \textsf{latexsym}, \textsf{amsfonts} or
     \textsf{amssymb}.

\end{table}

\begin{table}
\caption{Miscellaneous Symbols}\label{tab:ord}
\vspace{1ex}
\begin{tabular}{*8l}
\X\ldots	&\X\cdots	&\X\vdots	&\X\ddots	\\
\X\aleph	&\X\prime	&\X\forall	&\X\infty	\\
\X\hbar 	&\X\emptyset	&\X\exists	&\X\Box$^*$	\\
\X\imath	&\X\nabla	&\X\neg 	&\X\Diamond$^*$ \\
\X\jmath	&\X\surd	&\X\flat	&\X\triangle	\\
\X\ell		&\X\top 	&\X\natural	&\X\clubsuit	\\
\X\wp		&\X\bot 	&\X\sharp	&\X\diamondsuit \\
\X\Re		&\X\|		&\X\backslash	&\X\heartsuit	\\
\X\Im		&\X\angle	&\X\partial	&\X\spadesuit	\\
\X\mho$^*$	&\X.		&\X|
\end{tabular}

$^*$ Not predefined in \LaTeXe.
     Use one of the packages  \textsf{latexsym}, \textsf{amsfonts} or
     \textsf{amssymb}.

\end{table}

\begin{table}
\caption{Variable-sized  Symbols}\label{tab:op}
\vspace{1ex}
\begin{tabular}{*6l}
\X\sum		&\X\bigcap	&\X\bigodot	\\
\X\prod 	&\X\bigcup	&\X\bigotimes	\\
\X\coprod	&\X\bigsqcup	&\X\bigoplus	\\
\X\int		&\X\bigvee	&\X\biguplus	\\
\X\oint 	&\X\bigwedge
\end{tabular}
\end{table}


\begin{table}
\caption{Log-like Symbols}\label{tab:log}
\vspace{1ex}
\begin{tabular}{*8l}
\Z\arccos &\Z\cos  &\Z\csc &\Z\exp &
	   \Z\ker    &\Z\limsup &\Z\min &\Z\sinh \\
\Z\arcsin &\Z\cosh &\Z\deg &\Z\gcd &
	   \Z\lg     &\Z\ln	&\Z\Pr	&\Z\sup  \\
\Z\arctan &\Z\cot  &\Z\det &\Z\hom &
	   \Z\lim    &\Z\log	&\Z\sec &\Z\tan  \\
\Z\arg	  &\Z\coth &\Z\dim &\Z\inf &
	   \Z\liminf &\Z\max	&\Z\sin &\Z\tanh
\end{tabular}
\end{table}


\begin{table}
\caption{Delimiters\label{tab:dels}}
\vspace{1ex}
\begin{tabular}{*8l}
\X(		&\X)		&\X\uparrow	&\X\Uparrow	\\
\X[		&\X]		&\X\downarrow	&\X\Downarrow	\\
\X\{		&\X\}		&\X\updownarrow &\X\Updownarrow \\
\X\lfloor	&\X\rfloor	&\X\lceil	&\X\rceil	\\
\X\langle	&\X\rangle	&\X/		&\X\backslash	\\
\X|		&\X\|
\end{tabular}
\end{table}

\begin{table}
\caption{Large Delimiters\label{tab:ldels}}
\vspace{1ex}
\begin{tabular}{*8l}
\Y\rmoustache&	\Y\lmoustache&	\Y\rgroup&	\Y\lgroup\\[5pt]
\Y\arrowvert&	\Y\Arrowvert&	\Y\bracevert
\end{tabular}
\end{table}

\begin{table}
\caption{Math mode accents}\label{tab:accent}
\vspace{1ex}
\begin{tabular}{*{10}l}
\W\hat{a}     &\W\acute{a}  &\W\bar{a}	  &\W\dot{a}	&\W\breve{a}\\
\W\check{a}   &\W\grave{a}  &\W\vec{a}	  &\W\ddot{a}	&\W\tilde{a}\\
\end{tabular}
\end{table}

\begin{table}
\caption{Some other constructions}\label{tab:other}
\vspace{1ex}
\begin{tabular}{*4l}
\W\widetilde{abc}	&\W\widehat{abc}			\\
\W\overleftarrow{abc}	&\W\overrightarrow{abc} 		\\
\W\overline{abc}	&\W\underline{abc}			\\
\W\overbrace{abc}	&\W\underbrace{abc}			\\[5pt]
\W\sqrt{abc}		&$\sqrt[n]{abc}$&\verb|\sqrt[n]{abc}|	\\
$f'$&\verb|f'|          &$\frac{abc}{xyz}$&\verb|\frac{abc}{xyz}|
\end{tabular}
\end{table}


%%--------------------------------------------------------------------------
\chapter{THIRD APPENDIX}
%%--------------------------------------------------------------------------

Here is the third appendix.
Watch the number of Figure \ref{fig:3} in this appendix.

\begin{figure}[h]
  \centering
  \unitlength 1in	    % make unit length to be 1 inch
  \begin{picture}(6,4)(0,0) % picture coordinates 6 in width, 4 in height,
			    % origin 0,0
    \put(1.4,2.6){\line(3,-1){3.0}} % draw a straight line at slope -1/3
				% starting at (1.4,2.6) of length 3.0
    \put(0,0){\vector(1,0){5.5}}
    \put(0,0){\vector(0,1){3}}
  \end{picture}
  \caption{A Picture Drawn with \LaTeX\ Commands}\label{fig:3}
\end{figure}

%%--------------------------------------------------------------------------
\end{appendices}
%%--------------------------------------------------------------------------

 % Appendix

\backmatter

\bibliographystyle{plain}
\bibliography{refs}

\end{document}
