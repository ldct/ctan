\documentclass[11pt,dvips]{mtn}
\usepackage{palatino}
\begin{document}
%
\title{The D Operator and Algorithmic \\ Differentiation}
\author{Michael Monagan
\thanks{Informatik E.T.H., Z\"{u}rich, Switzerland. monagan@inf.ethz.ch}
\and J. S. Devitt\thanks{Faculty  of Mathematics, University of Waterloo, 
Canada, jsdevitt@daisy.uwaterloo.ca}
}
\shorttitle{D Operator}
\maketitle

\section{Introduction}
In this article we would like to inform our readers and users about
the development of the \mexpr{D} operator in Maple.
As with many major tasks in system development, getting something like
this nicely integrated into a system, and working correctly,
notationally correct, and making it easy to use, requires the design
of new facilities and changes to many parts of the system.

Although ``differentiation'' is often regarded as a
relatively simple task for a computer algebra system, it turns
out that this is not actually the case.
A paper by Stanly Steinberg and Michael Wester {\cite {MUC84}} presented
at the 1984 Macsyma Users Conference pointed out problems with
the differentiation facility in the computer algebra systems available
at that time.  In particular, Maple and other systems could not distinguish
correctly between total and partial derivatives.

Operators were first introduced into Maple in version 4.2
by Gaston Gonnet {\cite {OPERATORS}}.  The addition of the \mexpr{D} operator
addressed the distinction of total and partial derivatives, and also
a representation of the derivative of a function evaluated at a
point (for example, $y'(0)$ by \mexpr{D(y)(0)}).  
Partial derivatives and the ability to
apply the chain rule to an unknown function were added in Maple V.
Presently the \mexpr{D} operator is being extended to address the
problem of algorithmic differentiation, that is, to differentiate
Maple procedures.

In this article we follow the development of the \mexpr{D} operator
by way of examples discussing some of details and system design
issues as we go.

\section{Functions -- Expressions or Mappings?}

Users will find two facilities for differentiation in Maple,
the \mexpr{diff} procedure, and the \mexpr{D} procedure.

The \mexpr{diff} procedure takes as input what Maple calls
an {\em expression} or a {\em formula} which is a function of
zero of more variables ($x_1, x_2, \ldots x_n$) which appear explicitly
in the expression.
It computes the partial derivative of the formula with respect to
a given variable.

The \mexpr{D} procedure in Maple (often called the
\mexpr{D} operator) takes as input a function which is
a mapping from $R^N \rightarrow R$.  In Maple this is called an
{\em operator} or a {\em mapping}.
For example, $sin(x)$ is an expression in $x$ but $sin$ by itself is a
mapping from $R \rightarrow R$.
Another mapping in Maple is $sin+cos^2$.
A {\em mapping} can always be applied to an argument.
For example, given the mapping

\begin{mapleinput}
F := sin+cos^2;
\end{mapleinput}
\begin{maplelatex}
\[
{F} := {\rm sin} + {\rm cos}^{2}
\]
\end{maplelatex}
if we apply it to a number we get a number and if we apply
it to a formula we get a formula, e.g.

\begin{mapleinput}
F(1.0);
\end{mapleinput}
\begin{maplelatex}
\[
1.133397567
\]
\end{maplelatex}
\begin{mapleinput}
F(Pi/3);
\end{mapleinput}
\begin{maplelatex}
\[
{\displaystyle \frac {1}{2}}\,\sqrt {3} + {\displaystyle \frac {1
}{4}}
\]
\end{maplelatex}
\begin{mapleinput}
F(x);
\end{mapleinput}
\begin{maplelatex}
\[
{\rm sin}(\,{x}\,) + {\rm cos}(\,{x}\,)^{2}
\]
\end{maplelatex}

Strictly speaking,
Maple would call both the formula $sin(x) + cos(x)^2$ and
the mapping $sin + cos^2$ expressions.
Any distinction between the two comes from how we use them.
Mappings are mappings because our intention is
to {\em apply} them to arguments, while formulae are the
result of applying mappings to their arguments.
Throughout this article we will call the former
mappings and the latter formulae.
Thus the \mexpr{diff} procedure differentiates a formula and returns
a formula.  The \mexpr{D} operator differentiates a mapping
and returns a mapping.  Compare

\begin{mapleinput}
diff( sin(x)^2, x );
\end{mapleinput}
\begin{maplelatex}
\[
2\,{\rm sin}(\,{x}\,)\,{\rm cos}(\,{x}\,)
\]
\end{maplelatex}
\begin{mapleinput}
D(sin^2);
\end{mapleinput}
\begin{maplelatex}
\[
2\,{\rm cos}\,{\rm sin}
\]
\end{maplelatex}

Note, for mappings, functional composition is represented
explicitly by use of the \mexpr{@} operator, and repeated composition
is represented by the \mexpr{@@} operator.  Compare

\begin{mapleinput}
diff( sin(cos(x)), x );
\end{mapleinput}
\begin{maplelatex}
\[
 - {\rm cos}(\,{\rm cos}(\,{x}\,)\,)\,{\rm sin}(\,{x}\,)
\]
\end{maplelatex}
\begin{mapleinput}
D( sin@cos );
\end{mapleinput}
\begin{maplelatex}
\[
 - {\rm cos}^{(\,2\,)}\,{\rm sin}
\]
\end{maplelatex}

\section{Derivatives of Unknown Functions}

As well as being able to compute with known functions, like
$\sin, \cos, \exp, \ln$, etc., Maple has always supported the ability to
compute with unknown functions.  In the following examples
of partial and repeated partial differentiation of an unknown function $f$, the
notation \mexpr{D[i](f)} means the partial derivative of $f$ with
respect to the $i^{th}$ argument.

\begin{mapleinput}
diff(f(x,y),y);
\end{mapleinput}
\begin{maplelatex}
\[
{\frac {{ \partial}}{{ \partial}{y}}}\,{\rm f}(\,{x}, {y}\,)
\]
\end{maplelatex}
\begin{mapleinput}
D[2](f);
\end{mapleinput}
\begin{maplettyout}
\end{maplettyout}
\begin{maplelatex}
\[
{{D}_{2}}(\,{f}\,)
\]
\end{maplelatex}
\begin{mapleinput}
diff(f(x,y),x,y,x) = diff(diff(diff(f(x,y),x),y),x);
\end{mapleinput}
\begin{maplelatex}
\[
{\frac {{ \partial}^{3}}{{ \partial}{y}\,{ \partial}{x}^{2}}}\,
{\rm f}(\,{x}, {y}\,)={\frac {{ \partial}^{3}}{{ \partial}{y}\,{ 
\partial}{x}^{2}}}\,{\rm f}(\,{x}, {y}\,)
\]
\end{maplelatex}
\begin{mapleinput}
D[1,2,1](f) = D[1](D[2](D[1](f)));
\end{mapleinput}
\begin{maplettyout}
\end{maplettyout}
\begin{maplelatex}
\[
{{D}_{1, 1, 2}}(\,{f}\,)={{D}_{1, 1, 2}}(\,{f}\,)
\]
\end{maplelatex}

In Maple, the \mexpr{D} operator is an ordinary Maple procedure.
When we input \mexpr{D[1,2,1](f)}, what happens?
If \mexpr{D} was an array or table then the entry \mexpr{D[1,2,1]}
would be applied to $f$.  In the case of a procedure, what happens
is that it is called with the given arguments and inside
the procedure the $procname$ variable's value will be the subscript.
In our example, \mexpr{D} is called with $f$ as an argument
and the value of $procname$ will be \mexpr{D[1,2,1]}.
The \mexpr{D} code sorts the indices (assumes partial derivatives commute)
and returns \mexpr{D[1,1,2](f)} unevaluated.
This subscripted function calling facility is new in Maple V.
It is also used for the log function for different bases.
For example, \mexpr{log[b](x)} means $\log _b x$ i.e. logarithm base $b$ of $x$.

One of the main reasons why Maple has a \mexpr{D} operator as well
as a \mexpr{diff} procedure is because it is not possible to specify $y'(0)$ using
\mexpr{diff}.  Maple users reading this article might think of
using the Maple \mexpr{subs} procedure e.g. \mexpr{subs(x=0,diff(sin(x),x))}.
This works if Maple can actually differentiate the function
but it will not work for an unknown function $y$.
Being able to represent $y'(0)$ simply as $D(y)(0)$ motivated the
introduction of operators, in particular the \mexpr{D} operator in Maple.
This notation is used to specify the initial conditions for the
\mexpr{dsolve} procedure, which solves systems of ODE's,
replacing an earlier defunct notation $yp(0), ypp(0), \ldots$
It is also used in series; for example, here is the
Taylor series for an unknown function $f$ to order $O(x^6)$

\begin{mapleinput}
taylor(f(x),x);
\end{mapleinput}
\begin{maplettyout}
\end{maplettyout}
\begin{maplelatex}
\begin{eqnarray*}
\lefteqn{{\rm f}(\,0\,) + {\rm D}(\,{f}\,)(\,0\,)\,{x} + 
{\displaystyle \frac {1}{2}}\,{D}^{(\,2\,)}(\,{f}\,)(\,0\,)\,{x}
^{2} + {\displaystyle \frac {1}{6}}\,{D}^{(\,3\,)}(\,{f}\,)(\,0\,
)\,{x}^{3} + {\displaystyle \frac {1}{24}}\,{D}^{(\,4\,)}(\,{f}\,
)(\,0\,)\,{x}^{4} + } \\
 & & {\displaystyle \frac {1}{120}}\,{D}^{(\,5\,)}(\,{f}\,)(\,0\,
)\,{x}^{5} + {\rm O}(\,{x}^{6}\,)\mbox{\hspace{200pt}}
\end{eqnarray*}
\end{maplelatex}
And here is a multivariate Taylor series to third order.

\begin{mapleinput}
readlib(mtaylor); # load the multivariate series package
\end{mapleinput}
\begin{maplelatex}
\end{maplelatex}
\begin{maplettyout}
proc() ... end

\end{maplettyout}
\begin{mapleinput}
mtaylor(f(x,y),[x,y],3);
\end{mapleinput}
\begin{maplettyout}
\end{maplettyout}
\begin{maplelatex}
\begin{eqnarray*}
\lefteqn{{\rm f}(\,0, 0\,) + {{D}_{1}}(\,{f}\,)(\,0, 0\,)\,{x} + 
{{D}_{2}}(\,{f}\,)(\,0, 0\,)\,{y} + {\displaystyle \frac {1}{2}}
\,{{D}_{1, 1}}(\,{f}\,)(\,0, 0\,)\,{x}^{2} + {x}\,{{D}_{1, 2}}(\,
{f}\,)(\,0, 0\,)\,{y}} \\
 & & \mbox{} + {\displaystyle \frac {1}{2}}\,{{D}_{2, 2}}(\,{f}\,
)(\,0, 0\,)\,{y}^{2}\mbox{\hspace{250pt}}
\end{eqnarray*}
\end{maplelatex}

\newpage
Without use of the \mexpr{D} operator it would be difficult to apply
the chain rule to unknown functions.  For example, we have

\begin{mapleinput}
diff(f(x^2),x);
\end{mapleinput}
\begin{maplettyout}
\end{maplettyout}
\begin{maplelatex}
\[
2\,{\rm D}(\,{f}\,)(\,{x}^{2}\,)\,{x}
\]
\end{maplelatex}
\begin{mapleinput}
diff(f(x^2,x*y),x);
\end{mapleinput}
\begin{maplettyout}
\end{maplettyout}
\begin{maplelatex}
\[
2\,{{D}_{1}}(\,{f}\,)(\,{x}^{2}, {y}\,{x}\,)\,{x} + {{D}_{2}}(\,{
f}\,)(\,{x}^{2}, {y}\,{x}\,)\,{y}
\]
\end{maplelatex}
\begin{mapleinput}
D(f@g);
\end{mapleinput}
\begin{maplettyout}
\end{maplettyout}
\begin{maplelatex}
\[
{\rm D}(\,{f}\,){\rm @}{g}\,{\rm D}(\,{g}\,)
\]
\end{maplelatex}
There are distinct advantages to manipulating mappings as if
they were expressions.  The following example illustrates implicit
differentiation of $y$ as a function of $x$.
Given the equation

\begin{mapleinput}
eq := y^2*x + y^3*x^2 + y + 3*x = 0;
\end{mapleinput}
\begin{maplelatex}
\[
{\it eq} := {y}^{2}\,{x} + {y}^{3}\,{x}^{2} + {y} + 3\,{x}=0
\]
\end{maplelatex}
we can regard $x$ and $y$ as arbitrary mappings.  Then applying
\mexpr{D} to both sides of the equation we have

\begin{mapleinput}
map(D,eq);
\end{mapleinput}
\begin{maplettyout}
\end{maplettyout}
\begin{maplelatex}
\[
2\,{\rm D}(\,{y}\,)\,{y}\,{x} + {y}^{2}\,{\rm D}(\,{x}\,) + 3\,
{\rm D}(\,{y}\,)\,{y}^{2}\,{x}^{2} + 2\,{y}^{3}\,{\rm D}(\,{x}\,)
\,{x} + {\rm D}(\,{y}\,) + 3\,{\rm D}(\,{x}\,)=0
\]
\end{maplelatex}
This equation can be interpreted in many different ways.  For example,
it can be solved to obtain a formula for $D(y)$
\begin{mapleinput}
D(y) = solve(",D(y));
\end{mapleinput}
\begin{maplettyout}
\end{maplettyout}
\begin{maplelatex}
\[
{\rm D}(\,{y}\,)= - \,{\displaystyle \frac {{y}^{2}\,{\rm D}(\,{x
}\,) + 2\,{y}^{3}\,{\rm D}(\,{x}\,)\,{x} + 3\,{\rm D}(\,{x}\,)}{2
\,{y}\,{x} + 3\,{y}^{2}\,{x}^{2} + 1}}
\]
\end{maplelatex}
while the interpretation that $x$ is an independent variable can
be indicated by the substitution
\begin{mapleinput}
subs(D(x)=1,");
\end{mapleinput}
\begin{maplettyout}
\end{maplettyout}
\begin{maplelatex}
\[
{\rm D}(\,{y}\,)= - \,{\displaystyle \frac {{y}^{2} + 2\,{y}^{3}
\,{x} + 3}{2\,{y}\,{x} + 3\,{y}^{2}\,{x}^{2} + 1}}
\]
\end{maplelatex}

Earlier we mentioned that application of a mapping to a symbolic
variable yields a formula.  And hence the identity
\mexpr{D(f)(x) = diff(f(x),x)}.  For example

\begin{mapleinput}
F := sin+cos^2;
\end{mapleinput}
\begin{maplelatex}
\[
{F} := {\rm sin} + {\rm cos}^{2}
\]
\end{maplelatex}
\begin{mapleinput}
F(x);
\end{mapleinput}
\begin{maplelatex}
\[
{\rm sin}(\,{x}\,) + {\rm cos}(\,{x}\,)^{2}
\]
\end{maplelatex}
\begin{mapleinput}
diff(F(x),x) - D(F)(x);
\end{mapleinput}
\begin{maplelatex}
\[
0
\]
\end{maplelatex}

Given the formula, how can we get back the mapping $F$?
This is called {\em lambda abstraction} in the language of lambda calculus.
In Maple it is called \mexpr{unapply} because it
is the inverse of application, that is, it takes a formula and returns
a mapping. For example

\begin{mapleinput}
G := unapply(F(x),x);
\end{mapleinput}
\begin{maplelatex}
\[
{G} := {x} \rightarrow {\rm sin}(\,{x}\,) + {\rm cos}(\,{x}\,)^{2
}
\]
\end{maplelatex}
is a mapping in the form of a Maple procedure equivalent to 
\mexpr{proc(x) sin(x)+cos(x)^2 end:} except it has been
displayed using a more succinct format, 
known as arrow operators.  The arrow notation above is used often in algebra.
This new notation is essentially equivalent to Maple's
older angle bracket notation

\begin{mapleinput}
<sin(x)+cos(x)^2|x>;
\end{mapleinput}
\begin{maplelatex}
\[
 \langle {\rm sin}(\,{x}\,) + {\rm cos}(\,{x}\,)^{2}\,{ \mid}\,{x
} \rangle 
\]
\end{maplelatex}
differing only in how the procedure is entered and displayed.

Another notation for functions that is used often in applied mathematics
is the $F(x)=sin(x)+cos(x)^2$ notation.  In Maple one might use
\mexpr{F(x):=sin(x)+cos(x)^2}, but this already has a meaning in Maple (which
unfortunately is different) and this does lead to confusion.
The meaning of \mexpr{F(x):=y} is to enter the entry $(x,y)$ in $F$'s remember
table so that when $F$ is called with the literal symbol
$x$, $y$ is returned.  It is rather like making $F$ work like a table
of values.

\section{Equivalence of Mappings}

Notice though, that \mexpr{unapply} did not return the mapping
in the same {\em form} $F$ that we started with.
This raises another question, namely, given two mappings, how could one
test if they are the same?  Users are familiar with the problem of testing
whether two formulae are the same.  For example, suppose we are
given the two formulae

\begin{mapleinput}
f1 := sin(x)+cos(x)^2:
f2 := sin(x)+cos(2*x)/2+1/2:
\end{mapleinput}

How would we test whether $f_1 = f_2$?  This is the problem of
simplification, or zero recognition.
In Maple, one would use the expand (or simplify) function as follows

\begin{mapleinput}
expand(f1-f2);
\end{mapleinput}
\begin{maplelatex}
\[
0
\]
\end{maplelatex}

In this case, expand applies the transformation $cos(2 x) = 2 cos(x)^2 - 1$
hence recognizing that $f_1 = f_2$.
But what about mappings?

\begin{mapleinput}
expand(eval(G));
\end{mapleinput}
\begin{maplelatex}
\[
{\rm sin} + {\rm cos}^{2}
\]
\end{maplelatex}

Expand tries to write an arrow (or angle bracket)
operator as an algebraic combination of other mappings, in this case
allowing us to recognize that $F$ and $G$ are effectively
the same mapping.

\begin{mapleinput}
expand(F-eval(G));
\end{mapleinput}
\begin{maplelatex}
\[
0
\]
\end{maplelatex}

Maple can differentiate those arrow operators too, even
in their unexpanded form.

\begin{mapleinput}
D(G);
\end{mapleinput}
\begin{maplettyout}
\end{maplettyout}
\begin{maplelatex}
\[
{x} \rightarrow {\rm cos}(\,{x}\,) - 2\,{\rm cos}(\,{x}\,)\,{\rm 
sin}(\,{x}\,)
\]
\end{maplelatex}

\section{Notational Equivalence and Conversions}

Two important identities relating \mexpr{D} and
\mexpr{diff} are \mexpr{D(f)(x)} = \mexpr{diff(f(x),x)} (and its
multiviate counterpart), and \mexpr{D(f)} = \mexpr{unapply(diff(f(x),x))}.
When two notations are involved for essentially the same
expression, it is essential to be able to convert from
one notation to the other.  For example

\begin{mapleinput}
diff(f(x),x);
\end{mapleinput}
\begin{maplelatex}
\[
{\frac {{ \partial}}{{ \partial}{x}}}\,{\rm f}(\,{x}\,)
\]
\end{maplelatex}
\begin{mapleinput}
convert(",D);
\end{mapleinput}
\begin{maplettyout}
\end{maplettyout}
\begin{maplelatex}
\[
{\rm D}(\,{f}\,)(\,{x}\,)
\]
\end{maplelatex}
\begin{mapleinput}
unapply(",x);
\end{mapleinput}
\begin{maplettyout}
\end{maplettyout}
\begin{maplelatex}
\[
{\rm D}(\,{f}\,)
\]
\end{maplelatex}

\section{Algorithmic Differentiation}

The use of arrow operators, or more generally, arbitrary procedures
leads us to the interesting problem of program differentiation.
Consider the function $f$ defined by the following Maple procedure

\begin{mapleinput}
f := proc(x) local s,t; s := sin(x); t := x^2; s*t+2*t end:
\end{mapleinput}

\vspace{1mm}
What is its derivative?
We could compute its value as a formula and differentiate
the formula.

\begin{mapleinput}
f(x);
\end{mapleinput}
\begin{maplelatex}
\[
{\rm sin}(\,{x}\,)\,{x}^{2} + 2\,{x}^{2}
\]
\end{maplelatex}
\begin{mapleinput}
diff(",x);
\end{mapleinput}
\begin{maplelatex}
\[
{\rm cos}(\,{x}\,)\,{x}^{2} + 2\,{\rm sin}(\,{x}\,)\,{x} + 4\,{x}
\]
\end{maplelatex}

The Maple V Release 2 Share Library%
\footnote{Information about the Share Library
is included in the {\em News and Announcements} section.}
introduced a facility for differentiating programs.  This has
since been incoporated into Release 3 and is accessible directly
as
\begin{mapleinput}
PD := readlib('`PD/PD`'):
\end{mapleinput}
The \mexpr{PD} procedure takes as input a Maple procedure $f$ which
is a function of $n$ parameters, and a positive integer $i$, and it returns
a Maple procedure which computes the partial derivative of $f$ with
respect to the $i^{th}$ parameter.
For example

\begin{mapleinput}
g := PD(f,1);
\end{mapleinput}
\begin{maplelatex}
\end{maplelatex}
\begin{maplettyout}
g := proc(x)
     local s,t,sx,tx;
         sx := cos(x); s := sin(x); tx := 2*x; t := x^2; sx*t+s*tx+2*tx
     end

\end{maplettyout}
Does the procedure $g$ really compute $f'$?
In this case we can \emph{prove} that it does by executing the procedure
on symbolic parameters, in effect converting the function represented
by the procedure into a formula.

\begin{mapleinput}
diff(f(x),x) - g(x);
\end{mapleinput}
\begin{maplelatex}
\[
0
\]
\end{maplelatex}

Clearly one couldn't do this if a procedure had a conditional
statement involving the formal parameter $x$ and one called the procedure
with a symbolic parameter,  but
under what conditions could one differentiate a 
procedure involving more than just an expression?
For example, can this be done
if the procedure had loops or subroutine calls?
It turns out that the answer to this question is, surprisingly, yes.
And moreover, there exists a very simple algorithm for computing the
derivative of a procedure or a program.

To construct the derivative procedure,
for each assignment statement $v := f(v_1, \ldots , v_n)$ that appears
in the procedure, where the $v_i$ are local variables or formal parameters,
precede it by $v_x := g(v_1, \ldots , v_n)$ where
$g(v_1, \ldots , v_n)$ is obtained by differentiating $f(v_1, \ldots , v_n)$
formally.  That is, $v_i$ may depend on $x$, hence its derivative
will be $v_{i_x}$.
Replace the last statement (or any RETURN value) by its derivative.
This very simple algorithm is called ``forward differentiation''.
There is actually quite a lot of literature on this subject. Many different
algorithms, and quite a number of implementations have been
written to differentiate Fortran code.
For a good reference see {\cite {PROCEEDINGS}}, which also contains a
fairly complete bibliography on algorithmic differentiation.

Here is an example which illustrates the power of algorithmic
differentiation.
In this example, the use of a loop allows us to represent
a very large formula in a very compact way.
There is a theoretical gain here.  In general, a function that can
be represented by a formula can be represented by a program
in an exponentially more compact way by using local variables and loops.

\begin{mapleinput}
f := proc(x,n) local i,t;
     t := x;
     for i to n do t := ln(t) od;
     t
end:
g := PD(f,1,2); # compute D(D(f))
\end{mapleinput}
\begin{maplelatex}
\end{maplelatex}
\begin{maplettyout}
g := proc(x,n)
     local tx,i,t,txx;
         txx := 0;
         tx := 1;
         t := x;
         for i to n do  txx := txx/t-tx^2/t^2; tx := tx/t; t := ln(t) od;
         txx
     end

\end{maplettyout}

Another nice theoretical result is that the size of the resulting
program which computes the derivative is linear in the size of the
original program.  In fact, in most cases, it is not much bigger.
What can algorithmic differentiation be used for?
In numerical computation, one often is given a function $f:R^N \rightarrow R$,
where $f$ is given by a program, rather than an analytic formula.
The standard fast methods for computing the zeros
of $f$ or the extrema require the derivatives of $f$.
An example is given in the article \emph{The Billiard Problem}
in this issue, by Walter Gander and Dominik Gruntz, where a Newton
iteration is used to find the zeroes of a function.

Here is another example where we are given a function which
evaluates a polynomial input as an array of coefficients using Horner's
rule and we compute its derivative.

\begin{mapleinput}
f := proc(x,b,n) local i,s;
    # the array b represents the polynomial b = sum( b[i]*x^i, i=0..n )
    s := 0;
    for i from n by -1 to 0 do s := s*x+b[i]; od;
    s
end:
g := PD(f,1);
\end{mapleinput}
\begin{maplelatex}
\end{maplelatex}
\begin{maplettyout}
g := proc(x,b,n)
     local i,s,sx;
         sx := 0;
         s := 0;
         for i from n by -1 to 0 do  sx := sx*x+s; s := s*x+b[i] od;
         sx
     end

\end{maplettyout}

It may not be apparent from these examples, but the difficult
part of algorithmic differentiation is program optimization.
This is because differentiation produces redundant computation.
For example, repeatedly differentiating formulae results in
lots of repeated common subexpressions.  And, when computing partial
derivatives, a lot of zeroes may result.
Thus two of the main focuses of algorithmic differentiation is
to avoid as much of this redundancy as possible and 
do program optimization.
We are presently working on extending Maple's capabilities for
differentiating procedures to compute gradients and jacobians,
and improving the optimization of the resulting procedures.

\begin{thebibliography}{10}

\bibitem{MUC84}
Michael Wester and Stanly Steinberg,
A Survey of Symbolic Differentiation Implementations,
\emph{Proceedings of the 1984 MACSYMA Users' Conference},
Schenectady NY, (1984).

\bibitem{OPERATORS}
Gaston Gonnet,
An Implementation of Operators for Symbolic Algebra Systems,
\emph{Proceedings of the 1986 Symposium on Symbolic and Algebraic Computations},
Symsac `86, ACM, (1986).

\bibitem{PROCEEDINGS}
Automatic Differentiation of Algorithms: Theory, Implementation
and Application, SIAM, Philadelphia 1991.
\textit{Proceedings of the SIAM Workshop on Automatic Differentiation},
Breckenridge, CO, (1991).

\end{thebibliography}

\end{document}
