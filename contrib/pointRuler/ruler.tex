\documentclass[12pt,legalpaper]{article}

%##############
%# Use PiCTeX #
%##############
\font\fivrm=cmr5 \relax
\input{prepictex}
\input{pictex}
\input{postpictex}

%#############################################
%# Use lscape to make the document landscape #
%#############################################
\usepackage{lscape}

%#################
%# Setup margins #
%#################
\setlength{\topmargin}{0pt}
\setlength{\headheight}{0pt}
\setlength{\headsep}{0pt}
\setlength{\textheight}{867.24pt}
\setlength{\footskip}{0pt}
\setlength{\oddsidemargin}{0pt}
\setlength{\evensidemargin}{0pt}
\setlength{\textwidth}{469.755pt}
\setlength{\marginparsep}{0pt}
\setlength{\marginparwidth}{0pt}

%###########################
%# Force no page numbering #
%###########################
\pagestyle{empty}

\begin{document}
\begin{landscape}

%############################
%# Start the PiCTeX picture #
%############################
\beginpicture
\setcoordinatesystem units <1pt,1pt>
\setplotarea x from 0 to 867.24, y from -36.135 to 36.135
\plot 0 -36.135 867.24 -36.135 867.24 36.135 0 36.135 0 -36.135 /
\plot 0 0 867.24 0 /
\input points.tex		% Input the points.tex for the points scale
\plot 0 28 10 28 10 36.135 /
\put {\tiny pt} at 5 32.0675
\input inches.tex		% Input the inches.tex for the inches scale
\plot 0 -28 10 -28 10 -36.135 /
\put {\scriptsize 12} [tr] at 865.24 -29
\put {\tiny in} at 5 -32.0675
\endpicture

\end{landscape}
\end{document}
