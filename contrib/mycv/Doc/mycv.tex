% ----------------------------------------------------------
% ** This file was automatically generated by <DoPackage.sh>
% ** with option(s) 'ctan'.
% ** Date: Sun May 20 21:31:48 CEST 2012::1337542308
% ----------------------------------------------------------
% -------------------------------------------------------
% Copyright 2012 Ghersi Andrea (ghanhawk@gmail.com).
%
% This work may be distributed and/or modified under the
% conditions of the LaTeX Project Public License version
% 1.3c, available at 'http://www.latex-project.org/lppl'.
% -------------------------------------------------------
\def\version{1.5.6}
\newif\ifdraft

\newif\ifparchment
\parchmenttrue

\documentclass[
   10pt,
   a4paper,oneside,openany,
   titlepage,
   fleqn,
   headinclude,footinclude,
   BCOR5mm,
   numbers=noenddot,
   cleardoublepage=empty,
   captions=tableheading
   \ifdraft
   ,draft
   \else
   \fi
]{scrreprt}

\usepackage[backref]{classicthesis-ldpkg}
\usepackage[
   eulerchapternumbers,
   subfig,
   beramono,
   eulermath,
   pdfspacing,
   dottedtoc
]{classicthesis}

\usepackage[english]{arsclassica}

\usepackage[utf8x]{inputenc}

\usepackage[english]{babel}

\usepackage{textcomp}

\usepackage{empheq}

\usepackage{calligra}

\usepackage{todonotes}

\setlength{\parindent}{0pt}

\usepackage{comment}

\usepackage{listings}


\ifdraft
   \overfullrule=5pt
\fi

\hypersetup{citecolor=webgreen}
\hypersetup{pdfstartpage=1}

\hypersetup{%
   pdfstartpage = 1,
   pdfauthor    = Andrea Ghersi,
   pdftitle     = MyCV,
   pdfsubject   = MyCV class documentation,
   pdfproducer  = \LaTeX{},
   pdfkeywords  = {},
   pdfcreator   = \LaTeX{} with 'arsclassica'
}

\newcommand{\myTitle}{MyCV\xspace}
\newcommand{\myName}{Andrea Ghersi\xspace}
\newcommand{\myMail}{ghanhawk@gmail.com}

\areaset{412pt}{749pt}

\titleformat{\section}%
   {\color{red}\normalfont\Large\bfseries\bf}%
   {\color{red}\MakeTextLowercase{\thesection}}{1em}{\spacedlowsmallcaps}

\titleformat{\subsection}%
   {\color{orange}\normalfont\bfseries\bf}
   {\textsc{\MakeTextLowercase{\thesubsection}}}{1em}{\normalsize}

\titleformat{\subsubsection}%
   {\color{cyan}\normalfont\bfseries\bf}
   {\textsc{\MakeTextLowercase{\thesubsection}}}{1em}{\small}

\renewcommand*\cftchapfont{\color{red}\bfseries}
\renewcommand*\cftsecfont{\color{gray}\bfseries}

\def\chapterautorefname{chapter}
\def\sectionautorefname{section}
\def\subsectionautorefname{subsection}

\def\myitemsep{5pt}

\newcommand{\squishlist}[1][\myitemsep]{%
   \begin{list}{$\triangleright$}%
   {
      \setlength{\parskip}{5pt}%
      \setlength{\itemsep}{#1}%
      \setlength{\parsep}{5pt}%
      \setlength{\topsep}{5pt}%
      \setlength{\partopsep}{0pt}%
      \setlength{\leftmargin}{2.5em}%
      \setlength{\labelwidth}{1em}%
      \setlength{\labelsep}{0.5em}%
   }}

\newcommand{\squishend}{%
   \end{list}}

\newcommand{\quotes}[3][0]{%
   \definecolor{authorcolor}{RGB}{51, 153, 51}
   \definecolor{datecolor}  {RGB}{51, 153, 51}
   %
   \noindent\makebox[0.90\textwidth]{%
      \begin{minipage}[t]{0.50\textwidth}%
         \hfill%
      \end{minipage}%
      \begin{minipage}[t]{0.50\textwidth}%
         {\small\ignorespaces\slshape #2}\\[0.5em]%
         \medskip%
         \hfill\texttt{---} {{\color{authorcolor}{#3}}}%
         \ifnum#1>0%
            \xspace[{\color{datecolor}{#1}}]%
         \fi%
      \end{minipage}%
   }}

\newcommand{\myref}[1]{{\color{blue}(\autoref{#1})}\xspace}
\newcommand{\myreftwo}[1]{{\color{blue}(see~\autoref{#1})}\xspace}

\newcommand{\keyword}[1]{{\footnotesize\textbf{#1}}\xspace}

\newcommand{\argvsep}{\\[5pt]}
\newcommand{\arghsep}{\;\;}
\newcommand{\cmdvsep}{\\[5pt]}
\newcommand{\cmdhsep}{\;\;}
\newcommand{\cmddsep}{\par\medskip}
\newcommand{\optvsep}{\\[5pt]}
\newcommand{\opthsep}{\;\;}
\newcommand{\optdsep}{\par}
\newcommand{\sep}{\medskip}
\newcommand{\argname}[2][black]{{\color{#1}{$\langle$\textit{#2}$\rangle$}}}
\newcommand{\cmdname}[2][brown]{{\color{#1}{\textbf{\textbackslash#2}}}}
\newcommand{\optname}[2][brown]{{\color{#1}{\textbf{#2}}}}
\newcommand{\classname}{{\color{blue}{\textit{MyCV}}}\xspace}
\newcommand{\codeskip}{\medskip}
\newcommand{\DPL}{\keyword{DPL}}
\newcommand{\SPL}{\keyword{SPL}}
\newcommand{\CV}{\keyword{CV}}

\newcommand{\marg}[1]{%
   \fbox{%
      {\color{red}\textbf{\{}}$\langle$%
      \textit{#1}%
      $\rangle${\color{red}\textbf{\}}}}%
   }

\newcommand{\dargone}[2]{%
   \fbox{%
      {\color{cyan}\textbf{[}}$\langle$%
      \textit{#1}%
      $\rangle${\color{cyan}\textbf{]}}%
   }\;$\longrightarrow$\;{\color{blue}\textbf{[}\textit{#2}\textbf{]}}}

\newcommand{\dargtwo}[2]{%
   \fbox{%
      {\color{cyan}\textbf{<}}$\langle$%
      \textit{#1}%
      $\rangle${\color{cyan}\textbf{>}}%
   }\;$\longrightarrow$\;{\color{blue}\textbf{<}\textit{#2}\textbf{>}}}

\newenvironment{myindent}[1][0.2in]%
 { \begin{list}{}{%
   \setlength{\topsep}{0pt}%
    \setlength{\leftmargin}{#1}%
    \setlength{\partopsep}{0pt}%
    \setlength{\parsep}{\parskip}}\item[]%
 }
 {\end{list}}

\newcommand{\ctext}[2][red]{{\color{#1}{#2}}}

\newcommand{\omissis}{[\dots\negthinspace]}

\definecolor{lightergray}{gray}{0.99}

\lstnewenvironment{latexcode}[1][]{%
   \small
   \lstset{language=[LaTeX]Tex,
      keywordstyle=\color{RoyalBlue},
      basicstyle=\normalfont\ttfamily,
      commentstyle=\color{Emerald}\ttfamily,
      stringstyle=\rmfamily,
      numbers=none,
      numberstyle=\scriptsize\color{gray},
      stepnumber=1,
      numbersep=8pt,
      showstringspaces=false,
      breaklines=true,
      frameround=ftff,
      frame=lines,
      moredelim=[is][\color{orange}]{|}{|},
      moredelim=[is][\color{brown}]{$}{$},
      moredelim=[is][\color{red}]{&}{&},
      moredelim=[is][\textbf]{||}{||},
      backgroundcolor=\color{lightergray},
      #1
   }
   \lstset{
      morekeywords=%
      {%
         RequirePackage,newboolean,DeclareOption,setboolean,%
         ProcessOptions,PackageError,ifthenelse,boolean,%
         MakeTextLowercase,@ifpackageloaded,undefined,%
         DeclareRobustCommand,MakeTextUppercase,conditionalblock,%
         color,titlerule,titlespacing,ifmodel,ifoption,mypdftitle,%
         labelitemi,hypersetup,setlength,mydecorationsSetLineWidth,
         mydecorationsPathmorphing,mycfoot,myrenderlayout,cvdec,%
         definecolor,includegraphics,ifdefined,mcbegin,mcend,%
         ProvidesPackage,PackageInfo,PackageWarningNoLine,%
         @ifclassloaded,ExecuteOptions,PackageWarning,textcolor%
      },%
      commentstyle=\color{Emerald}\ttfamily,%
      frame=lines%
   }
}{\codeskip}

\ifparchment

\usepackage{framed}
\usetikzlibrary{decorations.pathmorphing,decorations.shapes,calc}
\pgfmathsetseed{1}
\pgfdeclarelayer{background}
\pgfsetlayers{background,main}

\tikzset{
  normal border/.style={lightgray!12,
      decorate, decoration={shape, segment length=0.5cm, amplitude=.7mm}},
  torn border/.style={orange!30!black!5, decorate,
      decoration={random steps, segment length=.5cm, amplitude=1.7mm}}}

\def\parchmentframe#1{
\tikz{
   \node[inner sep=2em] (A) {#1};
   \begin{pgfonlayer}{background}
   \fill[normal border]
      (A.south east) -- (A.south west) --
      (A.north west) -- (A.north east) -- cycle;
   \end{pgfonlayer}}}

\def\parchmentframetop#1{
\tikz{
   \node[inner sep=2em] (A) {#1};
   \begin{pgfonlayer}{background}
   \fill[normal border]
      (A.south east) -- (A.south west) --
      (A.north west) -- (A.north east) -- cycle;
   \fill[torn border]
      ($(A.south east)-(0,.2)$) -- ($(A.south west)-(0,.2)$) --
      ($(A.south west)+(0,.2)$) -- ($(A.south east)+(0,.2)$) -- cycle;
   \end{pgfonlayer}}}

\def\parchmentframebottom#1{
\tikz{
   \node[inner sep=2em] (A) {#1};
   \begin{pgfonlayer}{background}
   \fill[normal border]
      (A.south east) -- (A.south west) --
      (A.north west) -- (A.north east) -- cycle;
   \fill[torn border]
      ($(A.north east)-(0,.2)$) -- ($(A.north west)-(0,.2)$) --
      ($(A.north west)+(0,.2)$) -- ($(A.north east)+(0,.2)$) -- cycle;
   \end{pgfonlayer}}}

\def\parchmentframemiddle#1{
\tikz{
   \node[inner sep=2em] (A) {#1};
   \begin{pgfonlayer}{background}
   \fill[normal border]
      (A.south east) -- (A.south west) --
      (A.north west) -- (A.north east) -- cycle;
   \fill[torn border]
      ($(A.south east)-(0,.2)$) -- ($(A.south west)-(0,.2)$) --
      ($(A.south west)+(0,.2)$) -- ($(A.south east)+(0,.2)$) -- cycle;
   \fill[torn border]
      ($(A.north east)-(0,.2)$) -- ($(A.north west)-(0,.2)$) --
      ($(A.north west)+(0,.2)$) -- ($(A.north east)+(0,.2)$) -- cycle;
   \end{pgfonlayer}}}

\newenvironment{parchment}{%[1][Example]{%
   \def\FrameCommand{\parchmentframe}%
   \def\FirstFrameCommand{\parchmentframetop}%
   \def\LastFrameCommand{\parchmentframebottom}%
   \def\MidFrameCommand{\parchmentframemiddle}%
   \vskip\baselineskip
   \MakeFramed {\FrameRestore}}
{\endMakeFramed}
\fi

\begin{document}
\pagestyle{useheadings}
\pagestyle{plain}

\pagenumbering{roman}

\begin{titlepage}
\begin{center}

\renewcommand{\thefootnote}{\fnsymbol{footnote}}
\newcommand{\HRule}{\rule{\linewidth}{0.5mm}}

\def\versionMsg{This file has version number \version{} \texttt{---} %
   documentation dated April 13, 2012 \texttt{---} %
   last revised \today}

\textsc{\large\color{gray}\today}\\[0.5cm]

\HRule\\[0.4cm]%
{\huge\medskip\bfseries\myTitle\footnote{\versionMsg}}\\[0.4cm]%
\HRule\\[1.5cm]

\large\emph{Author:}\\
\href{mailto:\myMail}{\textsc{\myName}}

\vspace{20pt}
\includegraphics[width=0.30\textwidth]{Images/logo-1.png}\\[1cm]

\vspace{10pt}
\textbf{\textit{Abstract}}\normalsize\par\bigskip
\begin{minipage}{0.75\textwidth}
\noindent
This \LaTeX{} class provides a set of functionality for writing \textit{curriculum vit\ae{}}
with different layouts. To achieve this goal, it adopts a different approach with respect
to the other c.v. classes or packages.

Basically, the idea is that a user can write some custom configuration directives, by means of which
is possible both to produce different c.v. layouts and quickly switch among them.

In order to process such directives, this class uses a set of lists, provided by the package
\textit{etextools}. A basic support for \textit{TikZ} decorations is also provided.

\end{minipage}

\renewcommand{\thefootnote}{\arabic{footnote}}\setcounter{footnote}{0}

\end{center}
\end{titlepage}
\pagestyle{scrheadings}

\pdfbookmark{\contentsname}{tableofcontents}
\setcounter{tocdepth}{2}
\tableofcontents
\cleardoublepage

\pagenumbering{arabic}
\chapter{Fundamentals}
\quotes[1973]{
   ``Computer programming is an art, because it applies
     accumulated knowledge to the world, because it
     requires skill and ingenuity, and especially because
     it produces objects of beauty''
}{Knuth}
\ifparchment\begin{parchment}\fi
\section{Introduction}

The main goal of this class (\classname) is to give support for creating \textit{curriculum
vit\ae{}} (\CV) with different layouts, allowing easy switching among them.
The class also provides basic support for using the \textit{TikZ} decorations and defines a bunch
of commands for managing the contents of a \CV, though that is not its primary goal.
On \keyword{CTAN} archives, there are available various \CV packages more
\textit{contents-oriented}, as it were, and they may be used together with this class.
Before starting to describe the class more in details, it goes without saying that any advice or
constructive criticism is greatly appreciated.

\section{Class files}
The class \classname is composed by six files. A short brief of each one is given here:
\squishlist
   \item \ctext{mycv.cls}
      \begin{myindent}
      it is the main file and, basically, handles the class options~\myref{sec:class-options} as
      well as the inclusion of all other files, except for \textit{mycv\_version.def};
      \end{myindent}
   \item \ctext{mycv\_base.def}
      \begin{myindent}
      it contains all the commands and definitions dealing with the layout components of a
      \CV~\myref{sec:layout-components}: it is the core file;
      \end{myindent}
   \item \ctext{mycv\_style.sty}
      \begin{myindent}
      it contains the default style commands~\myref{sub:style-commands} provided by the class: if
      the default style is not used,
      this file will not be included by \textit{mycv.cls};
      \end{myindent}
   \item \ctext{mycv\_dec.sty}
      \begin{myindent}
      it contains the decoration commands~\myref{sub:decoration-commands} provided by the class: if
      decorations are not enabled%
      , this file will not be included by \textit{mycv.cls};
      \end{myindent}
   \item \ctext{mycv\_misc.def}\hspace{5pt}and\hspace{5pt}\ctext{mycv\_version.def}
      \begin{myindent}
      they respectively contain some miscellaneous commands and the version string.
      \end{myindent}
\squishend

\section {Layout components}
\label{sec:layout-components}

This class considers a \textit{curriculum vit\ae{}} as logically divided into three main components:
\textit{header}, \textit{body} and \textit{footer}.
To each of these ones, a list, that basically contains some sub-components, is associated; files
with the \CV contents are also considered sub-components.
For these reasons, we can actually say that \classname uses a sort of \textit{list-driven} approach.

\subsection{Main components}

\classname recognizes the following three lists that, for all intends and purposes, are a concrete
representation of the main logical components:

\bigskip
\begin{myindent}
\textbf{headerlayoutlist},\hspace{5pt}\textbf{bodylayoutlist},\hspace{5pt}\textbf{footerlayoutlist}.
\end{myindent}

\bigskip
It is mandatory, for the correct behavior of the class, to not change the above list's names. In
the case a component is not required, the relative list may be omitted: for example, if a \CV does
not have a footer component, the list \textit{footerlayoutlist} is not strictly necessary.\\
What follows is an example of a list definition:\codeskip
\begin{latexcode}[numbers=none]
\def\headerlayoutlist{sub-component1,sub-component2,[...]}
\end{latexcode}

\subsection{Sub-components}

We previously said that \classname is based on three main components (\textit{header},
\textit{body} and \textit{footer}) and that each of these ones are represented by a list. A list
(therefore a main component), in turn, may have one or more sub-component, separated by
a comma, which are identified as follows:

\squishlist[0pt]
   \item \texttt{\ctext{Main}[Header|Body|Footer]\ctext{PageBegin}};
   \item \texttt{\ctext{Main}[Header|Body|Footer]\ctext{PageEnd}};
   \item \texttt{\ctext[blue]{Sub}[Header|Body|Footer]\ctext[blue]{PageBegin}};
   \item \texttt{\ctext[blue]{Sub}[Header|Body|Footer]\ctext[blue]{PageEnd}};
   \item \texttt{\ctext[orange]{filename}} with the (partial) \CV contents.
\squishend\sep

Both ``\verb|Main[...]PageBegin|'' and ``\verb|Sub[...]PageBegin|'' are \textit{minipages}; the
difference is that the former have a default width of $100\%$ of the \textit{textwidth}
macro, while that value is $44\text{-}45\%$ for the latter (it depends on the components type).

A \textit{filename} sub-component may either be the name of a file or a macro with it:
depending on the case, the syntax slightly changes.%~\myreftwo{subsub:component-options}.

\subsubsection{Sub-components options}
\label{subsub:component-options}

Each sub-component, \textit{filename} included, may have associated options, with colons
as separators, so that the syntax is something like:\codeskip
\begin{latexcode}[numbers=none]
   $sub-component$:option1:option2:[...]
\end{latexcode}

If truth be told, each option already has its own separator, so colons are not strictly necessary
and, as a separator, any other symbol may be used. If wanted, it is also possible to not have any
separator at all, but this is not recommended if only for a matter of clarity.

\medskip
Options for a sub-component are of different types, as listed below:

\def\tmpcolor{gray}
\squishlist
   \item <\optname{[pre\textbar post]cmd}\verb|:command1:command2:[...]|>\optdsep
      a sequence of commands is executed \textit{before}/\textit{after} the
      beginning or ending of a sub-component (\textit{filename} included).
      A command may have a sequence of arguments, separated by ``='',
      each of which can either be \textit{optional} or \textit{mandatory}.
      In total, the class recognizes four types of arguments:
      \squishlist[0pt]
         \item \ctext[blue]{arg} (mandatory argument equivalent to \{arg\});
         \item \ctext[blue]{@arg} (optional argument equivalent to [arg]);
         \item \ctext[blue]{!arg} (optional argument equivalent to <arg>);
         \item \ctext[blue]{*} (optional argument equivalent to *).
      \squishend
   \item /\optname{m[l\textbar r]}\verb|<value>|/\hfill\opthsep{\small($1$)}\optvsep
      /\optname{endm[l\textbar r]}/\hfill\opthsep{\small($2$)}\optdsep
      changes the \textit{left}/\textit{right} margin of a text portion of a document, between
      option {\small($1$)} and option {\small($2$)}; in a typical usage, these options are
      associated with different sub-components, such as \verb|*PageBegin| and \verb|*PageEnd|.
      Each time the option {\small($1$)} is used, the option {\small($2$)} is also required
      for ending the margin modification, except for the \textit{filename}
      sub-component that automatically does that. Example (it moves the left margin to the
      right of $0.2$in):
      \begin{myindent}
      \texttt{%
         \ctext[\tmpcolor]{%
            SubBodyPageBegin:</ml0.2in/>\\
            \omissis\\
            SubBodyPageEnd:</endml>}%
      }.
      \end{myindent}
   \item \verb|<width-value>|\optdsep
      sets the width of a sub-component in terms of \textit{textwidth} percentage.
      This option only exists for ``\verb|*PageBegin|`` sub-components. Example:
      \ctext[\tmpcolor]{\texttt{SubBodyPageBegin:<0.48>}}.
   \item /\optname{pagesize}\verb|<value>|/\optdsep
      sets the width of a sub-component, as the option above, but in terms of absolute reference
      (instead of \textit{textwidth} percentage). Also this option only exists for
      ``\verb|*PageBegin|`` sub-components. Example:
      \ctext[\tmpcolor]{\texttt{SubBodyPageBegin:/pagesize5.5in/}}.
   \item /\optname{pagebreak}/\optdsep
      permits to break two contiguous sub-components, aligning them one above the other, instead
      of side by side (that is the default behavior). This option only exists for
      ``\verb|*PageEnd|'' sub-components.
      Example: \ctext[\tmpcolor]{\texttt{SubBodyPageEnd:/pagebreak/}}.
   \item \optname{*}macroname\optvsep
      filename\optname{@}\optdsep
      \argname{macroname} is a macro expanding to the name of a file (with the \CV contents),
      while \argname{filename} is directly the name itself (the only \textit{non-alphanumeric}
      characters allowed there are ``\_'' and ``-'').
      Example: \ctext[\tmpcolor]{\texttt{*headerfile}}, where the macro \textit{headerfile}
      is somewhere defined.
\squishend
\ifparchment\end{parchment}\fi
\chapter{Usage}
\quotes{%
   “There are two ways to write error-free programs; only the third one works”%
}{Alan J. Perlis}
\ifparchment\begin{parchment}\fi
\section{Requirements}
When \textit{decorations} are not enabled and the \textit{default style} is not used, \classname
has
the following requirements:\codeskip
\begin{latexcode}[numbers=none]
\RequirePackage{kvoptions} % for class options with key-value format
\RequirePackage{etextools} % for lists and other useful tools
\RequirePackage{ifthen}    % for the \ifthenelse command
\RequirePackage{xstring}   % for string utilities
\RequirePackage{svn-prov}  % for file info extracted from SVN
\RequirePackage{hyperref}  % for hypertext links and other stuff
\end{latexcode}

If the default style is used, by means of the class option
``\textit{style}''~\myref{sec:class-options}, this class requires (in addiction to
\textit{hyperref} and \textit{svn-prov}):\codeskip
\begin{latexcode}[numbers=none]
\RequirePackage{xparse}    % for commands with multiple default arguments
\RequirePackage{pifont}    % for the 'ding' style (itemize environment)
\RequirePackage{titlesec}  % for title format and spacing
\RequirePackage{fancyhdr}  % for custom headers and footers
\RequirePackage{xcolor}    % for colors
\RequirePackage{calligra}  % for the calligra font
\RequirePackage{times}     % for the times font
\RequirePackage{marvosym}  % symbols - phone
\RequirePackage{amssymb}   % symbols - email
\end{latexcode}

Finally, if decorations are enabled, by using the class
option ``\textit{withDec}``~\myref{sec:class-options}, this class also requires (in addiction
to \textit{svn-prov}, \textit{xparse} and \textit{xstring}):\codeskip
\begin{latexcode}[firstnumber=1,numbers=none]
\RequirePackage{tikz}      % for graphics
\end{latexcode}

\section{Class options}
\label{sec:class-options}

\classname can use any option supported by the \textit{article} class, on which is based. In
addiction, it provides the following options:
\def\tmpcolor{brown}
\squishlist
\item {\color{red}language=<\argname[\tmpcolor]{string}>}
   \begin{myindent}
   string language to pass to the \textit{babel} package for the document (\CV) language;
   \end{myindent}
\item {\color{red}cntdir=<\argname[\tmpcolor]{dirname}>}
   \begin{myindent}
   sets the directory name where \classname will search for files with the \CV contents.\\
   The default one is ``Contents'';
   \end{myindent}
\item {\color{red}style=<\argname[\tmpcolor]{filemane}>}
   \begin{myindent}
   specifies the file name (with or without the extension ``.sty'') containing the style commands.
   By default, the file \textit{mycv\_style.sty}, provided by the class itself, is that used.
   It is also possible to not use any style file by specifying the value ``none'' as file name;
   \end{myindent}
\item {\color{red}mdlname=<\argname[\tmpcolor]{name}>}
   \begin{myindent}
   registers a name for the layout (model) intended to be used: in this way is possible, for
   example, to select the appropriate layout configuration file or a layout-specific portion of
   code;
   \end{myindent}
\item {\color{red}withDec}
   \begin{myindent}
   enables support for decorations (provided by the \textit{TikZ} package).
   \end{myindent}
\squishend

\section{Class commands}

Here follows the complete list of the commands provided by \classname. The style commands
are only available if the class option ``\textit{style}'' was used;
the same goes for the decoration commands,
which need the class option ``\textit{withDec}''.\\
In the following text of this section, when present, the form {\color{blue}\omissis} (or
{\color{blue}<...>}) indicates the default choice for an optional argument of a command.

\subsection{Conditionals}

\squishlist
\item \cmdname{ifoption}
   \marg{option}\arghsep
   \marg{true}\arghsep\marg{false}\argvsep
   \cmdname{ifmodel}
   \marg{mdlname}\arghsep
   \marg{true}\arghsep\marg{false}\cmddsep
   \textit{ifoption} checks whether \argname{option} was used,
   while \textit{ifmodel} checks whether \argname{mdlname} was registered in the class; then both
   commands use the appropriate \argname{true} or \argname{false} block of code.
\squishend

\subsection{Default style}
\label{sub:style-commands}

\squishlist
\item \cmdname{mysectionTitleFormat}
   \begin{myindent}
   \dargone{titlerule-color-above}{myheadingscolor}\argvsep
   \dargone{titlerule-color-below}{myheadingscolor}
   \end{myindent}\cmddsep
   \argname{titlerule-color-above} is the color for the rule above a section name, while
   \argname{titlerule-color-below} is for the one below. \textit{myheadingscolor} is the default
   color.
\item \cmdname{mysectionTitleSpacing}
   \begin{myindent}
   \dargone{left}{$0\text{pt}$}\arghsep
   \dargone{beforesep}{$0\text{pt}$}\arghsep
   \dargone{aftersep}{$5\text{pt}$}
   \end{myindent}\cmddsep
   this command is just an alias for
   \textit{\textbackslash{titlespacing}\{\textbackslash{section}\}\{\argname{left}\}%
   \{\argname{beforesep}\}\{\argname{aftersep}\}}. See the \textit{titlesec} package for further
   information.
\item \cmdname{mycfoot}
   \marg{text}\cmddsep
   adds \argname{text} to the page footer. It may be useful, for example, to show information
   about the last update of the document.
\item \cmdname{myitemize}
   \cmddsep
   a list environment that uses the \textit{ding} style.
\squishend

\subsection{Decorations}
\label{sub:decoration-commands}

\classname provides some commands for \textit{TikZ} decorations. The support
provided is not complete at all (on the other hand \textit{TikZ} has a huge amount of
functionality), but it is enough for this class purposes. The only \textit{TikZ} path supported is
\textit{rectangle}.

\squishlist
\item \cmdname{mydecorationsPathmorphing}[*]
   \begin{myindent}
   \dargone{show-decoration}{$1$}\argvsep
   \marg{decoration-type}\argvsep
   \dargone{decoration-color}{gray}\argvsep
   \dargtwo{shading-type}{radial}\argvsep
   \dargtwo{background-color}{white}%
   \end{myindent}\cmddsep
   \argname{show-decoration}, if equals $1$, shows the decoration \argname{decoration-type},
   while if $0$ does not.
   The \textit{starred} version of the command uses the shading technique and the last argument is
   the background shading color.\\
   The \textit{not starred} version does not consider the argument \argname{shading-type}
   (just for a matter of clarity, a ``none'' value may be used) and the last argument is simply the
   background color.\\
   \argname{decoration-type} was tested with the following values: ``shape'', ``straight'',
   ``zigzag``, ``random steps``, ``saw``, ``bent``, ``bumps``, ``coil``, ``snake'' and
   ``Koch snowflake``.\\
   \argname{shading-type} was tested with ``radial'' and ``ball'' shadings.
\item \cmdname{mydecorationsShape}
   \begin{myindent}
   \dargone{show-decoration}{$1$}\arghsep
   \marg{decoration-type}\arghsep
   \dargone{decoration-color}{gray}
   \end{myindent}\cmddsep
   \argname{show-decoration}, if equals $1$, shows the decoration \argname{decoration-type},
   while if $0$ does not.\\
   \argname{decoration-type} was tested with the following decorations: ``dart'', ``diamond'',
   ``rectangle'' and ``star''.
\item \cmdname{mydecorationsFading}
   \begin{myindent}
   \dargone{path-fading}{north}\argvsep
   \marg{primary-color}\argvsep
   \dargone{color-gradient}{$80$}\argvsep
   \dargone{secondary-color}{black}\argvsep
   \dargtwo{opacity}{$1.0$}%
   \end{myindent}\cmddsep
   the resulting fill color is given by \argname{primary-color}, \argname{color-gradient}
   and \argname{secondary-color}, which are composed as follows:
   \argname{primary-color}!\argname{color-gradient}!\argname{secondary-color}.
\item \cmdname{mydecorationsSetPos[XTL|YTL|XBR|YBR]}
   \begin{myindent}
   \dargone{coordinate-value}{$1\text{cm}\mid-1\text{cm}\mid-1\text{cm}\mid1\text{cm}$}
   \end{myindent}\cmddsep
   sets the position for the decoration in use. Since the decoration path is \textit{rectangle}, it
   is sufficient to have the $(x,y)$ coordinates of two points: the top-left and bottom-right.
   \textit{XTL} stands for ``\texttt{X-Top-Left}'', \textit{XBR} for ``\texttt{X-Bottom-Right}''
   and so on.
\item
   \cmdname{mydecorationsSetLineWidth}[*]
   \dargone{line-width}{tikz value}\cmdvsep
   \cmdname{mydecorationsSetSegmentAmplitude}[*]
   \dargone{segment-amplitude}{tikz value}\cmdvsep
   \cmdname{mydecorationsSetSegmentLength}[*]
   \dargone{segment-length}{tikz value}\cmddsep
   these commands may respectively be used for modifying the properties \argname{line-width},
   \argname{segment-amplitude} and \argname{segment-length} for the decoration in use.
   \textit{Starred} versions do not require any argument and reinitialize the properties to their
   default values.
\squishend

\subsection{Miscellaneous}

\squishlist
\item
   \cmdname{mypdfauthor}
   \marg{author}\cmdvsep
   \cmdname{mypdftitle}
   \marg{title}\cmdvsep
   \cmdname{mypdfsubject}
   \marg{subject}\cmddsep
   these commands do nothing but register \argname{author}, \argname{title} and \argname{subject}
   information in the document properties of the pdf is being produced.

\item \cmdname{mylang}
   \marg{text}\arghsep
   \dargone{language}{english}\cmddsep
   temporarily changes the language in use (\textit{babel} package) to \argname{language} for
   \argname{text}.

\item \cmdname{mychangemargin}
   \marg{left-margin}\arghsep
   \marg{right-margin}\cmddsep
   \textit{mychangemargin} environment changes the left and right margin of a portion of
   text. The environments \textit{mychangemarginLeft} and \textit{mychangemarginRight}, whose
   meaning is straight forward, are also available.

\item \cmdname{myrenderlayout}
   \dargone{component}{a}\cmddsep
   processes and draws the layout component(s). The option value ``h'' is for the header component,
   ``b'' and ``f'', respectively, for the body and footer ones, while ``a'' is for all components.
\squishend

\section{Some examples}

This section gives some \textit{minimal} examples and does some considerations about the use of
\classname (the class permits to do much better with a little patience). This is done by creating
two \textit{curriculum vit\ae{}} with the same contents, but different layouts: one \CV will use a
double page layout (abbreviated \DPL from here forward), while the other will use a single page
layout (\SPL).

The sample code presented here can be found in the ``Examples'' directory shipped with the
\textit{mycv} bundle, which this document is part of, and that also contains files with the \CV
contents: these files are not listed in the present document, as they do not contain anything worth
being mentioned for the purpose of these notes.

First and foremost, to keep the code organized, we need a file containing the layout components
for the \DPL and \SPL (\textit{model-layouts.tex}). We opt to share the \textit{header} and
\textit{footer} components, so we also create a second file named \textit{model-common.tex},
such as in listing~\ref{lst:model-common}.
\begin{latexcode}[firstnumber=1,caption=model-common.tex,label=lst:model-common]
% ----------------------
% the shared header list
% ----------------------
\def\headerlayoutlist{%
   &MainHeaderPageBegin&:<postcmd:vspace=10pt>,
      % --------------------------------- left header
      |SubHeaderPageBegin|:<precmd:hfill>,
         % header file (1)
         $header_title@$,
      |SubHeaderPageEnd|:<postcmd:hfill>,
      % --------------------------------- right header
      |SubHeaderPageBegin|,
         % header file (2)
         $header_contacts@$,
      |SubHeaderPageEnd|,
   &MainHeaderPageEnd&%
}

% ----------------------
% the shared footer list
% ----------------------
\def\footerlayoutlist{footer_sign@}
\end{latexcode}

\subsection{Double page layout}

Here we deal with the layout components specific for the \DPL, as showed in
listing~\ref{lst:dplmodel}.
\begin{latexcode}[firstnumber=1,caption=DPL model (part of 'model-layouts.tex'),label=lst:dplmodel]
\def\bodylayoutlist{% the DPL's body list
   % ---------------------------------------------------------
   % moves the right margin to the left (text and title rules)
   % ---------------------------------------------------------
   &MainBodyPageBegin&:<0.96>,
      % ---------------------------------------------------------
      % the 2 directives below are just used as a trick to do the
      % same thing for the left margin (it is moved to the right)
      % ---------------------------------------------------------
      |SubBodyPageBegin|,
      |SubBodyPageEnd|,
      % -----------------------------------------
      % left page (0.48 of textwidth)
      % -----------------------------------------
      |SubBodyPageBegin|:<0.48>,
         $contents_partA@$:<precmd:vspace=10pt:sectionnumber=1>,
         $contents_partB@$:<precmd:vspace=10pt:sectionnumber=2>,
         $contents_partC@$:<precmd:vspace=10pt:sectionnumber=3>,
      |SubBodyPageEnd|:<postcmd:hfill>,
      % -----------------------------------------
      % right page (0.48 of textwidth)
      % -----------------------------------------
      |SubBodyPageBegin|:<0.48>,
         $contents_partA@$:<precmd:vspace=10pt:sectionnumber=4>,
         $contents_partB@$:<precmd:vspace=10pt:sectionnumber=5>,
      |SubBodyPageEnd|,
      % -----------------------------------------
   &MainBodyPageEnd&%
}
\end{latexcode}

\subsection{Single page layout}

As far as the \SPL, we do not need to use the \verb|*PageBegin| components, but it is sufficient to
directly include the files with the contents. The resulting code is showed
in listing~\ref{lst:splmodel}.
\begin{latexcode}[firstnumber=1,caption=SPL model (part of
'model-layouts.tex'),label=lst:splmodel]
\def\bodylayoutlist{% the SPL's body list
   % ---------------------------------------------------
   $contents_partA@$:<precmd:vspace=10pt:sectionnumber=1>,
   $contents_partB@$:<precmd:vspace=10pt:sectionnumber=2>,
   $contents_partC@$:<precmd:vspace=10pt:sectionnumber=3>,
   $contents_partA@$:<precmd:vspace=10pt:sectionnumber=4>,
   $contents_partB@$:<precmd:vspace=10pt:sectionnumber=5>
   % ---------------------------------------------------
}
\end{latexcode}

\subsection{Main file}

Since we both have the components for the double and single page layouts, we can proceed writing the
main file (\textit{mycv-example-main.tex}) that picks and use them.\\
We start by setting up some options for the
class, such as those related to the decorations and language support, as well as the name of the
model (layout) we mean to register.
Besides, we opt to store the \CV contents files in the current directory (that is not the default one
where the class searches for the contents files), so there is  need to specify its path with
the option ``\textit{cntdir}''.

In listing~\ref{lst:mycv-example-main} we take the \DPL as an example, but switching to the
\SPL would just be a matter of changing the ``\textit{mdlname}'' option from \keyword{verDPL} to
\keyword{verSPL}.
\begin{latexcode}[firstnumber=1,label=lst:mycv-example-main,caption=mycv-example-main.tex]
\documentclass[10pt,mdlname=verDPL,withDec,cntdir=.,language=english]{mycv}
\input{mycv-example-common}
\begin{document}
\cvdec\myrenderlayout\mycfoot{Last update: \today}
\end{document}
\end{latexcode}

The file \textit{mycv-example-common[.tex]}, showed in listing~\ref{lst:example-common},
selects the appropriate layout components (listings~\ref{lst:dplmodel} or \ref{lst:splmodel}) to be
included and, subsequently, processed by \textit{\textbackslash{}myrenderlayout}.
In addiction, the file \textit{mycv-example-common.tex} contains some decoration commands, which
need to be used only if the option ``\textit{withDec}' was given to the class; this is the
reason of the conditional command \textit{\textbackslash{}ifoption}.
\begin{latexcode}[firstnumber=1,caption=mcv-example-common.tex,label=lst:example-common]
[...]
\ifoption{withDec}{%
   \newcommand{\cvdec}{
      \mydecorationsSetLineWidth[0.3mm]%
      \mydecorationsPathmorphing*{coil}<radial><lightgray>
   }
   [...]
}{\newcommand{\cvdec}{}}

% include layouts components
\input{model-layouts}
[...]
\end{latexcode}

\subsection{Layout notes}

When a double layout page is used, it may occur, for example, that a section is too long for a
page: this would not be a problem with a single page layout, since \LaTeX{} would automatically
break the section contents. Unfortunately, with a double page layout the behavior is substantially
different: this is because the class uses a \textit{minipage-based mechanism} and a minipage is by
itself not breakable. Thus, what happens is that part of the section contents comes out from the
page margins.

When a problem such as this occurs, a possible workaround is to manually break the section contents.
This can be done by using a counter that keeps track of the number of times a same file is included:
when the counter is equal $1$, a part of the section contents is included in the left page,
otherwise is the remaining one to be included in the right page.
Listing~\ref{lst:split-contents-example} shows a practical example of what just discussed.
\begin{latexcode}[firstnumber=1,label=lst:split-contents-example,
caption=workaround example]
% ---------------------------------------------------------
% file with the section contents: i.e. <section_skills.tex>
% ---------------------------------------------------------
% increases the counter 'cnt': it's defined outside this file
\stepcounter{cnt}

% selects the appropriate part of the file contents
\newcommand{\condblock}[2]{\ifthenelse{\value{cnt}<2}{#1}{#2}}

\condblock{skills section contents part A}%
   {skills section contents part B (the remaining part)}

% ------------------------------------------------------
% file with the DPL components: i.e. <model-layouts.tex>
% ------------------------------------------------------
\def\bodylayoutlist{%
   |SubBodyPageBegin|:<0.48>, % left page
      % include part A in the left page
      $section_skills@$,
      [...]
   |SubBodyPageEnd|,
   |SubBodyPageBegin|:<0.48>, % right page
      % include part B in the right page
      $section_skills@$,
      [...]
   |SubBodyPageEnd|
}
\end{latexcode}

Of course the proposed workaround is not the best we could wish for, since it requires manual
operations.
As an alternative, it is possible to not use the \verb|*PageBegin| and \verb|*PageEnd| mechanism,
delegating the double page layout to an external package (i.e. \textit{multicols}).
This kind of approach is showed in listings~\ref{lst:model-dpl2}.
\begin{latexcode}[firstnumber=1,caption=DPL$2$ model (part of
'model-layouts.tex'),label=lst:model-dpl2]
\newcommand{\mcbegin}{\begin{multicols}{2}}
\newcommand{\mcend}{\end{multicols}}

\def\bodylayoutlist{%
    $contents_partA@$:<precmd:vspace=10pt:mcbegin:sectionnumber=1>,%
    [...]
    $contents_partB@$:<precmd:vspace=10pt:sectionnumber=6>:<postcmd:mcend>,%
}
\end{latexcode}

Notice the usage of the commands \textit{\textbackslash{mcbegin}} and
\textit{\textbackslash{mcend}}, which act as a wrapper for the \textit{multicols} environment.

\section{Split file contents}

This class uses a file based approach as far as the contents of a \CV, that is to say it needs that
the contents be into separated file/s (at least one) with respect to the main file.

As the class works, having the contents in just one single file is not as good as would be with
multiple files (ideally a file for each section of the document), since multiple files allow to
more easily customize the \CV layout without directly acting on the contents.
On the other hand, a multiple file approach may compel to use several files and not be
convenient.\\
For these reasons, this class provides a \keyword{PERL} script named
\textit{mycv\_split\_contents.pl} (for a list of all the options provided, type
\textit{mycv\_split\_contents.pl -h} inside a shell environment)
that can process a file and, according to what specified by the
directives in the file itself, split its contents into several files (which will be those
effectively used by the class). In this way is possible to directly manage one single file for the
contents.

If requested, the script can also create a basic model file with the layout components:
it has to be considered just as a skeleton and probably needs to be edited by hand afterward.

The directives that may be used inside a contents file have the following form:\codeskip
\begin{latexcode}[numbers=none]
###::|filename|:$component$:&commands&
\end{latexcode}

{\color{orange}\textit{filemane}} is the name of the file to create,
{\color{brown}\textit{component}} is the main component (\textit{header}, \textit{body} or
\textit{footer}) that the file represents, while {\color{red}\textit{commands}} are the relative
commands associated to the component.
Both \textit{component} and \textit{commands} are not mandatory, but may be useful, especially
\textit{component}, when is requested to generate a basic model file.\\
The script \textit{mycv\_split\_contents.pl} may be called inside the main file as showed in
listing~\ref{lst:mycv-example-main2}.

\begin{latexcode}[firstnumber=1,caption=Split contents example,label=lst:mycv-example-main2]
% -------------------------------------------------------------------
% To compile this file is necessary to use the option '-shell-escape'
% to enable the 'write18' construct.
% -------------------------------------------------------------------
\documentclass[10pt,mdlname=verSPL,withDec,language=english]{mycv}
\immediate\write18{<dirpath>/mycv_split_contents.pl -i cv-contents.tex}
[...]
\end{latexcode}

When the file \textit{cv-contents.tex} (listing~\ref{lst:cv-contents}) is processed by the script,
the files \textit{*.tex} are created in the directory ''Contents`` (the default one).
\begin{latexcode}[firstnumber=1,caption=cv-contents.tex,label=lst:cv-contents]
% -----------------------------
% ###::header_title.tex::header
% -----------------------------
[...]
% -------------------------------------------------------------------
% ###::contents_partA.tex::body::<precmd:vspace=10pt:sectionnumber=1>
% -------------------------------------------------------------------
[...]
% -------------------------------------------------------------------
% ###::contents_partB.tex::body::<precmd:vspace=10pt:sectionnumber=2>
% -------------------------------------------------------------------
[...]
% ----------------------------
% ###::footer_sign.tex::footer
% ----------------------------
[...]
\end{latexcode}

\bigskip
\fbox{That's all, {\color{brown}happy \LaTeX{}ing!}}
\hfill\Large\calligra{AndreaGhersi}
\ifparchment\end{parchment}\fi

\end{document}