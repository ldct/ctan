\documentclass[a4paper,twoside]{article}
\usepackage{topfloat}

\renewcommand\figurename{Fig.}
\renewcommand\tablename{Tab.}

\begin{document}

\centerline{\huge Topfloat.sty}\vskip1cm
\centerline{\Large pacchetto di A. Macchia (sett. '99)}\vskip1.5cm

Topfloat \`e un pacchetto che permette di impaginare fino a due elementi
mobili (``{\sl floats\/}'') nella parte superiore della pagina. Il pacchetto
consiste principalmente di un ambinete {\tt topfloat} al quale si deve dare
un parametro rappresentane il numero di elemeni da impaginare (cio\`e 1 o 2).
Nell'ambiente vanno inseriti gli elemeti mobili racchiusi dai comandi
\verb!\topI<...>\endtopI! e se c'\`e il secondo elemento da impaginare, lo
si deve racchiudere tra i comandi \verb!\topII<...>\endtopII!; le eventuali
didascalie sono inserite grazie ai comandi \verb!\tabcap{...}! per le tabelle
e \verb!\figcap{...}! per le figure inseriti tra un \verb!\endtopI! e un
\verb!\topII! oppure tra un \verb!\endtopII! ed un \verb!\end{topfloat}!
oppure ancora tra un \verb!\endtopI! ed un \verb!\end{topfloat}!.

I comandi \verb!\tabcap! e \verb!\figcap! ammettono un parametro opzionale
che se \`e settatto ad una misura maggiore di $0\,$pt allora si \`e dichiarato
quanto deve essere larga la didascalia. Se si specifica un $0\,$pt allora
il pacchetto assumer\`a una larghezza in modo che la didascalia entri in una
sola riga. Se non si specificher\`a nulla, la larghezza della didascalia
sar\`a quella dell'elemento mobile impaginato.

\noindent Le fonts che si usano per impaginare le didascalie sono:

\begin{tabular}{  r l}
        \verb!\@topdidascalia! &: per il testo della didascalia \\
        \verb!\@toptipo!       &: per il tipo e il numero dell'elemento
        (per esempio: \\ & ``figura 1'' o ``tabella II.2'')
\end{tabular}

\vskip.5cm\noindent Ricapitolando le sintassi dell'ambiente e dei comandi:

\begin{tabular}{  l p{6cm}}
   \verb!\begin{topfloat}!$\{numero\}$ 
        & impagina $numero$ elementi mobili \\
   \verb!\topI!
        & marca l'inizio del I$^o$ el. da impaginare \\
   \verb!\endtopI!
        & marca la fine del I$^o$ el. da impaginare \\
   \verb!\topII!
        & marca l'inizio del II$^o$ el. da impaginare \\
   \verb!\endtopII!
        & marca la fine del II$^o$ el. da impaginare \\
   \verb!\tabcap![{\it larghezza}\/]$\{didascalia\}$
        & inserice una didascalia per tabelle con 
          una larghezza di ``$larghezza$\/'' se \`e specificata ed \`e 
          maggiore di $0\,$pt, pari alla lunghezza della didascalia intera se
          \`e specificato $0\,$pt oppure pari alla larghezza dell'elemento 
          impaginato se il parametro opzionale non viene specificato; 
          comunque se la larghezza dedicata per la didascalia \`e maggiore
          della didascalia stessa, allora questa viene centrata \\
   \verb!\figcap![{\it larghezza}\/]$\{didascalia\}$
        & stesso discorso visto per \verb!\tabcap! \\
\end{tabular}

\vskip.5cm\noindent Ora per chiarire meglio i concetti, vediamo alcuni esempi:

\begin{verbatim}
\begin{topfloat}{1}     % impagina un solo elemento
   \topI
       \begin{tabular}{|c|c|} 
                \hline
                Nome    & Et\`a \\ \hline
                Angelo  & 24 anni \\
                Maria   & 29 anni \\
                Luigi   & 30 anni \\
                Andrea  & non mi ricordo proprio \\
                \hline
       \end{tabular}
   \endtopI
   \tabcap{questo \`e un esempio di tabella impaginata 
           con una lunga didascalia}\label{tab:1}
\end{topfloat}
\end{verbatim}

\begin{topfloat}{1}     % impagina un solo elemento
   \topI
       \begin{tabular}{|c|c|} 
                \hline
                Nome    & Et\`a \\ \hline
                Angelo  & 24 anni \\
                Maria   & 29 anni \\
                Luigi   & 30 anni \\
                Andrea  & non mi ricordo proprio \\
                \hline
       \end{tabular}
   \endtopI
   \tabcap{questo \`e un esempio di tabella impaginata con una lunga 
           didascalia}\label{tab:1}
\end{topfloat}

\noindent Questo esempio produce una impaginazione come si pu\`o vedere nella
tabella \ref{tab:1} (pagina \pageref{tab:1}); nota che non \`e stato
specificato nessun parametro opzionale per \verb!\tabcap!.
Nella tabella \ref{tab:2} il comando \verb!\tabcap! \`e stato cos\`\i\
specificato:

\begin{verbatim}
   \tabcap[0pt]{questo \`e un esempio di tabella impaginata 
               con una lunga didascalia}\label{tab:2}
\end{verbatim}

\begin{topfloat}{1}     % impagina un solo elemento
   \topI
       \begin{tabular}{|c|c|} 
                \hline
                Nome    & Et\`a \\ \hline
                Angelo  & 24 anni \\
                Maria   & 29 anni \\
                Luigi   & 30 anni \\
                Andrea  & non mi ricordo proprio \\
                \hline
       \end{tabular}
   \endtopI
   \tabcap[0pt]{questo \`e un esempio di tabella impaginata con una lunga 
           didascalia}\label{tab:2}
\end{topfloat}

\noindent mentre nella tabella \ref{tab:3}:

\begin{verbatim}
   \tabcap[7cm]{questo \`e un esempio di tabella impaginata 
               con una lunga didascalia}\label{tab:3}
\end{verbatim}

\begin{topfloat}{1}     % impagina un solo elemento
   \topI
       \begin{tabular}{|c|c|} 
                \hline
                Nome    & Et\`a \\ \hline
                Angelo  & 24 anni \\
                Maria   & 29 anni \\
                Luigi   & 30 anni \\
                Andrea  & non mi ricordo proprio \\
                \hline
       \end{tabular}
   \endtopI
   \tabcap[7cm]{questo \`e un esempio di tabella impaginata con una lunga 
           didascalia}\label{tab:3}
\end{topfloat}

\noindent Vediamo, per concludere un esempio pi\`u complesso:

\begin{verbatim}
\begin{topfloat}{2}        % impagina due elementi
  \topI
     \begin{tabular}{ cc }
     \hline
     Statistiche        & giocate \\ \hline
     ambi               & 124   \\
     terni              & 24    \\
     quaterne           & 13    \\
     cinquine           & 2 \\
     \hline
     \end{tabular}
  \endtopI
  \figcap{statistiche}\label{fig:1}
  \topII
    \fbox{$\displaystyle f(x)=\sum_{i=0}^\infty {(-1)^n\over n!}
          x^n$}
  \endtopII
  \figcap{formula}\label{fig:2}
\end{topfloat}
\end{verbatim}

\begin{topfloat}{2}        % impagina due elementi
  \topI
     \begin{tabular}{ cc }
     \hline
     Statistiche        & giocate \\ \hline
     ambi               & 124   \\
     terni              & 24    \\
     quaterne           & 13    \\
     cinquine           & 2 \\
     \hline
     \end{tabular}
  \endtopI
  \figcap{statistiche}\label{fig:1}
  \topII
    \fbox{$\displaystyle f(x)=\sum_{i=0}^\infty {(-1)^n\over n!}x^n$}
  \endtopII
  \figcap{formula}\label{fig:2}
\end{topfloat}

che produce il risultato prodotta nella figura \ref{fig:1} e nella figua
\ref{fig:2}; nota che anche se la figura \ref{fig:1} \`e in realt\`a una
tabella, siccome la didascalia \`e stata trattata con \verb!\figcap!
allora verr\`a trattata in tutto e per tutto come una figura: dunque 
nella didascalia comparir\`a la scritta ``figura'' e le informazioni
sull'elemento sar\`a inserita nella {\sl list of figures} piuttosto che nella
{\sl list of tables\/}.
Inoltre in queste 2 figure si vede come, se la didascalia \`e piu` corta
della figura, allora il testo viene centrato.

\end{document}
