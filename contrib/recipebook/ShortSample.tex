%	Filename: ShortSample.tex
% Minimal sample cookbook using RecipeBook class.  Demonstrates
% syntax for creating a title page, hyperlinked table of
% contents, and several recipe formats.

\documentclass{RecipeBook}

\begin{document}

% \titlepage[<Text>]{<image1>}{<image2>}{<image3>}
% Provide title text and three image files to \titlepage to
% generate a title page.  In the following the images are located
% in the subdirectory images.
\titlepage[Sample Title]
	{images/sample_image.jpg}
	{images/sample_image.jpg}
	{images/sample_image.jpg}
	
% \makecontents
% Generate hyperlinked table of contents
\makecontents

% Section and subsection commands for making ToC entries and 
% managing pagebreaks.
\section{Sample section num. 1}
\subsection{Sample recipe num. 1}

% \author{<name>}
% Author, to be displayed in a footnote
\author{Sample Author}

% Recipe title and subtitle
\title{Sample Title}
\subtitle{\hspace*{0.5in}sample subtitle}

% info environment.  This is a tabular environment with two
% columns, so entries on the same line are separated by & and
% lines must be ended with \\.
\begin{info}
	Makes: & X servings \\
	Cooking time: & X hours
\end{info}

% \begin[<size>]{shrink} ... \end{shrink}
% Shrink the text within this environment by value <size>.  Default
% value is 1 pt, a good place to start if your recipe is too long.
\begin{shrink}

% \recipesection{<text>}
% Formats a heading with text <text>. 
\recipesection{Ingredients}

% \begin{ingredients} ... \end{ingredients}
% \ingr[<bullet>]{<text>}
% Format ingredients in a tabular environment with bullet points
% specified by <bullet>.  The default is \square.
% \group{<text>} removes bullet point and underlines the text.

\begin{ingredients}
	\group{Sample group heading}
	\item{Sample ingredient}
	\group{Sample group heading 2}
	\item{Sample ingredient 2}
\end{ingredients}

\recipesection{Instructions}

% \instr{<step_number>}{<text>}
% Used for numbered instructions.
\step{1.}{
	Sample instruction with some extra text text text text text 
	text text text text text text text.
}

% \hlgroup{<text>}
% Used for heading within instruction set.
\hlgroup{Sample group heading}
\step{2.}{
	Sample instruction with some extra text text text text text 
	text text text text text text text.
}

\pagebreak
\pic[t]{images/sample_image.jpg}

\end{shrink}

\section{Sample side-by-side recipes}
\subsection{Sample recipe (a)}
\subsection{Sample recipe (b)}
\author{Sample Author}

% \leftbgpic[<offset>][<transparency>]{<image_file>}
% Places a transparent image with edge fading in the left
% column, with <offset> setting the vertical distance from
% the top of the page (default is \pagetop). 
% \rightbgpic[<offset>][<transparency>]{<image>}
% Same syntax for \rightbgpic and leftbgpic.  Default value
% for <offset> is \pagebottom.
\rightbgpic{images/sample_image.jpg}
\leftbgpic{images/sample_image.jpg}

%----Title and info----------------------------------------------------------%
\title{Sample recipe (a)}

\begin{info}
	Makes: & X servings \\
	Cooking time: & X hours
\end{info}

\recipesection{Ingredients}
\begin{ingredients}
	\item{Sample ingredient}
	\item{Sample ingredient}
\end{ingredients}

\recipesection{Instructions}

\step{1.}{
	Sample instruction with some extra text text text text text 
	text text text text text text text.
}

\pagebreak

\title{Sample recipe (b)}

\begin{info}
	Makes: & X servings \\
	Cooking time: & X hours
\end{info}

\recipesection{Ingredients}
\begin{ingredients}
	\item{Sample ingredient}
\end{ingredients}

\recipesection{Instructions}

\step{1.}{
	Sample instruction with some extra text text text text text 
	text text text text text text text.
}
	
\clearpage

\end{document}

