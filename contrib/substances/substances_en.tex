% arara: pdflatex: { shell: on }
% arara: pdflatex: { shell: on }
% --------------------------------------------------------------------------
% the SUBSTANCES package
% 
%   A Chemical Database
% 
% --------------------------------------------------------------------------
% Clemens Niederberger
% Web:    https://bitbucket.org/cgnieder/substances/
% E-Mail: contact@mychemistry.eu
% --------------------------------------------------------------------------
% Copyright 2012--2016 Clemens Niederberger
% 
% This work may be distributed and/or modified under the
% conditions of the LaTeX Project Public License, either version 1.3
% of this license or (at your option) any later version.
% The latest version of this license is in
%   http://www.latex-project.org/lppl.txt
% and version 1.3 or later is part of all distributions of LaTeX
% version 2005/12/01 or later.
% 
% This work has the LPPL maintenance status `maintained'.
% 
% The Current Maintainer of this work is Clemens Niederberger.
% --------------------------------------------------------------------------
% The substances package consists of the files
%  - substances.sty, substances-default.def, substances-examples.sub,
%    substances_en.tex, substances_en.pdf, README
% --------------------------------------------------------------------------
% If you have any ideas, questions, suggestions or bugs to report, please
% feel free to contact me.
% --------------------------------------------------------------------------
%
% if you want to compile this documentation you'll need the document class
% `cnpkgdoc' which you can get here:
%    https://bitbucket.org/cgnieder/cnpkgdoc/
% the class is licensed LPPL 1.3 or later
%
\documentclass[load-preamble+]{cnltx-doc}
\usepackage[utf8]{inputenc}
\usepackage[index]{substances}

\LoadSubstances{substances-examples}

\usepackage{imakeidx}
\makeindex[name=\jobname-chem,title=Chemicals]

\setcnltx{
  package  = {substances},
  url      = http://www.mychemistry.eu/forums/forum/substances/ ,
  authors  = Clemens Niederberger ,
  email    = contact@mychemistry.eu ,
  abstract = {%
    The \substances\ package allows you to create a database like file that
    contains data of various chemicals. These data can be retrieved in the
    document. An index creation of the chemicals used in the document is
    directly supported.
  } ,
  quote-format = \small\biolinumLF ,
  add-cmds = {
    @CAS, @EC,
    ac, AllSubstancesClist, AllSubstancesSequence, arraybackslash,
    bottomrule,
    CAS, celsius, centi, ch, chem, chemfig, chemsetup, cmc, cubic,
    DeclareSubstance, DeclareSubstanceProperty,
    EC, endinput,
    ForAllSubstancesDo,
    GetSubstanceProperty, ghs, ghspic, ghspictograms, ghsstatements, gram,
    IfSubstanceExistF, IfSubstanceExistT, IfSubstanceExistTF,
      IfSubstanceFieldF, IfSubstanceFieldT, IfSubstanceFieldTF,
      IfSubstancePropertyF, IfSubstancePropertyT, IfSubstancePropertyTF,
      iupac,
    kilo,
    Lewis, LoadSubstances,
    metre, midrule, milli, MolMass,
    NewDocumentCommand, newindex, normal, num,
    ProvideChemIUPAC,
    per, printindex,
    RetrieveSubstanceProperty,
    SI, sisetup, SubstanceIndexAltEntry, SubstanceIndexNameAltEntry,
      SubstanceIndexNameEntry, SubstancesDatabase, SubstancesStyle,
    toprule
  } ,
  index-setup = { level = \section , othercode = \small }
}

\defbibheading{bibliography}{\section{References}}

\usepackage{booktabs}

\expandafter\def\csname libertine@figurestyle\endcsname{LF}
\usepackage[libertine]{newtxmath}
\expandafter\def\csname libertine@figurestyle\endcsname{OsF}

\usepackage[biblatex]{embrac}
\ChangeEmph{[}[,.02em]{]}[.055em,-.08em]
\ChangeEmph{(}[-.01em,.04em]{)}[.04em,-.05em]

\usepackage{acro}
\acsetup{short-format=\scshape}
\DeclareAcronym{CAS}{
  short = cas ,
  long = Chemical Abstract Service
}
\DeclareAcronym{EC}{
  short = ec ,
  long  = European Commission Number
}
\DeclareAcronym{LD50}{
  short = ld50 ,
  long  = Median Lethal Dose
}
\DeclareAcronym{IUPAC}{
  short = iupac ,
  long  = International Union of Pure and Applied Chemistry
}
\DeclareAcronym{ghs}{
  short = ghs ,
  long  = Globally Harmonized System of Classification and Labelling of
  Chemicals
}

\ExplSyntaxOn
\DeclareExpandableDocumentCommand \substancesdatabase {m}
  { \__substances_database:n {#1} }
\DeclareExpandableDocumentCommand \substancesstyle {m}
  { \__substances_style:n {#1} }
\ExplSyntaxOff

\newidxcmd\substyle{\texttt{#1}}[ (\substances\ style)]

\begin{document}

\part{Preliminaries}

\section{Licence and Requirements}
\license

\substances\ loads and needs the following packages:
\pkg{expl3}~\cite{bnd:l3kernel}, \pkg{xparse}, \pkg{xtemplate} and
\pkg{l3keys2e}~\cite{bnd:l3packages}. It also needs the chemistry packages
\pkg{chemmacros}~\cite{pkg:chemmacros}, \pkg{chemfig}~\cite{pkg:chemfig} and
\pkg{ghsystem}~\cite{pkg:ghsystem}.

\section{About}
The \substances\ package allows you to create a database like file that contains
data of various chemicals. These data can be retrieved in the document. An index
creation of the chemicals used in the document is directly supported.

\part{Package Description}

\section{Options}\label{sec:options}
The \substances\ package has a few options:
\begin{options}
  \keybool{draft}\Default{false}
    If set to true all warnings will be errors.
  \keybool{final}\Default{true}
    The opposite of \option{draft}.
  \keybool{index}\Default{false}
    Add index entries when \cs{chem} is called, see section~\ref{sec:index}.
  \keyval{style}{style}\Default{default}
    Load specific style, see sections~\ref{ssec:styles}
    and~\ref{sec:default-style}.
  \keybool{strict}\Default{false}
    If set to true all warnings will be errors. This option overwrites any
    \option{draft} or \option{final} option that is passed on by the document
    class.
\end{options}

The most important option is \option{style}.  Details concerning this option
are explaind in sections~\ref{ssec:styles} and~\ref{sec:default-style}.

\section{The Database}
\subsection{Declaring the Chemicals}
The data about substances are stored via the command
\begin{commands}
  \command{DeclareSubstance}[\marg{id}\marg{list of properties}]
    This declares substance \meta{id}.
\end{commands}
An entry could look like this:
\begin{sourcecode}
  \DeclareSubstance{NaCl}{
    name    = Sodiumchloride ,
    sum     = NaCl ,
    CAS     = 7647-14-5,
    mass    = 58.44 ,
    mp      = 801 ,
    bp      = 1465 ,
    phase   = solid ,
    density = 2.17
  }
\end{sourcecode}

Such\changedversion{0.2} entries can either be declared in the document
preamble or probably more useful in a database file. Such a file can be input
in the document via
\begin{commands}
  \command{LoadSubstances}[\marg{database name}]
    Input the database \meta{database name}.  The name of a database file must
    follow the structure
    \code{\substancesdatabase{\meta{database name}}}.
\end{commands}
Suppose you have the file \code{\substancesdatabase{mydatabase}} then you
input it in the document preamble via \cs{LoadSubstances}\Marg{mydatabase}.  A
database file should typically start with the following declaration:
\begin{commands}
  \command{SubstancesDatabase}[\marg{database name}]
    Declare\sinceversion{0.2} the database \meta{database name}
\end{commands}

\subsection{Available Fields}
\subsubsection{Always Defined Fields}
Below all fields defined by \substances\ are listed\footnote{Look in the file
  \code{\substancesdatabase{substances-examples}} which is part of this
  package and should be in the same place as this documentation for example
  uses.}.
\begin{options}
  \keyval{name}{name}\Default!
    The \acs{IUPAC} name of the substance. This is the only field that
    \emph{has} to be used. The field's input is parsed with
    \pkg{chemmacros}' command \cs{iupac}.
  \keyval{sort}{sort name}
    If you plan to use the \option{index} option you should specify this field
    to get the sorting of the index right. This then creates index entries
    \cs{index}\Marg{\meta{sort field}@\meta{name field}}.
  \keyval{alt}{alt name}
    An alternative name. The field's input is parsed with \pkg{chemmacros}'
    command \cs{iupac}.
  \keyval{altsort}{sort alt name}
    This is the same as the \option{sort} field but for the alternative name.
  \keyval{CAS}{CAS number}
    The \ac{CAS} number. The input needs to be input in the form
    \code{\meta{num}-\meta{num}-\meta{num}}.
  \keyval{PubChem}{PubChem number}
    The PubChem number.
\end{options}

The \option{CAS} field processes the number using the macro
\cs{CAS}\marg{number} which is defined like this:
\begin{sourcecode}
  \def\@CAS#1-#2-#3\relax{\iupac{#1-#2-#3}}
  \NewDocumentCommand\CAS{m}{\@CAS#1\relax}
\end{sourcecode}
You're free to redefine it to your needs.

\subsubsection{Style-dependend Fields}
\substances\ defines the style \substyle{default} (see also
sections~\ref{sec:options}, \ref{ssec:styles}, and~\ref{sec:default-style})
which is loaded if no other style has been specified. It defines the following
additional fields and loads the packages \pkg{chemfig}~\cite{pkg:chemfig} and
\pkg{siunitx}~\cite{pkg:siunitx}.
\begin{options}
  \keyval{formula}{formula}
    The molecular formula of the substance. The field's input is parsed with
    \pkg{chemmacros}' command \cs{ch}.
  \keyval{structure}{structure}
    The structural formula of the substance. The field's input is parsed with
    \pkg{chemfig}'s command \cs{chemfig}.
  \keyval{mp}{melting point}
    The melting point. The field's entry is input into the \pkg{siunitx}
    command \cs{SI} in the following way: \\
    \cs{SI}\marg{field}\Marg{\cs{celsius}}.
  \keyval{bp}{boiling point}
    The boiling point. The field's entry is input into the \pkg{siunitx}
    command \cs{SI} in the following way: \\
    \cs{SI}\marg{field}\Marg{\cs{celsius}}.
  \keyval{density}{density}
    The density. The field's entry is input into the \pkg{siunitx} command
    \cs{SI} in the following way: \\
    \cs{SI}\marg{field}\Marg{\cs{gram}\cs{per}\cs{cmc}}.
  \keyval{phase}{phase}
    The state of aggregation.
  \keyval{pKa}{\pKa}
    The \pKa\ value. The field's entry is input into the \pkg{siunitx} command
    \cs{num}.
  \keyval{pKa1}{\pKa[1]}
    The first of several \pKa\ values. The field's entry is input into the
    \pkg{siunitx} command \cs{num}.
  \keyval{pKa2}{\pKa[2]}
    The second of several \pKa\ values. The field's entry is input into the
    \pkg{siunitx} command \cs{num}.
  \keyval{pKb}{\pKb}
    The \pKb\ value. The field's entry is input into the \pkg{siunitx} command
    \cs{num}.
  \keyval{pKb1}{\pKb[1]}
    The first of several \pKb\ values. The field's entry is input into the
    \pkg{siunitx} command \cs{num}.
  \keyval{pKb2}{\pKb[2]}
    The second of several \pKb\ values. The field's entry is input into the
    \pkg{siunitx} command \cs{num}.
  \keyval{pictograms}{csv list of pictograms}
    The \acs{ghs} pictograms. This field takes a list of pictogram names as
    they're input into \pkg{ghsystem}'s command
    \cs{ghspic}~\cite{pkg:ghsystem}.
  \keyval{H}{csv list of hazard statements}
    The H statements. This field takes a list of numbers as they're input into
    \pkg{ghsystem}'s command \cs{ghs}\Marg{h}\marg{number}.
  \keyval{P}{csv list of precautionary statements}
    The P statements. This field takes a list of pictogram names as they're
    input into \pkg{ghsystem}'s command \cs{ghs}\Marg{p}\marg{number}.
  \keyval{EUH}{csv list of EUH statements}
    The EUH statements. This field takes a list of pictogram names as they're
    input into \pkg{ghsystem}'s command \cs{ghs}\Marg{euh}\marg{number}.
  \keyval{LD50}{\acl{LD50}}
    The \ac{LD50} in \si{\milli\gram\per\kilo\gram}. The field's entry is
    input into the \pkg{siunitx} command \cs{SI} in the following way: \\
    \cs{SI}\marg{field}\Marg{\cs{milli}\cs{gram}\cs{per}\cs{kilo}\cs{gram}}.
\end{options}

\section{Define Custom Styles}\label{ssec:styles}
\subsection{Background}
You might have other needs for fields than the ones defined by \substances\
and the \substyle{default} style.  \emph{All fields except the required
  \option{name} field whixh are explained in this manual are defined by the
  \substyle{default} style.}

You can easily define your own style which means that you save a file with the
name \code{\substancesstyle{\meta{style}}}. In it you both define the commands
you need and you declare substance properties with the command
\cs{DeclareSubstanceProperty} (which is explained in
section~\ref{sec:declare-new-fields}) to declare your own fields.

Such a style file should start with a \cs{SubstancesStyle} declaration:
\begin{commands}
  \command{SubstancesStyle}[\sarg\marg{style name}]
    This\sinceversion{0.2} declares the style \meta{style name}. The starred
    version also switches to the expl3 programming environment.  Either way
    \code{@} has category code~11 in a style file.
  \command{LoadSubstancesStyle}[\marg{style name}]
    This\sinceversion{0.2} loads the style \meta{style name}. It can be used
    inside of a style file.  This can be useful if you want to extend the
    default style without copy-pasting every definition of the default style.
    Outside of a style file this command does nothing.
\end{commands}

The implementation of the \substyle{default} style is shown in
section~\ref{sec:default-style} as an example.

\subsection{Declare New Fields or Change Existing Fields}\label{sec:declare-new-fields}
You might want other fields or change the definition of the predefined ones.
For this there's
\begin{commands}
  \command{DeclareSubstanceProperty}[\sarg\marg{field name}\oarg{pre
    code}\oarg{post code}]
    This command declares a new property field for a substance. The star makes
    the property a required one which means an error will be issued if a
    substance is declared without it. The optional arguments \meta{pre code}
    and \meta{post code} specify any code that should be input directly before
    or after the field entry, respectively. The \meta{pre code} may end with a
    command that takes one mandatory argument. In this case the field entry
    will be its argument.
\end{commands}

The following example would define a field \option*{EC} which uses a custom
command to parse the field entry. The \ac{EC} is assigned to chemical
substances for regulatory purposes within the European Union by the regulatory
authorities.
\begin{sourcecode}
  \makeatletter
  \def\@EC#1-#2-#3\relax{#1-#2-#3}
  \newcommand*\EC[1]{\@EC#1\relax}
  \makeatother
  \DeclareSubstanceProperty{EC}[\EC]
\end{sourcecode}
For further examples of the usage of pre and post code look at the definition of
the \option{name} and the \option{mp} field:
\begin{sourcecode}
  \DeclareSubstanceProperty*{name}[\iupac]
  \DeclareSubstanceProperty{mp}[\SI][{\celsius}]
\end{sourcecode}

\section{Retrieving the Data}
There are two commands defined by \substances\ that allow the retrieving of
the data. The command \cs{chem} is intended as user command, the command
\cs{GetSubstanceProperty} can be used to define your own user command (perhaps
in your own style file, see section~\ref{ssec:styles}).
\begin{commands}
  \command{chem}[\sarg\oarg{pre}\oarg{post}\marg{id}\oarg{property}]
    If the command \cs{chem} is called without the optional \meta{property}
    argument the \option{name} entry will be called. The starred version calls
    the \option{alt} entry if it is defined and the \option{name} entry
    otherwise. The arguments \meta{pre} and \meta{post} add arbitrary input
    before or after the output, respectively.
  \command{GetSubstanceProperty}[\marg{id}\marg{property}]
    Retrieves \meta{property} for substance \meta{id}.
\end{commands}

All of the next examples use the data defined in the file
\code{substances-examples.sub} that is part of this package, see
section~\ref{sec:examples}.
\begin{example}
  \chem{H2SO4}[structure] \newline
  \chem{H2SO4} has the boiling point $\chem[T_b =]{H2SO4}[bp]$ and a
  density of $\chem[\rho =]{H2SO4}[density]$.

  Compare the melting points of methane and ethane,
  $\chem[T_m=]{methane}[mp]$ and $\chem[T_m=]{ethane}[mp]$,
  with the boiling points $\chem[T_b=]{methane}[bp]$ and
  $\chem[T_b=]{ethane}[bp]$.
 
  \chem{NaCl} has the \ac{CAS} number \chem{NaCl}[CAS].
  
  \chem{acetone} (\chem*{acetone}) is the most simple ketone:
 
  \chem{acetone}[structure]
\end{example}

The following code creates table~\ref{tab:methane}.
\begin{example}[outside]
  \begin{table}[htp]
    \centering
    \ghssetup{hide}
    \sisetup{scientific-notation=fixed,fixed-exponent=0,per-mode=symbol}
    \begin{tabular}{l>{\raggedright\arraybackslash}p{.6\linewidth}}
      \toprule
        name              & \chem{methane} \\
        formula           & \chem{methane}[formula] \\
                          & \chem{methane}[structure] \\
      \midrule
        \ac{CAS}          & \chem{methane}[CAS] \\
        PubChem           & \chem{methane}[PubChem] \\
      \midrule
        boiling point     & \chem{methane}[bp] \\
        melting point     & \chem{methane}[mp] \\
        density           & \chem{methane}[density] \\
        molar mass        & \chem{methane}[mass] \\
      \midrule
                          & \chem{methane}[pictograms] \\
        H statements      & \chem{methane}[H] \\
        P statements      & \chem{methane}[P] \\
      \bottomrule
    \end{tabular}
    \caption{\label{tab:methane}All properties of \chem{methane} that have
      been saved in the example database.}
  \end{table}
\end{example}

\section{Additional Commands}
\substances\ provides a few commands that maybe are useful in building custom
macros for styles. A field exists if it has been defined with
\cs{DeclareSubstanceProperty} regardless if it has been used or not. A
substance exists if it has been defined with \cs{DeclareSubstance}.

\begin{commands}
  \command{GetSubstanceProperty}[\marg{id}\marg{field}]
    Retrieve the property specified in \meta{field} for substance \meta{id}.
    This command is \emph{not} expandable.
  \expandable\command{RetrieveSubstanceProperty}[\marg{id}\marg{field}]
    The same as \cs{GetSubstanceProperty} but expandable.
  \expandable\command{ForAllSubstancesDo}[\marg{code}]
    Loops through all existing substances. Inside \meta{code} \code{\#1} may
    be used to refer to the \meta{id} of the current substance. This command
    is expandable.
  \expandable\command{AllSubstancesSequence}
    A sequence of all substances. This is a sequence of balanced groups each
    containing the \meta{id} of a substance. This command is expandable.
  \expandable\command{AllSubstancesClist}
    A comma separated list of all substances. Every \meta{id} is separated
    from the next with a comma. This command is expandable.
  \expandable\command{IfSubstancePropertyTF}[\marg{id}\marg{field}\marg{true
    code}\marg{false code}]
    Tests if the property \meta{field} is defined for the substance \meta{id}
    and returns either \meta{true code} or \meta{false code}. This command is
    expandable.
  \expandable\command{IfSubstancePropertyT}[\marg{id}\marg{field}\marg{true
    code}]
    Tests if the property \meta{field} is defined for the substance \meta{id}
    and returns \meta{true code} if it is. This command is expandable.
  \expandable\command{IfSubstancePropertyF}[\marg{id}\marg{field}\marg{false
    code}]
    Tests if the property \meta{field} is defined for the substance \meta{id}
    and returns \meta{false code} if it isn't. This command is expandable.
  \expandable\command{IfSubstanceFieldTF}[\marg{field}\marg{true
    code}\marg{false code}]
    Tests if the property \meta{field} exists and returns either
    \meta{true code} or \meta{false code}. This command is expandable.
  \expandable\command{IfSubstanceFieldT}[\marg{field}\marg{true code}]
    Tests if the property \meta{field} exists and returns \meta{true code} if
    it does. This command is expandable.
  \expandable\command{IfSubstanceFieldF}[\marg{field}\marg{false code}]
    Tests if the property \meta{field} exists and returns \meta{false code} if
    it doesn't. This command is expandable.
  \expandable\command{IfSubstanceExistTF}[\marg{id}\marg{true code}\marg{false
    code}]
    Tests if the substance \meta{id} exists and returns either \meta{true
      code} or \meta{false code}. This command is expandable.
  \expandable\command{IfSubstanceExistT}[\marg{id}\marg{true code}]
    Tests if the substance \meta{id} exists and returns \meta{true code} if
    it does. This command is expandable.
  \expandable\command{IfSubstanceExistF}[\marg{id}\marg{false code}]
    Tests if the substance \meta{id} exists and returns \meta{false code} if
    it doesn't. This command is expandable.
\end{commands}

\begin{example}
  Just to demonstrate how these commands can be used. And to get
  our demonstration index filled.\par
  \newcounter{substances}
  \ForAllSubstancesDo{%
    \ifnum0=\value{substances}\relax
    \else,
    \fi
    \stepcounter{substances}%
    \chem{#1}%
    \IfSubstancePropertyT{#1}{alt}{ (\chem*{#1})}%
  }
\end{example}

\section{Create an Index}\label{sec:index}
When \substances\ is called with \keyis{index}{true} the command \cs{chem}
will add index entries each time it is used. In this case the entries of the
fields \option{name}, \option{sort}, \option{alt} and \option{altsort} will be
expanded during the process. You should keep that in mind if some error
arises. It might be due to a \cs*{textbf} or similar in your database. In this
case you either need to replace it with some robust command or put a
\cs*{noexpand} in front of it.

Alternative names as specified in the \option{alt} also get an index entry
with a reference to the one of the corresponding \option{name} field. The
entry of the \option{name} field in this case gets the \option{alt} name
appended in braces.

This behaviour is not customizable for the time being. It is planned for future
versions of this package, though.

As a demonstration an index for all chemicals used in this documentation is
created with the help of the package \pkg{imakeidx}~\cite{pkg:imakeidx}.

\subsection{Formatting Commands}
The index entries are formatted with the following commands. You can redefine
them to your needs. If you do make sure they have the same number of required
arguments and are expandable!
\begin{commands}
  \expandable\command{SubstanceIndexNameEntry}[\marg{sort}\marg{name}]
    Formats the name if no \option{alt} field is given. The default definition
    is \code{\#1@\#2}.
  \expandable\command{SubstanceIndexNameAltEntry}[\marg{sort}\marg{name}\marg{alt}]
    Formats the name if the \option{alt} field is given. The default definition is
    \code{\#1@\#2 (\#3)}.
  \expandable\command{SubstanceIndexAltEntry}[\marg{alt sort}\marg{name}\marg{alt}]
    Formats the entry for the \option{alt} field.  The default definition is
    \code{\#1@\#3|see\#2}
\end{commands}

\subsection{Using makeidx}
Using the option \keyis{index}{true} with the standard way to create an index
will add the entries \cs{index}\marg{name} to the index. This means you would
mix them with other entries if you have any. Below a sample document is shown.

\begin{sourcecode}
  \documentclass{article}
  \usepackage[T1]{fontenc}
  \usepackage[index]{substances}
  \LoadSubstances{substances-examples}

  \usepackage{makeidx}
  \makeindex
  \begin{document}

  \newcounter{substances}
  \ForAllSubstancesDo{%
    \ifnum0=\value{substances}\relax
    \else, \fi
    \stepcounter{substances}\chem{#1}
  }

  \printindex
  \end{document}
\end{sourcecode}

\subsection{Using \pkg*{splitidx}}
Maybe a seperate index for the chemicals will make more sense. In this case
you could use the package \pkg{splitidx}~\cite{pkg:splitidx}. \substances\
will recognize this and create \cs{sindex}\Oarg{\cs*{jobname}-chem}\marg{name}
entries each time \cs{chem} is used.

\begin{sourcecode}
  \documentclass{article}
  \usepackage[T1]{fontenc}
  \usepackage[index]{substances}
  \LoadSubstances{substances-examples}
  
  \usepackage{splitidx}
  \makeindex
  \newindex[Chemicals]{\jobname-chem}
  \begin{document}
  
  \newcounter{substances}
  \ForAllSubstancesDo{%
    \ifnum0=\value{substances}\relax
    \else, \fi
    \stepcounter{substances}\chem{#1}
  }

  \printindex[\jobname-chem]
  \end{document}
\end{sourcecode}

\subsection{Using \pkg*{imakeidx}}
Another way to create multiple indexes is the package
\pkg{imakeidx}~\cite{pkg:imakeidx}. \substances\ recognizes its usage and
creates index entries \cs{index}\Oarg{\cs*{jobname}-chem}\marg{name}.

\begin{sourcecode}
  \documentclass{article}
  \usepackage[T1]{fontenc}
  \usepackage[index]{substances}
  \LoadSubstances{substances-examples}

  \usepackage{imakeidx}
  \makeindex[name=\jobname-chem,title=Chemicals]
  \begin{document}
  
  \newcounter{substances}
  \ForAllSubstancesDo{%
    \ifnum0=\value{substances}\relax
    \else, \fi
    \stepcounter{substances}\chem{#1}
  }

  \printindex[\jobname-chem]
  \end{document}
\end{sourcecode}

\part{Appendix}
\appendix

\section{The Default Style}\label{sec:default-style}
The following code shows the contents of the file
\code{\substancesstyle{default}} which defines the \substyle{default} style
which is part of this package.

\implementation[basicstyle=\ttfamily\footnotesize]{\substancesstyle{default}}

\section{The Example Database}\label{sec:examples}
The following code shows the example database
\code{\substancesdatabase{substances-examples}} that is part of this package.

\implementation[basicstyle=\ttfamily\footnotesize]{\substancesdatabase{substances-examples}}

\printindex[\jobname-chem]

\end{document}
