\section{The package sets}\label{sets}

As described in the introduction, sets was designed to support set operations. The elements of a set are normally simple Text, but you can insert commands in a set, too.
These commands will---except when printing the set---not expanded.
The usage of braces (`\{' and `\}') unfortunately doesn't work. In this case you have to define a shortcut that doesn't need braces. However, Parameters without braces work.
``\verb|H"agar|'' is therefore a valid element of a set. ``\verb|\endset|''  and ``\verb|\empty|'' can not be part of a set.

As a document has only a few authors, not much effort was put in improving efficiency.
Sets should therefore be relatively small. If you nevertheless try to create a set with hundreds or thousands of items, \TeX's stack might overflow.

In most cases the sequence of items doesn't matter. This is also the case in most commands here. Exceptions will be marked.

The package sets needs \LaTeXe.

\subsection{Usage}\label{sets-usage}

In this subsection the usage of the package sets is described. In the description, some example sets will be used:
\begin{eqnarray*}
	A &=& \{Alice, Bob, Charly\}\\
	B &=& \{Alice, Bob\}\\
	C &=& \{Bob, Dean\}\\
	D &=& \{Dean\}\\
	L &=& \emptyset
\end{eqnarray*}

%-----------------------------------------------------------------------------
\subsubsection{Constructors}\label{constructors}
To create a set, the commands\\
\mbox{}\hspace{2em}\verb$\newset{<set>}{<content>}$\\
and\\
\mbox{}\hspace{2em}\verb$\newsetsimple{<set>}{<content>}$\\
can be used. \texttt{<set>} is a command name to access the set from now on.
The items of a set are separated with \texttt{|}. The set $A$ could therefore be defined with:\\
\mbox{}\hspace{2em}\verb$\newset{\sA}{Alice|Bob|Charly}$\\
The set $L$ is defined with:\\
\mbox{}\hspace{2em}\verb$\newset{\sL}{}$\\

\verb$\newset$ creates a new set. This set will be sorted in alphabetical order and duplicates will be removed. So it would be no matter if in the definition of $A$ after ``Charly'' a second ``Alice'' was inserted.

The effort for sorting and duplicate deleting is unnecessary at this point. If you want to skip these expensive steps, you can create a set with the command
\verb$\newsetsimple$, too.

Because they are needed later, we will create all sets mentioned above:\\
\mbox{}\hspace{2em}\verb$\newsetsimple{\sA}{Alice|Bob|Charly}$ \newsetsimple{\sA}{Alice|Bob|Charly}\\
\mbox{}\hspace{2em}\verb$\newsetsimple{\sD}{Alice|Bob}$ \newsetsimple{\sB}{Alice|Bob}\\
\mbox{}\hspace{2em}\verb$\newsetsimple{\sC}{Bob|Dean}$ \newsetsimple{\sC}{Bob|Dean}\\
\mbox{}\hspace{2em}\verb$\newsetsimple{\sD}{Dean}$ \newsetsimple{\sD}{Dean}\\
\mbox{}\hspace{2em}\verb$\newsetsimple{\sL}{}$ \newsetsimple{\sL}{}
%-----------------------------------------------------------------------------
\subsubsection{Inspectors}\label{inspectors}
Inspectores help you to retrieve informations about sets and to print sets.

\paragraph{Printing:}A set can be printed using the command\\
\mbox{}\hspace{2em}\verb$\listset$.\\
The elements will be put in the sequence they are in the set. A comma is used as separator.\\
\mbox{}\hspace{2em}\verb$\listset{\sA}$ therefore leads to the following output:\\
\centerline{\listset{\sA}}

Sometimes you might want to separate the items in a different way, for example with a \texttt{\&} to put them in a table. In this case a (temporary) redefinition of\\
\mbox{}\hspace{2em}\verb$\setseparator$\\
helps you. Normally this command expands to `\verb*$,\ $'.

\paragraph{Determining the size of a set:} The next inspector has the syntax\\
\mbox{}\hspace{2em}\verb$\sizeofset{$$S$\verb$}\is{<counter>}$,\\
where \texttt{<counter>} is the name of a \LaTeX\ counter, which afterwards will contain the number of elements in set $S$. The sequence\\
\mbox{}\hspace{2em}\verb$\newcounter{mycounter}$\newcounter{mycounter}\\
\mbox{}\hspace{2em}\verb$\sizeofset{\sB}\is{mycounter}$\sizeofset{\sB}\is{mycounter}\\
\mbox{}\hspace{2em}\verb$\arabic{mycounter}$\\
leads to the output: ``\arabic{mycounter}''
If you determine the size of set $L$, the result\sizeofset{\sL}\is{mycounter} (as you might have expected) is ``\arabic{mycounter}''.

\paragraph{Testing for membership:}By using the command\\
\mbox{}\hspace{2em}\verb$\iselementofset{$$e$\verb$}{$$S$\verb$}$\\
you can check, whether $e \in S$ is true. The effort is $O(1)$, because all work is done by the pattern matching of \TeX. The sequence\\
\mbox{}\hspace{2em}\verb$\if \iselementofset{Bob}{\sC}Yes\else No\fi$\\
would result in the output ``\if \iselementofset{Bob}{\sC}Yes\else No\fi'', the same test with set $D$ in ``\if \iselementofset{Bob}{\sD}Yes\else No\fi''.

%-----------------------------------------------------------------------------
\subsubsection{Modificators}\label{modificators}

\paragraph{Union of sets:}The operation $R := S_1 \cup S_2$ is realized in the command\\
\mbox{}\hspace{2em}\verb|\unionsets{|$S_1$\verb|}{|$S_2$\verb|}\to{|$R$\verb|}|.\\
Table \ref{tab:ops} contains some examples. The result of the operation is a sorted set without duplicates containing the items of sets $S_1$ and $S_2$.

\paragraph{Difference of sets:}The operation $R := S_1 - S_2$ (also written as $R := S_1 \backslash S_2$) can be carried out with\\
\mbox{}\hspace{2em}\verb|\minussets{|$S_1$\verb|}\minus{|$S_2$\verb|}\to{|$R$\verb|}|.\\
If $S_1$ is a sorted set, $R$ will be sorted, too. If $S_1$ contains duplicates, $R$ might also contains these duplicate elements. Table \ref{tab:ops} contains several examples for the usage of this command.

Colloquially you can formulate the operation as follows: Check for every element $e$ in $S_1$ if $e \in S_2$ is true. If not, insert $e$ into $R$. And that's exactly the way it has been implemented!

\paragraph{Intersection of sets:} The operation $R := S_1 \cap S_2$ is made possible with the command\\
\mbox{}\hspace{2em}\verb|\intersectsets{|$S_1$\verb|}{|$S_2$\verb|}\to{|$R$\verb|}|.\\
As above: If $S_1$ is a sorted set, $R$ will be sorted, too. If $S_1$ contains duplicates, $R$ might also contains these duplicate elements. Table \ref{tab:ops} contains several examples for the usage of this command, too.

This operation can colloquially be written down as: Check for every element $e$ in $S_1$ if $e \in S_2$ is true. If yes, insert $e$ into $R$. If you compare this with the formulation above, one can recognize that the only difference is the small word ``yes''. 
In the source code, this expresses in a missing \verb|\else|. Acutally amazingly simple if you remember the formal relation $S_1 \cap S_2 \equiv S_1 \backslash (S_1\backslash S_2)$, which lets one expect a much higher complexity.

\begin{table}%[htb]
\begin{center}
\begin{tabular}{ll}\hline
\textbf{Operation} & \textbf{Result} \\ \hline 
\unionsets{\sA}{\sC}\to{\sR}\global\let\sR\sR
	\verb$\unionsets{\sA}{\sC}\to{\sR}$ & ``\listset{\sR}'' \\ 
\unionsets{\sB}{\sD}\to{\sR}\global\let\sR\sR
	\verb$\unionsets{\sB}{\sD}\to{\sR}$ & ``\listset{\sR}'' \\
\unionsets{\sL}{\sC}\to{\sR}\global\let\sR\sR
	\verb$\unionsets{\sL}{\sC}\to{\sR}$ & ``\listset{\sR}'' \\
\unionsets{\sL}{\sL}\to{\sR}\global\let\sR\sR
	\verb$\unionsets{\sL}{\sL}\to{\sR}$ & ``\listset{\sR}'' \\\hline
%
\minussets{\sA}\minus{\sC}\to{\sR}\global\let\sR\sR
	\verb$\minussets{\sA}\minus{\sC}\to{\sR}$ & ``\listset{\sR}'' \\ 
\minussets{\sD}\minus{\sC}\to{\sR}\global\let\sR\sR
	\verb$\minussets{\sD}\minus{\sC}\to{\sR}$ & ``\listset{\sR}'' \\
\minussets{\sD}\minus{\sB}\to{\sR}\global\let\sR\sR
	\verb$\minussets{\sD}\minus{\sB}\to{\sR}$ & ``\listset{\sR}'' \\
\minussets{\sA}\minus{\sL}\to{\sR}\global\let\sR\sR
	\verb$\minussets{\sA}\minus{\sL}\to{\sR}$ & ``\listset{\sR}'' \\\hline
%
\intersectsets{\sA}{\sB}\to{\sR}\global\let\sR\sR
	\verb$\intersectsets{\sA}{\sB}\to{\sR}$ & ``\listset{\sR}'' \\ 
\intersectsets{\sC}{\sB}\to{\sR}\global\let\sR\sR
	\verb$\intersectsets{\sC}{\sB}\to{\sR}$ & ``\listset{\sR}'' \\
\intersectsets{\sB}{\sD}\to{\sR}\global\let\sR\sR
	\verb$\intersectsets{\sB}{\sD}\to{\sR}$ & ``\listset{\sR}'' \\
\intersectsets{\sA}{\sL}\to{\sR}\global\let\sR\sR
	\verb$\intersectsets{\sA}{\sL}\to{\sR}$ & ``\listset{\sR}'' \\\hline
\end{tabular}
\caption{Set operations, examples}
\label{tab:ops}
\end{center}
\end{table}

\paragraph{Sorting:} A set $S$ can be sorted alphabetically by using the command\\
\mbox{}\hspace{2em}\verb|\sortset{|$S$\verb|}{|$R$\verb|}|.\\
After execution of the command, $R$ contains the sorted set. The sorting is done by the bubblesort algorithm, an algorithm, that can be implemented in \TeX\ without having to perform many contortions.

At sorting, the elements are compared as they are, i.\,e. possibly contained macros are not expanded but compared by their name (including the backslash).

\paragraph{Removing duplicates:} The operations \emph{only} works on sorted sets! You actually will not need it very often, because the creation of a set using \verb|\newset| does the work automatically (by using this macro). However, I decided to make this macro available to public; probably sometimes someone really needs it.
Duplicate removal is called with:\\
\mbox{}\hspace{2em}\verb|\deleteduplicates{|$S$\verb|}{|$R$\verb|}|,\\
with $R$ being the result set and $S$ being the sorted set, whose duplicates shall be removed.

%------------------------------------------------------------------------------
\subsection{Effort estimations}
Table \ref{tab:complexity} lists the complexity of the operations in O-notation. The following Assumptions are made:
\begin{itemize}
  \item Let the length of an element of a set be $m$.
  \item Let the number of elements in a set be $n$. If an operation need two sets, $n_1$ is the cardinality of the first set and $n_2$ the cardinality of the second one.
  \item For simplicity reasons the effort for pattern matching in parameter processing is assumed to be constant.
\end{itemize}

The given complexity classes can help you to arrange a set of operations in the optimal sequence. For example when using the commands \verb@\intersectsets@ or \verb@\minussets@ it is better to let the smaller set be the first parameter.

\begin{table}[htb]
  \begin{center}
  \begin{tabular}{lc}\hline
    \textbf{Operation} & \textbf{Operation}\\ \hline
    Compare elements & $m$\\
    \verb@\sizeofset@ & $n$\\
    \verb@\listset@ & $n$\\
    \verb@\iselementofset@ & $1$\\%\hline
    \verb@\sortset@ & $m \cdot n^2$\\
    \verb@\deleteduplicates@ & $n$\\%\hline
    \verb@\newset@ & $m \cdot n^2$\\
    \verb@\newsetsimple@ & $1$\\%\hline
    \verb@\unionsets@ & $m \cdot (n_1 + n_2)^2$\\
    \verb@\intersectsets@ & $n_1$\\
    \verb@\minussets@ & $n_1$\\ \hline
  \end{tabular}
  \caption{Complexity classes of set operations}%
  \label{tab:complexity}%
  \end{center}
\end{table}
