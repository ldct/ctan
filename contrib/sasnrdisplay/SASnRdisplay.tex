% $LastChangedDate: 2012-03-27 12:21:37 +0200 (Tue, 27 Mar 2012) $
% $LastChangedRevision: 1479 $
% $LastChangedBy: daleif $


\documentclass[a4paper,article]{memoir}
\setlrmarginsandblock{3cm}{3cm}{*}
\setulmarginsandblock{2.5cm}{3cm}{*}
\checkandfixthelayout[nearest]
\usepackage[T1]{fontenc}
\usepackage{kpfonts}
\usepackage{xspace,calc,listings,enumitem,url,ragged2e}
\setlist{  
  listparindent=\parindent,
  parsep=0pt,
}
\usepackage[scaled=0.8]{beramono}

\hfuzz=5pt

%\setlength\cftbeforechapterskip{0em plus 0.2em}

\AtBeginDocument{\providecommand\href[2]{#2}}

\DeclareUrlCommand\mypath{\urlstyle{sf}}
%\urlstyle{rm}
\newcommand\addCTAN[2][\textsf{CTAN}:]{%
    {\footnotesize\href{http://mirror.ctan.org/#2}{#1\:{\mypath{#2}}}}% 
}

\newcommand\CTAN[1]{%
  {\footnotesize\href{http://mirror.ctan.org/#1}{\textsf{CTAN:}~{\mypath{#1}}}}}


\newcommand\pkg{\texttt{SASnRdisplay}\xspace}
%\setsecnumdepth{none}
\setsecheadstyle{\large\slshape\bfseries\memRTLraggedright}

\usepackage[svgnames,dvipsnames]{xcolor}
 \definecolor{felinesrcbgcolor}{rgb}{0.94,0.97,1}
 \definecolor{felineframe}{rgb}{0.79,0.88,1}


\usepackage[
english,
%grayscale,
%noautotitles-sas,
noneedspace
]{SASnRdisplay}

\captionsetup[lstlisting]{
  font=small,
  labelfont=bf
}  



\newcommand\R{\textsf{R}\xspace}
\newcommand\SAS{\textsf{SAS}\xspace}
\newcommand\sweave{\textsf{Sweave}\xspace}

\usepackage{xkeyval}


\setlength\unitlength{\linewidth+1em}
\lstdefinestyle{Show}{
  basicstyle=\ttfamily\small,
  breaklines,
  breakatwhitespace,
  columns=flexible,
  backgroundcolor=\color{BrickRed!10},
  rulecolor=\color{BrickRed},
  frame=single,
  framesep=0.25em,
  escapeinside=||,
}
\lstdefinestyle{inShow}{
  style=Show,
  basicstyle=\ttfamily\normalsize\color{BrickRed},
}
\lstnewenvironment{Show}[1][]{
\lstset{
  style=Show,#1
}
}{}
%\setsecnumdepth{none}
\newcommand\error[1]{\textcolor{OliveGreen}{>>#1<<}}

\def\NAME{SASnRdisplay.sty}
\newcommand\getsniplet[2]{
  \begingroup
  \Needspace{4\onelineskip}
  \ifblank{#2}{
    \lstinputlisting[style=Show,
    rangeprefix=\%\ ,
    includerangemarker=false,
    linerange=#1-end\ #1,
    columns=fixed,
    xleftmargin=0.5em,
    framexleftmargin=0.5em,
    ]{\NAME}
  }{
    \lstinputlisting[style=Show,
    rangeprefix=\%\ ,
    includerangemarker=false,
    linerange=#1-end\ #1,
    title={Style: #2},
    columns=fixed,
    xleftmargin=0.5em,
    framexleftmargin=0.5em,
    ]{\NAME}}
  \endgroup
  \noindent\ignorespaces
}

\newcommand\showcolor[1]{>>{\color{#1}\rule{1em}{0.5em}}<<}

\newenvironment{question}[1]{
  \subsubsection*{\textnormal{\textit{\color{BrickRed}#1}}}
  \phantomsection
  \addcontentsline{toc}{subsection}{#1}
}{}
\lstnewenvironment{solution}[1][]{
  \lstset{style=Show,gobble=2,columns=fixed,#1}
}{}


\usepackage{hyperref}

\settocdepth{subsection}

\pagestyle{plain}
\begin{document}

\title{The \textsf{SASnRdisplay} package}
\author{Lars Madsen\thanks{Email: daleif@imf.au.dk,\SnRversion}}
\maketitle

\chapter*{Introdution}
\label{cha:introdution}


The \pkg package acts as a frontend to the versatile \texttt{listings}
package in order assist the user in typesetting \SAS or
\R code or output. The package replaces the similar
\texttt{SASdisplay} package, which was only available to my local
users.

\fancybreak{}

Please be aware that \pkg is \emph{not} fully compatible with
\texttt{SASdisplay}, the default settings are different and some
macros are named differently.

\subsubsection*{Acknowledgements}
\label{sec:acknowledgements}

I'd like to thank Ulrike Fischer for hints about some very useful
features in the \texttt{caption} package, Heiko Oberdiek for his help
with \texttt{hyperref} details.  Plus a thanks to J�rgen Granfeldt and
Preben Bl�sild for answering my various questions about details within
the \R or \SAS languages.

\subsubsection*{Loaded package}
\label{sec:loaded-package}

The following packages will be loaded (without any options): \texttt{listings},
\texttt{xkeyval}, \texttt{xcolor}, \texttt{etoolbox},
\texttt{caption}, \texttt{needspace}. If you need to pass options to
these packages, load them \emph{before} the \pkg package.



\tableofcontents*

\newpage

\chapter{Interface}
\label{cha:interface}

\section{Environments}
\label{sec:environments}

\begin{solution}[escapeinside=XX]
  \begin{env}[X\meta{options}X]
    ...
  \end{env}
\end{solution}
%
Each environment support an optional \oarg{options} argument. The
\meta{options} part should be listings related configuration.

\renewcommand\descriptionlabel[1]{\hspace\labelsep\parbox{\textwidth-\labelsep}{\textbf{#1}}}
\begin{description}\firmlist
\item[SAScode, SAScode*] 
  For typesetting \SAS code. The starred version is not automatically 
  numbered. 
\item[SASoutput, SASoutput*] 
  Similar for \SAS output. Note that it might be an idea to decrease the
  width of the \SAS output from within the \SAS programme.
\item[Rcode, Rcode*] 
  Typesets \R code
\item[Routput, Routput*] 
  Typesets \R output
\end{description}

\section{Input from external files}
\label{sec:input-from-external}

General syntax:
\begin{quote}
  \cs{macro}\oarg{options}\marg{filename}
\end{quote}
Available macros:
\begin{description}\firmlist
\item[\cs{inputSAScode}, \cs{inputSAScode*}] 
Similar to the SAScode(*) environment, but get the data from external file.
\item[\cs{inputSASoutput}, \cs{inputSASoutput*}] 
Similar to the SASoutput(*) environment.
\item[\cs{inputRcode}, \cs{inputRcode*}] 
Similar to the Rcode(*) environment.
\item[\cs{inputRoutput}, \cs{inputRoutput*}] 
Similar to the Rcode(*) environment.
\end{description}

\section{Sniplets}
\label{sec:sniplets}

To typeset inline sniplets we provide
\begin{description}\tightlist
\item[\cs{SASinline}]
\item[\cs{Rinline}] 
\end{description}
They both behave like \cs{verb}, thus one can write
\begin{quote}
  \cs{Rinline}\oarg{options}\verb+|x <- 34|+
\end{quote}
%
Note that for \verb|\SASinline|, key words are marked, i.e.\ it is
\SAS aware. This is not the case for \R.



\section{Package options}
\label{sec:package-options}

\begin{solution}[escapeinside=XX]
  \usepackage[X\meta{options}X]{SASnRdisplay}
\end{solution}
Available options:
\begin{description}\firmlist
\item[danish]
  Loads Danish translations for some keywords. (Executed by default)
\item[english] Similar for English.
% \item[babel] Adding this option and we will bind the four \cs{...name}
%   macros to their respective languages, i.e. automatically use
%   \cs{addto}\cs{captionsX}. This will require that \pkg is loaded
%   \emph{after} \texttt{babel}.
\item[grayscale] Changes some build in colors to monochrome.
\item[countbysection] Force counters to be dominated by the section counter.
\item[countbychapter] Force counters to be dominated by the chapter
  counter. This is the default if \cs{chapter} exist, otherwise
  \texttt{countbysection} will be used.
\item[consecutive] Use this if you just want consecutive
  numbering throughout, that is the number of say \SAS code, is not
  reset at every new chapter or section.
\item[noautotitles-r, noautotitles-sas] Do not automatically add a
  number to any of the code and outputs. It can still be added
  manually by using the \texttt{caption=\marg{text}} option.
\item[needspace=\meta{length}] 
  This issue a \cs{needspace}\marg{length} before each code or output
  environmnt or inclusion macro. It will ensure that if there is less
  than \meta{length} space left on the page, a page break is issued
  before the construction start. 

  This feature is enabled by default with a default \meta{length} of
  \verb|3\baselineskip|.
\item[noneedspace] This disables the \texttt{needspace} feature.
\item[sweave] Overloads the \sweave package, i.e.\ it makes
  the \texttt{Sinput} and \texttt{Scode} environments behave as the
  \texttt{Rcode} environment , and the \texttt{Soutput} like the
  \texttt{Routput} environment.

  If \texttt{Sinput} should have a slightly different look than
  \texttt{Rcode}, then use the \texttt{Sinput} style to add your extra
  configuration. (The \sweave package typesets the contents of the
  \texttt{Sinput} env to be in the typewriter/monospace font plus
  italic, whereas we just set it in typewriter/monospace.)
  
\item[sasweave] Similar for the sasweave package, adding this options,
  we will overwrite \texttt{SASinput}, \texttt{SASoutput} and
  \texttt{SAScode} environments with our versions. Note you will have
  to load \pkg \emph{after} the \texttt{SasWeave}
  package.\footnote{Because \texttt{SasWeave} use the same environment
    names as we do.}

  Please note: This option has \emph{not} been throughly
  tested. Please let me know if it works as advertised.  

\item[\meta{other}] Other options will be passed on to the
  \texttt{listings} package.
\end{description}


\chapter{Configuration}
\label{cha:configuration}

As a frontend to listings, the configuration is based upon listings
\emph{styles}, i.e.\ collections of listings configurations. These are
applied in left to right fashion, the last configuration loaded takes
precedence.


\section{Titles}
\label{sec:macros}

These macros holds the titles for the four types of displays. English:
\getsniplet{englishnames}{}
And for Danish:
\getsniplet{danishnames}{}


\fancybreak{}

Note if you are using \texttt{babel} and the \texttt{babel} option to
\pkg, then these names are added to the language setup, and thus if
you want to change then you will have to add your changes to the
language setup as well.

\section{Handling listings configuration of our environments and macros}
\label{sec:handl-list-conf}

The listings configuration we use in this package is based on the
listings concept of \emph{styles}. A style is basically just giving a
collection of listings key-value sets a more convenient name. At first
glance, this may seem a tedious method for configuration instead of
giving various features macro names and letting the user change those
macros. Been there, done that. Given the shear number of
\texttt{listings} options, this would make configuration very un-flexible.


The general listings syntax for styles is
\begin{Show}
  \lstdefinestyle|\marg{name}|{
    |\meta{key-value set}|
  }
\end{Show}
One drawback is that if \meta{name} already exist, then you will
replace the contents of this style. Currently there is \emph{no}
manner in which to \emph{add} to a style.  

Thus it is a rather bad idea to provide the configuration as one long
style, because then changing a small thing, would require the user to
retype the rest of the configuration. Instead we split the
configuration into smaller themed pieces. The user then have a choice
of either overwriting one of these pieces or override a special user
style which is executed as the last style (and thus overwrites any
former style).

The settings can be seen in section~\ref{cha:defa-list-style}. The
styles are broken into smaller pieces. In some cases it make sense to
change one of these smaller pieces, in other cases it is easier to
add stuff to the provided \texttt{\meta{name}-\meta{type}-user}
styles. 



\section{Configuration FAQ/Examples}
\label{sec:configuration-faq}

Here follows a list of FAQs as to how one would make some
configuration changes. Again we remind the user, that it is not
possible to add to a \texttt{listings} style. Thus you will have to
add all your setting for, say, \texttt{r-user}, into one single call
to \cs{lstdefinestyle}.

\subsection*{Colors}
\label{sec:colors}

The default colors are \texttt{SnRFrame} \showcolor{SnRFrame} for the
frame, and \texttt{SnRBG} \showcolor{SnRBG} for the background. If the
\texttt{grayscale} option is used, the mentioned colored are mapped on
to \texttt{SnRFrameGray} \showcolor{SnRFrameGray} and \texttt{SnRBGGray}
\showcolor{SnRBGGray}.

\begin{question}{I'd like other colors}
  We automatically load the \texttt{xcolor} package for colors, so we
  refer that package for details. If you just want to change, say, the
  background color, try (\texttt{after} \pkg)
  \begin{solution}
    \definecolor{SnRBG}{gray}{0.8}
  \end{solution}
  for a gray tone (1 means white). If one loads the \texttt{xcolor}
  package \emph{before} \pkg, then one can pass certain options to it
  and get access to a lot of color names, a survey of these can be
  found in the \texttt{xcolor} manual, \cite{xcolor}.

  The two default colores are defined as
  \begin{solution}
    \definecolor{SnRBG}   {rgb}{0.94,0.97,1}
    \definecolor{SnRFrame}{rgb}{0.79,0.88,1}
  \end{solution}
\end{question}

\begin{question}{I'd like different colors for code and output}
  Since the \texttt{\meta{name}-\meta{type}-user} styles are executed
  at the very end of the configuration, it will be suitable to add
  them there. Here is how to make the \SAS code have a blue background
  while leaving the \SAS output with the default.
  \begin{solution}
    \lstdefinestyle{sas-code-user}{
      backgroundcolor = \color{blue},
    }
  \end{solution}
  Note: Remember that there will only be one \texttt{sas-code-user},
  thus if you have several configurations to add to it, collect them
  in one such \cs{lstdefinestyle}.
\end{question}

\begin{question}{How about the color of the text}
  This can be seen in three different ways: regular text, comments and
  keyword. (In our case keywords only apply to \SAS code.) This means
  that colors will have to be inserted into say the
  \texttt{basicstyle} key to change the basic font color. Here is
  instead how to make all \SAS comments green, note that we have to
  copy the rest of the font settings for \SAS comments, as we cannot
  add to a setting.
  \begin{solution}
    \lstdefinestyle{sas-code-user}{
      commentstyle = \normalfont\slshape\ttfamily\footnotesize\color{green},
    }
  \end{solution}
  When dealing with \SAS keywords one can even add different colors to separate
  groups of keywords, though this is a bit out of our scope in this manual.
\end{question}



\subsection*{Frames}
\label{sec:frames}

Listings supports a number of different types of frames, see the
manual (\cite{listings}) for details.

\begin{question}{I like the settings from the old \protect\texttt{SASdisplay}
    package with the line above and below}
  Here we choose to simply overwrite a \texttt{frame} style.
  \begin{solution}
    \lstdefinestyle{r-frame}{
      frame     = lines,
      framesep  = 0.5em,
      framerule = 1mm,    % thickness of the rule
    }
  \end{solution}
\end{question}

\begin{question}{No rules}
  \begin{solution}
    \lstdefinestyle{r-frame}{}
  \end{solution}
\end{question}

\fancybreak{}

\begingroup\sloppy
A user may want to experiment with the keys
\texttt{x\meta{left|right}margin} and
\texttt{framex\meta{left|right|\allowbreak top|\allowbreak bottom}margin}.

\endgroup

\subsection*{Captions}
\label{sec:captions-1}

This is not a configuration as such but rather a hint to how one adds
a caption.

If the \texttt{noautotitles} is not activated, all non-starred
environments and input macros will get an automatic caption, including
a number. If one wish to add extra text use the following to the
options of the environment or input macro.
\begin{solution}[gobble=0]
  caption=|\marg{My text}|
\end{solution}
Remember the \texttt{\{\}} pair around the text.

\fancybreak{}

In this version of \pkg `list of \dots' are not supported due to
technical difficulties.

\fancybreak{}

If you want to configure captions related to \texttt{listings}, please
use
\begin{solution}[gobble=0]
  \captionsetup[lstlisting]{|\meta{options}|}
\end{solution}
For example, in this document we use
\begin{solution}[gobble=0]
  \captionsetup[lstlisting]{
    font=small,
    labelfont=bf
  }  
\end{solution}
to make the label text bold, and the entire caption text in \cs{small}.



Note that numbered construction without a caption are typeset as
\verb|Name Num|, with a caption this change into \verb|Name Num: Caption|.

\begin{question}{I do not like numbers, but I'd like to add some
    titling info for some of my code.}
  If you do not want to use the auto numbering scheme, then use the
  \texttt{noautotitles-sas} or \texttt{noautotitles-r} package
  options.  Then to add just a title, add the following to the
  \meta{options}
  \begin{solution}
    title={My title text}
  \end{solution}
  It is simular to the \texttt{caption} option, but has no numbers or
  preceeding text. 
\end{question}

\begin{question}{How do I refer to code or output?}
  First of all, as with floats, it is the caption that provide the
  number that one can refer to. So as long as the code or output is
  numbered, then one can just add
  \begin{Show}
    label=|\meta{keyname}|
  \end{Show}
  to the environment or inclusion macro \meta{options}. 

  You can of course also add a label even if it is not numbered, then
  \cs{ref}\marg{key} will just not be weldefined. But
  \cs{pageref}\marg{key} will!
\end{question}


\subsection*{Keywords}
\label{sec:keywords}

\begin{question}{In a presentaion, I'd like to highlight a word}
  See the \texttt{emph} and \texttt{emphstyle} keys. Here is an
  example.
  \begin{solution}
    \begin{SAScode*}[emph={INSIGHT},emphstyle=\color{red}\bfseries]
      PROC INSIGHT DATA data=fisk;
    \end{SAScode*}
  \end{solution}
  resulting in
  \begin{SAScode*}[emph={INSIGHT},emphstyle=\color{red}\bfseries,gobble=2]
    PROC INSIGHT DATA data=fisk;
  \end{SAScode*}
\end{question}

\begin{question}{How do I disable the keyword marking?}
  You could either specify en empty language, i.e.\ \texttt{language=}
  to eithor the \meta{options} or to a global style. 

  Or you could redefine the keyword style:
  \begin{solution}
    keywordstyle=
  \end{solution}
\end{question}


\begin{question}{I typeset SAS code, but keywords are not being marked!?}
  This is usually because the mono space font (i.e.\ the font behind
  \cs{ttfamily}) does not support boldface (as that is the default
  manner which we mark keywords).

  One such example is the default \LaTeX\ font: Computer Modern. Its
  mono space has no bold version.

  Solutions: see
  \url{http://www.tug.dk/FontCatalogue/typewriterfonts.html}, you will
  need to look for fonts that are shown to support
  \cs{bfseries}. 
  
  In this manual we use \texttt{beramono}. Another interesting
  solution is to use
  \begin{solution}
    \renewcommand\ttdefault{txtt}
  \end{solution}
\end{question}


\subsection*{Escape to \LaTeX}
\label{sec:escape-latex}

This is a very handy feature and can e.g.\ be used to get formatted
\LaTeX\ code inside, say, a comment. For example using
\begin{Show}[escapeinside={}]
  escapeinside=||,
\end{Show}
means that if one write \verb+|$a_{ij}$|+ in a comment one would get
$a_{ij}$ typeset in the output.

A feature like this is \emph{not} enabled by default. Though a user
can always add it globally to the settings of his/her document, say using
\begin{Show}[escapeinside=]
  \lstdefinestyle{sas-code-user}{
    escapeinside=||,
  }
\end{Show}
It does not have to be >>\texttt{|}<< that is the escape character.

It can be locally disabled by adding 
\begin{Show}
  escapeinside={},
\end{Show}
in the \meta{options} for the environment or the input macro.

\fancybreak{}

The \texttt{listings} manual, \cite[section~4.14]{listings} list other
features related to escaping back to normal \LaTeX\ formatting.


\subsection*{Input encodings}
\label{sec:input-encodings}


\begin{question}{I keep getting errors when I include my program,
    something about undefined chars!?}

  There are two things at play here: firstly \texttt{listings} does
  not support two-byte UTF8 (\texttt{listings} need to do a lot of
  parsing, that and parsing may break when dealing with two-byte
  UTF8), secondly one may have a situation where the main document
  encoding is different from the encoding being used in code filers
  being included.

  Let us try and handle the later first. We assume you are using the
  \texttt{inputenc} package. Then one can specify that, say, all code
  should be interpreted with, say, the \texttt{latin1} encoding. Here
  is how for included \SAS code:
  \begin{solution}
    \lstdefinestyle{sas-include-code-user}{
      inputencoding=latin1
    }
  \end{solution}

  \fancybreak{}

  The other situation, i.e.\ the included\footnote{The solution is
    only available for included files.} file is written in UTF8. Here
  a solution exist if the two-byte UTF8 file, can safely be
  transformed into a one-byte encoding such as \texttt{latin1}. And
  example of this could be the situation with an \R file, written in
  UTF8, including the Danish vowels ���, but other that those only US
  ascii is being used. A file like this can easily be recoded as a
  \texttt{latin1} based file.

  There is a package (\texttt{listingsutf8}) can than handle this. It
  extends the file inclusion feature and extend the input encoding
  syntax. In the case described above, adding \texttt{listingsutf8} to
  your preamble, and using 
  \begin{solution}
    \lstdefinestyle{r-include-code-user}{
      inputencoding=utf8/latin1
    }
  \end{solution}
  may solve the problem. See \cite{listingsutf8} for a bit more details. 
\end{question}


\subsection*{Other}
\label{sec:other}

\begin{question}{I'd like to have line numbers}
  Here is how to add line numbers to all \R code.
  \begin{solution}
    \lstdefinestyle{r-code-user}{
      numbers     = left,
      numberstyle = \tiny
    }
  \end{solution}
  Line numbers can be configured further, see section 4.8 in the
  \texttt{listings} manual, \cite{listings}.

  Line numbers can be very handy when displaying source code. For
  output, it might not be that relevant.

  It \emph{is} possible to actually label and refer to specific lines
  in a piece of code, see section 7 in the \texttt{listings} manual,
  \cite{listings}.
\end{question}




\begin{question}{I have many blank lines, can some be ignored?}
  Yes with the \texttt{emptylines} key. It determine the number of
  consecutive blank lines to allow in the output. By default
  \texttt{listings} will already ignore blank lines at the end of what
  ever is shown. To show only one blank line in the output for \R, try
  \begin{solution}
    \lstdefinestyle{r-user}{
      emptylines=1,
    }
  \end{solution}
  If you are also using line numbers, you may want to use 
  \begin{solution}
    \lstdefinestyle{r-user}{
      emptylines=*1,
    }
  \end{solution}
  then the line numbers `jump' correctly in regards to the blank lines.
\end{question}

\begin{question}{Can also blank space at the start of lines be
    ignored?}
  Of course, that key is called \texttt{gobble}, its value will
  indicate the number of characters to eat (from the left). Note that
  it will not distinguish between spaces and non-spaces, it will just
  eat a set number of characters at the start of each line.
\end{question}

\begin{question}{By the way, can one control the width of the \SAS
    output from within a \SAS programme?}
  Yes, try the >>\SASinline|OPTIONS LS=80;|<< setting.
\end{question}


\begin{question}{Can I show sniplets of code?}
  Sure. See the \texttt{firstline} and \texttt{lastline} keys or the
  \texttt{linerange}. They cannot be set globally, so can only be
  added into environment or input macro options.

  \fancybreak{}

  There is also an experimental feature where instead of line numbers
  one specify certain strings inside the external file. This can be
  quite handy if the contents of the external file may
  change.\footnote{Then one does not have to manually change the line
    number references all the time.} Section 5.7 in the
  \texttt{listings} manual (\cite{listings}) has more details. 
\end{question}


\begin{question}{The quotes look odd in my code listings, can it look
    more like keyboard keys?}
  Of course. Add the \texttt{textcomp} package, and issue
  \begin{solution}
    \lstdefinestyle{r-user}{
      upquote=true,
    }
    \lstdefinestyle{sas-user}{
      upquote=true,
    }
  \end{solution}
\end{question}


\begin{question}{I use the \sweave package and overload the look using
    \pkg.  . I'd like \texttt{Sinput} to look more like the default in
    \sweave.}
  This can be done by using the extra \texttt{Sinput} style to
  overload the basic style:
  \begin{solution}
    \lstdefinestyle{Sinput}{
      basicstyle = \ttfamily\itshape
    }
  \end{solution}
\end{question}


\chapter{Examples}
\label{sec:examples}


\section{\SAS}
\label{sec:sas}

inline: \verb+\SASinline|RANGE xxx|+ results in 
\SASinline|RANGE xxx|. 

Personally I often add >>\dots<< around inline sniplets to indicate
where they start and end. Sadly this is apparently not something one
can add into the inline macro definition because of its \cs{verb}-like
nature.

\begin{SAScode}[caption={Test of caption}]
PROC INSIGHT DATA data=fisk;
SCATTER x1 x2 x3 x4 x5 * dosis vgt;
RUN; 
OUTPUT
QUIT; /* a standard SAS comment */
\end{SAScode}
was typeset via
\begin{Show}
\begin{SAScode}[caption={Test of caption}]
PROC INSIGHT DATA data=fisk;
SCATTER x1 x2 x3 x4 x5 * dosis vgt;
RUN; 
OUTPUT
QUIT; /* a standard SAS comment */
\end{SAScode}
\end{Show}
%
whereas

\begin{SASoutput}
                           TABLE OF NIVEAU BY FAG

           NIVEAU     FAG

           Frequency|
           Percent  |
           Row Pct  |
           Col Pct  |kem     |mat     |mus     |samf    |  Total
           ---------+--------+--------+--------+--------+
           h        |      0 |      1 |      1 |      0 |      2
                    |   0.00 |  20.00 |  20.00 |   0.00 |  40.00
                    |   0.00 |  50.00 |  50.00 |   0.00 |
                    |   0.00 |  50.00 | 100.00 |   0.00 |
           ---------+--------+--------+--------+--------+
           m        |      0 |      1 |      0 |      1 |      2
                    |   0.00 |  20.00 |   0.00 |  20.00 |  40.00
                    |   0.00 |  50.00 |   0.00 |  50.00 |
                    |   0.00 |  50.00 |   0.00 | 100.00 |
           ---------+--------+--------+--------+--------+
           o        |      1 |      0 |      0 |      0 |      1
                    |  20.00 |   0.00 |   0.00 |   0.00 |  20.00
                    | 100.00 |   0.00 |   0.00 |   0.00 |
                    | 100.00 |   0.00 |   0.00 |   0.00 |
           ---------+--------+--------+--------+--------+
           Total           1        2        1        1        5
                       20.00    40.00    20.00    20.00   100.00

\end{SASoutput}
%
comes from
\begin{Show}
\begin{SASoutput}
                           TABLE OF NIVEAU BY FAG

           NIVEAU     FAG

           Frequency|
           Percent  |
           Row Pct  |
           Col Pct  |kem     |mat     |mus     |samf    |  Total
           ---------+--------+--------+--------+--------+
           h        |      0 |      1 |      1 |      0 |      2
                    |   0.00 |  20.00 |  20.00 |   0.00 |  40.00
                    |   0.00 |  50.00 |  50.00 |   0.00 |
                    |   0.00 |  50.00 | 100.00 |   0.00 |
           ---------+--------+--------+--------+--------+
           m        |      0 |      1 |      0 |      1 |      2
                    |   0.00 |  20.00 |   0.00 |  20.00 |  40.00
                    |   0.00 |  50.00 |   0.00 |  50.00 |
                    |   0.00 |  50.00 |   0.00 | 100.00 |
           ---------+--------+--------+--------+--------+
           o        |      1 |      0 |      0 |      0 |      1
                    |  20.00 |   0.00 |   0.00 |   0.00 |  20.00
                    | 100.00 |   0.00 |   0.00 |   0.00 |
                    | 100.00 |   0.00 |   0.00 |   0.00 |
           ---------+--------+--------+--------+--------+
           Total           1        2        1        1        5
                       20.00    40.00    20.00    20.00   100.00

\end{SASoutput}
\end{Show}



\section{\R}
\label{sec:r}



\verb+\Rinline|x <- a|+ result in \Rinline|x <- a|

\inputRcode[
 emptylines=*1,
 numbers=left,
 numberstyle=\tiny,
 caption={With just one blank line showing, plus line numbers. Note we
 only show the first 25 lines.},
 linerange=1-25,
]{opg_G17.R}
\noindent
was typeset via
\begin{Show}
\inputRcode[
 emptylines=*1,
 numbers=left,
 numberstyle=\tiny,
 caption={With just one blank line showing, plus line numbers. Note we
 only show the first 25 lines.},
 linerange=1-25
]{opg_G17.R}
\end{Show}



\begin{Routput}[caption={Copied from an Sweave result.}]
 [1] 0.0495
\end{Routput}
is just sumple use of the \texttt{Routput} environment.


\chapter{Default style settings for \R and \SAS}
\label{cha:defa-list-style}






\section{Style settings for \R}
\label{sec:style-settings-r}

Note how there are three user styles. The style \texttt{r-user} apply
to both \R code and \R output, whereas \texttt{r-code-user} and
\texttt{r-output-user} apply only to code and output
respectively. This means that if, say, the user want to change the
framing to a line above and below instead of the default box, the user
can either overwrite the \texttt{r-frame} style, or the
\texttt{r-user} style.

The style settings are divided into three separate groups: (a) the
settings themselves, (b) user styles and (c) collector styles, which
are just a common name and loading order for other styles and

\subsection{\R configuration styles}
\label{sec:r-conf-styl}


\getsniplet{rvskip}{\texttt{r-vskips}}
\getsniplet{rfonts}{\texttt{r-fonts}} The font is the base font for
the rest. If for example one make use of \texttt{emphstyle=\bfseries},
then one will get the \texttt{basicstyle} plus bold face (if possible)
for things that are emphasized.
\getsniplet{rcharsandbreaks}{\texttt{r-chars-and-breaks}}
\getsniplet{rmarkup}{\texttt{r-markup}}
\getsniplet{rframe}{\texttt{r-frame}}
\getsniplet{rcolors}{\texttt{r-colors}}
\noindent
The colors being used as standard have special names. By default they
are in color. If the \texttt{monochrome} option is issued they are
mapped onto \texttt{SnRBGGray} and \texttt{SnRFrameGray}, the later
being black by default.


\getsniplet{rinline}{\texttt{r-inline}}
Note that by default, the inline style for \verb|\Rinline|, shares
nothing with the rest of the \R configuration and is loaded on its own.

\subsection{\R user styles}
\label{sec:user-styles}

These are all empty by default. There are two types of these: (i)
styles applying to both types (code and output) in one go, and (ii)
ones that are specific to either code or output. We also have a third
level \emph{file include} versus \emph{in document}. The more specific
the name, the later it will come in the collector styles (i.e.\ its
settings will apply last).

\getsniplet{ruser}{\texttt{r-user} -- user stuff for everything \R}
\getsniplet{rcodeuser}{\texttt{r-code-user} -- only for \R code}
\getsniplet{routputuser}{\texttt{r-output-user} -- only for \R output}

\fancybreak{}

These three apply only to inclusion macros. Can be handy to specify an
input encoding for, say, all included code files.

\getsniplet{rincludeuser}{\texttt{r-include-user}}
\getsniplet{rincludecodeuser}{\texttt{r-include-code-user}}
\getsniplet{rincludeoutputuser}{\texttt{r-include-output-user}}

\subsection{\R collector styles}
\label{sec:r-collector-styles}

Please note the calling sequence.

\getsniplet{rstyle}{\texttt{r-style} -- all common code for \R}

\getsniplet{rcode}{\texttt{r-code} -- specific for \R code}
\getsniplet{routput}{\texttt{r-output} -- specific for \R output}


\fancybreak{}

Next we have the extra styles used for inclusion macros. They are the
same as for \R code and \R output, with the addition of two extra styles.

\getsniplet{rincludecode}{\texttt{r-include-code}}
\getsniplet{rincludeoutput}{\texttt{r-include-output}}


\section{Style settings for \SAS}
\label{sec:style-settings-sas}

The structure is similar to the one used for \R, though the font
settings are split in two.

\subsection{\SAS configuration styles}

\getsniplet{sasinline}{\texttt{sas-inline}}
\getsniplet{sasvskips}{\texttt{sas-vskips}}
\getsniplet{sascolors}{\texttt{sas-colors}}
\getsniplet{sascodefonts}{\texttt{sas-code-fonts}}
\getsniplet{sasoutputfonts}{\texttt{sas-output-fonts}}
\noindent
Note how we actually split the font settings in two. This is because
it is common to have \SAS code and \SAS output i different font
sizes. As \R output is not that common, this split has not been made
with \R, it will be up to the user.
\getsniplet{sascharsandbreaks}{\texttt{sas-chars-and-breaks}}
\getsniplet{sasmarkup}{\texttt{sas-markup}}
Note that only the \texttt{/*...*/} style \SAS comments are supported
for styling. The \texttt{*...;} syntax is \emph{not} supported. Also
note we had to change the \SAS settings for \texttt{listings}
otherwise we would be unable to style \SAS comments.

\getsniplet{sasframe}{\texttt{sas-frame}}


\subsection{\SAS user styles}

\getsniplet{sasuser}{\texttt{sas-user} -- user stuff for everything \SAS}
\getsniplet{sascodeuser}{\texttt{sas-code-user} -- only for \SAS code}
\getsniplet{sasoutputuser}{\texttt{sas-output-user} -- only for \SAS output}

\fancybreak{}

These three apply only to inclusion macros. Can be handy to specify an
input encoding for, say, all included code files.

\getsniplet{sasincludeuser}{\texttt{sas-include-user}}
\getsniplet{sasincludecodeuser}{\texttt{sas-include-code-user}}
\getsniplet{sasincludeoutputuser}{\texttt{sas-include-output-user}}




\subsection{\SAS collector styles}

\getsniplet{sasstyle}{\texttt{sas-style} -- all common code for \SAS}
\getsniplet{sascode}{\texttt{sas-code} -- specific for \SAS code}
\getsniplet{sasoutput}{\texttt{sas-output} -- specific for \SAS output}
\getsniplet{sasincludecode}{\texttt{sas-include-code}}
\getsniplet{sasincludeoutput}{\texttt{sas-include-output}}



\subsection*{Extra SAS keywords}
\label{sec:extra-sas-keywords}

J�rgen Granfeldt supplied extra \SAS keywords to supplement those
supported by \texttt{listings}. The keywords are found in
\texttt{\pkg.cfg} and is labelled as the \texttt{sas-more-keywords}
style. Note that even though not required by \SAS, all supported
keywords are written in upper case. JG explains that this is
encouraged because that it makes it easier to tell the difference
between build in \SAS commands and user supplied (lower case)
variables and procedure names.

Please note that we also change a list of \emph{other} keywords,
otherwise we will be unable to style \SAS comments.


Here is the current list.
\def\NAME{SASnRdisplay.cfg}
\getsniplet{sasmorekeywords}{\texttt{sas-more-keywords}, from \texttt{\pkg.cfg}}

\begin{thebibliography}{9}
\RaggedRight
\bibitem{xcolor} Uwe Kern, \emph{Extending \LaTeX's color facilities:
    the \textsf{xcolor} package}, 2006.
  \addCTAN{/macros/latex/contrib/xcolor/}.

\bibitem{listings} Various, \emph{The \textsf{listings} Package},
  1996--2007.  \addCTAN{/macros/latex/contrib/listings/}.

\bibitem{etoolbox}
  Philipp Lehman,
\emph{The etoolbox Package -- An e-TeX Toolbox for Class and Package
  Authors}, 2011. \addCTAN{/macros/latex/contrib/listings/etoolbox}.


\bibitem{listingsutf8}
  Heiko Oberdiek, \emph{The \textsf{listingsutf8} package}, 2007.
  \addCTAN{/macros/latex/contrib/oberdiek/}.

\bibitem{caption} Axel Sommerfelt, \emph{Customizing captions of
    floating environments using the caption package}, 2011.
  \addCTAN{/macros/latex/contrib/caption}.




\end{thebibliography}


\end{document}

%%% Local Variables: 
%%% mode: latex
%%% TeX-master: t
%%% End: 

