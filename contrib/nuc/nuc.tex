\documentclass{article}
\author{Neal Davis}
\title{The \textbf{nuc} package, v0.1}

\usepackage{nuc}

\begin{document}

\maketitle

\texttt{nuc.sty} is provided by Neal Davis (davis68@illinois.edu).  This package provides automatic notation for nuclear isotopes.  Isotopes which have Z with more digits than A require special spacing to appear properly.  nuc.sty is released under the LaTeX Project Public License.

This is an early draft (v0.1) of this package.

It is invoked in the usual way,
    \texttt{\textbackslash usepackage{nuc}}
and each element and isotope is indicated by \texttt{\textbackslash(element name)\{(element weight)\}}.
For instance, \texttt{\textbackslash Pa\{231\}} produces \Pa{231}; \texttt{\textbackslash O\{18\}} produces \O{18}; and \texttt{\textbackslash H\{3\}} produces \H{3}.  This of necessity overrides the basic \LaTeX~commands \texttt{\textbackslash H}, \texttt{\textbackslash O}, \texttt{\textbackslash P}, \texttt{\textbackslash S}, \texttt{\textbackslash Pr}, and \texttt{\textbackslash Re}.

It appears that under certain circumstances, \protect may be required prior to
invoking the element (i.e., \texttt{\textbackslash protect\textbackslash Fe\{56\}}).

All elements are supported, with additional support for deprecated atomic symbols such as \Cb{93} (columbium, now niobium) and \Ku{263} (kurchatovium, now rutherfordium).

This package does not currently support omission of the atomic number subscript,
although that is planned for the next version.

\texttt{nuc.sty} is released under the LaTeX Project Public license v1.3 or later.

\end{document}