\def\fileversion{1.04}
\def\filedate {07 Apr 95}
\def\docdate  {06 Dec 94}
\def\docdatede{06 Dec 94}
% 
% \iffalse metacomment
%   This file is part of the mapcodes package, version 1.04.
% -----------------------------------------------------------
% Copyright (C) 1994 Michael Piotrowski. All rights reserved.
%
% This file is distributed in the hope that it will be useful,
% but WITHOUT ANY WARRANTY; without even the implied warranty
% of MERCHANTABILITY or FITNESS FOR A PARTICULAR PURPOSE.
% -----------------------------------------------------------
%
% IMPORTANT NOTICE:
%
% For error reports in case of UNCHANGED versions see readme file.
%
% You are not allowed to change this file.
%
% You are allowed to distribute this file under the condition that
% it is distributed together with all files mentioned in manifest.txt.
%
% If you receive only some of these files from someone, complain!
%
% You are NOT ALLOWED to distribute this file alone. You are NOT
% ALLOWED to take money for the distribution or use of either this
% file or a changed version, except for a nominal charge for copying.
%
% \fi
%
% \changes{v1.0}{94/12/04}{First release}
% \changes{v1.01}{94/12/06}{ISO 8859-2 added, all chars of all charsets are mapped now}
% \changes{v1.02}{95/02/07}{Modifications for \TeX 3.1415 under HP-UX}
% \changes{v1.03}{95/03/14}{Serious (pronounce: stupid) bug fixed}
% \changes{v1.04}{95/04/07}{Support for quotation marks in \texttt{german.dtx} added}
% 
% \MakeShortVerb{\|}
%
% \newcommand{\mapcodes}{\textsf{mapcodes}}
% \newcommand{\package}[1]{\textsf{#1}}
% \newcommand{\option}[1]{\texttt{#1}}
%
% \title{Using 8-bit Character Sets with \mapcodes{}
%    \thanks{This document describes version \fileversion{} of \filedate{}.}}
% \author{Michael Piotrowski\\
%    \texttt{mlpiotro@linguistik.uni-erlangen.de}}
% \date{Printed on \today}
% \maketitle
% \begin{abstract}
%    The \mapcodes{} package allows you to use 8-bit input files
%    in a variety of encodings, like ISO 8859-1 or the IBM codepage 850.
%    It can be used with both T1 and OT1 fonts and language-specific
%    extension packages.
% \end{abstract}
%
% \section{Deutsche Kurzbeschreibung}
%
%    \mapcodes{} erm\"oglicht die Verwendung von 8-bit-Eingabedateien
%    mit verschiedenen Zeichens\"atzen.
%
%    In dieser Version sind das:
%    \begin{itemize}
%       \item ISO 8859-1 (Latin 1)
%       \item ISO 8859-2 (Latin 2)
%       \item IBM Kodeseite 850
%       \item IBM Kodeseite 852
%       \item HP Roman 8
%       \item Macintosh Roman
%       \item Atari TOS
%    \end{itemize}
%
%    Um \mapcodes{} zu benutzen, schreiben Sie eine Zeile der folgenden
%    Form in die Pr\"aambel der Eingabedatei:
%    \begin{quote}
%       |\usepackage[|\textit{Zeichensatz}|]{mapcodes}|
%    \end{quote}
%
%    F\"ur \textit{Zeichensatz} k\"onnen sie im Augenblick folgendes
%    einsetzen:\footnote{F\"ur iso8859-$x$ kann auch latin$x$ geschrieben
%    werden.}
%    \begin{center}
%       \begin{tabular}{llll}
%           |iso8859-1|&|iso8859-2|&|ibm850|&|ibm852|\\
%           |hproman8|&|macroman|&|atari|&\\
%       \end{tabular}
%    \end{center}
%
%    \mapcodes{} erkennt selbst\"andig, ob Sie T1- oder OT1-Schriften
%    verwenden, wobei erstere zu empfehlen sind. Wenn Sie OT1-Schriften
%    und ein LaTeX-Paket wie \package{german} oder eine \package{babel}-Option
%    wie \option{german} oder \option{spanish} verwenden, kann
%    \mapcodes{} die entsprechenden Erweiterungen nutzen, die z. B.
%    Trennungen in W\"ortern mit Umlauten erlauben. Geben sie hierzu
%    die Sprache als zus\"atzliche Option an. Beispiel:
%    \begin{quote}
%      |\usepackage[hproman8,german]{mapcodes}|
%    \end{quote}
%    Sie k\"onnen die Sprache auch als globale Option zum
%    |\documentclass|-Befehl angeben. Im Augenblick werden folgende
%    Sprachoptionen unterst\"utzt:
%    
%    \begin{center}
%    \begin{tabular}{llll}
%       |german|&|spanish|&&\\
%       \end{tabular}
%    \end{center}
%
%
% \section{Introduction}
%    When using \TeX{} with languages other than English you always have the
%    problem of how to enter language-specific characters like \"a, \'e or
%    \'\i. The \LaTeX{} commands |\"{a}|, |\'{e}| and |\`{\i{}}| are neither
%    intended nor adequate for writing texts in German, French or Italian,
%    as they are too difficult to type; besides of that, they inhibit
%    proper hyphenation of the words in which they are contained.
%
%    Packages like \package{german} or \package{babel} offer some help
%    in producing non-English texts, providing commands like |"a| for \"a
%    which also allow correct hyphenation. However, this is still not
%    perfect; it would be much nicer if you could simply use the ``\"a'' key
%    on a German keyboard and directly see ``\"a'' on your screen.
%    The drawback here is that unfortunately nearly every computer system
%    is still using its own character sets and tables, which makes it
%    difficult to correctly process 8-bit coded texts from one system
%    on an other system.
%
%    The \mapcodes{} package allows you to use (nearly) all of the
%    characters in the character set you are using while providing a high
%    level of portability.
%  
% \section{Specifications}
%
%    Currently supported character sets:
%    \begin{itemize}
%       \item ISO 8859-1 (Latin 1)
%       \item ISO 8859-2 (Latin 2)
%       \item IBM codepage 850
%       \item IBM codepage 852
%       \item HP Roman 8
%       \item Macintosh Roman
%       \item Atari TOS
%    \end{itemize}
%
%    \mapcodes{} can easily be extended in a modular way to support
%    further encodings.
%
%    Output can be produced for:
%    \begin{itemize}
%       \item OT1 fonts; |"cf| in Latin 1 encoding will be mapped to |\ss|
%       \item OT1 with extension packages; currently supported are the
%             \package{german} package and the \option{german} and
%             \option{spanish} options of \package{babel}. For the
%             \package{german} package the above code would
%             be mapped to |"s| and thus allowing hyphenation.
%       \item T1 fonts; |"cf| is mapped to the the corresponding character
%             code in the T1 set, namely |"ff|.
%    \end{itemize}
%
% \section{The User Interface}
%
%    To use \package{mapcodes}, simply specifiy it in a |\usepackage| command,
%    with the name of the desired encoding as an option. Example:
%    \begin{quote}
%                 |\usepackage[iso8859-1]{mapcodes}|
%    \end{quote}
%
%    If you are using em\TeX\, make sure that you use a \LaTeX{}
%    format \textbf{without} a character translation table built-in
%    (this is done by specifying the option \option{/c} when building it).
%    Use the option \option{/8} instead to enable 8-bit character processing.
%
%    \package{mapcodes} automatically detects whether you are using T1 or OT1
%    fonts. Note, however, that the usage of OT1 fonts can in some
%    cases yield esthetically dubious results due to limitations of these
%    fonts. Usage of T1 fonts is recommended. If you are using OT1 fonts with
%    one of the extension packages named above, you can request support for
%    them by specifying the language name in addition to the encoding.
%    Example:
%    \begin{quote}
%                 |\usepackage[hproman8,german]{mapcodes}|
%    \end{quote}
%
%    When you are using T1 fonts, the request will be ignored.
%
%    Valid encoding options in this version are:\footnote{|iso8859-|$x$ can
%    be replaced by |latin|$x$}
%    \begin{center}
%    \begin{tabular}{llll}
%       |iso8859-1|&|iso8859-1|&|ibm850|&|ibm852|\\
%       |hproman8|&|macroman|&|atari|&\\
%    \end{tabular}
%    \end{center}
%
%    Valid language options in this version are:
%    \begin{center}
%    \begin{tabular}{llll}
%       |german|&|spanish|&&\\
%       \end{tabular}
%    \end{center}
%
% \section{Bugs and features}
%    Suggestions and bug reports are welcome. Direct them to the
%    E-mail address indicated on the first page.
%
% \StopEventually{
%   \section{Acknowledgments}
%      The character tables in Kosta Kostis' \texttt{trans097} conversion
%      tools were very helpful.
%   
%   \begin{thebibliography}{9}
%      \bibitem{Lamport} Lamport, Leslie. \emph{\LaTeX: A Document
%      Preparation System.} 2nd ed. Reading, Mass.: Addison-Wesley 1994.
%      \bibitem{PSMan} Adobe Systems Incorporated. \emph{POSTSCRIPT language
%      reference manual.} Reading, Mass.: Addison-Wesley 1986.
%   \end{thebibliography}
% }
%
% \section{The Code}
%
% The preliminaries:
%    \begin{macrocode}
%<*package>
\NeedsTeXFormat{LaTeX2e}
\ProvidesPackage{mapcodes}[1995/02/07 mapcodes 1.02]
\typeout{Package `mapcodes' \fileversion\space<\filedate> (Michael Piotrowski)}
\typeout{English documentation\space\space<\docdate>}
\typeout{Deutsche Beschreibung\space\space<\docdatede>}
%    \end{macrocode}
%
% Some constants and varibles are defined:
%    \begin{macrocode}
\newcommand\map@corkenc{T1}
\newcommand\map@oldenc{OT1}
\newlength{\map@width}
\newlength{\map@height}
%    \end{macrocode}
%
% \begin{macro}{\map@overlay}
% This little macro is used to produce the yen sign and the registered
% trademark symbol. The second argument is centered over the first one.
% For details consult \cite[p. 108ff]{Lamport}
%    \begin{macrocode}
\newcommand{\map@overlay}[2]{%
   \settowidth{\map@width}{#1}%
   \makebox[0pt][l]{\makebox[\map@width]{#2}}%
   {#1}%
}
%    \end{macrocode}
% \end{macro}
% \begin{macro}{\map@accent}
% The following macro is my own implementation of the \TeX{} |\accent|
% command. The second argument is lifted up and placed over the first one.
%    \begin{macrocode}
\newcommand{\map@accent}[2]{%
   \settowidth{\map@width}{#1}%
   \settoheight{\map@height}{#1}%
   \addtolength{\map@height}{0.2\map@height}%
   \raisebox{\map@height}[0pt][0pt]{%
      \makebox[0pt][l]{%
         \makebox[\map@width]{#2}}}%
   {#1}%
}
%    \end{macrocode}
% \end{macro}
%
% The following definitions are used for both T1 and OT1 fonts. The names
% for the characters are the names defined in \cite{PSMan} with a
% prefixed ``map@'' to be in any case different from possibly existing
% \LaTeX{} commands.
%    \begin{macrocode}
\newcommand\map@nil{\ensuremath{\bigotimes}}
\newcommand\map@ordfeminine{{\raise1ex\hbox{\underbar{\scriptsize a}}}}
\newcommand\map@ordmasculine{{\raise1ex\hbox{\underbar{\scriptsize o}}}}
\newcommand\map@cent{\leavevmode\hbox{\rm\rlap/c}}
\newcommand\map@copyright{\copyright{}}
\newcommand\map@paragraph{\P{}}
\newcommand\map@trademark{{\raise1ex\hbox{\scriptsize TM}}}
\newcommand\map@registered{\map@overlay{\ensuremath{\bigcirc}}{\scriptsize R}}
\newcommand\map@multiply{\ensuremath{\times}}
\newcommand\map@divide{\ensuremath{\div}}
\newcommand\map@onesuperior{\ensuremath{^1}}
\newcommand\map@twosuperior{\ensuremath{^2}}
\newcommand\map@threesuperior{\ensuremath{^3}}
\newcommand\map@onequarter{\ensuremath{\frac{1}{4}}}
\newcommand\map@onehalf{\ensuremath{\frac{1}{2}}}
\newcommand\map@threequarters{\ensuremath{\frac{3}{4}}}
\newcommand\map@space{~}
\newcommand\map@hyphen{-}
\newcommand\map@mu{\ensuremath{\mu}}
\newcommand\map@yen{\map@overlay{Y}{--}}
\newcommand\map@logicalnot{\ensuremath{\neg}}
\newcommand\map@plusminus{\ensuremath{\pm}}
\newcommand\map@periodcentered{\ensuremath{\cdot}}
\newcommand\map@degree{\ensuremath{^\circ}}
\newcommand\map@brokenbar{\ensuremath{\mid}}
\newcommand\map@currency{\ensuremath{\circ}}
\newcommand\map@dagger{\ensuremath{\dagger}}
\newcommand\map@daggerdbl{\ensuremath{\ddagger}}
\newcommand\map@bullet{\ensuremath{\bullet}}
\newcommand\map@aleph{\ensuremath{\aleph}}
\newcommand\map@alpha{\ensuremath{\alpha}}
\newcommand\map@beta{\ensuremath{\beta}}
\newcommand\map@gamma{\ensuremath{\gamma}}
\newcommand\map@pi{\ensuremath{\pi}}
\newcommand\map@productdot{\ensuremath{\cdot}}
\newcommand\map@notequal{\ensuremath{\not=}}
\newcommand\map@infinity{\ensuremath{\infty}}
\newcommand\map@lessequal{\ensuremath{\leq}}
\newcommand\map@greaterequal{\ensuremath{\geq}}
\newcommand\map@partialdiff{\ensuremath{\partial}}
\newcommand\map@summation{\ensuremath{\sum}}
\newcommand\map@product{\ensuremath{\prod}}
\newcommand\map@integral{\ensuremath{\int}}
\newcommand\map@Omega{\ensuremath{\Omega}}
\newcommand\map@radical{\ensuremath{\surd}}
\newcommand\map@florin{\textrm{\textit{f}}}
\newcommand\map@approxequal{\ensuremath{\approx}}
\newcommand\map@Delta{\ensuremath{\Delta}}
\newcommand\map@quoteleft{`}
\newcommand\map@quoteright{'}
\newcommand\map@lozenge{\ensuremath{\diamond}}
\newcommand\map@fi{fi}
\newcommand\map@fl{fl}
\newcommand\map@cdots{\ensuremath{\cdots}} % not a Postscript name!
\newcommand\map@fraction{/}
\newcommand\map@{\ensuremath{\pi}}
\newcommand\map@Gamma{\ensuremath{\Gamma}}
\newcommand\map@logicaland{\ensuremath{\wedge}}
\newcommand\map@Sigma{\ensuremath{\Sigma}}
\newcommand\map@sigma{\ensuremath{\sigma}}
\newcommand\map@tau{\ensuremath{\tau}}
\newcommand\map@Phi{\ensuremath{\Phi}}
\newcommand\map@Theta{\ensuremath{\Theta}}
\newcommand\map@delta{\ensuremath{\delta}}
\newcommand\map@phi{\ensuremath{\phi}}
\newcommand\map@epsilon{\ensuremath{\varepsilon}}
\newcommand\map@intersection{\ensuremath{\bigcap}}
\newcommand\map@equivalence{\ensuremath{\equiv}}
\newcommand\map@nsuperior{\ensuremath{^n}}
\newcommand\map@filledbox{\ensuremath{\diamondsuit}}


%    \end{macrocode}
%
% If T1 fonts are used the following definitions are made:
%    \begin{macrocode}
\ifx\encodingdefault\map@corkenc
\newcommand\map@grave{\symbol{00}}
\newcommand\map@acute{\symbol{01}}
\newcommand\map@circumflex{\symbol{02}}
\newcommand\map@tilde{\symbol{03}}
\newcommand\map@dieresis{\symbol{04}}
\newcommand\map@hungarumlaut{\symbol{05}}
\newcommand\map@ring{\symbol{06}}
\newcommand\map@caron{\symbol{07}}
\newcommand\map@breve{\symbol{08}}
\newcommand\map@macron{\symbol{09}}
\newcommand\map@dotaccent{\symbol{10}}
\newcommand\map@cedilla{\symbol{11}}
\newcommand\map@ogonek{\symbol{12}}
\newcommand\map@quotesinglbase{\symbol{13}}
\newcommand\map@guilsinglleft{\symbol{14}}
\newcommand\map@guilsinglright{\symbol{15}}
\newcommand\map@quotedblleft{\symbol{16}}
\newcommand\map@quotedblright{\symbol{17}}
\newcommand\map@quotedblbase{\symbol{18}}
\newcommand\map@guillemotleft{\symbol{19}}
\newcommand\map@guillemotright{\symbol{20}}
\newcommand\map@endash{\symbol{21}}
\newcommand\map@emdash{\symbol{22}}
\newcommand\map@perthousand{\%\symbol{23}}
\newcommand\map@dotlessi{\symbol{24}}
\newcommand\map@dotlessj{\symbol{25}}
\newcommand\map@Abreve{\symbol{128}}
\newcommand\map@Aogonek{\symbol{129}}
\newcommand\map@Cacute{\symbol{130}}
\newcommand\map@Ccaron{\symbol{131}}
\newcommand\map@Dcaron{\symbol{132}}
\newcommand\map@Ecaron{\symbol{133}}
\newcommand\map@Eogonek{\symbol{134}}
\newcommand\map@Gbreve{\symbol{135}}
\newcommand\map@Lacute{\symbol{136}}
\newcommand\map@Lquoteright{\symbol{137}}
\newcommand\map@Lslash{\symbol{138}}
\newcommand\map@Nacute{\symbol{139}}
\newcommand\map@Ncaron{\symbol{140}}
\newcommand\map@NJ{\symbol{141}}
\newcommand\map@Ohungarumlaut{\symbol{142}}
\newcommand\map@Racute{\symbol{143}}
\newcommand\map@Rcaron{\symbol{144}}
\newcommand\map@Sacute{\symbol{145}}
\newcommand\map@Scaron{\symbol{146}}
\newcommand\map@Scedilla{\symbol{147}}
\newcommand\map@Tcaron{\symbol{148}}
\newcommand\map@Tcedilla{\symbol{149}}
\newcommand\map@Uhungarumlaut{\symbol{150}}
\newcommand\map@Uring{\symbol{151}}
\newcommand\map@Ydieresis{\symbol{152}}
\newcommand\map@Zacute{\symbol{153}}
\newcommand\map@Zcaron{\symbol{154}}
\newcommand\map@Zdotaccent{\symbol{155}}
\newcommand\map@IJ{\symbol{156}}
\newcommand\map@Idotaccent{\symbol{157}}
\newcommand\map@dslash{\symbol{158}}
\newcommand\map@section{\symbol{159}}
\newcommand\map@abreve{\symbol{160}}
\newcommand\map@aogonek{\symbol{161}}
\newcommand\map@cacute{\symbol{162}}
\newcommand\map@ccaron{\symbol{163}}
\newcommand\map@dquoteright{\symbol{164}}
\newcommand\map@ecaron{\symbol{165}}
\newcommand\map@eogonek{\symbol{166}}
\newcommand\map@gbreve{\symbol{167}}
\newcommand\map@lacute{\symbol{168}}
\newcommand\map@lquoteright{\symbol{169}}
\newcommand\map@lslash{\symbol{170}}
\newcommand\map@nacute{\symbol{171}}
\newcommand\map@ncaron{\symbol{172}}
\newcommand\map@nj{\symbol{173}}
\newcommand\map@ohungarumlaut{\symbol{174}}
\newcommand\map@racute{\symbol{175}}
\newcommand\map@rcaron{\symbol{176}}
\newcommand\map@sacute{\symbol{177}}
\newcommand\map@scaron{\symbol{178}}
\newcommand\map@scedilla{\symbol{179}}
\newcommand\map@tquoteright{\symbol{180}}
\newcommand\map@tcedilla{\symbol{181}}
\newcommand\map@uhungarumlaut{\symbol{182}}
\newcommand\map@uring{\symbol{183}}
\newcommand\map@ydieresis{\symbol{184}}
\newcommand\map@zacute{\symbol{185}}
\newcommand\map@zcaron{\symbol{186}}
\newcommand\map@zdotaccent{\symbol{187}}
\newcommand\map@ij{\symbol{188}}
\newcommand\map@exclamdown{\symbol{189}}
\newcommand\map@questiondown{\symbol{190}}
\newcommand\map@sterling{\symbol{191}}
\newcommand\map@Agrave{\symbol{192}}
\newcommand\map@Aacute{\symbol{193}}
\newcommand\map@Acircumflex{\symbol{194}}
\newcommand\map@Atilde{\symbol{195}}
\newcommand\map@Adieresis{\symbol{196}}
\newcommand\map@Aring{\symbol{197}}
\newcommand\map@AE{\symbol{198}}
\newcommand\map@Ccedilla{\symbol{199}}
\newcommand\map@Egrave{\symbol{200}}
\newcommand\map@Eacute{\symbol{201}}
\newcommand\map@Ecircumflex{\symbol{202}}
\newcommand\map@Edieresis{\symbol{203}}
\newcommand\map@Igrave{\symbol{204}}
\newcommand\map@Iacute{\symbol{205}}
\newcommand\map@Icircumflex{\symbol{206}}
\newcommand\map@Idieresis{\symbol{207}}
\newcommand\map@Eth{\symbol{208}}%%%%% Achtung! Eth und Dslash
\newcommand\map@Dslash{\symbol{208}}%% sind Synonyme! (gleiche Nummern!)
\newcommand\map@Ntilde{\symbol{209}}
\newcommand\map@Ograve{\symbol{210}}
\newcommand\map@Oacute{\symbol{211}}
\newcommand\map@Ocircumflex{\symbol{212}}
\newcommand\map@Otilde{\symbol{213}}
\newcommand\map@Odieresis{\symbol{214}}
\newcommand\map@OE{\symbol{215}}
\newcommand\map@Oslash{\symbol{216}}
\newcommand\map@Ugrave{\symbol{217}}
\newcommand\map@Uacute{\symbol{218}}
\newcommand\map@Ucircumflex{\symbol{219}}
\newcommand\map@Udieresis{\symbol{220}}
\newcommand\map@Yacute{\symbol{221}}
\newcommand\map@Thorn{\symbol{222}}
\newcommand\map@SS{\symbol{223}}
\newcommand\map@agrave{\symbol{224}}
\newcommand\map@aacute{\symbol{225}}
\newcommand\map@acircumflex{\symbol{226}}
\newcommand\map@atilde{\symbol{227}}
\newcommand\map@adieresis{\symbol{228}}
\newcommand\map@aring{\symbol{229}}
\newcommand\map@ae{\symbol{230}}
\newcommand\map@ccedilla{\symbol{231}}
\newcommand\map@egrave{\symbol{232}}
\newcommand\map@eacute{\symbol{233}}
\newcommand\map@ecircumflex{\symbol{234}}
\newcommand\map@edieresis{\symbol{235}}
\newcommand\map@igrave{\symbol{236}}
\newcommand\map@iacute{\symbol{237}}
\newcommand\map@icircumflex{\symbol{238}}
\newcommand\map@idieresis{\symbol{239}}
\newcommand\map@eth{\symbol{240}}
\newcommand\map@ntilde{\symbol{241}}
\newcommand\map@ograve{\symbol{242}}
\newcommand\map@oacute{\symbol{243}}
\newcommand\map@ocircumflex{\symbol{244}}
\newcommand\map@otilde{\symbol{245}}
\newcommand\map@odieresis{\symbol{246}}
\newcommand\map@oe{\symbol{247}}
\newcommand\map@oslash{\symbol{248}}
\newcommand\map@ugrave{\symbol{249}}
\newcommand\map@uacute{\symbol{250}}
\newcommand\map@ucircumflex{\symbol{251}}
\newcommand\map@udieresis{\symbol{252}}
\newcommand\map@yacute{\symbol{253}}
\newcommand\map@thorn{\symbol{254}}
\newcommand\map@germandbls{\symbol{255}}
%    \end{macrocode}
%
% With OT1 fonts, however, the following definitions are made:
%    \begin{macrocode}
\else
\newcommand\map@grave{\`{}}
\newcommand\map@acute{\'{}}
\newcommand\map@circumflex{\^{}}
\newcommand\map@tilde{\~{}}
\newcommand\map@dieresis{\"{}}
\newcommand\map@hungarumlaut{\H{}}
\newcommand\map@ring{}
\newcommand\map@caron{\v{}}
\newcommand\map@breve{\u{}}
\newcommand\map@macron{\={}}
\newcommand\map@dotaccent{\.{}}
\newcommand\map@cedilla{\c{}}
\newcommand\map@ogonek{\map@nil}
\newcommand\map@endash{-}
\newcommand\map@emdash{--}

\newcommand\map@aacute{\'a}
\newcommand\map@Aacute{\'A}
\newcommand\map@abreve{\u{a}}
\newcommand\map@Abreve{\u{A}}
\newcommand\map@Acircumflex{\^A}
\newcommand\map@acircumflex{\^a}
\newcommand\map@adieresis{\"a}
\newcommand\map@Adieresis{\"A}
\newcommand\map@ae{\ae{}}
\newcommand\map@AE{\AE{}}
\newcommand\map@agrave{\`a}
\newcommand\map@Agrave{\`A}
\newcommand\map@aogonek{\c{a}}
\newcommand\map@Aogonek{\c{A}}
\newcommand\map@Aring{\AA{}}
\newcommand\map@aring{\aa{}}
\newcommand\map@atilde{\~a}
\newcommand\map@Atilde{\~A}
\newcommand\map@cacute{\'c}
\newcommand\map@Cacute{\'C}
\newcommand\map@ccaron{\v{c}}
\newcommand\map@Ccaron{\v{C}}
\newcommand\map@Ccedilla{\c C}
\newcommand\map@ccedilla{\c c}
\newcommand\map@Dcaron{\v{D}}
\newcommand\map@dcaron{d\ensuremath{\!}'}
\newcommand\map@dotlessi{\i{}}
\newcommand\map@dotlessj{\j{}}
\newcommand\map@dslash{\map@overlay{d}{\ensuremath{^-}}}
\newcommand\map@Dslash{\map@Eth}
\newcommand\map@Eacute{\'E}
\newcommand\map@eacute{\'e}
\newcommand\map@ecaron{\v{E}}
\newcommand\map@Ecaron{\v{E}}
\newcommand\map@Ecircumflex{\^E}
\newcommand\map@ecircumflex{\^e}
\newcommand\map@Edieresis{\"E}
\newcommand\map@edieresis{\"e}
\newcommand\map@egrave{\`e}
\newcommand\map@Egrave{\`E}
\newcommand\map@eogonek{\c{e}}
\newcommand\map@Eogonek{\c{E}}
\newcommand\map@Eth{\makebox[0pt][l]{--}D}
\newcommand\map@eth{\ensuremath{\partial}}
\newcommand\map@exclamdown{!`}
\newcommand\map@gbreve{\u{g}}
\newcommand\map@Gbreve{\u{G}}
\newcommand\map@germandbls{\ss{}}
\newcommand\map@guillemotleft{\ensuremath{\scriptstyle\ll}}
\newcommand\map@guillemotright{\ensuremath{\scriptstyle\gg}}
\newcommand\map@guilsinglleft{\ensuremath{\scriptstyle <}}
\newcommand\map@guilsinglright{\ensuremath{\scriptstyle >}}
\newcommand\map@Iacute{\'I}
\newcommand\map@iacute{\'\i{}}
\newcommand\map@Icircumflex{\^I}
\newcommand\map@icircumflex{\^\i{}}
\newcommand\map@Idieresis{\"I}
\newcommand\map@idieresis{\"\i{}}
\newcommand\map@Idotaccent{\.{I}}
\newcommand\map@Igrave{\`I}
\newcommand\map@igrave{\`\i{}}
\newcommand\map@ij{ij}
\newcommand\map@IJ{IJ}
\newcommand\map@lacute{\'l}
\newcommand\map@Lacute{\'L}
\newcommand\map@lcaron{l\ensuremath{\!}'}% siehe Dudentaschenbuch Satz- und 
\newcommand\map@Lcaron{L\ensuremath{\!}'}% Korrekturvorschriften
\newcommand\map@lslash{\map@overlay{l}{-}}
\newcommand\map@Lslash{\makebox[0pt][l]{--}L}
\newcommand\map@nacute{\'n}
\newcommand\map@Nacute{\'N}
\newcommand\map@ncaron{\v{n}}
\newcommand\map@Ncaron{\v{n}}
\newcommand\map@ntilde{\~n}
\newcommand\map@Ntilde{\~N}
\newcommand\map@Oacute{\'O}
\newcommand\map@oacute{\'o}
\newcommand\map@Ocircumflex{\^O}
\newcommand\map@ocircumflex{\^o}
\newcommand\map@odieresis{\"o}
\newcommand\map@Odieresis{\"O}
\newcommand\map@oe{\oe{}}
\newcommand\map@OE{\OE{}}
\newcommand\map@Ograve{\`O}
\newcommand\map@ograve{\`o}
\newcommand\map@ohungarumlaut{\H{o}}
\newcommand\map@Ohungarumlaut{\H{O}}
\newcommand\map@oslash{\o{}}
\newcommand\map@Oslash{\O{}}
\newcommand\map@otilde{\~o}
\newcommand\map@Otilde{\~O}
\newcommand\map@perthousand{\ensuremath{^{0}/_{00}}}
\newcommand\map@questiondown{?`}
\newcommand\map@quotedblbase{,,}
\newcommand\map@quotedblleft{``}
\newcommand\map@quotedblright{''}
\newcommand\map@quotesinglbase{,}
\newcommand\map@racute{\'r}
\newcommand\map@Racute{\'R}
\newcommand\map@rcaron{\v{r}}
\newcommand\map@Rcaron{\v{R}}
\newcommand\map@Sacute{\'S}
\newcommand\map@sacute{\'s}
\newcommand\map@scaron{\v{s}}
\newcommand\map@Scaron{\v{S}}
\newcommand\map@Scedilla{\c{s}}
\newcommand\map@scedilla{\c{s}}
\newcommand\map@section{\S{}}
\newcommand\map@SS{SS}
\newcommand\map@sterling{\pounds{}}
\newcommand\map@tcaron{t\ensuremath{\!}'}
\newcommand\map@Tcaron{\v{T}}
\newcommand\map@Tcedilla{\c{T}}
\newcommand\map@tcedilla{\c{t}}
\newcommand\map@Thorn{\makebox[0pt][l]{l}\raisebox{0,3ex}{p}}
\newcommand\map@thorn{\makebox[0pt][l]{l}p}
\newcommand\map@uacute{\'u}
\newcommand\map@Uacute{\'U}
\newcommand\map@ucircumflex{\^u}
\newcommand\map@Ucircumflex{\^U}
\newcommand\map@Udieresis{\"U}
\newcommand\map@udieresis{\"u}
\newcommand\map@ugrave{\`u}
\newcommand\map@Ugrave{\`U}
\newcommand\map@Uhungarumlaut{\H{U}}
\newcommand\map@uhungarumlaut{\H{u}}
\newcommand\map@uring{\map@accent{u}{\ensuremath{\scriptscriptstyle\circ}}}
\newcommand\map@Uring{\map@accent{U}{\ensuremath{\scriptscriptstyle\circ}}}
\newcommand\map@Yacute{\'Y}
\newcommand\map@yacute{\'y}
\newcommand\map@Ydieresis{\"Y}
\newcommand\map@ydieresis{\"y}
\newcommand\map@zacute{\'z}
\newcommand\map@Zacute{\'Z}
\newcommand\map@zcaron{\v{z}}
\newcommand\map@Zcaron{\v{Z}}
\newcommand\map@zdotaccent{\.{z}}
\newcommand\map@Zdotaccent{\.{Z}}
\fi
%    \end{macrocode}
%
% That's it for the definitions. Now the options are evaluated:
%    \begin{macrocode}
\DeclareOption{iso8859-1}{% \iffalse metacomment
%   This file is part of the mapcodes package, version 1.04.
% -----------------------------------------------------------
% Copyright (C) 1994 Michael Piotrowski. All rights reserved.
%
% This file is distributed in the hope that it will be useful,
% but WITHOUT ANY WARRANTY; without even the implied warranty
% of MERCHANTABILITY or FITNESS FOR A PARTICULAR PURPOSE.
% -----------------------------------------------------------
%
% IMPORTANT NOTICE:
%
% For error reports in case of UNCHANGED versions see readme file.
%
% You are not allowed to change this file.
%
% You are allowed to distribute this file under the condition that
% it is distributed together with all files mentioned in manifest.txt.
%
% If you receive only some of these files from someone, complain!
%
% You are NOT ALLOWED to distribute this file alone. You are NOT
% ALLOWED to take money for the distribution or use of either this
% file or a changed version, except for a nominal charge for copying.
%
%
%<*driver>
\documentclass{article}
\usepackage{doc}
\DontCheckModules
\newcommand{\mapcodes}{\textsf{mapcodes}}
\begin{document}
 \DocInput{iso88591.dtx}
\end{document}
%</driver>
% \fi
%
% \def\filename{iso88591.dtx}
% \def\fileversion{v1.0}
% \def\filedate{1994/12/05}
% \CheckSum{387}
%
% \changes{iso88591-1.0}{1994/12/05}{First version}
%
% \section{ISO 8859-1 Character Table}
%
%    This is \filename{} \fileversion{} of \filedate. It contains the
%    ISO 8859-1 conversion table for \mapcodes.
% 
% \StopEventually{}
%
%    \begin{macrocode}
\typeout{Mapping ISO 8859-1 (Latin 1) to \encodingdefault\space encoding.}

\catcode160=\active  \def^^a0{\map@space}%
\catcode161=\active  \def^^a1{\map@exclamdown}%
\catcode162=\active  \def^^a2{\map@cent}%
\catcode163=\active  \def^^a3{\map@sterling}%
\catcode164=\active  \def^^a4{\map@currency}%
\catcode165=\active  \def^^a5{\map@yen}%
\catcode166=\active  \def^^a6{\map@brokenbar}%
\catcode167=\active  \def^^a7{\map@section}%
\catcode168=\active  \def^^a8{\map@dieresis}%
\catcode169=\active  \def^^a9{\map@copyright}%
\catcode170=\active  \def^^aa{\map@ordfeminine}%
\catcode171=\active  \def^^ab{\map@guillemotleft}%
\catcode172=\active  \def^^ac{\map@logicalnot}%
\catcode173=\active  \def^^ad{\map@hyphen}%
\catcode174=\active  \def^^ae{\map@registered}%
\catcode175=\active  \def^^af{\map@macron}%
\catcode176=\active  \def^^b0{\map@degree}%
\catcode177=\active  \def^^b1{\map@plusminus}%
\catcode178=\active  \def^^b2{\map@twosuperior}%
\catcode179=\active  \def^^b3{\map@threesuperior}%
\catcode180=\active  \def^^b4{\map@acute}%
\catcode181=\active  \def^^b5{\map@mu}%
\catcode182=\active  \def^^b6{\map@paragraph}%
\catcode183=\active  \def^^b7{\map@periodcentered}%
\catcode184=\active  \def^^b8{\map@cedilla}%
\catcode185=\active  \def^^b9{\map@onesuperior}%
\catcode186=\active  \def^^ba{\map@ordmasculine}%
\catcode187=\active  \def^^bb{\map@guillemotright}%
\catcode188=\active  \def^^bc{\map@onequarter}%
\catcode189=\active  \def^^bd{\map@onehalf}%
\catcode190=\active  \def^^be{\map@threequarters}%
\catcode191=\active  \def^^bf{\map@questiondown}%
\catcode192=\active  \def^^c0{\map@Agrave}%
\catcode193=\active  \def^^c1{\map@Aacute}%
\catcode194=\active  \def^^c2{\map@Acircumflex}%
\catcode195=\active  \def^^c3{\map@Atilde}%
\catcode196=\active  \def^^c4{\map@Adieresis}%
\catcode197=\active  \def^^c5{\map@Aring}%
\catcode198=\active  \def^^c6{\map@AE}%
\catcode199=\active  \def^^c7{\map@Ccedilla}%
\catcode200=\active  \def^^c8{\map@Egrave}%
\catcode201=\active  \def^^c9{\map@Eacute}%
\catcode202=\active  \def^^ca{\map@Ecircumflex}%
\catcode203=\active  \def^^cb{\map@Edieresis}%
\catcode204=\active  \def^^cc{\map@Igrave}%
\catcode205=\active  \def^^cd{\map@Iacute}%
\catcode206=\active  \def^^ce{\map@Icircumflex}%
\catcode207=\active  \def^^cf{\map@Idieresis}%
\catcode208=\active  \def^^d0{\map@Eth}%
\catcode209=\active  \def^^d1{\map@Ntilde}%
\catcode210=\active  \def^^d2{\map@Ograve}%
\catcode211=\active  \def^^d3{\map@Oacute}%
\catcode212=\active  \def^^d4{\map@Ocircumflex}%
\catcode213=\active  \def^^d5{\map@Otilde}%
\catcode214=\active  \def^^d6{\map@Odieresis}%
\catcode215=\active  \def^^d7{\map@multiply}%
\catcode216=\active  \def^^d8{\map@Oslash}%
\catcode217=\active  \def^^d9{\map@Ugrave}%
\catcode218=\active  \def^^da{\map@Uacute}%
\catcode219=\active  \def^^db{\map@Ucircumflex}%
\catcode220=\active  \def^^dc{\map@Udieresis}%
\catcode221=\active  \def^^dd{\map@Yacute}%
\catcode222=\active  \def^^de{\map@Thorn}%
\catcode223=\active  \def^^df{\map@germandbls}%
\catcode224=\active  \def^^e0{\map@agrave}%
\catcode225=\active  \def^^e1{\map@aacute}%
\catcode226=\active  \def^^e2{\map@acircumflex}%
\catcode227=\active  \def^^e3{\map@atilde}%
\catcode228=\active  \def^^e4{\map@adieresis}%
\catcode229=\active  \def^^e5{\map@aring}%
\catcode230=\active  \def^^e6{\map@ae}%
\catcode231=\active  \def^^e7{\map@ccedilla}%
\catcode232=\active  \def^^e8{\map@egrave}%
\catcode233=\active  \def^^e9{\map@eacute}%
\catcode234=\active  \def^^ea{\map@ecircumflex}%
\catcode235=\active  \def^^eb{\map@edieresis}%
\catcode236=\active  \def^^ec{\map@igrave}%
\catcode237=\active  \def^^ed{\map@iacute}%
\catcode238=\active  \def^^ee{\map@icircumflex}%
\catcode239=\active  \def^^ef{\map@idieresis}%
\catcode240=\active  \def^^f0{\map@eth}%
\catcode241=\active  \def^^f1{\map@ntilde}%
\catcode242=\active  \def^^f2{\map@ograve}%
\catcode243=\active  \def^^f3{\map@oacute}%
\catcode244=\active  \def^^f4{\map@ocircumflex}%
\catcode245=\active  \def^^f5{\map@otilde}%
\catcode246=\active  \def^^f6{\map@odieresis}%
\catcode247=\active  \def^^f7{\map@divide}%
\catcode248=\active  \def^^f8{\map@oslash}%
\catcode249=\active  \def^^f9{\map@ugrave}%
\catcode250=\active  \def^^fa{\map@uacute}%
\catcode251=\active  \def^^fb{\map@ucircumflex}%
\catcode252=\active  \def^^fc{\map@udieresis}%
\catcode253=\active  \def^^fd{\map@yacute}%
\catcode254=\active  \def^^fe{\map@thorn}%
\catcode255=\active  \def^^ff{\map@ydieresis}%
%    \end{macrocode}
% \Finale
\endinput
}
\DeclareOption{iso8859-2}{\input{iso88592.map}}
\DeclareOption{latin1}{% \iffalse metacomment
%   This file is part of the mapcodes package, version 1.04.
% -----------------------------------------------------------
% Copyright (C) 1994 Michael Piotrowski. All rights reserved.
%
% This file is distributed in the hope that it will be useful,
% but WITHOUT ANY WARRANTY; without even the implied warranty
% of MERCHANTABILITY or FITNESS FOR A PARTICULAR PURPOSE.
% -----------------------------------------------------------
%
% IMPORTANT NOTICE:
%
% For error reports in case of UNCHANGED versions see readme file.
%
% You are not allowed to change this file.
%
% You are allowed to distribute this file under the condition that
% it is distributed together with all files mentioned in manifest.txt.
%
% If you receive only some of these files from someone, complain!
%
% You are NOT ALLOWED to distribute this file alone. You are NOT
% ALLOWED to take money for the distribution or use of either this
% file or a changed version, except for a nominal charge for copying.
%
%
%<*driver>
\documentclass{article}
\usepackage{doc}
\DontCheckModules
\newcommand{\mapcodes}{\textsf{mapcodes}}
\begin{document}
 \DocInput{iso88591.dtx}
\end{document}
%</driver>
% \fi
%
% \def\filename{iso88591.dtx}
% \def\fileversion{v1.0}
% \def\filedate{1994/12/05}
% \CheckSum{387}
%
% \changes{iso88591-1.0}{1994/12/05}{First version}
%
% \section{ISO 8859-1 Character Table}
%
%    This is \filename{} \fileversion{} of \filedate. It contains the
%    ISO 8859-1 conversion table for \mapcodes.
% 
% \StopEventually{}
%
%    \begin{macrocode}
\typeout{Mapping ISO 8859-1 (Latin 1) to \encodingdefault\space encoding.}

\catcode160=\active  \def^^a0{\map@space}%
\catcode161=\active  \def^^a1{\map@exclamdown}%
\catcode162=\active  \def^^a2{\map@cent}%
\catcode163=\active  \def^^a3{\map@sterling}%
\catcode164=\active  \def^^a4{\map@currency}%
\catcode165=\active  \def^^a5{\map@yen}%
\catcode166=\active  \def^^a6{\map@brokenbar}%
\catcode167=\active  \def^^a7{\map@section}%
\catcode168=\active  \def^^a8{\map@dieresis}%
\catcode169=\active  \def^^a9{\map@copyright}%
\catcode170=\active  \def^^aa{\map@ordfeminine}%
\catcode171=\active  \def^^ab{\map@guillemotleft}%
\catcode172=\active  \def^^ac{\map@logicalnot}%
\catcode173=\active  \def^^ad{\map@hyphen}%
\catcode174=\active  \def^^ae{\map@registered}%
\catcode175=\active  \def^^af{\map@macron}%
\catcode176=\active  \def^^b0{\map@degree}%
\catcode177=\active  \def^^b1{\map@plusminus}%
\catcode178=\active  \def^^b2{\map@twosuperior}%
\catcode179=\active  \def^^b3{\map@threesuperior}%
\catcode180=\active  \def^^b4{\map@acute}%
\catcode181=\active  \def^^b5{\map@mu}%
\catcode182=\active  \def^^b6{\map@paragraph}%
\catcode183=\active  \def^^b7{\map@periodcentered}%
\catcode184=\active  \def^^b8{\map@cedilla}%
\catcode185=\active  \def^^b9{\map@onesuperior}%
\catcode186=\active  \def^^ba{\map@ordmasculine}%
\catcode187=\active  \def^^bb{\map@guillemotright}%
\catcode188=\active  \def^^bc{\map@onequarter}%
\catcode189=\active  \def^^bd{\map@onehalf}%
\catcode190=\active  \def^^be{\map@threequarters}%
\catcode191=\active  \def^^bf{\map@questiondown}%
\catcode192=\active  \def^^c0{\map@Agrave}%
\catcode193=\active  \def^^c1{\map@Aacute}%
\catcode194=\active  \def^^c2{\map@Acircumflex}%
\catcode195=\active  \def^^c3{\map@Atilde}%
\catcode196=\active  \def^^c4{\map@Adieresis}%
\catcode197=\active  \def^^c5{\map@Aring}%
\catcode198=\active  \def^^c6{\map@AE}%
\catcode199=\active  \def^^c7{\map@Ccedilla}%
\catcode200=\active  \def^^c8{\map@Egrave}%
\catcode201=\active  \def^^c9{\map@Eacute}%
\catcode202=\active  \def^^ca{\map@Ecircumflex}%
\catcode203=\active  \def^^cb{\map@Edieresis}%
\catcode204=\active  \def^^cc{\map@Igrave}%
\catcode205=\active  \def^^cd{\map@Iacute}%
\catcode206=\active  \def^^ce{\map@Icircumflex}%
\catcode207=\active  \def^^cf{\map@Idieresis}%
\catcode208=\active  \def^^d0{\map@Eth}%
\catcode209=\active  \def^^d1{\map@Ntilde}%
\catcode210=\active  \def^^d2{\map@Ograve}%
\catcode211=\active  \def^^d3{\map@Oacute}%
\catcode212=\active  \def^^d4{\map@Ocircumflex}%
\catcode213=\active  \def^^d5{\map@Otilde}%
\catcode214=\active  \def^^d6{\map@Odieresis}%
\catcode215=\active  \def^^d7{\map@multiply}%
\catcode216=\active  \def^^d8{\map@Oslash}%
\catcode217=\active  \def^^d9{\map@Ugrave}%
\catcode218=\active  \def^^da{\map@Uacute}%
\catcode219=\active  \def^^db{\map@Ucircumflex}%
\catcode220=\active  \def^^dc{\map@Udieresis}%
\catcode221=\active  \def^^dd{\map@Yacute}%
\catcode222=\active  \def^^de{\map@Thorn}%
\catcode223=\active  \def^^df{\map@germandbls}%
\catcode224=\active  \def^^e0{\map@agrave}%
\catcode225=\active  \def^^e1{\map@aacute}%
\catcode226=\active  \def^^e2{\map@acircumflex}%
\catcode227=\active  \def^^e3{\map@atilde}%
\catcode228=\active  \def^^e4{\map@adieresis}%
\catcode229=\active  \def^^e5{\map@aring}%
\catcode230=\active  \def^^e6{\map@ae}%
\catcode231=\active  \def^^e7{\map@ccedilla}%
\catcode232=\active  \def^^e8{\map@egrave}%
\catcode233=\active  \def^^e9{\map@eacute}%
\catcode234=\active  \def^^ea{\map@ecircumflex}%
\catcode235=\active  \def^^eb{\map@edieresis}%
\catcode236=\active  \def^^ec{\map@igrave}%
\catcode237=\active  \def^^ed{\map@iacute}%
\catcode238=\active  \def^^ee{\map@icircumflex}%
\catcode239=\active  \def^^ef{\map@idieresis}%
\catcode240=\active  \def^^f0{\map@eth}%
\catcode241=\active  \def^^f1{\map@ntilde}%
\catcode242=\active  \def^^f2{\map@ograve}%
\catcode243=\active  \def^^f3{\map@oacute}%
\catcode244=\active  \def^^f4{\map@ocircumflex}%
\catcode245=\active  \def^^f5{\map@otilde}%
\catcode246=\active  \def^^f6{\map@odieresis}%
\catcode247=\active  \def^^f7{\map@divide}%
\catcode248=\active  \def^^f8{\map@oslash}%
\catcode249=\active  \def^^f9{\map@ugrave}%
\catcode250=\active  \def^^fa{\map@uacute}%
\catcode251=\active  \def^^fb{\map@ucircumflex}%
\catcode252=\active  \def^^fc{\map@udieresis}%
\catcode253=\active  \def^^fd{\map@yacute}%
\catcode254=\active  \def^^fe{\map@thorn}%
\catcode255=\active  \def^^ff{\map@ydieresis}%
%    \end{macrocode}
% \Finale
\endinput
}
\DeclareOption{latin2}{\input{iso88592.map}}
\DeclareOption{ibm850}{% \iffalse metacomment
%   This file is part of the mapcodes package, version 1.04.
% -----------------------------------------------------------
% Copyright (C) 1994 Michael Piotrowski. All rights reserved.
%
% This file is distributed in the hope that it will be useful,
% but WITHOUT ANY WARRANTY; without even the implied warranty
% of MERCHANTABILITY or FITNESS FOR A PARTICULAR PURPOSE.
% -----------------------------------------------------------
%
% IMPORTANT NOTICE:
%
% For error reports in case of UNCHANGED versions see readme file.
%
% You are not allowed to change this file.
%
% You are allowed to distribute this file under the condition that
% it is distributed together with all files mentioned in manifest.txt.
%
% If you receive only some of these files from someone, complain!
%
% You are NOT ALLOWED to distribute this file alone. You are NOT
% ALLOWED to take money for the distribution or use of either this
% file or a changed version, except for a nominal charge for copying.
%
%
%<*driver>
\documentclass{article}
\usepackage{doc}
\DontCheckModules
\newcommand{\mapcodes}{\textsf{mapcodes}}
\begin{document}
 \DocInput{ibm850.dtx}
\end{document}
%</driver>
% \fi
%
% \def\filename{ibm850.dtx}
% \def\fileversion{v1.0}
% \def\filedate{1994/12/05}
% \CheckSum{511}
%
% \changes{ibm850-1.0}{1994/12/05}{First version}
%
% \section{IBM Codepage 850 Character Table}
%
%    This is \filename{} \fileversion{} of \filedate. It contains the
%    IBM Codepage 850 conversion table for \mapcodes.
% 
% \StopEventually{}
%
%    \begin{macrocode}
\typeout{Mapping IBM PC Codepage 850 to \encodingdefault\space encoding}

\catcode128=\active  \def^^80{\map@Ccedilla}%
\catcode129=\active  \def^^81{\map@udieresis}%
\catcode130=\active  \def^^82{\map@eacute}%
\catcode131=\active  \def^^83{\map@acircumflex}%
\catcode132=\active  \def^^84{\map@adieresis}%
\catcode133=\active  \def^^85{\map@agrave}%
\catcode134=\active  \def^^86{\map@aring}%
\catcode135=\active  \def^^87{\map@ccedilla}%
\catcode136=\active  \def^^88{\map@ecircumflex}%
\catcode137=\active  \def^^89{\map@edieresis}%
\catcode138=\active  \def^^8a{\map@egrave}%
\catcode139=\active  \def^^8b{\map@idieresis}%
\catcode140=\active  \def^^8c{\map@icircumflex}%
\catcode141=\active  \def^^8d{\map@igrave}%
\catcode142=\active  \def^^8e{\map@Adieresis}%
\catcode143=\active  \def^^8f{\map@Aring}%
\catcode144=\active  \def^^90{\map@Eacute}%
\catcode145=\active  \def^^91{\map@ae}%
\catcode146=\active  \def^^92{\map@AE}%
\catcode147=\active  \def^^93{\map@ocircumflex}%
\catcode148=\active  \def^^94{\map@odieresis}%
\catcode149=\active  \def^^95{\map@ograve}%
\catcode150=\active  \def^^96{\map@ucircumflex}%
\catcode151=\active  \def^^97{\map@ugrave}%
\catcode152=\active  \def^^98{\map@ydieresis}%
\catcode153=\active  \def^^99{\map@Odieresis}%
\catcode154=\active  \def^^9a{\map@Udieresis}%
\catcode155=\active  \def^^9b{\map@oslash}%
\catcode156=\active  \def^^9c{\map@sterling}%
\catcode157=\active  \def^^9d{\map@Oslash}%
\catcode158=\active  \def^^9e{\map@multiply}%
\catcode159=\active  \def^^9f{\map@florin}%
\catcode160=\active  \def^^a0{\map@aacute}%
\catcode161=\active  \def^^a1{\map@iacute}%
\catcode162=\active  \def^^a2{\map@oacute}%
\catcode163=\active  \def^^a3{\map@uacute}%
\catcode164=\active  \def^^a4{\map@ntilde}%
\catcode165=\active  \def^^a5{\map@Ntilde}%
\catcode166=\active  \def^^a6{\map@ordfeminine}%
\catcode167=\active  \def^^a7{\map@ordmasculine}%
\catcode168=\active  \def^^a8{\map@questiondown}%
\catcode169=\active  \def^^a9{\map@registered}%
\catcode170=\active  \def^^aa{\map@logicalnot}%
\catcode171=\active  \def^^ab{\map@onehalf}%
\catcode172=\active  \def^^ac{\map@onequarter}%
\catcode173=\active  \def^^ad{\map@exclamdown}%
\catcode174=\active  \def^^ae{\map@guillemotleft}%
\catcode175=\active  \def^^af{\map@guillemotright}%
\catcode176=\active  \def^^b0{\map@nil}%
\catcode177=\active  \def^^b1{\map@nil}%
\catcode178=\active  \def^^b2{\map@nil}%
\catcode179=\active  \def^^b3{\map@nil}%
\catcode180=\active  \def^^b4{\map@nil}%
\catcode181=\active  \def^^b5{\map@Aacute}%
\catcode182=\active  \def^^b6{\map@Acircumflex}%
\catcode183=\active  \def^^b7{\map@Agrave}%
\catcode184=\active  \def^^b8{\map@copyright}%
\catcode185=\active  \def^^b9{\map@nil}%
\catcode186=\active  \def^^ba{\map@nil}%
\catcode187=\active  \def^^bb{\map@nil}%
\catcode188=\active  \def^^bc{\map@nil}%
\catcode189=\active  \def^^bd{\map@cent}%
\catcode190=\active  \def^^be{\map@yen}%
\catcode191=\active  \def^^bf{\map@nil}%
\catcode192=\active  \def^^c0{\map@nil}%
\catcode193=\active  \def^^c1{\map@nil}%
\catcode194=\active  \def^^c2{\map@nil}%
\catcode195=\active  \def^^c3{\map@nil}%
\catcode196=\active  \def^^c4{\map@nil}%
\catcode197=\active  \def^^c5{\map@nil}%
\catcode198=\active  \def^^c6{\map@atilde}%
\catcode199=\active  \def^^c7{\map@Atilde}%
\catcode200=\active  \def^^c8{\map@nil}%
\catcode201=\active  \def^^c9{\map@nil}%
\catcode202=\active  \def^^ca{\map@nil}%
\catcode203=\active  \def^^cb{\map@nil}%
\catcode204=\active  \def^^cc{\map@nil}%
\catcode205=\active  \def^^cd{\map@nil}%
\catcode206=\active  \def^^ce{\map@nil}%
\catcode207=\active  \def^^cf{\map@currency}%
\catcode208=\active  \def^^d0{\map@eth}%
\catcode209=\active  \def^^d1{\map@Eth}%
\catcode210=\active  \def^^d2{\map@Ecircumflex}%
\catcode211=\active  \def^^d3{\map@Edieresis}%
\catcode212=\active  \def^^d4{\map@Egrave}%
\catcode213=\active  \def^^d5{\map@onesuperior}%
\catcode214=\active  \def^^d6{\map@Iacute}%
\catcode215=\active  \def^^d7{\map@Icircumflex}%
\catcode216=\active  \def^^d8{\map@Idieresis}%
\catcode217=\active  \def^^d9{\map@nil}%
\catcode218=\active  \def^^da{\map@nil}%
\catcode219=\active  \def^^db{\map@nil}%
\catcode220=\active  \def^^dc{\map@nil}%
\catcode221=\active  \def^^dd{\map@brokenbar}%
\catcode222=\active  \def^^de{\map@Igrave}%
\catcode223=\active  \def^^df{\map@nil}%
\catcode224=\active  \def^^e0{\map@Oacute}%
\catcode225=\active  \def^^e1{\map@germandbls}%
\catcode226=\active  \def^^e2{\map@Ocircumflex}%
\catcode227=\active  \def^^e3{\map@Ograve}%
\catcode228=\active  \def^^e4{\map@otilde}%
\catcode229=\active  \def^^e5{\map@Otilde}%
\catcode230=\active  \def^^e6{\map@mu}%
\catcode231=\active  \def^^e7{\map@thorn}%
\catcode232=\active  \def^^e8{\map@Thorn}%
\catcode233=\active  \def^^e9{\map@Uacute}%
\catcode234=\active  \def^^ea{\map@Ucircumflex}%
\catcode235=\active  \def^^eb{\map@Ugrave}%
\catcode236=\active  \def^^ec{\map@yacute}%
\catcode237=\active  \def^^ed{\map@Yacute}%
\catcode238=\active  \def^^ee{\map@macron}%
\catcode239=\active  \def^^ef{\map@acute}%
%\catcode240=\active  \def^^f0{\map@}%
\catcode241=\active  \def^^f1{\map@plusminus}%
%\catcode242=\active  \def^^f2{\map@}%
\catcode243=\active  \def^^f3{\map@threequarters}%
\catcode244=\active  \def^^f4{\map@paragraph}%
\catcode245=\active  \def^^f5{\map@section}%
\catcode246=\active  \def^^f6{\map@divide}%
\catcode247=\active  \def^^f7{\map@cedilla}%
\catcode248=\active  \def^^f8{\map@degree}%
\catcode249=\active  \def^^f9{\map@dieresis}%
\catcode250=\active  \def^^fa{\map@productdot}%
\catcode251=\active  \def^^fb{\map@onesuperior}%
\catcode252=\active  \def^^fc{\map@threesuperior}%
\catcode253=\active  \def^^fd{\map@twosuperior}%
\catcode254=\active  \def^^fe{\map@filledbox}%
%    \end{macrocode}
% \Finale
\endinput
}
\DeclareOption{ibm852}{% \iffalse metacomment
%   This file is part of the mapcodes package, version 1.04.
% -----------------------------------------------------------
% Copyright (C) 1994 Michael Piotrowski. All rights reserved.
%
% This file is distributed in the hope that it will be useful,
% but WITHOUT ANY WARRANTY; without even the implied warranty
% of MERCHANTABILITY or FITNESS FOR A PARTICULAR PURPOSE.
% -----------------------------------------------------------
%
% IMPORTANT NOTICE:
%
% For error reports in case of UNCHANGED versions see readme file.
%
% You are not allowed to change this file.
%
% You are allowed to distribute this file under the condition that
% it is distributed together with all files mentioned in manifest.txt.
%
% If you receive only some of these files from someone, complain!
%
% You are NOT ALLOWED to distribute this file alone. You are NOT
% ALLOWED to take money for the distribution or use of either this
% file or a changed version, except for a nominal charge for copying.
%
%
%<*driver>
\documentclass{article}
\usepackage{doc}
\DontCheckModules
\newcommand{\mapcodes}{\textsf{mapcodes}}
\begin{document}
 \DocInput{ibm852.dtx}
\end{document}
%</driver>
% \fi
%
% \def\filename{ibm852.dtx}
% \def\fileversion{v1.0}
% \def\filedate{1994/12/05}
% \CheckSum{511}
%
% \changes{ibm852-1.0}{1994/12/05}{First version}
%
% \section{IBM Codepage 852 Character Table}
%
%    This is \filename{} \fileversion{} of \filedate. It contains the
%    IBM Codepage 852 conversion table for \mapcodes.
% 
% \StopEventually{}
%
%    \begin{macrocode}
\typeout{Mapping IBM PC Codepage 852 to \encodingdefault\space encoding}

\catcode128=\active  \def^^80{\map@Ccedilla}%
\catcode129=\active  \def^^81{\map@udieresis}%
\catcode130=\active  \def^^82{\map@eacute}%
\catcode131=\active  \def^^83{\map@acircumflex}%
\catcode132=\active  \def^^84{\map@adieresis}%
\catcode133=\active  \def^^85{\map@uring}%
\catcode134=\active  \def^^86{\map@cacute}%
\catcode135=\active  \def^^87{\map@ccedilla}%
\catcode136=\active  \def^^88{\map@lslash}%
\catcode137=\active  \def^^89{\map@edieresis}%
\catcode138=\active  \def^^8a{\map@Otilde}%
\catcode139=\active  \def^^8b{\map@otilde}%
\catcode140=\active  \def^^8c{\map@icircumflex}%
\catcode141=\active  \def^^8d{\map@Zacute}%
\catcode142=\active  \def^^8e{\map@Adieresis}%
\catcode143=\active  \def^^8f{\map@Cacute}%
\catcode144=\active  \def^^90{\map@Eacute}%
\catcode145=\active  \def^^91{\map@Lacute}%
\catcode146=\active  \def^^92{\map@lacute}%
\catcode147=\active  \def^^93{\map@ocircumflex}%
\catcode148=\active  \def^^94{\map@odieresis}%
\catcode149=\active  \def^^95{\map@Lcaron}%
\catcode150=\active  \def^^96{\map@lcaron}%
\catcode151=\active  \def^^97{\map@Sacute}%
\catcode152=\active  \def^^98{\map@sacute}%
\catcode153=\active  \def^^99{\map@Odieresis}%
\catcode154=\active  \def^^9a{\map@Udieresis}%
\catcode155=\active  \def^^9b{\map@Tcaron}%
\catcode156=\active  \def^^9c{\map@tcaron}%
\catcode157=\active  \def^^9d{\map@Lslash}%
\catcode158=\active  \def^^9e{\map@multiply}%
\catcode159=\active  \def^^9f{\map@ccaron}%
\catcode160=\active  \def^^a0{\map@aacute}%
\catcode161=\active  \def^^a1{\map@iacute}%
\catcode162=\active  \def^^a2{\map@oacute}%
\catcode163=\active  \def^^a3{\map@uacute}%
\catcode164=\active  \def^^a4{\map@Aogonek}%
\catcode165=\active  \def^^a5{\map@aogonek}%
\catcode166=\active  \def^^a6{\map@Zcaron}%
\catcode167=\active  \def^^a7{\map@zcaron}%
\catcode168=\active  \def^^a8{\map@Eogonek}%
\catcode169=\active  \def^^a9{\map@eogonek}%
\catcode170=\active  \def^^aa{\map@logicalnot}%
\catcode171=\active  \def^^ab{\map@zacute}%
\catcode172=\active  \def^^ac{\map@Ccaron}%
\catcode173=\active  \def^^ad{\map@scedilla}%
\catcode174=\active  \def^^ae{\map@guillemotleft}%
\catcode175=\active  \def^^af{\map@guillemotright}%
\catcode176=\active  \def^^b0{\map@nil}%
\catcode177=\active  \def^^b1{\map@nil}%
\catcode178=\active  \def^^b2{\map@nil}%
\catcode179=\active  \def^^b3{\map@nil}%
\catcode180=\active  \def^^b4{\map@nil}%
\catcode181=\active  \def^^b5{\map@Aacute}%
\catcode182=\active  \def^^b6{\map@Acircumflex}%
\catcode183=\active  \def^^b7{\map@Ecaron}%
\catcode184=\active  \def^^b8{\map@Scedilla}%
\catcode185=\active  \def^^b9{\map@nil}%
\catcode186=\active  \def^^ba{\map@nil}%
\catcode187=\active  \def^^bb{\map@nil}%
\catcode188=\active  \def^^bc{\map@nil}%
\catcode189=\active  \def^^bd{\map@Zdotaccent}%
\catcode190=\active  \def^^be{\map@zdotaccent}%
\catcode191=\active  \def^^bf{\map@nil}%
\catcode192=\active  \def^^c0{\map@nil}%
\catcode193=\active  \def^^c1{\map@nil}%
\catcode194=\active  \def^^c2{\map@nil}%
\catcode195=\active  \def^^c3{\map@nil}%
\catcode196=\active  \def^^c4{\map@nil}%
\catcode197=\active  \def^^c5{\map@nil}%
\catcode198=\active  \def^^c6{\map@Abreve}%
\catcode199=\active  \def^^c7{\map@abreve}%
\catcode200=\active  \def^^c8{\map@nil}%
\catcode201=\active  \def^^c9{\map@nil}%
\catcode202=\active  \def^^ca{\map@nil}%
\catcode203=\active  \def^^cb{\map@nil}%
\catcode204=\active  \def^^cc{\map@nil}%
\catcode205=\active  \def^^cd{\map@nil}%
\catcode206=\active  \def^^ce{\map@nil}%
\catcode207=\active  \def^^cf{\map@currency}%
\catcode208=\active  \def^^d0{\map@dslash}%
\catcode209=\active  \def^^d1{\map@Dslash}%
\catcode210=\active  \def^^d2{\map@Dcaron}%
\catcode211=\active  \def^^d3{\map@Edieresis}%
\catcode212=\active  \def^^d4{\map@dcaron}%
\catcode213=\active  \def^^d5{\map@Ncaron}%
\catcode214=\active  \def^^d6{\map@Iacute}%
\catcode215=\active  \def^^d7{\map@Icircumflex}%
\catcode216=\active  \def^^d8{\map@ecaron}%
\catcode217=\active  \def^^d9{\map@nil}%
\catcode218=\active  \def^^da{\map@nil}%
\catcode219=\active  \def^^db{\map@nil}%
\catcode220=\active  \def^^dc{\map@nil}%
\catcode221=\active  \def^^dd{\map@Tcedilla}%
\catcode222=\active  \def^^de{\map@Uring}%
\catcode223=\active  \def^^df{\map@nil}%
\catcode224=\active  \def^^e0{\map@Oacute}%
\catcode225=\active  \def^^e1{\map@germandbls}%
\catcode226=\active  \def^^e2{\map@Ocircumflex}%
\catcode227=\active  \def^^e3{\map@Nacute}%
\catcode228=\active  \def^^e4{\map@nacute}%
\catcode229=\active  \def^^e5{\map@ncaron}%
\catcode230=\active  \def^^e6{\map@Scaron}%
\catcode231=\active  \def^^e7{\map@scaron}%
\catcode232=\active  \def^^e8{\map@Racute}%
\catcode233=\active  \def^^e9{\map@Uacute}%
\catcode234=\active  \def^^ea{\map@racute}%
\catcode235=\active  \def^^eb{\map@Uhungarumlaut}%
\catcode236=\active  \def^^ec{\map@yacute}%
\catcode237=\active  \def^^ed{\map@Yacute}%
\catcode238=\active  \def^^ee{\map@tcedilla}%
\catcode239=\active  \def^^ef{\map@acute}%
%\catcode240=\active  \def^^f0{\map@}%
\catcode241=\active  \def^^f1{\map@hungarumlaut}%
\catcode242=\active  \def^^f2{\map@ogonek}%
\catcode243=\active  \def^^f3{\map@caron}%
\catcode244=\active  \def^^f4{\map@breve}%
\catcode245=\active  \def^^f5{\map@section}%
\catcode246=\active  \def^^f6{\map@divide}%
\catcode247=\active  \def^^f7{\map@cedilla}%
\catcode248=\active  \def^^f8{\map@degree}%
\catcode249=\active  \def^^f9{\map@dieresis}%
\catcode250=\active  \def^^fa{\map@productdot}%
\catcode251=\active  \def^^fb{\map@uhungarumlaut}%
\catcode252=\active  \def^^fc{\map@Rcaron}%
\catcode253=\active  \def^^fd{\map@rcaron}%
\catcode254=\active  \def^^fe{\map@filledbox}%
%    \end{macrocode}
% \Finale
\endinput
}
\DeclareOption{atari}{% \iffalse metacomment
%   This file is part of the mapcodes package, version 1.04.
% -----------------------------------------------------------
% Copyright (C) 1994 Michael Piotrowski. All rights reserved.
%
% This file is distributed in the hope that it will be useful,
% but WITHOUT ANY WARRANTY; without even the implied warranty
% of MERCHANTABILITY or FITNESS FOR A PARTICULAR PURPOSE.
% -----------------------------------------------------------
%
% IMPORTANT NOTICE:
%
% For error reports in case of UNCHANGED versions see readme file.
%
% You are not allowed to change this file.
%
% You are allowed to distribute this file under the condition that
% it is distributed together with all files mentioned in manifest.txt.
%
% If you receive only some of these files from someone, complain!
%
% You are NOT ALLOWED to distribute this file alone. You are NOT
% ALLOWED to take money for the distribution or use of either this
% file or a changed version, except for a nominal charge for copying.
%
%
%<*driver>
\documentclass{article}
\usepackage{doc}
\DontCheckModules
\newcommand{\mapcodes}{\textsf{mapcodes}}
\begin{document}
 \DocInput{atari.dtx}
\end{document}
%</driver>
% \fi
%
% \def\filename{atari.dtx}
% \def\fileversion{v1.0}
% \def\filedate{1994/12/05}
% \CheckSum{515}
%
% \changes{atari-1.0}{1994/12/05}{First version}
%
% \section{Atari TOS Character Table}
%
%    This is \filename{} \fileversion{} of \filedate. It contains the Atari
%    conversion table for \mapcodes.
% 
% \StopEventually{}
%
%    \begin{macrocode}
%
\typeout{Mapping Atari to \encodingdefault\space encoding}

\catcode128=\active  \def^^80{\map@Ccedilla}%
\catcode129=\active  \def^^81{\map@udieresis}%
\catcode130=\active  \def^^82{\map@eacute}%
\catcode131=\active  \def^^83{\map@acircumflex}%
\catcode132=\active  \def^^84{\map@adieresis}%
\catcode133=\active  \def^^85{\map@agrave}%
\catcode134=\active  \def^^86{\map@aring}%
\catcode135=\active  \def^^87{\map@ccedilla}%
\catcode136=\active  \def^^88{\map@ecircumflex}%
\catcode137=\active  \def^^89{\map@edieresis}%
\catcode138=\active  \def^^8a{\map@egrave}%
\catcode139=\active  \def^^8b{\map@idieresis}%
\catcode140=\active  \def^^8c{\map@icircumflex}%
\catcode141=\active  \def^^8d{\map@igrave}%
\catcode142=\active  \def^^8e{\map@Adieresis}%
\catcode143=\active  \def^^8f{\map@Aring}%
\catcode144=\active  \def^^90{\map@Eacute}%
\catcode145=\active  \def^^91{\map@ae}%
\catcode146=\active  \def^^92{\map@AE}%
\catcode147=\active  \def^^93{\map@ocircumflex}%
\catcode148=\active  \def^^94{\map@odieresis}%
\catcode149=\active  \def^^95{\map@ograve}%
\catcode150=\active  \def^^96{\map@ucircumflex}%
\catcode151=\active  \def^^97{\map@ugrave}%
\catcode152=\active  \def^^98{\map@ydieresis}%
\catcode153=\active  \def^^99{\map@Odieresis}%
\catcode154=\active  \def^^9a{\map@Udieresis}%
\catcode155=\active  \def^^9b{\map@cent}%
\catcode156=\active  \def^^9c{\map@sterling}%
\catcode157=\active  \def^^9d{\map@yen}%
\catcode158=\active  \def^^9e{\map@germandbls}%
\catcode159=\active  \def^^9f{\map@florin}%
\catcode160=\active  \def^^a0{\map@aacute}%
\catcode161=\active  \def^^a1{\map@iacute}%
\catcode162=\active  \def^^a2{\map@oacute}%
\catcode163=\active  \def^^a3{\map@uacute}%
\catcode164=\active  \def^^a4{\map@ntilde}%
\catcode165=\active  \def^^a5{\map@Ntilde}%
\catcode166=\active  \def^^a6{\map@ordfeminine}%
\catcode167=\active  \def^^a7{\map@ordmasculine}%
\catcode168=\active  \def^^a8{\map@questiondown}%
%\catcode169=\active  \def^^a9{\map@}%
\catcode170=\active  \def^^aa{\map@logicalnot}%
\catcode171=\active  \def^^ab{\map@onehalf}%
\catcode172=\active  \def^^ac{\map@onequarter}%
\catcode173=\active  \def^^ad{\map@exclamdown}%
\catcode174=\active  \def^^ae{\map@guillemotleft}%
\catcode175=\active  \def^^af{\map@guillemotright}%
\catcode176=\active  \def^^b0{\map@atilde}%
\catcode177=\active  \def^^b1{\map@otilde}%
\catcode178=\active  \def^^b2{\map@Oslash}%
\catcode179=\active  \def^^b3{\map@oslash}%
\catcode180=\active  \def^^b4{\map@oe}%
\catcode181=\active  \def^^b5{\map@OE}%
\catcode182=\active  \def^^b6{\map@Agrave}%
\catcode183=\active  \def^^b7{\map@Atilde}%
\catcode184=\active  \def^^b8{\map@Otilde}%
\catcode185=\active  \def^^b9{\map@dieresis}%
\catcode186=\active  \def^^ba{\map@acute}%
\catcode187=\active  \def^^bb{\map@dagger}%
\catcode188=\active  \def^^bc{\map@paragraph}%
\catcode189=\active  \def^^bd{\map@copyright}%
\catcode190=\active  \def^^be{\map@registered}%
\catcode191=\active  \def^^bf{\map@trademark}%
\catcode192=\active  \def^^c0{\map@ij}%
\catcode193=\active  \def^^c1{\map@IJ}%
\catcode194=\active  \def^^c2{\map@aleph}%
\catcode195=\active  \def^^c3{\map@nil}%
\catcode196=\active  \def^^c4{\map@nil}%
\catcode197=\active  \def^^c5{\map@nil}%
\catcode198=\active  \def^^c6{\map@nil}%
\catcode199=\active  \def^^c7{\map@nil}%
\catcode200=\active  \def^^c8{\map@nil}%
\catcode201=\active  \def^^c9{\map@nil}%
\catcode202=\active  \def^^ca{\map@nil}%
\catcode203=\active  \def^^cb{\map@nil}%
\catcode204=\active  \def^^cc{\map@nil}%
\catcode205=\active  \def^^cd{\map@nil}%
\catcode206=\active  \def^^ce{\map@nil}%
\catcode207=\active  \def^^cf{\map@nil}%
\catcode208=\active  \def^^d0{\map@nil}%
\catcode209=\active  \def^^d1{\map@nil}%
\catcode210=\active  \def^^d2{\map@nil}%
\catcode211=\active  \def^^d3{\map@nil}%
\catcode212=\active  \def^^d4{\map@nil}%
\catcode213=\active  \def^^d5{\map@nil}%
\catcode214=\active  \def^^d6{\map@nil}%
\catcode215=\active  \def^^d7{\map@nil}%
\catcode216=\active  \def^^d8{\map@nil}%
\catcode217=\active  \def^^d9{\map@nil}%
\catcode218=\active  \def^^da{\map@nil}%
\catcode219=\active  \def^^db{\map@nil}%
\catcode220=\active  \def^^dc{\map@nil}%
\catcode221=\active  \def^^dd{\map@section}%
\catcode222=\active  \def^^de{\map@logicaland}%
\catcode223=\active  \def^^df{\map@infinity}%
\catcode224=\active  \def^^e0{\map@alpha}%
\catcode225=\active  \def^^e1{\map@beta}%
\catcode226=\active  \def^^e2{\map@Gamma}%
\catcode227=\active  \def^^e3{\map@pi}%
\catcode228=\active  \def^^e4{\map@Sigma}%
\catcode229=\active  \def^^e5{\map@sigma}%
\catcode230=\active  \def^^e6{\map@mu}%
\catcode231=\active  \def^^e7{\map@tau}%
\catcode232=\active  \def^^e8{\map@Phi}%
\catcode233=\active  \def^^e9{\map@Theta}%
\catcode234=\active  \def^^ea{\map@Omega}%
\catcode235=\active  \def^^eb{\map@delta}%
%\catcode236=\active  \def^^ec{\map@}%
\catcode237=\active  \def^^ed{\map@phi}%
\catcode238=\active  \def^^ee{\map@epsilon}%
\catcode239=\active  \def^^ef{\map@intersection}%
\catcode240=\active  \def^^f0{\map@equivalence}%
\catcode241=\active  \def^^f1{\map@plusminus}%
\catcode242=\active  \def^^f2{\map@greaterequal}%
\catcode243=\active  \def^^f3{\map@lessequal}%
\catcode244=\active  \def^^f4{\map@nil}%
\catcode245=\active  \def^^f5{\map@nil}%
\catcode246=\active  \def^^f6{\map@divide}%
\catcode247=\active  \def^^f7{\map@approxequal}%
\catcode248=\active  \def^^f8{\map@degree}%
\catcode249=\active  \def^^f9{\map@bullet}%
\catcode250=\active  \def^^fa{\map@productdot}%
\catcode251=\active  \def^^fb{\map@radical}%
\catcode252=\active  \def^^fc{\map@nsuperior}%
\catcode253=\active  \def^^fd{\map@twosuperior}%
\catcode254=\active  \def^^fe{\map@threesuperior}%
\catcode255=\active  \def^^ff{\map@macron}%
%
%    \end{macrocode}
% \Finale
\endinput}
\DeclareOption{hproman8}{% \iffalse metacomment
%   This file is part of the mapcodes package, version 1.04.
% -----------------------------------------------------------
% Copyright (C) 1994 Michael Piotrowski. All rights reserved.
%
% This file is distributed in the hope that it will be useful,
% but WITHOUT ANY WARRANTY; without even the implied warranty
% of MERCHANTABILITY or FITNESS FOR A PARTICULAR PURPOSE.
% -----------------------------------------------------------
%
% IMPORTANT NOTICE:
%
% For error reports in case of UNCHANGED versions see readme file.
%
% You are not allowed to change this file.
%
% You are allowed to distribute this file under the condition that
% it is distributed together with all files mentioned in manifest.txt.
%
% If you receive only some of these files from someone, complain!
%
% You are NOT ALLOWED to distribute this file alone. You are NOT
% ALLOWED to take money for the distribution or use of either this
% file or a changed version, except for a nominal charge for copying.
%
%
%<*driver>
\documentclass{article}
\usepackage{doc}
\DontCheckModules
\newcommand{\mapcodes}{\textsf{mapcodes}}
\begin{document}
 \DocInput{hproman8.dtx}
\end{document}
%</driver>
% \fi
%
% \def\filename{hproman8.dtx}
% \def\fileversion{v1.0}
% \def\filedate{1994/12/05}
% \CheckSum{379}
%
% \changes{hproman8-1.0}{1994/12/05}{First version}
%
% \section{HP Roman 8 Character Table}
%
%    This is \filename{} \fileversion{} of \filedate. It contains the HP
%    Roman 8 conversion table for \mapcodes.
% 
% \StopEventually{}
%
%    \begin{macrocode}
\typeout{Mapping HP Roman 8 to \encodingdefault\space encoding}

\catcode161=\active  \def^^a1{\map@Agrave}%
\catcode162=\active  \def^^a2{\map@Acircumflex}%
\catcode163=\active  \def^^a3{\map@Egrave}%
\catcode164=\active  \def^^a4{\map@Ecircumflex}%
\catcode165=\active  \def^^a5{\map@Edieresis}%
\catcode166=\active  \def^^a6{\map@Icircumflex}%
\catcode167=\active  \def^^a7{\map@Idieresis}%
\catcode168=\active  \def^^a8{\map@acute}%
\catcode169=\active  \def^^a9{\map@grave}%
\catcode170=\active  \def^^aa{\map@circumflex}%
\catcode171=\active  \def^^ab{\map@dieresis}%
\catcode172=\active  \def^^ac{\map@tilde}%
\catcode173=\active  \def^^ad{\map@Ugrave}%
\catcode174=\active  \def^^ae{\map@Ucircumflex}%
\catcode175=\active  \def^^af{\map@sterling}%
\catcode176=\active  \def^^b0{\map@macron}%
\catcode177=\active  \def^^b1{\map@Yacute}%
\catcode178=\active  \def^^b2{\map@yacute}%
\catcode179=\active  \def^^b3{\map@degree}%
\catcode180=\active  \def^^b4{\map@Ccedilla}%
\catcode181=\active  \def^^b5{\map@ccedilla}%
\catcode182=\active  \def^^b6{\map@Ntilde}%
\catcode183=\active  \def^^b7{\map@ntilde}%
\catcode184=\active  \def^^b8{\map@exclamdown}%
\catcode185=\active  \def^^b9{\map@questiondown}%
\catcode186=\active  \def^^ba{\map@currency}%
\catcode187=\active  \def^^bb{\map@sterling}%
\catcode188=\active  \def^^bc{\map@yen}%
\catcode189=\active  \def^^bd{\map@section}%
\catcode190=\active  \def^^be{\map@florin}%
\catcode191=\active  \def^^bf{\map@cent}%
\catcode192=\active  \def^^c0{\map@acircumflex}%
\catcode193=\active  \def^^c1{\map@ecircumflex}%
\catcode194=\active  \def^^c2{\map@ocircumflex}%
\catcode195=\active  \def^^c3{\map@ucircumflex}%
\catcode196=\active  \def^^c4{\map@aacute}%
\catcode197=\active  \def^^c5{\map@eacute}%
\catcode198=\active  \def^^c6{\map@oacute}%
\catcode199=\active  \def^^c7{\map@uacute}%
\catcode200=\active  \def^^c8{\map@agrave}%
\catcode201=\active  \def^^c9{\map@egrave}%
\catcode202=\active  \def^^ca{\map@ograve}%
\catcode203=\active  \def^^cb{\map@ugrave}%
\catcode204=\active  \def^^cc{\map@adieresis}%
\catcode205=\active  \def^^cd{\map@edieresis}%
\catcode206=\active  \def^^ce{\map@odieresis}%
\catcode207=\active  \def^^cf{\map@udieresis}%
\catcode208=\active  \def^^d0{\map@Aring}%
\catcode209=\active  \def^^d1{\map@icircumflex}%
\catcode210=\active  \def^^d2{\map@Oslash}%
\catcode211=\active  \def^^d3{\map@AE}%
\catcode212=\active  \def^^d4{\map@aring}%
\catcode213=\active  \def^^d5{\map@iacute}%
\catcode214=\active  \def^^d6{\map@oslash}%
\catcode215=\active  \def^^d7{\map@ae}%
\catcode216=\active  \def^^d8{\map@Adieresis}%
\catcode217=\active  \def^^d9{\map@igrave}%
\catcode218=\active  \def^^da{\map@Odieresis}%
\catcode219=\active  \def^^db{\map@Udieresis}%
\catcode220=\active  \def^^dc{\map@Eacute}%
\catcode221=\active  \def^^dd{\map@idieresis}%
\catcode222=\active  \def^^de{\map@germandbls}%
\catcode223=\active  \def^^df{\map@Ocircumflex}%
\catcode224=\active  \def^^e0{\map@Aacute}%
\catcode225=\active  \def^^e1{\map@Atilde}%
\catcode226=\active  \def^^e2{\map@atilde}%
\catcode227=\active  \def^^e3{\map@Eth}%
\catcode228=\active  \def^^e4{\map@eth}%
\catcode229=\active  \def^^e5{\map@Iacute}%
\catcode230=\active  \def^^e6{\map@Igrave}%
\catcode231=\active  \def^^e7{\map@Oacute}%
\catcode232=\active  \def^^e8{\map@Ograve}%
\catcode233=\active  \def^^e9{\map@Otilde}%
\catcode234=\active  \def^^ea{\map@otilde}%
\catcode235=\active  \def^^eb{\map@Scaron}%
\catcode236=\active  \def^^ec{\map@scaron}%
\catcode237=\active  \def^^ed{\map@Uacute}%
\catcode238=\active  \def^^ee{\map@Ydieresis}%
\catcode239=\active  \def^^ef{\map@ydieresis}%
\catcode240=\active  \def^^f0{\map@Thorn}%
\catcode241=\active  \def^^f1{\map@thorn}%
\catcode242=\active  \def^^f2{\map@periodcentered}%
\catcode243=\active  \def^^f3{\map@mu}%
\catcode244=\active  \def^^f4{\map@paragraph}%
\catcode245=\active  \def^^f5{\map@threequarters}%
\catcode246=\active  \def^^f6{\map@emdash}%
\catcode247=\active  \def^^f7{\map@onequarter}%
\catcode248=\active  \def^^f8{\map@onehalf}%
\catcode249=\active  \def^^f9{\map@ordfeminine}%
\catcode250=\active  \def^^fa{\map@ordmasculine}%
\catcode251=\active  \def^^fb{\map@guillemotleft}%
\catcode252=\active  \def^^fc{\map@filledbox}%
\catcode253=\active  \def^^fd{\map@guillemotright}%
\catcode254=\active  \def^^fe{\map@plusminus}%
%    \end{macrocode}
% \Finale    
\endinput}
\DeclareOption{macroman}{% \iffalse metacomment
%   This file is part of the mapcodes package, version 1.04.
% -----------------------------------------------------------
% Copyright (C) 1994 Michael Piotrowski. All rights reserved.
%
% This file is distributed in the hope that it will be useful,
% but WITHOUT ANY WARRANTY; without even the implied warranty
% of MERCHANTABILITY or FITNESS FOR A PARTICULAR PURPOSE.
% -----------------------------------------------------------
%
% IMPORTANT NOTICE:
%
% For error reports in case of UNCHANGED versions see readme file.
%
% You are not allowed to change this file.
%
% You are allowed to distribute this file under the condition that
% it is distributed together with all files mentioned in manifest.txt.
%
% If you receive only some of these files from someone, complain!
%
% You are NOT ALLOWED to distribute this file alone. You are NOT
% ALLOWED to take money for the distribution or use of either this
% file or a changed version, except for a nominal charge for copying.
%
%
%<*driver>
\documentclass{article}
\usepackage{doc}
\DontCheckModules
\newcommand{\mapcodes}{\textsf{mapcodes}}
\begin{document}
 \DocInput{macroman.dtx}
\end{document}
%</driver>
% \fi
%
% \def\filename{macroman.dtx}
% \def\fileversion{v1.0}
% \def\filedate{1994/12/05}
% \CheckSum{515}
%
% \changes{macroman-1.0}{1994/12/05}{First version}
%
% \section{Macintosh Roman Character Table}
%
%    This is \filename{} \fileversion{} of \filedate. It contains the 
%    Macintosh Roman conversion table for \mapcodes.
% 
% \StopEventually{}
%
%    \begin{macrocode}
\typeout{Mapping Macintosh Roman to \encodingdefault\space encoding}

\catcode128=\active  \def^^80{\map@Adieresis}%
\catcode129=\active  \def^^81{\map@Aring}%
\catcode130=\active  \def^^82{\map@Ccedilla}%
\catcode131=\active  \def^^83{\map@Eacute}%
\catcode132=\active  \def^^84{\map@Ntilde}%
\catcode133=\active  \def^^85{\map@Odieresis}%
\catcode134=\active  \def^^86{\map@Udieresis}%
\catcode135=\active  \def^^87{\map@aacute}%
\catcode136=\active  \def^^88{\map@agrave}%
\catcode137=\active  \def^^89{\map@acircumflex}%
\catcode138=\active  \def^^8a{\map@adieresis}%
\catcode139=\active  \def^^8b{\map@atilde}%
\catcode140=\active  \def^^8c{\map@aring}%
\catcode141=\active  \def^^8d{\map@ccedilla}%
\catcode142=\active  \def^^8e{\map@eacute}%
\catcode143=\active  \def^^8f{\map@egrave}%
\catcode144=\active  \def^^90{\map@ecircumflex}%
\catcode145=\active  \def^^91{\map@edieresis}%
\catcode146=\active  \def^^92{\map@iacute}%
\catcode147=\active  \def^^93{\map@igrave}%
\catcode148=\active  \def^^94{\map@icircumflex}%
\catcode149=\active  \def^^95{\map@idieresis}%
\catcode150=\active  \def^^96{\map@ntilde}%
\catcode151=\active  \def^^97{\map@oacute}%
\catcode152=\active  \def^^98{\map@ograve}%
\catcode153=\active  \def^^99{\map@ocircumflex}%
\catcode154=\active  \def^^9a{\map@odieresis}%
\catcode155=\active  \def^^9b{\map@otilde}%
\catcode156=\active  \def^^9c{\map@uacute}%
\catcode157=\active  \def^^9d{\map@ugrave}%
\catcode158=\active  \def^^9e{\map@ucircumflex}%
\catcode159=\active  \def^^9f{\map@udieresis}%
\catcode160=\active  \def^^a0{\map@dagger}%
\catcode161=\active  \def^^a1{\map@degree}%
\catcode162=\active  \def^^a2{\map@cent}%
\catcode163=\active  \def^^a3{\map@sterling}%
\catcode164=\active  \def^^a4{\map@section}%
\catcode165=\active  \def^^a5{\map@productdot}%
\catcode166=\active  \def^^a6{\map@paragraph}%
\catcode167=\active  \def^^a7{\map@germandbls}%
\catcode168=\active  \def^^a8{\map@registered}%
\catcode169=\active  \def^^a9{\map@copyright}%
\catcode170=\active  \def^^aa{\map@trademark}%
\catcode171=\active  \def^^ab{\map@acute}%
\catcode172=\active  \def^^ac{\map@dieresis}%
\catcode173=\active  \def^^ad{\map@notequal}%
\catcode174=\active  \def^^ae{\map@AE}%
\catcode175=\active  \def^^af{\map@Oslash}%
\catcode176=\active  \def^^b0{\map@infinity}%
\catcode177=\active  \def^^b1{\map@plusminus}%
\catcode178=\active  \def^^b2{\map@lessequal}%
\catcode179=\active  \def^^b3{\map@greaterequal}%
\catcode180=\active  \def^^b4{\map@yen}%
\catcode181=\active  \def^^b5{\map@mu}%
\catcode182=\active  \def^^b6{\map@partialdiff}%
\catcode183=\active  \def^^b7{\map@summation}%
\catcode184=\active  \def^^b8{\map@product}%
\catcode185=\active  \def^^b9{\map@pi}%
\catcode186=\active  \def^^ba{\map@integral}%
\catcode187=\active  \def^^bb{\map@ordfeminine}%
\catcode188=\active  \def^^bc{\map@ordmasculine}%
\catcode189=\active  \def^^bd{\map@Omega}%
\catcode190=\active  \def^^be{\map@ae}%
\catcode191=\active  \def^^bf{\map@oslash}%
\catcode192=\active  \def^^c0{\map@questiondown}%
\catcode193=\active  \def^^c1{\map@exclamdown}%
\catcode194=\active  \def^^c2{\map@logicalnot}%
\catcode195=\active  \def^^c3{\map@radical}%
\catcode196=\active  \def^^c4{\map@florin}%
\catcode197=\active  \def^^c5{\map@approxequal}%
\catcode198=\active  \def^^c6{\map@Delta}%
\catcode199=\active  \def^^c7{\map@guillemotleft}%
\catcode200=\active  \def^^c8{\map@guillemotright}%
\catcode201=\active  \def^^c9{\map@cdots}% not a Postscript name!
\catcode202=\active  \def^^ca{\map@space}%
\catcode203=\active  \def^^cb{\map@Agrave}%
\catcode204=\active  \def^^cc{\map@Atilde}%
\catcode205=\active  \def^^cd{\map@Otilde}%
\catcode206=\active  \def^^ce{\map@OE}%
\catcode207=\active  \def^^cf{\map@oe}%
\catcode208=\active  \def^^d0{\map@endash}%
\catcode209=\active  \def^^d1{\map@emdash}%
\catcode210=\active  \def^^d2{\map@quotedblleft}%
\catcode211=\active  \def^^d3{\map@quotedblright}%
\catcode212=\active  \def^^d4{\map@quoteleft}%
\catcode213=\active  \def^^d5{\map@quoteright}%
\catcode214=\active  \def^^d6{\map@divide}%
\catcode215=\active  \def^^d7{\map@lozenge}%
\catcode216=\active  \def^^d8{\map@ydieresis}%
\catcode217=\active  \def^^d9{\map@Ydieresis}%
\catcode218=\active  \def^^da{\map@fraction}%
\catcode219=\active  \def^^db{\map@currency}%
\catcode220=\active  \def^^dc{\map@guilsinglleft}%
\catcode221=\active  \def^^dd{\map@guilsinglright}%
\catcode222=\active  \def^^de{\map@fi}%
\catcode223=\active  \def^^df{\map@fl}%
\catcode224=\active  \def^^e0{\map@daggerdbl}%
\catcode225=\active  \def^^e1{\map@periodcentered}%
\catcode226=\active  \def^^e2{\map@quotesinglbase}%
\catcode227=\active  \def^^e3{\map@quotedblbase}%
\catcode228=\active  \def^^e4{\map@perthousand}%
\catcode229=\active  \def^^e5{\map@Acircumflex}%
\catcode230=\active  \def^^e6{\map@Ecircumflex}%
\catcode231=\active  \def^^e7{\map@Aacute}%
\catcode232=\active  \def^^e8{\map@Edieresis}%
\catcode233=\active  \def^^e9{\map@Egrave}%
\catcode234=\active  \def^^ea{\map@Iacute}%
\catcode235=\active  \def^^eb{\map@Icircumflex}%
\catcode236=\active  \def^^ec{\map@Idieresis}%
\catcode237=\active  \def^^ed{\map@Igrave}%
\catcode238=\active  \def^^ee{\map@Oacute}%
\catcode239=\active  \def^^ef{\map@Ocircumflex}%
\catcode240=\active  \def^^f0{\map@nil}%
\catcode241=\active  \def^^f1{\map@Ograve}%
\catcode242=\active  \def^^f2{\map@Uacute}%
\catcode243=\active  \def^^f3{\map@Ucircumflex}%
\catcode244=\active  \def^^f4{\map@Ugrave}%
\catcode245=\active  \def^^f5{\map@dotlessi}%
\catcode246=\active  \def^^f6{\map@circumflex}%
\catcode247=\active  \def^^f7{\map@tilde}%
\catcode248=\active  \def^^f8{\map@macron}%
\catcode249=\active  \def^^f9{\map@breve}%
\catcode250=\active  \def^^fa{\map@dotaccent}%
\catcode251=\active  \def^^fb{\map@ring}%
\catcode252=\active  \def^^fc{\map@cedilla}%
\catcode253=\active  \def^^fd{\map@hungarumlaut}%
\catcode254=\active  \def^^fe{\map@ogonek}%
\catcode255=\active  \def^^ff{\map@caron}%
%    \end{macrocode}
% \Finale
\endinput}
%    \end{macrocode}
%
% If the specified option is not listed above we look for a
% file of this name. Note that we don't care if it doesn't
% exist either (should be changed).
%    \begin{macrocode}
\DeclareOption*{\InputIfFileExists{\CurrentOption.map}{}{}}
%    \end{macrocode}
%
% The language options are read last.
%    \begin{macrocode}
\DeclareOption{german}{% \iffalse metacomment
%   This file is part of the mapcodes package, version 1.04.
% -----------------------------------------------------------
% Copyright (C) 1994 Michael Piotrowski. All rights reserved.
%
% This file is distributed in the hope that it will be useful,
% but WITHOUT ANY WARRANTY; without even the implied warranty
% of MERCHANTABILITY or FITNESS FOR A PARTICULAR PURPOSE.
% -----------------------------------------------------------
%
% IMPORTANT NOTICE:
%
% For error reports in case of UNCHANGED versions see readme file.
%
% You are not allowed to change this file.
%
% You are allowed to distribute this file under the condition that
% it is distributed together with all files mentioned in manifest.txt.
%
% If you receive only some of these files from someone, complain!
%
% You are NOT ALLOWED to distribute this file alone. You are NOT
% ALLOWED to take money for the distribution or use of either this
% file or a changed version, except for a nominal charge for copying.
%
%
%<*driver>
\documentclass{article}
\usepackage{doc}
\DontCheckModules
\newcommand{\mapcodes}{\textsf{mapcodes}}
\begin{document}
 \DocInput{german.dtx}
\end{document}
%</driver>
% \fi
%
% \def\filename{german.dtx}
% \def\fileversion{v1.1}
% \def\filedate{1995/04/07}
% \CheckSum{44}
%
% \changes{german-1.0}{1994/12/05}{First version}
% \changes{german-1.1}{1995/04/07}{Support for quotation marks added}
%
% \section{\mapcodes{} Support for German Extensions}
%
%    This is \filename{} \fileversion{} of \filedate. It contains the 
%    \mapcodes support for the \textsf{german} package and the
%    \texttt{german} option of \textsf{babel}.
% 
% \StopEventually{}
%
%    \begin{macrocode}
\typeout{Compatibility with German OT1 extensions requested...}

\ifx\encodingdefault\map@oldenc
    \typeout{enabled.}
    \renewcommand\map@Adieresis{"A}
    \renewcommand\map@Odieresis{"O}
    \renewcommand\map@Udieresis{"U}
    \renewcommand\map@adieresis{"a}
    \renewcommand\map@odieresis{"o}
    \renewcommand\map@udieresis{"u}
    \renewcommand\map@germandbls{"s}
    \renewcommand\map@quotedblbase{"`}
    \renewcommand\map@quotedblleft{"'}
    \renewcommand\map@quotesinglbase{\glq}
    \renewcommand\map@quoteleft{\grq}
    \renewcommand\map@guilsinglleft{\flq}
    \renewcommand\map@guilsinglright{\flq}
    \renewcommand\map@guillemotleft{\flqq}
    \renewcommand\map@guillemotright{\frqq}
\else
    \typeout{rejected, T1 encoding used.}
\fi
%    \end{macrocode}
% \Finale
\endinput}
\DeclareOption{spanish}{% \iffalse metacomment
%   This file is part of the mapcodes package, version 1.04.
% -----------------------------------------------------------
% Copyright (C) 1994 Michael Piotrowski. All rights reserved.
%
% This file is distributed in the hope that it will be useful,
% but WITHOUT ANY WARRANTY; without even the implied warranty
% of MERCHANTABILITY or FITNESS FOR A PARTICULAR PURPOSE.
% -----------------------------------------------------------
%
% IMPORTANT NOTICE:
%
% For error reports in case of UNCHANGED versions see readme file.
%
% You are not allowed to change this file.
%
% You are allowed to distribute this file under the condition that
% it is distributed together with all files mentioned in manifest.txt.
%
% If you receive only some of these files from someone, complain!
%
% You are NOT ALLOWED to distribute this file alone. You are NOT
% ALLOWED to take money for the distribution or use of either this
% file or a changed version, except for a nominal charge for copying.
%
%
%<*driver>
\documentclass{article}
\usepackage{doc}
\DontCheckModules
\newcommand{\mapcodes}{\textsf{mapcodes}}
\begin{document}
 \DocInput{spanish.dtx}
\end{document}
%</driver>
% \fi
%
% \def\filename{spanish.dtx}
% \def\fileversion{v1.0}
% \def\filedate{1994/12/05}
% \CheckSum{42}
%
% \changes{spanish-1.0}{1994/12/05}{First version}
%
% \section{\mapcodes{} Support for \textsf{babel} Spanish Option}
%
%    This is \filename{} \fileversion{} of \filedate. It contains the
%    \mapcodes support for the \textsf{babel} |spanish| option.
% 
% \StopEventually{}
%
%    \begin{macrocode}
\typeout{Compatibility with babel spanish option OT1 facilities requested...}

\ifx\encodingdefault\map@oldenc
    \typeout{enabled.}
    \renewcommand\map@Aacute{'A}
    \renewcommand\map@Eacute{'E}
    \renewcommand\map@Iacute{'I}
    \renewcommand\map@Oacute{'O}
    \renewcommand\map@Uacute{'U}
    \renewcommand\map@aacute{'a}
    \renewcommand\map@eacute{'e}
    \renewcommand\map@iacute{'i}
    \renewcommand\map@oacute{'o}
    \renewcommand\map@uacute{'u}
    \renewcommand\map@udieresis{"u}
    \renewcommand\map@ordfeminine{"a}
    \renewcommand\map@ordmasculine{"o}
    \renewcommand\map@guillemotleft{"<}
    \renewcommand\map@guillemotright{">}
    \renewcommand\map@Ntilde{~N}
    \renewcommand\map@ntilde{~n}
\else
    \typeout{rejected, T1 encoding used.}
\fi
%    \end{macrocode}
% \Finale
\endinput}
%    \end{macrocode}
%
% And finally the options are processed.
%    \begin{macrocode}
\ProcessOptions
%</package>
%    \end{macrocode}
%      
% \section{The Documentation Driver}
%    This will generate the documentation root file.
%    \begin{macrocode}
%<*driver>
\documentclass{article}
\usepackage{doc}
\usepackage[ibm850]{mapcodes}
\OnlyDescription
\DisableCrossrefs
\RecordChanges
\setlength{\parindent}{0pt}
\begin{document}
   \DocInput{mapcodes.dtx} \PrintChanges
\end{document}
%</driver>
%    \end{macrocode}
%
%    The end.
%    \begin{macrocode}
\endinput
%    \end{macrocode}
%
% \Finale
