% ======================================================================
% scrlttr2.tex
% Copyright (c) Markus Kohm, 2002-2017
%
% This file is part of the LaTeX2e KOMA-Script bundle.
%
% This work may be distributed and/or modified under the conditions of
% the LaTeX Project Public License, version 1.3c of the license.
% The latest version of this license is in
%   http://www.latex-project.org/lppl.txt
% and version 1.3c or later is part of all distributions of LaTeX 
% version 2005/12/01 or later and of this work.
%
% This work has the LPPL maintenance status "author-maintained".
%
% The Current Maintainer and author of this work is Markus Kohm.
%
% This work consists of all files listed in manifest.txt.
% ----------------------------------------------------------------------
% scrlttr2.tex
% Copyright (c) Markus Kohm, 2002-2017
%
% Dieses Werk darf nach den Bedingungen der LaTeX Project Public Lizenz,
% Version 1.3c, verteilt und/oder veraendert werden.
% Die neuste Version dieser Lizenz ist
%   http://www.latex-project.org/lppl.txt
% und Version 1.3c ist Teil aller Verteilungen von LaTeX
% Version 2005/12/01 oder spaeter und dieses Werks.
%
% Dieses Werk hat den LPPL-Verwaltungs-Status "author-maintained"
% (allein durch den Autor verwaltet).
%
% Der Aktuelle Verwalter und Autor dieses Werkes ist Markus Kohm.
% 
% Dieses Werk besteht aus den in manifest.txt aufgefuehrten Dateien.
% ======================================================================
%
% Chapter about scrlttr2 of the KOMA-Script guide
% Maintained by Markus Kohm
%
% ----------------------------------------------------------------------
%
% Kapitel ueber scrlttr2 in der KOMA-Script-Anleitung
% Verwaltet von Markus Kohm
%
% ============================================================================

\KOMAProvidesFile{scrlttr2.tex}%
                 [$Date: 2017-01-02 13:30:07 +0100 (Mon, 02 Jan 2017) $
                  KOMA-Script guide (chapter: scrlttr2)]

\translator{Harald Bongartz\and Georg Grandke\and Raimund Kohl\and Jens-Uwe
  Morawski\and Stephan Hennig\and Gernot Hassenpflug\and Markus Kohm}

% Date of the translated German file: 2017-01-02

\chapter{The New Letter Class \Class{scrlttr2}}
\labelbase{scrlttr2}
\BeginIndexGroup%
\BeginIndex{Class}{scrlttr2}%
\BeginIndex{}{letters}%

\iffalse
  Since the June 2002 release {\KOMAScript} provides a completely
  rewritten letter class\ChangedAt{v2.8q}{\Class{scrlttr2}}. Although
  part of the code is identical to that of the main classes described
  in \autoref{cha:maincls}, letters
\else
  Letters
\fi
are quite different from articles,
reports, books, and suchlike.  That alone justifies a separate
chapter about the letter class. But there is another reason for a
chapter on \Class{scrlttr2}. This class has been redeveloped from
scratch and provides a new user interface different from every other
class the author knows of. This new user interface may be uncommon,
but the author is convinced both experienced and new {\KOMAScript}
users will benefit from its advantages.


\section{Variables}
\seclabel{variables}%
\BeginIndexGroup
\BeginIndex{}{variables}

Apart from options, commands, environments, counters and lengths,
additional elements have already been introduced in \KOMAScript.  A
typical property of an element is the font style and the option to
change it (see \autoref{sec:\LabelBase.textmarkup},
\DescPageRef{\LabelBase.cmd.setkomafont}). At this point we now
introduce variables. Variables have a name by which they are called,
and they have a content. The content of a variable can be set
independently from time and location of the actual usage in the same
way as the contents of a command can be separated from its usage. The
main difference between a command and a variable is that a command
usually triggers an action, whereas a variable only consists of plain
text which is then output by a command. Furthermore, a variable can
additionally have a description which can be set and output.

This section specifically only gives an introduction to the concept of
variables. The following examples have no special meaning. More
detailed examples can be found in the explanation of predefined
variables of the letter class in the following sections. An overview
of all variables is given in \autoref{tab:\LabelBase.variables}.
%
\begin{desclist}
  % Umbruchkorrektur
  \renewcommand*{\abovecaptionskipcorrection}{-\normalbaselineskip}%
  \desccaption{%
    Alphabetical list of all supported variables in
    \Class{scrlttr2}\label{tab:\LabelBase.variables}%
  }{%
    Alphabetical list of all supported variables in \Class{scrlttr2}
    (\emph{continuation})%
  }%
  \ventry{addresseeimage}{%
    instuctions, that will be used to print the Port-Pay\'e head of option
    \OptionValueRef{\LabelBase}{addrfield}{backgroundimage} or the Port-Pay\'e
    addressee of option \OptionValueRef{\LabelBase}{addrfield}{image} %
    (\autoref{sec:\LabelBase.firstpage},
    \DescPageRef{\LabelBase.variable.addresseeimage})}%
  \ventry{backaddress}{%
    return address for window envelopes %
    (\autoref{sec:\LabelBase.firstpage},
    \DescPageRef{\LabelBase.variable.backaddress})}%
  \ventry{%
    backaddressseparator}{separator within the return address %
    (\autoref{sec:\LabelBase.firstpage},
    \DescPageRef{\LabelBase.variable.backaddressseparator})}%
  \ventry{ccseparator}{%
    separator between title of additional addressees, and additional
    addressees %
    (\autoref{sec:\LabelBase.document},
    \DescPageRef{\LabelBase.variable.ccseparator})}%
  \ventry{customer}{%
    customer number %
    (\autoref{sec:\LabelBase.firstpage},
    \DescPageRef{\LabelBase.variable.customer})}%
  \ventry{date}{%
    date %
    (\autoref{sec:\LabelBase.firstpage},
    \DescPageRef{\LabelBase.variable.date})}%
  \ventry{emailseparator}{%
    separator between e-mail name and e-mail address %
    (\autoref{sec:\LabelBase.firstpage},
    \DescPageRef{\LabelBase.variable.emailseparator})}%
  \ventry{enclseparator}{%
    separator between title of enclosure, and enclosures %
    (\autoref{sec:\LabelBase.document},
    \DescPageRef{\LabelBase.variable.enclseparator})}%
  \ventry{faxseparator}{%
    separator between title of fax, and fax number %
    (\autoref{sec:\LabelBase.firstpage},
    \DescPageRef{\LabelBase.variable.faxseparator})}%
  \ventry{firstfoot}{%
    page\ChangedAt{v3.08}{\Class{scrlttr2}} foot of the note paper %
    (\autoref{sec:\LabelBase.firstpage},
    \DescPageRef{\LabelBase.variable.firstfoot})}%
  \ventry{firsthead}{%
    page\ChangedAt{v3.08}{\Class{scrlttr2}} head of the note paper %
    (\autoref{sec:\LabelBase.firstpage},
    \DescPageRef{\LabelBase.variable.firsthead})}%
  \ventry{fromaddress}{%
    sender's address without sender name %
    (\autoref{sec:\LabelBase.firstpage},
    \DescPageRef{\LabelBase.variable.fromaddress})}%
  \ventry{frombank}{%
    sender's bank account %
    (\autoref{sec:\LabelBase.firstpage},
    \DescPageRef{\LabelBase.variable.frombank})}%
  \ventry{fromemail}{%
    sender's e-mail %
    (\autoref{sec:\LabelBase.firstpage},
    \DescPageRef{\LabelBase.variable.fromemail})}%
  \ventry{fromfax}{%
    sender's fax number %
    (\autoref{sec:\LabelBase.firstpage},
    \DescPageRef{\LabelBase.variable.fromfax})}%
  \ventry{fromlogo}{%
    commands for inserting the sender's logo %
    (\autoref{sec:\LabelBase.firstpage},
    \DescPageRef{\LabelBase.variable.fromlogo})}%
  \ventry{frommobilephone}{%
    \ChangedAt{v3.12}{\Class{scrlttr2}}%
    sender's mobile telephone number %
    (\autoref{sec:\LabelBase.firstpage},
    \DescPageRef{\LabelBase.variable.frommobilephone})}%
  \ventry{fromname}{%
    complete name of sender %
    (\autoref{sec:\LabelBase.firstpage},
    \DescPageRef{\LabelBase.variable.fromname})}%
  \ventry{fromphone}{%
    sender's telephone number %
    (\autoref{sec:\LabelBase.firstpage},
    \DescPageRef{\LabelBase.variable.fromphone})}%
  \ventry{fromurl}{%
    a URL of the sender, e.\,g., the URL of his homepage %
    (\autoref{sec:\LabelBase.firstpage},
    \DescPageRef{\LabelBase.variable.fromurl})}%
  \ventry{fromzipcode}{%
    zip code or postcode of the sender used at the Port-Pay\'e head of option
    \OptionValueRef{\LabelBase}{addrfield}{PP} %
    (\autoref{sec:\LabelBase.firstpage},
    \DescPageRef{\LabelBase.variable.fromzipcode})}%
  \ventry{invoice}{%
    invoice number %
    (\autoref{sec:\LabelBase.firstpage},
    \DescPageRef{\LabelBase.variable.invoice})}%
  \ventry{location}{%
    more details of the sender %
    (\autoref{sec:\LabelBase.firstpage},
    \DescPageRef{\LabelBase.variable.location})}%
  \ventry{myref}{%
    sender's reference %
    (\autoref{sec:\LabelBase.firstpage},
    \DescPageRef{\LabelBase.variable.myref})}%
  \ventry{nextfoot}{%
    page\ChangedAt{v3.08}{\Class{scrlttr2}} foot using page style
    \PageStyle{headings}\IndexPagestyle{headings} or
    \PageStyle{myheadings}\IndexPagestyle{myheadings} %
    (\autoref{sec:\LabelBase.pagestyle},
    \DescPageRef{\LabelBase.variable.nextfoot})}%
  \ventry{nexthead}{%
    page\ChangedAt{v3.08}{\Class{scrlttr2}} head using page style
    \PageStyle{headings}\IndexPagestyle{headings} or
    \PageStyle{myheadings}\IndexPagestyle{myheadings} %
    (\autoref{sec:\LabelBase.pagestyle},
    \DescPageRef{\LabelBase.variable.nexthead})}%
  \ventry{phoneseparator}{%
    separator between title of telephone and telephone number %
    (\autoref{sec:\LabelBase.firstpage},
    \DescPageRef{\LabelBase.variable.phoneseparator})}%
  \ventry{place}{%
    sender's place used near date %
    (\autoref{sec:\LabelBase.firstpage},
    \DescPageRef{\LabelBase.variable.place})}%
  \ventry{placeseparator}{%
    separator between place and date %
    (\autoref{sec:\LabelBase.firstpage},
    \DescPageRef{\LabelBase.variable.placeseparator})}%
  \ventry{PPdatamatrix}{%
    instruction, that print the data array of option
    \OptionValueRef{\LabelBase}{addrfield}{PP} %
    (\autoref{sec:\LabelBase.firstpage},
    \DescPageRef{\LabelBase.variable.PPdatamatrix})}%
  \ventry{PPcode}{%
    instructions for the sender's identification code of option
    \OptionValueRef{\LabelBase}{addrfield}{PP} %
    (\autoref{sec:\LabelBase.firstpage},
    \DescPageRef{\LabelBase.variable.PPcode})}%
  \ventry{signature}{%
    signature beneath the ending of the letter %
    (\autoref{sec:\LabelBase.closing},
    \DescPageRef{\LabelBase.variable.signature})}%
  \ventry{specialmail}{%
    mode of dispatch %
    (\autoref{sec:\LabelBase.firstpage},
    \DescPageRef{\LabelBase.variable.specialmail})}%
  \ventry{subject}{%
    letter's subject %
    (\autoref{sec:\LabelBase.firstpage},
    \DescPageRef{\LabelBase.variable.subject})}%
  \ventry{subjectseparator}{%
    separator between title of subject and subject %
    (\autoref{sec:\LabelBase.firstpage},
    \DescPageRef{\LabelBase.variable.subjectseparator})}%
  \ventry{title}{%
    letter title %
    (\autoref{sec:\LabelBase.firstpage},
    \DescPageRef{\LabelBase.variable.title})}%
  \ventry{toaddress}{%
    address of addressee without addressee name %
    (\autoref{sec:\LabelBase.firstpage},
    \DescPageRef{\LabelBase.variable.toaddress})}%
  \ventry{toname}{%
    complete name of addressee %
    (\autoref{sec:\LabelBase.firstpage},
    \DescPageRef{\LabelBase.variable.toname})}%
  \ventry{yourmail}{%
    date of addressee's referenced mail %
    (\autoref{sec:\LabelBase.firstpage},
    \DescPageRef{\LabelBase.variable.yourmail})}%
  \ventry{yourref}{%
    addressee's reference %
    (\autoref{sec:\LabelBase.firstpage},
    \DescPageRef{\LabelBase.variable.yourref})}%
  \ventry{zipcodeseparator}{%
    separator between the zip code's or postcode's title and the code itself %
    (\autoref{sec:\LabelBase.firstpage},
    \DescPageRef{\LabelBase.variable.zipcodeseparator})}%
\end{desclist}

\begin{Declaration}
  \Macro{setkomavar}%
    \Parameter{name}\OParameter{description}\Parameter{content}%
  \Macro{setkomavar*}\Parameter{name}\Parameter{description}
\end{Declaration}
With the command \Macro{setkomavar} you determine the
\PName{content} of the variable \PName{name}. Using an optional
argument you can at the same time  change  the \PName{description}
of the variable. In contrast, \Macro{setkomavar*} can only set the
\PName{description} of the variable \PName{name}.
% Wir haben hier ein anderes Beispiel als in der deutschen Anleitung, aber das
% passt hier sehr gut so!
\begin{Example}
  Suppose you have defined a direct dialling as mentioned above
  and you now want to set the content. You write:
\begin{lstlisting}
  \setkomavar{myphone}{-\,11}
\end{lstlisting}
  In addition, you want to replace the term ``direct dialling''
  with ``Connection''. Thus you add the description:
\begin{lstlisting}
  \setkomavar*{myphone}{Connection}
\end{lstlisting}
  or you can combine both in one command:
\begin{lstlisting}
  \setkomavar{myphone}[Connection]{-\,11}
\end{lstlisting}
\end{Example}
By the way: You may delete the content of a variable using an empty
\PName{content} argument. You can also delete the description using an
empty \PName{description} argument.
\begin{Example}
  Suppose you have defined a direct dialling as mentioned above and
  you now no longer want a description to be set. You write:
\begin{lstlisting}
  \setkomavar*{myphone}{}
\end{lstlisting}
  You can combine this with the definition of the content:
\begin{lstlisting}
  \setkomavar{myphone}[]{-\,11}
\end{lstlisting}
  So you may setup the content and delete the description using only
  one command.
\end{Example}
%
\EndIndexGroup


\begin{Declaration}
  \Macro{usekomavar}\OParameter{command}\Parameter{name}%
  \Macro{usekomavar*}\OParameter{command}\Parameter{name}
\end{Declaration}
In\ChangedAt{v2.9i}{\Class{scrlttr2}} some cases it is necessary for the user
to access the content or the description of a variable, and not to leave this
only up to the class. This is specially important when you have defined a
variable which is not added to the reference fields line. Using the command
\Macro{usekomavar} you have access to the content of the variable
\PName{name}, whereas the starred version \Macro{usekomavar*} allows you to
access the description or title. In \autoref{sec:scrlttr2-experts.variables},
\DescPageRef{scrlttr2-experts.cmd.newkomavar} you may find more
information about defining variable on your own.%
\EndIndexGroup


\begin{Declaration}
  \Macro{ifkomavar}\Parameter{name}\Parameter{true-code}\Parameter{false-code}
\end{Declaration}
This\ChangedAt{v3.03}{\Class{scrlttr2}} command may be used to test, whether
or not a variable has already been defined. The \PName{true-code} will be
executed only, if the variable already exists. The contents of the variable
will not be examined, so it may be empty. The \PName{false-code} will be
executed if the variable does not yet exist. Such tests make sense if a
variable will be defined at one
\File{lco}-file\Index{lco-file=\File{lco}-file} (see
\autoref{sec:\LabelBase.lcoFile} from \autopageref{sec:\LabelBase.lcoFile}
onward), but used in another \File{lco}-file if it exists only.%
%
\EndIndexGroup


\begin{Declaration}
  \Macro{ifkomavarempty}\Parameter{name}\Parameter{true-code}%
    \Parameter{false-code}%
  \Macro{ifkomavarempty*}\Parameter{name}\Parameter{true-code}%
    \Parameter{false-code}
\end{Declaration}
With\ChangedAt{v2.9i}{\Class{scrlttr2}} these commands you may check
whether or not the expanded content or description of a variable is
empty. The \PName{true-code} will be executed if the content or
description is empty. Otherwise the \PName{false-code} will be
executed. The starred variant handles the description of a variable,
the unstarred variant handles the contents.%
\EndIndexGroup
%
\EndIndexGroup


\section{Pseudo-Lengths}
\seclabel{pseudoLength}

\BeginIndexGroup
\BeginIndex{}{pseudo-lengths}
\LaTeX{} processes length with commands \Macro{newlength}, \Macro{setlength},
\Macro{addtolength}, and \Macro{the\PName{length}}. Many packages also use
macros, that are commands, to storage lengths. \KOMAScript{} extends the
method of storing length at macros by some commands similar to the commands
above, that are used to handle real lengths. \KOMAScript calls this kind of
lengths, that are stored at macros instead of real \LaTeX{} lengths,
pseudo-lengths.

A list of all pseudo-lengths in \Class{scrlttr2} is shown in
\autoref{tab:scrlttr2-experts.pseudoLength} starting at
\autopageref{tab:scrlttr2-experts.pseudoLength}. The meaning of the various
pseudo-lengths is shown graphically in
\autoref{fig:scrlttr2-experts.pseudoLength}. The dimensions used in the figure
correspond to the default settings of \Class{scrlttr2}. More detailed
description of the individual pseudo-lengths is found in the individual
sections of this chapter.

Normally users would not need to define a pseudo-length on their own. Because
of this, definition of pseudo-lengths will be described in the expert part at
\autoref{sec:scrlttr2-experts.pseudoLengths},
\DescPageRef{scrlttr2-experts.cmd.@newplength}. Setting pseudo-lengths to
new values is also a work for advanced users. So this will be described in the
expert part too at \DescPageRef{scrlttr2-experts.cmd.@setplength}.

Please note\textnote{Attention!}: Even though these pseudo-lengths are
internally implemented as macros, the commands for pseudo-length management
expect only the names of the pseudo-lengths not the macros representing the
pseudo-lengths. The names of pseudo-lengths are without backslash at the very
beginning similar to the names of \LaTeX{} counters and in opposite to macros
or \LaTeX{} lengths.

\begin{Declaration}
  \Macro{useplength}\Parameter{name}
\end{Declaration}
Using this command you can access the value of the pseudo-length with
the given \PName{name}. This is one of the few user commands in
connection with pseudo-lengths. Of course this command can also be
used with an \File{lco}-file\Index{lco-file=\File{lco}-file} (see
\autoref{sec:\LabelBase.lcoFile} ab \autopageref{sec:\LabelBase.lcoFile}).%
%
\EndIndexGroup


\begin{Declaration}
  \Macro{setlengthtoplength}%
    \OParameter{factor}\Parameter{length}\Parameter{pseudo-length}%
  \Macro{addtolengthplength}%
    \OParameter{factor}\Parameter{length}\Parameter{pseudo-length}
\end{Declaration}
\begin{Explain}%
  While you can simply prepend a factor to a length, this is not
  possible with pseudo-lengths. Suppose you have a length \Macro{test}
  with the value 2\Unit{pt}; then \texttt{3\Macro{test}} gives you the
  value 6\Unit{pt}. Using pseudo-lengths instead,
  \texttt{3\Macro{useplength}\PParameter{test}} would give you
  32\Unit{pt}. This is especially annoying if you want a real
  \PName{length} to take the value of a \PName{pseudo-length}.
\end{Explain}
Using the command \Macro{setlengthtoplength} you can assign the
multiple of a \PName{pseudo-length} to a real \PName{length}.  Here,
instead of prepending the \PName{factor} to the \PName{pseudo-length},
it is given as an optional argument. You should also use this command
when you want to assign the negative value of a \PName{pseudo-length}
to a \PName{length}. In this case you can either use a minus sign or
\PValue{-1} as the \PName{factor}. The command
\Macro{addtolengthplength} works very similarly; it adds the multiple
of a \PName{pseudo-length} to the \PName{length}.
%
\EndIndexGroup
%
\EndIndexGroup


\LoadCommonFile{options} % \section{Early or Late Selection of Options}

\LoadCommonFile{compatibility} % \section{Compatibility with Earlier Versions of
                        %   \KOMAScript{}}

\LoadCommonFile{draftmode} % \section{Draft-Mode}

\LoadCommonFile{typearea} % \section{Page Layout}

Normally it makes no sense to distinguish letters with single-side layout and
letters with double-side layout. Mostly letters are not bound like
books. Therefor each page will be viewed on its own. This is also true if both
sides of the paper sheet will be used for printing. Vertical adjustment is
unusual at letters too. Nevertheless, if you need or want it, you may read the
description of \Macro{raggedbottom} and \Macro{flushbottom} in
\autoref{sec:maincls.typearea} at \DescPageRef{maincls.cmd.flushbottom}.%
%
\EndIndexGroup


\section{General Structure of Letter Documents}
\seclabel{document}
\BeginIndexGroup
\BeginIndex{}{letter>structure}

The general structure of a letter document differs somewhat from the
structure of a normal document. Whereas a book document in general
contains only one book, a letter document can contain several
letters. As illustrated in \autoref{fig:\LabelBase.document}, a letter
document consists of a preamble, the individual letters, and the
closing.

\begin{figure}
  \KOMAoptions{captions=bottombeside}%
  \setcapindent{0pt}%
  \begin{captionbeside}[{%
      General structure of a letter document with several individual letters%
    }]{%
      General structure of a letter document with several individual letters
      (the structure of a single letter is shown in
      \autoref{fig:\LabelBase.letter})%
      \label{fig:\LabelBase.document}%
    }[l]
    \begin{minipage}[b]{.667\linewidth}
      \centering\small\setlength{\fboxsep}{1.5ex}%
      \addtolength{\linewidth}{-\dimexpr2\fboxrule+2\fboxsep\relax}%
      \fbox{\parbox{\linewidth}{\raggedright%
          \Macro{documentclass}\OParameter{\dots}\PParameter{scrlttr2}\\
          \dots\\
          {\centering\emph{settings for all letters}\\}
          \dots\\
          \Macro{begin}\PParameter{document}\\
          \dots\\
          {\centering\emph{settings for all letters}\\}
          \dots\\
        }}\\
      \fbox{\parbox{\linewidth}{\raggedright%
          \Macro{begin}\PParameter{letter}\Parameter{addressee}\\
          \dots\\
          {\centering\emph{content of the individual letter}\\}
          \dots\\
          \Macro{end}\PParameter{letter}\\
        }}\\[2pt]
      \parbox{\linewidth}{\raggedright\vspace{-.5ex}\vdots\vspace{1ex}}\\
      \fbox{\parbox{\linewidth}{\raggedright%
          \Macro{end}\PParameter{document}\\
        }}\\[\dimexpr\fboxsep+\fboxrule\relax]
    \end{minipage}
  \end{captionbeside}
\end{figure}

The preamble comprises all settings that in general concern all letters. Most
of them can also be overwritten in the settings of the individual letters. The
only setting which can not be changed within a single letter is compatibility
to prior versions of \Class{scrlttr2} (see option \DescRef{\LabelBase.option.version} in
\autoref{sec:\LabelBase.compatibilityOptions},
\DescPageRef{\LabelBase.option.version}).

It is recommended that only general settings such as the loading of packages
and the setting of options be placed before
\Macro{begin}\PParameter{document}. All settings that comprise the setting of
variables or other text features should be done after
\Macro{begin}\PParameter{document}. This is particularly recommended when the
\Package{babel} package\IndexPackage{babel} (see \cite{package:babel}) is
used, or language-dependent variables of \Class{scrlttr2} are to be changed.

The closing usually consists only of
\Macro{end}\PParameter{document}. Of course you can also insert
additional comments at this point.

\begin{figure}
  \KOMAoptions{captions=bottombeside}%
  \setcapindent{0pt}%
  \begin{captionbeside}[{%
      General structure of a single letter within a letter document%
    }]{%
      General structure of a single letter within a letter document (see also
      \autoref{fig:\LabelBase.document})%
      \label{fig:\LabelBase.letter}}[l]
    \begin{minipage}[b]{.667\linewidth}%
      \centering\small\setlength{\fboxsep}{1.5ex}%
      \addtolength{\linewidth}{-\dimexpr2\fboxrule+2\fboxsep\relax}%
      \fbox{\parbox{\linewidth}{\raggedright%
          \Macro{begin}\PParameter{letter}%
          \OParameter{Optionen}\Parameter{addressee}\\
          \dots\\[\dp\strutbox]
          {\centering\emph{settings for this letter}\\}
          \dots\\
          \DescRef{\LabelBase.cmd.opening}\Parameter{opening}\\
        }}\\[1pt]
      \fbox{\parbox{\linewidth}{\raggedright%
          \dots\\[\dp\strutbox]
          {\centering\emph{letter text}\\}
          \dots\\
        }}\\[1pt]
      \fbox{\parbox{\linewidth}{\raggedright%
          \DescRef{\LabelBase.cmd.closing}\Parameter{closing}\\
          \DescRef{\LabelBase.cmd.ps}\\
          \dots\\[\dp\strutbox]
          {\centering\emph{postscript}\\}
          \dots\\
          \DescRef{\LabelBase.cmd.encl}\Parameter{enclosures}\\
          \DescRef{\LabelBase.cmd.cc}\Parameter{additional addressees}\\
          \Macro{end}\PParameter{letter}\\
        }}\\[\dimexpr\fboxsep+\fboxrule\relax]
    \end{minipage}
  \end{captionbeside}
\end{figure}

As shown in \autoref{fig:\LabelBase.letter}, every single letter itself consists
of an introduction, the letter body, and the closing. In the introduction, all
settings pertaining only to the current letter are defined. It is important
that this introduction always ends with
\DescRef{\LabelBase.cmd.opening}\IndexCmd{opening}. Similarly, the closing always starts with
\DescRef{\LabelBase.cmd.closing}\IndexCmd{closing}. The two arguments \PName{opening} and
\PName{closing} can be left empty, but both commands must be used and must
have an argument.

It should be noted that several settings can be changed between the individual
letters. Such changes then have an effect on all subsequent letters. For
reasons of maintainability of your letter documents, it is however not
recommended to use further general settings with limited scope between the
letters.

\begin{Declaration}
  \begin{Environment}{letter}\OParameter{options}\Parameter{addressee}
  \end{Environment}
\end{Declaration}
\BeginIndex{}{addressee}%
The \Environment{letter} environment is one of the key environments of the
letter class. A special\textnote{\KOMAScript{} vs. standard classes}
\Class{scrlttr2} feature are optional arguments to the \Environment{letter}
environment. These \PName{options} are executed internally via the
\DescRef{\LabelBase.cmd.KOMAoptions} command.

The \PName{addressee} is a mandatory argument passed to the
\Environment{letter} environment. Parts\textnote{Attention!} of the addressee
contents are separated by double backslashes. These parts are output on
individual lines in the address field. Nevertheless, the double backslash
should not be interpreted as a certain line break. Vertical material such as
paragraphs or vertical space is not permitted within \PName{addressee}, and
could lead to unexpected results and error messages, as is the case also for
the standard letter class.

\begin{Example}
  \phantomsection\label{desc:\LabelBase.env.letter.example}%
  Assumed, someone wants to send a letter to Joana Public. A minimalistic
  letter document for this may be:
\begin{lstcode}
  \documentclass[version=last]{scrlttr2}
  \usepackage[english]{babel}
  \begin{document}
  \begin{letter}{Joana Public\\
      Hillside 1\\
      12345 Public-City}
  \end{letter}
  \end{document}
\end{lstcode}
  However, this would not result in any printable output. At least there
  would not be an addressee at the note paper sheet. The reason for this will
  be explained at the description of command \DescRef{\LabelBase.cmd.opening} at
  \DescPageRef{\LabelBase.cmd.opening}.
\end{Example}
%
\EndIndexGroup


\begin{Declaration}
  \Macro{AtBeginLetter}\Parameter{instruction code}%
  \Macro{AtEndLetter}\Parameter{instruction code}
\end{Declaration}
{\LaTeX} enables the user to declare \PName{instruction code} whose execution
is delayed until a determined point. Such points are called
\emph{hooks}\Index{hook}. Known macros for using hooks are
\Macro{AtBeginDocument}\IndexCmd{AtBeginDocument} and
\Macro{AtEndOfClass}\IndexCmd{AtEndOfClass} at the \LaTeX{} kernel. The class
\Class{scrlttr2} provides two more hooks. The \PName{instruction code} for
these may be declared using \Macro{AtBeginLetter} and
\Macro{AtEndLetter}\ChangedAt{v2.95}{\Class{scrlttr2}}. Originally, hooks were
provided for package and class authors, so they are documented in
\cite{latex:clsguide} only, and not in \cite{latex:usrguide}. However, with
letters there are useful applications of \Macro{AtBeginLetter} as the
following example may illustrate.
%
\begin{Example}
  It is given that one has to set multiple letters with questionnaires
  within one document. Questions are numbered automatically within
  single letters using a counter. Since, in contrast to page
  numbering, that counter is not known by \Class{scrlttr2}, it would
  not be reset at the start of each new letter. Given that each
  questionnaire contains ten questions, question~1 would get number~11
  in the second letter. A solution is to reset this counter at the
  beginning of each new letter:
\begin{lstlisting}
  \newcounter{Question}
  \newcommand{\Question}[1]{%
    \refstepcounter{Question}\par
    \noindent\begin{tabularx}{\textwidth}{l@{}X}
      \theQuestion:~ & #1\\
    \end{tabularx}%
  }%
  \AtBeginLetter{\setcounter{Question}{0}}
\end{lstlisting}
  This way first question remains question~1, even in the 1001st letter. Of
  course the definition at this example needs package
  \Package{tabularx}\IndexPackage{tabularx} (see \cite{package:tabularx}).
\end{Example}
%
\EndIndexGroup


\begin{Declaration}
  \Counter{letter}%
  \Macro{thisletter}%
  \Macro{letterlastpage}
\end{Declaration}
If\ChangedAt{v3.19}{\Class{scrlttr2}\and \Package{scrletter}} you have more
than one letter in one document, it is useful to have a letter number. Since
version~3.19 \KOMAScript{} provides counter \Counter{letter} and increases it
at every \Macro{begin}\PParameter{letter}.
\begin{Example}
  Have one more look into the \DescRef{\LabelBase.cmd.AtBeginLetter}
  example. Instead of resetting the counter explicitly at
  \Macro{begin}\PParameter{letter}, we can do it implicitly by defining
  counter \Counter{Question} depending on counter \Counter{letter}:
\begin{lstlisting}
  \newcounter{Question}[letter]
  \newcommand{\Question}[1]{%
    \refstepcounter{Question}\par
    \noindent\begin{tabularx}{\textwidth}{l@{}X}
      \theQuestion:~ & #1\\
    \end{tabularx}%
  }%
\end{lstlisting}
  Now, the new counter will be reset at every start of a new letter and
  the first question of every letter will be number one.
\end{Example}

If you want the output of current value of \Counter{letter}, you may usually
use \Macro{theletter}. Indeed the letter can also be used for
cross-references. So you can use \Macro{label}\Parameter{name} to generate a
label immediately after \Macro{begin}\PParameter{letter} and reference it
somewhere in the document using \Macro{ref}\Parameter{name}. Inside the same
letter you can simply use \Macro{thisletter} without generating a label to get
the same result.

\KOMAScript{} itself uses \Macro{thisletter} to put a label onto the last page
of every letter. You can use
\Macro{letterlastpage}\IndexCmd{label}\IndexCmd{pageref} to reference the last
page number of the current letter. Please note, the value of
\Macro{letterlastpage} is valid after some \LaTeX{} runs, because it uses
\Macro{label} and \Macro{pageref}. So you need at least two or three \LaTeX{}
runs, if you use \Macro{letterlastpage}. Please have a look at \emph{Rerun}
terminal or \File{log}-file messages about labels that have been changed.%
\EndIndexGroup


\begin{Declaration}
  \Macro{opening}\Parameter{opening}
\end{Declaration}
This is one of the most important commands in \Class{scrlttr2}.  For the user
it may seem that only the \PName{opening}\Index{letter>opening}, e.\,g.,
``Dear Mrs~\dots'', is typeset, but\textnote{Attention!} the command also
typesets the folding marks\Index{folding marks},
letterhead\Index{letter>head}, address field\Index{address field}, reference
fields line\Index{reference line}, subject\Index{subject}, the page
footer\Index{page>footer} and others. In short, without \Macro{opening} there
is no letter. And if you want to print a letter without opening you have to
use an \Macro{opening} command with an empty argument.

\begin{Example}
  Let's extend the example from
  \DescPageRef{\LabelBase.env.letter.example} by an opening:
  \lstinputcode[xleftmargin=1em]{letter-0.tex}
  This will result in a note paper sheet shown in
  \autoref{fig:\LabelBase.letter-0}.
  \begin{figure}
    \setcapindent{0pt}%
    \begin{captionbeside}[{Example: letter with addressee and opening}]{%
        result of a minimalistic letter with addressee and opening only 
        (date and folding marks are defaults of DIN-letters)}[l]
    \frame{\includegraphics[width=.4\textwidth]{letter-0}}
    \end{captionbeside}
    \label{fig:\LabelBase.letter-0}
  \end{figure}
\end{Example}
\iffalse% Umbruchkorrekturtext

In the early days of computer-generated letters, programs did not have many
capabilities, therefore the letters seldom had an opening.  Today the
capabilities have been enhanced. Thus personal openings are very common, even
in mass-production advertising letters.%
\fi
%
\EndIndexGroup

\begin{Declaration}
  \Macro{closing}\Parameter{closing phrase}
\end{Declaration}
The main purpose of the command \Macro{closing} is to typeset the
\PName{closing phrase}\Index{closing}. This may even consists of multiple
lines. The lines should be separated by double backslash. Paragraph breaks
inside the \PName{closing phrase} are not allowed.

Beyond that the command also typesets the content of the variable
\DescRef{\LabelBase.variable.signature}. More information about the signature and the
configuration of the signature may be found at \autoref{sec:\LabelBase.closing}
ab \DescPageRef{\LabelBase.variable.signature}.

\begin{Example}
  Let's extend the our example by some lines of text and a closing phrase:
  \lstinputcode[xleftmargin=1em]{letter-1.tex}
  This will result in a the letter shown in \autoref{fig:\LabelBase.letter-1}.
  \begin{figure}
    \setcapindent{0pt}%
    \begin{captionbeside}[{Example: letter with addressee, opening, text, and
        closing}]{%
        result of a small letter with addressee, opening, text, and closing
        (date and folding marks are defaults of DIN-letters)}[l]
      \frame{\includegraphics[width=.4\textwidth]{letter-1}}
    \end{captionbeside}
    \label{fig:\LabelBase.letter-1}
  \end{figure}
\end{Example}
%
\EndIndexGroup

\begin{Declaration}
  \Macro{ps}
\end{Declaration}%
This instruction merely switches to the postscript.  Hence, a new
paragraph begins, and a vertical distance\,---\,usually below the
signature\,---\,is inserted.  The command \Macro{ps} is followed by
normal text. If you want the postscript to be introduced with the
acronym ``PS:'' , which by the way is written without a full stop, you
have to type this yourself. The acronym is typeset neither
automatically nor optionally by the class \Class{scrlttr2}.

\begin{Example}
  The example letter extended by a postscript
  \lstinputcode[xleftmargin=1em]{letter-2.tex}
  results in \autoref{fig:\LabelBase.letter-2}.
  \begin{figure}
    \setcapindent{0pt}%
    \begin{captionbeside}[{Example: letter with addressee, opening, text,
        closing, and postscript}]{%
        result of a small letter with addressee, opening, text, closing, and
        postscript
        (date and folding marks are defaults of DIN-letters)}[l]
      \frame{\includegraphics[width=.4\textwidth]{letter-2}}
    \end{captionbeside}
    \label{fig:\LabelBase.letter-2}
  \end{figure}
\end{Example}

\begin{Explain}
  In the time when letters were written by hand it was quite common to use a
  postscript because this was the only way to add information which one had
  forgotten to mention in the main part of the letter. Of course, in letters
  written with {\LaTeX} you can insert additional lines easily. Nevertheless,
  it is still popular to use the postscript. It gives one a good possibility
  to underline again the most important or sometimes the less important things
  of the particular letter.
\end{Explain}
%
\EndIndexGroup


\begin{Declaration}
  \Macro{cc}\Parameter{distribution list}%
  \Variable{ccseparator}%
\end{Declaration}
With the command \Macro{cc}%
\Index{addressee>additional}\Index{distribution list}\Index{carbon copy} it is
possible to typeset a \PName{distribution list}.  The command takes the
\PName{distribution list} as its argument. If the content of the variable
\Variable{ccseparator}\Index{separator}\Index{delimiter} is not empty, then
the name and the content of this variable is inserted before
\PName{distribution list}.  In this case the \PName{distribution list} will be
indented appropriately.  It is a good idea\textnote{Hint!} to set the
\PName{distribution list} \Macro{raggedright}\IndexCmd{raggedright} and to
separate the individual entries with a double backslash.
\begin{Example}
  This time, the example letter should be send not only to the chairman, but
  also to all club members:
  \lstinputcode[xleftmargin=1em]{letter-3.tex}%
  The result is shown in \autoref{fig:\LabelBase.letter-3}.
  \begin{figure}
    \setcapindent{0pt}%
    \begin{captionbeside}[{Example: letter with addressee, opening, text,
        closing, postscript, and distribution list}]{%
        result of a small letter with addressee, opening, text, closing,
        postscript, and distribution list
        (date and folding marks are defaults of DIN-letters)}[l]
      \frame{\includegraphics[width=.4\textwidth]{letter-3}}
    \end{captionbeside}
    \label{fig:\LabelBase.letter-3}
  \end{figure}
\end{Example}
In front of the distribution list a vertical gap is inserted automatically.%
%
\EndIndexGroup


\begin{Declaration}
  \Macro{encl}\Parameter{enclosures}%
  \Variable{enclseparator}%
\end{Declaration}
The \PName{enclosures} have the same structure as the distribution list.
The only difference is that here the enclosures starts with the name
and content of the variable
\Variable{enclseparator}\Index{separator}\Index{delimiter}.
\begin{Example}
  Now, the example letter will be extended by some paragraphs from the
  constitution. These will be added as an enclosure. The description title
  will be changed also, because there is only one enclosure and the default
  may be prepared for several enclosures:
  \lstinputcode[xleftmargin=1em]{letter-4.tex}
  This will result in \autoref{fig:\LabelBase.letter-4}.
  \begin{figure}
    \setcapindent{0pt}%
    \begin{captionbeside}[{Example: letter with addressee, opening, text,
        closing, postscript, distribution list, and enclosure}]{%
        result of a small letter with addressee, opening, text, closing,
        postscript, distribution list, and enclosure
        (date and folding marks are defaults of DIN-letters)}[l]
      \frame{\includegraphics[width=.4\textwidth]{letter-4}}
    \end{captionbeside}
    \label{fig:\LabelBase.letter-4}
  \end{figure}
\end{Example}
%
\EndIndexGroup
%
\EndIndexGroup


\LoadCommonFile{fontsize} % \section{Selection of Document Font Size}

\LoadCommonFile{textmarkup} % \section{Text Markup}

\section{Note Paper}
\seclabel{firstpage}
\BeginIndexGroup
\BeginIndex{}{note paper}%
\BeginIndex{}{letter>first page}

The note paper is the first page and therefore the signboard of each
letter. In business scope often preprinted forms are used, that already
contains elements like the letter head with the sender's information and
logo. \KOMAScript{} provides to position these elements independent. With this
it is not only possible to replicate the note paper directly, but also to
complete the destined fields instantaneously. The independent position is
provided by pseudo-lengths (see \autoref{sec:\LabelBase.pseudoLength} from
\autopageref{sec:\LabelBase.pseudoLength} onward). A schematic display of the
note page and the used variable is shown by
\autoref{fig:\LabelBase.variables}. Thereby the names of the variables are
printed boldly for better distinction from the commands and their arguments.

Following pages\Index{page>following}\Index{following page} are different from
the note paper. Following pages in the meaning of this manual are all letter
pages but the first one.

\begin{figure}
  \centering
  \includegraphics[scale=0.99]{varDIN}
  \caption{schematic display of the note paper with the most important
    commands and variables for the drafted elements}
  \label{fig:\LabelBase.variables}
\end{figure}

\begin{Declaration}
  \OptionVName{foldmarks}{selection}
\end{Declaration}
Foldmarks\Index{foldmark} or folding marks\Index{folding mark} are tiny
horizontal rules at the left margin or tiny vertical rules at the top
margin. \KOMAScript{} currently provides three configurable horizontal folding
marks and one configurable vertical folding mark. Additionally it provides a
horizontal hole puncher mark, also known as page middle mark. This additional
mark cannot be moved vertically.

This option activates or deactivates folding marks for vertical
two-, three- or four-panel folding, and a single horizontal folding, of the
letter, whereby the folding need not result in equal-sized parts. The position
of the four horizontal and the single vertical marks are configurable via
pseudo-lengths (see \autoref{sec:scrlttr2-experts.foldmarks} from
\DescPageRef{scrlttr2-experts.plength.foldmarkvpos} onwards).

The user has a choice: Either one may use the standard values for simple
switches, as described in \autoref{tab:truefalseswitch},
\autopageref{tab:truefalseswitch}, to activate or deactivate at once all
configured folding marks on the left and upper edges of the paper;
or\ChangedAt{v2.97e}{\Class{scrlttr2}} one may specify by one or more letters,
as listed in \autoref{tab:\LabelBase.foldmark}, the use of the individual
folding marks independently. Also in the latter case the folding marks will only be
shown if they have not been switched off generally with one of \PValue{false},
\PValue{off}, or \PValue{no}. The exact positioning of the folding marks is
specified in the user settings, that is, the \File{lco} files (see
\autoref{sec:\LabelBase.lcoFile} from \autopageref{sec:\LabelBase.lcoFile} onward)
chosen for a letter. Default values are \PValue{true} and \PValue{TBMPL}.
%
\begin{table}
%  \centering
  \KOMAoptions{captions=topbeside}%
  \setcapindent{0pt}%
%  \caption
  \begin{captionbeside}{%
      Combinable values for the configuration of folding marks with 
      option \Option{foldmarks}%
    }[l]
    \begin{tabular}[t]{ll}
      \toprule
      \PValue{B} & activate upper horizontal foldmark on left paper edge\\%
      \PValue{b} & deactivate upper horizontal foldmark on left paper edge\\%
      \PValue{H} & activate all horizontal folding marks on left paper edge\\%
      \PValue{h} & deactivate all horizontal folding marks on left paper edge\\%
      \PValue{L} & activate left vertical foldmark on upper paper edge\\%
      \PValue{l} & deactivate left vertical foldmark on upper paper edge\\%
      \PValue{M} & activate middle horizontal foldmark on left paper edge\\%
      \PValue{m} & deactivate middle horizontal foldmark on left paper edge\\%
      \PValue{P} & activate punch or center mark on left paper edge\\%
      \PValue{p} & deactivate punch or center mark on left paper edge\\%
      \PValue{T} & activate lower horizontal foldmark on left paper edge\\%
      \PValue{t} & deactivate lower horizontal foldmark on left paper edge\\%
      \PValue{V} & activate all vertical folding marks on upper paper edge\\%
      \PValue{v} & deactivate all vertical folding marks on upper paper edge\\%
      \bottomrule
    \end{tabular}
  \end{captionbeside}
  \label{tab:\LabelBase.foldmark}
\end{table}
%
\begin{Example}
  Assume that you would like to deactivate all folding marks except the punching
  mark.  This you can accomplish with, for example:
\begin{lstlisting}
  \KOMAoptions{foldmarks=blmt}
\end{lstlisting}
  as long as the defaults have not been changed previously. If some changes
  might have been made before, a safer method should be used. This changes our
  example a little bit:
  \lstinputcode[xleftmargin=1em]{letter-7}%
  The result is shown in \autoref{fig:\LabelBase.letter-7}.
  \begin{figure}
    \setcapindent{0pt}%
    \begin{captionbeside}[{Example: letter with addressee, opening, text,
        closing, postscript, distribution list, enclosure, and hole puncher
        mark}]{%
        result of a small letter with addressee, opening, text, closing,
        postscript, distribution list, enclosure, and hole puncher mark
        (the date is a default of DIN-letters)}[l]
      \frame{\includegraphics[width=.4\textwidth]{letter-7}}
    \end{captionbeside}
    \label{fig:\LabelBase.letter-7}
  \end{figure}
\end{Example}
\BeginIndex{FontElement}{foldmark}\LabelFontElement{foldmark}%
The color of the folding mark may be changed
using\ChangedAt{v2.97c}{\Class{scrlttr2}} using the commands
\DescRef{\LabelBase.cmd.setkomafont} and
\DescRef{\LabelBase.cmd.addtokomafont} (see
\autoref{sec:\LabelBase.textmarkup}, \DescPageRef{\LabelBase.cmd.setkomafont})
with element \FontElement{foldmark}. Default is not change.%
\EndIndex{FontElement}{foldmark}%
%
\EndIndexGroup


\begin{Declaration}
  \OptionVName{enlargefirstpage}{simple switch}
\end{Declaration}
The first page of a letter always uses a different page layout. The
\Class{scrlttr2} class provides a mechanism to calculate height and vertical
alignment of header and footer of the first page independently of the
following pages. If, as a result, the footer of the first page would reach
into the text area, this text area is automatically made smaller using the
\Macro{enlargethispage}\IndexCmd{enlargethispage} macro. On the other hand, if
the text area should become larger, supposing that the footer on the first
page allows that, you can use this option. At best, a little more text will
then fit on the first page. See also the description of the pseudo-length
\PLength{firstfootvpos} on
\DescPageRef{scrlttr2-experts.plength.firstfootvpos}.  This option can
take the standard values for simple switches, as listed in
\autoref{tab:truefalseswitch}, \autopageref{tab:truefalseswitch}. Default is
\PValue{false}.
%
\EndIndexGroup


\begin{Declaration}
  \OptionVName{firsthead}{simple switch}
\end{Declaration}
\BeginIndex{}{letterhead}%
\BeginIndex{}{letter>head}%
The\ChangedAt{v2.97e}{\Class{scrlttr2}} letterhead is usually the topmost
element of the note paper.  This option determines whether the letterhead will
be typeset at all. The option accepts the standard values for simple keys,
given in \autoref{tab:truefalseswitch} at
\autopageref{tab:truefalseswitch}. Default is for the letterhead to be set.%
%
\EndIndexGroup


\begin{Declaration}
  \OptionVName{fromalign}{method}
\end{Declaration}
\BeginIndex{}{letterhead}%
\BeginIndex{}{letter>head}%
Option\important{\Option{fromalign}} \Option{fromalign} defines the placement
of the return address in the letterhead of the first page. Apart from the
various options for positioning the return address in the letterhead, there is
also the option\ChangedAt{v2.97e}{\Class{scrlttr2}} of adding the return
address to the sender's extension\Index{sender's
  extension}. Further\textnote{Attention!}, this option serves as a switch to
activate or deactivate the letterhead extensions. If these extensions are
deactivated, some other options will have no effect. This will be noted in the
explanations of the respective options. Possible values for \Option{fromalign}
are shown in \autoref{tab:\LabelBase.fromalign}. Default is \PValue{left}.%
%
\begin{table}
  \caption[{Available values for option \Option{fromalign} with
    \Class{scrlttr2}}]{Available values for option \Option{fromalign} to
    define the position of the from address in the letterhead with
    \Class{scrlttr2}}
  \label{tab:\LabelBase.fromalign}
  \begin{desctabular}
    \entry{\PValue{center}, \PValue{centered}, \PValue{middle}}{%
      return address centered; an optional logo will be above the extended
      return address; letterhead extensions will be activated}%
    \entry{\PValue{false}, \PValue{no}, \PValue{off}}{%
      standard design will be used for the return address; the letterhead
      extensions are deactivated}%
    \entry{\PValue{left}}{%
      left-justified return address; an optional logo will be right justified;
      letterhead extensions will be activated}%
    \entry{\PValue{locationleft}, \PValue{leftlocation}}{%
      return address is set left-justified in the sender's extension; a logo,
      if applicable, will be placed above it; the letterhead is automatically
      deactivated but can be reactivated using option
      \DescRef{\LabelBase.option.firsthead}.}%
    \entry{\PValue{locationright}, \PValue{rightlocation},
      \PValue{location}}{%
      return address is set right-justified in the sender's extension; a logo,
      if applicable, will be placed above it; the letterhead is automatically
      deactivated but can be reactivated using option
      \DescRef{\LabelBase.option.firsthead}.}%
    \entry{\PValue{right}}{%
      right-justified return address; an optional logo will be left justified;
      letterhead extensions will be activated}%
  \end{desctabular}
\end{table}
%
\EndIndexGroup


\begin{Declaration}
  \OptionVName{fromrule}{position}%
  \Variable{fromname}%
  \Variable{fromaddress}%
\end{Declaration}
\BeginIndex{}{letterhead}%
\BeginIndex{}{letter>head}%
The\important{\Variable{fromname}} sender's name will be determined by
variable \Variable{fromname}. Thereby the \PName{description} (see also
\autoref{tab:\LabelBase.fromTerm}, \autopageref{tab:\LabelBase.fromTerm}) will not
be used by the predefined letterheads.

At\important{\OptionValue{fromrule}{aftername}} the extended letterhead an
optional horizontal rule below the name may be selected using
\OptionValue{fromrule}{aftername}.
Alternatively\important[i]{\begin{tabular}{@{}l@{}}
    \KOption{fromrule}\\\quad\PValue{afteraddress}\end{tabular}} this rule may
be placed below the while sender using \OptionValue{fromrule}{afteraddress}. A
summary of all available rule position settings shows
\autoref{tab:\LabelBase.fromrule}. The length of this rule is determined by
pseudo-length \PLength{fromrulewidth}\IndexPLength[indexmain]{fromrulewidth}.

\begin{table}
  \caption[{Possible values of option \Option{fromrule} with
    \Class{scrlttr2}}]{Possible values of option \Option{fromrule} for the
    position of the rule in the from address with
    \Class{scrlttr2}}
  \label{tab:\LabelBase.fromrule}
  \begin{desctabular}
  \entry{\PValue{afteraddress}, \PValue{below}, \PValue{on}, \PValue{true},
    \PValue{yes}}{%
    rule below the return address}%
  \entry{\PValue{aftername}}{%
    rule directly below the sender's name}%
  \entry{\PValue{false}, \PValue{no}, \PValue{off}}{%
    no rule}%
  \end{desctabular}
\end{table}

Default for the rule at the extended letterhead is \PValue{false}. But at the
standard letterhead the rule will always be placed below the sender's name.

The\important{\Variable{fromaddress}} sender's address follows below the
name. The \PName{content} of variable \Variable{fromaddress} determines this
address. The \PName{description} (see also \autoref{tab:\LabelBase.fromTerm})
will not be used at the predefined letterheads

\BeginIndexGroup
\BeginIndex{FontElement}{fromaddress}\LabelFontElement{fromaddress}%
\BeginIndex{FontElement}{fromname}\LabelFontElement{fromname}%
\BeginIndex{FontElement}{fromrule}\LabelFontElement{fromrule}%
The font of the whole address is determined by the element
\FontElement{fromaddress}\IndexFontElement{fromaddress}%
\important{\FontElement{fromaddress}}. Modifications to this may be defined
with element \FontElement{fromname}\IndexFontElement{fromname}%
\important{\FontElement{fromname}} for the sender's name and with element
\FontElement{fromrule}\IndexFontElement{fromrule}%
\important{\FontElement{fromrule}} for the rule, that may be activated using
option \Option{fromrule}. Nevertheless changing the font style of the rule
would make sense. But you may use the elements also to change the color,
e.\,g. color the rule gray instead of black. See \cite{package:xcolor} for
information about colors.%
%
\EndIndexGroup

\begin{Example}
  Let's now define the name of the sender at our letter example:
  \lstinputcode[xleftmargin=1em]{letter-8.tex}
  \begin{figure}
    \centering
    \frame{\includegraphics[width=.4\textwidth]{letter-8}}\quad
    \frame{\includegraphics[width=.4\textwidth]{letter-9}}
    \caption[{Example: letter with sender, addressee, opening, text, closing,
      postscript, distribution list, and enclosure}]
    {result of a small letter with sender, addressee, opening, text, closing,
      postscript, distribution list, and enclosure (date and folding marks are
      defaults of DIN-letters): at left one the standard letterhead using
      \OptionValueRef{\LabelBase}{fromalign}{false}, at right one the
      extended letterhead using \OptionValueRef{\LabelBase}{fromalign}{center}}
    \label{fig:\LabelBase.letter-8-9}
  \end{figure}
  First of all not the extended but the standard letterhead has been used. The
  result is shown at the left side of \autoref{fig:\LabelBase.letter-8-9}. The
  right side shows almost the same letter but with
  \OptionValueRef{\LabelBase}{fromalign}{center} and therefore with the extended
  letterhead. You may see, that this variation is without any rule.

  For the first time \autoref{fig:\LabelBase.letter-8-9} also shows a signature
  below the closing phrase. This has been generated automatically from the
  sender's name. More information about configuration of the signature may be
  found in  \autoref{sec:\LabelBase.closing} from
  \autopageref{sec:\LabelBase.closing} onward.

  Now, the letter with extended letterhead should use option \Option{fromrule}
  to print a rule below the sender's name:%
  \lstinputcode[xleftmargin=1em]{letter-11.tex}%
  The result may be found at the right side of
  \autoref{fig:\LabelBase.letter-10-11}.
  \begin{figure}
    \centering
    \frame{\includegraphics[width=.4\textwidth]{letter-10}}\quad
    \frame{\includegraphics[width=.4\textwidth]{letter-11}}
    \caption[{Example: letter with sender, separation rule, addressee,
      opening, text, closing, signature, postscript, distribution list,
      enclosure, and puncher hole mark}]
    {result of a small letter with sender, separation rule, addressee,
      opening, text, closing, signature, postscript, distribution list,
      enclosure and puncher hole mark (the date is a default of DIN-letters):
      at left one the standard letterhead using
      \OptionValueRef{\LabelBase}{fromalign}{false}, at right one the extended letterhead
      using \OptionValueRef{\LabelBase}{fromalign}{center}}
    \label{fig:\LabelBase.letter-10-11}
  \end{figure}
  In difference to this, the left letter has been set once again with the
  standard letter head, that does not react on the additional option.
\end{Example}

An\textnote{Attention!} important note concerns the sender's address: within
the sender's address, parts such as street, P.O.~Box, state, country, etc.,
are separated with a double backslash. Depending on how the sender's address
is used, this double backslash will be interpreted differently and therefore
is not strictly always a line break. Paragraphs, vertical spacing and the like
are usually not allowed within the sender's address declaration. One has to
have very good knowledge of \Class{scrlttr2} to use things like those
mentioned above, intelligently. Another point to note is the one should most
certainly set the variables for return address (see variable
\DescRef{\LabelBase.variable.backaddress}, \DescPageRef{\LabelBase.variable.backaddress}) and
signature (see variable \DescRef{\LabelBase.variable.signature},
\DescPageRef{\LabelBase.variable.signature}) oneself.%
%
\EndIndexGroup


\begin{Declaration}
  \OptionVName{symbolicnames}{simple switch}%
  \OptionVName{fromphone}{simple switch}%
  \OptionVName{frommobilephone}{simple switch}%
  \OptionVName{fromfax}{simple switch}%
  \OptionVName{fromemail}{simple switch}%
  \OptionVName{fromurl}{simple switch}%
  \Variable{fromphone}%
  \Variable{frommobilephone}%
  \Variable{fromfax}%
  \Variable{fromemail}%
  \Variable{fromurl}%
  \Variable{phoneseparator}%
  \Variable{mobilephoneseparator}%
  \Variable{faxseparator}%
  \Variable{emailseparator}%
  \Variable{urlseparator}%
\end{Declaration}%
\BeginIndex{}{letterhead}%
\BeginIndex{}{letter>head}%
Whether or not a telephone number\Index{telephone}\Index{phone}, a
mobile\ChangedAt{v3.12}{\Class{scrlttr2}} telephone number\Index{mobile
  phone}\Index{cell phone}\Index{cellphone}, a fax number\Index{fax}, an
e-mail address\Index{e-mail}, or a sender's URL should be set as part of the
letterhead may be configured by the options \Option{fromphone},
\Option{fromfax}, \Option{fromemail}, and \Option{fromurl}. Any standard value
for simple switches from \autoref{tab:truefalseswitch},
\autopageref{tab:truefalseswitch} may be assigned to these options. Default is
\PValue{false} for all of them. The \PName{contents} depends on the
corresponding variable. The default of the \PName{description} or title of
each variable may be found in \autoref{tab:\LabelBase.fromTerm}. The separators,
that will be inserted between \PName{description} and \PName{content}, may be
found in \autoref{tab:\LabelBase.fromSeparator}.

You may\ChangedAt{v3.12}{\Class{scrlttr2}}\important{\Option{symbolicnames}}
change the defaults for the description of the separator variables at once
using option \Option{symbolicnames}. The option understands the values for
simple switches from \autoref{tab:truefalseswitch},
\autopageref{tab:truefalseswitch}. Switching on the option replaces the
descriptions from the language depending terms \DescRef{scrlttr2-experts.cmd.emailname},
\DescRef{scrlttr2-experts.cmd.faxname}, \DescRef{scrlttr2-experts.cmd.mobilephonename}, and \DescRef{scrlttr2-experts.cmd.phonename} into symbols
of the package \Package{marvosym}\IndexPackage{marvosym}. Also the colon will
be removed from the content of the separator variables. And in this case the
description and the content of the URL separator will be
empty. Note\textnote{Attention!}, that you have to load package
\Package{marvosym} on your own, if you switch on \Option{symbolicnames}
first time after \Macro{begin}\PParameter{document}.

\begin{table}
  \centering
  \caption[{Predefined descriptions of the variables of the
    letterhead}]{Predefined descriptions of the variables of the letterhead
    (the description and contents of the separator variables may be found at
    \autoref{tab:\LabelBase.fromSeparator}}
  \begin{desctabular}[1.8em]
    \ventry{fromemail}{\DescRef{\LabelBase.cmd.usekomavar*}\PParameter{emailseparator}%
      \DescRef{\LabelBase.cmd.usekomavar}\PParameter{emailseparator}}%
    \ventry{fromfax}{\DescRef{\LabelBase.cmd.usekomavar*}\PParameter{faxseparator}%
      \DescRef{\LabelBase.cmd.usekomavar}\PParameter{faxseparator}}%
    \ventry{frommobilephone}{%
      \ChangedAt{v3.12}{\Class{scrlttr2}}%
      \DescRef{\LabelBase.cmd.usekomavar*}\PParameter{mobilephoneseparator}%
      \DescRef{\LabelBase.cmd.usekomavar}\PParameter{mobilephoneseparator}}%
    \ventry{fromname}{\DescRef{scrlttr2-experts.cmd.headfromname}}%
    \ventry{fromphone}{\DescRef{\LabelBase.cmd.usekomavar*}\PParameter{phoneseparator}%
      \DescRef{\LabelBase.cmd.usekomavar}\PParameter{phoneseparator}}%
    \ventry{fromurl}{\DescRef{\LabelBase.cmd.usekomavar*}\PParameter{urlseparator}%
      \DescRef{\LabelBase.cmd.usekomavar}\PParameter{urlseparator}}%
  \end{desctabular}
  \label{tab:\LabelBase.fromTerm}
\end{table}

\begin{table}[tp]
%  \centering
  \KOMAoptions{captions=topbeside}%
  \setcapindent{0pt}%
%  \caption
  \begin{captionbeside}{Predefined description and content of the separators
      at the letterhead without option \Option{symbolicnames}}
    [l]
  \begin{tabular}[t]{lll}
    \toprule
    variable name             & description       & content \\
    \midrule
    \Variable{emailseparator} & \DescRef{scrlttr2-experts.cmd.emailname} & \texttt{:\~} \\
    \Variable{faxseparator}   & \DescRef{scrlttr2-experts.cmd.faxname}   & \texttt{:\~} \\
    \Variable{mobilephoneseparator} & \DescRef{scrlttr2-experts.cmd.mobilephonename} & \Macro{usekomavaer}\PParameter{phoneseparator} \\
    \Variable{phoneseparator} & \DescRef{scrlttr2-experts.cmd.phonename} & \texttt{:\~} \\
    \Variable{urlseparator}   & \DescRef{scrlttr2-experts.cmd.wwwname}   & \texttt{:\~} \\
    \bottomrule
  \end{tabular}
  \end{captionbeside}
  \label{tab:\LabelBase.fromSeparator}
\end{table}

\begin{Example}
  Mr Public from the example letter has telephone and e-mail. He wants to show
  this also in the letterhead. Furthermore the separation rule should be
  placed below the letterhead. So he uses the corresponding options and
  defines the content of the needed variables:%
  \lstinputcode[xleftmargin=1em]{letter-12.tex}%
  Nevertheless the result at the left side of
  \autoref{fig:\LabelBase.letter-12-13} is not disillusioning: the options are
  ignored. That's the case because the additional variables and options will
  be used at the extended letterhead only. So option \DescRef{\LabelBase.option.fromalign} has to
  be used, like done at the right letter of
  \autoref{fig:\LabelBase.letter-12-13}.
  \begin{figure}
    \centering
    \frame{\includegraphics[width=.4\textwidth]{letter-12}}\quad
    \frame{\includegraphics[width=.4\textwidth]{letter-13}}
    \caption[{Example: letter with extended sender, separation rule, addressee,
      opening, text, closing, signature, postscript, distribution list,
      enclosure, and puncher hole mark; standard vs. extended letterhead}]
    {result of a small letter with sender, separation rule, addressee,
      opening, text, closing, signature, postscript, distribution list,
      enclosure and puncher hole mark (the date is a default of DIN-letters):
      at left one the standard letterhead using
      \OptionValueRef{\LabelBase}{fromalign}{false}, at right one the extended letterhead
      using \OptionValueRef{\LabelBase}{fromalign}{center}}
    \label{fig:\LabelBase.letter-12-13}
  \end{figure}
  \lstinputcode[xleftmargin=1em]{letter-13.tex}

  An arrangement of alternatives with left aligned using
  \OptionValueRef{\LabelBase}{fromalign}{left} and right aligned sender
  using \OptionValueRef{\LabelBase}{fromalign}{right} may be found in
  \autoref{fig:\LabelBase.letter-14-15}.
  \begin{figure}
    \centering
    \frame{\includegraphics[width=.4\textwidth]{letter-14}}\quad
    \frame{\includegraphics[width=.4\textwidth]{letter-15}}
    \caption[{Example: letter with extended sender, separation rule, addressee,
      opening, text, closing, signature, postscript, distribution list,
      enclosure, and puncher hole mark; left vs. right aligned letterhead}]
    {result of a small letter with sender, separation rule, addressee,
      opening, text, closing, signature, postscript, distribution list,
      enclosure and puncher hole mark (the date is a default of DIN-letters):
      at left one left aligned letterhead using
      \OptionValueRef{\LabelBase}{fromalign}{left}, at right one right aligned letterhead
      using \OptionValueRef{\LabelBase}{fromalign}{right}}
    \label{fig:\LabelBase.letter-14-15}
  \end{figure}
\end{Example}
%
\EndIndexGroup


\begin{Declaration}
  \OptionVName{fromlogo}{simple switch}%
  \Variable{fromlogo}%
\end{Declaration}
\BeginIndex{}{letterhead}%
\BeginIndex{}{letter>head}%
With option \Option{fromlogo} you may configure whether or not to use a
logo\Index{Logo} at the letterhead. The option value may be any \PName{simple
  switch} from \autoref{tab:truefalseswitch},
\autopageref{tab:truefalseswitch}. Default is \PValue{false}, that means no
logo. The logo itself is defined by the \PName{content} of variable
\Variable{fromlogo}. The \PName{description} of the logo is empty by default
and \KOMAScript{} does not use it anywhere at the predefined note papers (see
also \autoref{tab:\LabelBase.fromTerm}).%
\begin{Example}
  Mr Public likes to use a letterhead with logo. He generated a graphics file
  with the logo and would like to include this using
  \Macro{includegraphics}. Therefor he uses the additional package
  \Package{graphics}\IndexPackage{graphics} (see \cite{package:graphics}).%
  \lstinputcode[xleftmargin=1em]{letter-16}%
  The result may be seen at the left top position of
  \autoref{fig:\LabelBase.letter-16-18}. The additional letter prints at this
  figure shows the result with right aligned or centered sender.
  \begin{figure}
    \setcapindent{0pt}%
    {\hfill
      \frame{\includegraphics[width=.4\textwidth]{letter-16}}\quad
      \frame{\includegraphics[width=.4\textwidth]{letter-17}}\par\bigskip}
    \begin{captionbeside}[{Example: letter with extended sender, logo,
        separation rule, addressee, opening, text, closing, signature,
        postscript, distribution list, enclosure, and puncher hole mark; left
        vs. right aligned vs. centered sender}]
      {result of a small letter with extended sender, logo, separation rule,
        addressee, opening, text, closing, signature, postscript, distribution
        list, enclosure and puncher hole mark (the date is a default of
        DIN-letters): at top left one left aligned sender using, at right
        beneath one with centered sender, and at right one right aligned
        sender}[l]
      \frame{\includegraphics[width=.4\textwidth]{letter-18}}\quad
    \end{captionbeside}
  \label{fig:\LabelBase.letter-16-18}
  \end{figure}
\end{Example}%
%
\EndIndexGroup


\begin{Declaration}
  \Variable{firsthead}%
\end{Declaration}
\BeginIndex{}{letterhead}%
\BeginIndex{}{letter>head}%
For many cases, \Class{scrlttr2} with its options and variables offers
enough possibilities to create a letterhead. In very rare situations
one may wish to have more freedom in terms of layout. In those
situations one will have to do without predefined letterheads, which
could have been chosen via options. Instead, one needs to create one's
own letterhead from scratch. To do so, one has to define the preferred
construction as content of variable \Variable{firsthead}. Within
\Macro{firsthead}, and with the help of the \Macro{parbox} command
(see \cite{latex:usrguide}), one can set several boxes side by side,
or one underneath the other. An advanced user will thus be able to
create a letterhead on his own. Of course the construct may
and should use other variables with the help of
\DescRef{\LabelBase.cmd.usekomavar}. \KOMAScript{} does not use the
\PName{description} of variable \Variable{firsthead}.
\iffree{}{A detailed example for the definition of a letterhead will be shown
  in \autoref{cha:modernletters}.}

The\textnote{Attention!} command
\Macro{firsthead}\IndexCmd[indexmain]{firsthead} exists only for reason of
compatibility to former \Class{scrlttr2} versions. However it is deprecated
and you should not use it anymore.%
%
\EndIndexGroup


\begin{Declaration}
  \OptionVName{addrfield}{mode}%
  \OptionVName{backaddress}{value}%
  \OptionVName{priority}{priority}%
  \Variable{toname}%
  \Variable{toaddress}%
  \Variable{backaddress}%
  \Variable{backaddressseparator}%
  \Variable{specialmail}%
  \Variable{fromzipcode}%
  \Variable{zipcodeseparator}%
  \Variable{place}%
  \Variable{PPcode}%
  \Variable{PPdatamatrix}%
  \Variable{addresseeimage}%
\end{Declaration}%
\BeginIndex{}{addressee}%
\BeginIndexGroup
\BeginIndex{FontElement}{addressee}\LabelFontElement{addressee}%
\BeginIndex{FontElement}{toname}\LabelFontElement{toname}%
\BeginIndex{FontElement}{toaddress}\LabelFontElement{toaddress}%
Option \Option{addrfield} defines whether or not to set an address
field. Default is to use an address field. This option can take the mode
values from
\autoref{tab:\LabelBase.addrfield}\ChangedAt{v3.03}{\Class{scrlttr2}}. Default
is \PValue{true}. With values \PValue{true},
\PValue{topaligned}\ChangedAt{v3.17}{\Class{scrlttr2}\and
  \Package{scrletter}}, \PValue{PP}, and
\PValue{backgroundimage} name and address of the addressee will be defined by
the mandatory argument of the \DescRef{\LabelBase.env.letter} environment (see
\autoref{sec:\LabelBase.document},
\DescPageRef{\LabelBase.env.letter}). These elements will additionally be
copied into the variables \Variable{toname} and \Variable{toaddress}.  The
predefined font styles may be changed\ChangedAt{v2.97c}{\Class{scrlttr2}} by
execution of command \DescRef{\LabelBase.cmd.setkomafont} or \DescRef{\LabelBase.cmd.addtokomafont} (siehe
\autoref{sec:\LabelBase.textmarkup},
\DescPageRef{\LabelBase.cmd.setkomafont}). Thereby three elements
exist. First of all element
\FontElement{addressee}\important{\FontElement{addressee}}, that is
responsible for the addressee overall. The additional elements
\FontElement{toname}\important{\FontElement{toname}} and
\FontElement{toaddress}\important{\FontElement{toaddress}} are responsible
only either for the name or the address of the addressee. They may be used to
define modifications from the \FontElement{addressee} configuration.%
\EndIndexGroup

\begin{table}
  \caption[{available values for option \Option{addrfield} using
    \Class{scrlttr2}}]{available values for option \Option{addrfield} to
    change the addressee mode of \Class{scrlttr2}}%
    \label{tab:\LabelBase.addrfield}%
  \begin{desctabular}
    \entry{\PValue{backgroundimage}, \PValue{PPbackgroundimage},
      \PValue{PPBackgroundImage}, \PValue{PPBackGroundImage},
      \PValue{ppbackgroundimage}, \PValue{ppBackgroundImage},
      \PValue{ppBackGroundImage}}{%
      Prints an address field with Port-Pay\'e head, in variable
      \Variable{addresseimage} defined background graphics, but without return
      address or mode of dispatch.}%
    \entry{\PValue{false}, \PValue{off}, \PValue{no}}{%
      Omits the address field.}%
    \entry{\PValue{image}, \PValue{Image}, \PValue{PPimage}, \PValue{PPImage},
      \PValue{ppimage}, \PValue{ppImage}}{%
      Prints variable \Variable{addresseeimage} as addressee with Port-Pay\'e,
      but ignores addressee information and definitions for return address,
      mode of dispatch or priority.}%
    \entry{\PValue{PP}, \PValue{pp}, \PValue{PPexplicite},
      \PValue{PPExplicite}, \PValue{ppexplicite}, \PValue{ppExplicite}}{%
      Prints an address field with Port-Pay\'e head, defined by the variables
      \Variable{fromzipcode}, \Variable{place}, and \Variable{PPcode} and when
      indicated with priority and data array defined by
      \Variable{PPdatamatrix} but without return address and mode of
      dispatch.}%
    \entry{\PValue{topaligned}, \PValue{alignedtop}%
      \ChangedAt{v3.17}{\Class{scrlttr2}\and \Package{scrletter}}}{%
      Prints an address field including a return address and a mode of
      dispatch or priority without centring the address vertically in the
      available space.}%
    \entry{\PValue{true}, \PValue{on}, \PValue{yes}}{%
      Prints an address field including a return address and a mode of
      dispatch or priority.}%
  \end{desctabular}
\end{table}%

With the default \OptionValue{addrfield}{true} an additional return address
will be printed. Option \Option{backaddress} defines whether a return address
for window envelopes will be set. This
option\important{\OptionValue{backaddress}{false}} can take the standard
values for simple switches, as listed in \autoref{tab:truefalseswitch},
\autopageref{tab:truefalseswitch}. These values do not change style of the
return address. On the other hand,
additionally\ChangedAt{v2.96}{\Class{scrlttr2}} to switching on the return
address, the style of the return address may be selected. Option value
\PValue{underlined} selects an underlined return address. In opposite to
this value \PValue{plain}\important{\OptionValue{backaddress}{plain}} selects
a style without underline. Default is \PValue{underlined} and therefore
printing an underlined return address.

\BeginIndex{FontElement}{backaddress}\LabelFontElement{backaddress}%
The return address itself is defined by the \PName{content} of variable
\Variable{backaddress}. Default is a combination of variable \Variable{toname}
and \Variable{toaddress} with redefinition of the double backslash to set the
\PName{content} of variable \Variable{backaddressseparator}. The predefined
separator \PName{content} is a comma followed by a non-breakable white
space. The \PName{description} of variable \Variable{backaddress} is not used
by \KOMAScript.  The font style of the return address is configurable via
element
\FontElement{backaddress}\important{\FontElement{backaddress}}. Default for
this is \Macro{sffamily} (see
\autoref{tab:\LabelBase.AddresseeElements}). Before execution of the element's
font style \KOMAScript{} switches to \Macro{scriptsize}.%
\EndIndex{FontElement}{backaddress}%

By default, \OptionValue{addrfield}{true}, the address will be vertically
centred in the available space below the mode of dispatch or
priority. In\ChangedAt{v3.17}{\Class{scrlttr2}\and \Package{scrletter}}
opposite to this, \OptionValue{addrfield}{topaligned}%
\important{\OptionValue{addrfield}{topaligned}} will not centre the address
but follow it aligned top at the available space. This is the only difference
to \OptionValue{addrfield}{true}.%

\begin{table}
%  \centering
  \KOMAoptions{captions=topbeside}%
  \setcapindent{0pt}%
%  \caption
  \begin{captionbeside}{%
      Predefined font style for the elements of the address field.%
    }%
    [l]
  \begin{tabular}[t]{ll}
    \toprule
    element & font style \\
    \midrule
    \DescRef{\LabelBase.fontelement.addressee}\IndexFontElement{addressee} & 
    \\
    \DescRef{\LabelBase.fontelement.backaddress}\IndexFontElement{backaddress} & 
    \Macro{sffamily}%
    \\
    \DescRef{\LabelBase.fontelement.PPdata}\IndexFontElement{PPdata} &
    \Macro{sffamily}%
    \\
    \DescRef{\LabelBase.fontelement.PPlogo}\IndexFontElement{PPlogo} &
    \Macro{sffamily}\Macro{bfseries}%
    \\
    \DescRef{\LabelBase.fontelement.priority}\IndexFontElement{priority} &
    \Macro{fontsize}\PParameter{10pt}\PParameter{10pt}%
    \Macro{sffamily}\Macro{bfseries}%
    \\
    \DescRef{\LabelBase.fontelement.prioritykey}\IndexFontElement{prioritykey} &
    \Macro{fontsize}\PParameter{24.88pt}\PParameter{24.88pt}%
    \Macro{selectfont}%
    \\
    \DescRef{\LabelBase.fontelement.specialmail}\IndexFontElement{specialmail} & 
    \\
    \DescRef{\LabelBase.fontelement.toaddress}\IndexFontElement{toaddress} & 
    \\
    \DescRef{\LabelBase.fontelement.toname}\IndexFontElement{toname} & 
    \\
    \bottomrule
  \end{tabular}
  \end{captionbeside}
  \label{tab:\LabelBase.AddresseeElements}%
\end{table}

\BeginIndex{FontElement}{specialmail}\LabelFontElement{specialmail}%
At the default \OptionValue{addrfield}{true} an optional dispatch
type\Index{mode of dispatch}\Index{dispatch type} can be output within the
address field between the return address and the addressee address, This will
be done only if variable \Variable{specialmail} is not empty and
\OptionValue{priority}{manual}\ChangedAt{v3.03}{\Class{scrlttr2}} has been
selected, which is also the default. Class \Class{scrlttr2} itself does not use
the \PName{description} of variable \Variable{specialmail}. The alignment is
defined by the pseudo-lengths PLength{specialmailindent} and
\PLength{specialmailrightindent} (siehe
\DescPageRef{scrlttr2-experts.plength.specialmailindent}). The predefined
font style of element\ChangedAt{v2.97c}{\Class{scrlttr2}}
\FontElement{specialmail}\important{\FontElement{specialmail}}, that has been
listed in \autoref{tab:\LabelBase.AddresseeElements}, may be changed using
commands \DescRef{\LabelBase.cmd.setkomafont} and \DescRef{\LabelBase.cmd.addtokomafont} (see
\autoref{sec:\LabelBase.textmarkup},
\DescPageRef{\LabelBase.cmd.setkomafont}).%
\EndIndex{FontElement}{specialmail}%

\BeginIndexGroup
\BeginIndex{FontElement}{priority}\LabelFontElement{priority}%
\BeginIndex{FontElement}{prioritykey}\LabelFontElement{prioritykey}%
\BeginIndex{FontElement}{PPlogo}\LabelFontElement{PPlogo}%
On\ChangedAt{v3.03}{\Class{scrlttr2}}\important[i]{\OptionValue{priority}{A}\\
  \OptionValue{priority}{B}} the other hand, using an international priority
with \OptionValue{priority}{A} or \OptionValue{priority}{B} (see
\autoref{tab:\LabelBase.priority}) together with \OptionValue{addrfield}{true}
will print the priority as mode of dispatch. Using it together with
\OptionValue{addrfield}{PP}\important{\OptionValue{addrfield}{PP}} will print
the priority at the corresponding position in the Port-Pay\'e head. Thereby
the element \FontElement{priority} defines the basic font style and
\FontElement{prioritykey} the modification of the basic font style for the
priority key, ``A'' or ``B''. The Port-Pay\'e logo ``P.~P.'' uses the font
style of element \FontElement{PPlogo}. The default font styles are listed in
\autoref{tab:\LabelBase.AddresseeElements} and may be changed using commands
\DescRef{\LabelBase.cmd.setkomafont} und
\DescRef{\LabelBase.cmd.addtokomafont} (siehe
\autoref{sec:\LabelBase.textmarkup},
\DescPageRef{\LabelBase.cmd.setkomafont}).%
\EndIndexGroup


\begin{table}
  \caption[{available values for option \Option{priority} in
    \Class{scrlttr2}}]{available values for option \Option{priority} to select
    the international priority at the address field of\Class{scrlttr2}}
  \label{tab:\LabelBase.priority}
  \begin{desctabular}
    \entry{\PValue{false}, \PValue{off}, \PValue{no}, \PValue{manual}}{%
      Priority will not be printed.}%
    \entry{\PValue{B}, \PValue{b}, \PValue{economy}, \PValue{Economy},
      \PValue{ECONOMY}, \PValue{B-ECONOMY}, \PValue{B-Economy}, 
      \PValue{b-economy}}{%
      Use international priority B-Economy. With
      \OptionValue{addrfield}{true} this will be printed instead of the mode
      of dispatch.}%
    \entry{\PValue{A}, \PValue{a}, \PValue{priority}, \PValue{Priority},
      \PValue{PRIORITY}, \PValue{A-PRIORITY}, \PValue{A-Priority}, 
      \PValue{a-priority}}{%
      Use international priority A-Priority. With
      \OptionValue{addrfield}{true} this will be printed instead of the mode
      of dispatch.}%
  \end{desctabular}
\end{table}

With\ChangedAt{v3.03}{\Class{scrlttr2}}\important{\OptionValue{addrfield}{PP}}
\OptionValue{addrfield}{PP} also the zip-code of
variable \Variable{fromzipcode} and the place from \PName{content} of variable
\Variable{place} will be set in the Port-Pay\'e head. Thereby the
\PName{content} of variable \Variable{fromzipcode} will be preceded by the
\PName{description} of variable \Variable{fromzipcode} and the \PName{content}
of variable \Variable{zipcodeseparator}. The predefined \PName{description}
depends on the used \File{lco}-file (see \autoref{sec:\LabelBase.lcoFile} from
\autopageref{sec:\LabelBase.lcoFile} onward). On the other hand the default of
the \PName{content} of variable \Variable{zipcodeseparator} is a small
distance followed by an endash followed by one more small distance
(\Macro{,}\texttt{-{}-}\Macro{,}).

Beyond\ChangedAt{v3.03}{\Class{scrlttr2}} that, with
\OptionValue{addrfield}{PP}\important{\OptionValue{addrfield}{PP}} and
sender's identification code will be used at the Port-Pay\'e head. This is the
\PName{content} of variable \Variable{PPcode}. Right beside the address of the
addressee an additional data array may be printed. The \PName{content} of
variable \Variable{PPdatamatrix} will be used for this.

\BeginIndex{FontElement}{PPdata}\LabelFontElement{PPdata}%
Zip-code\ChangedAt{v3.03}{\Class{scrlttr2}}, place and code will be printed
with default font size 8\Unit{pt}. Thereby the font style of element
\FontElement{PPdata}\important{\FontElement{PPdata}} will be used. The default
of the element is listed in \autoref{tab:\LabelBase.AddresseeElements} and may
be changed using commands \DescRef{\LabelBase.cmd.setkomafont} and \DescRef{\LabelBase.cmd.addtokomafont} (see
\autoref{sec:\LabelBase.textmarkup},
\DescPageRef{\LabelBase.cmd.setkomafont}).%
\EndIndex{FontElement}{PPdata}%

With\important[i]{\begin{tabular}[t]{@{}l@{}}
    \KOption{addrfield}\\\quad\PValue{backgroundimage}\end{tabular}\strut\\
  \strut\OptionValue{addrfield}{image}} options
\OptionValue{addrfield}{backgroundimage}\ChangedAt{v3.03}{\Class{scrlttr2}}
or \OptionValue{addrfield}{image} a picture will be print inside the address
field. The \PName{content} of variable \Variable{addresseeimage} will be used
for this. \KOMAScript{} does not use the \PName{description} of that
variable. Nothing else but the picture will be printed with option
\OptionValue{addrfield}{image}. But with option
\OptionValue{addrfield}{backgroundimage} the addressee's name and address from
the mandatory argument of the \DescRef{\LabelBase.env.letter} environment will
be output.

The alignment of the Port-Pay\'e head as long as the alignment of the
Port-Pay\'e address is defined by pseudo-length \PLength{toaddrindent} (see
\DescPageRef{scrlttr2-experts.plength.toaddrindent}) as well as
\PLength{PPheadwidth} and \PLength{PPheadheight} (siehe
\DescPageRef{scrlttr2-experts.plength.PPheadheight}). The alignment of
the data array is defined by pseudo-length \PLength{PPdatamatrixvskip}
(see \DescPageRef{scrlttr2-experts.plength.PPdatamatrixvskip}).

Please note\textnote{Attention!} that \KOMAScript{} itself cannot set any
external graphics or pictures. So, if you want to put external picture files
into variables like \Variable{addresseeimage} or \Variable{PPdatamatrix}, you
have to use an additional graphics package like
\Package{graphics}\IndexPackage{graphics} or
\Package{graphicx}\IndexPackage{graphicx} to use, i.\,e., the command
\Macro{includegraphics} at the \PName{content} of the variables.%
%
\EndIndexGroup


\begin{Declaration}
  \OptionVName{locfield}{selection}
\end{Declaration}
\BeginIndex{}{sender>additional information}%
\Class{scrlttr2} places a field with additional sender attributes next to the
address field. This can be used, for example, for bank account
or similar additional information.
Depending\important[i]{\OptionValueRef{\LabelBase}{fromalign}{center}\\
  \begin{tabular}[t]{@{}l@{}}
    \KOption{fromalign}\\\quad\PValue{locationleft}\end{tabular}\\
  \begin{tabular}[t]{@{}l@{}}
    \KOption{fromalign}\\\quad\PValue{locationright}\end{tabular}} on the
\DescRef{\LabelBase.option.fromalign} option, it will also be used for the
sender logo. The width of this field may be defined within an \File{lco} file
(see \autoref{sec:\LabelBase.lcoFile}). If the width is set to 0 in that file,
then the \Option{locfield} option can toggle between two presets for
the field width. See the explanation on the \PLength{locwidth} pseudo
length in \autoref{sec:scrlttr2-experts.locationField},
\DescPageRef{scrlttr2-experts.plength.locwidth}. Possible values for this
option are shown in \autoref{tab:\LabelBase.locfield}. Default is
\PValue{narrow}.%
%
\begin{table}
%  \centering
  \KOMAoptions{captions=topbeside}%
  \setcapindent{0pt}%
%  \caption
  \begin{captionbeside}
    [{Possible values of option \Option{locfield} with
      \Class{scrlttr2}}]{Possible values of option \Option{locfield} for
      setting the width of the field with additional sender attributes with
      \Class{scrlttr2}%
      \label{tab:\LabelBase.locfield}}%
    [l]
    \begin{minipage}[t]{.45\linewidth}
      \begin{desctabular}[t]
        \pventry{narrow}{narrow sender supplement field}%
        \pventry{wide}{wide sender supplement field}%
      \end{desctabular}
    \end{minipage}
  \end{captionbeside}
\end{table}

\begin{Declaration}
  \Variable{location}%
\end{Declaration}
\BeginIndex{}{sender>additional information}%
The contents of the sender's extension field is determined by the
variable \Variable{location}. To set this variable's \PName{content}, it is
permitted to use formatting commands like \Macro{raggedright}. \KOMAScript{}
does not use the \PName{description} of the variable.

\begin{Example}
  Mr Public likes to show some additional information about his
  membership. Therefor he uses the \Variable{location} variable:%
  \lstinputcode[xleftmargin=1em]{letter-19.tex}%
  This will define the field beside the addressee's address like shown in
  \autoref{fig:\LabelBase.letter-19}.
  \begin{figure}
    \setcapindent{0pt}%
    \begin{captionbeside}[{Example: letter with extended sender, logo,
        addressee, additional sender information, opening,
        text, closing, signature, postscript, distribution list, enclosure,
        and puncher hole mark}]
      {result of a small letter with extended sender, logo,
        addressee, additional sender information, opening, text, closing,
        signature, postscript, distribution list, enclosure and puncher hole
        mark (the date is a default of DIN-letters)}[l]
      \frame{\includegraphics[width=.4\textwidth]{letter-19}}
    \end{captionbeside}
    \label{fig:\LabelBase.letter-19}
  \end{figure}
\end{Example}
%
\EndIndexGroup
%
\EndIndexGroup


\begin{Declaration}
  \OptionVName{numericaldate}{simple switch}
\end{Declaration}
This option toggles between the standard, language-dependent
date\Index{date}\Index{reference line} presentation, and a short, numerical
one. {\KOMAScript} does not provide the standard presentation. It should be
defined by packages such as \Package{ngerman}\IndexPackage{ngerman},
\Package{babel}\IndexPackage{babel}, or
\Package{isodate}\IndexPackage{isodate}.
The\important{\OptionValue{numericaldate}{true}} short, numerical
presentation, on the other hand, is produced by \Class{scrlttr2} itself. This
option can take the standard values for simple switches, as listed in
\autoref{tab:truefalseswitch}, \autopageref{tab:truefalseswitch}. Default is
\PValue{false}, which results in standard date presentation.

\begin{Declaration}
  \Variable{date}%
\end{Declaration}
The date in the selected presentation will become the \PName{content} of
variable \Variable{date}. The selection of option
\DescRef{\LabelBase.option.numericaldate} does not influence the date any
longer, if the \PName{content} of this variable will be changed by the
user. Normally the date will be part of the reference line. If all other
elements of the reference line will be empty and therefore unused a date line
instead of a reference line will be printed. See the description of variable
\DescRef{\LabelBase.variable.place} on
\DescPageRef{\LabelBase.variable.placeseparator} for more information about
the date line. You should note, that you can change the automatic printing of
the date using option \DescRef{\LabelBase.option.refline}, that will be described next.
%
\EndIndexGroup
\EndIndexGroup


\begin{Declaration}
  \OptionVName{refline}{selection}
\end{Declaration}
\BeginIndex{}{reference line}%
\index{business line|see{reference line}}% |
Especially in business letters a line with information like identification
code\Index{identification>code}, direct dial, customer's number, invoice's
number, or references to previous letters may be found usually. This line will
be named \emph{reference line}\textnote{reference line} in this manual.

With the \Class{scrlttr2} class, the header, footer, address, and sender
attributes may extend beyond the normal type area to the left and to the
right. Option \OptionValue{refline}{wide} defines that this should also apply to
the reference fields line. Normally, the reference fields line contains at
least the date, but it can hold additional data. Possible values for this
option are shown in \autoref{tab:\LabelBase.refline}. Default is \PValue{narrow}
and \PValue{dateright}\ChangedAt{v3.09}{\Class{scrlttr2}}.%
%
\begin{table}
  \caption[{Possible value of option \Option{refline} with
    \Class{scrlttr2}}]{Possible value of option \Option{refline} for setting
    the width of the reference fields line with
    \Class{scrlttr2}}
  \label{tab:\LabelBase.refline}
  \begin{desctabular}
    \pventry{dateleft}{\ChangedAt{v3.09}{\Class{scrlttr2}}%
      The date will be placed leftmost at the reference line.}%
    \pventry{dateright}{\ChangedAt{v3.09}{\Class{scrlttr2}}%
      The date will be placed rightmost at the reference line.}%
    \pventry{narrow}{The reference line will be restricted to type
      area.}%
    \pventry{nodate}{\ChangedAt{v3.09}{\Class{scrlttr2}}%
      The date is not placed automatically into the reference line.}%
    \pventry{wide}{The with of the reference line corresponds to address and
      sender's additional information.}%
  \end{desctabular}
\end{table}

\begin{Declaration}
  \Variable{yourref}%
  \Variable{yourmail}%
  \Variable{myref}%
  \Variable{customer}%
  \Variable{invoice}%
\end{Declaration}
These five variables represent typical fields of the reference line. Their
meanings are given in \autoref{tab:\LabelBase.variables} on
\autopageref{tab:\LabelBase.variables}. Each variable has also a predefined
\PName{description}, shown in \autoref{tab:\LabelBase.reflineTerm}. Additional
information about adding other variables to the reference line may be found
in \autoref{sec:scrlttr2-experts.variables} from
\DescPageRef{scrlttr2-experts.cmd.newkomavar} onward.%
%
\begin{table}
%  \centering
  \KOMAoptions{captions=topbeside}%
  \setcapindent{0pt}%
%  \caption
  \begin{captionbeside}[{predefined descriptions of variables of the reference
      line}]{predefined descriptions of typical variables of the reference
      fields line using macros depending on the current language}%
    [l]
  \begin{tabular}[t]{lll}
    \toprule
    variable name       & description          & e.\,g., in english \\
    \midrule
    \Variable{yourref}  & \DescRef{scrlttr2-experts.cmd.yourrefname}  & Your reference \\
    \Variable{yourmail} & \DescRef{scrlttr2-experts.cmd.yourmailname} & Your letter from \\
    \Variable{myref}    & \DescRef{scrlttr2-experts.cmd.myrefname}    & Our reference \\
    \Variable{customer} & \DescRef{scrlttr2-experts.cmd.customername} & Customer No.: \\
    \Variable{invoice}  & \DescRef{scrlttr2-experts.cmd.invoicename}  & Invoice No.: \\
    \DescRef{\LabelBase.variable.date}     & \DescRef{scrlttr2-experts.cmd.datename}     & date \\
    \bottomrule
  \end{tabular}
  \end{captionbeside}
  \label{tab:\LabelBase.reflineTerm}
\end{table}

\BeginIndex{FontElement}{refname}\LabelFontElement{refname}%
\BeginIndex{FontElement}{refvalue}\LabelFontElement{refvalue}%
Font style and color\ChangedAt{v2.97c}{\Class{scrlttr2}} of the
\PName{description} and \PName{content} of the fields of the reference line
may be changed with elements \FontElement{refname}%
\important[i]{\FontElement{refname}\\\FontElement{refvalue}} and
\FontElement{refvalue}. Therefor the commands \DescRef{\LabelBase.cmd.setkomafont} and \DescRef{\LabelBase.cmd.addtokomafont} (see
\autoref{sec:\LabelBase.textmarkup},
\DescPageRef{\LabelBase.cmd.setkomafont}) should be used. The default
configuration of both elements is listed in
\autoref{tab:\LabelBase.refnamerefvalue}.%
\begin{table}[tp]
%  \centering
  \KOMAoptions{captions=topbeside}%
  \setcapindent{0pt}%
%  \caption
  \begin{captionbeside}
    [{font style default of elements of the reference line}]{default font
      style configuration of the elements of the reference line%
      \label{tab:\LabelBase.refnamerefvalue}}
    [l]
    \begin{tabular}[t]{ll}
      \toprule
      element & default configuration \\
       \midrule
      \DescRef{\LabelBase.fontelement.refname} & \Macro{sffamily}\Macro{scriptsize} \\
      \DescRef{\LabelBase.fontelement.refvalue} & \\
      \bottomrule
    \end{tabular}
  \end{captionbeside}
\end{table}%
%
\EndIndex{FontElement}{refvalue}%
\EndIndex{FontElement}{refname}%
\EndIndexGroup


\begin{Declaration}
  \Variable{placeseparator}%
\end{Declaration}%
\BeginIndex{Variable}{place}%
If all variables of the reference line are empty, the line will be omitted.
In this case, the \PName{content} of \DescRef{\LabelBase.variable.place} and
\Variable{placeseparator} will be set, followed by the \PName{content} of
\DescRef{\LabelBase.variable.date}. The predefined \PName{content} of the
\Variable{placeseparator} is a comma followed by a non-breaking space. If the
variable \DescRef{\LabelBase.variable.place} has no \PName{content} value then
the hyphen remains unset also.  The predefined \PName{content} of
\DescRef{\LabelBase.variable.date} is \Macro{today}\IndexCmd{today} and
depends on the setting of the option \DescRef{\LabelBase.option.numericaldate}
(see \DescPageRef{\LabelBase.option.numericaldate}).

Since version~3.09\ChangedAt{v3.09}{\Class{scrlttr2}} place and date the
alignment of place and date is given by option \DescRef{\LabelBase.option.refline}. The available
alignment values for this option are listed in \autoref{tab:\LabelBase.refline}.

\BeginIndex{FontElement}{placeanddate}\LabelFontElement{placeanddate}%
The\ChangedAt{v3.12}{\Class{scrlttr2}} font setting of place and date in the
date line are given by font
element\FontElement{placeanddate}\important[i]{\FontElement{placeanddate}}
instead of element \DescRef{\LabelBase.fontelement.refvalue}, which would be used for general
reference lines. You can change the empty default of the font element using
commands \DescRef{\LabelBase.cmd.setkomafont} and \DescRef{\LabelBase.cmd.addtokomafont} (see
\autoref{sec:\LabelBase.textmarkup},
\DescPageRef{\LabelBase.cmd.setkomafont}).%
\EndIndex{FontElement}{placeanddate}%

\begin{Example}
  Now Mr Public also sets the place:%
  \lstinputcode[xleftmargin=1em]{letter-20.tex}%
  In this case \autoref{fig:\LabelBase.letter-20} shows the place and the
  automatic separator in front of the date. The date has been defined
  explicitly to make the printed date independent from the date of the
  \LaTeX{} run.
  \begin{figure}
    \setcapindent{0pt}%
    \begin{captionbeside}[{Example: letter with extended sender, logo,
        addressee, additional sender information, place, date, opening,
        text, closing, signature, postscript, distribution list, enclosure,
        and puncher hole mark}]
      {result of a small letter with extended sender, logo, addressee,
        additional sender information, place, date, opening, text, closing,
        signature, postscript, distribution list, enclosure and puncher hole
        mark}[l]
      \frame{\includegraphics[width=.4\textwidth]{letter-20}}
    \end{captionbeside}
    \label{fig:\LabelBase.letter-20}
  \end{figure}
\end{Example}
%
\EndIndexGroup
%
\EndIndexGroup


\begin{Declaration}
  \Variable{title}%
\end{Declaration}
\BeginIndex{}{title}%
\BeginIndex{FontElement}{lettertitle}\LabelFontElement{lettertitle}%
\BeginIndex{FontElement}{title}\LabelFontElement{title}%
With \Class{scrlttr2} a letter can carry an additional title. The title is
centered and set with font size \Macro{LARGE} directly after and beneath the
reference fields line.  The predefined font setup for the element
\FontElement{lettertitle}\important{\FontElement{lettertitle}} can be changed
with commands \DescRef{\LabelBase.cmd.setkomafont} und \DescRef{\LabelBase.cmd.addtokomafont} (see
\autoref{sec:\LabelBase.textmarkup},
\DescPageRef{\LabelBase.cmd.setkomafont}). Font size declarations are
allowed. Font size \Macro{LARGE} is not part of the predefined default
\Macro{normalcolor}\Macro{sffamily}\Macro{bfseries} but nevertheless will be
used before the font style of the element.  With \Class{scrlttr2} you can also
use \FontElement{title}\important{\FontElement{title}} as an alias of
\FontElement{lettertitle}. This is not provided using \Package{scrletter} with
a \KOMAScript{} class, because these classes already define an element
\FontElement{title} with different meaning for the document title.%
\EndIndex{FontElement}{title}%
\EndIndex{FontElement}{lettertitle}%
\begin{Example}
  Assume that you are to write a reminder. Thus you put as title:
\begin{lstlisting}
  \setkomavar{title}{Reminder}
\end{lstlisting}
  This way the addressee will recognize a reminder as such.
\end{Example}
Like shown in the example, the \PName{content} of the variable defines the
title. \KOMAScript{} will not use the \PName{description}.%
%
\EndIndexGroup


\begin{Declaration}
  \OptionVName{subject}{selection}%
  \Variable{subject}%
  \Variable{subjectseparator}%
\end{Declaration}
\BeginIndex{}{subject}%
In case a subject should be set, the \PName{content} of the variable
\Variable{subject} need to be defined. First of all with option
\OptionValue{subject}{true}\important{\OptionValue{subject}{titled}} the usage 
of the \PName{description} before the output of the subject may be
configured. See \autoref{tab:\LabelBase.subjectTerm} for the predefined
\PName{description}. In case of using the \PName{description} the
\PName{content} of variable \Variable{subjectseparator}\Index{separator} will
be set between the \PName{description} and \PName{content} of the
\Variable{subject}. The predefined \PName{content} of \PName{subjectseparator}
is a colon followed by a white space.

\begin{table}
%  \centering
  \KOMAoptions{captions=topbeside}%
  \setcapindent{0pt}%
%  \caption
  \begin{captionbeside}{predefined descriptions of subject-related variables}
    [l]
    \begin{tabular}[t]{lll}
      \toprule
      variable name      & description \\
      \midrule
      \Variable{subject} & \DescRef{\LabelBase.cmd.usekomavar*}\PParameter{subjectseparator}%
                           \texttt{\%} \\ 
                         & \texttt{\quad}%
                           \DescRef{\LabelBase.cmd.usekomavar}\PParameter{subjectseparator} \\
      \Variable{subjectseparator} & \DescRef{scrlttr2-experts.cmd.subjectname} \\
      \bottomrule
    \end{tabular}
  \end{captionbeside}
  \label{tab:\LabelBase.subjectTerm}
\end{table}

On the other hand, \OptionValue{subject}{afteropening}%
\important{\OptionValue{subject}{afteropening}} may be used to place the
subject below instead of before the letter opening.  Furthermore, the
formatting\important[i]{\OptionValue{subject}{underlined}\\
  \OptionValue{subject}{centered}\\
  \OptionValue{subject}{right}} of the subject may be changed using either
\OptionValue{subject}{underlined}, \OptionValue{subject}{centered}, or
\OptionValue{subject}{right}\ChangedAt{v2.97c}{\Class{scrlttr2}}. All
available values are listed in \autoref{tab:\LabelBase.subject}. Please
note\textnote{Attention!}, that with option \OptionValue{subject}{underlined}
the while subject must fit at one line! Defaults are
\OptionValue{subject}{left}, \OptionValue{subject}{beforeopening}, and
\OptionValue{subject}{untitled}.%
%
\begin{table}
%  \centering
  \KOMAoptions{captions=topbeside}%
  \setcapindent{0pt}%
%  \caption
  \begin{captionbeside}
    [{available values of option \Option{subject} with \Class{scrlttr2}}]
    {available values of option \Option{subject} for the position and
      formatting of the subject with
      \Class{scrlttr2}\label{tab:\LabelBase.subject}}%
    [l]
    \begin{minipage}[t]{.6\linewidth}
      \begin{desctabular}[t]
        \pventry{afteropening}{subject after opening}%
        \pventry{beforeopening}{subject before opening}%
        \pventry{centered}{subject centered}%
        \pventry{left}{subject left-justified}%
        \pventry{right}{subject right-justified}%
        \pventry{titled}{add title/description to subject}%
        \pventry{underlined}{set subject underlined (see note in text please)}%
        \pventry{untitled}{do not add title/description to subject}%
      \end{desctabular}
    \end{minipage}
  \end{captionbeside}
\end{table}

\BeginIndex{FontElement}{lettersubject}\LabelFontElement{lettersubject}%
\BeginIndex{FontElement}{subject}\LabelFontElement{subject}%
The subject line is set in a separate font\Index{font>style}. To change this
use the commands \DescRef{\LabelBase.cmd.setkomafont} and
\DescRef{\LabelBase.cmd.addtokomafont} (siehe
\autoref{sec:\LabelBase.textmarkup},
\DescPageRef{\LabelBase.cmd.setkomafont}). For element
\FontElement{lettersubject}\important{\FontElement{lettersubject}} the
predetermined font is \Macro{normalcolor}\Macro{bfseries}. With
\Class{scrlttr2} you can also use
\FontElement{subject}\important{\FontElement{subject}} as an alias of
\FontElement{lettersubject}. This is not provided using \Package{scrletter}
with a \KOMAScript{} class, because these classes already define an element
\FontElement{subject} with different meaning for the document title.%
\EndIndex{FontElement}{subject}%
\EndIndex{FontElement}{lettersubject}%
\begin{Example}
  Now, Mr Public sets a subject. He's a more traditional person, so he likes
  to have a kind of heading to the subject and therefor sets the corresponding
  option:%
  \lstinputcode[xleftmargin=1em]{letter-21.tex}%
  The result is shown by \autoref{fig:\LabelBase.letter-21}.
  \begin{figure}
    \setcapindent{0pt}%
    \begin{captionbeside}[{Example: letter with extended sender, logo,
        addressee, additional sender information, place, date, subject,
        opening, text, closing, signature, postscript, distribution list,
        enclosure, and puncher hole mark}]
      {result of a small letter with extended sender, logo, addressee,
        additional sender information, place, date, subject opening, text,
        closing, signature, postscript, distribution list, enclosure and
        puncher hole mark}[l]
      \frame{\includegraphics[width=.4\textwidth]{letter-21}}
    \end{captionbeside}
    \label{fig:\LabelBase.letter-21}
  \end{figure}
\end{Example}
%
\EndIndexGroup


\begin{Declaration}
  \OptionVName{firstfoot}{simple switch}
\end{Declaration}
\BeginIndex{}{letter>foot}%
\BeginIndex{}{letterfoot}%
This\ChangedAt{v2.97e}{\Class{scrlttr2}} option determines whether the
letterfoot is set or not. Switching off the letterfoot, e.\,g., using
\OptionValue{firstfoot}{false}\important{\OptionValue{firstfoot}{false}}, has
an effect when the option \DescRef{\LabelBase.option.enlargefirstpage} (see
\DescPageRef{\LabelBase.option.enlargefirstpage}) is used concurrently. In
this case the text area of the page will be enlarged down to the bottom. Then
the normal distance between typing area and the letter foot will become the
only distance remaining between the end of the typing area and the end of the
page.

The option understands the standard values for simple switches, as given in
\autoref{tab:truefalseswitch}, \autopageref{tab:truefalseswitch}. Default is
the setting of the letter foot.
\EndIndexGroup


\begin{Declaration}
  \Variable{firstfoot}%
\end{Declaration}%
\BeginIndex{}{letter>foot}%
\BeginIndex{}{letterfoot}%
The first page's footer is preset to
empty. However\ChangedAt{v3.08}{\Class{scrlttr2}}, a new construction may be
made at the \PName{content} of variable \Variable{firstfoot}. \KOMAScript{}
does not use the \PName{description} of the variable. For more information see
the example following the next description. Only for compatibility reason the
deprecated command \Macro{firstfoot}\IndexCmd[indexmain]{firstfoot} of
\Class{scrlttr2} before release~3.08 still exists. Nevertheless it should not
be used any longer.

\begin{Declaration}
  \Variable{frombank}%
\end{Declaration}%
This variable at the moment takes on a special meaning: it is not used
internally at this point, and the user can make use of it to set, for example,
his bank account\Index{bank account} within the sender's additional
information (see variable \DescRef{\LabelBase.variable.location},
\DescPageRef{\LabelBase.variable.location}) or the footer.%
%
\begin{Example}
  In the first page's footer, you may want to set the \PName{content} of the
  variable \Variable{frombank} (the bank account). The double
  backslash should be exchanged with a comma at the same time:
\begin{lstcode}
  \setkomavar{firstfoot}{%
    \parbox[b]{\linewidth}{%
      \centering\def\\{, }\usekomavar{frombank}%
    }%
  }
\end{lstcode}
  For the hyphen you might define a variable of your own if you like.
  This is left as an exercise for the reader.

  Nowadays it has become very common to create a proper footer in
  order to obtain some balance with respect to the letterhead. This
  can be done as follows:
\begin{lstcode}
  \setkomavar{firstfoot}{%
    \parbox[t]{\textwidth}{\footnotesize
      \begin{tabular}[t]{l@{}}%
        \multicolumn{1}{@{}l@{}}{Partners:}\\
        Jim Smith\\
        Russ Mayer
      \end{tabular}%
      \hfill
      \begin{tabular}[t]{l@{}}%
        \multicolumn{1}{@{}l@{}}{Manager:}\\
        Jane Fonda\\[1ex]
        \multicolumn{1}{@{}l@{}}{Court Of Jurisdiction:}\\
        Great Plains
      \end{tabular}%
      \ifkomavarempty{frombank}{}{%
        \hfill
        \begin{tabular}[t]{l@{}}%
          \multicolumn{1}{@{}l@{}}{\usekomavar*{frombank}:}\\
          \usekomavar{frombank}
        \end{tabular}%
      }%
    }%
  }
\end{lstcode}
  This example, by the way, came from Torsten Kr\"uger. With
\begin{lstcode}
  \setkomavar{frombank}{Account No. 12\,345\,678\\
    at Citibank\\
    bank code no: 876\,543\,21}
\end{lstcode}
  the bank account can be set accordingly.
%  If the footer will have such a large height then it might happen that you
%  have to shift its position. You can do this with the pseudo-length
%  \PLength{firstfootvpos}\IndexPLength{firstfootvpos}, which is
%  described above in this section.
\end{Example}
In the previous example a multi-line footer was set.  With a compatibility
setting to version 2.9u (see \DescRef{\LabelBase.option.version} in
\autoref{sec:\LabelBase.compatibilityOptions},
\DescPageRef{\LabelBase.option.version}) the space will in general not
suffice. In that case, you may need to reduce \PLength{firstfootvpos} (see
\DescPageRef{scrlttr2-experts.plength.firstfootvpos}) appropriately.%
%
\EndIndexGroup
%
\EndIndexGroup
%
\EndIndexGroup


\LoadCommonFile{parmarkup} % \section{Paragraph Markup}

\LoadCommonFile{oddorevenpage} % \section{Detection of Odd and Even Pages}


\section{Head and Foot Using Predefined Page Style}
\seclabel{pagestyle}
\BeginIndexGroup
\BeginIndex{}{page>style}%
\BeginIndex{}{page>head}%
\BeginIndex{}{page>foot}%

One of the general properties of a document is the page style. In {\LaTeX}
this means mostly the contents of headers and
footers. Like\textnote{Attention!} already mentioned in
\autoref{sec:\LabelBase.firstpage}, the head and foot of the note paper are
treated as elements of the note paper. Therefore they do not depend on the
settings of the page style. So this section describes the page styles of the
consecutive pages of letter after the note paper. At single-side letters this is
the page style of the secondary letter sheet. At double-side letters this is
also the page style of all backsides.


\begin{Declaration}
  \Macro{letterpagestyle}
\end{Declaration}
The\ChangedAt{v3.19}{\Class{scrlttr2}\and \Package{scrletter}} default page
style used for letters is specified by the contents of command
\Macro{letterpagestyle}. With \Class{scrlttr2} an empty value has been
predefined that does not change the
page style of the letter from the page style of the document. This has been
done, because \Class{scrlttr2} has been made for letter-only-documents. So it
is much easier to define the page style of all letters with the page style of
the document using \DescRef{\LabelBase.cmd.pagestyle} inside the preamble or before starting
the first letter.

The \PageStyle{plain} page style of letters differs from the \PageStyle{plain}
page style of most classes. Therefore package \Package{scrletter} defines
\Macro{letterpagestyle} with the contents
\PageStyle{plain.letter}\IndexPagestyle{plain.letter}. So all letters will use
the \PageStyle{plain} style of page style pair
\PageStyle{letter}\IndexPagestyle{letter} by default and independent from the
page style of the rest of the document. See
\autoref{sec:scrlttr2-experts.pagestyleatscrletter} for more information about
the characteristics of the page style of package \Package{scrletter}.
\begin{Example}
  You are using package \Package{scrletter} and want to have the same page
  style inside letters as for the rest of the document. So you are using
\begin{lstcode}
  \renewcommand*{\letterpagestyle}{}
\end{lstcode}
  in your document preamble. Take care about the star of
  \Macro{renewcommand*}!
\end{Example}
Nevertheless, if you use \DescRef{\LabelBase.cmd.pagestyle} or \DescRef{\LabelBase.cmd.thispagestyle} inside a
letter, this will overwrite the setting of \Macro{letterpagestyle} that is
only the default for the initialisation inside
\Macro{begin}\PParameter{document}.%
\EndIndexGroup


\begin{Declaration}
  \OptionVName{headsepline}{simple switch}%
  \OptionVName{footsepline}{simple switch}
\end{Declaration}
These two options select whether or not to insert a separator
line\Index{line>separator}\Index{rule>separator} below the header or above the
footer, respectively, on consecutive pages.  This option can take the standard
values for simple switches, as listed in \autoref{tab:truefalseswitch},
\autopageref{tab:truefalseswitch}.
Activation\important{\OptionValue{headsepline}{true}} of option
\Option{headsepline} switches on a rule below the page
head. Activation\important{\OptionValue{footsepline}{true}} of option
\Option{footsepline} switches on a rule above the page foot. Deactivation of
the options switches of the corresponding rules.

Obviously option \Option{headsepline} does not influence page style
\PValue{empty}\important[i]{\PageStyle{empty}} (see afterwards at this
section). This page style explicitly does not use any page head.

Typographically\important[i]{\PageStyle{headings}\\
  \PageStyle{myheadings}\\\PageStyle{plain}} such a rule make a hard optical
connection of head or foot and the text area. This would not mean, that the
distance between head and text or text and foot should be increased. Instead
of this the head or foot should be seen as parts of the typing area, while
calculation of margins and typing area. To achieve this \KOMAScript{} will
pass option \DescRef{typearea.option.headinclude}%
\important[i]{\DescRef{typearea.option.headinclude}\\
  \DescRef{typearea.option.footinclude}} or
\DescRef{typearea.option.footinclude}, respectively, to the \Package{typearea}
package, if option \Option{headsepline} or \Option{footsepline}, respectively,
are used as a class option. In\important{\PageStyle{plain}} opposite to
\Option{headsepline} option \Option{footsepline} does influence page style
\PValue{plain} also, because \PValue{plain} uses a page number at the
foot. Package \Package{scrlayer-scrpage}\IndexPackage{scrlayer-scrpage}%
\important{\Package{scrlayer-scrpage}} (see \autoref{cha:scrlayer-scrpage})
provides additional features for rules at head and foot and may be combined
with \Class{scrlttr2}.%
%
\EndIndexGroup


\begin{Declaration}
  \OptionVName{pagenumber}{position}
\end{Declaration}
This option defines if and where a page number will be placed on consecutive
pages. This option affects the page
styles\important[i]{\PageStyle{headings}\\
  \PageStyle{myheadings}\\
  \PageStyle{plain}} \PageStyle{headings}, \PageStyle{myheadings}, and
\PageStyle{plain}. It also affects the default page styles of the
\Package{scrlayer-scrpage}\important{\Package{scrlayer-scrpage}} package, if
set before loading the package (see \autoref{cha:scrlayer-scrpage}). It can
take values only influencing horizontal, only vertical, or both
positions. Available value are shown in
\autoref{tab:\LabelBase.pagenumber}. Default is \PValue{botcenter}.

\begin{table}
  \caption[{available values of option \Option{pagenumber} with
    \Class{scrlttr2}}]{available values of option \Option{pagenumber} for the
    position of the page number in page styles \PageStyle{headings},
    \PageStyle{myheadings}, and \PValue{plain} with \Class{scrlttr2}}
 \label{tab:\LabelBase.pagenumber}
  \begin{desctabular}
    \entry{\PValue{bot}, \PValue{foot}}{%
      page number in footer, horizontal position not changed}%
    \entry{\PValue{botcenter}, \PValue{botcentered}, \PValue{botmittle},
      \PValue{footcenter}, \PValue{footcentered}, \PValue{footmiddle}}{%
      page number in footer, centered}%
    \entry{\PValue{botleft}, \PValue{footleft}}{%
      page number in footer, left justified}%
    \entry{\PValue{botright}, \PValue{footright}}{%
      page number in footer, right justified}%
    \entry{\PValue{center}, \PValue{centered}, \PValue{middle}}{%
      page number centered horizontally, vertical position not changed}%
    \entry{\PValue{false}, \PValue{no}, \PValue{off}}{%
      no page number}%
    \entry{\PValue{head}, \PValue{top}}{%
      page number in header, horizontal position not changed}%
    \entry{\PValue{headcenter}, \PValue{headcentered}, \PValue{headmiddle},
      \PValue{topcenter}, \PValue{topcentered}, \PValue{topmiddle}}{%
      page number in header, centered}%
    \entry{\PValue{headleft}, \PValue{topleft}}{%
      page number in header, left justified}%
    \entry{\PValue{headright}, \PValue{topright}}{%
      page number in header, right justified}%
    \entry{\PValue{left}}{%
      page number left, vertical position not changed}%
    \entry{\PValue{right}}{%
      page number right, vertical position not changed}%
  \end{desctabular}
\end{table}
%
\EndIndexGroup


\begin{Declaration}
  \Macro{pagestyle}\Parameter{page style}%
  \Macro{thispagestyle}\Parameter{local page style}
\end{Declaration}%
In letters written with \Class{scrlttr2} there are four different
page styles.
\begin{description}
\item[{\PageStyle{empty}%
    \IndexPagestyle[indexmain]{empty}\LabelPageStyle{empty}}] is the page
  style, in which the header and footer of consecutive pages are completely
  empty. This page style is also used for the first page, because header and
  footer of this page are set by other means using the macro
  \DescRef{\LabelBase.cmd.opening}\IndexCmd{opening} (see
  \autoref{sec:\LabelBase.firstpage}, \DescPageRef{\LabelBase.cmd.opening}).
\item[{\PageStyle{headings}%
    \IndexPagestyle[indexmain]{headings}\LabelPageStyle{heading}}] is the page
  style for running (automatic) headings on consecutive pages. The inserted
  marks are the sender's name from the variable
  \DescRef{\LabelBase.variable.fromname}\IndexVariable{fromname} and the
  subject from the variable \DescRef{\LabelBase.variable.subject}\IndexVariable{subject} (see
  \autoref{sec:\LabelBase.firstpage},
  \DescPageRef{\LabelBase.variable.fromname} and
  \DescPageRef{\LabelBase.variable.subject}).  At which position these marks
  and the page numbers are placed, depends on the previously described option
  \DescRef{\LabelBase.option.pagenumber} and the \PName{content} of the variables
  \DescRef{\LabelBase.variable.nexthead}\IndexVariable{nexthead} and
  \DescRef{\LabelBase.variable.nextfoot}\IndexVariable{nextfoot}. The author can also change
  these marks manually after the \DescRef{\LabelBase.cmd.opening} command. To
  this end, the commands \DescRef{\LabelBase.cmd.markboth} and
  \DescRef{\LabelBase.cmd.markright} are available as usual, and with the use
  of package \Package{scrlayer-scrpage} also
  \DescRef{scrlayer-scrpage.cmd.markleft} (see
  \autoref{sec:scrlayer-scrpage.pagestyle.content},
  \DescPageRef{scrlayer-scrpage.cmd.automark}) is available.
\item[{\PageStyle{myheadings}%
    \IndexPagestyle[indexmain]{myheadings}\LabelPageStyle{myheadings}}] is the
  page style for manual page headings on consecutive pages. This is very
  similar to \PValue{headings}, but here the marks must be set by the author
  using the commands \DescRef{\LabelBase.cmd.markboth}\IndexCmd{markboth} and
  \DescRef{\LabelBase.cmd.markright}\Index{markright}.  With the use of package
  \Package{scrlayer-scrpage} also \DescRef{scrlayer-scrpage.cmd.markleft} can be utilized.
\item[{\PageStyle{plain}%
    \IndexPagestyle[indexmain]{plain}\LabelPageStyle{plain}}] is the page
  style with only page numbers in the header or footer on consecutive
  pages. The placement of these page numbers is influenced by the previously
  described option \DescRef{\LabelBase.option.pagenumber}.
\end{description}

Page styles are also influenced by the previously described
options\important[i]{\DescRef{\LabelBase.option.headsepline}\\
  \DescRef{\LabelBase.option.footsepline}}
\DescRef{\LabelBase.option.headsepline}\IndexOption{headsepline} and
\DescRef{\LabelBase.option.footsepline}\IndexOption{footsepline}. The page
style beginning with the current page is switched using \Macro{pagestyle}. In
contrast, \Macro{thispagestyle} changes only the page style of the current
page. The\textnote{Attention!} letter class itself uses
\Macro{thispagestyle}\PParameter{empty} within
\DescRef{\LabelBase.cmd.opening}\IndexCmd{opening} for the first page of the
letter.

\BeginIndexGroup
\BeginIndex[indexother]{}{font>style}%
\BeginIndex{FontElement}{pageheadfoot}\LabelFontElement{pageheadfoot}%
\BeginIndex{FontElement}{pagefoot}\LabelFontElement{pagefoot}%
\BeginIndex{FontElement}{pagenumber}\LabelFontElement{pagenumber}%
For changing the font style of headers or footers you should use the user
interface described in \autoref{sec:maincls.textmarkup}.  For header and
footer the same element is used, which you can name either
\FontElement{pageheadfoot}\important{\FontElement{pagehead}} or
\FontElement{pagehead}.  There\ChangedAt{v3.00}{\Class{scrlttr2}} is an
additional element \FontElement{pagefoot}\important{\FontElement{pagefoot}}
for the page foot. This will be used after \FontElement{pageheadfoot} at page
foots, that has been defined either with variable
\DescRef{\LabelBase.variable.nextfoot}\IndexVariable{nextfoot} or Package
\Package{scrlayer-scrpage}\IndexPackage{scrlayer-scrpage} (see
\autoref{cha:scrlayer-scrpage},
\DescPageRef{scrlayer-scrpage.fontelement.pagefoot}). The element for the
page number within the header or footer is named
\FontElement{pagenumber}\important{\FontElement{pagenumber}}. Default settings
are listed in \autoref{tab:maincls.defaultFontsHeadFoot},
\autopageref{tab:maincls.defaultFontsHeadFoot}.  Please have also a look at
the example in \autoref{sec:maincls.pagestyle},
\PageRefxmpl{maincls.cmd.pagestyle}.
%
\EndIndexGroup
%
\EndIndexGroup


\begin{Declaration}
  \Macro{markboth}\Parameter{left mark}\Parameter{right mark}%
  \Macro{markright}\Parameter{right mark}
\end{Declaration}
The possibilities that are offered with variables and options in
\Class{scrlttr2} should be good enough in most cases to create letterheads and
footers for the consecutive pages that follow the first letter page. Even more
so since you can additionally change with \Macro{markboth} and
\Macro{markright} the sender's statements that \Class{scrlttr2} uses to create
the letterhead. The commands \Macro{markboth}\IndexCmd{markboth} and
\Macro{markright}\IndexCmd{markright} can in particular be used together with
pagestyle \PageStyle{myheadings}\IndexPagestyle{myheadings}%
\important{\PageStyle{myheadings}}. If the package
\Package{scrlayer-scrpage}\IndexPackage{scrlayer-scrpage} is used then this,
of course, is valid also for pagestyle
\PValue{scrheadings}\IndexPagestyle{scrheadings}. There the command
\DescRef{scrlayer-scrpage.cmd.markleft}\IndexCmd{markleft} is furthermore
available.

\begin{Declaration}
  \Variable{nexthead}%
  \Variable{nextfoot}%
\end{Declaration}
At times one wants to have more freedom with creating the letterhead or footer
of subsequent pages. Then one has to give up the previously described
possibilities of predefined letterheads or footers that could have been chosen
via the option
\DescRef{\LabelBase.option.pagenumber}\IndexOption{pagenumber}. Instead one is
free to create the letterhead and footer of consecutive pages just the way one
wants to have them set with page style
\PageStyle{headings}\IndexPagestyle{headings}\important{\PageStyle{headings},
  \PageStyle{myheadings}} or
\PageStyle{myheadings}\IndexPagestyle{myheadings}.  For that, one creates the
desired letterhead or footer construction using the \PName{content} of
variable\ChangedAt{v3.08}{\Class{scrlttr2}} \Variable{nexthead} or
\Variable{nextfoot}, respectively. Within the \PName{content} of
\Variable{nexthead} and \Variable{nextfoot} you can, for example, have several
boxes side by side or one beneath the other by use of the \Macro{parbox}
command (see \cite{latex:usrguide}). A more advanced user should have no
problems creating letterheads of footers of his own.  Within \PName{content}
you can and should of course also make use of other variables by using
\DescRef{\LabelBase.cmd.usekomavar}. \KOMAScript{} does not use the
\PName{description} of both variables.

Only\textnote{Attention!} for compatibility reason the deprecated commands
\Macro{nexthead}\IndexCmd[indexmain]{nexthead} and
\Macro{nextfoot}\IndexCmd[indexmain]{nextfoot} of former \Class{scrlttr2}
releases before 3.08 are also implemented. Nevertheless, you should not use
those.%
%
\EndIndexGroup
%
\EndIndexGroup
%
\EndIndexGroup


\LoadCommonFile{interleafpage}% \section{Interleaf Pages}

\LoadCommonFile{footnotes}% \section{Footnotes}

\LoadCommonFile{lists}% \section{Lists}


\section{Math}
\seclabel{math}%
\BeginIndexGroup
\BeginIndex{}{equations}%
\BeginIndex{}{formulas}%
\BeginIndex{}{mathematics}%

There are not math environments implemented at the \KOMAScript{}
classes. Instead of this, the math features of the \LaTeX{} kernel have been
supported. Furthermore\textnote{Attention!} regular math with numbered
equations or formulas is very unusual at letters. Because of this
\Class{scrlttr2} does not actively support numbered equations. Therefore
options \DescRef{maincls.option.leqno} and \DescRef{maincls.option.fleqn},
that has been described for \Class{scrbook}, \Class{scrreprt}, and
\Class{scrartcl} at \autoref{sec:maincls.math} are not available from
\Class{scrlttr2}.

You will not find a description of the math environments of the \LaTeX{} kernel
here. If you want to use \Environment{displaymath}\IndexEnv{displaymath},
\Environment{equation}\IndexEnv{equation} and
\Environment{eqnarray}\IndexEnv{eqnarray} you should read a short introduction
into \LaTeX{} like \cite{lshort}. But\textnote{Hint!} if you want more than
very simple mathematics, usage of package \Package{amsmath} would be
recommended (see \cite{package:amsmath}).%
%
\EndIndexGroup


\section{Floating Environments of Tables and Figures}
\seclabel{floats}

Floating environments for tables or figures are very unusual in
letters. Therefore\textnote{Attention!} \Class{scrlttr2} does not provide
them. If someone nevertheless needs floating environments, then this is often
points out a malpractice of the letter class. In such cases you may try to
define the floating environments with help of packages like \Package{tocbasic}\important{\Package{tocbasic}} (siehe
\autoref{cha:tocbasic}). Nevertheless, tabulars and pictures or graphics
without floating environment may still be done with the letter class
\Class{scrlttr2}.


\LoadCommonFile{marginpar} % \section{Margin Notes}


\section{Closing}
\seclabel{closing}
\BeginIndexGroup%
\BeginIndex{}{closing}%
\BeginIndex{}{letter>closing}%
\BeginIndex{}{signature}%
\BeginIndex{}{letter>signature}%

That the letter closing will be set by
\DescRef{\LabelBase.cmd.closing}\IndexCmd{closing} has been explained already
in \autoref{sec:\LabelBase.document}, \DescPageRef{\LabelBase.cmd.closing}. A kind
of annotation to the signature is often placed below the signature of the
letter. The signing or hand-written inscription itself is placed between this
signature annotation and the closing phrase.

\begin{Declaration}
  \Variable{signature}%
\end{Declaration}
The variable \Variable{signature} holds an explanation or annotation for the
inscription, the signing of the letter. The \PName{content} is predefined as
\DescRef{\LabelBase.cmd.usekomavar}\PParameter{fromname}.  The annotation may
consist of multiple lines. The lines should then be separated by a double
backslash. Paragraph\textnote{Attention!} breaks in the annotation are however
not permitted.%
\EndIndexGroup


\begin{Declaration}
  \Macro{raggedsignature}
\end{Declaration}
Closing phrase and signature will be typeset in a box. The width of
the box is determined by the length of the longest line of the closing
phrase or signature's \PName{content}.

Where to place this box is determined by the pseudo-lengths
\PLength{sigindent}\IndexPLength{sigindent} and
\PLength{sigbeforevskip}\IndexPLength{sigbeforevskip} (see
\autoref{sec:scrlttr2-experts.closing},
\DescPageRef{scrlttr2-experts.plength.sigindent}). The command
\Macro{raggedsignature} defines the alignment inside the box. In the
predefined \File{lco} files the command is either defined as \Macro{centering}
(all besides \Option{KOMAold}) or \Macro{raggedright} (\Option{KOMAold}).  In
order to obtain flush-right or flush-left alignment inside the box, the
command can be redefined in the same way as
\DescRef{maincls.cmd.raggedsection} (see \autoref{sec:maincls.structure},
\DescPageRef{maincls.cmd.raggedsection}).

\begin{Example}
  Now, Mr Public really wants to aggrandize himself. Therefor he uses the
  signature to show again, that he himself was formerly chairman. Because of
  this he changes \PName{contents} of variable
  \DescRef{\LabelBase.variable.signature}. Additionally he wants the signature
  be flush-left aligned and so he also redefines \Macro{raggedsignature}:%
  \lstinputcode[xleftmargin=1em]{letter-22}%
  See \autoref{fig:\LabelBase.letter-22} for the result.
  \begin{figure}
    \setcapindent{0pt}%
    \begin{captionbeside}[{Example: letter with extended sender, logo,
        addressee, additional sender information, place, date, subject,
        opening, text, closing, modified signature, postscript, distribution
        list, enclosure, and puncher hole mark}]
      {result of a small letter with extended sender, logo, addressee,
        additional sender information, place, date, subject opening, text,
        closing, modified signature, postscript, distribution list, enclosure
        and puncher hole mark}[l]
      \frame{\includegraphics[width=.4\textwidth]{letter-22}}
    \end{captionbeside}
    \label{fig:\LabelBase.letter-22}
  \end{figure}
\end{Example}
%
\EndIndexGroup
%
\EndIndexGroup


\section{Letter Class Option Files}
\seclabel{lcoFile}%
\BeginIndexGroup
\BeginIndex{}{lco-file=\File{lco}-file}%
\BeginIndex{}{letter class option}%
\BeginIndex{}{letter>class option}%

Normally, you would not redefine selections like the sender's information
every time you write a letter. Instead, you would reuse a whole set of
parameters for certain occasions. It will be much the same for the letterhead
and footer used on the first page. Therefore, it is reasonable to save these
settings in a separate file. For this purpose, the \Class{scrlttr2} class
offers the \File{lco}-files. The \File{lco} suffix is an abbreviation for
\emph{\emph{l}etter \emph{c}lass \emph{o}ption}.

In an \File{lco} file you can use all commands available to the document at
the time the \File{lco} file is loaded.  Additionally, it can contain internal
commands available to package writers. For \Class{scrlttr2}, these are in
particular the commands
\DescRef{scrlttr2-experts.cmd.@newplength}\IndexCmd{@newplength},
\DescRef{scrlttr2-experts.cmd.@setplength}\IndexCmd{@setplength}, and
\DescRef{scrlttr2-experts.cmd.@setplength}\IndexCmd{@addtoplength} (see
\autoref{sec:scrlttr2-experts.pseudoLengths}).

There are already some \File{lco} files included in the {\KOMAScript}
distribution. The \File{DIN.lco}, \File{DINmtext.lco}, \File{SNleft.lco},
\File{SN.lco}, \File{UScommercial9}, \File{UScommercial9DW},
\File{NF.lco}\ChangedAt{v3.04}{\Class{scrlttr2}} files serve to adjust
{\KOMAScript} to different layout standards. They are well suited as templates
for your own parameter sets, while you become a \KOMAScript{} expert. The
\File{KOMAold.lco} file, on the other hand, serves to improve compatibility
with the old letter class \Class{scrlettr}.  Since it contains internal
commands not open to package writers, you should not use this as a template
for your own \File{lco} files. You can find a list of predefined \File{lco}
files in \autoref{tab:lcoFiles}, \autopageref{tab:lcoFiles}.

If you have defined a parameter set for a letter standard not yet supported by
\KOMAScript, you are explicitly invited to send this parameter set to the
{\KOMAScript} support address. Please do not forget to include the permission
for distribution under the {\KOMAScript} license (see the \File{lppl.txt}
file). If you know the necessary metrics for an unsupported letter standard,
but are not able to write a corresponding \File{lco} file yourself, you can
also contact the {\KOMAScript} author, Markus Kohm, directly. More
particular complex examples of \File{lco}-files are shown at \cite{homepage}
or in \cite{DANTE:TK0203:MJK}. Both locations are mainly in German. 

\begin{Declaration}
  \Macro{LoadLetterOption}\Parameter{name}%
  \Macro{LoadLetterOptions}\Parameter{list of names}
\end{Declaration}
With \Class{scrlttr2} \File{lco}-files can be loaded by the
\Macro{documentclass} command. You enter the name of the \File{lco}-file
without suffix as an option\Index{option}. In this case, the \File{lco}-file
will be loaded right after the class file.

However, it is recommended to load an \File{lco}-file using
\Macro{LoadLetterOption} or
\Macro{LoadLetterOptions}\ChangedAt{v3.14}{\Class{scrlttr2}}. You can do this
even from within another \File{lco}-file. Both commands take the \PName{name}
of the \File{lco}-file without suffix, \Macro{LoadLetterOption} as a single
parameter, \Macro{LoadLetterOptions} as one member of a comma-separated
\PName{list of names}. The corresponding \File{lco}-files will be loaded in
the order given by the list.
\begin{Example}
  Mr Public also writes a document containing several letters. Most of them
  should comply with the German DIN standard. So he starts with:
\begin{lstcode}
  \documentclass{scrlttr2}
\end{lstcode}
  However, one letter should use the \File{DINmtext} variant, with the
  address field placed more toward the top, which results in more text
  fitting on the first page. The folding will be modified so that the
  address field still matches the address window in a DIN~C6/5 envelope.
  You can achieve this as follows:
\begin{lstcode}
  \begin{letter}{%
      Joana Public\\
      Hillside 1\\
      12345 Public-City}
    \LoadLetterOption{DINmtext}
    \opening{Hello,}
\end{lstcode}
  Since construction of the page does not start before the
  \DescRef{\LabelBase.cmd.opening}\IndexCmd{opening} command, it is sufficient
  to load the \File{lco}-file before this. In particular, the loading need not
  be done before \Macro{begin}\PParameter{letter}. Therefore the changes made
  by loading the \File{lco}-file are local to the corresponding letter.
\end{Example}

If\ChangedAt{v2.97}{\Class{scrlttr2}} an \File{lco}-file is loaded via
\Macro{documentclass}, then it may no longer have the same name as an option.

\begin{Example}
  Since Mr~Public often writes letters with the same options and parameters,
  he does not like to copy all these to each new letter. To simplify the effort
  of writing a new letter, he therefore makes a \File{lco}-file:%
  \lstinputcode[xleftmargin=1em]{ich.lco}%
  With this the size of the previous letter decreases to:
  \lstinputcode[xleftmargin=1em]{letter-23.tex}%
  Nevertheless, as shown in \autoref{fig:\LabelBase.letter-23}, the result
  does not change.
  \begin{figure}
    \setcapindent{0pt}%
    \begin{captionbeside}[{Example: letter with extended sender, logo,
        addressee, additional sender information, place, date, subject,
        opening, text, closing, modified signature, postscript, distribution
        list, enclosure, and puncher hole mark using a \File{lco}-file}]
      {result of a small letter with extended sender, logo, addressee,
        additional sender information, place, date, subject opening, text,
        closing, modified signature, postscript, distribution list, enclosure
        and puncher hole mark using a \File{lco}-file}[l]
      \frame{\includegraphics[width=.4\textwidth]{letter-23}}
    \end{captionbeside}
    \label{fig:\LabelBase.letter-23}
  \end{figure}
\end{Example}

Please note\textnote{Attention!} that immediately after loading the
document class normally neither a package for the input encoding nor a
language package has been loaded. Because of this, you should use \TeX's 7-bit
encoding for all characters, e.\,g., use ``\Macro{ss}'' to produce a German
``\ss''.

In \autoref{tab:lcoFiles}, \autopageref{tab:lcoFiles} you can find a list of
all predefined \File{lco} files. If\textnote{Attention!} you use a printer
that has large unprintable areas on the left or right side, you might have
problems with the \Option{SN}\IndexOption{SN} option. Since the Swiss standard
SN~101\,130 defines the address field to be placed 8\Unit{mm} from the right
paper edge, the headline and the sender attributes too will be set with the
same small distance from the paper edge. This also applies to the reference
line when using the \DescRef{\LabelBase.option.refline}\PValue{=wide} option
(see \autoref{sec:\LabelBase.firstpage},
\DescPageRef{\LabelBase.option.refline}). If\textnote{Hint!} you have this
kind of problem, create your own \File{lco} file that loads \Option{SN} first
and then changes \PLength{toaddrhpos}\IndexPLength{toaddrhpos} (see
\autoref{sec:scrlttr2-experts.addressee},
\DescPageRef{scrlttr2-experts.plength.toaddrvpos}) to a smaller
value. Additionally, also reduce \PLength{toaddrwidth} accordingly.%

By\textnote{Hint!} the way, the \File{DIN} \File{lco}-file will always be
loaded as the first \File{lco} file. This ensures that all pseudo-lengths will
have more or less reasonable default values. Because of this, it is not
necessary to load this default file on your own.

Please\textnote{Attention!} note that it is not possible to use
\Macro{PassOptionsToPackage} to pass options to packages from within an
\File{lco}-file that have already been loaded by the class. Normally, this
only applies to the \Package{typearea}, \Package{scrlfile}, \Package{scrbase},
and \Package{keyval} packages.%

\begin{desclist}
  \renewcommand*{\abovecaptionskipcorrection}{-\normalbaselineskip}%
  \desccaption{%
    predefined \File{lco}-files\label{tab:lcoFiles}%
  }{%
    predefined \File{lco}-files (\emph{continuation})%
  }%
  \oentry{DIN}{%
    parameter set for letters on A4-size paper, complying with German standard
    DIN~676; suitable for window envelopes in the sizes C4, C5, C6, and C6/5
    (C6 long).}%
  \oentry{DINmtext}{%
    parameter set for letters on A4-size paper, complying with DIN~676, but
    using an alternate layout with more text on the first page; only suitable
    for window envelopes in the sizes C6 and C6/5 (C6 long).}%
  \oentry{KakuLL}{%
    parameter set for Japanese letters in A4 format; suitable for Japanese
    window envelopes of type Kaku A4, in which the windows is approximately
    90\Unit{mm} wide, 45\Unit{mm} high, and positioned 25\Unit{mm} from the
    left and 24\Unit{mm} from the top edge (see \autoref{cha:japanlco}).}%
  \oentry{KOMAold}{%
    parameter set for letters on A4-size paper using a layout close to the now
    obsolete \Class{scrlettr} letter class; suitable for window envelopes in
    the sizes C4, C5, C6, and C6/5 (C6 long); some additional commands to
    improve compatibility with obsolete \Class{scrlettr} commands are defined;
    \Class{scrlttr2} may behave slightly different when used with this
    \File{lco}-file than with the other \File{lco}-files.}%
  \oentry{NF}{%
    parameter set for French letters, according to NF~Z~11-001; suitable for
    window envelopes of type DL (110\Unit{mm} to 220\Unit{mm}) with a window
    of about 20\Unit{mm} from right and bottom with width 45\Unit{mm} and
    height 100\Unit{mm}; this file was originally developed by Jean-Marie
    Pacquet, who provides an extended version and additional information on
    \cite{www:NF.lco}.}%
  \oentry{NipponEH}{%
    parameter set for Japanese letters in A4 format; suitable for Japanese
    window envelopes of types Chou or You 3 or 4, in which the windows is
    approximately 90\Unit{mm} wide, 55\Unit{mm} high, and positioned
    22\Unit{mm} from the left and 12\Unit{mm} from the top edge (see
    \autoref{cha:japanlco}).}%
  \oentry{NipponEL}{%
    parameter set for Japanese letters in A4 format; suitable for Japanese
    window envelopes of types Chou or You 3 or 4, in which the windows is
    approximately 90\Unit{mm} wide, 45\Unit{mm} high, and positioned
    22\Unit{mm} from the left and 12\Unit{mm} from the top edge (see
    \autoref{cha:japanlco}).}%
  \oentry{NipponLH}{%
    parameter set for Japanese letters in A4 format; suitable for Japanese
    window envelopes of types Chou or You 3 or 4, in which the windows is
    approximately 90\Unit{mm} wide, 55\Unit{mm} high, and positioned
    25\Unit{mm} from the left and 12\Unit{mm} from the top edge (see
    \autoref{cha:japanlco}).}%
  \oentry{NipponLL}{%
    parameter set for Japanese letters in A4 format; suitable for Japanese
    window envelopes of types Chou or You 3 or 4, in which the windows is
    approximately 90\Unit{mm} wide, 45\Unit{mm} high, and positioned
    25\Unit{mm} from the left and 12\Unit{mm} from the top edge (see
    \autoref{cha:japanlco}).}%
  \oentry{NipponRL}{%
    parameter set for Japanese letters in A4 format; suitable for Japanese
    window envelopes of types Chou or You 3 or 4, in which the windows is
    approximately 90\Unit{mm} wide, 45\Unit{mm} high, and positioned
    25\Unit{mm} from the left and 24\Unit{mm} from the top edge (see
    \autoref{cha:japanlco}).}%
  \oentry{SN}{%
    parameter set for Swiss letters with address field on the right side,
    according to SN~010\,130; suitable for Swiss window envelopes in the sizes
    C4, C5, C6, and C6/5 (C6 long).}%
  \oentry{SNleft}{%
    parameter set for Swiss letters with address field on the left side;
    suitable for Swiss window envelopes with window on the left side in the
    sizes C4, C5, C6, and C6/5 (C6 long).}%
  \oentry{UScommercial9}{%
    parameter set for US-American letters with paper size letter; suitable for
    US-American window envelopes of size \emph{commercial~No.\,9} with window
    width of 4\,1/2\Unit{in}, height of 1\,1/8\Unit{in}, and position of
    7/8\Unit{in} from the left and 1/2\Unit{in} from the bottom, without
    sender's return address inside of the window; with folding it first at the
    puncher mark then at the top folder mark also legal paper may be used but
    would result in a page size warning}%
  \oentry{UScommercial9DW}{%
    parameter set for US-American letters with paper size letter; suitable for
    US-American window envelopes of size \emph{commercial~No.\,9} with
    addressee's window width of 3\,5/8\Unit{in}, height of 1\,1/8\Unit{in},
    and position of 3/4\Unit{in} from the left and 1/2\Unit{in} from the
    bottom, and with a sender's window width of 3\,1/2\Unit{in}, height of
    7/8\Unit{in}, and position of 5/16\Unit{in} from the left and
    2\,1/2\Unit{in} from the bottom, but without a sender's return address at
    any of the windows; with folding it first at the puncher mark then at the
    top folder mark also legal paper may be used but would result in a page
    size warning}%
\end{desclist}
%
\EndIndexGroup
%
\EndIndexGroup


\section{Address Files and Circular Letters}
\seclabel{addressFile}%
\BeginIndexGroup
\BeginIndex{}{address>file}%
\BeginIndex{}{circular letters}%
\index{serial letters|see circular letters}%

When people write circular letters one of the more odious tasks is the typing
of many different addresses.  The class \Class{scrlttr2}%
\iffalse% Umbruchkorrekturtext
, as did its predecessor \Class{scrlettr} as well,%
\fi%
\ provides basic support for this task.%
\iffalse% Umbruchkorrekturtext
\ Currently there are plans for much enhanced support.%
\fi

\begin{Declaration}
  \Macro{adrentry}\Parameter{last-name}\Parameter{first-name}%
  \Parameter{address}\Parameter{phone}\Parameter{F1}\Parameter{F2}%
  \Parameter{comment}\Parameter{key}
\end{Declaration}%
The class \Class{scrlttr2} supports the use of address files which
contain address entries, very useful for circular letters. The file
extension of the address file has to be \File{.adr}. Each entry is an
\Macro{adrentry} command with eight parameters, for example:
\begin{lstcode}
  \adrentry{McEnvy}
           {Flann}
           {Main Street 1\\ Glasgow}
           {123 4567}
           {male}
           {}
           {niggard}
           {FLANN}
\end{lstcode}
The 5\textsuperscript{th} and 6\textsuperscript{th} elements, \PValue{F1} and
\PValue{F2}, can be used freely: for example, for the gender, the academic
grade, the birthday, or the date on which the person joined a society.  The
last parameter \PName{key} should only consist of more than one uppercase
letters in order to not interfere with existing {\TeX} or {\LaTeX} commands.

\begin{Example}
  Mr McEnvy is one of your most important business partners, but
  every day you receive correspondence from him.  Before long you do
  not want to bother typing his boring address again and again.  Here
  \Class{scrlttr2} can help.  Assume that all your business partners
  have an entry in your \File{partners.adr} address file.  If you now
  have to reply to Mr~McEnvy again, then you can save typing as
  follows:
  \begin{lstcode}
    \input{partners.adr}
    \begin{letter}{\FLANN}
      Your correspondence of today \dots
    \end{letter}
  \end{lstcode}
  Your {\TeX} system must be configured to have access to your address
  file. Without access, the \Macro{input} command results in an
  error. You can either put your address file in the same directory
  where you are running {\LaTeX}, or configure your system to find the
  file in a special directory.
\end{Example}
% 
\EndIndexGroup

\begin{Declaration}
  \Macro{addrentry}\Parameter{last-name}\Parameter{first-name}%
  \Parameter{address}\Parameter{phone}\Parameter{F1}\Parameter{F2}%
  \Parameter{F3}\Parameter{F4}\Parameter{key}
\end{Declaration}%
Over the years people have objected that the \DescRef{\LabelBase.cmd.adrentry} has only two
free parameters. To cater to this demand, there now exists a new command
called \Macro{addrentry}\,---\,note the additional ``d''\,---\,which supports
four freely-definable parameters. Since {\TeX} supports maximally nine
parameters per command, the comment parameter has fallen away. Other than this
difference, the use is the same as that of \DescRef{\LabelBase.cmd.adrentry}.

Both \DescRef{\LabelBase.cmd.adrentry} and \Macro{addrentry} commands can be freely mixed in
the \File{adr} files. However, it should be noted that there are some packages
which are not suited to the use of \Macro{addrentry}.  For example, the
\Package{adrconv} by Axel Kielhorn can be used to create address lists from
\File{adr} files, but it has currently no support for command
\Macro{addrentry}.  In this case, the only choice is to extend the package
yourself.%
%
\EndIndexGroup

Besides the simple access to addresses, the address files can be
easily used in order to write circular letters.  Thus, there is no
requirement to access a complicated database system via {\TeX}.
%
\begin{Example}
  Suppose you are member of a society and want write an invitation for
  the next general meeting to all members.
\begin{lstcode}
  \documentclass{scrlttr2}
  \begin{document}
  \renewcommand*{\adrentry}[8]{
    \begin{letter}{#2 #1\\#3}
      \opening{Dear members,} Our next general meeting will be on
      Monday, August 12, 2002. The following topics are \dots
      \closing{Regards,}
    \end{letter}
  }
  \chapter{Members of the consortium}
\label{cha:members}

\instructions{
\textit{This section is not covered by the page limit.}
\vskip0.2cm
\textit{The information provided here will be used to judge the operational capacity.}
}

\section{Participants (applicants)}
\label{sec:participants}

\instructions{
Please provide, for each participant, the following (if available):\\
\begin{itemize}
\item a description of the legal entity and its main tasks, with an explanation of how its profile matches the tasks in the proposal;
\item a curriculum vitae or description of the profile of the persons, including their gender, who will be primarily responsible for carrying out the proposed research and/or innovation activities;
\item a list of up to 5 relevant publications, and/or products, services (including widely-used datasets or software), or other achievements relevant to the call content;
\item a list of up to 5 relevant previous projects or activities, connected to the subject of this proposal;
\item a description of any significant infrastructure and/or any major items of technical equipment, relevant to the proposed work;
\item any other supporting documents specified in the work programme for this call.
\end{itemize}
}

\section{Third parties involved in the project (third party resources)}
\label{sec:third-parties}

\instructions{
\textit{Please complete, for each participant, the following table (or simply state "No third parties involved", if applicable).} \\
If yes in first row, please describe and justify the tasks to be subcontracted. If yes in second row, please describe the third party, the link of the participant to the third party, and describe and justify the foreseen tasks to be performed by the third party\footnote{A third party that is an affiliated entity or has a legal link to a participant implying a collaboration not limited to the action. (Article 14 of the Model Grant Agreement).}. If yes in third row, please describe the third party and their contributions.}

\begin{tabular}{|p{.85\textwidth}|p{.05\textwidth}|}
  \hline  
  \multicolumn{2}{|l|}{\cellcolor[gray]{0.8}\textbf{UoC}}\\
  \hline
  Does the participant plan to subcontract certain tasks (please note that core tasks of the project should not be sub-contracted) &
  \textbf{Y/N} \\
  \hline
Does the participant envisage that part of its work is performed by linked
third parties &
  \textbf{Y/N} \\
  \hline
  Does the participant envisage the use of contributions in kind provided by
third parties (Articles 11 and 12 of the General Model Grant Agreement) &
  \textbf{Y/N}\\
  \hline
\end{tabular}

\begin{tabular}{|p{.85\textwidth}|p{.05\textwidth}|}
  \hline  
  \multicolumn{2}{|l|}{\cellcolor[gray]{0.8}\textbf{UoP1}}\\
  \hline
  Does the participant plan to subcontract certain tasks (please note that core tasks of the project should not be sub-contracted) &
  \textbf{Y/N} \\
  \hline
Does the participant envisage that part of its work is performed by linked
third parties &
  \textbf{Y/N} \\
  \hline
  Does the participant envisage the use of contributions in kind provided by
third parties (Articles 11 and 12 of the General Model Grant Agreement) &
  \textbf{Y/N}\\
  \hline
\end{tabular}


\begin{tabular}{|p{.85\textwidth}|p{.05\textwidth}|}
  \hline  
  \multicolumn{2}{|l|}{\cellcolor[gray]{0.8}\textbf{UoP2}}\\
  \hline
  Does the participant plan to subcontract certain tasks (please note that core tasks of the project should not be sub-contracted) &
  \textbf{Y/N} \\
  \hline
Does the participant envisage that part of its work is performed by linked
third parties &
  \textbf{Y/N} \\
  \hline
  Does the participant envisage the use of contributions in kind provided by
third parties (Articles 11 and 12 of the General Model Grant Agreement) &
  \textbf{Y/N}\\
  \hline
\end{tabular}

  \end{document}
\end{lstcode}
  If the address file contains \DescRef{\LabelBase.cmd.addrentry} commands too, than an
  additional definition for \DescRef{\LabelBase.cmd.addrentry} is required before loading
  the address file:
\begin{lstcode}
  \renewcommand*{\addrentry}[9]{%
    \adrentry{#1}{#2}{#3}{#4}{#5}{#6}{#7}{#9}%
  }
\end{lstcode}
  In this simple example the extra freely-definable parameter is not
  used, and therefore \DescRef{\LabelBase.cmd.addrentry} is defined with the help of
  \DescRef{\LabelBase.cmd.adrentry}.
\end{Example}

With some additional programming one can let the content of the
letters depend on the address data. For this the free parameters of
the \DescRef{\LabelBase.cmd.adrentry} and and \DescRef{\LabelBase.cmd.addrentry} commands can be used.

\begin{Example}
  Suppose the 5\textsuperscript{th} parameter of the \DescRef{\LabelBase.cmd.adrentry}
  command contains the gender of a member (\PValue{m/f}), and the
  6\textsuperscript{th} parameter contains what amount of subscription
  has not yet been paid by the member. If you would like to write a
  more personal reminder to each such member, then the next example
  can help you:
\begin{lstcode}
  \renewcommand*{\adrentry}[8]{
    \ifdim #6pt>0pt\relax
    % #6 is an amount greater than 0.
    % Thus, this selects all members with due subscription.
      \begin{letter}{#2 #1\\#3}
        \if #5m \opening{Dear Mr.\,#2,} \fi
        \if #5f \opening{Dear Mrs.\,#2,} \fi

        Unfortunately we have to remind you that you have
        still not paid the member subscription for this
        year.

        Please remit EUR #6 to the account of the society.
       \closing{Regards,}
      \end{letter}
     \fi
  }
\end{lstcode}
\end{Example}
As you can see, the letter text can be made more personal by depending on
attributes of the letter's addressee.  The number of attributes is only
restricted by number of two free parameters of the \DescRef{\LabelBase.cmd.adrentry} command,
or four free parameters of the \DescRef{\LabelBase.cmd.addrentry} command.


\begin{Declaration}
  \Macro{adrchar}\Parameter{initial letter}%
  \Macro{addrchar}\Parameter{initial letter}
\end{Declaration}
\Index[indexmain]{address>list}\Index[indexmain]{telephone list}%
As already mentioned above, it is possible to create address and telephone
lists using \File{adr} files.  For that, the additional package
\Package{adrconv} by Axel Kielhorn (see \cite{package:adrconv}) is needed.
This package contains interactive {\LaTeX} documents which help to create
those lists.

The address files have to be sorted already in order to obtain sorted
lists. It is recommended to separate the sorted entries at each different
initial letter of \PName{last name}.  As a separator, the commands
\Macro{adrchar} and \Macro{addrchar} can be used. These commands will be
ignored if the address files are utilized in \Class{scrlettr2}.
%
\begin{Example}
  Suppose you have the following short address file:
\begin{lstlisting}
  \adrchar{A}
  \adrentry{Angel}{Gabriel}
           {Cloud 3\\12345 Heaven's Realm}
           {000\,01\,02\,03}{}{}{archangel}{GABRIEL}
  \adrentry{Angel}{Michael}
           {Cloud 3a\\12345 Heaven's Realm}
           {000\,01\,02\,04}{}{}{archangel}{MICHAEL}
  \adrchar{K}
  \adrentry{Kohm}{Markus}
           {Freiherr-von-Drais-Stra\ss e 66\\68535 Edingen-Neckarhausen}
           {+49~62\,03~1\,??\,??}{}{}{no angel at all}
           {KOMA}
\end{lstlisting}
  This address file can be treated with \File{adrdir.tex} of the
  \Package{adrconv} package~\cite{package:adrconv}.
  The result should look like this:
  \begin{center}
    \setlength{\unitlength}{1mm}
    \begin{picture}(80,57)
      \put(0,57){\line(1,0){80}}
      \put(0,3){\line(0,1){54}}
      \put(80,3){\line(0,1){54}}
      \thicklines
      \put(10,42){\line(1,0){60}}
      \put(70,45){\makebox(0,0)[r]{\textsf{\textbf{A}}}}
      \put(10,23){\makebox(0,20)[l]{\parbox{5cm}{\raggedright
            \textsc{Angel}, Gabriel\\\quad\small Cloud 3\\
            \quad 12345 Heaven's Realm\\
            \quad (archangel)}}}
      \put(70,23){\makebox(0,20)[r]{\parbox{2cm}{\raggedright~\\
            \small~\\\textsc{gabriel}\\000\,01\,02\,03}}}
      \put(10,4){\makebox(0,20)[l]{\parbox{5cm}{\raggedright
            \textsc{Angel}, Michael\\\quad\small Cloud 3a\\
            \quad 12345 Heaven's Realm\\
            \quad (archangel)}}}
      \put(70,4){\makebox(0,20)[r]{\parbox{2cm}{\raggedright~\\
            \small~\\\textsc{michael}\\000\,01\,02\,04}}}
      \qbezier(0,3)(10,6)(40,3)\qbezier(40,3)(60,0)(80,3)
    \end{picture}
  \end{center}
  The letter in the page header is created by the \Macro{adrchar}
  command. The definition can be found in \File{adrdir.tex}.
\end{Example}
More about the \Package{adrconv} package can be found in its
documentation. There you should also find information about whether the
current version of \Package{adrconv} supports the
\DescRef{\LabelBase.cmd.addrentry} and \Macro{addrchar} commands.  Former
versions only know the commands \DescRef{\LabelBase.cmd.adrentry} and
\Macro{adrchar}.%
%
\EndIndexGroup
%
\EndIndexGroup
%
\EndIndexGroup


%%% Local Variables: 
%%% mode: latex
%%% mode: flyspell
%%% coding: us-ascii
%%% ispell-local-dictionary: "en_GB"
%%% TeX-master: "../guide"
%%% End: 

% LocalWords:  scrlttr simpleSwitchValues afteropening beforeopening Combinable
% LocalWords:  locfield Hennig Kohm Foldmarks
