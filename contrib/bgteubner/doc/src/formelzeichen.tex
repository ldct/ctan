%
% bgteubner class bundle
%
% formelzeichen.tex
% Copyright 2003--2012 Harald Harders
%
% This program may be distributed and/or modified under the
% conditions of the LaTeX Project Public License, either version 1.3
% of this license or (at your opinion) any later version.
% The latest version of this license is in
%    http://www.latex-project.org/lppl.txt
% and version 1.3 or later is part of all distributions of LaTeX
% version 1999/12/01 or later.
%
% This program consists of all files listed in manifest.txt.
% ===================================================================
\addchap{Verwendete Formelzeichen}

% -------------------------------------------------------------------
\begin{theglossary}[\addsec{Skalare}]%
%
\alpha_k& Kerbformzahl\\
%
\varepsilon& technische Dehnung\\
%
\varphi& wahre Dehnung\\
%
\sigma& Spannung\\
%
a& Gitterkonstante\\
a& Rissl�nge bei Oberfl�chenrissen und halbe Rissl�nge bei inneren Rissen\\
%
c& Gitterkonstante\\
%
K& Spannungsintensit�tsfaktor\\
%
R_{eH}& obere Streckgrenze f�r Materialien mit ausgepr�gter Streckgrenze\\
R_{eL}& untere Streckgrenze f�r Materialien mit ausgepr�gter Streckgrenze\\
R_m& Zugfestigkeit\\
R_{p0.2}& Dehngrenze f�r Materialien ohne ausgepr�gte Streckgrenze\\
%
\end{theglossary}%
                                
% -------------------------------------------------------------------
\begin{theglossary}[\addsec{Vektoren}]%
%
(\varepsilon_{\alpha})=\vec{\varepsilon}& Dehnungsvektor (Voigtsche
Schreibweise)\\
%
(\sigma_{\alpha})=\vec{\sigma}& Spannungsvektor (Voigtsche
Schreibweise)\\
%
\end{theglossary}%
                                
% -------------------------------------------------------------------
\begin{theglossary}[\addsec{Matrizen und Tensoren}]%
%
\delta_{ij}& Kroneckersymbol\\
%
(\varepsilon_{ij})=\matr{\varepsilon}& Dehnungstensor\\
%
(\sigma_{ij})=\matr{\sigma}& Spannungstensor\\
%
\end{theglossary}%
                                
% -------------------------------------------------------------------
\begin{theglossary}[\addsec{Indizes}]%
%
\alpha, \beta& In der Tensorrechnung: Laufindizes in der Voigtschen
Schreibweise, m�gliche Werte: $1$ bis $6$\\
%
1, 2, 3& Unsortierte Hauptwerte f�r Spannungstensoren\\
%
\mathrm{I}, \mathrm{II}, \mathrm{III}& Sortierte Hauptwerte f�r
Spannungstensoren\\
i, j, k, l& In der Tensorrechnung: Laufindizes f�r die
Tensorkomponenten, m�gliche Werte: $1$ bis $3$\\
%
\end{theglossary}%
                                
% -------------------------------------------------------------------
% EOF