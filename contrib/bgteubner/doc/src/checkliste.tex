%
% bgteubner class bundle
%
% checkliste.tex
% Copyright 2003--2012 Harald Harders
%
% This program may be distributed and/or modified under the
% conditions of the LaTeX Project Public License, either version 1.3
% of this license or (at your opinion) any later version.
% The latest version of this license is in
%    http://www.latex-project.org/lppl.txt
% and version 1.3 or later is part of all distributions of LaTeX
% version 1999/12/01 or later.
%
% This program consists of all files listed in manifest.txt.
% ===================================================================
\chapter{Checkliste}%
\index{Checkliste|textbf}%
\label{sec:checkliste}%

An dieser Stelle wird eine Checkliste angeboten, die man durcharbeiten
sollte, bevor man dem Verlag eine endg�ltige Version seines
Buches zukommen l�sst.
%
\begin{enumerate}
\item Sind so wenig zus�tzliche Pakete geladen wie m�glich?
  (Es m�ssen weniger Pakete geladen werden als bei den meisten anderen
  Dokumentklassen.
  Beispielsweise m�ssen weder \texttt{babel} noch \texttt{fontenc}
  zus�tzlich geladen werden.)
\item L�uft der �bersetzungslauf fehlerfrei durch?\footnote{Gerade
    unter Windows findet man manchmal Installationen, die auch an
    Fehlern die �bersetzung nicht unterbrechen.
    Schauen Sie auf jeden Fall in die Log"=Datei, ob keine Fehler
    aufgetreten sind.}
\item Wurde im gesamten Werk, inklusive aller Vorworte, einheitlich
  entweder die neue oder die alte Rechtschreibung verwendet und auch
  in der \verb|\documentclass|"=Zeile die entsprechende Sprache als
  letztes angegeben?
  Die neue Rechtschreibung ist vorzuziehen.
\item Entsprechen die Abs�tze vor und nach abgesetzten Formeln der
  inhaltlichen Gliederung? Keine unn�tigen Abs�tze!
\item Sind durch Flie�umgebungen keine unn�tigen Abs�tze oder
  Leerzeichen entstanden?
\item Sind alle Angaben zum Buch (Autoren, Titel etc.) korrekt?
\item Entsprechen alle Datumsangaben (z.\,B.\ im Vorwort) dem
  Erscheinungsjahr und "~monat?
\item Wurde in allen F�llen zwischen Zahlen im Text"= und Mathemodus
  unterschieden?
\item Wurden alle Formelzeichen auch wirklich im mathematischen Modus
  gesetzt?
\item Haben alle Zahlen Kommata statt Punkte?
\item Tabellen haben �berschriften, keine Unterschriften!
\item Ist das Stichwortverzeichnis erstellt und nicht zu knapp
  gehalten?
\item L�uft die Erzeugung des Literaturverzeichnisses mit \BibTeX\
  fehlerfrei und ohne Warnung ab (siehe Log"=Datei mit der Endung
  \verb|.blg|)?
\item Sind alle fehlenden Referenzen und doppelten Labels behoben?
\item Wurde bei den letzten �bersetzungsl�ufen ohne die Option
  \verb|draft| gearbeitet?
\item\label{enum:uebersetzen}
  Wurde oft genug �bersetzt, so dass alle Referenzen und
  Seitenzahlen in den entsprechenden Verzeichnissen stimmen?
\item Wurde der Index sp�t genug neu erzeugt, so dass auch dort alle
  Seitenzahlen stimmen?
\item\label{enum:overfull}
  Sind alle "`Overfull \textbackslash hbox"' (schwarze Balken,
  falls mit der Option \verb|draft| �bersetzt) behoben?
\item Wenn Sie lange Tabellen mit der \env{longtable}"=Umgebung
  verwenden: Sind alle doppelten Linien an Seitenumbr�chen beseitigt
  worden (vgl.\ Abschnitt~\ref{sec:tex:tabellen}), und ist die
  Reihenfolge der Tabellen korrekt?
\item\label{enum:feinschliff}
  Wurde der Feinschliff entsprechend
  Abschnitt~\ref{sec:tex_feinschliff} zur Vermeidung unsch�ner
  Umbr�che durchgef�hrt (u.\,A.\ Beheben von "`Underfull
  \textbackslash vbox"')? 
\item\label{enum:indexwiderspruch}
  Wurde darauf geachtet, dass sich keine widerspr�chlichen
  Indexangaben entsprechend Abschnitt~\ref{sec:tex_index_doppelt}
  ergeben haben?
\item Wurde die endg�ltige \acro{PDF}"=Datei ausgedruckt und
  sorgf�ltig auf Fehler hin kontrolliert?
  Wenn noch Fehler gefunden werden, m�ssen wahrscheinlich die
  Punkte~\ref{enum:uebersetzen} bis \ref{enum:indexwiderspruch}
  wiederholt werden.
\end{enumerate}

Einige dieser Fragen lassen sich am einfachsten dadurch beantworten,
dass man sich die Log"=Datei von \pdfLaTeX\ ansieht.
Beispielsweise werden fehlende und doppelte Verweise, sowie "`Overfull
\textbackslash hbox"' und "`Underfull \textbackslash vbox"' dort genannt.

% ===================================================================
%%% Local Variables: 
%%% mode: latex
%%% TeX-master: "bgteubner"
%%% End: 
