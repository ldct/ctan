%
% bgteubner class bundle
%
% optionen-advanced.tex
% Copyright 2003--2012 Harald Harders
%
% This program may be distributed and/or modified under the
% conditions of the LaTeX Project Public License, either version 1.3
% of this license or (at your opinion) any later version.
% The latest version of this license is in
%    http://www.latex-project.org/lppl.txt
% and version 1.3 or later is part of all distributions of LaTeX
% version 1999/12/01 or later.
%
% This program consists of all files listed in manifest.txt.
% ===================================================================
\chapter{Anmerkungen f�r versierte Nutzer}%
\label{chap:optionen-advanced}%

Da der normale Anwender die in diesem Anhang beschriebenen Dinge
normalerweise nicht ben�tigt, sind sie kleiner gedruckt.

\section{Erweiterte Klassenoptionen}
\begin{advanced}
  \setkomafont{caption}{\rmfamily\footnotesize\RaggedRight}%
  \setkomafont{float}{\normalfont\normalcolor\footnotesize}%
  \renewcommand{\subcapsize}{\footnotesize}%
  %
  Gegen�ber den in Abschnitt~\ref{sec:tex:aufbau} beschriebenen
  Klassenoptionen unterst�tzt \verb|bgteubner| einige weitere
  Optionen, die normalerweise nicht verwendet werden sollen. 
  In den meisten F�llen f�hren sie zu einem Layout, das inkonsistent
  mit den Vorgaben des Verlags ist.
  In Tabelle~\ref{tab:klassenoptionen-advanced} sind diese Optionen
  ohne ausf�hrliche Erkl�rungen zusammengefasst.
  %
  \begin{table}%
    \centering
    \def\default{{\rmfamily*}}%
    \caption{Selten ben�tigte Klassenoptionen der Klasse \texttt{bgteubner}.
      Defaultm��ig aktivierte Optionen sind mit einem \default\
      gekennzeichnet.}%
    \label{tab:klassenoptionen-advanced}%
    \begin{tabular}{>{\ttfamily}ll}
      \toprule
      \rmfamily Option& Erkl�rung \\
      \midrule
      headingoutside\default & Lebender Kolumnentitel au�en \\
      headinginside& Lebender Kolumnentitel innen \\
      \midrule
      tocindent\default & Inhaltsverzeichnis mit Einr�ckung setzen \\
      tocleft& Inhaltsverzeichnis linksb�ndig \\
      \midrule
      springervieweg\default & Verlagsname Springer Vieweg auf der
      Titelseite \\
      viewegteubner & Verlagsname Vieweg+Teubner auf der Titelseite \\
      bgteubner & Verlagsname B.\,G.\ Teubner auf der Titelseite \\
%      \midrule
%      epsfigures& Setzt Bilder auch bei normalem \LaTeX\ (dvi"=Ausgabe) \\
      \bottomrule
    \end{tabular}
  \end{table}%
\end{advanced}

                                
\section{\acro{PDF}"=Informationen f�r den Verlag}%
\label{sec:pdf-informationen}%
\begin{advanced}
  \setkomafont{caption}{\rmfamily\footnotesize\RaggedRight}%
  \setkomafont{float}{\normalfont\normalcolor\small}%
  \renewcommand{\subcapsize}{\footnotesize}%
  %
  Bei der Erstellung des Dokuments sollen die Autoren einige Felder
  bez�glich des Titels und der Autoren ausf�llen (Befehle \cs{title},
  \cs{subtitle}, \cs{author}, \cs{edition}).
  Diese werden in die Info"=Felder der \acro{PDF}"=Datei
  geschrieben, die im \emph{Acrobat Reader} abgerufen werden k�nnen
  (\emph{File -- Document Properties -- Summary}).
  Im Feld \emph{Title} erscheint der Titel.
  Das \emph{Subject}"=Feld wird f�r den Untertitel und die Auf"|lage
  zweckentfremdet.
  Die Autoren erscheinen im Feld \emph{Author}.
  Das Feld \emph{Keywords} wird zweckentfremdet f�r die Angabe der
  Anzahl der Abbildungen, Tabellen, Aufgaben, Beispiele usw.
  Hier kann der Lektor des Teubner Verlags immer die aktuelle Anzahl
  der entsprechenden Umgebungen ablesen, da diese f�r das Titelblatt
  notwendig ist.
  Es kann so nicht passieren, dass eine inkorrekte Zahl angegeben
  wird.

  Unter \emph{Creator} wird die Version der Dokumentklasse angegeben. 
  Dies ist in seltenen F�llen von Nutzen, um zu kontrollieren, ob eine
  eventuelle alte Version verwendet wurde.
  
  Das Erstellungsdatum im Feld \emph{Created} kann helfen, die neueste
  Version des Manuskriptes zu finden.
\end{advanced}


\section{Befehle f�r Fortgeschrittene}%
\label{sec:fortgeschrittene}%

\begin{advanced}
  Manchmal kann es n�tzlich sein zu pr�fen, ob eine bestimmte Version
  der Dokumentklasse verwendet wird.
  Dazu k�nnen Sie in der Dokumentpr�ambel den Befehl
\begin{verbatim}[\footnotesize\makeescape\|\makebgroup\[\makeegroup\]]
\version|marg[Versionsnummer]
\end{verbatim}
  verwenden.

  M�chten Sie zum Beispiel pr�fen, ob Version 1.03 der Dokumentklasse
  verwendet wird, schreiben Sie
\begin{verbatim}[\footnotesize]
\version{1.03}
\end{verbatim}
  in die Pr�ambel.
  Stimmen die angegebene und die tats�chliche Versionsnummer nicht
  �berein, wird eine Warnung in die Log"=Datei geschrieben.
\end{advanced}


% ===================================================================

%%% Local Variables: 
%%% mode: latex
%%% TeX-master: "bgteubner"
%%% End: 
