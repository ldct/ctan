% -*- coding: utf-8; time-stamp-format: "%02d-%02m-%:y at %02H:%02M:%02S %Z" -*-
% N.B.: this dtx file is NOT for use with \DocInput. The latex source of the
% user manual isn't prefixed with percent signs.
%<*none>
\def\dtxtimestamp {Time-stamp: <19-09-2016 at 11:46:39 CEST>}%
%</none>
%<*!readme>
%%
%% Package: filecontentsdef
%% Version: 1.2 (2016/09/19)
%% License: LPPL 1.3c
%% Copyright (C) 2016 Jean-Francois Burnol <jfbu at free dot fr>
%%
%</!readme>
%<*insfile>
\def\pkgname        {filecontentsdef}
\def\pkgdate        {2016/09/19}
\def\pkgdocdate     {2016/09/19}
\def\pkgversion     {v1.2}
\def\pkgdescription {filecontents + macro + verbatim (JFB)}
%</insfile>
%<*none>
\iffalse
%</none>
%<*readme>
<!-- -->

    Source:  filecontentsdef.dtx (v1.2 2016/09/19)
    Author:  Jean-Francois Burnol
    Info:    filecontents + macro + verbatim
    License: LPPL 1.3c
    Copyright (C) 2016 Jean-Francois Burnol.
    <jfbu at free dot fr>

ABSTRACT
========

This lightweight LaTeX2e package provides two environments called
`filecontentsdef` and `filecontentshere`. They are derived from the
LaTeX `filecontents` environment as extended by Scott Pakin's
[filecontents] [1] package. In addition to the file creation they
either store the (verbatim) contents in a macro (`filecontentsdef`)
or typeset them (verbatim) on the spot (`filecontentshere`).

I developed this to display TeX code verbatim in documentation and
simultaneously produce during the LaTeX run the corresponding files
in order to embed them in the PDF as _file attachment annotations_
(via the services of Scott Pakin's further package [attachfile] [2].)

[1]: http://www.ctan.org/pkg/filecontents "filecontents package"
[2]: http://www.ctan.org/pkg/attachfile   "attachfile package"

INSTALLATION
============

To extract the package (.sty) run etex on the dtx file. To produce
the PDF documentation, either:

1. latex (twice) filecontentsdef.dtx, then dvips, then ps2pdf
2. or pdflatex (twice), 
3. or latex (twice) then dvipdfmx

This will also extract automatically the style file.

Installation:

    filecontentsdef.sty -> TDS:tex/latex/filecontentsdef/filecontentsdef.sty
    filecontentsdef.dtx -> TDS:source/latex/filecontentsdef/filecontentsdef.dtx
    filecontentsdef.pdf -> TDS:doc/latex/filecontentsdef/filecontentsdef.pdf

    README.md   -> TDS:doc/latex/filecontentsdef/README.md

The other files may be discarded.

LICENSE
=======

This Work may be distributed and/or modified under the
conditions of the LaTeX Project Public License 1.3c.
This version of this license is in

> <http://www.latex-project.org/lppl/lppl-1-3c.txt>

and the latest version of this license is in

> <http://www.latex-project.org/lppl.txt>

and version 1.3 or later is part of all distributions of
LaTeX version 2005/12/01 or later.

The Author of this Work is:

- Jean-Francois Burnol `<jfbu at free dot fr>`

This Work consists of the main source file filecontentsdef.dtx and
the derived files filecontentsdef.sty, filecontentsdef.ins, 
filecontentsdef.pdf, filecontentsdef.dvi, README.md.


CHANGE LOG
==========

v1.2 \[2016/09/19\]
-------------------

Initial version.
%</readme>
%<*tex>-------------------------------------------------------------------------
\chardef\Withdvipdfmx 1 % replace 1 by 0 for using pdflatex
\chardef\NoSourceCode 0 % replace 0 by 1 for the doc *without* the source code
\NeedsTeXFormat{LaTeX2e}
\ProvidesFile {filecontentsdef.tex}[Driver for filecontentsdef documentation]%
\PassOptionsToClass   {a4paper,fontsize=11pt,oneside}{scrdoc}
\PassOptionsToPackage {english}{babel}
\input filecontentsdef.dtx
%%% Local Variables:
%%% mode: latex
%%% End:
%</tex>-------------------------------------------------------------------------
%<*insfile>---------------------------------------------------------------------
\input docstrip.tex
\askforoverwritefalse
\def\pkgpreamble{\defaultpreamble^^J\MetaPrefix^^J%
\string\NeedsTeXFormat{LaTeX2e}^^J%
\string\ProvidesPackage{filecontentsdef}^^J%
\space[\pkgdate\space\pkgversion\space\pkgdescription]}%
\generate{\nopreamble\nopostamble
\file{README.md}{\from{\pkgname.dtx}{readme}}%
\usepostamble\defaultpostamble
\file{\pkgname.tex}{\from{\pkgname.dtx}{tex}}%
\usepreamble\pkgpreamble
\file{\pkgname.sty}{\from{\pkgname.dtx}{package}}}%
\catcode32=13\relax% active space
\let =\space%
\Msg{************************************************************************}
\Msg{*}
\Msg{* To finish the installation you have to move the following}
\Msg{* file into a directory searched by TeX:}
\Msg{*}
\Msg{*     \pkgname.sty} 
\Msg{*}
\Msg{* To produce the documentation run latex twice on file \pkgname.tex}
\Msg{* and then run dvipdfmx on \pkgname.dvi.}
\Msg{*}
\Msg{* Happy TeXing!}
\Msg{*}
\Msg{************************************************************************}
\ifx\numexpr\undefined
\Msg{* warning: to get correct utf-8 encoded README.md }%
\Msg{* do etex \pkgname.ins and not tex \pkgname.ins   }%
\Msg{************************************************************************}
\fi
\endbatchfile
%</insfile>---------------------------------------------------------------------
%<*none>------------------------------------------------------------------------
\fi
%
\chardef\noetex 0
\ifx\numexpr\undefined\chardef\noetex 1 \fi
\ifnum\noetex=1 \chardef\extractfiles 0 % extract files, then stop
\else
    \ifx\ProvidesFile\undefined
      \chardef\extractfiles 0 % etex etc.. on dtx, only file extraction.
    \else % latex/pdflatex
      \ifx\Withdvipdfmx\undefined
        % latex/pdflatex on dtx
        \chardef\extractfiles 1 % 1 = extract files and typeset manual, 2 = only typeset
        \chardef\Withdvipdfmx 1 % 0 = pdflatex or latex+dvips, 1 = dvipdfmx
        \chardef\NoSourceCode 0 % 0 =  include source code, 1 = do not
        \NeedsTeXFormat {LaTeX2e}%
        \PassOptionsToClass   {a4paper,fontsize=11pt,oneside}{scrdoc}% 
        \PassOptionsToPackage {english}{babel}%
      \else % latex on tex
        \chardef\extractfiles 2 % do not extract files, only typeset
      \fi
      \ProvidesFile{\pkgname.dtx}%
        [\pkgname\space source and documentation (\dtxtimestamp)]%
    \fi
\fi
\ifnum\extractfiles<2 % extract files
\def\MessageDeFin{\newlinechar10 \let\Msg\message
\Msg{********************************************************************^^J}%
\Msg{*^^J}%
\Msg{* To finish the installation you have to move the following^^J}%
\Msg{* file into a directory searched by TeX:^^J}%
\Msg{*^^J}%
\Msg{*\space\space\space\space \pkgname.sty^^J}%
\Msg{*^^J}%
\Msg{* To produce the documentation with source code included run latex^^J}%
\Msg{* twice on file \pkgname.tex and then dvipdfmx on \pkgname.dvi^^J}%
\Msg{*^^J}%
\Msg{* Happy TeXing!^^J}%
\Msg{*^^J}%
\Msg{********************************************************************^^J}%
}%
\begingroup
    \input docstrip.tex
    \askforoverwritefalse
    \def\pkgpreamble{\defaultpreamble^^J\MetaPrefix^^J%
    \string\NeedsTeXFormat{LaTeX2e}^^J%
    \string\ProvidesPackage{\pkgname}^^J%
    \space[\pkgdate\space\pkgversion\space\pkgdescription]}%
    \generate{\nopreamble\nopostamble
    \file{README.md}{\from{\pkgname.dtx}{readme}}%
    \usepostamble\defaultpostamble
    \file{\pkgname.ins}{\from{\pkgname.dtx}{insfile}}%
    \file{\pkgname.tex}{\from{\pkgname.dtx}{tex}}%
    \usepreamble\pkgpreamble
    \file{\pkgname.sty}{\from{\pkgname.dtx}{package}}}%
\endgroup
\fi % end of file extraction (from etex/latex/pdflatex run)
\ifnum\noetex=1 % warning for README.md
    \expandafter\def\expandafter\MessageDeFin\expandafter 
{\MessageDeFin
\Msg{* warning: to get correct utf-8 encoded README.md ^^J}%
\Msg{* do etex \pkgname.dtx and not tex \pkgname.dtx   ^^J}%
\Msg{********************************************************************^^J}}%
\fi
\ifnum\extractfiles=0 % tex/etex/xetex/etc files extracted, stop
      \MessageDeFin\expandafter\end
\fi
% From this point on, run is necessarily with e-TeX.
% Check if \MessageDeFin got defined, if yes put it at end of run.
\ifdefined\MessageDeFin\AtEndDocument{\MessageDeFin}\fi
%-------------------------------------------------------------------------------
% START OF USER MANUAL TEX SOURCE
\documentclass[abstract]{scrdoc}

\ifnum\NoSourceCode=1 \OnlyDescription\fi

\usepackage{ifpdf}
\ifpdf\chardef\Withdvipdfmx 0 \fi

\makeatletter
\ifnum\Withdvipdfmx=1
   \@for\@tempa:=hyperref,bookmark,graphicx,xcolor,pict2e\do
            {\PassOptionsToPackage{dvipdfmx}\@tempa}
   %
   \PassOptionsToPackage{dvipdfm}{geometry}
   \PassOptionsToPackage{bookmarks=true}{hyperref}
   \PassOptionsToPackage{dvipdfmx-outline-open}{hyperref}
   \PassOptionsToPackage{dvipdfmx-outline-open}{bookmark}
   %
   \def\pgfsysdriver{pgfsys-dvipdfm.def}
\else
   \PassOptionsToPackage{bookmarks=true}{hyperref}
\fi
   \let\original@check@percent\check@percent
   \let\check@percent\relax
\makeatother

\usepackage[T1]{fontenc}
\usepackage[utf8]{inputenc}
\usepackage[hscale=0.66,vscale=0.75]{geometry}
\pagestyle{headings}

\def\MacroFont{\ttfamily\small\hyphenchar\font45 \baselineskip11pt\relax}

% ATTENTION: (avec doc.sty ou les classes scrdoc ou ltxdoc)
%
% - l'environnement macrocode se fait avec \macro@font qui est le
% \MacroFont du \begin{document}.
%
% - les environnements verbatim utilisent le \MacroFont courant.
%
% - \verb utilise un \ttfamily !! et non pas \verbatim@font que l'on
% peut customiser.
%
% - (9 mars 2015) il ne faut PAS commencer des macrocode directement après
% un \section, il faut un paragraphe (par exemple un \indent\par)

\usepackage{xspace}
\usepackage[dvipsnames]{xcolor}

\definecolor{joli}{RGB}{225,95,0}
\definecolor{JOLI}{RGB}{225,95,0}

\newcommand\fcdname{%
  \texorpdfstring{{\color{joli}\ttfamily\bfseries \pkgname}}{\pkgname}\xspace}

\DeclareRobustCommand\csa [1] {{\ttfamily\hyphenchar\font45 \char`\\ #1}}

\newcommand\csh[1]{\texorpdfstring{\csa{#1}}{\textbackslash\detokenize{#1}}}

\usepackage {babel} % ngerman and english options have been passed to babel
\AtBeginDocument {%
  \renewcommand\partname{Part}%
  \addto\captionsenglish {\renewcommand\partname{Part}}%
}

\usepackage[pdfencoding=pdfdoc]{hyperref}
\hypersetup{%
linktoc=all,%
%% bookmarksdepth=3,%
breaklinks=true,%
colorlinks,%   
linkcolor=OliveGreen,%RoyalBlue,%
urlcolor=RoyalBlue,%OliveGreen,%
pdfauthor={Jean-Fran\c cois Burnol},%
pdftitle={The \pkgname\space package},%
pdfsubject={\pkgdescription},%
pdfkeywords={LaTeX, files, verbatim},%
pdfstartview=FitH,%
pdfpagemode=UseOutlines}
\usepackage{bookmark} 

\usepackage[zerostyle=a,scaled=0.95,straightquotes]{newtxtt}
\renewcommand\familydefault\sfdefault
\frenchspacing

\usepackage{framed}

%\usepackage{varioref}

%\usepackage{footnotehyper}

\usepackage{filecontentsdef}

\makeatletter
\newcommand\inmarg [1]{\@bsphack
    \vadjust{\vskip-\dp\strutbox
             \smash{\hbox to 0pt
                       {\color[named]{PineGreen}\normalfont\small
                        \hsize 2.5cm\rightskip.5cm minus.5cm
                        \hyphenpenalty\z@\exhyphenpenalty\z@
                        \doublehyphendemerits\z@\baselineskip9pt
                        \hss\vtop{\noindent#1}\kern.25cm }}%
             \vskip\dp\strutbox }\strut\@esphack}
\makeatother

\begin{document}
\rmfamily
\thispagestyle{empty}

\bookmark[named=FirstPage,level=1]{Title page}

%\ttzfamily

{%
\normalfont\Large\parindent0pt \parfillskip 0pt\relax
 \leftskip 2cm plus 1fil \rightskip 2cm plus 1fil
 The \fcdname package\par
}

{\centering
  \textsc{Jean-François Burnol}\par
  \footnotesize
  jfbu (at) free (dot) fr\par
  Package version: \pkgversion\ (\pkgdate);
            documentation date: \pkgdocdate.\par
  {From source file \texttt{\pkgname.dtx}. \dtxtimestamp.}\par
}

\begin{abstract}
  This lightweight \LaTeX2e package provides environments
  |filecontentsdef| and |filecontentshere|. They are derived from
  the \LaTeX\ |filecontents| environment as extended by
  \textsc{Scott Pakin}'s
  \href{http://www.ctan.org/pkg/filecontents}{filecontents}
  package.\footnote{\fcdname works independently from
    \href{http://www.ctan.org/pkg/filecontents}{filecontents} and
    does not load it.} In addition to the file creation they either
  store the (verbatim) contents in a macro (|filecontentsdef|) or
  typeset them (verbatim) on the spot (|filecontentshere|).

  I developed this to display \TeX\ code verbatim in documentation
  and simultaneously produce during the LaTeX run the corresponding
  files in order to embed them in the PDF as \emph{file attachment
    annotations} (via the services of \textsc{Scott Pakin}'s
  further package
  \href{http://www.ctan.org/pkg/attachfile}{attachfile}.)
\end{abstract}

\section{Description}

The environment\inmarg{|file\-contents\-here|}
\begin{verbatim}
    \begin{filecontentshere}{<filename>}
    ... arbitrary contents ...
    \end{filecontentshere}
\end{verbatim}
creates on the fly a file with these contents, and simultaneously
it typesets them in a verbatim environment. There is no syntax
  highlighting whatsoever.
\begin{footnotesize}
\begin{enumerate}
\item 
    The contents are not completely arbitrary, as they may not contain
    |\end{filecontentshere}| itself...\par
\item This uses underneath the |verbatim| environment and this has been
  tested to be compatible with the standard |verbatim|, with the one from
  package \href{http://www.ctan.org/pkg/doc}{doc} (classes |ltxdoc.cls|,
  |scrdoc.cls|) and also with the one from package
  \href{http://www.ctan.org/pkg/verbatim}{verbatim} (whose mechanism is quite
  different from the one of the default |verbatim| environment.)
\end{enumerate}
\end{footnotesize}

\medskip
The other environment is |filecontentsdef|.\inmarg{|file\-contents\-def|}
It has a second mandatory argument, a macro.

\begin{verbatim}
    \begin{filecontentsdef}{<filename>}{\macro}
    ... arbitrary contents ...
    \end{filecontentsdef}
\end{verbatim}


It creates the file and rather than typesetting it verbatim
simultaneously, it stores its (verbatim) contents in its second argument
|\macro|.

\begin{enumerate}
\item the scope of the macro definition is global,
\item |filecontentshere| is a wrapper of the |filecontentsdef|
  environment using \csa{filecontentsheremacro} as the macro where the
  contents are stored.\inmarg{\csa{file\-contents\-here\-macro}} This macro
  can then be reused elsewhere if wanted.
\item both environments admit the starred form which does \emph{not} add
the usual three comment lines at the top of the written file (those lines
are anyhow not typeset by |filecontentshere| nor are they included in the
macro by |filecontentsdef|).
\end{enumerate}

\begin{framed}
  Please note that |filecontentsdef| basically stores sanitized
  (i.e. verbatim) tokens in its macro argument |\macro|.
  
  If the material consists of \LaTeX\ code, the expansion of the
  macro will only typeset some \emph{verbatim} rendering of the
  \LaTeX\ code.
\end{framed}

\begin{footnotesize}
  \def\MacroFont{\ttfamily\footnotesize\hyphenchar\font45
    \baselineskip\the\baselineskip\relax} Using \eTeX's
  \csa{scantokens}, one can re-tokenize the macro contents and (here
  naturally, we are talking about the situation where the macro
  contains \TeX/\LaTeX\ code): we obtain its ``execution'' via
  |\scantokens\expandafter{\macro}|. Due to the way |\scantokens|
  works, this must be done with |\newlinechar| set to |13| (to match
  the |^^M|'s; see later in this documentation). Example:
\begin{verbatim}
\begin{filecontentsdef}{\jobname.test}{\macro}
  \begin{framed}
    \noindent 
                We have coded this in \LaTeX: both
      $E=mc^2$ (input as |$E=mc^2$|)
    and     $E=h\nu$  owe much    to \textsc{Albert Einstein}.
  \end{framed}
\end{filecontentsdef}
{\newlinechar13 \scantokens\expandafter{\macro}}
\end{verbatim}
\begin{filecontentsdef}{\jobname.test}{\macro}
  \begin{framed}
    \noindent 
                We have coded this in \LaTeX: both
      $E=mc^2$ (input as |$E=mc^2$|)
    and     $E=h\nu$  owe much    to \textsc{Albert Einstein}.
  \end{framed}
\end{filecontentsdef}
\filecontentsexec\macro Notice\footnote{The absence of indentation
  at the start of this paragraph is a funny effect due to it
  immediately following |framed| which itself immediately follows a
  |verbatim|.} that a space token will generally appear at the end of
the expansion, due to |\scantokens|'s way of working. This is an
end-of-line space, which we could suppress via |\endlinechar-1\relax|
before the |\scantokens|, but that is an option only in the case of
single-line contents. If we had written above \verb|\end{framed}%| or
|\end{framed}\relax| in our use of |filecontentsdef| this would have
prevented |\scantokens| from inserting this final space (naturally in
this example the space is issued while \TeX\ is in vertical mode and
leaves no trace anyhow).\footnote{for basic information on this issue,
  see:
\url{http://tex.stackexchange.com/questions/117906/use-of-everyeof-and-endlinechar-with-scantokens}}

Although \fcdname itself makes no use of \eTeX, it provides as a
convenience
\csa{filecontentsexec}\inmarg{\csa{file\-contents\-exec}\csa{macro}}
which will use |\scantokens| to execute a |\macro| as above (thus
assuming it contains legitimate but possibly verbatimized \LaTeX\
code.) Rather than using a group (possibly |\macro| makes some
non-global definitions) it issues |\newlinechar10\relax| (as this is
the default -- we could have stored and restored current value, but
well...) after the |\scantokens|.\par
\end{footnotesize}

As an example consider the following (with some |utf8| characters
among those which are available in |T1|-encoded \TeX-fonts as used
by this document with the help of |fontenc| and |inputenc|):
\begin{verbatim}
\begin{filecontentsdef}{filecontentsdef.license}{\fcdlicense}
This Work may be distributed and/or modified under the
conditions of the LaTeX Project Public License 1.3c.
This version of this license is in

> <http://www.latex-project.org/lppl/lppl-1-3c.txt>

and the latest version of this license is in

> <http://www.latex-project.org/lppl.txt>

and version 1.3 or later is part of all distributions of
LaTeX version 2005/12/01 or later.

The Author of this Work is:

- Jean-François Burnol `<jfbu at free dot fr>`

This Work consists of the main source file filecontentsdef.dtx and
the derived files filecontentsdef.sty, filecontentsdef.ins, 
filecontentsdef.pdf, filecontentsdef.dvi, README.md.

CHANGE LOG
==========

v1.2 \[2016/09/19\]
-------------------

Initial version.

test: éèàùÉÈÇÀÙÛÎåðñòóôõöœøùúûüýþߟŽ§
\end{filecontentsdef}
\end{verbatim}

\begin{filecontentsdef}{filecontentsdef.license}{\fcdlicense}
This Work may be distributed and/or modified under the
conditions of the LaTeX Project Public License 1.3c.
This version of this license is in

> <http://www.latex-project.org/lppl/lppl-1-3c.txt>

and the latest version of this license is in

> <http://www.latex-project.org/lppl.txt>

and version 1.3 or later is part of all distributions of
LaTeX version 2005/12/01 or later.

The Author of this Work is:

- Jean-François Burnol `<jfbu at free dot fr>`

This Work consists of the main source file filecontentsdef.dtx and
the derived files filecontentsdef.sty, filecontentsdef.ins, 
filecontentsdef.pdf, filecontentsdef.dvi, README.md.

CHANGE LOG
==========

v1.2 \[2016/09/19\]
-------------------

Initial version.

test: éèàùÉÈÇÀÙÛÎåðñòóôõöœøùúûüýþߟŽ§
\end{filecontentsdef}

The file |filecontentsdef.license| is created with the usual three commentary
lines at top of it. And macro \csa{fcdlicense} contains the verbatim material.
It can then be expanded anywhere in the document. Here are some relevant
details:

\begin{enumerate}
\item the usual special characters are sanitized like in a verbatim
  environment,
\item spaces become the active character of ascii code |32|, and
  end of lines are converted into the active character of ascii
  code |13| (i.e. |^^M|),
\item the tabs |CTRL-I| have been converted to active spaces,
\item the form feeds |CTRL-L| have been converted to blank lines (|^^M^^M|),
\item the characters of ascii code between |128| and |255| have been
  either given the catcode |letter|, or if they were active (which
  will be the case with package |inputenc|), they are just inserted in
  the produced macro as active characters.
\end{enumerate}

Thus what is needed before inserting \csa{fcdlicense} in the document
is to give definitions to the active space and the active |^^M|.
\LaTeX\ and \TeX\ both provide \csa{obeyspaces} and \csa{obeylines}.
For a true verbatim printout, these are usually not enough because
spaces at start of lines will disappear, and multiple empty lines give
multiple |\par|'s which collapse into a single one (hence no empty
line can be observed in the output). The usual |verbatim| environment
uses a special definition of |\par| which prevents the disappearance
of empty lines, and for the spaces it has macro \csa{@vobeyspaces}
which makes the spaces issue |\leavevmode| so they are not skipped at
the start of lines. Let's define:

\begin{verbatim}
\makeatletter
% this redefines active spaces, but does not make spaces active
\def\niceactivespaces{\@vobeyspaces\catcode32=10\relax}% 
\makeatother
\begingroup
% this redefines active end of lines, but does not make them active
  \catcode`\^^M\active %
  \gdef\niceactiveCRs{\def^^M{\leavevmode\par}}%
\endgroup %
\end{verbatim}

Then we can issue something like (the output is not shown):
\begin{verbatim}
{\setlength{\parindent}{1cm}\niceactivespaces\niceactiveCRs\fcdlicense}
\end{verbatim}
Notice however this will still allow hyphenation and ligatures, which
are usually inhibited in standard |verbatim| (and also we have not
switched to the monospace font.) To emulate exactly what a real
|verbatim| would give, \fcdname provides
\csa{filecontentsprint}\inmarg{\csa{file\-contents\-print}\csa{macro}}
which is a command with one mandatory argument whose invocation will
produce the same as what
\begingroup\makeatletter\def\x{\let\@xverbatim\relax
       \verbatim
\string\begin\string{verbatim\string}\par
<contents of \string\macro>\par
\string\end\string{verbatim\string}\par
\endverbatim%
\endgroup}\x

% % bizarre
% % \texttt{\parindent0pt\relax\niceactivespaces\niceactiveCRs\fcdlicense}% 
% % crée l'indentation il faut faire le \parindent0pt avant !
% % hmm, bon pas envie de regarder ce que fait \texttt

% % pour le fun
% % {\setlength{\parindent}{11cm}\niceactivespaces\niceactiveCRs\ttfamily\fcdlicense}

\noindent would have given.\footnote{This is compatible with
  \href{http://www.ctan.org/pkg/verbatim}{verbatim.sty}'s |verbatim| and
  hopefully also with other packages modifiying the way the |verbatim|
  environment works.} The tokens stored in the macro must be of the type
described above. As an illustration, here is the output from
|\filecontentsprint\fcdlicense|:


\filecontentsprint\fcdlicense


\StopEventually{\end{document}}

\makeatletter
    \let\check@percent\original@check@percent
\makeatother

\small

\section{Implementation}

\indent % !!!!!!!!!!!!!!!!! Lundi 09 mars 2015 à 09:32:22


\makeatletter
\begingroup
\topsep\MacrocodeTopsep
\trivlist\parskip\z@\item[]
\macro@font
\leftskip\@totalleftmargin  \advance\leftskip\MacroIndent
\rightskip\z@  \parindent\z@  \parfillskip\@flushglue
\global\@newlistfalse \global\@minipagefalse
\ifcodeline@index
  \everypar{\global\advance\c@CodelineNo\@ne
  \llap{\theCodelineNo\ \hskip\@totalleftmargin}}%
\fi
\string\NeedsTeXFormat\string{LaTeX2e\string}[1999/12/01]\par
\string\ProvidesPackage\string{\pkgname\string}\par
\noindent\space [\pkgdate\space\pkgversion\space\pkgdescription]\par
\nointerlineskip
\global\@inlabelfalse
\endtrivlist
\endgroup
\makeatother

% The catcode hackery next is to avoid to have <*package> to be listed
% in the commented source code...
% (c) 2012/11/19 jf burnol ;-)

\MakePercentIgnore

%
% \catcode`\<=0 \catcode`\>=11 \catcode`\*=11 \catcode`\/=11
% \let</none>\relax
% \def<*package>{\catcode`\<=12 \catcode`\>=12 \catcode`\*=12 \catcode`\/=12}
%
%</none>
%<*package>
% Most of
% the code is still identical to the one in \textsc{Scott Pakin}'s
% \href{http://www.ctan.org/pkg/filecontents}{filecontents} hence to
% the original one in \LaTeX's sources.
%    \begin{macrocode}
\begingroup
\catcode`\^^M\active%
\catcode`\^^L\active\let^^L\relax%
\catcode`\^^I\active%
\gdef\filec@ntentsdef#1#2{%
  \let#2\@empty%
  \openin\@inputcheck#1 %
  \ifeof\@inputcheck%
    \@latex@warning@no@line%
        {Writing file `\@currdir#1'}%
  \else%
    \@latex@warning@no@line%
        {Overwriting file `\@currdir#1'}%
  \fi%
  \closein\@inputcheck%
  \chardef\reserved@c15 %
  \ch@ck7\reserved@c\write%
  \immediate\openout\reserved@c#1\relax%
  \if@tempswa%
    \immediate\write\reserved@c{%
      \@percentchar\@percentchar\space%
          \expandafter\@gobble\string\LaTeX2e file `#1'^^J%
      \@percentchar\@percentchar\space  generated by the %
        `\@currenvir' \expandafter\@gobblefour\string\newenvironment^^J%
      \@percentchar\@percentchar\space from source `\jobname' on %
         \number\year/\two@digits\month/\two@digits\day.^^J%
      \@percentchar\@percentchar}%
  \fi%
  \let\do\@makeother\dospecials%
%    \end{macrocode}
% SP's |filecontents| sets here in the loop all catcodes to |11|, but we
% need for correct rendering in verbatim that the constructed macro
% stores active characters as active characters.
%
% We don't check for unusual active characters of ascii code |<128|
% as this is not done by original or SP's |filecontents|. But if
% present then they will expand similarly both in the |\write| and
% in the construction of the macro.
%    \begin{macrocode}
  \count@=128\relax%
  \loop%
    \ifnum\catcode\count@=\active%
          \lccode`~\count@%
          \lowercase{\def~{\noexpand~}}%
    \else%
          \catcode\count@=11 %
    \fi%
    \advance\count@ by \@ne%
    \ifnum\count@<\@cclvi%
  \repeat%
%    \end{macrocode}
% The default active |^^L|  is |\outer|. But |\reserved@b| will be |def'd|
% with an active |^^L| in its replacement text.
%    \begin{macrocode}
  \let^^L\relax%
  \edef\E{\@backslashchar end\string{\@currenvir\string}}%
  \edef\reserved@b{\def\noexpand\reserved@b####1\E####2\E####3\relax}%
  \reserved@b{%
    \ifx\relax##3\relax%
      \immediate\write\reserved@c{##1}%
%    \end{macrocode}
% This is where the original |filecontents| is extended to store the
% parsed material in a macro (in my very first hack I simply
% patched it to redefine |\write| to also do the macro storage, but
% considerations like the one relative to active characters due to
% |inputenc| made me decide to re-write the whole thing, hence make
% a new package.)
%
% Active characters were defined with a single |\noexpand| in the
% loop, and this is enough because after each new line is processed
% the characters it contains are protected from further expansion in
% the |\xdef|'s. And the single |\noexpand| is enough also for the
% |\write| done above.
%
% The |lccode| of the tilde is |32| when this gets executed. Multiple
% form feeds produce the same effect in the macro (insertion of two
% |^^M| per form feed) as in the written out file (via two |^^J|).
%    \begin{macrocode}
      \toks@\expandafter{#2}%
      {\def^^L{\noexpand^^M\noexpand^^M}\lowercase{\let^^I~}%
       \xdef#2{\the\toks@##1\noexpand^^M}}%
    \else%
      \edef^^M{\noexpand\end{\@currenvir}}%
      \ifx\relax##1\relax%
      \else%
          \@latex@warning{Writing text `##1' before %
             \string\end{\@currenvir}\MessageBreak as last line of #1}%
        \immediate\write\reserved@c{##1}%
%    \end{macrocode}
% Same added code as above.
%    \begin{macrocode}
        \toks@\expandafter{#2}%
        {\def^^L{\noexpand^^M\noexpand^^M}\lowercase{\let^^I~}%
         \xdef#2{\the\toks@##1\noexpand^^M}}%
      \fi%
      \ifx\relax##2\relax%
      \else%
         \@latex@warning{%
           Ignoring text `##2' after \string\end{\@currenvir}}%
      \fi%
    \fi%
    ^^M}%
  \catcode`\^^L\active%
  \let\L\@undefined%
  \def^^L{\@ifundefined L^^J^^J^^J}%
  \catcode`\^^I\active%
  \let\I\@undefined%
  \def^^I{\@ifundefined I\space\space}%
  \catcode`\^^M\active%
  \edef^^M##1^^M{\noexpand\reserved@b##1\E\E\relax}%
%    \end{macrocode}
% We want space characters to be active in the produced macro.
% We only need to protect them once from expansion.
%    \begin{macrocode}
  \catcode32\active\lccode`~32 \lowercase{\def~{\noexpand~}}%
}%
\endgroup
\begingroup
\catcode`\*=11
\gdef\filecontentsdef {\@tempswatrue\filec@ntentsdef}%
\gdef\filecontentsdef*{\@tempswafalse\filec@ntentsdef}%
\global\let\endfilecontentsdef \endfilecontents
\global\let\endfilecontentsdef*\endfilecontents
\gdef\filecontentshere #1{\@tempswatrue
                         \filec@ntentsdef{#1}\filecontentsheremacro}%
\gdef\filecontentshere*#1{\@tempswafalse
                          \filec@ntentsdef{#1}\filecontentsheremacro}%
\gdef\endfilecontentshere{\endfilecontentsdef\aftergroup\filecontents@verbatim}%
\global\let\endfilecontentshere*\endfilecontentshere
%    \end{macrocode}
% Package |verbatim.sty| modifies the standard |verbatim| environment. For
% both the original and the modified version we need to insert an active |^^M|
% upfront, else an empty first line would not be obeyed. The |verbatim.sty|'s
% |verbatim| needs that we feed it with the macro expanded once, as it uses
% active end of lines as delimiters and they thus need to be immediately
% visible. It also needs an active |^^M| after the |\end{verbatim}|. To avoid
% to check at |\AtBeginDocument| if package |verbatim.sty| is loaded, we use a
% slightly tricky common definition. The advantage is that this may help make
% the code compatible with further packages (I have not looked for them)
% modifying the |verbatim| environment. For better code readibility I use
% |^^M%|'s rather than exploiting the active ends of lines here.
%    \begin{macrocode}
\catcode`\^^M\active%
\gdef\filecontentsprint #1{\let\filecontentsprint@EOL^^M\let^^M\relax%
   \begingroup\toks@\expandafter{#1}\edef\x{\endgroup%
            \noexpand\begin{verbatim}^^M%
            \the\toks@\@backslashchar end\string{verbatim\string}}\x^^M%
   \filecontentsprint@resetEOL}%
\gdef\filecontentsprint@resetEOL{\let^^M\filecontentsprint@EOL}%
\endgroup
\def\filecontents@verbatim {\filecontentsprint\filecontentsheremacro}%
\def\filecontentsexec #1{\newlinechar13 
    \scantokens\expandafter{#1}\newlinechar10\relax}%
\endinput
%    \end{macrocode}
% \MakePercentComment
\Finale
%%
%% End of file `filecontentsdef.dtx'.
