\chapter{Introduction}
\label{chap:Intro}

This is the introduction.  And this is how a Figure would look like (see Figure~\ref{fig:FEUP_logo}).


\begin{figure}[t]
	\centering
		\includegraphics[scale=0.5]{Figures/uporto-feup.pdf} %Use 'scale' to set how large the image should be
	\caption{FEUP's Logotype}
	\label{fig:FEUP_logo}
\end{figure}

A quote should look like this:

\begin{quote}
  ``Like the Abstract, the Introduction should be written to engage the
  interest of the reader. It should also give the reader an idea of
  how the dissertation is structured, and in doing so, define the
  thread of the contents.''~\citep[chap.\ Introduction]{kn:Tha01} 
\end{quote}

And a table should be written like Table~\ref{tab:example2}, or even better, Table~\ref{tab:example1}

\begin{itemize}
\item For code, set your preferences using the \emph{lstset} option (see \emph{listings} package documentation on CTAN).

\item For importing figures in Matlab, use Nico Schlömer's \emph{matlab2tikz} files provided in the Matlab central \url{http://www.mathworks.com/matlabcentral/fileexchange/22022-matlab2tikz}

\item If you see yourself in trouble, ask your question at \url{http://tex.stackexchange.com}.

\end{itemize}




\begin{table}
\centering
\caption{Default Specifications}
\label{tab:example1}
\begin{tabular}{ll}
\toprule
\textbf{Parameter} & \textbf{Values}\\
\midrule
\textit{Area} & 3-20 pixels\\
$T$ & 0.65\\
\textit{Number of frames} & 10\\
\textit{Frames per second} & 1\\
\textit{Define movement for }& 3 frames\\
\multirow{2}{*}{\textit{Number of candidate vesicles}} & Strong: 10\\
& Ambiguous: 10\\
\bottomrule
\end{tabular}
\end{table}



\begin{table}[t]
  \centering
  \caption{Example Table}
\begin{tabular}{c|r@{.}lr@{.}lr@{.}l|r}
\multicolumn{8}{c}
	{\rule[-3mm]{0mm}{8mm}Iteration $k$ of $f(x_n)$} \\
\textbf{\em k}
	& \multicolumn{2}{c}{$x_1^k$}
	& \multicolumn{2}{c}{$x_2^k$}
	& \multicolumn{2}{c|}{$x_3^k$}
	& Comments \\ \hline \hline
0   & -0&3                 & 0&6                 &  0&7   & - \\
1   &  0&47102965 & 0&04883157 & -0&53345964  & $\delta<\epsilon$ \\
2   &  0&49988691 & 0&00228830 & -0&52246185  & $\delta < \varepsilon$ \\
3   &  0&49999976 & 0&00005380 & -0&523656   &   $N$ \\
4   &  0&5                 & 0&00000307 & -0&52359743  & \\
\vdots	& \multicolumn{2}{c}{\vdots}
	& \multicolumn{2}{c}{$\ddots$}
	& \multicolumn{2}{c|}{\vdots}  & \\
7   &  0&5   & 0&0    & \textbf{-0}&\textbf{52359878}
		 & $\delta<10^{-8}$ \\ 
\end{tabular}
  \label{tab:example2}
\end{table}


\newpage
For \emph{longtables} check the one available on the \emph{Abbs.tex} file, or Table~\ref{tab:longtableexample}

	\begin{longtable}{lc}
	\caption{Longtable Example}\label{tab:longtableexample}\\
	\toprule
	\textbf{School} & \textbf{Meaning} \\
	\midrule
	\endfirsthead
	\caption*{Table \ref{tab:longtableexample} (Continue): Longtable Example}\\
	\toprule
	\textbf{School} & \textbf{Meaning} \\
	\midrule
	\endhead
		FEUP	&  Faculdade de Engenharia da Universidade do Porto\\
		FEUP	&  Faculdade de Engenharia da Universidade do Porto\\
		FEUP	&  Faculdade de Engenharia da Universidade do Porto\\
		FEUP	&  Faculdade de Engenharia da Universidade do Porto\\
		FEUP	&  Faculdade de Engenharia da Universidade do Porto\\
		FEUP	&  Faculdade de Engenharia da Universidade do Porto\\
		FEUP	&  Faculdade de Engenharia da Universidade do Porto\\
		FEUP	&  Faculdade de Engenharia da Universidade do Porto\\
		FEUP	&  Faculdade de Engenharia da Universidade do Porto\\
		FEUP	&  Faculdade de Engenharia da Universidade do Porto\\
		FEUP	&  Faculdade de Engenharia da Universidade do Porto\\
		FEUP	&  Faculdade de Engenharia da Universidade do Porto\\
		FEUP	&  Faculdade de Engenharia da Universidade do Porto\\
		FEUP	&  Faculdade de Engenharia da Universidade do Porto\\
				FEUP	&  Faculdade de Engenharia da Universidade do Porto\\
		FEUP	&  Faculdade de Engenharia da Universidade do Porto\\
		FEUP	&  Faculdade de Engenharia da Universidade do Porto\\
		FEUP	&  Faculdade de Engenharia da Universidade do Porto\\
		FEUP	&  Faculdade de Engenharia da Universidade do Porto\\
		FEUP	&  Faculdade de Engenharia da Universidade do Porto\\
		FEUP	&  Faculdade de Engenharia da Universidade do Porto\\
		FEUP	&  Faculdade de Engenharia da Universidade do Porto\\
		FEUP	&  Faculdade de Engenharia da Universidade do Porto\\
		FEUP	&  Faculdade de Engenharia da Universidade do Porto\\
		FEUP	&  Faculdade de Engenharia da Universidade do Porto\\
		FEUP	&  Faculdade de Engenharia da Universidade do Porto\\
		FEUP	&  Faculdade de Engenharia da Universidade do Porto\\
		FEUP	&  Faculdade de Engenharia da Universidade do Porto\\
				FEUP	&  Faculdade de Engenharia da Universidade do Porto\\
		FEUP	&  Faculdade de Engenharia da Universidade do Porto\\
		FEUP	&  Faculdade de Engenharia da Universidade do Porto\\
		FEUP	&  Faculdade de Engenharia da Universidade do Porto\\
		FEUP	&  Faculdade de Engenharia da Universidade do Porto\\
		FEUP	&  Faculdade de Engenharia da Universidade do Porto\\
		FEUP	&  Faculdade de Engenharia da Universidade do Porto\\
		FEUP	&  Faculdade de Engenharia da Universidade do Porto\\
		FEUP	&  Faculdade de Engenharia da Universidade do Porto\\
		FEUP	&  Faculdade de Engenharia da Universidade do Porto\\
		FEUP	&  Faculdade de Engenharia da Universidade do Porto\\
		FEUP	&  Faculdade de Engenharia da Universidade do Porto\\
		FEUP	&  Faculdade de Engenharia da Universidade do Porto\\
		FEUP	&  Faculdade de Engenharia da Universidade do Porto\\
		\bottomrule
	\end{longtable}

\begin{landscape}

Just to show you how to turn your page completely.  If you just one to turn one figure, use the \emph{rotate} option instead

\end{landscape}


\section{Example of a Matlab figure (using \emph{matlab2tikx})}

Here is the code obtained from the file in Matlab (just copy and paste it):

\begin{figure}[h]
\centering
\begin{tikzpicture}
\begin{axis}[%
scale only axis,
xmin=0,
xmax=120,
xlabel={Time [s]},
ymin=0,
ymax=120,
ylabel={Distance [m]},
title={Distance vs Time for dataset}
]
\addplot [
color=cyan,
solid,
forget plot
]
table[row sep=crcr]{
1.12 0.9667792\\
8.45 2.5287607\\
8.54 9.913476\\
9.99 10.594364\\
11.1 10.754158\\
12.18 12.246912\\
13.22 13.620711\\
14.27 14.872885\\
15.33 16.090015\\
16.42 17.278852\\
17.54 18.32099\\
18.65 19.388561\\
19.92 20.324896\\
21.43 20.586063\\
22.5 21.477476\\
23.59 22.690266\\
30.08 24.050547\\
30.13 30.270317\\
31.83 31.315056\\
32.94 31.488174\\
33.99 33.107754\\
35.03 34.373497\\
36.08 35.63807\\
37.13 36.723347\\
38.19 37.94308\\
39.31 39.134987\\
40.43 40.300858\\
41.94 41.369823\\
43.19 42.55742\\
44.31 44.03123\\
45.34 45.18666\\
52.38 46.42933\\
53.76 53.1462\\
54.16 53.2205\\
55.2 54.755077\\
56.24 56.024307\\
57.27 57.08118\\
58.32 58.184376\\
59.37 59.19243\\
60.44 60.24227\\
61.53 61.134487\\
62.95 61.925873\\
64.13 63.123123\\
65.19 64.73521\\
66.21 66.29376\\
72.82 67.88995\\
74.62 75.39087\\
75.15 75.57744\\
76.22 77.04989\\
77.27 78.37899\\
78.31 79.68579\\
79.39 81.0405\\
80.48 82.293175\\
81.61 83.4804\\
82.74 84.5754\\
84.09 85.688416\\
85.53 86.14043\\
86.65 87.05914\\
87.7 88.49611\\
97.14 89.8551\\
98.68 98.40683\\
99 99.85365\\
100.08 101.18116\\
101.17 102.46642\\
102.27 103.613914\\
103.43 104.70302\\
104.57 105.7168\\
105.83 106.64546\\
};
\end{axis}
\end{tikzpicture}%
\caption{Complete distance vs time for the cyan dataset}
\label{fig:ALLDistanceTimeFig}
\end{figure}
