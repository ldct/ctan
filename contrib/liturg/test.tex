% +AMDG  This document was begun on 17 Jan 2008, the feast of St.
% Antonius the Abbot, and it is humbly dedicated to him, to St.
% Gregory the Great, and to the Immaculate Heart of Mary for their
% prayers, and to the Sacred Heart of Jesus for His mercy.

\documentclass{article}
\title{Testing the \texttt{liturg} Package}
\author{Donald P.\ Goodman III}
\date{\today}

\usepackage{liturg}
\usepackage{multicol}

\begin{document}

\maketitle
\tableofcontents

\section{Introduction}
\label{sect:intro}

There really isn't much to say here, or much to introduce.  The
purpose of this document is to demonstrate the capabilities of the
\texttt{liturg} package for typesetting liturgical texts,
specifically Catholic liturgical texts.  So I've selected two
fairly typical liturgical texts, one from the Missal of
1962\footnote{\textit{See infra}, Section \ref{sect:missal}, at
\pageref{sect:missal}.} and one from the Breviary of
1961.\footnote{\textit{See infra}, Section \ref{sect:brev}, at
\pageref{sect:brev}.}

Nothing about the package makes it unsuitable for the new liturgy,
and indeed it can be used to typeset liturgical texts from the new
liturgy quite beautifully.  It was, however, quite unapologetically
designed with the old liturgy specifically in mind.

\section{A Selection from the Missal}
\label{sect:missal}

\begin{multicols}{2}

\latinpart

\prayerheading{Orationes ad Pedem Altaris}

\instruct{The priest, bowing down at the foot of the altar, makes
the Sign of the Cross and says,}

\priest{In n'omine Patris, et F'ilii, + et Sp'iritus Sancti.}
\server{Amen.}
\priest{Intro'ibo ad alt'are Dei.}
\server{Ad Deum qui l"aet'ificat juvent'utem meam.}

\psalmheading{Psalm 42---Judica me}

\priest{J'udica me, Deus, et disc'erne causam meam de gente non
sancta: ab h'omine in'iquo, et dol'oso 'erue me.}
\server{Quia tu es Deus fortit'udo mea: quare me repul'isti, et
quare tristis inc'edo, dum affl'igit me inim'icus?}
\priest{Em'itte lucem tuam, et verit'atem tuam: ipsa me
dedux'erunt, et addux'erunt in montem sanctum tuum et in
tabern'acula tua.}
\server{Et intro'ibo ad alt'are Dei: ad Deum qui l"aet'ificat
juvent'utem meam.}
\priest{Confit'ebor tibi in c'ithara, Deus, Deus meus: quare
tristis es 'aima mea, et quare cont'urbas me?}
\server{Spera in Deo, qu'oniam adhuc confit'ebor illi: salut'are
vultus mei, et Deus meus.}
\priest{Gl'oria Patri, et F'ilio et Spir'itui Sancto.}
\server{Sicut erat in princ'ipio, et nunc, et semper: et in
s'aecula s"aecul'orum.  Amen.}
\instruct{The priest repeats the anthem.}
\priest{Intro'ibo ad alt'are Dei.}
\server{Ad Deum qui l"aet'ificat juvent'utem meam.}
\instruct{The priest, signing himself with the Sign of the Cross,
says:}
\priest{Adjut'orium nostrum + in n'omine D'omini.}
\server{Qui fecit c"aelum et terram.}
\instruct{Then, joining his hands, and humbly bowing down, he says
the Confiteor:}
\priest{Confiteor Deo\ldots}
\server{Miseare'atur tui omn'ipotens Deus, et dim'issis pecc'atis
tuis, perd'ucat te ad vitam "aet'ernam.}
\priest{Amen.}
\instruct{The server says the Confiteor:}

\leslettrine{C}onf'iteor Deo omnipot'enti, be'at"ae Mar'i"ae semper
V'irgini, be'ato Micha'eli Arch'angelo, be'ato Jo'anni Bapt'ist"ae,
sanctis Ap'ostolis Petro et Paulo, 'omnibus Sanctis, et tibi Pater:
quia pecc'avi nimis congitati'one, verbo, et 'opere:

\instruct{Here he strikes his breast thrice.}
Mea culpa, mea culpa, mea m'axima culpa.  Ideo precor be'atam
Mar'iam semper V'irginem, be'atum Micha'elem Arch'angelum, be'atum
Jo'annem Baptistam, sanctos Ap'ostolos Patrum et Paulum, omnes

\instruct{Then the priest, with his hands joined, says:}
\priest{Misere'atur vestri omn'ipotens Deus, et dim'issis pecc'atis
vestris, perd'ucat vos ad vitam "aet'ernam.}
\server{Amen.}
\instruct{Signing himself with the Sign of the Cross, he says:}
\priest{Indulg'entiam, + absoluti'onem, et remissi'onem peccat'orum
nostr'orum, tr'ibuat nobis omn'ipotens et mis'ericors D'ominus.}
\server{Amen.}
\instruct{Bowing down, he proceeds:}
\priest{Deus, tu conv'ersus vivific'abis nos.}
\server{Et plebs tua l"aet'abitur in te.}
\priest{Ost'ende nobis D'omine, miseric'ordiam tuam.}
\server{Et salut'are tuum da nobis.}
\priest{D'omine, ex'audi orati'onem meam.}
\server{Et clamor meus ad te v'eniat.}
\priest{D'ominus vob'iscum.}
\server{Et cum sp'iritu tuo.}
\priest{Or'emus.}

\englishpart

\end{multicols}

\newpage

\section{A Selection from the Breviary}
\label{sect:brev}

\begin{multicols}{2}

\latinpart

\feasttitle[Commune vide \page{(132)}]{Die 7 martii}{S.\ THOM\AE\ DE 
AQUINO}{Conf.\ et Eccl.\ Doct.}{III}
\hourheading{Ad Matutinum}

\lessonheading{Lectio iii}

\leslettrine{T}homas Aqu'inas, nob'ilibus par'entibus natus, iam
adul'escens, inv'itis matre et fr'atribus, Ordinem Pr"aedicat'orum
susc'epit et Lut'etiam Parisi'orum missus est.  Verum fratres, in
it'inere eum aggr'essi, in arcem castri sancti Jo'annis perd'ucunt,
ubi ang'elicus i'uvenis mul'ierem, qu"ae ad labefact'andam eius
castit'atem introd'ucta f'uerat, titi'one fug'avit.  Patr'isiis
philosoph'i"ae ac theolog'i"ae ita 'operam dedit, ut vix vig'inti
quinque annos natus, p'ublice phil'osophos ac the'ologos summa cum
laude interpret'atus sit.  Nunquam se lecti'oni aut scripti'oni
dedit, nisi post orati'onem.  Cum aliqu'ando hanc Iesu crucif'ixi
vocem aud'isset:  Bene scrips'isti de me, Thoma, quam ergo
merc'edem acc'ipies? amandt'issime resp'ondit: Non 'aliam, D'omine,
nisi te'ipsum.  Nullum fuit script'orum genus, in quo non eset
diligent'issime vers'atus.  Ab Urb'ano quarto Romam voc'atus, eius
iussu lucubr'avit Officium pro solemnit'ate C'orporis Christi.
Missus a be'ato Greg'orio d'ecimo ad conc'ilium Lugdun'ense, in
monast'erio Foss"ae Nov"ae in morbum 'incidit, et ibi "aegr'otus
C'antica cantic'orum explan'avit.  Ib'idem 'obiit quiquagen'arius,
anno mill'esimo ducent'esimo septuag'esimo quarto, Nonis m'artii.
Ipsum Leo d'ecimus t'ertius c"ael'estem patr'onum schol'arum
'omnium catholic'arum declar'avit in inst'ituit.

\blackinstruct{Te Deum laud'amus 7*.}

\hourheading{Ad Laudes}

C  \versic{Iustum deduxit D'ominus per vias rectas.}  \response{Et
ost'endit illi regnum Dei.}

C  \markup{Ad Bened.\ ant.  }  Euge, serve bone * et fid'elis,
quia in pauca fu'isti fid'elis, supra multa te const'ituam, intra
in g'audium D'omini tui.

\prayerheading{Oratio}

\leslettrine{D}eus, qui Eccl'esiam tuam be'ati Thom"ae Confess'oris
tui mira eruditi'one clar'ificas, et sancta operati'one fec'undas:
da nobis, qu'aesumus; et qu"ae d'ocuit, intell'ectu consp'icere, et
qu"ae egit, imitati'one compl'ere.  Per D'ominum.

\hourheading{Ad Vesperas}

C  \versic{Iustum ded'uxit D'ominus per vias rectas.}  \response{Et
ost'endit illi regnum Dei.}

C  \markup{Ad Magnif.\ ant.}  O Doctor 'optime, * Eccl'esi"ae
sanct"ae lumen, be'ate Thoma, div'in"ae legis am'ator, deprec'are
pro nobis F'ilium Dei.

\end{multicols}

\englishpart

Here we see that simply by issuing the \verb|\englishpart| command,
we can return to normal English typesetting without difficulty, or
even a noticeable transition in the text.  This is a feature, not a
bug, no matter what it looks like.  The user is responsible for
providing the logical structure; once they do that, this package
does provide for the appearance.  However, if the user doesn't give
structure, the package just prints out what it gets.

Now I'll provide a quick example of the typesetting of one
of the great feast's headings, much simpler than that of a
normal feast:
\greatfeast{Feria Quinta in Cena Domini}{I}
As one can see, it's a very simple algorithm for a very
simple problem, and is largely trivial.

\end{document}
