% isoman.tex   User guide for ISO style files for LaTeX(2e) August 2002
%
\documentclass[wd,letterpaper,copyright]{isov2}
%\documentclass[wd,letterpaper,draft]{isov2}
%\documentclass{isov2}

%%%\usepackage[isoman]{tex4ht}

\usepackage{comment}

%%%\usepackage{times}
\usepackage{isorot}
\usepackage{xtab}
\usepackage{hyphenat}

\ifpdf
  \pdfoutput=1
  \usepackage[plainpages=false,
              pdfpagelabels,
              bookmarksnumbered,
              hyperindex=true
             ]{hyperref}
\fi

\standard{LaTeX for ISO standards}
\yearofedition{2002}
\languageofedition{(E)}

%\partno{3456}
\changemarkstrue

\makeindex

%  Uncomment the following to change the Foreword heading
%\renewcommand{\forewordname}{Avant-propos} % change the Foreword title

\setcounter{tocdepth}{3} % add more levels to table of contents
%
%    remainder of preamble is some special macro definitions
\makeatletter
%   the \meta{} command
%
\begingroup
\obeyspaces%
\catcode`\^^M\active%
\gdef\meta{\begingroup\obeyspaces\catcode`\^^M\active%
\let^^M\do@space\let \do@space%
\def\-{\egroup\discretionary{-}{}{}\hbox\bgroup\it}%
\m@ta}%
\endgroup
\def\m@ta#1{\leavevmode\hbox\bgroup\texttt{<}\textit{#1}\/\texttt{>}\egroup
    \endgroup}
\def\do@space{\egroup\space
    \hbox\bgroup\it\futurelet\next\sp@ce}
\def\sp@ce{\ifx\next\do@space\expandafter\sp@@ce\fi}
\def\sp@@ce#1{\futurelet\next\sp@ce}
%
%
\makeatother
%
% the \latex command
\newcommand{\latex}{LaTeX}
\newcommand{\tex}{TeX}
%
%   the \file{} command
%
\newcommand{\file}[1]{\textsf{#1}}
%
\makeatletter
%   index a command
\newcommand{\bs}{\symbol{'134}}
\newcommand{\ixcom}[1]{\index{#1/ @{\tt \protect\bs #1}}}
%   index an environment
\newcommand{\ixenv}[1]{\index{#1 @{\tt #1} (environment)}}
%   index a starred environment
\newcommand{\ixenvs}[1]{\index{#1s @{\tt #1*} (environment)}}
%   index an option
\newcommand{\ixopt}[1]{\index{#1 @{\tt #1} (option)}}
%   index a package
\newcommand{\ixpack}[1]{\index{#1 @\file{#1} (package)}}
%   index a class
\newcommand{\ixclass}[1]{\index{#1 @\file{#1} (class)}}
%   index in typewriter font
\newcommand{\ixtt}[1]{\index{#1@{\tt #1}}}
%   index LaTeX
\newcommand{\ixltx}{\index{latex@\latex}}
%   index LaTeX 2e
\newcommand{\ixltxe}{\index{latex2e@\latex 2e}}
%   index LaTeX v2.09
\newcommand{\ixltxv}{\index{latex209@\latex{} v2.09}}
\makeatother
%
%   end of preamble
%
\begin{document}


\begin{cover}
\vspace*{4in}
\begin{center}
\Huge\bfseries LaTeX for ISO standards
\end{center}
\begin{center}
\bfseries 2002/08/10
\end{center}
\begin{center}
Peter Wilson \\
\texttt{peter.r.wilson@boeing.com}
\end{center}
\clearpage
\end{cover}
%%%\clearpage

\begin{foreword}
\fwdbp
\fwdnopatents
Annexes~\ref{anx:extraiso} and~\ref{anx:lord}
are an integral part of this document.
Annexes~\ref{anx:indexing}, \ref{anx:sgml}, \ref{anx:getstuff},
and \ref{anx:changes}
are for information only.
\end{foreword}


\begin{introduction}

    This document describes the use of the \file{isov2} \latex{} class 
file and some package files in preparing ISO standard documents.

\sclause*{Overview}

    This document describes a \latex{} class file, called \file{isov2},
 for typesetting ISO standards. It also
provides descriptions of some particular package files (e.g., \file{isorot}) 
that have been developed to support the writing of ISO standards.
The electronic source of this  document also provides an example of the
use of these files.



    The current set of files~\bref{isoe} have been developed by 
Peter Wilson (RPI, CUA and NIST, Boeing)
from files written by Kent Reed (NIST).
In turn, these were revisions of files originally created by 
Phil Spiby (CADDETC, UK), based on early work by 
Phil Kennicott (GE).\footnote{In mid 1994 \latex{} was upgraded from 
version 2.09 to what is called \latex 2e. The files described in 
earlier versions this document were compatible with both versions 
of \latex. Starting with the October 1997 release, 
support was withdrawn from any \latex{} v2.09 file versions.}


    Documents produced with the \latex{} files have been reviewed 
by the ISO Editorial Board in Geneva for conformance to their typographical
requirements. The first review was of a series of Draft International 
Standard documents. This review resulted in some changes to the style 
files. The second review was of a series of twelve International Standard 
documents (ISO 10303:1994). Likewise, this review led to changes in the 
style files to bring the documents into compliance.

    With the publication of the ISO~10303:1994 standard, the opportunity 
was taken
to provide a new baseline release of the package files.
The new baseline release was also designed to address the fact that
a major update of \latex{} to \latex 2e took place during 1994. \latex 2e
is now the officially supported version. However, some users needed time
to convert to using \latex 2e. Consequently, although the 
facilities described in the original version of this document 
were common for both 
\latex{} v2.09 and \latex 2e users, there were two different 
implementations.

    Because ISO standard documents have a very structured layout, the class
and package files have been designed to reflect the logical document structure
to a much greater extent than the `standard' \latex{} files. Further, ISO
documents are published in more than one language. The files described
here are written for the English language, but the language-specific elements
have been parameterized for easy modification for publication in other
official ISO languages, such as French.

    In 1997 ISO produced a new version of their Directives on the 
requirements for the layout of ISO documents. These were not completely
unambiguous as to their intent; the current version was released in
2001. Members of ISO TC184/SC4 have worked with
the ISO Editorial Board and have reached an agreement that more precisely
identifies the requirements. The current version of the \latex{} files 
implements that agreement.

\begin{note}
The original of this manual has been typeset using the \verb?draft?
option in order to display its effect of placing a black box at the 
end of each line that is longer than the normal width of the text.
\end{note}

\begin{note}
The author of this document may be contacted at 
\texttt{peter.r.wilson@boeing.com}.
\end{note}

\end{introduction}

\title{LaTeX for ISO standards: User manual}%
{Peter Wilson}%
{August 2002}


\scopeclause

This manual describes a set of \ixltx\latex{} files for typesetting 
documents according to the ISO Directives Part 3 (third edition, 1997), 
together with some elements from the ISO~10303 Supplementary Directives.

\begin{inscope}{manual}
\item use of \latex{} for preparing ISO standard documents.
\end{inscope}

\begin{outofscope}{manual}
\item use of \latex{} in general;
\item use of other document preparation systems.
\end{outofscope}

\textbf{IMPORTANT:} No matter whether or not there are copyright attributions
to ISO, this document is \emph{not} copyrighted by ISO. Any copyright
markings are for illustrative purposes only. This document is released under
the LaTeX Project Public Licence.

\normrefsclause \label{sec:nrefs}

\normrefbp{manual}
\begin{nreferences}

\isref{ISO/IEC Directives, Part 2}{Rules for the structure and drafting of
       International Standards. (Fourth edition, 2001)}

\isref{ISO TC 184/SC4 N1217:2001(E)}{SC4 Supplementary directives --- Rules
       for the structure and drafting of SC4
       standards for industrial data. (2001--11--01).}

\isref{ISO/IEC 8824-1:1998}{Information technology ---
       Abstract Syntax Notation One (ASN.1):
       Specification of basic notation.}

\disref{ISO}{Camera-ready copy (CRC) ---
       Submission requirements and ISO type specification.
       (Version 1.0, 1996-04-26)}

\end{nreferences}

\defabbclause
%\clause{Terms, definitions, and abbreviations}

\defsubclause
%\sclause{Terms and definitions}

    For the purposes of this manual, the following definitions
apply.

\begin{definitions}
\definition{boilerplate}{text whose wording is fixed and has been agreed
 to be present in a specific type of document} \index{boilerplate}
\definition{style file}{a set of \latex{} macros assembled into a single 
   file with an extension \file{.sty}}
            \index{style file}\ixltx\index{file!.sty}
\definition{package file}{a style file for use with \latex 2e}\ixltxe
\end{definitions}

\abbsubclause
%\sclause{Abbreviations}

    For the purposes of this manual, the following abbreviations
 apply.

\begin{symbols}
\symboldef{CD}{Committee Draft}\index{CD}
\symboldef{DIS}{Draft International Standard}\index{DIS}
\symboldef{FDIS}{Final Draft International Standard}\index{DIS}
\symboldef{IS}{International Standard}\index{IS}
\symboldef{IS-REVIEW}{The documentation style accepted by the ISO 
    Editorial Board review (September 1994) of twelve IS documents 
    (the initial release of ISO~10303) for compliance with ISO 
    typographical and layout requirements.}\index{IS-REVIEW}
\symboldef{ISOD}{ISO/IEC Directives, Part 2}\index{ISOD}\index{ISO/IEC Directives}
\symboldef{PAS}{Publicly Available Specification}\index{PAS}
\symboldef{SD}{SC4 Supplementary directives}\index{SD}\index{Supplementary directives}
\symboldef{TR}{Technical Report}\index{TR}
\symboldef{TS}{Technical Specification}\index{TS}
\symboldef{WD}{Working Draft}\index{WD}
\symboldef{CRC}{The ISO \emph{Camera-ready copy (CRC)} document}\index{CRC}
\symboldef{ToC}{table of contents}\index{ToC}
\end{symbols}

\clause{Conformance requirements}  \label{sec:iconform}

    The \latex{} macro source files shall not be modified.

    If there is a need to modify the macro definitions then the
modifications shall be defined in a separate \file{.sty}\index{file!.sty}
 file (or files), using the
\latex{} \verb|\renewcommand|\ixcom{renewcommand}
and/or the 
\verb|\renewenvironment|\ixcom{renewenvironment}
commands as appropriate. The resulting \file{.sty} file(s) shall then
be called in within the preamble\index{preamble} portion of the 
document to be typeset.

    Author specified \verb|\label{...}| commands shall not start with
the characters \verb|;i| (semicolon and `i'); definition of labels
starting with these characters is reserved for the maintainer of the
facility files.


\fcandaclause
%\clause{Fundamental concepts and assumptions}

    It is assumed that the reader of this document is familiar with the
\latex{} document preparation system.\ixltx

\begin{note}Reference~\bref{lamport} in the bibliography describes the
      \latex{} system.
\end{note}

    The reader is also assumed to be familiar with the ISO/IEC Directives 
Part~2 (ISOD\index{ISOD}). Agreements reached between the ISO Editorial Board
and ISO TC184/SC4 are documented in the SC4 Supplementary Directives
(SD\index{SD}).

    If there are any discrepancies between the layout and wording of this 
document and the requirements of the ISO/IEC Directives Part~2, 
then the requirements in that document shall be
followed for any ISO standard document.

   The \file{isov2}\ixclass{isov2} class requires the 
\file{url}\ixpack{url} package.

\begin{note}
Additional \latex{} facilities specifically designed for ISO~10303 are
defined and described elsewhere~\bref{stepsty}.
\end{note}

    Because of many revisions over the years to the files described
herein, a naming convention has been adopted for them.
The primary name of the file is suffixed by \file{v\#} or \file{\#}, where
\file{\#} is the version number of the file in question. 
All file primary names have been limited to a 
maximum of eight characters.

\begin{note}
    Table~\ref{tab:curfiles} shows the versions of the files that were
current at the time of publication.
\end{note} % end note
\begin{note}
Starting with the October 1997 release, files that were specific to 
\ixltxv\latex{} v2.09 are no
longer either supported or supplied.
\end{note}

\begin{note}
As of 1999, the \file{uschyp}\ixpack{uschyp} package is no longer supported. 
It has been replaced by the 
\file{hyphenat}\ixpack{hyphenat}\index{hyphenat.sty@\file{hyphenat.sty}}
package.
\end{note}

\ixclass{isov2}\index{isov2.cls@\file{isov2.cls}}
%             \index{isonev11.sty@\file{isonev11.sty}}
\ixopt{9pt}\index{iso9.clo@\file{iso9.clo}}
%           \index{iso9.sty@\file{iso9.sty}}
\ixopt{10pt}\index{iso10.clo@\file{iso10.clo}}
%            \index{iso10.sty@\file{iso10.sty}}
\ixopt{11pt}\index{iso11.clo@\file{iso11.clo}}
%            \index{iso11.sty@\file{iso11.sty}}
\ixpack{isorot}\index{isorot.sty@\file{isorot.sty}}
%               \index{isrotne1.sty@\file{isrotne1.sty}}
\ixpack{askinc}\index{askincv1.sty@\file{askincv1.sty}}
\ixpack{xtab}\index{xtab.sty@\file{xtab.sty}}
\ixpack{uschyp}\index{uschyp.sty@\file{uschyp.sty}}
%               \index{uschypne.sty@\file{uschypne.sty}}

\begin{table}
\centering
\caption{Current file versions} \label{tab:curfiles}
\begin{tabular}{|l|l|} \hline
\textbf{Facility} & \textbf{File}          \\ \hline\hline
\file{iso}     & \file{isov2.cls}    \\
9pt option     & \file{iso9.clo}     \\
10pt option    & \file{iso10.clo}    \\
11pt option    & \file{iso11.clo}    \\
\file{isorot}  & \file{isorot.sty}   \\
\file{askinc}  & \file{askincv1.sty} \\
\file{xtab}    & \file{xtab.sty}   \\
\file{hyphenat}  & \file{hyphenat.sty}   \\ \hline
\end{tabular}
\end{table}


\begin{anexample} At the time of 
publication of this document, any references to \file{iso.cls} should be
read as actually referring to \file{isov2.cls}, and similarly for references
to other files.
\end{anexample} % end example


\begin{note}This document is not intended for publication as a standard, 
although it has been laid out in a 
similar, but not necessarily identical, manner.\end{note} % end note


\clause{The \file{iso} class facility} 

    The \latex{} \file{isov2}\ixclass{isov2} class file
is a general file for use in preparing ISO
standard documents using the \latex{} document preparation system.

    As usual, any \latex{} document has the following structure:
\begin{verbatim}
\documentclass[<list of options>]{isov2}
% preamble goes here
\begin{document}
% document body goes here
\end{document}
\end{verbatim}

\sclause{Options}

    The \file{isov2}\ixclass{isov2} class file supports the following options: 
\begin{itemize}
\item \verb|draft|\ixopt{draft} for a draft document where 
      overfull horizontal boxes are marked, marginal notes are allowed,
      and ISO copyright text is not placed in the document;
\item \verb|final|\ixopt{final} the opposite of \verb|draft|
      (this is the default);
\item \verb|letterpaper|\ixopt{letterpaper} for printing 
      on US letter size paper;
\item \verb|a4paper|\ixopt{a4paper} for printing on A4
      size paper (this is the default);
\item \verb|twocolumn|\ixopt{twocolumn} for two column 
      formatting;
\item \verb|onecolumn|\ixopt{onecolumn} for single column
      printing (this is the default);
\item One of \verb|11pt|, \ixopt{11pt}
      \verb|10pt|, \ixopt{10pt}
      \verb|9pt|\ixopt{9pt} for 11pt, 10pt or 9pt printing
      respectively (the default is \verb|11pt|);
\item \verb|notcopyright|\ixopt{notcopyright} for 
      disabling the printing of copyright notices (this is the default);
\item \verb|copyright|\ixopt{copyright} enables printing
      of copyright notices;
\item \verb|is|\ixopt{is} for International Standard documents;
\item \verb|fdis|\ixopt{fdis} for  Final Draft
      International Standard documents; \index{FDIS}
\item \verb|dis|\ixopt{dis} for  Draft
      International Standard documents; \index{DIS}
\item \verb|cd|\ixopt{cd} for  Committee Draft
      standard documents; \index{CD}
\item \verb|wd|\ixopt{wd} for  Working Draft
      standard documents; \index{WD}
\item \verb|pas|\ixopt{pas} for Publicly Available Specification
      standard documents;
\item \verb|techrep|\ixopt{techrep} for Technical
      Report standard documents; \index{TR}
\item \verb|techspec|\ixopt{techspec} for Technical
      Spefication standard documents; \index{TS}
\item \verb|otherdoc|\ixopt{otherdoc} for documents
      that are not intended to become a standard (this is the default);
%\item \verb|uglycaption|\ixopt{uglycaption} to produce
%      an ugly style of captioning;
\item Any other facilities that are available via packages.
\end{itemize}

    When no options are specified, then the result is 11pt, single column
printing on A4 size paper, without copyright notice and a running header.
That is, by default, the options set are: 
\verb|final|\ixopt{final};
\verb|a4paper|\ixopt{a4paper}; 
\verb|onecolumn|\ixopt{onecolumn};
\verb|11pt|\ixopt{11pt};
\verb|notcopyright|\ixopt{notcopyright}; and
\verb|otherdoc|\ixopt{otherdoc}.


%\begin{note}ISOD\index{ISOD} calls for 9pt double column  printing but the 
%      SD\index{SD} calls for 11pt single column printing.
%      Using 9pt double column printing is awkward if any computer 
%      code has to
%      be typeset in the document. The IS-REVIEW\index{IS-REVIEW} accepted
%      11pt single column layout. \end{note} % end note
%\begin{note}The CRC\index{CRC} states that acceptable founts are: Univers,
%    Helvetica, and Times Roman with the body text in 10pt. 
%    The IS-REVIEW accepted camera-ready copy
%    using Computer Modern 11pt set in single column. \end{note}
\begin{note}The user of the \file{isov2}\ixclass{isov2} class is encouraged 
      to process this document using
      different combinations of the options to gain experience with
      their effects. This printing of the document is typeset using the
      \verb|draft|\ixopt{draft} option.
\end{note} % end note
\begin{note}
    The \verb|otherdoc|\ixopt{otherdoc} option 
was chosen as the default so that the
current stage of ISO standardardization has to be explicitly declared
as an option, and is therefore available to any software package that
might process the document source (e.g., a \latex{} to SGML translator).
\end{note}


\sclause{Sectioning commands}

    Table~\ref{tab:sec} gives the sectioning commands defined for \file{isov2}
class documents.
\ixcom{clause} \ixcom{normannex} \ixcom{infannex} \ixcom{repannex} 
\ixcom{sclause}
\ixcom{ssclause}
\ixcom{sssclause}
\ixcom{ssssclause}
\ixcom{sssssclause}

\begin{table}
\centering
\caption{Sectioning commands.} \label{tab:sec}
\begin{tabular}{|c|l|} \hline
\textbf{Level} & \textbf{Command} \\ \hline\hline
1   & \verb|\clause|, \verb|\normannex|, \verb|\infannex|, \verb|\repannex| \\
2           & \verb|\sclause| \\
3           & \verb|\ssclause| \\
4           & \verb|\sssclause| \\
5           & \verb|\ssssclause| \\
6           & \verb|\sssssclause| \\ \hline
\end{tabular}
\end{table}

%    The \verb|\section|\ixcom{section} command
% is similar to the normal \latex{} \verb|\part|\ixcom{part}
%command. It is only available when the \verb|sect|\ixopt{sect}
%option is specified.

    The \verb|\clause|\ixcom{clause} commands are similar to 
the normal \latex{} \verb|\section|ing 
commands. There are also starred versions of these commands 
(e.g., \verb|\clause*|).

    Three varieties of annex\index{annex} commands are available:
\begin{itemize}
\item \verb+\normannex{+\meta{title}\verb+}+\ixcom{normannex}
      for a normative annex;
\item \verb+\infannex{+\meta{title}\verb+}+\ixcom{infannex}
       for an informative annex;
\item \verb+\repannex{+\meta{title}\verb+}+\ixcom{repannex}
      for an annex that is neither normative or
     informative (e.g. an annex in a technical report).
\end{itemize}

    Only \verb|\sclause|\ixcom{sclause} and lower level 
sectioning commands can be used after
an annex\index{annex} command.


\begin{example}
    The command  \verb|\infannex{Technical discussion}|, assuming that this is the first
annex in the document, produces:
    \begin{center}
    \textbf{Annex A}\\
    (informative)

    \textbf{Technical discussion}
    \end{center}
\end{example} % end example
\begin{example}
  The command \verb|\repannex{Title of annex}|, assuming that this is the second
annex in the document, produces:

\begin{center}
  \textbf{Annex B}

  \textbf{Title of annex}
\end{center}
\end{example} % end example


\ssclause{The use of \texttt{tocdepth} and \texttt{secnumdepth}}

    In the standard \latex{} classes the
\verb|tocdepth|\ixtt{tocdepth} and 
\verb|secnumdepth|\ixtt{secnumdepth} counters
are set in the preamble\index{preamble} to respectively control the level at
which clause titles are inserted into a table of contents (ToC)\index{ToC} 
and at which clause numbering ceases.

    In the \file{isov2}\ixclass{isov2} class, 
the values of these can be changed at 
any point in the document. The change lasts until another change is
made to the value. 
\begin{example}
    Assume that in the preamble we have
\begin{verbatim}
\setcounter{secnumdepth}{3}  % number ssclauses and above
\setcounter{tocdepth}{3}     % ToC includes ssclauses and above
\end{verbatim}
and that a certain subclause has subsubclauses that should be numbered but
not put into the table of contents, then we could do:
\begin{verbatim}
...
\sclause{The certain subclause}
\setcounter{tocdepth}{2}
...
\ssclause{Numbered but not in ToC}
...
...
\setcounter{tocdepth}{3}
\sclause{Following subclause}
\end{verbatim}

\end{example}

    It can sometimes be difficult to remember which level number corresponds
to which kind of clause. Accordingly, a set of commands are provided to ease
this task. These commands can only be used after the preamble.
\begin{itemize}
\item \verb|\maxsecnumdepth{|\meta{sec}\verb|}|\ixcom{maxsecnumdepth}
      sets the level at which clauses will be numbered. This command
      should be used before the first sectioning command.
\item \verb|\maxtocdepth{|\meta{sec}\verb|}|\ixcom{maxtocdepth}
      sets the level at which clauses will be put into the ToC.
      This command should be used before the \verb|\tableofcontents| command.
\item \verb|\setsecnumdepth{|\meta{sec}\verb|}|\ixcom{setsecnumdepth}
      sets the current level at which clauses will be numbered. 
      This command can be used anywhere after the preamble.
\item \verb|\settocdepth{|\meta{sec}\verb|}|\ixcom{settocdepth}
      sets the current level at which clauses will put into the ToC. 
      This command can be used anywhere after the preamble.
\end{itemize}
The value of the argument \meta{sec} can be any of the following:
\texttt{clause}, \texttt{sclause}, \ldots, \texttt{sssssclause}.

\begin{example}
Using these commands, the previous example can also be coded as:
\begin{verbatim}
...
\begin{document}
\maxsecnumdepth{ssclause}
\maxtocdepth{ssclause}
...
\sclause{The certain subclause}
\settocdepth{sclause}
...
\ssclause{Numbered but not in ToC}
...
...
\settocdepth{ssclause}
\sclause{Following subclause}
\end{verbatim}

\end{example}



\sclause{LaTeX environments and commands}

    Many of the standard \latex{} environments and commands are available.
In particular, all the normal mathematical typesetting capabilities are
present.
However, there are some additional environments and commands defined.

\ssclause{Lists}

    The standard \latex{} \verb|itemize|,\ixenv{itemize}
\verb|enumerate|\ixenv{enumerate} and \verb|description|\ixenv{description}
environments are provided. The labels in these lists, though, differ from
those normally provided by \latex.

\begin{note}
The ISOD describes only a single level for an itemized list, being marked 
with either an em-dash or a bullet. The SD deprecates the bullet but 
provides four levels, each being marked with an em-dash. These are 
provided in the \file{isov2} class.
\end{note}
\begin{note}
The ISOD allows for two levels of enumerated lists. The SD extends this
to four levels, and these are provided in the \file{isov2} class.
\end{note}

\begin{example}
The list environments provided are shown below as:\ixenv{itemize}
\begin{verbatim}
\begin{itemize}
\item First level itemized element;
  \begin{itemize}
  \item Second level itemized element;
    \begin{itemize}
    \item Third level itemized element;
      \begin{itemize}
      \item Fourth level itemized element.
      \end{itemize}
    \end{itemize}
  \end{itemize}
\end{itemize}
\end{verbatim}

\ixenv{enumerate}
\begin{verbatim}
\begin{enumerate}
\item First level enumerated element;
  \begin{enumerate}
  \item Second level enumerated element;
    \begin{enumerate}
    \item Third level enumerated element;
      \begin{enumerate}
      \item Fourth level enumerated element.
      \end{enumerate}
    \end{enumerate}
  \end{enumerate}
\end{enumerate}
\end{verbatim}

\ixenv{description}
\begin{verbatim}
\begin{description}
\item[Description] a description element. Note that a colon is
                   automatically added to the item label.
\end{description}
\end{verbatim}
and they produce
\begin{itemize}
\item First level itemized element;
  \begin{itemize}
  \item Second level itemized element;
    \begin{itemize}
    \item Third level itemized element;
      \begin{itemize}
      \item Fourth level itemized element.
      \end{itemize}
    \end{itemize}
  \end{itemize}
\end{itemize}

\begin{enumerate}
\item First level enumerated element;
  \begin{enumerate}
  \item Second level enumerated element;
    \begin{enumerate}
    \item Third level enumerated element;
      \begin{enumerate}
      \item Fourth level enumerated element.
      \end{enumerate}
    \end{enumerate}
  \end{enumerate}
\end{enumerate}

\begin{description}
\item[Description] a description element. Note that a colon is
                   automatically added to the item label.
\end{description}

\end{example} % end example

\begin{example}
    Here is a more complex set of lists:

\begin{itemize}
\item First level itemization
  \begin{enumerate}
  \item First level enumeration
    \begin{itemize}
    \item Second level itemization
      \begin{enumerate}
      \item Second level enumeration
        \begin{itemize}
        \item Third level itemization
%          \begin{enumerate}    % At this point the list is too deeply nested.
%          \item Third level enumeration    % For the TeXnophile this is
%          \item 3rd level enumeration      % because the example environment
%         \end{enumerate}                   % is itself a list.
        \item 3rd level itemization
        \end{itemize}
      \item 2nd level enumeration
      \end{enumerate}
    \item 2nd level itemization
    \end{itemize}
  \item 1st level enumeration
  \end{enumerate}
\item 1st level itemization
\end{itemize}

\end{example} % end example

\begin{note}
On occasion, \latex{} objects to a well formed list. Typically, this happens
when a list has only one entry. \index{list!error}
 The \latex{} error message is of the form:
\begin{verbatim}
! Something's wrong--perhaps a missing \item.
\end{verbatim}
Hitting the \verb|<return>| key usually gets \latex{} to run on happily.
\end{note}% end note
\begin{note}
Similarly, you may get the message
\begin{verbatim}
! Too deeply nested
\end{verbatim}
Again, hit \verb|<return>| and processing should continue. However,
the indentation of later lists may not be correct.
\end{note} % end note


\ssclause{Notes}

    
    Two different kinds of environments are provided for typesetting notes.

\sssclause{Numbered notes}

    The environment 
\verb+\begin{note}+ \meta{text} \verb+\end{note}+\ixenv{note}
produces a numbered note whose body consists
of \meta{text}.

\begin{anexample}
The commands
\begin{verbatim}
\begin{note}Numbered note.\end{note}
\end{verbatim}
produce

\begin{note}Numbered note.\end{note}
\end{anexample}

\sssclause{Isolated notes}

    The environment 
\verb+\begin{anote}+ \meta{text} \verb+\end{anote}+\ixenv{anote}
produces an un-numbered note whose body
consists of \meta{text}.

\begin{anexample}
    The commands
\begin{verbatim}
\begin{anote}Plain note.\end{anote}
\end{verbatim}
produce

\begin{anote}Plain note.\end{anote}
\end{anexample} % end example


\ssclause{Examples} \label{ssc:ex}

    Two different kinds of environments are provided for typesetting 
examples.

\sssclause{Numbered examples} \label{cl:numex}

The environment \verb?\begin{example}? \meta{text} \verb?\end{example}?
produces a numbered example whose body consists of \meta{text}.

\begin{anexample}
The commands
\begin{verbatim}
\begin{example}Numbered example. \label{ex:exref} \end{example}
\end{verbatim}
produce

\begin{example}Numbered example. \label{ex:exref} \end{example}
\end{anexample}

\begin{anote}
Numbered examples (and numbered notes) may be identified using the
\verb?\label{...}?\ixcom{label} command, as exhibited in \eref{ex:exref}
in \ref{cl:numex}, and then referred to by the 
\verb?\ref{...}?\ixcom{ref} command.
\end{anote}

\sssclause{Isolated examples}

The environment \verb?\begin{anexample}? \meta{text} \verb?\end{anexample}?
produces an un-numbered example whose body consists of \meta{text}.
\ixenv{anexample}

\begin{anexample}
The commands
\begin{verbatim}
\begin{anexample}Isolated example.\end{anexample}
\end{verbatim}
produce

\begin{anexample}Isolated example.\end{anexample}
\end{anexample}



\ssclause{Bibliographic references}

    Two different kinds of environments are provided for bibliographic 
references. It should be noted that neither of these has anything to do with
BibTeX.\index{BibTeX}

\sssclause{Normative references}

    Normative references are listed in the 
\verb|nreferences|\ixenv{nreferences} environment.

    In this environment, bibliographic entries are denoted by either
\verb+\isref{+\meta{ref}\verb+}{+\meta{title}\verb+}+ or by 
\verb+\disref{+\meta{ref}\verb+}{+\meta{title}\verb+}+. The
\meta{ref} parameter is the number of the standard document and 
the \meta{title} parameter is the title of the standard. 

    Use the \verb|\isref|\ixcom{isref} command for published standards 
and the \verb|\disref|\ixcom{disref}
 command for documents that have not yet been 
finally approved as a standard. This latter command automatically adds a
footnote to the effect  that the document is
to be published.

\begin{example}
The \latex{} source for the references in clause~\ref{sec:nrefs} of 
this document is:
\begin{verbatim}
\begin{nreferences}
\isref{ISO/IEC Directives, Part 2}{Rules for the structure and drafting of
       International Standards. (Fourth edition, 2001)}
...
...
\disref{ISO}{Camera-ready copy (CRC) ---
       Submission requirements and ISO type specification.
       (Version 1.0, 1996-04-26)}
\end{nreferences}
\end{verbatim}
\end{example} % end example


\sssclause{Informative references}

    Informative bibliography elements are listed in the
\verb|references|\ixenv{references} environment. 
Each element in the list is specified as 
\verb+\reference{+\meta{author}\verb+}{+\meta{title}\verb+}{+\meta{publisher}\verb+}+.\ixcom{reference}

\begin{example}
The \latex{} source for the bibliography in one version of this document was:
\begin{verbatim}
\infannex{Bibliography}
\begin{references}
\reference{LAMPORT, L.,}{\latex\/ A Document Preparation System,}%
           {Addison-Wesley Publishing Co., 1986} \label{lamport}
\reference{GOOSSENS, M., MITTELBACH, F. and SAMARIN, A.,}{%
           The \latex\/ Companion,}
           Addison-Wesley Publishing Co., 1994} \label{goosens}
\reference{CHEN, P. and HARRISON, M.A.,}{Index preparation and
           processing,}{Software--Practice and Experience, 19(9):897--915,
           September 1988.} \label{chen}
....
\end{references}

\end{verbatim}
\end{example} % end example

    Informative references may be cited in the text via the \latex{}
\verb|\label|\ixcom{label} and \verb|\ref|\ixcom{ref}
mechanism. Note that \verb|\cite|\ixcom{cite} is not available
for references to bibliographic items. For the purposes of
ISO documents, the command 
\verb+\bref{+\meta{ref}\verb+}+\ixcom{bref}
is supplied which
properly formats a bibliographic reference.

\ssclause{Listing of scope items}

    The \verb|inscope|\ixenv{inscope} and 
\verb|outofscope|\ixenv{outofscope}
environments are provided for itemized listing of elements that
are within and outside the scope of the standard. Each list
element is introduced via the \verb|\item|\ixcom{item} command. 
Some boilerplate
text is also printed that introduces the scope list.

    The environments take one parameter, \meta{text}, that must read
naturally in a sentence of the form: `The following are within/outside
the scope of this \meta{text}:'. The non-parameterized part of this
sentence is specified by the commands 
|\verb|\inscopename|\ixcom{inscopename} and
\verb|\outofscopename|\ixcom{outofscopename}
 respectively for `within' and `outside'.

\begin{example}The following text was printed by the commands shown at the
end of the example.

\begin{inscope}{part of ISO~10303}
\item use of \latex{} for preparing ISO standard documents;
\item use of \latex{} for preparing ISO~10303 documents.
\end{inscope}

\begin{outofscope}{part of ISO~10303}
\item use of \latex{} in general;
\item use of other document preparation systems.
\end{outofscope}

\begin{verbatim}
\begin{inscope}{part of ISO~10303}
\item use of \latex{} for preparing ISO standard documents;
\item use of \latex{} for preparing ISO~10303 documents.
\end{inscope}

\begin{outofscope}{part of ISO~10303}
\item use of \latex{} in general;
\item use of other document preparation systems.
\end{outofscope}
\end{verbatim}

\end{example} % end example

\ssclause{Listing of definitions}

    The \verb|\begin{olddefinitions}| \ldots \verb|\end{olddefinitions}|
\ixenv{olddefinitions} environment is provided for 
listing terms that have been defined within
the normatively referenced documents. Each term in the list is specified as: \\
\verb+\olddefinition{+\meta{phrase}\verb+}{+\meta{supplement}\verb+}+\ixcom{olddefinition}.


    The \verb|\begin{definitions}| \ldots \verb|\end{definitions}|
\ixenv{definitions} environment is provided for listing 
the definitions of terms specific to the
document being written. Each term in the
list is specified as: \ixcom{definition} \\
\verb+\definition{+\meta{phrase}\verb+}{+\meta{definition text}\verb+}+.


\begin{example}A listing of terms defined elsewhere could be specified as:

\begin{verbatim}
\begin{olddefinitions}
\olddefinition{application protocol (AP)}{}
\olddefinition{integrated resource}{}
\end{olddefinitions}
\end{verbatim}
\end{example} % end example
\begin{example}The definition listing earlier in this document was produced by:

\begin{verbatim}
\begin{definitions}
\definition{boilerplate}{text whose wording is fixed ...}
\definition{style file}{a set of \latex{} macros assembled 
                        into a single file}
...
\end{definitions}
\end{verbatim}
\end{example} % end example


\ssclause{Listing of abbreviations}

    The \verb|\begin{symbols}| \ldots \verb|\end{symbols}|\ixenv{symbols}
environment is provided for listing symbols
and abbreviations. Each term (either symbol or abbreviation) in the list is
specified as: \ixcom{symboldef} \\
 \verb+\symboldef{+\meta{symbol}\verb+}{+\meta{definition text}\verb+}+.

\begin{example}The list of abbreviations earlier in this document was specified as:
\begin{verbatim}
\begin{symbols}
\symboldef{DIS}{Draft International Standard}
\symboldef{IS}{International Standard}
...
\end{symbols}
\end{verbatim}
\end{example} % end example

\sclause{Floating bodies}

    \latex{} provides the \verb|figure|\ixenv{figure}
and \verb|table|\ixenv{table} environments.
Captions (produced by the \verb|\caption|\ixcom{caption}
command) increment the figure or
table number and add the caption to the relevant contents listing file.

\index{floats} \index{floats!continuation}

    The command 
\verb+\contcaption{+\meta{text}\verb+}+\ixcom{contcaption}
may be used instead. This command neither increments the number nor adds
anything to the listing files.

\begin{example}
The following code:
\begin{verbatim}
\begin{table}[tbp]
\centering
\caption{Example table in three parts} \label{tab:cont}
\begin{tabular}{|c|c|} \hline
\textbf{Col 1} & \textbf{Col 2} \\ \hline\hline
A & 1 \\
B & 2 \\ \hline
\end{tabular}
\end{table}

\begin{table}[tbp]
\centering
\contcaption{(continued)}
\begin{tabular}{|c|c|} \hline
\textbf{Col 1} & \textbf{Col 2} \\ \hline\hline
C & 3 \\
D & 4 \\ \hline
\end{tabular}
\end{table}

\begin{table}[tbp]
\centering
\contcaption{(concluded)}
\begin{tabular}{|c|c|} \hline
\textbf{Col 1} & \textbf{Col 2} \\ \hline\hline
E & 5 \\
F & 6 \\ \hline
\end{tabular}
\end{table}
\end{verbatim}
produces the three-part \tref{tab:cont}. \end{example} %end example

\begin{table}[tbp]
\centering
\caption{Example table in three parts} \label{tab:cont}
\begin{tabular}{|c|c|} \hline
\textbf{Col 1} & \textbf{Col 2} \\ \hline\hline
A & 1 \\
B & 2 \\ \hline
\end{tabular}
\end{table}

\begin{table}[tbp]
\centering
\contcaption{(continued)}
\begin{tabular}{|c|c|} \hline
\textbf{Col 1} & \textbf{Col 2} \\ \hline\hline
C & 3 \\
D & 4 \\ \hline
\end{tabular}
\end{table}

\begin{table}[tbp]
\centering
\contcaption{(concluded)}
\begin{tabular}{|c|c|} \hline
\textbf{Col 1} & \textbf{Col 2} \\ \hline\hline
E & 5 \\
F & 6 \\ \hline
\end{tabular}
\end{table}

\sclause{Title command}


    The command to produce the title of the standard\ixcom{title}
is 
\verb+\title{+\meta{intro}\verb+}{+\meta{main}\verb+}{+\meta{compl}\verb+}+. 
The command takes three parameters
according to the three elements of the title as specified in the ISO directives.
\begin{enumerate}
\item \meta{intro} The introductory element of the title. This may be empty.
\item \meta{main} The main element of the title.
\item \meta{compl} The complementary element of the title. This may be empty.
\end{enumerate}

\begin{example}
If this were actually meant to be an ISO standard, then the \latex{} source 
for the title of this document might be:
\begin{verbatim}
\title{Industrial automation systems and integration}%
{Product data representation and exchange}%
{Part 3456 : Documentation methods: The LaTeX style
file reference manual}
\end{verbatim}
\end{example} % end example

\begin{anote}There must be a space on either side of the colon separating the
part number and the final part of the title.
\end{anote}% end of note

    Three other commands are used for setting the running header throughout the
document. These shall be placed in the preamble.\index{preamble}

    The command \verb+\standard{+\meta{number of standard}\verb+}+ is used to identify
the standard.\ixcom{standard}

    The command \verb+\yearofedition{+\meta{year}\verb+}+ is used to identify the
\ixcom{yearofedition} year of the edition.

    The command 
\verb+\languageofedition{+\meta{code}\verb+}+\ixcom{languageofedition}
 is used to identify the language of the edition.

\begin{example}
This document is identified by:
\begin{verbatim}
\standard{LaTeX for standards}
\yearofedition{1997}
\languageofedition{(E)}
\end{verbatim}
\end{example} % end of example

    As a convenience the vacuous command \verb+\extrahead+\ixcom{extrahead}
is supplied which will add it's contents, if any, to the header. It may
be used, for example, to add a document number to the header by
\verb+\renewcommand{\extrahead}{Doc number}+. If \verb+\extrahead+ is
modified it must be done in the preamble.

    The \verb|\title|\ixcom{title} command sets the page numbering 
style to be arabic,
starts a new page, numbered 1, and puts the title
at the start of the page. It also puts the appropriate header at the top
of the page, dependending on the particular combination of selected
options.

\begin{anote}
Remember that you have to use the \verb|copyright|\ixopt{copyright}
option to enable printing of copyright symbols and notices.
\end{anote}

\ssclause{The cover}

    For publication, ISO want to be supplied with a document starting
on page iii with the ToC. They supply the cover (page i) and page ii.

It can often be useful to have a cover page for documents while they
are in the process of being prepared for submittal to ISO.
    The \verb+cover+\ixenv{cover} environment is provided for that
purpose. The contents of the environment must not exceed one page and the
contents must have a \verb+\clearpage+\ixcom{clearpage} immediately before
the end. The \verb+cover+ environment also prints page ii, which has a 
copyright notice on it if the document is copyrighted.

    For example, this document starts with:
\begin{verbatim}
\begin{cover}
\vspace*{4in}
\begin{center}
\Huge\bfseries LaTeX for standards
\end{center}
\begin{center}
\bfseries 2001/07/06
\end{center}
\begin{center}
Peter Wilson \\
\texttt{peter.r.wilson@boeing.com}
\end{center}
\clearpage
\end{cover}
\end{verbatim}

\sclause{Cross referencing commands}

    The usual \latex{} \verb+\label+\ixcom{label} and
\verb+\ref+\ixcom{ref} commands are supported. The class also
provides some formatted referencing commands in addition to \verb+\ref+.

    The following commands are useful for referring to figures, clauses etc.
Each takes a parameter that is the parameter of a \latex{} \verb|\label{}|
command.

\begin{itemize}
\item The command \verb|`\aref{anx:lord}'|\ixcom{aref} prints `\aref{anx:lord}' 
      while \verb|`\ref{anx:lord}'|\ixcom{ref}
      prints `\ref{anx:lord}'
\item The command \verb|`\bref{lamport}'|\ixcom{bref} prints `\bref{lamport}'
      while \verb|`\ref{lamport}'|\ixcom{ref}
      prints `\ref{lamport}'
\item The command \verb|`\cref{sec:nrefs}'|\ixcom{cref} prints `\cref{sec:nrefs}'
      while \verb|`\ref{sec:nrefs}'|\ixcom{ref}
      prints `\ref{sec:nrefs}'
\item The command \verb|`\eref{sec:nrefs}'|\ixcom{eref} prints `\eref{sec:nrefs}'
      while \verb|`\ref{sec:nrefs}'|\ixcom{ref}
      prints `\ref{sec:nrefs}'
\item The command \verb|`\fref{sec:nrefs}'|\ixcom{fref} prints `\fref{sec:nrefs}'
      while \verb|`\ref{sec:nrefs}'|\ixcom{ref}
      prints `\ref{sec:nrefs}'
\item The command \verb|`\nref{sec:nrefs}'|\ixcom{nref} prints `\nref{sec:nrefs}'
      while \verb|`\ref{sec:nrefs}'|\ixcom{ref}
      prints `\ref{sec:nrefs}'
\item The command \verb|`\tref{sec:nrefs}'|\ixcom{tref} prints `\tref{sec:nrefs}'
      while \verb|`\ref{sec:nrefs}'|\ixcom{ref}
      prints `\ref{sec:nrefs}'
\item The command \verb|`\pref{sec:nrefs}'|\ixcom{pref} prints
      `\pref{sec:nrefs}'.
\end{itemize}




\sclause{Heading commands}

\ssclause{Foreword}

    The \verb|foreword|\ixenv{foreword} environment specifies
that a table of contents, list of
figures and list of tables be produced, and starts a new unnumbered
clause called Foreword.
Formatting is one column style only and roman page numbering is set. The 
table of contents starts on page iii.

    More precisely, the title of the clause is given by the value of
\verb|\forewordname|\ixcom{forewordname} (see \aref{anx:extraiso}).

\begin{note}
The default style only lists level 1 clauses in the table of contents. If you
need a more detailed listing, then put the command 
\verb|\setcounter{tocdepth}{<level #>}|\ixcom{setcounter}\ixtt{tocdepth}
in the preamble.\index{preamble}
For example \verb|\setcounter{tocdepth}{3}| will produce a contents
listing down to the level of \verb|\ssclause| (see \tref{tab:sec} for
the definitions of clause levels).
\end{note} %end note

    The command \verb+\fwdbp+\ixcom{fwdbp} prints the 
ISO specified boilerplate for
the initial paragraphs of a foreword.

    The boilerplate for a Technical Specification (ISO/TS) or a 
Publicly Available Specification (ISO/PAS) differs from that
for a standard. The command \verb|\tspasfwdbp|\ixcom{tspasfwdbp}
prints some of this boilerplate.
\begin{example}
In this document, \verb|\tspasfwdbp| prints:

\tspasfwdbp
\end{example}
\begin{note}
    The required paragraph immediately following this boilerplate is:

ISO/PAS [or ISO/TS] \ldots was prepared by Technical Committee ISO/TC \ldots
\end{note}

%%    The boilerplate for a Technical Report (TR) also differs from that for
%%a standard.
%%The command \verb|\trfwdbpi|\ixcom{trfwdbpi}
%%prints some of this boilerplate. 
%%\begin{example}
%%In this document, \verb|\trfwdbpi| prints:
%%
%%\trfwdbpi
%%\end{example}

    The \verb|\fwdnopatents|\ixcom{fwdnopatents} 
prints out the Foreword
boilerplate paragraph concerning potential patent rights.
\begin{example}
In this document, \verb|\fwdnopatents| prints:
\fwdnopatents 
\end{example}



\ssclause{Introduction}

    The \verb|introduction|\ixenv{introduction} environment 
starts a new unnumbered clause 
called `Introduction' with one column formatting.

    More precisely, the title of the clause is given by the value of
\verb|\introductionname|\ixcom{introductionname} (see \aref{anx:extraiso}).

    The \verb|\intropatents|\ixcom{intropatents} prints some of the 
patent boilerplate that may be required in the Introduction. 
\begin{anexample}
In this document, \verb|\intropatents| prints:
\fwdnopatents 
\end{anexample}


\ssclause{Scope clause}

    The \verb|\scopeclause|\ixcom{scopeclause} 
command starts a new numbered clause called `Scope', which is given
the label \verb|;i1|\index{;i1}.

    More precisely, the title of the clause is given by the value of
\verb|\scopename|\ixcom{scopename} (see \aref{anx:extraiso}).

\ssclause{Normative references clause}

    The \verb|\normrefsclause|\ixcom{normrefsclause}
command starts a new numbered clause called `Normative references',
which is given the label \verb|;i2|\index{;i2}.

    More precisely, the title of the clause is given by the value of
\verb|\normrefsname|\ixcom{normrefsname} (see \aref{anx:extraiso}).

    The command 
\verb+\normrefbp{+\meta{document identifier}\verb+}+\ixcom{normrefbp} 
prints the
ISO required text for the introduction to the listing of normative
references. The \meta{document identifier} parameter must be such
that it reads sensibly in a sentence of the form `\ldots constitute
provisions of this \meta{document identifier}.'.

\begin{example}Clause~\ref{sec:nrefs} in this document was started by the
commands:
\begin{verbatim}
\normrefsclause \label{sec:nrefs}

\normrefbp{manual}
\begin{nreferences}
...
\end{verbatim}
\end{example} % end example

\ssclause{Definitions, symbols and abbreviations}

    A variety of commands are provided that initiate new numbered
clauses for definitions, symbols and abbreviations. Depending on the
amount of material in these respective categories, one or more clauses
may be used. The commands and clause titles are listed in \tref{tab:dsa}.
The clause level headings are each given the label \verb|;i3|\index{;i3};
one and only one of these headings should appear in a standard document.
\ixcom{defclause} \ixcom{defname}
\ixcom{symclause} \ixcom{symname}
\ixcom{abbclause} \ixcom{abbname}
\ixcom{defsymclause} \ixcom{defsymname}
\ixcom{defabbclause} \ixcom{defabbname}
\ixcom{symabbclause} \ixcom{symabbname}
\ixcom{defsymabbclause} \ixcom{defsymabbname}
\ixcom{defsubclause}
\ixcom{symsubclause}
\ixcom{abbsubclause}
\ixcom{defsymsubclause}
\ixcom{defabbsubclause}
\ixcom{symabbsubclause}

\begin{table}
\centering
\caption{Definition, symbol and abbreviation clause commands}
\label{tab:dsa}
\begin{tabular}{|l|c|l|l|} \hline
\textbf{Command} & \textbf{Clause} & \textbf{Title} & \textbf{Default text} \\ \hline
\verb|\defclause| &       C  & \verb|\defname| & \defname{} \\
\verb|\symclause| &       C  & \verb|\symname| & \symname{}  \\
\verb|\abbclause| &       C  & \verb|\abbname| & \abbname{}  \\
\verb|\defsymclause| &    C  & \verb|\defsymname| & \defsymname{} \\
\verb|\defabbclause| &    C  & \verb|\defabbname| & \defabbname{} \\
\verb|\symabbclause| &    C  & \verb|\symabbname| & \symabbname{} \\
\verb|\defsymabbclause| & C  & \verb|\defsymabbname| & \defsymabbname{} \\
\verb|\defsubclause| &    SC & \verb|\defname| & \defname{} \\
\verb|\symsubclause| &    SC & \verb|\symname| & \symname{}  \\
\verb|\abbsubclause| &    SC & \verb|\abbname| & \abbname{}  \\
\verb|\defsymsubclause| & SC & \verb|\defsymname| & \defsymname{} \\
\verb|\defabbsubclause| & SC & \verb|\defabbname| & \defabbname{} \\
\verb|\symabbsubclause| & SC & \verb|\symabbname| & \symabbname{} \\
 \hline
\multicolumn{4}{|l|}{%
NOTE - In the table, C = clause, SC = subclause. } \\ \hline
\end{tabular}
\end{table}

\sssclause{Heading labels}

    Some of the defined clauses have associated \verb|\label|s. These
heading commands and their \verb|\label| values are listed in 
\tref{tab:clabels}.

\begin{table}
\centering
\caption{Defined clause headings with labels}
\label{tab:clabels}
\begin{tabular}{|l|c|} \hline
\textbf{Command}       &  \textbf{Label} \\ \hline
\verb|\scopeclause|    &  \texttt{;i1} \\
\verb|\normrefsclause| &  \texttt{;i2} \\
\verb|\defclause| &       \texttt{;i3} \\
\verb|\symclause| &       \texttt{;i3} \\
\verb|\abbclause| &       \texttt{;i3} \\
\verb|\defsymclause| &    \texttt{;i3} \\
\verb|\defabbclause| &    \texttt{;i3} \\
\verb|\symabbclause| &    \texttt{;i3} \\
\verb|\defsymabbclause| & \texttt{;i3} \\
 \hline
\end{tabular}
\end{table}



\ssclause{Bibliography}

    The command \verb|\bibannex|\ixcom{bibannex}
 starts an informative section of the document entitled `Bibliography'. 
Or, more precisely, 
by the value of the \verb|\bibname|\ixcom{bibname}
command.

\sclause{Urls, etc}

    The command \verb|\url{|\meta{text}\verb|}|\ixcom{url} can be used for
typesetting \meta{text} as an email address.

    The command \verb|\isourl{|\meta{text}\verb|}|\ixcom{isourl} can be 
used for
typesetting \meta{text} as an URL address.

\begin{example}
The following code
\begin{verbatim}
The Email address is \url{joe@fred.mik} or the homepage is at
\isourl{http://fred.mik/home/}.
\end{verbatim}
will be typeset as: \\
The Email address is \url{joe@fred.mik} or the homepage is at
\isourl{http://fred.mik/home/}.
\end{example}


\sclause{Version control commands}


    A set of commands are provided to assist when revising a document.
For these commands to flag the changes in the printed document the command
\verb|\changemarkstrue|\ixcom{changemarkstrue}
must be put in the preamble.\index{preamble}

    In the commands described below, the \meta{number} parameter can be used 
to correlate changes in a document
to some external (numbered) requirement for the change.

\ssclause{Editorial}

    The command \verb+\editorial{+\meta{number}\verb+}+\ixcom{editorial}
flags an editorial change to the
document text with ED$^{number}$.

\begin{anexample}

   Just to start things off, this is an original sentence, which should
take up about a line.
This example sentence contains an editorial \editorial{37}
change. The command \verb|\editorial{37}| was included in the previous
sentence.
This sentence, though, has no changes marked in it and may be
assumed to be unaltered from a prior version.
\end{anexample} % end example

\ssclause{Added}

    The command 
\verb+\added{+\meta{text}\verb+}{+\meta{number}\verb+}+\ixcom{added}
flags the
position of the additional \meta{text} and highlights it.

\begin{anexample}

   Just to start things off, this is an original sentence, which should
take up about a line.
This example sentence contains \added{some added text}{27} in
the middle of it. The command \verb|\added{some added text}{27}| was
included in the previous sentence.
This sentence, though, has no changes marked in it and may be
assumed to be unaltered from a prior version.
\end{anexample} % end example

\ssclause{Deleted}

    The command \verb+\deleted{+\meta{number}\verb+}+\ixcom{deleted}
flags the position of deleted text.

\begin{anexample}

   Just to start things off, this is an original sentence, which should
take up about a line.
Some text was deleted \deleted{34} from the middle of this sentence.
The command \verb|\deleted{34}| was included in the previous sentence.
This sentence, though, has no changes marked in it and may be
assumed to be unaltered from a prior version.
\end{anexample} % end example

\ssclause{Moved}

    The command 
\verb+\moved{+\meta{text}\verb+}{+\meta{number}\verb+}+\ixcom{moved}
flags the position of
moved \meta{text} and highlights it.

\begin{anexample}

   Just to start things off, this is an original sentence, which should
take up about a line.
This sentence contains some \moved{moved}{81} text in the middle of it.
The command \verb|\moved{moved}{81}| was included in the previous
sentence.
This sentence, though, has no changes marked in it and may be
assumed to be unaltered from a prior version.
\end{anexample} % end example

\sclause{PDF}

    The class provides a command, \verb|\ifpdf|\ixcom{ifpdf}, 
to test whether or not the document is being processed by \latex{} 
or by pdf\latex. \latex{} processing produces a \file{.dvi} file which
needs further processing, usually by \file{dvips}, to generate PostScript
for printing. pdf\latex, however, directly generates a \file{.pdf} file
which can then be printed.

\begin{anexample}
A document for processing by either \latex{} or pdf\latex{} could be 
started like:
\begin{verbatim}
\documentclass{iso}
\usepackage{times}
\ifpdf
  \pdfoutput=1
  \usepackage[pdftex]{graphicx}
\else
  \usepackage{graphicx}
\fi
....
\end{verbatim}

The \file{times} package is used in this example as PDF printers tend
to handle
PostScript fonts better than other kinds of fonts.
\end{anexample}


    The class also supports the use of the 
\file{hyperref}\ixpack{hyperref} package in conjunction with pdf\latex.
Typically bookmark processing would be specifed by:
\begin{verbatim}
\ifpdf
  \pdfoutput=1
  \usepackage[plainpages=false,
              pdfpagelabels,
              bookmarksnumbered,
              hyperindex=true
             ]{hyperref}
  ...
\else
  ...
\fi
....
\end{verbatim}
If the \verb|hyperindex=true| option to the \file{hyperref} package is used
the the index has to be processed via the
\textsc{MakeIndex}\index{makeIndex@{\sc MakeIndex}} program.

%%%%%%%%%%%
%%%\end{document}
%%%%%%%%%%%

\clause{The \file{isorot} package facility} 

   The \file{isorot}\ixpack{isorot} facility enables the rotation 
of document elements on
a page. It uses the \latex{} \verb|\special|\ixcom{special} 
command to perform its
effects, and thus can only be used
with a limited number of dvi to postscript programs. The facilities
available are summarized in \tref{tab4}.

    \file{isorot} is a modification of the \file{rotation.sty} file
created by Rahtz and Barroca~\bref{rahtz}. Further examples of the usage
of their style are given in Goosens \emph{et al}~\bref{goosens}.

\begin{note}Several examples of the effects of the commands described herein
are shown. In many cases the results are not pretty. This should act as
a warning that using rotational elements requires more care than
most other document elements.\end{note} %end note

\begin{sidewaystable}
\ixcom{rotdriver} \ixcom{clockwise} \ixcom{counterclockwise}
\ixcom{figuresright} \ixcom{figuresleft} \ixcom{rotcaption}
\ixcom{controtcaption}
\ixenv{sideways} \ixenv{turn} \ixenv{rotate} \ixenv{sidewaystable}
\ixenv{sidewaysfigure} \ixenv{landscape}
\centering
\caption{The rotation facilities} \label{tab4}
\begin{tabular}{|l|l|} \hline
\textbf{Facility} & \textbf{Effect} \\ \hline
\multicolumn{2}{|c|}{\textbf{Commands}} \\ \hline
\verb|\rotdriver{<driver>}| & 
declare the name of the dvi to Postscript translator (default {\tt dvips}) \\
\verb|\clockwise| & 
sets rotation direction clockwise for positive angles (the default) \\
\verb|\counterclockwise| &
sets rotation direction counterclockwise for positive angles \\
\verb|\figuresright| &
sets rotation direction for sideways floats counterclockwise (the default) \\
\verb|\figuresleft| &
sets rotation direction for sideways floats clockwise \\
\verb|\rotcaption| &
like the \verb|caption| command, but rotates the caption through 90 degrees \\
\verb|\controtcaption| &
like the \verb|contcaption| command, but rotates the caption through 90 degrees \\ \hline
\multicolumn{2}{|c|}{\textbf{Environments}} \\ \hline
\verb|sideways| &
rotates the contents through 90 degrees counterclockwise \\
\verb|turn| &
rotates the contents through the given angle \\
\verb|rotate| &
rotates the contents through the given angle, but no space allowed for the result\\
\verb|sidewaystable| &
like the \verb|table| environment, but rotated 90 degrees \\
\verb|sidewaystable*| &
twocolumn version of \verb|sidewaystable| \\
\verb|sidewaysfigure| &
like the \verb|figure| environment, but rotated 90 degrees \\
\verb|sidewaysfigure*| &
twocolumn version of \verb|sidewaysfigure| \\
\verb|landscape| &
prints all enclosed pages in landscape mode \\ \hline
\end{tabular}
\end{sidewaystable}

\sclause{Options}

    The \file{isorot} facility has one option, 
namely \verb|debugshow|\ixopt{debugshow}. Calling this option produces
messages on the screen and in the \file{log} file regarding the actions 
being taken.

\begin{anote} This option is principally of interest to the maintainer
of the facility. \end{anote}

\sclause{DVI drivers}

    The \file{isorot} facility supports only a limited number of
dvi to postscript translators. The default translator is \emph{dvips}.
 The following command must be put in
the preamble of the document if \emph{dvips} is not being used: 
\verb|\rotdriver{<drivername>}|,\ixcom{rotdriver} where
\verb|<drivername>| is one of the following:\footnote{I have been able to 
try the {\tt dvips} driver
but not the others. If anyone has experience with the other drivers, or has
extended the range of drivers, I would like to be given the results.} %end footnote

\begin{enumerate}
\item \verb|dvipdf| for the \emph{dvipdf} 
      translator;\ixtt{dvipdf}
\item \verb|dvips| for Tom Rockicki's \emph{dvips} 
      translator;\ixtt{dvips}
\item \verb|dvipsone| for Y\&Y's \emph{dvipsone} 
      translator;\ixtt{dvipsone}
\item \verb|dvitops| for James Clark's \emph{dvitops} 
      translator;\ixtt{dvitops}
\item \verb|dviwindo| for Y\&Y's \emph{dviwindo} 
      translator;\ixtt{dviwindo}
\item \verb|pctex32| for Personal TeX's PC TeX for 32 bit Windows 
      (\emph{pctex32})
      translator;\ixtt{pctex32}
\item \verb|pctexps| for Personal TeX's PC PTI Laser/PS (\emph{pctexps})
      translator;\ixtt{pctexps}
\item \verb|pubps| for the Arbortext's \emph{pubps} 
      translator.\ixtt{pubps}
\item \verb|textures| for Blue Sky's \emph{Textures} 
      translator;\ixtt{textures}

\end{enumerate}


\sclause{Rotational directions}

    \file{isorot} enables the textual and other elements of a document
to be rotated from their normal horizontal layout. In some cases elements
can be rotated through arbitrary angles, whereas in others only 90 degree
rotation is possible. 

    By default, a rotation through a positive number of
degrees corresponds to a clockwise rotation. The command 
\verb|\counterclockwise|\ixcom{counterclockwise}
sets the following rotations to be counterclockwise for positive angles.
The command \verb|\clockwise|\ixcom{clockwise}
sets the following rotations to be clockwise for positive angles.
These commands can be used to toggle the rotational behavior.
    

    Rotated floating environments are normally rotated so that they are
printed with a counterclockwise rotation (i.e. the original bottom of the float
is placed at the right hand side of the paper), 
which is what is normally required.
This behavior can be altered by the command
\verb|\figuresleft|,\ixcom{figuresleft}
which will give the reverse effect. The command
\verb|\figuresright|\ixcom{figuresright}
will set the behavior to the default.
These commands can be used to toggle the rotational behavior of
floats.


\sclause{Rotation of text}

    The \verb|sideways|\ixenv{sideways}
environment rotates the contents of the environment
by 90 degrees counterclockwise, and leaves space for the result.

    The \verb|\begin{turn}{|\meta{angle}\verb|}|\ixenv{turn}
environment rotates the contents by the given number
of degrees in the direction specified by the most recent of the
\verb|\clockwise|\ixcom{clockwise} or
\verb|\counterclockwise|\ixcom{counterclockwise}
commands, leaving space for the result.

    The \verb|\begin{rotate}{|\meta{angle}\verb|}|\ixenv{rotate}
environment rotates the contents by the given number
of degrees in the direction specified by the most recent of the
\verb|\clockwise|\ixcom{clockwise} or
\verb|\counterclockwise|\ixcom{counterclockwise}
commands, but no arrangements are made for leaving space for the result.


\begin{example}Some simple rotations: \label{ex:1}

This code
\begin{verbatim}
Default rotation direction: \\
A 
\begin{sideways}%
B C
\end{sideways}
D E F G H I J K L M
\begin{turn}{-90}%
Minus 90 turn
\end{turn}
N O P
\begin{rotate}{90}%
Plus 90 rotate
\end{rotate}
Q \\
and continue on with another line after rotations.
\end{verbatim}
produces the following (note how space is allowed for the \verb|turn|ed 
text, whereas the \verb|rotate|d text runs into the text below).

Default rotation direction: \\
A 
\begin{sideways}%
B C
\end{sideways}
D E F G H I J K L M
\begin{turn}{-90}%
Minus 90 turn
\end{turn}
N O P
\begin{rotate}{90}%
Plus 90 rotate
\end{rotate}
Q \\
and continue on with another line after rotations.
\end{example} % end example
\begin{example}This example shows the effect of using the 
\verb|\counterclockwise|\ixcom{counterclockwise}
command.

This code
\begin{verbatim}
Flip rotation direction: \\
\counterclockwise
A 
\begin{sideways}%
B C
\end{sideways}
D E F G H I J K L M
\begin{turn}{-90}%
Minus 90 turn
\end{turn}
N O P
\begin{rotate}{90}%
Plus 90 rotate
\end{rotate}
Q \\
Set rotation direction back to default value.
\clockwise
\end{verbatim}

produces the following, which should be compared with example~\ref{ex:1}.

Flip rotation direction: \\
\counterclockwise
A 
\begin{sideways}%
B C
\end{sideways}
D E F G H I J K L M
\begin{turn}{-90}%
Minus 90 turn
\end{turn}
N O P
\begin{rotate}{90}%
Plus 90 rotate
\end{rotate}
Q \\
Set rotation direction back to default value.
\clockwise
\end{example} % end example


    Although the examples so far have only shown the rotation of text, boxes
can also be rotated.

\begin{example}Rotating a box.

This code
\begin{verbatim}
\newsavebox{\foo}
\newlength{\fool}
\settowidth{\fool}{Hurrah for ISO.}
\savebox{\foo}{\parbox{\fool}{Hurrah for ISO. Hurrah for ISO.
                              Hurrah for ISO. Hurrah for ISO.}}
Start
\usebox{\foo}
\&
\begin{turn}{-45}\usebox{\foo}\end{turn}
\&
\begin{turn}{45}\usebox{\foo}\end{turn}
End
\end{verbatim}
produces:

\newsavebox{\foo}
\newlength{\fool}
\settowidth{\fool}{Hurrah for ISO.}
\savebox{\foo}{\parbox{\fool}{Hurrah for ISO. Hurrah for ISO.
                              Hurrah for ISO. Hurrah for ISO.}}
Start
\usebox{\foo}
\&
\begin{turn}{-45}\usebox{\foo}\end{turn}
\&
\begin{turn}{45}\usebox{\foo}\end{turn}
End

\end{example} %end example

    Elements can be rotated through arbitrary angles, and also rotated
elements can be nested inside other rotated elements.


\begin{example}Repeated rotation:

    The following example code shows that text can be rotated through any angle.
The result is shown in \fref{fig:wheel}.
\begin{verbatim}
\newcount\prwc
\newsavebox{\prwtext}
\newdimen\prwspace
\def\wheel#1#2{%
  \savebox{\prwtext}{#1\begin{sideways}#2\end{sideways}}%
  \prwspace\wd\prwtext%
  \advance\prwspace by 1cm%
  \centerline{%
  \rule{0pt}{\prwspace}%
  \rule[-\prwspace]{0pt}{\prwspace}%
  \prwc=-180\loop\ifnum\prwc<180
  \rlap{\begin{rotate}{\the\prwc}%
  \rule{1cm}{0pt}\usebox{\prwtext}\end{rotate}}%
  \advance\prwc by 20\repeat}}
\begin{figure}
\wheel{Express yourself ---}{Hooray for STEP!}
\caption{Example rotation through multiple angles}
\label{fig:wheel}
\end{figure}
\end{verbatim}

\newcount\prwc
\newsavebox{\prwtext}
\newdimen\prwspace
\def\wheel#1#2{%
  \savebox{\prwtext}{#1\begin{sideways}#2\end{sideways}}%
  \prwspace\wd\prwtext%
  \advance\prwspace by 1cm%
  \centerline{%
  \rule{0pt}{\prwspace}%
  \rule[-\prwspace]{0pt}{\prwspace}%
  \prwc=-180\loop\ifnum\prwc<180
  \rlap{\begin{rotate}{\the\prwc}%
  \rule{1cm}{0pt}\usebox{\prwtext}\end{rotate}}%
  \advance\prwc by 20\repeat}}
\begin{figure}
\vspace*{1cm}
\wheel{Express yourself ---}{Hooray for STEP!}
\vspace*{1cm}
\caption{Example rotation through multiple angles}
\label{fig:wheel}
\end{figure}

    Figures~\ref{fig:angles1} and~\ref{fig:angles2} also show rotations through a
range of angles, both positive and negative.
\end{example} %end example

\begin{example}Nested rotations. \label{ex:sidetabular}

This code
\begin{verbatim}
    Here is some text before a \verb|sideways| environment. 
And some more, and more and more garble gobble cluck
click clack clock cluck and so on and on and on.
\begin{center}
\begin{sideways}
\rule{1in}{0pt}
\begin{tabular}{|lr|}
\begin{rotate}{-45}\emph{Word}\end{rotate} & \begin{rotate}{-90}%
Occurrences\end{rotate}
\\
\hline
hello & 33 \\
goodbye & 34 \\
\hline
\end{tabular}
\end{sideways}
\end{center}
    Here is some text after a \verb|sideways| environment.
 And some more, and more and more garble gobble cluck
click clack clock cluck and so on and on and on.
\end{verbatim}
produces:

    Here is some text before a \verb|sideways| environment. 
And some more, and more and more garble gobble cluck
click clack clock cluck and so on and on and on.
\begin{center}
\begin{sideways}
%\rule{1in}{0pt}
\begin{tabular}{|lr|}
\begin{rotate}{-45}\emph{Word}\end{rotate} & \begin{rotate}{-90}%
Occurrences\end{rotate} \\ \hline
hello & 33 \\
goodbye & 34 \\ \hline
\end{tabular}
\end{sideways}
\end{center}
    Here is some text after a \verb|sideways| environment.
 And some more, and more and more garble gobble cluck
click clack clock cluck and so on and on and on.
\end{example} %end example



\sclause{Rotations of tables and figures}

    The previous examples have demonstrated the rotation of textual elements.
For instance, example~\ref{ex:sidetabular} shows that tabular material can be rotated using
the \verb|sideways|\ixenv{sideways}
environment. (Actually, any of the previously
mentioned environments may be used.)
Two further environments are provided which rotate a \latex{} float through
90 degrees. These are:
\begin{itemize}
\item \verb|sidewaystable|\ixenv{sidewaystable}, which
  corresponds to the standard \latex{} \verb|table|\ixenv{table}
  environment; and
\item \verb|sidewaysfigure|\ixenv{sidewaysfigure}, which
  corresponds to the standard \latex{} \verb|figure|\ixenv{figure}
  environment.
\end{itemize}
There are also starred versions of these, namely 
\verb|sidewaystable*|\ixenvs{sidewaystable} and
\verb|sidewaysfigure*|\ixenvs{sidewaysfigure}, for use in twocolumn mode.
However, the correspondence with the standard environments is not strictly
complete as a sideways float is alway placed on a page by itself.

    The direction of rotation may be controlled by the 
\verb|\figuresright|\ixcom{figuresright} and
\verb|\figuresleft|\ixcom{figuresleft} commands.

\begin{example}Table~\ref{tab4} is produced by the code below: \label{ex:4}

\begin{verbatim}
\begin{sidewaystable}
\centering
\caption{The rotation facilities} \label{tab4}
\begin{tabular}{|l|l|} \hline
\textbf{Facility} & \textbf{Effect} \\ \hline
\multicolumn{2}{|c|}{\textbf{Commands}} \\ \hline
\verb|\rotdriver{<driver>}| & 
declare the name of the dvi to Postscript translator (default {\tt dvips}) \\
......
\verb|sidewaysfigure| &
like the \verb|figure| environment, but rotated 90 degrees \\ \hline
\end{tabular}
\end{sidewaystable}
\end{verbatim}
\end{example} % end example



\sclause{Rotation of float captions and bodies}

    Sometimes it may be useful to rotate a caption independently of the
rotation of a figure or table. The command 
\verb|\rotcaption|\ixcom{rotcaption} is analogous
to the normal \verb|\caption|\ixcom{caption} command, 
and inserts the caption rotated
by 90~degrees. There is also the companion command 
\verb|\controtcaption|\ixcom{controtcaption}, analagous to the
\verb|\contcaption|\ixcom{contcaption} command,
for continuation captions.

%\newsavebox{\picbox}

\begin{figure}
\centering
\caption{Example figure with a standard caption.} \label{fig:nocrot}
\setlength{\unitlength}{0.2in}
\footnotesize
\begin{picture}(17,2)
\thicklines
\put(0,0){\begin{picture}(4,1)
  \put(1.5,0.5){\oval(3,1)}
  \put(1.5,0.5){\makebox(0,0){2,5 (1)}}
  \put(3,0.5){\line(1,0){1.0}}
  \put(4.25,0.5){\circle{0.5}}
  \end{picture}}

\put(4.5,0){\begin{picture}(8,1)
  \put(0,0){\dashbox{0.25}(4,1){date}}
  \put(4,0.5){\line(1,0){3.5}}
  \put(7.75,0.5){\circle{0.5}}
  \put(6,1){\makebox(0,0){A[1:3]}}
  \end{picture}}

\put(12.5,0){\begin{picture}(4,1)
  \put(0,0){\framebox(4,1){INTEGER}}
  \put(3.75,0){\line(0,1){1}}
  \end{picture}}
\end{picture}
\normalsize
\setlength{\unitlength}{1pt}
\end{figure}


\begin{example}Float with a regular caption.

Figure~\ref{fig:nocrot} is produced by the code below:
\begin{verbatim}
\begin{figure}
\centering
\caption{Example figure with a standard caption.} \label{fig:nocrot}
\setlength{\unitlength}{0.2in}
\footnotesize
\begin{picture}(17,2)
\thicklines
\put(0,0){\begin{picture}(4,1)
  \put(1.5,0.5){\oval(3,1)}
  \put(1.5,0.5){\makebox(0,0){2,5 (1)}}
  \put(3,0.5){\line(1,0){1.0}}
  \put(4.25,0.5){\circle{0.5}}
  \end{picture}}
\put(4.5,0){\begin{picture}(8,1)
  \put(0,0){\dashbox{0.25}(4,1){date}}
  \put(4,0.5){\line(1,0){3.5}}
  \put(7.75,0.5){\circle{0.5}}
  \put(6,1){\makebox(0,0){A[1:3]}}
  \end{picture}}
\put(12.5,0){\begin{picture}(4,1)
  \put(0,0){\framebox(4,1){INTEGER}}
  \put(3.75,0){\line(0,1){1}}
  \end{picture}}
\end{picture}
\normalsize
\setlength{\unitlength}{1pt}
\end{figure}
\end{verbatim}
\end{example} % end example

\begin{example}Float with a rotated caption.

Figure~\ref{fig:crot} is produced by the code below:
\begin{verbatim}
\begin{figure}
\centering
\rotcaption{Figure~\protect\ref{fig:nocrot} with a rotated caption.}
 \label{fig:crot}
\setlength{\unitlength}{0.2in}
\footnotesize
\begin{picture}(17,2)
...
\end{picture}
\normalsize
\setlength{\unitlength}{1pt}
\end{figure}
\end{verbatim}
\end{example} % end example


\begin{figure}
\centering
\rotcaption{Figure~\protect\ref{fig:nocrot} with a rotated caption.}
 \label{fig:crot}
\setlength{\unitlength}{0.2in}
\footnotesize
\begin{picture}(17,2)
\thicklines
\put(0,0){\begin{picture}(4,1)
  \put(1.5,0.5){\oval(3,1)}
  \put(1.5,0.5){\makebox(0,0){2,5 (1)}}
  \put(3,0.5){\line(1,0){1.0}}
  \put(4.25,0.5){\circle{0.5}}
  \end{picture}}
\put(4.5,0){\begin{picture}(8,1)
  \put(0,0){\dashbox{0.25}(4,1){date}}
  \put(4,0.5){\line(1,0){3.5}}
  \put(7.75,0.5){\circle{0.5}}
  \put(6,1){\makebox(0,0){A[1:3]}}
  \end{picture}}
\put(12.5,0){\begin{picture}(4,1)
  \put(0,0){\framebox(4,1){INTEGER}}
  \put(3.75,0){\line(0,1){1}}
  \end{picture}}
\end{picture}
\normalsize
\setlength{\unitlength}{1pt}
\end{figure}

    As can be seen from \fref{fig:crot} the advisability of rotating a caption
depends on the size of the body of the float. It may be better in certain
cases to leave the caption in its regular position and rotate the body of
the float instead.

\def\prwrot#1{%
\settowidth{\fool}{ISOROT}
\savebox{\foo}{\parbox{\fool}{ISOROT ISOROT ISOROT ISOROT}}%
\framebox{---\begin{turn}{#1}\framebox{\usebox{\foo}}\end{turn}---}}%
\def\degrees{{\small$^{o}$}}

\begin{figure}
\centering
\begin{tabular}{|c|c|c|} \hline
\prwrot{0} &\prwrot{-40}&\prwrot{-80}\\
0\degrees & -40\degrees & -80\degrees \\ \hline
\prwrot{-120}&\prwrot{-160}&\prwrot{-200}\\
-120\degrees & -160\degrees & -200\degrees \\ \hline
\prwrot{-240}&\prwrot{-280}&\prwrot{-320}\\
-240\degrees & -280\degrees & -320\degrees \\ \hline
\end{tabular}
\caption{Rotation of paragraphs between 0 and -320 degrees} \label{fig:angles1}
\end{figure}


\begin{example}Regular caption and float.

Figure~\ref{fig:angles1} is a regular figure and caption. It is produced by
the following code:
\begin{verbatim}
\def\prwrot#1{%
\settowidth{\fool}{ISOROT}
\savebox{\foo}{\parbox{\fool}{ISOROT ISOROT ISOROT ISOROT}}%
\framebox{---\begin{turn}{#1}\framebox{\usebox{\foo}}\end{turn}---}}%
\def\degrees{{\small$^{o}$}}
\end{verbatim}

\begin{verbatim}
\begin{figure}
\centering
\begin{tabular}{|c|c|c|} \hline
\prwrot{0} &\prwrot{-40}&\prwrot{-80}\\
0\degrees & -40\degrees & -80\degrees \\ \hline
\prwrot{-120}&\prwrot{-160}&\prwrot{-200}\\
-120\degrees & -160\degrees & -200\degrees \\ \hline
\prwrot{-240}&\prwrot{-280}&\prwrot{-320}\\
-240\degrees & -280\degrees & -320\degrees \\ \hline
\end{tabular}
\caption{Rotation of paragraphs between 0 and -320 degrees} \label{fig:angles1}
\end{figure}
\end{verbatim}
\end{example} % end example


\begin{figure}
\centering
\begin{sideways}
\begin{tabular}{|c|c|c|} \hline
\prwrot{0} &\prwrot{40}&\prwrot{80}\\
0\degrees & 40\degrees & 80\degrees \\ \hline
\prwrot{120}&\prwrot{160}&\prwrot{200}\\
120\degrees & 160\degrees & 200\degrees \\ \hline
\prwrot{240}&\prwrot{280}&\prwrot{320}\\
240\degrees & 280\degrees & 320\degrees \\ \hline
\end{tabular}
\end{sideways}
\caption[Rotation of paragraphs between 0 and 320 degrees]%
        {Rotation of paragraphs between 0 and 320 degrees (with figure
         body turned sideways)}\label{fig:angles2}
\end{figure}

\begin{example}Regular caption and rotated float body.

Figure~\ref{fig:angles2} is a regular figure and caption where the figure
contents have been rotated. It was produced by the following code.
\begin{verbatim}
\begin{figure}
\centering
\begin{sideways}
\begin{tabular}{|c|c|c|} \hline
\prwrot{0} &\prwrot{40}&\prwrot{80}\\
0\degrees & 40\degrees & 80\degrees \\ \hline
\prwrot{120}&\prwrot{160}&\prwrot{200}\\
120\degrees & 160\degrees & 200\degrees \\ \hline
\prwrot{240}&\prwrot{280}&\prwrot{320}\\
240\degrees & 280\degrees & 320\degrees \\ \hline
\end{tabular}
\end{sideways}
\caption[Rotation of paragraphs between 0 and 320 degrees]%
        {Rotation of paragraphs between 0 and 320 degrees (with figure
         body turned sideways)}\label{fig:angles2}
\end{figure}
\end{verbatim}
\end{example} % end example

\begin{landscape}
\sclause{Landscaping}

    \latex{} normally prints in portrait mode. 
The \verb|landscape|\ixenv{landscape} environment
prints all the enclosed stuff in landscape mode, except for headers
and footers which are not rotated.

\begin{example} Landscaping

The source for this part of the document is:
\begin{verbatim}
\begin{landscape}
\sclause{Landscaping}

    \latex{} normally prints in portrait mode. The ...
...
... long, wide tables.
\end{landscape}
\end{verbatim}
\end{example}

    The environment starts by clearing the current page and then switches
to portrait mode. At the end of the environment the current page is cleared
and the next page is back to normal portrait mode. 

    All the other rotation commands and environments produce boxes and
\latex{} will not break a box across a page. The \verb|landscape| environemt
does not produce a box and so many pages can be printed in landscape mode
with \latex{} taking care of the page breaking for you. 

    Landscape mode is not particularly useful for normal text as the
lines are far too long for comfortable reading. Where it can be useful
is where you have a table that is too wide to fit on a portrait page, so
needs to be rotated, yet is also too long to fit on the page when it is
rotated. The \file{supertabular}\ixpack{supertabular},
the \file{longtable}\ixpack{longtable},
and the \file{xtab}\ixpack{xtab}
packages provide facilities for automatically breaking long tables across 
pages. Any of these can be used in conjunction with landscaping to both 
rotate and automatically page break long, wide tables.
\end{landscape}

\clause{The \file{xtab} package facility}

    The \file{xtab} package is fully documented in \bref{bib:xtab}.
This clause provides an overview of the package.

    The \file{xtab}\ixpack{xtab} package facility is an extension 
of the \file{supertabular}\ixpack{supertabular} package originally
developed by Johannes Braams and Theo Jurriens.
The extension provides for the specification of a header to
go on the last page of a long table.

    The principal commands available are given in \tref{tab:xtab}.
\ixenv{xtabular}
\ixenv{mpxtabular}
\ixcom{topcaption}
\ixcom{bottomcaption}
\ixcom{tablecaption}
\ixcom{tablefirsthead}
\ixcom{tablehead}
\ixcom{tablelasthead}
\ixcom{notablelasthead}
\ixcom{tabletail}
\ixcom{tablelasttail}

\topcaption{The principal xtab package commands} \label{tab:xtab}
\tablefirsthead{\hline \multicolumn{1}{|c|}{\textbf{Command}} &
                       \multicolumn{1}{c|}{\textbf{Effect}} \\ \hline }
\tablehead{\multicolumn{2}{c}%
           {{\captionsize\bfseries \tablename\ \thetable{} -- continued from previous page}} \\
           \hline \multicolumn{1}{|c|}{\textbf{Command}} &
                  \multicolumn{1}{c|}{\textbf{Effect}} \\ \hline }
\tablelasthead{\multicolumn{2}{c}%
           {{\captionsize\bfseries \tablename\ \thetable{} -- concluded from previous page}} \\ 
           \hline \multicolumn{1}{|c|}{\textbf{Command}} &
                  \multicolumn{1}{c|}{\textbf{Effect}} \\ \hline }
\tabletail{\hline \multicolumn{2}{|r|}{{Continued on next page}} \\ \hline}
\tablelasttail{\\ \hline \hline}

\begin{center}
\begin{xtabular}{|l|p{0.5\textwidth}|}
\verb|\begin{xtabular}{...}| & This is equivalent to the normal
                         \verb|\begin{tabular}{...}| environment.
                         You supply the specification of the columns
                         just as for the normal tabular environment.
                         All commands that can be used within a tabular
                         environment can also be used within
                         the xtabular environment.
 \\
 &
    Unlike the tabular environment which prevents page breaking
within the tabular, the xtabular allows page breaking, so that
tabulars can extend automatically across several pages. Xtabular
starts off with a tabular environment and checks the amount of
space left on the page as it adds each row to the tabular.
If the space left on the page is too short for another row, then
it ends the current tabular, performs a page break and starts
another tabular on the following page. This process is repeated
until all the rows have been output. 
 \\
 &
    There are special commands for captioning a xtabular as a
table, and also elements can be automatically inserted after each
(internal) \verb|\begin{tabular}| and immediately before each
\verb|\end{tabular}|.
 \\
 &
    Do not put a xtabular in a table environment, as the table
environment keeps its contents on a single page (presumably you
are using xtabular because its contents are longer than
one page).
 \\
\verb|\end{xtabular}| & End the xtabular environment. \\ \hline
\verb|\begin{mpxtabular}| & Like the xtabular environment
except that each `page' is put into a \verb|minipage| first. Thus
it is possible to have footnotes inside an \verb|mpxtabular|.
The footnote text is printed at the end of each page.
 \\
\verb|\end{mpxtabular}| & End the mpxtabular environment. \\ \hline
                          & \textbf{Note:} If any of the following commands
                            are used, then they should be placed
                            before the particular xtabular
                            environment that they apply to. \\
\verb|\topcaption{...}| & A command to provide a caption for the
                          table. The caption is placed at the top
                          of the table. \\
\verb|\bottomcaption{...}| & A command to provide a caption for the
                          table. The caption is placed at the bottom
                          of the table. \\
\verb|\tablecaption{...}| & A command to provide a caption for the
                          table. The caption is placed at the default
                          position, which is at the top
                          of the table.
 \\
 & 
                           \textbf{Notes:} You cannot use the \verb|caption|
                            command but you can put a label after
                            any of these captioning commands. If you
                            want captioning, the command must be specified
                            before the start of the xtabular environment.
  \\
  &
                            The \verb|\...caption{}| command(s) remain
                            in effect until changed by another
                            \verb|\...caption| command.
   \\
\verb|\tablefirsthead{...}| & Defines the contents of the first occurence
                         of the tabular head. The tabular head is some
                         special treatment of the first row in the table.
                         This command is optional.
                         If used, the header must be closed by the
                         end of line command for tabulars (e.g., \verb|\\|). \\
\verb|\tablehead{...}| & Defines the contents of the table head on
                         subsequent pages.
 \\
 &
                         For example, you might want to note that
                         this is a continuation of the table on
                         the previous page, as well as repeating
                         any column headings that were given
                         at the start of the xtabular by
                         \verb|\tablefirsthead|.
 \\
\verb|\tablelasthead{...}| & Defines the contents of the table head
                            on the last page of the table.
 \\
 &
    This works by writing to the \file{.aux} file the number of
pages that the xtabular extends over. When the xtabular
comes to the last tabular (which it calculates from the 
information it reads from the \file{.aux} file) it replaces the
contents of \verb|\tablehead| by the contents of \verb|\tablelasthead|.
 \\
 &
It makes no attempt to measure the amount of space used by the last
table head, so if this is more than for \verb|tablehead| then the
tabular might be a litle too long.
  \\
  &
    In any case, when using this command, the document has to
be LaTeXed at least twice, just as it has to be to resolve
references and so on.
 \\
\verb|\notablelasthead|  & Switches off the last \verb|\tablelasthead|.
                           A \verb|tablelasthead| stays in effect until
                           overwritten by a new \verb|\tablelasthead|
                           or cancelled by this command.
 \\
\verb|\tabletail{...}| & The contents of this command are inserted before
                         the \verb|\end{tabular}| on each page except
                         for the last page of the table.
 \\
 &
                         For example, you might want to note that the
                         table is continued on the next page. 
 \\
\verb|\tablelasttail{...}| & The contents of this command are inserted before
                         the final \verb|\end{tabular}| of the table.
 \\
 &
                         For example, you might want to note that
                         this is where the table ends.
\end{xtabular}
\end{center}

    As well as the \verb|xtabular|\ixenv{xtabular} and
\verb|mpxtabular|\ixenv{mpxtabular} environments there are
the corresponding starred versions 
(i.e., \verb|xtabular*|\ixenvs{xtabular} and 
        \verb|mpxtabular*|\ixenvs{mpxtabular}) for use in
two column\ixopt{twocolumn} mode where the table is meant to span 
both columns.

\begin{example} \label{ex:xtab} Table~\ref{tab:xtab} was produced by 
the following code:
\begin{verbatim}
\topcaption{The principal xtab package commands} \label{tab:xtab}
\tablefirsthead{\hline \multicolumn{1}{|c|}{\textbf{Command}} &
                       \multicolumn{1}{c|}{\textbf{Effect}} \\ \hline }
\tablehead{\multicolumn{2}{c}%
           {{\captionsize\bfseries \tablename\ \thetable{} -- 
             continued from previous page}} \\
  \hline   \multicolumn{1}{|c|}{\textbf{Command}} &
           \multicolumn{1}{c|}{\textbf{Effect}} \\ \hline }
\tablelasthead{\multicolumn{2}{c}%
           {{\captionsize\bfseries \tablename\ \thetable{} -- 
             concluded from previous page}} \\
  \hline   \multicolumn{1}{|c|}{\textbf{Command}} &
           \multicolumn{1}{c|}{\textbf{Effect}} \\ \hline }
\tabletail{\hline \multicolumn{2}{|r|}{{Continued on next page}} \\ \hline}
\tablelasttail{\\ \hline \hline}

\begin{center}
\begin{xtabular}{|l|p{0.5\textwidth}|}
\verb|\begin{xtabular}{...}| & This is equivalent to the normal
                         \verb|\begin{tabular}{...}| environment.
                         You supply the specification of the columns
                         just as for the normal tabular environment.
                         All commands that can be used within a tabular
                         environment can also be used within
                         the xtabular environment.
 \\
 &
    Unlike the tabular environment which prevents page breaking
within the tabular, the xtabular allows page breaking, so that
tabulars can extend automatically across several pages. Xtabular
... ... ...
\verb|\tablelasttail{...}| & The contents of this command are inserted before
                         the final \verb|\end{tabular}| of the table.
 \\
 &
    For example, you might want to note that  this is where 
the table ends.
\end{xtabular}
\end{center}
\end{verbatim}
\end{example}       

    The table is only broken between rows --- a row will not be split
across pages. This can lead to some bad page breaks, especially if 
there are rows with a large vertical height (like some in \tref{tab:xtab}).
It is best to keep rows not too tall.

The command
\verb|\shrinkheight{|\meta{length}\verb|}|\ixcom{shrinkheight} may be
used after the first \verb|\\| in the table to modify the allowed
height of the table on each page. A positive \meta{length} decreases
the allowed space per page and a negative \meta{length} increases
the allowed space.

\begin{example}
This example illustrates changing the natural height of the pages
in a \verb|xtabular| and its relatives.

\verb|\shrinkheight{2\baselineskip}| decreases the space per page by 
two lines.

\verb|\shrinkheight{-\baselineskip}| increases the space per page by
one line.
\end{example}

    You have to specify the font used for the 
\verb|\tablehead|\ixcom{tablehead} and
\verb|tablelasthead|\ixcom{tablelasthead} yourself. 
Within ISO documents, captions shall
be in bold font. The \file{iso}\ixclass{iso} class also provides 
a command for
setting the size of the font used in captions, namely 
\verb|\captionsize|\ixcom{captionsize}. 
Note that this is used in~\eref{ex:xtab}.
    The value of \verb|\captionsize|\ixcom{captionsize} is set by the 
class.

\sclause{Options}

    The \file{xtab} package has three options which control the amount of
information that is written to the \file{.log}\index{file!.log} file. The
options are:
\begin{enumerate}
\item The option \verb|errorshow|\ixopt{errorshow} (the default) does not
      write any extra information;
\item The option \verb|pageshow|\ixopt{pageshow} writes information
      about when and why \verb|xtabular| decides to produce a new page;
\item The option \verb|debugshow|\ixopt{debugshow}, which also includes
      \verb|pageshow|, additionally writes information about each line
      that is added to the table.
\end{enumerate}

    Under normal circumstances \file{xtab} is used without invoking any 
option. The \verb|pageshow| option may be useful when attempting to cure
a bad page break. The \verb|debugshow| option, as its name implies, is
principally of use to the \file{xtab} developer.

    Independently of the options, the command 
\verb|\sstraceon|\ixcom{sstraceon} may be used at any
point in the document to turn on printing of \verb|debugshow| data. This
can be turned off later by the \verb|\sstraceoff|\ixcom{sstraceoff}
command, which will stop all \verb|...show| printing.

\sclause{Caveats}

    The authors of the original \file{supertabular} package note that
it has the following weaknesses:
\begin{itemize}
\item Sometimes the body of the first page of a table gets moved to the 
following, leaving the caption behind;
\item Sometimes the last page of a table consists of just the head and foot
with no contents.
\end{itemize}
The weaknesses are caused by trying to guess where \tex{} will put a page 
break. The package has to guesstimate how long the next entry will be in 
the table and, if it is too long for the available space, it puts in its
own page break. If its guess is off too much in one direction, \tex{} will 
break the page unexpectadly; if its off in the other direction
\file{supertabular} will put in an unnecessary page break.

    The \file{xtab} package has reduced, but perhaps not entirely
eliminated, these weaknesses. Some hand tuning may still be required.


%%%%%%%%%
%%%%\end{document}
%%%%%%%%%



\clause{The \file{askinc} package facility} 

    The \file{askinc}\ixpack{askinc} package facility contains \latex{} 
code to enable the interactive
input of files. This functionality is a cross between the \latex{}
\verb|\include|\ixcom{include} and 
\verb|\includeonly|\ixcom{includeonly} commands, and the
\verb|\input|\ixcom{input} command.

In the body of the root source document, use the command 
\verb+\infile{+\meta{file}\verb+}+\ixcom{infile}
for each \meta{file}
that comprises the document. That is, the command \verb|\infile| is similar
to the \verb|\input|\ixcom{input} and
\verb|\include|\ixcom{include} commands.

    When \latex ing the root document, you will be asked to provide a 
comma-separated list of the \verb|\infile|s to be processed (similar to the
argument to the \verb|\includeonly|\ixcom{includeonly}
command). If you want all the files to be processed, just hit the
\verb|<RETURN>| key (or its equivalent).

    Like \verb|\include|d files, a file that is \verb|\infile|d into a 
document shall not itself contain another \verb|\infile|d file.

\begin{example}The following root file has three files that are \verb|\infile|d.
\begin{verbatim}
\documentclass[...]{...}
\usepackage{askincv}
% other preamble stuff
\begin{document}
% perhaps some stuff
\infile{file1}
\infile{file2}
\infile{file3}
% perhaps more stuff
\end{document}
\end{verbatim}
\end{example} % end example

\clause{The \file{hyphenat} package facility} \label{sec:uschyp}

    The \file{hyphenat} package is fully described in~\bref{bib:hyphenat}.
This clause provides an overview of the portions of the package that are
most relevant to typesetting ISO standards.

    In \latex{} if you need to use the underscore (\verb|_|)
character in normal text, for example when documenting identifiers
in a programming language code, you have to use the 
\verb|\_|\index{_ /@\verb?\_?} command,
as in \verb|a\_multiword\_identifier|. \latex{} normally treats
\verb|_|\index{_@\verb?_? (underscore)} as a math mode subscript command. 
Further, if you want
the possibility of hyphenation\index{hyphenation} 
at the position of an underscore
you have to use the command pairing 
\verb|\_\-|;\index{_ /@\verb?\_?}\ixcom{-} this has the side
effect of disabling normal hyphenation in any succeeding `word' in
the identifier.

    The \file{hyphenat}\ixpack{hyphenat} package facility redefines the 
\verb|\_|\index{_ /@\verb?\_?} command
so that hyphenationen is automatically enabled at the position
of the underscore and in all succeeding words in the identifier.


\begin{note}
    Using the command pair \verb|\_\-| in conjunction with this facility
disables automatic hyphenation of succeeding words, so don't do it.
\end{note}



\begin{example} \label{eg:uschyp}
This set of \latex{} source
\begin{verbatim}
Amazingly An\_excessively\_long\_multiword\_identifier%
\_demonstrating\_hyphenation

\begin{minipage}{3cm}
\begin{itemize}
\item An\_excessively\_long\_multiword\_identifier%
\_demonstrating\_hyphenation
\item Underscore in math mode: $A_n$
\item \verb|\_| command in math mode: $A\_n$
\end{itemize}
\end{minipage}
\end{verbatim}
prints as:

Amazingly An\_excessively\_long\_multiword\_identifier%
\_demonstrating\_hyphenation

\begin{minipage}{3cm}
\begin{itemize}
\item An\_excessively\_long\_multiword\_identifier%
\_demonstrating\_hyphenation
\item Underscore in math mode: $A_n$
\item \verb|\_| command in math mode: $A\_n$
\end{itemize}
\end{minipage}

\end{example}

\begin{example}
Contrast this example with \eref{eg:uschyp}.
This set of \latex{} source
\begin{verbatim}
Amazingly An\_\-excessively\_\-long\_\-multiword\_\-identifier%
\_\-demonstrating\_\-hyphenation\_\-disabling

\begin{minipage}{3cm}
\begin{itemize}
\item An\_\-excessively\_\-long\_\-multiword\_\-identifier%
\_\-demonstrating\_\-hyphenation\_\-disabling
\item Underscore in math mode: $A_n$
\item \verb|\_| command in math mode: $A\_n$
\end{itemize}
\end{minipage}
\end{verbatim}
prints as:

Amazingly An\_\-excessively\_\-long\_\-multiword\_\-identifier%
\_\-demonstrating\_\-hyphenation\_\-disabling

\begin{minipage}{3cm}
\begin{itemize}
\item An\_\-excessively\_\-long\_\-multiword\_\-identifier%
\_\-demonstrating\_\-hyphenation\_\-disabling
\item Underscore in math mode: $A_n$
\item \verb|\_| command in math mode: $A\_n$
\end{itemize}
\end{minipage}

\end{example}


    The \file{hyphenat} package also provides some other commands for
enabling hyphenation of words that include 
analphabetic\index{character!analphabetic}\footnote{An {\it analphabetic} 
character is any character
that is not alphabetic. Typically it refers to punctuation characters.}
characters. In this context, the phrase 
`breakable character'\index{character!breakable} is used
to describe an analphabetic character that enables hyphenation immediately
after it and does not prevent further hyphenation in the `word'
in which it occurs. The \verb|\_|\index{_ /@\verb?\_?} command produces 
a breakable underscore. Table~\ref{tab:breakable} lists all the commands
that generate breakable characters.

\begin{table}
\centering
\caption{Commands producing breakable characters} \label{tab:breakable}
\begin{tabular}{|l|c|} \hline
\textbf{Command}    & \textbf{Character} \\ \hline
\verb|\_|        & \_ \\
%\verb|\?|        & \? \\
\verb|\bshyp|    & \bshyp \\
\verb|\colonhyp| & \colonhyp \\
\verb|\dothyp|   & \dothyp \\
\verb|\fshyp|    & \fshyp \\ \hline
\end{tabular}
\end{table}

    The \verb|\bshyp|\ixcom{bshyp} command prodes a breakable backslash
(\verb|\|), \verb|\fshyp|\ixcom{fshp} produces a breakable forward
slash (\verb|/|), the \verb|\dothyp|\ixcom{dothyp} command produces a
breakable full stop (\verb|.|), also known in some countries as a period,
and the \verb|\colonhyp|\ixcom{colonhyp} command produces a breakable 
colon (:).

\begin{example}
This is similar to \eref{eg:uschyp} except that it demonstrates other
breakable characters.

This set of \latex{} source
\begin{verbatim}
Analphabetically an\bshyp{}excessively\fshyp{}long\dothyp{}multiword\bshyp{}identifier%
\fshyp{}demonstrating\dothyp{}hyphenation

\begin{minipage}{3cm}
Analphabetically an\bshyp{}excessively\fshyp{}long\dothyp{}multiword\bshyp{}identifier%
\fshyp{}demonstrating\dothyp{}hyphenation
\end{minipage}
\end{verbatim}
prints as:

Analphabetically an\bshyp{}excessively\fshyp{}long\dothyp{}multiword\bshyp{}identifier%
\fshyp{}demonstrating\dothyp{}hyphenation

\begin{minipage}{3cm}
Analphabetically an\bshyp{}excessively\fshyp{}long\dothyp{}multiword\bshyp{}identifier%
\fshyp{}demonstrating\dothyp{}hyphenation
\end{minipage}

\end{example}

\begin{note}
\latex{} will not hyphenate the first word in a paragraph.
\end{note}

    Just as with the \verb|\_|\index{_ /@\verb?\_?} command, 
the discretionary hyphen
command (i.e., \verb|\-|\ixcom{-}) should not be used in conjunction with 
any of the breakable character commands as it will then inhibit any
further potential hyphenation points. In general, any analphabetic
character in a word will inhibit further hyphenation.

\normannex{Additional commands} \label{anx:extraiso}

\sclause{Language configuration commands}

    There is an additional set of commands in the
\file{iso}\ixclass{iso} class
facility that
are specified here. This set has been provided to enable the style to be
easily configured for a non-English language.\index{non-English languages}

    The modified commands may be put in the document 
preamble\index{preamble} or, preferably,
placed in a separate \file{.sty}\index{file!.sty} file and 
called as a package. This latter option improves reuseability.

\ssclause{Words and phrases}

    To produce a non-English version of the \file{iso}\ixclass{iso}
class the relevant commands
from the following list may require re-definition together with
the heading commands in \tref{tab:dsa}.

\begin{note}In the listing, the default values are printed \emph{in this
font} to distinguish them from the explanatory text.
\end{note} % end note}

\begin{itemize}
\item \verb|\annexname|\ixcom{annexname}: Header for
      an annex.
      Default value is: \emph{\annexname}\index{Annex}
\item \verb|\contentsname|\ixcom{contentsname}: Header
      for table of contents listing. 
      Default value is: \emph{\contentsname}\index{Contents}
\item \verb|\copyrightname|\ixcom{copyrightname}:
      The copyright owner.
      Default value is: \emph{\copyrightname}\index{ISO}
\item \verb|\examplename|\ixcom{examplename}:
      Identification of an example.
      Default value is: \emph{\examplename}\index{EXAMPLE}
%\item \verb|\examplesname|\ixcom{examplesname}:
%      Header for a list of examples.
%      Default value is: \emph{\examplesname}\index{EXAMPLES}
\item \verb|\figurename|\ixcom{figurename}: Start of
      the caption for a figure. 
      Default value is: \emph{\figurename}\index{Figure}
\item \verb|\forewordname|\ixcom{forewordname}:
      Title of the Foreword.
      Default value is: \emph{\forewordname}\index{Foreword}
\item \verb|\indexname|\ixcom{indexname}: Header for
      the index.
      Default value is: \emph{\indexname}\index{Index}
\item \verb|\informativename|\ixcom{informativename}:
      Identification of an informative annex.
      Default value is: \emph{\informativename}\index{informative}
\item \verb|\inscopename|\ixcom{inscopename}:
      Introduction to in-scope listing.
      Default value is: \emph{\inscopename}
      \index{The following are within the scope of this}
\item \verb|\ISname|\ixcom{ISname}:
     `INTERNATIONAL STANDARD' as used in the header for an IS title page.
      Default value is: 
      \emph{\ISname}\index{INTERNATIONAL STANDARD}
\item \verb|\introductionname|\ixcom{introductionname}:
      Title of the Introduction.
      Default value is: 
      \emph{\introductionname}\index{Introduction}
\item \verb|\listannexname|\ixcom{listannexname}:
      Header for list of annexes listing.
      Default value is: \emph{\listannexname}\index{Annexes}
\item \verb|\listfigurename|\ixcom{listfigurename}:
      Header for list of figures listing.
      Default value is: \emph{\listfigurename}\index{Figures}
\item \verb|\listtablename|\ixcom{listtablename}:
      Header for list of tables listing.
      Default value is: \emph{\listtablename}\index{Tables}
\item \verb|\normativename|\ixcom{normativename}:
      Identification of a normative annex.
      Default value is: \emph{\normativename}\index{normative}
\item \verb|\normrefsname|\ixcom{normrefsname}:
      Title of normative references clause.
      Default value is: 
      \emph{\normrefsname}\index{Normative references}
\item \verb|\notename|\ixcom{notename}:
      Identification of a note.
      Default value is: \emph{\notename}\index{NOTE}
%\item \verb|\notesname|\ixcom{notesname}:
%      Header for a list of notes.
%      Default value is: \emph{\notesname}\index{NOTES}
\item \verb|\outofscopename|\ixcom{outofscopename}:
      Introduction to out-of-scope listing.
      Default value is: \emph{\outofscopename}
      \index{The following are outside the scope of this}
\item \verb|\pagename|\ixcom{pagename}:
      The word for the page header in the table of contents.
      Default value is: \emph{\pagename}\index{Page}
\item \verb|\scopename|\ixcom{scopename}:
      Title of the Scope.
      Default value is: \emph{\scopename}\index{Scope}
%\item \verb|\sectionname|\ixcom{sectionname}:
%      The word for a 'section'.
%      Default value is: \emph{\sectionname}\index{Section}
\item \verb|\tablename|\ixcom{tablename}:
      Start of the caption for a table.
      Default value is: \emph{\tablename}\index{Table}
\item \verb|\tbpname|\ixcom{tbpname}:
      Footnote text for `to be published.'.
      Default value is: 
      \emph{\tbpname}\index{To be published.}
\end{itemize}

    The following commands provide the names for referenced document elements.
\begin{itemize}
\item \verb|\annexrefname|\ixcom{annexrefname}:
      Reference to an annex.
      Default value is: \emph{\annexrefname}\index{annex}
\item \verb|\clauserefname|\ixcom{clauserefname}:
      Reference to a clause.
      Default value is: \emph{\clauserefname}\index{clause}
\item \verb|\examplerefname|\ixcom{examplerefname}:
      Reference to an example.
      Default value is: \emph{\examplerefname}\index{example}
\item \verb|\figurerefname|\ixcom{figurerefname}:
      Reference to a figure.
      Default value is: \emph{\figurerefname}\index{figure}
\item \verb|\noterefname|\ixcom{noterefname}:
      Reference to a note.
      Default value is: \emph{\noterefname}
\item \verb|\tablerefname|\ixcom{tablerefname}:
      Reference to a table.
      Default value is: \emph{\tablerefname}\index{table}
\item \verb|\pagerefname|\ixcom{pagerefname}:
      Reference to a page.
      Default value is: \emph{\pagerefname}\index{page}
\end{itemize}



\begin{note} The above commands, 
 may be changed via the \latex{} 
\verb|\renewcommand|\ixcom{renewcommand}.
\end{note} % end note
\begin{note}The \latex{} command \verb|\today|\ixcom{today}
      will probably also require modification. This is not something
      for the casual user to attempt.
\end{note} % end note


\begin{example}The following is a partial list of the commands to convert to a 
         French language\index{French language} style. 
\begin{verbatim}
\renewcommand{\annexname}{Annexe}
\renewcommand{\contentsname}{Sommaire}
\renewcommand{\examplename}{EXEMPLE}
%\renewcommand{\examplesname}{EXEMPLES}
\renewcommand{\forewordname}{Avant-propos}
\renewcommand{\ISname}{NORME INTERNATIONALE}
\renewcommand{\listtablename}{Tableaux}
\renewcommand{\scopename}{Domaine d'application}
\renewcommand{\tablename}{Tableau}
\end{verbatim}
\end{example} % end example

\ssclause{Boilerplate}

    Some commands print boilerplate text; again, the default text is in English.
For \file{iso}\ixclass{iso} there are several such commands. The first is 
\verb|\copyrightnotice|\ixcom{copyrightnotice}
 which contains the text of the copyright notice
for an International Standard. This can be changed via the 
\verb|\renewcommand|\ixcom{renewcommand} command.

    The second is \verb|\normrefbp|\ixcom{normrefbp}
which prints the boilerplate for the introduction to the Normative
references clause. Like the \verb|\copyrightnotice| command, this can
be redefined using the \verb|\renewcommand|.

    Another is the \verb|\fwdbp|\ixcom{fwdbp} command which
\verb|\input|s the boilerplate text from a file called \file{isofwdbp.tex}.
\index{isofwdbp.tex@\file{isofwdbp.tex}} For non-English text either 
modify the contents of this
file or create a new file and modify the \verb|\fwdbp| command to call in
the new file.



\begin{example}This is how \verb|\normrefbp| could be written for the 
French language and in accordance with the second edition of the 
ISO Directives.
\label{eg:frenchnfbp}
\begin{verbatim}
\renewcommand{\normrefbp}[1]{%
    Les normes suivantes contiennent des dispositions qui, par
suite de la r\'{e}f\'{e}nce qui en est faite, constituent des
dispositions valables pour la pr\'{e}sente #1. 
Au moment de la publication, les \'{e}ditions indiqu\'{e}es 
\'{e}taient en viguer. Toute norms est sujette \`{a} r\'{e}vision et
les parties prenantes des accords fond\'{e}s sur la pr\'{e}sente #1 
sont invit\'{e}es \`{a} rechercher la possibilit\'{e} d'appliquer
les \'{e}ditions les plus r\'{e}centes des normes indiqu\'{e}es
ci-apr\`{e}s. Les membres de la CEI et de l'ISO poss\`{e}dent
le registre des Normes Internationales en vigueur \`{a} un
moment donn\'{e}.
}% end renewcommand
\end{verbatim}
\end{example} % end example
\begin{example}Given that \verb|\normrefbp| has been redefined as in \eref{eg:frenchnfbp},
then the command \\
 \verb|\normrefbp{norme internationale}| will print:

\newcommand{\tempbp}[1]{%
    Les normes suivantes contiennent des dispositions qui, par
suite de la r\'{e}f\'{e}nce qui en est faite, constituent des
dispositions valables pour la pr\'{e}sente #1. 
Au moment de la publication, les \'{e}ditions indiqu\'{e}es 
\'{e}taient en viguer. Toute norms est sujette \`{a} r\'{e}vision et
les parties prenantes des accords fond\'{e}s sur la pr\'{e}sente #1 
sont invit\'{e}es \`{a} rechercher la possibilit\'{e} d'appliquer
les \'{e}ditions les plus r\'{e}centes des normes indiqu\'{e}es
ci-apr\`{e}s. Les membres de la CEI et de l'ISO poss\`{e}dent
le registre des Normes Internationales en vigueur \`{a} un
moment donn\'{e}.
}% end newcommand

\tempbp{norme internationale}
\end{example} % end example



    The \verb|\tspasfwdbp|\ixcom{tspasfwdbp} also maintains
boilerplate text in the file 
\file{tspasfwdbp.tex}\index{tspasfwdbp.tex@\file{tspasfwdbp.tex}}.

\sclause{Caption text size}

    The size of the font used for typestting the captions of figures and
tables is defined within the \file{iso}\ixclass{iso} class.
% The
%\verb|uglycaption|\ixopt{uglycaption} option resets the size to larger
%than normal for the captioning text.

    The size of the captioning font is controlled by the value of the
\verb|\captionsize|\ixcom{captionsize} command. The default definition
of \verb|\captionsize| is similar to:
\verb|\captionsize{\normalsize}| \ixcom{normalsize}\\
%The \verb|uglycaption| option resets this to (approximately): \\
%\verb|\captionsize{\large}|. \ixcom{large}

    You can reset the \verb|\captionsize| at any point in your
document to change the size of captions from then onwards.

    All the normal \latex{} font size commands are available.
\begin{example}
The following shows the effects of the font size commands. \\
\verb|{\tiny tiny text}|\ixcom{tiny} 
prints: {\tiny tiny text} \\
\verb|{\scriptsize scriptsize text}|\ixcom{scriptsize} 
prints: {\scriptsize scriptsize text} \\
\verb|{\footnotesize footnotesize text}|\ixcom{footnotesize} 
prints: {\footnotesize footnotesize text} \\
\verb|{\small small text}|\ixcom{small}
prints: {\small small text} \\
\verb|{\normalsize normalsize text}|\ixcom{normalsize} 
prints: {\normalsize normalsize text} \\
\verb|{\large large text}|\ixcom{large} 
prints: {\large large text} \\
\verb|{\Large Large text}|\ixcom{Large} 
prints {\Large Large text} \\
\verb|{\LARGE LARGE text}|\ixcom{LARGE} 
prints: {\LARGE LARGE text} \\
\verb|{\huge huge text}|\ixcom{huge} 
prints: {\huge huge text} \\
\verb|{\Huge Huge text}|\ixcom{Huge} 
prints: {\Huge Huge text}
\end{example}


\normannex{Ordering of LaTeX commands} \label{anx:lord}

    The \latex{} commands for the logical structuring of an ISO standard 
document are:
\begin{verbatim}
\documentclass[<options>]{isov2}          % for LaTeX 2e
\usepackage{<name>}                       % additional packages (LaTeX 2e)
\standard{<standard identification>}
\yearofedition{<year>}
\languageofedition{<parenthesized code letter>}
  % other preamble commands
\begin{document}
\begin{foreword}                     % start Foreword
  \fwdbp                             % boilerplate
  % other text and perhaps \fwdnopatents
\end{foreword}
\begin{introduction}                 % start Introduction
  % text and perhaps \intropatents
\end{introduction}
\title{<intro>}{<main>}{<compl>}     % the title
\scopeclause                         % The Scope clause
   \begin{inscope}{<document>}       % in scope boilerplate
   % \item list
   \end{inscope}
   % text 
   \begin{outofscope}{<document>}    % out of scope boilerplate
   % \item list
   \end{outofscope}
   % text
\normrefsclause                      % The Normative references clause
  \normrefbp{<document identifier>}  % boilerplate
  \begin{nreferences}
    % \isref{<p1>}{<p2>} and/or \disref{<p1>}{<p2>} commands
  \end{nreferences}
% definitions, symbols, abbreviation clause as appropriate
\clause{<Clause title>}
% THE BODY OF THE DOCUMENT
% ...
% \normannex{<Normative annex title>}
% ...
% \infannex{<Infomative annex title>}
% ...
\bibannex                            % optional bibliography
  % bibliography listing
% the index
\end{document}
\end{verbatim}


\infannex{Creating an index} \label{anx:indexing}

    \latex, although providing some assistance in preparing the data for an
index\index{index}, only does part of the job. Providing the command
\verb|\makeindex|\ixcom{makeindex} is put in the document's 
preamble\index{preamble}
the \latex{} command \verb|\index{text}|\ixcom{index} 
writes out \verb|text| to an \file{.idx}\index{file!.idx} file 
in the following format:\ixcom{indexentry}
\begin{verbatim}
\indexentry{text}{pg}
\end{verbatim}
where \verb|pg| is the page number in the document where the \verb|\index|
command occurred.

    The \verb|theindex|\ixenv{theindex} environment 
is used for printing an index. The format of this is:
\begin{verbatim}
\begin{theindex}
\item <text and page numbers>
  \subitem <text and page numbers>
    \subsubitem <text and page numbers>
.
.
\end{theindex}
\end{verbatim}
where \verb|\item|\ixcom{item} is a major topic entry,
\verb|\subitem|\ixcom{subitem} is a sub-topic entry, and
\verb|\subsubitem|\ixcom{subsubitem} is a sub-subtopic
entry. The command \verb|\indexspace|\ixcom{indexspace}
can be used to add space between the entries. Other text and commands can also
occur within the environment.

    \latex{} provides no help in going from an \file{.idx} file to the
\verb|theindex| environment~\bref{lamport}. The data in the file has to be
sorted, duplicate page numbers deleted, etc, and then re-written in the 
desired environment format.

    The typical process for producing a document with an index is:
\begin{enumerate}
\item Prepare the source file, say \file{fred.tex}, with the command
      \verb|\makeindex|\ixcom{makeindex} in the
      preamble\index{preamble}, and \verb|\index|\ixcom{index}
      commands within the body of the text.
\item Run \latex. As well as producing the usual output files, it will also
      produce the file \file{fred.idx}.
\item By some means produce a file, let's call it \file{fredidx.tex}, from the 
      data in \file{fred.idx} that contains the appropriate \verb|theindex|
      formatted data.
\item Run \latex{} again on \file{fred.tex} which now has to contain (either
      via \verb|\input|\ixcom{input} or \verb|\include|\ixcom{include})
      the file \file{fredidx.tex}.
\end{enumerate}

    Chen and Harrison~\bref{chen} discuss the problems of creating an index 
in their
paper \emph{Index preparation and processing} and also describe the
{\sc MakeIndex}\index{makeIndex@{\sc MakeIndex}} program. Goosens, Mittelbach
and Samarin~\bref{goosens} also describe how to use the {\sc MakeIndex} 
program for producing indexes for \latex{} documents. For users of
{\sc MakeIndex} a style file called 
\file{iso.ist}\index{iso.ist@\file{iso.ist}}\index{file!.ist} is provided as
part of this distribution.

\sclause{The index command}\ixcom{index}

    The \verb|\index| command is one of the standard \latex{} commands.
The command format is \verb|\index{|\meta{str}\verb|}|, where \meta{str}
 is any
string of characters, and it writes an entry to the \file{.idx} file in
the form \verb|\indexentry{<str>}{pg}|, where \verb|pg| is the page
number in the document where the command is called.

    Some points to note:
\begin{itemize}
\item The \verb|\index| command is, in \latex{} terminology, 
      \emph{fragile}\index{fragile}. That is, if it appears in a moving
      argument (like the caption to a table or figure) it must be preceded
      by the \latex{} \verb|\protect|\ixcom{protect}
      command.
\item Any of the ten \latex{} special characters\index{special characters}
      (i.e., \verb|#|, \verb|$|, \verb|%|, \verb|&|, \verb|~|, \verb|_|,
             \verb|^|, \verb|\|, \verb|{| and \verb|}|) may appear
     within the argument, with the proviso that it must have no unmatched
     braces (the braces in \verb|\{| and \verb|\}| are counted in the
     matching process).
\item The \verb|\index| command must not appear inside another command's
     argument (e.g., within a 
     \verb|\footnote|\ixcom{footnote} or \verb|\clause|\ixcom{clause}
     command) unless the argument of the \verb|\index| command contains
     only letters, digits, and/or punctuation characters. In particular,
     it should not contain any of the special characters.
  \begin{note}This means that the argument of the \verb|\ix|\ixcom{ix}
        command should not contain any special characters. Remember that the
       \verb|\ix| command prints its argument in the body of the text and also
       calls \verb|\index| to place its argument into the \file{.idx} file.
       \end{note} % end note
\end{itemize}

\begin{note}Under some circumstances, an \verb|\index| command appearing in another
command's argument may contain special characters, provided they are 
\verb|\protect|ed. Determination of when this is satisfactory is a process
of trial and error.
\end{note} % end note

\begin{example}The command \verb|\ix{an\protect\_underscore}| will print the
characters \emph{an\_undescore}
in the text and also write the following to the \file{.idx} file:
\begin{verbatim}
\indexentry{an\_ underscore}{pg}
\end{verbatim}
Notice that there is a space between the underscore character and the word
`underscore' in the \file{.idx} file (but there is no space in the printed
body of the document text). This extraneous space may have to be edited out 
from the final index.
\end{example} % end example


\begin{comment}

\sclause{The program GenIndex} \index{genindex@GenIndex}

    GenIndex is a C program\index{C language} that converts \file{.idx}
data to \verb|theindex|\ixenv{theindex} data. 
Source code for GenIndex is freely available from
the NIST SOLIS\index{SOLIS} system (see \ref{anx:solis}). 
The GenIndex program is not
as sophisticated as {\sc MakeIndex} but does a reasonable 
job.\footnote{These two programs are not completely compatible but do have
a common subset of commands. The common subset excludes the {\tt SeeAlso}
and {\tt See} commands. At some later time GenIndex may be rewritten to
be compatible with {\sc MakeIndex}. It is probably advisable, therefore, not
to use the GenIndex {\tt SeeAlso} and {\tt See} commands.}


\ssclause{Input} \index{genindex@GenIndex!input}

    GenIndex reads lines of data of the form:
\begin{verbatim}
\indexentry{IndexData}{Page}
\end{verbatim}

    \verb|Page| is a character string representing a page number. GenIndex only
recognizes strings that represent integer values greater than or equal to zero,
or (upper and lower case) roman numerals. For any other numbering system, the
page number is treated as zero.

    \verb|IndexData| is a string of characters and command characters that
specify the data that is to be indexed. In the simplest case, this is just the
word or phrase to appear in the index, but much more can be done than this, as
is described below.

    The general form of \verb|IndexData| is given by the following partial
grammar:
\begin{verbatim}
IndexData  = MainData ['!' SubData ['!' SubSubData ] ] .
MainData   = Data .
SubData    = Data .
SubSubData = Data .
Data       = Key [ Entry ] [ SeeAlso ] [ See ] .
Key        = any string not containing the @, !, + or * characters .
Entry      = '@' any string not containing the !, + or * characters 
                 (unless enclosed in braces {} ) .
SeeAlso    = '+' any string not containing the ! or * characters
                 (unless enclosed in braces {} ) .
See        = '*' any string not containing the ! character
                 (unless enclosed in braces {} ) .
\end{verbatim}

    Upto three levels of indexing are enabled --- a main topic entry, an optional
sub-topic, and an optional sub-subtopic. The sub-topics for an entry are
introduced by the \verb|!| character. \index{(33@{\verb?!?}}

     Apart from \verb|Key|, braces within a string must be matched. That is,
they must appear in pairs of an opening and a closing brace.
Spaces are significant within the \verb|Key| string, but not in the others.
That is, \verb*|\indexentry{A}| differs from \verb*|\indexentry{ A}| 
both of which differ from \verb*|\indexentry{A }|.


\begin{example}Here is an index entry for a simple topic:
\begin{verbatim}
\indexentry{Topic}{27}
\end{verbatim}
\end{example} % end example
\begin{example}And here is one where the key and the main entry are different, a 
sub-topic is being indexed, and the page is in roman numerals:
\begin{verbatim}
\indexentry{main@\textbf{Main}!sub topic}{xviii}
\end{verbatim}
\end{example} % end example
\begin{example}This is how the characters \verb|@|, \verb|!|, \verb|+| and
\verb|*| characters are indexed in this document.
\begin{verbatim}
\index{(64@{\verb?@?}}
\index{(33@{\verb?!?}}
\index{(43@{\verb?+?}}
\index{(42@{\verb?*?}}
\end{verbatim}
\end{example} % end example


\sssclause{Key} \index{genindex@GenIndex!input!key}

    The indexed entries are sorted alphabetically on the value of \verb|Key|.
Sorting is case-insensitive. A \verb|Key| value is required.

    More precisely, the key entries are sorted according to the
C language\index{C language} implementation collating sequence, which is usually ASCII. 
Table~\ref{tab:ascii} gives the ASCII collating sequence for the \latex\
character set. \index{ASCII}


\begin{table}
\def\vissp{\hbox{\tt\char`\ }} % visible space
\centering
\caption{The \protect\latex{} ASCII character set} \label{tab:ascii}
\begin{tabular}{|r|c|c|c|c|c|c|c|c|c|c|} \hline
  & \textbf{0} & \textbf{1} & \textbf{2} & \textbf{3} & \textbf{4} &
\textbf{5} & \textbf{6} & \textbf{7} & \textbf{8} & \textbf{9} \\ \hline\hline
\textbf{30}  &          &          & \vissp   & \verb|!| & \verb|"| &
            \verb|#| & \verb|$| & \verb|%| & \verb|&| & \verb|'|    \\
\textbf{40}  & \verb|(| & \verb|)| & \verb|*| & \verb|+| & \verb|,| &
            \verb|-| & \verb|.| & \verb|/| & \verb|0| & \verb|1|     \\
\textbf{50}  & \verb|2| & \verb|3| & \verb|4| & \verb|5| & \verb|6| &
            \verb|7| & \verb|8| & \verb|9| & \verb|:| & \verb|;|     \\
\textbf{60}  & \verb|<| & \verb|=| & \verb|>| & \verb|?| & \verb|@| &
            \verb|A| & \verb|B| & \verb|C| & \verb|D| & \verb|E|     \\
\textbf{70}  & \verb|F| & \verb|G| & \verb|H| & \verb|I| & \verb|J| &
            \verb|K| & \verb|L| & \verb|M| & \verb|N| & \verb|O|     \\
\textbf{80}  & \verb|P| & \verb|Q| & \verb|R| & \verb|S| & \verb|T| &
            \verb|U| & \verb|V| & \verb|W| & \verb|X| & \verb|Y|     \\
\textbf{90}  & \verb|Z| & \verb|[| & \verb|\| & \verb|]| & \verb|^| &
            \verb|_| & \verb|`| & \verb|a| & \verb|b| & \verb|c|     \\
\textbf{100} & \verb|d| & \verb|e| & \verb|f| & \verb|g| & \verb|h| &
            \verb|i| & \verb|j| & \verb|k| & \verb|l| & \verb|m|     \\
\textbf{110} & \verb|n| & \verb|o| & \verb|p| & \verb|q| & \verb|r| &
            \verb|s| & \verb|t| & \verb|u| & \verb|v| & \verb|w|     \\
\textbf{120} & \verb|x| & \verb|y| & \verb|z| & \verb|{| & \verb/|/ &
            \verb|}| & \verb|~| &          &          &              \\ \hline
\end{tabular}
\end{table}

\sssclause{Entry} \index{genindex@GenIndex!input!entry} 
\index{(64@{\verb?@?}}

    \verb|Entry| is optional and is introduced by the \verb|@| character. If present, this will form the indexed string.
If absent, then the value of \verb|Key| is used instead.

\sssclause{See also} \index{genindex@GenIndex!input!see also}
 \index{(43@{\verb?+?}}

    \verb|SeeAlso| is optional and is introduced by the \verb|+| character. 
Within a document, it should only be used once
per entry. If used more than once, then only the last value is taken.

    This is used to produce an entry that refers to another indexed entry in
a \emph{see also \ldots} style.

\sssclause{See} \index{genindex@GenIndex!input!see} 
\index{(42@{\verb?*?}}

    \verb|See| is optional and is introduced by the \verb|*| character. 
Within a document, it should be used only once per
entry. If used more than once, then only the last value is taken.

    This is used to produce an entry that refers to another entry in a
\emph{see \ldots} style. Use of this option prohibits printing page numbers 
for the entry.

\ssclause{Output} \index{genindex@GenIndex!output}

    The program GenIndex sorts the entries into alphabetical order (based on
the \verb|Key| values), and produces a \verb|theindex| for the input data.
Several formatting commands are included in the output file to enable
adjustment of the appearance of the printed index. These are shown in
\tref{tab:indexc}.

\begin{table}
\centering
\caption{GenIndex formatting commands} \label{tab:indexc}
\begin{tabular}{|l|l|} \hline
\textbf{Command} & \textbf{Effect} \\ \hline
\verb|\indexfill|\ixcom{indexfill} & 
    spacing style between main topic and page numbers \\
\verb|\sindexfill|\ixcom{sindexfill} & 
    spacing style between subtopic and page numbers \\
\verb|\ssindexfill|\ixcom{ssindexfill} & 
    spacing between sub-subtopic and page numbers \\
\verb|\indexsee{text}|\ixcom{indexsee} & 
    produces \verb|text| as a \emph{see} entry \\
\verb|\indexseealso{text}|\ixcom{indexseealso} & 
    produces \verb|text| as a \emph{see also} entry \\
\verb|\otherindexspace{c}|\ixcom{otherindexspace} & 
    heading for non-alphabetic entry \\
\verb|\alphaindexspace{c}|\ixcom{alphaindexspace} & 
    heading for an alphabetic entry  \\ \hline
\end{tabular}
\end{table}


\begin{note}The commands in \tref{tab:indexc} are defined in the 
      \file{iso}\ixclass{iso} class file.
\end{note}
\begin{note}If the commands are not defined in your system, 
      then you can define them
      using the \latex{} \verb|\newcommand|\ixcom{newcommand}
      command. On the other hand, if they are defined in your system, 
      you can change them using the 
      \latex{} \verb|\renewcommand|\ixcom{renewcommand} command.
\end{note} % end note


\begin{example}
This input file:
\begin{verbatim}
\indexentry{Freddy+Fred}{27}
\indexentry{Frederick*Fred}{29}
\indexentry{Fred}{42}
\indexentry{Fred}{52}
\indexentry{Fred}{43}
\end{verbatim}
will produce output like:
\begin{verbatim}
\begin{theindex}
\alphaindexspace{F}
\item Fred \indexfill 42--43, 52
\item Frederick \indexsee{Fred}
\item Freddy \indexfill 27 \indexseealso{Fred}
\end{theindex}
\end{verbatim}
\end{example} % end example

\sssclause{Indexfill commands}

    The commands \verb|\indexfill{|\meta{style}\verb|}|\ixcom{indexfill}, 
\verb|\sindexfill{|\meta{style}\verb|}|\ixcom{sindexfill} and 
\verb|\ssindexfill{|\meta{style}\verb|}|\ixcom{ssindexfill}
control the style of spacing between an indexed entry and its corresponding
page numbers.

\begin{example}
If you want to have the page numbers right justified, with lines between the
main topics and their numbers, dots between sub-topics and pages, and 
sub-subtopic numbers right justified, then you could define these commands as:
\begin{verbatim}
\newcommand{\indexfill}{\hrulefill}
\newcommand{\sindexfill}{\dotfill}
\newcommand{\ssindexfill}{\hfill}
\end{verbatim}
\end{example} % end example


\begin{anote}The usual \latex{} style is to have a small gap between the
topic and page number, the whole being set ragged right.
If this is what you want, then define the commands as:
\begin{verbatim}
\newcommand{\indexfill}{}
\newcommand{\sindexfill}{}
\newcommand{\ssindexfill}{}
\end{verbatim}
\end{anote} % end note

\sssclause{Indexsee and indexseealso commands}

    The \verb?\indexsee{?\meta{text}\verb?}?\ixcom{indexsee} and 
\verb?\indexseealso{?\meta{text}\verb?}?\ixcom{indexseealso} commands 
control the appearance
of the \emph{see} and \emph{see also} text. 


\begin{example}
These commands could be defined as:
\begin{verbatim}
\newcommand{\indexsee}[1]{\par \hspace*{2em} \emph{see} #1}
\newcommand{\indexseealso}[1]{\par \hspace*{2em} \emph{see also} #1}
\end{verbatim}
\end{example} % end example

\sssclause{Index space commands}
\ixcom{alphaindexspace}
\ixcom{otherindexspace}

    The commands 
\verb?\alphaindexspace{?\meta{c}\verb?}?\ixcom{alphaindexspace} and 
\verb?\otherindexspace{?\meta{c}\verb?}?\ixcom{otherindexspace}
control the amount of space between blocks of index entries. 
These are an extension of the
\latex{} \verb|\indexspace|\ixcom{indexspace}
command, which just inserts some vertical space into the index listing.
Both these commands take a single parameter, which is typically a single
character.

\begin{anote}
GenIndex looks at the first character (call it \verb|c|) of the \verb|Key| 
and if this changes
as it processes the ordered list of main topics, it puts that character
as the parameter for these commands. It writes \verb|\alphaindexspace{c}| if
the character is alphabetic (and \verb|c| is put into upper-case), 
otherwise it writes \verb|\otherindexspace{c}|.
\end{anote} % end note

\begin{example}
These commands could be defined as:
\begin{verbatim}
\newcommand{\alphaindexspace}[1]{\indexspace
                       {\bfseries #1}}
\newcommand{\otherindexspace}[1]{}
\end{verbatim}
which would result in the printing of a vertical space and a bold font text
for an alphabetic header, or nothing for anything else.
\end{example} % end example

\ssclause{Running GenIndex} \index{genindex@GenIndex!run}

    To run GenIndex, first obtain a copy of the program, and compile it if
necessary. Execute the program on your data.

    GenIndex takes zero, one or two file names as parameters. If no files are
given then input and output is from and to \verb|stdin| and \verb|stdout|
respectively.

    If one file is given, then input is taken from this file, and output
is to \verb|stdout|.

    If two file names are given, then input is taken from the first and output
is to the second.

\begin{example}
A typical usage would be:
\begin{verbatim}
GenIndex fred.idx fredidx.tex
\end{verbatim}
which would read the \file{fred.idx} file and write the resulting index data to
file \file{fredidx.tex}.
\end{example} % end example

\end{comment}

% sgmlannx.tex    latex and SGML

\infannex{LaTeX, the Web, and *ML} \label{anx:sgml} \index{SGML}

    ISO are becoming more interested in electronic sources for their
standards as well as the traditional camera-ready copy. Acronyms like
PDF, HTML, SGML and XML have been bandied about. Fortunately documents
written using \latex{} are well placed to be provided in a variety of 
electronic formats. A comprehensive treatment of \latex{} with respect
to this topic is provided by Goossens and Rahtz~\bref{lwebcom}.

    SGML (Standard Generalized Markup Language) is a document tagging 
language that is described in ISO~8879~\bref{sgml} and whose usage is described 
in~\bref{bryan}, among others. The principal
mover behind SGML is Charles Goldfarb from IBM, who has authored a detailed 
handbook~\bref{goldfarb} on the SGML standard.

    The concepts lying behind both \latex{} and SGML are similar, but on the face
of it they are distinctly different in both syntax and capabilities. ISO is
migrating towards electronic versions of its standard documents and, naturally, 
would prefer these to be SGML tagged. 
     Like \latex, SGML has a
concept of style files, which are termed DTDs, and both systems support
powerful macro-like capabilities. SGML provides for logical document
markup and not typesetting --- commercial SGML systems often use
\TeX{} or \latex{} as their printing engine, as does the NIST SGML
environment for ISO~10303~\bref{pandl}.



NIST have SGML tagged some ISO~10303 documents 
using manual methods, which are time consuming and expensive. 
About 1997 there was a NIST 
effort underway to develop an auto-tagger that would (semi-) automatically convert
a \latex{} tagged document to one with SGML tags. This tool assumed a
fixed set of \latex{} macros and a fixed DTD.
 The design of an auto-tagger
essentially boils down to being able to convert from a source document tagged
according to a \latex{} style file to one which is tagged according to an
SGML DTD.
    Fully automatic conversion is really only possible if the authors'
of the documents to be translated avoid using any `non-standard' macros within
their documents. There is a program called \file{ltx2x}\index{ltx2x} available
from SOLIS, which replaces \latex{} commands within a document with
user-defined text strings~\bref{ltx2x}. This can be used as a basis for
a \latex{} to whatever auto-tagger, provided the \latex{} commands are not
too exotic.

    HTML is a simple markup language, based on SGML, and is used for the
publication of many documents on the Web. XML is a subset of SGML and appears
to being taken up by every man and his dog as \emph{the} document markup
language. HTML is being recast in terms of XML instead of SGML. PDF is a page
description language that is a popular format for display of documents 
on the Web.

    \latex{} documents can be output in PDF by using pdfLaTeX. Instead
of a \file{.dvi} file being produced a \file{.pdf} file is output directly.
The best 
results are obtained when PostScript fonts rather than Knuth's cm fonts 
are used. Noting that the \file{iso} class provides an \verb|\ifpdf| command,
a general form for documents to be processed by either \latex{} or pdfLaTeX
is
\begin{verbatim}
\documentclass{isov2}
\usepackage{times}     % PostScript fonts Times, Courier, Helvetica
\ifpdf
  \pdfoutput=1         % request PDF output
  \usepackage[pdftex]{graphicx}
\else
  \usepackage{graphicx}
\fi
...
\end{verbatim}

    There are several converters available to transform a \latex{} document 
into an HTML document, but like \file{ltx2x} they generally do their own
parsing of the source file, and unlike \file{ltx2x} are typically limited
to only generating HTML. Eitan Gurari's \file{TeX4ht}\index{TeX4ht} 
suite is a notable
exception (see Chapter~4 and Appendix~B of~\bref{lwebcom}). It uses the 
\file{.dvi} file as input, so that all the parsing is done by \TeX, and can be
configured to generate a wide variety of output formats.
A set of \file{TeX4ht} configuration files are available for converting
ISO \latex{} documents into HTML\footnote{Later, configuration files for XML
output will be developed.}.

    Some points to watch when writing \latex{} documents that will assist
in translations into *ML are given below. Typically, attention to these points
will make it easier to parse the \latex{} source.

\begin{itemize}
\item Avoid using the \verb|\label|\ixcom{label} command within
      clause headings or captions. It can just as easily be placed immediately
      after these constructs.
\item Avoid using the \verb|\index|\ixcom{index} command within
      clause headings or captions. It can just as easily be placed immediately
      after these constructs.
\end{itemize}






\infannex{Obtaining LaTeX and friends} \label{anx:getstuff}

    \latex{} is a freely available document typesetting system. There are many
public domain additions to the basic system.
The information below gives pointers to where
you can obtain \latex{} etc., from the\index{Internet} Internet. 

    \latex{} runs on a wide variety of hardware, from PCs to Crays. 
Source to build a \latex{} system is freely available via anonymous 
ftp\index{ftp} from what is called CTAN\index{CTAN} 
(Comprehensive \tex{} Archive Network). 
There are three sites; pick the one nearest to you.
\begin{itemize}
\item \url{ftp.dante.de} CTAN in Germany;
\item \url{ftp.tex.ac.uk} CTAN in the UK;
\item \url{ctan.tug.org} CTAN in the USA;
\end{itemize}
The top level CTAN directory for \latex{} and friends is 
\url{/tex-archive}. CTAN contains a wide variety
of (La)TeX sources, style files, and software tools and scripts 
to assist in document processing.

\begin{anote}
    CTAN is maintained by the \tex{} Users Group (TUG). Their homepage
\isourl{http://www.tug.org} should be consulted for the current
list of CTAN sites and mirrors.
\end{anote}

\begin{comment}

\sclause{SOLIS} \index{SOLIS} \label{anx:solis}

    SOLIS is the \emph{SC4 On Line Information Service}. It contains many electronic
sources of STEP related documents. The relevant top level directory is
\url{pub/subject/sc4}.
 In particular, SOLIS contains the source for this document
and the \file{.sty} files, as well as other \latex{} related files. 
The \latex{} root directory is \url{sc4/editing/latex}. The latest
versions of the \latex{} related files are kept in the sub-directory 
\url{latex/current}.
Some \latex{} related programs are also available in the 
\url{latex/programs} sub-directory.

    SOLIS can be reached at \isourl{http://www.nist.gov/sc4}.

\end{comment}


\infannex{Changes in this release} \label{anx:changes}

    Many of the commands and environments have been redefined in order
to match the change in requirements from the the third to the fourth
edition of the ISO Directives. Usage of these is unaffected.

    The following changes have been made in this release:
\begin{itemize}
\item The \verb|cover|\ixenv{cover} environment has been added;

\item The boolean test \verb|\ifpdf|\ixcom{ifpdf} has been added;

\item The command \verb|\fwdnopatents|\ixcom{fwdnopatents} has been added
      for patent boilerplate in the Foreword.

\item The command \verb|\tpasfwdbp|\ixcom{tpasfwdbp} has been added
      for TS/PAS Foreword boilerplate.

\item The command \verb|\intropatents|\ixcom{intropatents} has been added
      for patent boilerplate in the Introduction.

\item The commands \verb|\pref|\ixcom{pref} and 
      \verb|\pagerefname|\ixcom{pagerefname} have been added.

\item The command \verb|\trwfwdbpi|\ixcom{trfwdbpi} for Foreword boilerplate
      in a Technical Report has been deleted.

\item The \verb|notes|\ixenv{notes} and \verb|examples|\ixenv{examples}
      environments have been deleted.

\item Support is provided for the \file{hyperref}\ixpack{hyperref}
      package.

\end{itemize}



\bibannex
%\label{biblio}

\begin{references}
\reference{LAMPORT, L.,}{LaTeX --- A Document Preparation System,}
            {Addison-Wesley Publishing Co., 2nd edition, 1994.} \label{lamport}
\reference{WILSON, P.R.,}{LaTeX files for typesetting ISO standards:
           Source code,}
           {NISTIR,  National Institute of Standards and Technology,
           Gaithersburg, MD 20899. June 1996.} \label{isoe}
\reference{WILSON, P.R.,}{LaTeX package files for ISO~10303: User manual,}
           {NISTIR,  National Institute of Standards and Technology,
           Gaithersburg, MD 20899. June 1996.} \label{stepsty}
\reference{RAHTZ, S., and BARROCA, L.,}{A style option for rotated
           objects in \latex,}{ TUGBoat, volume 13, number 2, pp 156--180, 
           July 1992.} \label{rahtz}
\reference{GOOSSENS, M., MITTELBACH, F. and SAMARIN, A.,}{%
           The LaTeX Companion,}
           {Addison-Wesley Publishing Co., 1994.} \label{goosens}
\reference{GOOSSENS, M., and RAHTZ, S.,}{%
           The LaTeX Web Companion --- Integrating TeX, HTML and XML,}
           {Addison-Wesley Publishing Co., 1999.} \label{lwebcom}
\reference{CHEN, P. and HARRISON, M.A.,}{Index preparation and
           processing,}{Software--Practice and Experience, 19(9):897--915,
           September 1988.} \label{chen}
%\reference{KOPKA, H. and DALY, P.W.,}{A Guide to LaTeX,}
%           {Addison-Wesley Publishing Co., 1993.} \label{kopka}
%\reference{WALSH, N.,}{Making TeX Work,}{O'Reilly \& Associates, Inc.,
%           103 Morris Street, Suite A, Sebastopol, CA 95472. 1994. } \label{walsh}
\reference{ISO 8879:1986,}{Information processing --- 
                                Text and office systems ---
           Standard Generalized Markup Language (SGML).}{} \label{sgml}
\reference{GOLDFARB, C.F.,}{The SGML Handbook,}
           {Oxford University Press, 1990.} \label{goldfarb}
\reference{BRYAN, M.,}{SGML --- An Author's Guide to the Standard Generalized
           Markup Language,}{Addison-Wesley Publishing Co., 1988. } \label{bryan}
\reference{PHILLIPS, L. and LUBELL, J.,}{An SGML Environment for STEP,}%
          {NISTIR 5515, National Institute of Standards and Technology,
           Gaithersburg, MD 20899. November 1994.} \label{pandl}
\reference{WILSON, P. R.,}{LTX2X: A LaTeX to X Auto-tagger,}%
          {NISTIR, National Institute of Standards and Technology,
           Gaithersburg, MD 20899. June 1996.} \label{ltx2x}
\begin{comment}
\reference{RESSLER, S.,}{The National PDES Testbed Mail Server User's Guide,}
           {NSTIR 4508, National Institute of Standards and Technology,
           Gaithersburg, MD 20899. January 1991.} \label{ressler}
\reference{RINAUDOT, G.R.,}{STEP On Line Information Service (SOLIS),}
          {NISTIR 5511, National Institute of Standards and Technology,
          Gaithersburg, MD 20899. October 1994. } \label{rinaudot}
\reference{KROL, E.,}{The Whole Internet --- User's Guide \& Catalog,}
           {O'Reilly \& Associates, Inc.,
           103 Morris Street, Suite A, Sebastopol, CA 95472. 1993. } \label{krol}
\end{comment}
\reference{WILSON, P.R.,}{The hyphenat package,}%
          {1999. (Available from CTAN)} \label{bib:hyphenat}
\reference{WILSON, P.R.,}{The xtab package,}%
          {1998. (Available from CTAN)} \label{bib:xtab}
\end{references}

    
%%%%%%% here is the index at the end
%%\input{isomanidx}
% isoman.tex   User guide for ISO style files for LaTeX(2e) August 2002
%
\documentclass[wd,letterpaper,copyright]{isov2}
%\documentclass[wd,letterpaper,draft]{isov2}
%\documentclass{isov2}

%%%\usepackage[isoman]{tex4ht}

\usepackage{comment}

%%%\usepackage{times}
\usepackage{isorot}
\usepackage{xtab}
\usepackage{hyphenat}

\ifpdf
  \pdfoutput=1
  \usepackage[plainpages=false,
              pdfpagelabels,
              bookmarksnumbered,
              hyperindex=true
             ]{hyperref}
\fi

\standard{LaTeX for ISO standards}
\yearofedition{2002}
\languageofedition{(E)}

%\partno{3456}
\changemarkstrue

\makeindex

%  Uncomment the following to change the Foreword heading
%\renewcommand{\forewordname}{Avant-propos} % change the Foreword title

\setcounter{tocdepth}{3} % add more levels to table of contents
%
%    remainder of preamble is some special macro definitions
\makeatletter
%   the \meta{} command
%
\begingroup
\obeyspaces%
\catcode`\^^M\active%
\gdef\meta{\begingroup\obeyspaces\catcode`\^^M\active%
\let^^M\do@space\let \do@space%
\def\-{\egroup\discretionary{-}{}{}\hbox\bgroup\it}%
\m@ta}%
\endgroup
\def\m@ta#1{\leavevmode\hbox\bgroup\texttt{<}\textit{#1}\/\texttt{>}\egroup
    \endgroup}
\def\do@space{\egroup\space
    \hbox\bgroup\it\futurelet\next\sp@ce}
\def\sp@ce{\ifx\next\do@space\expandafter\sp@@ce\fi}
\def\sp@@ce#1{\futurelet\next\sp@ce}
%
%
\makeatother
%
% the \latex command
\newcommand{\latex}{LaTeX}
\newcommand{\tex}{TeX}
%
%   the \file{} command
%
\newcommand{\file}[1]{\textsf{#1}}
%
\makeatletter
%   index a command
\newcommand{\bs}{\symbol{'134}}
\newcommand{\ixcom}[1]{\index{#1/ @{\tt \protect\bs #1}}}
%   index an environment
\newcommand{\ixenv}[1]{\index{#1 @{\tt #1} (environment)}}
%   index a starred environment
\newcommand{\ixenvs}[1]{\index{#1s @{\tt #1*} (environment)}}
%   index an option
\newcommand{\ixopt}[1]{\index{#1 @{\tt #1} (option)}}
%   index a package
\newcommand{\ixpack}[1]{\index{#1 @\file{#1} (package)}}
%   index a class
\newcommand{\ixclass}[1]{\index{#1 @\file{#1} (class)}}
%   index in typewriter font
\newcommand{\ixtt}[1]{\index{#1@{\tt #1}}}
%   index LaTeX
\newcommand{\ixltx}{\index{latex@\latex}}
%   index LaTeX 2e
\newcommand{\ixltxe}{\index{latex2e@\latex 2e}}
%   index LaTeX v2.09
\newcommand{\ixltxv}{\index{latex209@\latex{} v2.09}}
\makeatother
%
%   end of preamble
%
\begin{document}


\begin{cover}
\vspace*{4in}
\begin{center}
\Huge\bfseries LaTeX for ISO standards
\end{center}
\begin{center}
\bfseries 2002/08/10
\end{center}
\begin{center}
Peter Wilson \\
\texttt{peter.r.wilson@boeing.com}
\end{center}
\clearpage
\end{cover}
%%%\clearpage

\begin{foreword}
\fwdbp
\fwdnopatents
Annexes~\ref{anx:extraiso} and~\ref{anx:lord}
are an integral part of this document.
Annexes~\ref{anx:indexing}, \ref{anx:sgml}, \ref{anx:getstuff},
and \ref{anx:changes}
are for information only.
\end{foreword}


\begin{introduction}

    This document describes the use of the \file{isov2} \latex{} class 
file and some package files in preparing ISO standard documents.

\sclause*{Overview}

    This document describes a \latex{} class file, called \file{isov2},
 for typesetting ISO standards. It also
provides descriptions of some particular package files (e.g., \file{isorot}) 
that have been developed to support the writing of ISO standards.
The electronic source of this  document also provides an example of the
use of these files.



    The current set of files~\bref{isoe} have been developed by 
Peter Wilson (RPI, CUA and NIST, Boeing)
from files written by Kent Reed (NIST).
In turn, these were revisions of files originally created by 
Phil Spiby (CADDETC, UK), based on early work by 
Phil Kennicott (GE).\footnote{In mid 1994 \latex{} was upgraded from 
version 2.09 to what is called \latex 2e. The files described in 
earlier versions this document were compatible with both versions 
of \latex. Starting with the October 1997 release, 
support was withdrawn from any \latex{} v2.09 file versions.}


    Documents produced with the \latex{} files have been reviewed 
by the ISO Editorial Board in Geneva for conformance to their typographical
requirements. The first review was of a series of Draft International 
Standard documents. This review resulted in some changes to the style 
files. The second review was of a series of twelve International Standard 
documents (ISO 10303:1994). Likewise, this review led to changes in the 
style files to bring the documents into compliance.

    With the publication of the ISO~10303:1994 standard, the opportunity 
was taken
to provide a new baseline release of the package files.
The new baseline release was also designed to address the fact that
a major update of \latex{} to \latex 2e took place during 1994. \latex 2e
is now the officially supported version. However, some users needed time
to convert to using \latex 2e. Consequently, although the 
facilities described in the original version of this document 
were common for both 
\latex{} v2.09 and \latex 2e users, there were two different 
implementations.

    Because ISO standard documents have a very structured layout, the class
and package files have been designed to reflect the logical document structure
to a much greater extent than the `standard' \latex{} files. Further, ISO
documents are published in more than one language. The files described
here are written for the English language, but the language-specific elements
have been parameterized for easy modification for publication in other
official ISO languages, such as French.

    In 1997 ISO produced a new version of their Directives on the 
requirements for the layout of ISO documents. These were not completely
unambiguous as to their intent; the current version was released in
2001. Members of ISO TC184/SC4 have worked with
the ISO Editorial Board and have reached an agreement that more precisely
identifies the requirements. The current version of the \latex{} files 
implements that agreement.

\begin{note}
The original of this manual has been typeset using the \verb?draft?
option in order to display its effect of placing a black box at the 
end of each line that is longer than the normal width of the text.
\end{note}

\begin{note}
The author of this document may be contacted at 
\texttt{peter.r.wilson@boeing.com}.
\end{note}

\end{introduction}

\title{LaTeX for ISO standards: User manual}%
{Peter Wilson}%
{August 2002}


\scopeclause

This manual describes a set of \ixltx\latex{} files for typesetting 
documents according to the ISO Directives Part 3 (third edition, 1997), 
together with some elements from the ISO~10303 Supplementary Directives.

\begin{inscope}{manual}
\item use of \latex{} for preparing ISO standard documents.
\end{inscope}

\begin{outofscope}{manual}
\item use of \latex{} in general;
\item use of other document preparation systems.
\end{outofscope}

\textbf{IMPORTANT:} No matter whether or not there are copyright attributions
to ISO, this document is \emph{not} copyrighted by ISO. Any copyright
markings are for illustrative purposes only. This document is released under
the LaTeX Project Public Licence.

\normrefsclause \label{sec:nrefs}

\normrefbp{manual}
\begin{nreferences}

\isref{ISO/IEC Directives, Part 2}{Rules for the structure and drafting of
       International Standards. (Fourth edition, 2001)}

\isref{ISO TC 184/SC4 N1217:2001(E)}{SC4 Supplementary directives --- Rules
       for the structure and drafting of SC4
       standards for industrial data. (2001--11--01).}

\isref{ISO/IEC 8824-1:1998}{Information technology ---
       Abstract Syntax Notation One (ASN.1):
       Specification of basic notation.}

\disref{ISO}{Camera-ready copy (CRC) ---
       Submission requirements and ISO type specification.
       (Version 1.0, 1996-04-26)}

\end{nreferences}

\defabbclause
%\clause{Terms, definitions, and abbreviations}

\defsubclause
%\sclause{Terms and definitions}

    For the purposes of this manual, the following definitions
apply.

\begin{definitions}
\definition{boilerplate}{text whose wording is fixed and has been agreed
 to be present in a specific type of document} \index{boilerplate}
\definition{style file}{a set of \latex{} macros assembled into a single 
   file with an extension \file{.sty}}
            \index{style file}\ixltx\index{file!.sty}
\definition{package file}{a style file for use with \latex 2e}\ixltxe
\end{definitions}

\abbsubclause
%\sclause{Abbreviations}

    For the purposes of this manual, the following abbreviations
 apply.

\begin{symbols}
\symboldef{CD}{Committee Draft}\index{CD}
\symboldef{DIS}{Draft International Standard}\index{DIS}
\symboldef{FDIS}{Final Draft International Standard}\index{DIS}
\symboldef{IS}{International Standard}\index{IS}
\symboldef{IS-REVIEW}{The documentation style accepted by the ISO 
    Editorial Board review (September 1994) of twelve IS documents 
    (the initial release of ISO~10303) for compliance with ISO 
    typographical and layout requirements.}\index{IS-REVIEW}
\symboldef{ISOD}{ISO/IEC Directives, Part 2}\index{ISOD}\index{ISO/IEC Directives}
\symboldef{PAS}{Publicly Available Specification}\index{PAS}
\symboldef{SD}{SC4 Supplementary directives}\index{SD}\index{Supplementary directives}
\symboldef{TR}{Technical Report}\index{TR}
\symboldef{TS}{Technical Specification}\index{TS}
\symboldef{WD}{Working Draft}\index{WD}
\symboldef{CRC}{The ISO \emph{Camera-ready copy (CRC)} document}\index{CRC}
\symboldef{ToC}{table of contents}\index{ToC}
\end{symbols}

\clause{Conformance requirements}  \label{sec:iconform}

    The \latex{} macro source files shall not be modified.

    If there is a need to modify the macro definitions then the
modifications shall be defined in a separate \file{.sty}\index{file!.sty}
 file (or files), using the
\latex{} \verb|\renewcommand|\ixcom{renewcommand}
and/or the 
\verb|\renewenvironment|\ixcom{renewenvironment}
commands as appropriate. The resulting \file{.sty} file(s) shall then
be called in within the preamble\index{preamble} portion of the 
document to be typeset.

    Author specified \verb|\label{...}| commands shall not start with
the characters \verb|;i| (semicolon and `i'); definition of labels
starting with these characters is reserved for the maintainer of the
facility files.


\fcandaclause
%\clause{Fundamental concepts and assumptions}

    It is assumed that the reader of this document is familiar with the
\latex{} document preparation system.\ixltx

\begin{note}Reference~\bref{lamport} in the bibliography describes the
      \latex{} system.
\end{note}

    The reader is also assumed to be familiar with the ISO/IEC Directives 
Part~2 (ISOD\index{ISOD}). Agreements reached between the ISO Editorial Board
and ISO TC184/SC4 are documented in the SC4 Supplementary Directives
(SD\index{SD}).

    If there are any discrepancies between the layout and wording of this 
document and the requirements of the ISO/IEC Directives Part~2, 
then the requirements in that document shall be
followed for any ISO standard document.

   The \file{isov2}\ixclass{isov2} class requires the 
\file{url}\ixpack{url} package.

\begin{note}
Additional \latex{} facilities specifically designed for ISO~10303 are
defined and described elsewhere~\bref{stepsty}.
\end{note}

    Because of many revisions over the years to the files described
herein, a naming convention has been adopted for them.
The primary name of the file is suffixed by \file{v\#} or \file{\#}, where
\file{\#} is the version number of the file in question. 
All file primary names have been limited to a 
maximum of eight characters.

\begin{note}
    Table~\ref{tab:curfiles} shows the versions of the files that were
current at the time of publication.
\end{note} % end note
\begin{note}
Starting with the October 1997 release, files that were specific to 
\ixltxv\latex{} v2.09 are no
longer either supported or supplied.
\end{note}

\begin{note}
As of 1999, the \file{uschyp}\ixpack{uschyp} package is no longer supported. 
It has been replaced by the 
\file{hyphenat}\ixpack{hyphenat}\index{hyphenat.sty@\file{hyphenat.sty}}
package.
\end{note}

\ixclass{isov2}\index{isov2.cls@\file{isov2.cls}}
%             \index{isonev11.sty@\file{isonev11.sty}}
\ixopt{9pt}\index{iso9.clo@\file{iso9.clo}}
%           \index{iso9.sty@\file{iso9.sty}}
\ixopt{10pt}\index{iso10.clo@\file{iso10.clo}}
%            \index{iso10.sty@\file{iso10.sty}}
\ixopt{11pt}\index{iso11.clo@\file{iso11.clo}}
%            \index{iso11.sty@\file{iso11.sty}}
\ixpack{isorot}\index{isorot.sty@\file{isorot.sty}}
%               \index{isrotne1.sty@\file{isrotne1.sty}}
\ixpack{askinc}\index{askincv1.sty@\file{askincv1.sty}}
\ixpack{xtab}\index{xtab.sty@\file{xtab.sty}}
\ixpack{uschyp}\index{uschyp.sty@\file{uschyp.sty}}
%               \index{uschypne.sty@\file{uschypne.sty}}

\begin{table}
\centering
\caption{Current file versions} \label{tab:curfiles}
\begin{tabular}{|l|l|} \hline
\textbf{Facility} & \textbf{File}          \\ \hline\hline
\file{iso}     & \file{isov2.cls}    \\
9pt option     & \file{iso9.clo}     \\
10pt option    & \file{iso10.clo}    \\
11pt option    & \file{iso11.clo}    \\
\file{isorot}  & \file{isorot.sty}   \\
\file{askinc}  & \file{askincv1.sty} \\
\file{xtab}    & \file{xtab.sty}   \\
\file{hyphenat}  & \file{hyphenat.sty}   \\ \hline
\end{tabular}
\end{table}


\begin{anexample} At the time of 
publication of this document, any references to \file{iso.cls} should be
read as actually referring to \file{isov2.cls}, and similarly for references
to other files.
\end{anexample} % end example


\begin{note}This document is not intended for publication as a standard, 
although it has been laid out in a 
similar, but not necessarily identical, manner.\end{note} % end note


\clause{The \file{iso} class facility} 

    The \latex{} \file{isov2}\ixclass{isov2} class file
is a general file for use in preparing ISO
standard documents using the \latex{} document preparation system.

    As usual, any \latex{} document has the following structure:
\begin{verbatim}
\documentclass[<list of options>]{isov2}
% preamble goes here
\begin{document}
% document body goes here
\end{document}
\end{verbatim}

\sclause{Options}

    The \file{isov2}\ixclass{isov2} class file supports the following options: 
\begin{itemize}
\item \verb|draft|\ixopt{draft} for a draft document where 
      overfull horizontal boxes are marked, marginal notes are allowed,
      and ISO copyright text is not placed in the document;
\item \verb|final|\ixopt{final} the opposite of \verb|draft|
      (this is the default);
\item \verb|letterpaper|\ixopt{letterpaper} for printing 
      on US letter size paper;
\item \verb|a4paper|\ixopt{a4paper} for printing on A4
      size paper (this is the default);
\item \verb|twocolumn|\ixopt{twocolumn} for two column 
      formatting;
\item \verb|onecolumn|\ixopt{onecolumn} for single column
      printing (this is the default);
\item One of \verb|11pt|, \ixopt{11pt}
      \verb|10pt|, \ixopt{10pt}
      \verb|9pt|\ixopt{9pt} for 11pt, 10pt or 9pt printing
      respectively (the default is \verb|11pt|);
\item \verb|notcopyright|\ixopt{notcopyright} for 
      disabling the printing of copyright notices (this is the default);
\item \verb|copyright|\ixopt{copyright} enables printing
      of copyright notices;
\item \verb|is|\ixopt{is} for International Standard documents;
\item \verb|fdis|\ixopt{fdis} for  Final Draft
      International Standard documents; \index{FDIS}
\item \verb|dis|\ixopt{dis} for  Draft
      International Standard documents; \index{DIS}
\item \verb|cd|\ixopt{cd} for  Committee Draft
      standard documents; \index{CD}
\item \verb|wd|\ixopt{wd} for  Working Draft
      standard documents; \index{WD}
\item \verb|pas|\ixopt{pas} for Publicly Available Specification
      standard documents;
\item \verb|techrep|\ixopt{techrep} for Technical
      Report standard documents; \index{TR}
\item \verb|techspec|\ixopt{techspec} for Technical
      Spefication standard documents; \index{TS}
\item \verb|otherdoc|\ixopt{otherdoc} for documents
      that are not intended to become a standard (this is the default);
%\item \verb|uglycaption|\ixopt{uglycaption} to produce
%      an ugly style of captioning;
\item Any other facilities that are available via packages.
\end{itemize}

    When no options are specified, then the result is 11pt, single column
printing on A4 size paper, without copyright notice and a running header.
That is, by default, the options set are: 
\verb|final|\ixopt{final};
\verb|a4paper|\ixopt{a4paper}; 
\verb|onecolumn|\ixopt{onecolumn};
\verb|11pt|\ixopt{11pt};
\verb|notcopyright|\ixopt{notcopyright}; and
\verb|otherdoc|\ixopt{otherdoc}.


%\begin{note}ISOD\index{ISOD} calls for 9pt double column  printing but the 
%      SD\index{SD} calls for 11pt single column printing.
%      Using 9pt double column printing is awkward if any computer 
%      code has to
%      be typeset in the document. The IS-REVIEW\index{IS-REVIEW} accepted
%      11pt single column layout. \end{note} % end note
%\begin{note}The CRC\index{CRC} states that acceptable founts are: Univers,
%    Helvetica, and Times Roman with the body text in 10pt. 
%    The IS-REVIEW accepted camera-ready copy
%    using Computer Modern 11pt set in single column. \end{note}
\begin{note}The user of the \file{isov2}\ixclass{isov2} class is encouraged 
      to process this document using
      different combinations of the options to gain experience with
      their effects. This printing of the document is typeset using the
      \verb|draft|\ixopt{draft} option.
\end{note} % end note
\begin{note}
    The \verb|otherdoc|\ixopt{otherdoc} option 
was chosen as the default so that the
current stage of ISO standardardization has to be explicitly declared
as an option, and is therefore available to any software package that
might process the document source (e.g., a \latex{} to SGML translator).
\end{note}


\sclause{Sectioning commands}

    Table~\ref{tab:sec} gives the sectioning commands defined for \file{isov2}
class documents.
\ixcom{clause} \ixcom{normannex} \ixcom{infannex} \ixcom{repannex} 
\ixcom{sclause}
\ixcom{ssclause}
\ixcom{sssclause}
\ixcom{ssssclause}
\ixcom{sssssclause}

\begin{table}
\centering
\caption{Sectioning commands.} \label{tab:sec}
\begin{tabular}{|c|l|} \hline
\textbf{Level} & \textbf{Command} \\ \hline\hline
1   & \verb|\clause|, \verb|\normannex|, \verb|\infannex|, \verb|\repannex| \\
2           & \verb|\sclause| \\
3           & \verb|\ssclause| \\
4           & \verb|\sssclause| \\
5           & \verb|\ssssclause| \\
6           & \verb|\sssssclause| \\ \hline
\end{tabular}
\end{table}

%    The \verb|\section|\ixcom{section} command
% is similar to the normal \latex{} \verb|\part|\ixcom{part}
%command. It is only available when the \verb|sect|\ixopt{sect}
%option is specified.

    The \verb|\clause|\ixcom{clause} commands are similar to 
the normal \latex{} \verb|\section|ing 
commands. There are also starred versions of these commands 
(e.g., \verb|\clause*|).

    Three varieties of annex\index{annex} commands are available:
\begin{itemize}
\item \verb+\normannex{+\meta{title}\verb+}+\ixcom{normannex}
      for a normative annex;
\item \verb+\infannex{+\meta{title}\verb+}+\ixcom{infannex}
       for an informative annex;
\item \verb+\repannex{+\meta{title}\verb+}+\ixcom{repannex}
      for an annex that is neither normative or
     informative (e.g. an annex in a technical report).
\end{itemize}

    Only \verb|\sclause|\ixcom{sclause} and lower level 
sectioning commands can be used after
an annex\index{annex} command.


\begin{example}
    The command  \verb|\infannex{Technical discussion}|, assuming that this is the first
annex in the document, produces:
    \begin{center}
    \textbf{Annex A}\\
    (informative)

    \textbf{Technical discussion}
    \end{center}
\end{example} % end example
\begin{example}
  The command \verb|\repannex{Title of annex}|, assuming that this is the second
annex in the document, produces:

\begin{center}
  \textbf{Annex B}

  \textbf{Title of annex}
\end{center}
\end{example} % end example


\ssclause{The use of \texttt{tocdepth} and \texttt{secnumdepth}}

    In the standard \latex{} classes the
\verb|tocdepth|\ixtt{tocdepth} and 
\verb|secnumdepth|\ixtt{secnumdepth} counters
are set in the preamble\index{preamble} to respectively control the level at
which clause titles are inserted into a table of contents (ToC)\index{ToC} 
and at which clause numbering ceases.

    In the \file{isov2}\ixclass{isov2} class, 
the values of these can be changed at 
any point in the document. The change lasts until another change is
made to the value. 
\begin{example}
    Assume that in the preamble we have
\begin{verbatim}
\setcounter{secnumdepth}{3}  % number ssclauses and above
\setcounter{tocdepth}{3}     % ToC includes ssclauses and above
\end{verbatim}
and that a certain subclause has subsubclauses that should be numbered but
not put into the table of contents, then we could do:
\begin{verbatim}
...
\sclause{The certain subclause}
\setcounter{tocdepth}{2}
...
\ssclause{Numbered but not in ToC}
...
...
\setcounter{tocdepth}{3}
\sclause{Following subclause}
\end{verbatim}

\end{example}

    It can sometimes be difficult to remember which level number corresponds
to which kind of clause. Accordingly, a set of commands are provided to ease
this task. These commands can only be used after the preamble.
\begin{itemize}
\item \verb|\maxsecnumdepth{|\meta{sec}\verb|}|\ixcom{maxsecnumdepth}
      sets the level at which clauses will be numbered. This command
      should be used before the first sectioning command.
\item \verb|\maxtocdepth{|\meta{sec}\verb|}|\ixcom{maxtocdepth}
      sets the level at which clauses will be put into the ToC.
      This command should be used before the \verb|\tableofcontents| command.
\item \verb|\setsecnumdepth{|\meta{sec}\verb|}|\ixcom{setsecnumdepth}
      sets the current level at which clauses will be numbered. 
      This command can be used anywhere after the preamble.
\item \verb|\settocdepth{|\meta{sec}\verb|}|\ixcom{settocdepth}
      sets the current level at which clauses will put into the ToC. 
      This command can be used anywhere after the preamble.
\end{itemize}
The value of the argument \meta{sec} can be any of the following:
\texttt{clause}, \texttt{sclause}, \ldots, \texttt{sssssclause}.

\begin{example}
Using these commands, the previous example can also be coded as:
\begin{verbatim}
...
\begin{document}
\maxsecnumdepth{ssclause}
\maxtocdepth{ssclause}
...
\sclause{The certain subclause}
\settocdepth{sclause}
...
\ssclause{Numbered but not in ToC}
...
...
\settocdepth{ssclause}
\sclause{Following subclause}
\end{verbatim}

\end{example}



\sclause{LaTeX environments and commands}

    Many of the standard \latex{} environments and commands are available.
In particular, all the normal mathematical typesetting capabilities are
present.
However, there are some additional environments and commands defined.

\ssclause{Lists}

    The standard \latex{} \verb|itemize|,\ixenv{itemize}
\verb|enumerate|\ixenv{enumerate} and \verb|description|\ixenv{description}
environments are provided. The labels in these lists, though, differ from
those normally provided by \latex.

\begin{note}
The ISOD describes only a single level for an itemized list, being marked 
with either an em-dash or a bullet. The SD deprecates the bullet but 
provides four levels, each being marked with an em-dash. These are 
provided in the \file{isov2} class.
\end{note}
\begin{note}
The ISOD allows for two levels of enumerated lists. The SD extends this
to four levels, and these are provided in the \file{isov2} class.
\end{note}

\begin{example}
The list environments provided are shown below as:\ixenv{itemize}
\begin{verbatim}
\begin{itemize}
\item First level itemized element;
  \begin{itemize}
  \item Second level itemized element;
    \begin{itemize}
    \item Third level itemized element;
      \begin{itemize}
      \item Fourth level itemized element.
      \end{itemize}
    \end{itemize}
  \end{itemize}
\end{itemize}
\end{verbatim}

\ixenv{enumerate}
\begin{verbatim}
\begin{enumerate}
\item First level enumerated element;
  \begin{enumerate}
  \item Second level enumerated element;
    \begin{enumerate}
    \item Third level enumerated element;
      \begin{enumerate}
      \item Fourth level enumerated element.
      \end{enumerate}
    \end{enumerate}
  \end{enumerate}
\end{enumerate}
\end{verbatim}

\ixenv{description}
\begin{verbatim}
\begin{description}
\item[Description] a description element. Note that a colon is
                   automatically added to the item label.
\end{description}
\end{verbatim}
and they produce
\begin{itemize}
\item First level itemized element;
  \begin{itemize}
  \item Second level itemized element;
    \begin{itemize}
    \item Third level itemized element;
      \begin{itemize}
      \item Fourth level itemized element.
      \end{itemize}
    \end{itemize}
  \end{itemize}
\end{itemize}

\begin{enumerate}
\item First level enumerated element;
  \begin{enumerate}
  \item Second level enumerated element;
    \begin{enumerate}
    \item Third level enumerated element;
      \begin{enumerate}
      \item Fourth level enumerated element.
      \end{enumerate}
    \end{enumerate}
  \end{enumerate}
\end{enumerate}

\begin{description}
\item[Description] a description element. Note that a colon is
                   automatically added to the item label.
\end{description}

\end{example} % end example

\begin{example}
    Here is a more complex set of lists:

\begin{itemize}
\item First level itemization
  \begin{enumerate}
  \item First level enumeration
    \begin{itemize}
    \item Second level itemization
      \begin{enumerate}
      \item Second level enumeration
        \begin{itemize}
        \item Third level itemization
%          \begin{enumerate}    % At this point the list is too deeply nested.
%          \item Third level enumeration    % For the TeXnophile this is
%          \item 3rd level enumeration      % because the example environment
%         \end{enumerate}                   % is itself a list.
        \item 3rd level itemization
        \end{itemize}
      \item 2nd level enumeration
      \end{enumerate}
    \item 2nd level itemization
    \end{itemize}
  \item 1st level enumeration
  \end{enumerate}
\item 1st level itemization
\end{itemize}

\end{example} % end example

\begin{note}
On occasion, \latex{} objects to a well formed list. Typically, this happens
when a list has only one entry. \index{list!error}
 The \latex{} error message is of the form:
\begin{verbatim}
! Something's wrong--perhaps a missing \item.
\end{verbatim}
Hitting the \verb|<return>| key usually gets \latex{} to run on happily.
\end{note}% end note
\begin{note}
Similarly, you may get the message
\begin{verbatim}
! Too deeply nested
\end{verbatim}
Again, hit \verb|<return>| and processing should continue. However,
the indentation of later lists may not be correct.
\end{note} % end note


\ssclause{Notes}

    
    Two different kinds of environments are provided for typesetting notes.

\sssclause{Numbered notes}

    The environment 
\verb+\begin{note}+ \meta{text} \verb+\end{note}+\ixenv{note}
produces a numbered note whose body consists
of \meta{text}.

\begin{anexample}
The commands
\begin{verbatim}
\begin{note}Numbered note.\end{note}
\end{verbatim}
produce

\begin{note}Numbered note.\end{note}
\end{anexample}

\sssclause{Isolated notes}

    The environment 
\verb+\begin{anote}+ \meta{text} \verb+\end{anote}+\ixenv{anote}
produces an un-numbered note whose body
consists of \meta{text}.

\begin{anexample}
    The commands
\begin{verbatim}
\begin{anote}Plain note.\end{anote}
\end{verbatim}
produce

\begin{anote}Plain note.\end{anote}
\end{anexample} % end example


\ssclause{Examples} \label{ssc:ex}

    Two different kinds of environments are provided for typesetting 
examples.

\sssclause{Numbered examples} \label{cl:numex}

The environment \verb?\begin{example}? \meta{text} \verb?\end{example}?
produces a numbered example whose body consists of \meta{text}.

\begin{anexample}
The commands
\begin{verbatim}
\begin{example}Numbered example. \label{ex:exref} \end{example}
\end{verbatim}
produce

\begin{example}Numbered example. \label{ex:exref} \end{example}
\end{anexample}

\begin{anote}
Numbered examples (and numbered notes) may be identified using the
\verb?\label{...}?\ixcom{label} command, as exhibited in \eref{ex:exref}
in \ref{cl:numex}, and then referred to by the 
\verb?\ref{...}?\ixcom{ref} command.
\end{anote}

\sssclause{Isolated examples}

The environment \verb?\begin{anexample}? \meta{text} \verb?\end{anexample}?
produces an un-numbered example whose body consists of \meta{text}.
\ixenv{anexample}

\begin{anexample}
The commands
\begin{verbatim}
\begin{anexample}Isolated example.\end{anexample}
\end{verbatim}
produce

\begin{anexample}Isolated example.\end{anexample}
\end{anexample}



\ssclause{Bibliographic references}

    Two different kinds of environments are provided for bibliographic 
references. It should be noted that neither of these has anything to do with
BibTeX.\index{BibTeX}

\sssclause{Normative references}

    Normative references are listed in the 
\verb|nreferences|\ixenv{nreferences} environment.

    In this environment, bibliographic entries are denoted by either
\verb+\isref{+\meta{ref}\verb+}{+\meta{title}\verb+}+ or by 
\verb+\disref{+\meta{ref}\verb+}{+\meta{title}\verb+}+. The
\meta{ref} parameter is the number of the standard document and 
the \meta{title} parameter is the title of the standard. 

    Use the \verb|\isref|\ixcom{isref} command for published standards 
and the \verb|\disref|\ixcom{disref}
 command for documents that have not yet been 
finally approved as a standard. This latter command automatically adds a
footnote to the effect  that the document is
to be published.

\begin{example}
The \latex{} source for the references in clause~\ref{sec:nrefs} of 
this document is:
\begin{verbatim}
\begin{nreferences}
\isref{ISO/IEC Directives, Part 2}{Rules for the structure and drafting of
       International Standards. (Fourth edition, 2001)}
...
...
\disref{ISO}{Camera-ready copy (CRC) ---
       Submission requirements and ISO type specification.
       (Version 1.0, 1996-04-26)}
\end{nreferences}
\end{verbatim}
\end{example} % end example


\sssclause{Informative references}

    Informative bibliography elements are listed in the
\verb|references|\ixenv{references} environment. 
Each element in the list is specified as 
\verb+\reference{+\meta{author}\verb+}{+\meta{title}\verb+}{+\meta{publisher}\verb+}+.\ixcom{reference}

\begin{example}
The \latex{} source for the bibliography in one version of this document was:
\begin{verbatim}
\infannex{Bibliography}
\begin{references}
\reference{LAMPORT, L.,}{\latex\/ A Document Preparation System,}%
           {Addison-Wesley Publishing Co., 1986} \label{lamport}
\reference{GOOSSENS, M., MITTELBACH, F. and SAMARIN, A.,}{%
           The \latex\/ Companion,}
           Addison-Wesley Publishing Co., 1994} \label{goosens}
\reference{CHEN, P. and HARRISON, M.A.,}{Index preparation and
           processing,}{Software--Practice and Experience, 19(9):897--915,
           September 1988.} \label{chen}
....
\end{references}

\end{verbatim}
\end{example} % end example

    Informative references may be cited in the text via the \latex{}
\verb|\label|\ixcom{label} and \verb|\ref|\ixcom{ref}
mechanism. Note that \verb|\cite|\ixcom{cite} is not available
for references to bibliographic items. For the purposes of
ISO documents, the command 
\verb+\bref{+\meta{ref}\verb+}+\ixcom{bref}
is supplied which
properly formats a bibliographic reference.

\ssclause{Listing of scope items}

    The \verb|inscope|\ixenv{inscope} and 
\verb|outofscope|\ixenv{outofscope}
environments are provided for itemized listing of elements that
are within and outside the scope of the standard. Each list
element is introduced via the \verb|\item|\ixcom{item} command. 
Some boilerplate
text is also printed that introduces the scope list.

    The environments take one parameter, \meta{text}, that must read
naturally in a sentence of the form: `The following are within/outside
the scope of this \meta{text}:'. The non-parameterized part of this
sentence is specified by the commands 
|\verb|\inscopename|\ixcom{inscopename} and
\verb|\outofscopename|\ixcom{outofscopename}
 respectively for `within' and `outside'.

\begin{example}The following text was printed by the commands shown at the
end of the example.

\begin{inscope}{part of ISO~10303}
\item use of \latex{} for preparing ISO standard documents;
\item use of \latex{} for preparing ISO~10303 documents.
\end{inscope}

\begin{outofscope}{part of ISO~10303}
\item use of \latex{} in general;
\item use of other document preparation systems.
\end{outofscope}

\begin{verbatim}
\begin{inscope}{part of ISO~10303}
\item use of \latex{} for preparing ISO standard documents;
\item use of \latex{} for preparing ISO~10303 documents.
\end{inscope}

\begin{outofscope}{part of ISO~10303}
\item use of \latex{} in general;
\item use of other document preparation systems.
\end{outofscope}
\end{verbatim}

\end{example} % end example

\ssclause{Listing of definitions}

    The \verb|\begin{olddefinitions}| \ldots \verb|\end{olddefinitions}|
\ixenv{olddefinitions} environment is provided for 
listing terms that have been defined within
the normatively referenced documents. Each term in the list is specified as: \\
\verb+\olddefinition{+\meta{phrase}\verb+}{+\meta{supplement}\verb+}+\ixcom{olddefinition}.


    The \verb|\begin{definitions}| \ldots \verb|\end{definitions}|
\ixenv{definitions} environment is provided for listing 
the definitions of terms specific to the
document being written. Each term in the
list is specified as: \ixcom{definition} \\
\verb+\definition{+\meta{phrase}\verb+}{+\meta{definition text}\verb+}+.


\begin{example}A listing of terms defined elsewhere could be specified as:

\begin{verbatim}
\begin{olddefinitions}
\olddefinition{application protocol (AP)}{}
\olddefinition{integrated resource}{}
\end{olddefinitions}
\end{verbatim}
\end{example} % end example
\begin{example}The definition listing earlier in this document was produced by:

\begin{verbatim}
\begin{definitions}
\definition{boilerplate}{text whose wording is fixed ...}
\definition{style file}{a set of \latex{} macros assembled 
                        into a single file}
...
\end{definitions}
\end{verbatim}
\end{example} % end example


\ssclause{Listing of abbreviations}

    The \verb|\begin{symbols}| \ldots \verb|\end{symbols}|\ixenv{symbols}
environment is provided for listing symbols
and abbreviations. Each term (either symbol or abbreviation) in the list is
specified as: \ixcom{symboldef} \\
 \verb+\symboldef{+\meta{symbol}\verb+}{+\meta{definition text}\verb+}+.

\begin{example}The list of abbreviations earlier in this document was specified as:
\begin{verbatim}
\begin{symbols}
\symboldef{DIS}{Draft International Standard}
\symboldef{IS}{International Standard}
...
\end{symbols}
\end{verbatim}
\end{example} % end example

\sclause{Floating bodies}

    \latex{} provides the \verb|figure|\ixenv{figure}
and \verb|table|\ixenv{table} environments.
Captions (produced by the \verb|\caption|\ixcom{caption}
command) increment the figure or
table number and add the caption to the relevant contents listing file.

\index{floats} \index{floats!continuation}

    The command 
\verb+\contcaption{+\meta{text}\verb+}+\ixcom{contcaption}
may be used instead. This command neither increments the number nor adds
anything to the listing files.

\begin{example}
The following code:
\begin{verbatim}
\begin{table}[tbp]
\centering
\caption{Example table in three parts} \label{tab:cont}
\begin{tabular}{|c|c|} \hline
\textbf{Col 1} & \textbf{Col 2} \\ \hline\hline
A & 1 \\
B & 2 \\ \hline
\end{tabular}
\end{table}

\begin{table}[tbp]
\centering
\contcaption{(continued)}
\begin{tabular}{|c|c|} \hline
\textbf{Col 1} & \textbf{Col 2} \\ \hline\hline
C & 3 \\
D & 4 \\ \hline
\end{tabular}
\end{table}

\begin{table}[tbp]
\centering
\contcaption{(concluded)}
\begin{tabular}{|c|c|} \hline
\textbf{Col 1} & \textbf{Col 2} \\ \hline\hline
E & 5 \\
F & 6 \\ \hline
\end{tabular}
\end{table}
\end{verbatim}
produces the three-part \tref{tab:cont}. \end{example} %end example

\begin{table}[tbp]
\centering
\caption{Example table in three parts} \label{tab:cont}
\begin{tabular}{|c|c|} \hline
\textbf{Col 1} & \textbf{Col 2} \\ \hline\hline
A & 1 \\
B & 2 \\ \hline
\end{tabular}
\end{table}

\begin{table}[tbp]
\centering
\contcaption{(continued)}
\begin{tabular}{|c|c|} \hline
\textbf{Col 1} & \textbf{Col 2} \\ \hline\hline
C & 3 \\
D & 4 \\ \hline
\end{tabular}
\end{table}

\begin{table}[tbp]
\centering
\contcaption{(concluded)}
\begin{tabular}{|c|c|} \hline
\textbf{Col 1} & \textbf{Col 2} \\ \hline\hline
E & 5 \\
F & 6 \\ \hline
\end{tabular}
\end{table}

\sclause{Title command}


    The command to produce the title of the standard\ixcom{title}
is 
\verb+\title{+\meta{intro}\verb+}{+\meta{main}\verb+}{+\meta{compl}\verb+}+. 
The command takes three parameters
according to the three elements of the title as specified in the ISO directives.
\begin{enumerate}
\item \meta{intro} The introductory element of the title. This may be empty.
\item \meta{main} The main element of the title.
\item \meta{compl} The complementary element of the title. This may be empty.
\end{enumerate}

\begin{example}
If this were actually meant to be an ISO standard, then the \latex{} source 
for the title of this document might be:
\begin{verbatim}
\title{Industrial automation systems and integration}%
{Product data representation and exchange}%
{Part 3456 : Documentation methods: The LaTeX style
file reference manual}
\end{verbatim}
\end{example} % end example

\begin{anote}There must be a space on either side of the colon separating the
part number and the final part of the title.
\end{anote}% end of note

    Three other commands are used for setting the running header throughout the
document. These shall be placed in the preamble.\index{preamble}

    The command \verb+\standard{+\meta{number of standard}\verb+}+ is used to identify
the standard.\ixcom{standard}

    The command \verb+\yearofedition{+\meta{year}\verb+}+ is used to identify the
\ixcom{yearofedition} year of the edition.

    The command 
\verb+\languageofedition{+\meta{code}\verb+}+\ixcom{languageofedition}
 is used to identify the language of the edition.

\begin{example}
This document is identified by:
\begin{verbatim}
\standard{LaTeX for standards}
\yearofedition{1997}
\languageofedition{(E)}
\end{verbatim}
\end{example} % end of example

    As a convenience the vacuous command \verb+\extrahead+\ixcom{extrahead}
is supplied which will add it's contents, if any, to the header. It may
be used, for example, to add a document number to the header by
\verb+\renewcommand{\extrahead}{Doc number}+. If \verb+\extrahead+ is
modified it must be done in the preamble.

    The \verb|\title|\ixcom{title} command sets the page numbering 
style to be arabic,
starts a new page, numbered 1, and puts the title
at the start of the page. It also puts the appropriate header at the top
of the page, dependending on the particular combination of selected
options.

\begin{anote}
Remember that you have to use the \verb|copyright|\ixopt{copyright}
option to enable printing of copyright symbols and notices.
\end{anote}

\ssclause{The cover}

    For publication, ISO want to be supplied with a document starting
on page iii with the ToC. They supply the cover (page i) and page ii.

It can often be useful to have a cover page for documents while they
are in the process of being prepared for submittal to ISO.
    The \verb+cover+\ixenv{cover} environment is provided for that
purpose. The contents of the environment must not exceed one page and the
contents must have a \verb+\clearpage+\ixcom{clearpage} immediately before
the end. The \verb+cover+ environment also prints page ii, which has a 
copyright notice on it if the document is copyrighted.

    For example, this document starts with:
\begin{verbatim}
\begin{cover}
\vspace*{4in}
\begin{center}
\Huge\bfseries LaTeX for standards
\end{center}
\begin{center}
\bfseries 2001/07/06
\end{center}
\begin{center}
Peter Wilson \\
\texttt{peter.r.wilson@boeing.com}
\end{center}
\clearpage
\end{cover}
\end{verbatim}

\sclause{Cross referencing commands}

    The usual \latex{} \verb+\label+\ixcom{label} and
\verb+\ref+\ixcom{ref} commands are supported. The class also
provides some formatted referencing commands in addition to \verb+\ref+.

    The following commands are useful for referring to figures, clauses etc.
Each takes a parameter that is the parameter of a \latex{} \verb|\label{}|
command.

\begin{itemize}
\item The command \verb|`\aref{anx:lord}'|\ixcom{aref} prints `\aref{anx:lord}' 
      while \verb|`\ref{anx:lord}'|\ixcom{ref}
      prints `\ref{anx:lord}'
\item The command \verb|`\bref{lamport}'|\ixcom{bref} prints `\bref{lamport}'
      while \verb|`\ref{lamport}'|\ixcom{ref}
      prints `\ref{lamport}'
\item The command \verb|`\cref{sec:nrefs}'|\ixcom{cref} prints `\cref{sec:nrefs}'
      while \verb|`\ref{sec:nrefs}'|\ixcom{ref}
      prints `\ref{sec:nrefs}'
\item The command \verb|`\eref{sec:nrefs}'|\ixcom{eref} prints `\eref{sec:nrefs}'
      while \verb|`\ref{sec:nrefs}'|\ixcom{ref}
      prints `\ref{sec:nrefs}'
\item The command \verb|`\fref{sec:nrefs}'|\ixcom{fref} prints `\fref{sec:nrefs}'
      while \verb|`\ref{sec:nrefs}'|\ixcom{ref}
      prints `\ref{sec:nrefs}'
\item The command \verb|`\nref{sec:nrefs}'|\ixcom{nref} prints `\nref{sec:nrefs}'
      while \verb|`\ref{sec:nrefs}'|\ixcom{ref}
      prints `\ref{sec:nrefs}'
\item The command \verb|`\tref{sec:nrefs}'|\ixcom{tref} prints `\tref{sec:nrefs}'
      while \verb|`\ref{sec:nrefs}'|\ixcom{ref}
      prints `\ref{sec:nrefs}'
\item The command \verb|`\pref{sec:nrefs}'|\ixcom{pref} prints
      `\pref{sec:nrefs}'.
\end{itemize}




\sclause{Heading commands}

\ssclause{Foreword}

    The \verb|foreword|\ixenv{foreword} environment specifies
that a table of contents, list of
figures and list of tables be produced, and starts a new unnumbered
clause called Foreword.
Formatting is one column style only and roman page numbering is set. The 
table of contents starts on page iii.

    More precisely, the title of the clause is given by the value of
\verb|\forewordname|\ixcom{forewordname} (see \aref{anx:extraiso}).

\begin{note}
The default style only lists level 1 clauses in the table of contents. If you
need a more detailed listing, then put the command 
\verb|\setcounter{tocdepth}{<level #>}|\ixcom{setcounter}\ixtt{tocdepth}
in the preamble.\index{preamble}
For example \verb|\setcounter{tocdepth}{3}| will produce a contents
listing down to the level of \verb|\ssclause| (see \tref{tab:sec} for
the definitions of clause levels).
\end{note} %end note

    The command \verb+\fwdbp+\ixcom{fwdbp} prints the 
ISO specified boilerplate for
the initial paragraphs of a foreword.

    The boilerplate for a Technical Specification (ISO/TS) or a 
Publicly Available Specification (ISO/PAS) differs from that
for a standard. The command \verb|\tspasfwdbp|\ixcom{tspasfwdbp}
prints some of this boilerplate.
\begin{example}
In this document, \verb|\tspasfwdbp| prints:

\tspasfwdbp
\end{example}
\begin{note}
    The required paragraph immediately following this boilerplate is:

ISO/PAS [or ISO/TS] \ldots was prepared by Technical Committee ISO/TC \ldots
\end{note}

%%    The boilerplate for a Technical Report (TR) also differs from that for
%%a standard.
%%The command \verb|\trfwdbpi|\ixcom{trfwdbpi}
%%prints some of this boilerplate. 
%%\begin{example}
%%In this document, \verb|\trfwdbpi| prints:
%%
%%\trfwdbpi
%%\end{example}

    The \verb|\fwdnopatents|\ixcom{fwdnopatents} 
prints out the Foreword
boilerplate paragraph concerning potential patent rights.
\begin{example}
In this document, \verb|\fwdnopatents| prints:
\fwdnopatents 
\end{example}



\ssclause{Introduction}

    The \verb|introduction|\ixenv{introduction} environment 
starts a new unnumbered clause 
called `Introduction' with one column formatting.

    More precisely, the title of the clause is given by the value of
\verb|\introductionname|\ixcom{introductionname} (see \aref{anx:extraiso}).

    The \verb|\intropatents|\ixcom{intropatents} prints some of the 
patent boilerplate that may be required in the Introduction. 
\begin{anexample}
In this document, \verb|\intropatents| prints:
\fwdnopatents 
\end{anexample}


\ssclause{Scope clause}

    The \verb|\scopeclause|\ixcom{scopeclause} 
command starts a new numbered clause called `Scope', which is given
the label \verb|;i1|\index{;i1}.

    More precisely, the title of the clause is given by the value of
\verb|\scopename|\ixcom{scopename} (see \aref{anx:extraiso}).

\ssclause{Normative references clause}

    The \verb|\normrefsclause|\ixcom{normrefsclause}
command starts a new numbered clause called `Normative references',
which is given the label \verb|;i2|\index{;i2}.

    More precisely, the title of the clause is given by the value of
\verb|\normrefsname|\ixcom{normrefsname} (see \aref{anx:extraiso}).

    The command 
\verb+\normrefbp{+\meta{document identifier}\verb+}+\ixcom{normrefbp} 
prints the
ISO required text for the introduction to the listing of normative
references. The \meta{document identifier} parameter must be such
that it reads sensibly in a sentence of the form `\ldots constitute
provisions of this \meta{document identifier}.'.

\begin{example}Clause~\ref{sec:nrefs} in this document was started by the
commands:
\begin{verbatim}
\normrefsclause \label{sec:nrefs}

\normrefbp{manual}
\begin{nreferences}
...
\end{verbatim}
\end{example} % end example

\ssclause{Definitions, symbols and abbreviations}

    A variety of commands are provided that initiate new numbered
clauses for definitions, symbols and abbreviations. Depending on the
amount of material in these respective categories, one or more clauses
may be used. The commands and clause titles are listed in \tref{tab:dsa}.
The clause level headings are each given the label \verb|;i3|\index{;i3};
one and only one of these headings should appear in a standard document.
\ixcom{defclause} \ixcom{defname}
\ixcom{symclause} \ixcom{symname}
\ixcom{abbclause} \ixcom{abbname}
\ixcom{defsymclause} \ixcom{defsymname}
\ixcom{defabbclause} \ixcom{defabbname}
\ixcom{symabbclause} \ixcom{symabbname}
\ixcom{defsymabbclause} \ixcom{defsymabbname}
\ixcom{defsubclause}
\ixcom{symsubclause}
\ixcom{abbsubclause}
\ixcom{defsymsubclause}
\ixcom{defabbsubclause}
\ixcom{symabbsubclause}

\begin{table}
\centering
\caption{Definition, symbol and abbreviation clause commands}
\label{tab:dsa}
\begin{tabular}{|l|c|l|l|} \hline
\textbf{Command} & \textbf{Clause} & \textbf{Title} & \textbf{Default text} \\ \hline
\verb|\defclause| &       C  & \verb|\defname| & \defname{} \\
\verb|\symclause| &       C  & \verb|\symname| & \symname{}  \\
\verb|\abbclause| &       C  & \verb|\abbname| & \abbname{}  \\
\verb|\defsymclause| &    C  & \verb|\defsymname| & \defsymname{} \\
\verb|\defabbclause| &    C  & \verb|\defabbname| & \defabbname{} \\
\verb|\symabbclause| &    C  & \verb|\symabbname| & \symabbname{} \\
\verb|\defsymabbclause| & C  & \verb|\defsymabbname| & \defsymabbname{} \\
\verb|\defsubclause| &    SC & \verb|\defname| & \defname{} \\
\verb|\symsubclause| &    SC & \verb|\symname| & \symname{}  \\
\verb|\abbsubclause| &    SC & \verb|\abbname| & \abbname{}  \\
\verb|\defsymsubclause| & SC & \verb|\defsymname| & \defsymname{} \\
\verb|\defabbsubclause| & SC & \verb|\defabbname| & \defabbname{} \\
\verb|\symabbsubclause| & SC & \verb|\symabbname| & \symabbname{} \\
 \hline
\multicolumn{4}{|l|}{%
NOTE - In the table, C = clause, SC = subclause. } \\ \hline
\end{tabular}
\end{table}

\sssclause{Heading labels}

    Some of the defined clauses have associated \verb|\label|s. These
heading commands and their \verb|\label| values are listed in 
\tref{tab:clabels}.

\begin{table}
\centering
\caption{Defined clause headings with labels}
\label{tab:clabels}
\begin{tabular}{|l|c|} \hline
\textbf{Command}       &  \textbf{Label} \\ \hline
\verb|\scopeclause|    &  \texttt{;i1} \\
\verb|\normrefsclause| &  \texttt{;i2} \\
\verb|\defclause| &       \texttt{;i3} \\
\verb|\symclause| &       \texttt{;i3} \\
\verb|\abbclause| &       \texttt{;i3} \\
\verb|\defsymclause| &    \texttt{;i3} \\
\verb|\defabbclause| &    \texttt{;i3} \\
\verb|\symabbclause| &    \texttt{;i3} \\
\verb|\defsymabbclause| & \texttt{;i3} \\
 \hline
\end{tabular}
\end{table}



\ssclause{Bibliography}

    The command \verb|\bibannex|\ixcom{bibannex}
 starts an informative section of the document entitled `Bibliography'. 
Or, more precisely, 
by the value of the \verb|\bibname|\ixcom{bibname}
command.

\sclause{Urls, etc}

    The command \verb|\url{|\meta{text}\verb|}|\ixcom{url} can be used for
typesetting \meta{text} as an email address.

    The command \verb|\isourl{|\meta{text}\verb|}|\ixcom{isourl} can be 
used for
typesetting \meta{text} as an URL address.

\begin{example}
The following code
\begin{verbatim}
The Email address is \url{joe@fred.mik} or the homepage is at
\isourl{http://fred.mik/home/}.
\end{verbatim}
will be typeset as: \\
The Email address is \url{joe@fred.mik} or the homepage is at
\isourl{http://fred.mik/home/}.
\end{example}


\sclause{Version control commands}


    A set of commands are provided to assist when revising a document.
For these commands to flag the changes in the printed document the command
\verb|\changemarkstrue|\ixcom{changemarkstrue}
must be put in the preamble.\index{preamble}

    In the commands described below, the \meta{number} parameter can be used 
to correlate changes in a document
to some external (numbered) requirement for the change.

\ssclause{Editorial}

    The command \verb+\editorial{+\meta{number}\verb+}+\ixcom{editorial}
flags an editorial change to the
document text with ED$^{number}$.

\begin{anexample}

   Just to start things off, this is an original sentence, which should
take up about a line.
This example sentence contains an editorial \editorial{37}
change. The command \verb|\editorial{37}| was included in the previous
sentence.
This sentence, though, has no changes marked in it and may be
assumed to be unaltered from a prior version.
\end{anexample} % end example

\ssclause{Added}

    The command 
\verb+\added{+\meta{text}\verb+}{+\meta{number}\verb+}+\ixcom{added}
flags the
position of the additional \meta{text} and highlights it.

\begin{anexample}

   Just to start things off, this is an original sentence, which should
take up about a line.
This example sentence contains \added{some added text}{27} in
the middle of it. The command \verb|\added{some added text}{27}| was
included in the previous sentence.
This sentence, though, has no changes marked in it and may be
assumed to be unaltered from a prior version.
\end{anexample} % end example

\ssclause{Deleted}

    The command \verb+\deleted{+\meta{number}\verb+}+\ixcom{deleted}
flags the position of deleted text.

\begin{anexample}

   Just to start things off, this is an original sentence, which should
take up about a line.
Some text was deleted \deleted{34} from the middle of this sentence.
The command \verb|\deleted{34}| was included in the previous sentence.
This sentence, though, has no changes marked in it and may be
assumed to be unaltered from a prior version.
\end{anexample} % end example

\ssclause{Moved}

    The command 
\verb+\moved{+\meta{text}\verb+}{+\meta{number}\verb+}+\ixcom{moved}
flags the position of
moved \meta{text} and highlights it.

\begin{anexample}

   Just to start things off, this is an original sentence, which should
take up about a line.
This sentence contains some \moved{moved}{81} text in the middle of it.
The command \verb|\moved{moved}{81}| was included in the previous
sentence.
This sentence, though, has no changes marked in it and may be
assumed to be unaltered from a prior version.
\end{anexample} % end example

\sclause{PDF}

    The class provides a command, \verb|\ifpdf|\ixcom{ifpdf}, 
to test whether or not the document is being processed by \latex{} 
or by pdf\latex. \latex{} processing produces a \file{.dvi} file which
needs further processing, usually by \file{dvips}, to generate PostScript
for printing. pdf\latex, however, directly generates a \file{.pdf} file
which can then be printed.

\begin{anexample}
A document for processing by either \latex{} or pdf\latex{} could be 
started like:
\begin{verbatim}
\documentclass{iso}
\usepackage{times}
\ifpdf
  \pdfoutput=1
  \usepackage[pdftex]{graphicx}
\else
  \usepackage{graphicx}
\fi
....
\end{verbatim}

The \file{times} package is used in this example as PDF printers tend
to handle
PostScript fonts better than other kinds of fonts.
\end{anexample}


    The class also supports the use of the 
\file{hyperref}\ixpack{hyperref} package in conjunction with pdf\latex.
Typically bookmark processing would be specifed by:
\begin{verbatim}
\ifpdf
  \pdfoutput=1
  \usepackage[plainpages=false,
              pdfpagelabels,
              bookmarksnumbered,
              hyperindex=true
             ]{hyperref}
  ...
\else
  ...
\fi
....
\end{verbatim}
If the \verb|hyperindex=true| option to the \file{hyperref} package is used
the the index has to be processed via the
\textsc{MakeIndex}\index{makeIndex@{\sc MakeIndex}} program.

%%%%%%%%%%%
%%%\end{document}
%%%%%%%%%%%

\clause{The \file{isorot} package facility} 

   The \file{isorot}\ixpack{isorot} facility enables the rotation 
of document elements on
a page. It uses the \latex{} \verb|\special|\ixcom{special} 
command to perform its
effects, and thus can only be used
with a limited number of dvi to postscript programs. The facilities
available are summarized in \tref{tab4}.

    \file{isorot} is a modification of the \file{rotation.sty} file
created by Rahtz and Barroca~\bref{rahtz}. Further examples of the usage
of their style are given in Goosens \emph{et al}~\bref{goosens}.

\begin{note}Several examples of the effects of the commands described herein
are shown. In many cases the results are not pretty. This should act as
a warning that using rotational elements requires more care than
most other document elements.\end{note} %end note

\begin{sidewaystable}
\ixcom{rotdriver} \ixcom{clockwise} \ixcom{counterclockwise}
\ixcom{figuresright} \ixcom{figuresleft} \ixcom{rotcaption}
\ixcom{controtcaption}
\ixenv{sideways} \ixenv{turn} \ixenv{rotate} \ixenv{sidewaystable}
\ixenv{sidewaysfigure} \ixenv{landscape}
\centering
\caption{The rotation facilities} \label{tab4}
\begin{tabular}{|l|l|} \hline
\textbf{Facility} & \textbf{Effect} \\ \hline
\multicolumn{2}{|c|}{\textbf{Commands}} \\ \hline
\verb|\rotdriver{<driver>}| & 
declare the name of the dvi to Postscript translator (default {\tt dvips}) \\
\verb|\clockwise| & 
sets rotation direction clockwise for positive angles (the default) \\
\verb|\counterclockwise| &
sets rotation direction counterclockwise for positive angles \\
\verb|\figuresright| &
sets rotation direction for sideways floats counterclockwise (the default) \\
\verb|\figuresleft| &
sets rotation direction for sideways floats clockwise \\
\verb|\rotcaption| &
like the \verb|caption| command, but rotates the caption through 90 degrees \\
\verb|\controtcaption| &
like the \verb|contcaption| command, but rotates the caption through 90 degrees \\ \hline
\multicolumn{2}{|c|}{\textbf{Environments}} \\ \hline
\verb|sideways| &
rotates the contents through 90 degrees counterclockwise \\
\verb|turn| &
rotates the contents through the given angle \\
\verb|rotate| &
rotates the contents through the given angle, but no space allowed for the result\\
\verb|sidewaystable| &
like the \verb|table| environment, but rotated 90 degrees \\
\verb|sidewaystable*| &
twocolumn version of \verb|sidewaystable| \\
\verb|sidewaysfigure| &
like the \verb|figure| environment, but rotated 90 degrees \\
\verb|sidewaysfigure*| &
twocolumn version of \verb|sidewaysfigure| \\
\verb|landscape| &
prints all enclosed pages in landscape mode \\ \hline
\end{tabular}
\end{sidewaystable}

\sclause{Options}

    The \file{isorot} facility has one option, 
namely \verb|debugshow|\ixopt{debugshow}. Calling this option produces
messages on the screen and in the \file{log} file regarding the actions 
being taken.

\begin{anote} This option is principally of interest to the maintainer
of the facility. \end{anote}

\sclause{DVI drivers}

    The \file{isorot} facility supports only a limited number of
dvi to postscript translators. The default translator is \emph{dvips}.
 The following command must be put in
the preamble of the document if \emph{dvips} is not being used: 
\verb|\rotdriver{<drivername>}|,\ixcom{rotdriver} where
\verb|<drivername>| is one of the following:\footnote{I have been able to 
try the {\tt dvips} driver
but not the others. If anyone has experience with the other drivers, or has
extended the range of drivers, I would like to be given the results.} %end footnote

\begin{enumerate}
\item \verb|dvipdf| for the \emph{dvipdf} 
      translator;\ixtt{dvipdf}
\item \verb|dvips| for Tom Rockicki's \emph{dvips} 
      translator;\ixtt{dvips}
\item \verb|dvipsone| for Y\&Y's \emph{dvipsone} 
      translator;\ixtt{dvipsone}
\item \verb|dvitops| for James Clark's \emph{dvitops} 
      translator;\ixtt{dvitops}
\item \verb|dviwindo| for Y\&Y's \emph{dviwindo} 
      translator;\ixtt{dviwindo}
\item \verb|pctex32| for Personal TeX's PC TeX for 32 bit Windows 
      (\emph{pctex32})
      translator;\ixtt{pctex32}
\item \verb|pctexps| for Personal TeX's PC PTI Laser/PS (\emph{pctexps})
      translator;\ixtt{pctexps}
\item \verb|pubps| for the Arbortext's \emph{pubps} 
      translator.\ixtt{pubps}
\item \verb|textures| for Blue Sky's \emph{Textures} 
      translator;\ixtt{textures}

\end{enumerate}


\sclause{Rotational directions}

    \file{isorot} enables the textual and other elements of a document
to be rotated from their normal horizontal layout. In some cases elements
can be rotated through arbitrary angles, whereas in others only 90 degree
rotation is possible. 

    By default, a rotation through a positive number of
degrees corresponds to a clockwise rotation. The command 
\verb|\counterclockwise|\ixcom{counterclockwise}
sets the following rotations to be counterclockwise for positive angles.
The command \verb|\clockwise|\ixcom{clockwise}
sets the following rotations to be clockwise for positive angles.
These commands can be used to toggle the rotational behavior.
    

    Rotated floating environments are normally rotated so that they are
printed with a counterclockwise rotation (i.e. the original bottom of the float
is placed at the right hand side of the paper), 
which is what is normally required.
This behavior can be altered by the command
\verb|\figuresleft|,\ixcom{figuresleft}
which will give the reverse effect. The command
\verb|\figuresright|\ixcom{figuresright}
will set the behavior to the default.
These commands can be used to toggle the rotational behavior of
floats.


\sclause{Rotation of text}

    The \verb|sideways|\ixenv{sideways}
environment rotates the contents of the environment
by 90 degrees counterclockwise, and leaves space for the result.

    The \verb|\begin{turn}{|\meta{angle}\verb|}|\ixenv{turn}
environment rotates the contents by the given number
of degrees in the direction specified by the most recent of the
\verb|\clockwise|\ixcom{clockwise} or
\verb|\counterclockwise|\ixcom{counterclockwise}
commands, leaving space for the result.

    The \verb|\begin{rotate}{|\meta{angle}\verb|}|\ixenv{rotate}
environment rotates the contents by the given number
of degrees in the direction specified by the most recent of the
\verb|\clockwise|\ixcom{clockwise} or
\verb|\counterclockwise|\ixcom{counterclockwise}
commands, but no arrangements are made for leaving space for the result.


\begin{example}Some simple rotations: \label{ex:1}

This code
\begin{verbatim}
Default rotation direction: \\
A 
\begin{sideways}%
B C
\end{sideways}
D E F G H I J K L M
\begin{turn}{-90}%
Minus 90 turn
\end{turn}
N O P
\begin{rotate}{90}%
Plus 90 rotate
\end{rotate}
Q \\
and continue on with another line after rotations.
\end{verbatim}
produces the following (note how space is allowed for the \verb|turn|ed 
text, whereas the \verb|rotate|d text runs into the text below).

Default rotation direction: \\
A 
\begin{sideways}%
B C
\end{sideways}
D E F G H I J K L M
\begin{turn}{-90}%
Minus 90 turn
\end{turn}
N O P
\begin{rotate}{90}%
Plus 90 rotate
\end{rotate}
Q \\
and continue on with another line after rotations.
\end{example} % end example
\begin{example}This example shows the effect of using the 
\verb|\counterclockwise|\ixcom{counterclockwise}
command.

This code
\begin{verbatim}
Flip rotation direction: \\
\counterclockwise
A 
\begin{sideways}%
B C
\end{sideways}
D E F G H I J K L M
\begin{turn}{-90}%
Minus 90 turn
\end{turn}
N O P
\begin{rotate}{90}%
Plus 90 rotate
\end{rotate}
Q \\
Set rotation direction back to default value.
\clockwise
\end{verbatim}

produces the following, which should be compared with example~\ref{ex:1}.

Flip rotation direction: \\
\counterclockwise
A 
\begin{sideways}%
B C
\end{sideways}
D E F G H I J K L M
\begin{turn}{-90}%
Minus 90 turn
\end{turn}
N O P
\begin{rotate}{90}%
Plus 90 rotate
\end{rotate}
Q \\
Set rotation direction back to default value.
\clockwise
\end{example} % end example


    Although the examples so far have only shown the rotation of text, boxes
can also be rotated.

\begin{example}Rotating a box.

This code
\begin{verbatim}
\newsavebox{\foo}
\newlength{\fool}
\settowidth{\fool}{Hurrah for ISO.}
\savebox{\foo}{\parbox{\fool}{Hurrah for ISO. Hurrah for ISO.
                              Hurrah for ISO. Hurrah for ISO.}}
Start
\usebox{\foo}
\&
\begin{turn}{-45}\usebox{\foo}\end{turn}
\&
\begin{turn}{45}\usebox{\foo}\end{turn}
End
\end{verbatim}
produces:

\newsavebox{\foo}
\newlength{\fool}
\settowidth{\fool}{Hurrah for ISO.}
\savebox{\foo}{\parbox{\fool}{Hurrah for ISO. Hurrah for ISO.
                              Hurrah for ISO. Hurrah for ISO.}}
Start
\usebox{\foo}
\&
\begin{turn}{-45}\usebox{\foo}\end{turn}
\&
\begin{turn}{45}\usebox{\foo}\end{turn}
End

\end{example} %end example

    Elements can be rotated through arbitrary angles, and also rotated
elements can be nested inside other rotated elements.


\begin{example}Repeated rotation:

    The following example code shows that text can be rotated through any angle.
The result is shown in \fref{fig:wheel}.
\begin{verbatim}
\newcount\prwc
\newsavebox{\prwtext}
\newdimen\prwspace
\def\wheel#1#2{%
  \savebox{\prwtext}{#1\begin{sideways}#2\end{sideways}}%
  \prwspace\wd\prwtext%
  \advance\prwspace by 1cm%
  \centerline{%
  \rule{0pt}{\prwspace}%
  \rule[-\prwspace]{0pt}{\prwspace}%
  \prwc=-180\loop\ifnum\prwc<180
  \rlap{\begin{rotate}{\the\prwc}%
  \rule{1cm}{0pt}\usebox{\prwtext}\end{rotate}}%
  \advance\prwc by 20\repeat}}
\begin{figure}
\wheel{Express yourself ---}{Hooray for STEP!}
\caption{Example rotation through multiple angles}
\label{fig:wheel}
\end{figure}
\end{verbatim}

\newcount\prwc
\newsavebox{\prwtext}
\newdimen\prwspace
\def\wheel#1#2{%
  \savebox{\prwtext}{#1\begin{sideways}#2\end{sideways}}%
  \prwspace\wd\prwtext%
  \advance\prwspace by 1cm%
  \centerline{%
  \rule{0pt}{\prwspace}%
  \rule[-\prwspace]{0pt}{\prwspace}%
  \prwc=-180\loop\ifnum\prwc<180
  \rlap{\begin{rotate}{\the\prwc}%
  \rule{1cm}{0pt}\usebox{\prwtext}\end{rotate}}%
  \advance\prwc by 20\repeat}}
\begin{figure}
\vspace*{1cm}
\wheel{Express yourself ---}{Hooray for STEP!}
\vspace*{1cm}
\caption{Example rotation through multiple angles}
\label{fig:wheel}
\end{figure}

    Figures~\ref{fig:angles1} and~\ref{fig:angles2} also show rotations through a
range of angles, both positive and negative.
\end{example} %end example

\begin{example}Nested rotations. \label{ex:sidetabular}

This code
\begin{verbatim}
    Here is some text before a \verb|sideways| environment. 
And some more, and more and more garble gobble cluck
click clack clock cluck and so on and on and on.
\begin{center}
\begin{sideways}
\rule{1in}{0pt}
\begin{tabular}{|lr|}
\begin{rotate}{-45}\emph{Word}\end{rotate} & \begin{rotate}{-90}%
Occurrences\end{rotate}
\\
\hline
hello & 33 \\
goodbye & 34 \\
\hline
\end{tabular}
\end{sideways}
\end{center}
    Here is some text after a \verb|sideways| environment.
 And some more, and more and more garble gobble cluck
click clack clock cluck and so on and on and on.
\end{verbatim}
produces:

    Here is some text before a \verb|sideways| environment. 
And some more, and more and more garble gobble cluck
click clack clock cluck and so on and on and on.
\begin{center}
\begin{sideways}
%\rule{1in}{0pt}
\begin{tabular}{|lr|}
\begin{rotate}{-45}\emph{Word}\end{rotate} & \begin{rotate}{-90}%
Occurrences\end{rotate} \\ \hline
hello & 33 \\
goodbye & 34 \\ \hline
\end{tabular}
\end{sideways}
\end{center}
    Here is some text after a \verb|sideways| environment.
 And some more, and more and more garble gobble cluck
click clack clock cluck and so on and on and on.
\end{example} %end example



\sclause{Rotations of tables and figures}

    The previous examples have demonstrated the rotation of textual elements.
For instance, example~\ref{ex:sidetabular} shows that tabular material can be rotated using
the \verb|sideways|\ixenv{sideways}
environment. (Actually, any of the previously
mentioned environments may be used.)
Two further environments are provided which rotate a \latex{} float through
90 degrees. These are:
\begin{itemize}
\item \verb|sidewaystable|\ixenv{sidewaystable}, which
  corresponds to the standard \latex{} \verb|table|\ixenv{table}
  environment; and
\item \verb|sidewaysfigure|\ixenv{sidewaysfigure}, which
  corresponds to the standard \latex{} \verb|figure|\ixenv{figure}
  environment.
\end{itemize}
There are also starred versions of these, namely 
\verb|sidewaystable*|\ixenvs{sidewaystable} and
\verb|sidewaysfigure*|\ixenvs{sidewaysfigure}, for use in twocolumn mode.
However, the correspondence with the standard environments is not strictly
complete as a sideways float is alway placed on a page by itself.

    The direction of rotation may be controlled by the 
\verb|\figuresright|\ixcom{figuresright} and
\verb|\figuresleft|\ixcom{figuresleft} commands.

\begin{example}Table~\ref{tab4} is produced by the code below: \label{ex:4}

\begin{verbatim}
\begin{sidewaystable}
\centering
\caption{The rotation facilities} \label{tab4}
\begin{tabular}{|l|l|} \hline
\textbf{Facility} & \textbf{Effect} \\ \hline
\multicolumn{2}{|c|}{\textbf{Commands}} \\ \hline
\verb|\rotdriver{<driver>}| & 
declare the name of the dvi to Postscript translator (default {\tt dvips}) \\
......
\verb|sidewaysfigure| &
like the \verb|figure| environment, but rotated 90 degrees \\ \hline
\end{tabular}
\end{sidewaystable}
\end{verbatim}
\end{example} % end example



\sclause{Rotation of float captions and bodies}

    Sometimes it may be useful to rotate a caption independently of the
rotation of a figure or table. The command 
\verb|\rotcaption|\ixcom{rotcaption} is analogous
to the normal \verb|\caption|\ixcom{caption} command, 
and inserts the caption rotated
by 90~degrees. There is also the companion command 
\verb|\controtcaption|\ixcom{controtcaption}, analagous to the
\verb|\contcaption|\ixcom{contcaption} command,
for continuation captions.

%\newsavebox{\picbox}

\begin{figure}
\centering
\caption{Example figure with a standard caption.} \label{fig:nocrot}
\setlength{\unitlength}{0.2in}
\footnotesize
\begin{picture}(17,2)
\thicklines
\put(0,0){\begin{picture}(4,1)
  \put(1.5,0.5){\oval(3,1)}
  \put(1.5,0.5){\makebox(0,0){2,5 (1)}}
  \put(3,0.5){\line(1,0){1.0}}
  \put(4.25,0.5){\circle{0.5}}
  \end{picture}}

\put(4.5,0){\begin{picture}(8,1)
  \put(0,0){\dashbox{0.25}(4,1){date}}
  \put(4,0.5){\line(1,0){3.5}}
  \put(7.75,0.5){\circle{0.5}}
  \put(6,1){\makebox(0,0){A[1:3]}}
  \end{picture}}

\put(12.5,0){\begin{picture}(4,1)
  \put(0,0){\framebox(4,1){INTEGER}}
  \put(3.75,0){\line(0,1){1}}
  \end{picture}}
\end{picture}
\normalsize
\setlength{\unitlength}{1pt}
\end{figure}


\begin{example}Float with a regular caption.

Figure~\ref{fig:nocrot} is produced by the code below:
\begin{verbatim}
\begin{figure}
\centering
\caption{Example figure with a standard caption.} \label{fig:nocrot}
\setlength{\unitlength}{0.2in}
\footnotesize
\begin{picture}(17,2)
\thicklines
\put(0,0){\begin{picture}(4,1)
  \put(1.5,0.5){\oval(3,1)}
  \put(1.5,0.5){\makebox(0,0){2,5 (1)}}
  \put(3,0.5){\line(1,0){1.0}}
  \put(4.25,0.5){\circle{0.5}}
  \end{picture}}
\put(4.5,0){\begin{picture}(8,1)
  \put(0,0){\dashbox{0.25}(4,1){date}}
  \put(4,0.5){\line(1,0){3.5}}
  \put(7.75,0.5){\circle{0.5}}
  \put(6,1){\makebox(0,0){A[1:3]}}
  \end{picture}}
\put(12.5,0){\begin{picture}(4,1)
  \put(0,0){\framebox(4,1){INTEGER}}
  \put(3.75,0){\line(0,1){1}}
  \end{picture}}
\end{picture}
\normalsize
\setlength{\unitlength}{1pt}
\end{figure}
\end{verbatim}
\end{example} % end example

\begin{example}Float with a rotated caption.

Figure~\ref{fig:crot} is produced by the code below:
\begin{verbatim}
\begin{figure}
\centering
\rotcaption{Figure~\protect\ref{fig:nocrot} with a rotated caption.}
 \label{fig:crot}
\setlength{\unitlength}{0.2in}
\footnotesize
\begin{picture}(17,2)
...
\end{picture}
\normalsize
\setlength{\unitlength}{1pt}
\end{figure}
\end{verbatim}
\end{example} % end example


\begin{figure}
\centering
\rotcaption{Figure~\protect\ref{fig:nocrot} with a rotated caption.}
 \label{fig:crot}
\setlength{\unitlength}{0.2in}
\footnotesize
\begin{picture}(17,2)
\thicklines
\put(0,0){\begin{picture}(4,1)
  \put(1.5,0.5){\oval(3,1)}
  \put(1.5,0.5){\makebox(0,0){2,5 (1)}}
  \put(3,0.5){\line(1,0){1.0}}
  \put(4.25,0.5){\circle{0.5}}
  \end{picture}}
\put(4.5,0){\begin{picture}(8,1)
  \put(0,0){\dashbox{0.25}(4,1){date}}
  \put(4,0.5){\line(1,0){3.5}}
  \put(7.75,0.5){\circle{0.5}}
  \put(6,1){\makebox(0,0){A[1:3]}}
  \end{picture}}
\put(12.5,0){\begin{picture}(4,1)
  \put(0,0){\framebox(4,1){INTEGER}}
  \put(3.75,0){\line(0,1){1}}
  \end{picture}}
\end{picture}
\normalsize
\setlength{\unitlength}{1pt}
\end{figure}

    As can be seen from \fref{fig:crot} the advisability of rotating a caption
depends on the size of the body of the float. It may be better in certain
cases to leave the caption in its regular position and rotate the body of
the float instead.

\def\prwrot#1{%
\settowidth{\fool}{ISOROT}
\savebox{\foo}{\parbox{\fool}{ISOROT ISOROT ISOROT ISOROT}}%
\framebox{---\begin{turn}{#1}\framebox{\usebox{\foo}}\end{turn}---}}%
\def\degrees{{\small$^{o}$}}

\begin{figure}
\centering
\begin{tabular}{|c|c|c|} \hline
\prwrot{0} &\prwrot{-40}&\prwrot{-80}\\
0\degrees & -40\degrees & -80\degrees \\ \hline
\prwrot{-120}&\prwrot{-160}&\prwrot{-200}\\
-120\degrees & -160\degrees & -200\degrees \\ \hline
\prwrot{-240}&\prwrot{-280}&\prwrot{-320}\\
-240\degrees & -280\degrees & -320\degrees \\ \hline
\end{tabular}
\caption{Rotation of paragraphs between 0 and -320 degrees} \label{fig:angles1}
\end{figure}


\begin{example}Regular caption and float.

Figure~\ref{fig:angles1} is a regular figure and caption. It is produced by
the following code:
\begin{verbatim}
\def\prwrot#1{%
\settowidth{\fool}{ISOROT}
\savebox{\foo}{\parbox{\fool}{ISOROT ISOROT ISOROT ISOROT}}%
\framebox{---\begin{turn}{#1}\framebox{\usebox{\foo}}\end{turn}---}}%
\def\degrees{{\small$^{o}$}}
\end{verbatim}

\begin{verbatim}
\begin{figure}
\centering
\begin{tabular}{|c|c|c|} \hline
\prwrot{0} &\prwrot{-40}&\prwrot{-80}\\
0\degrees & -40\degrees & -80\degrees \\ \hline
\prwrot{-120}&\prwrot{-160}&\prwrot{-200}\\
-120\degrees & -160\degrees & -200\degrees \\ \hline
\prwrot{-240}&\prwrot{-280}&\prwrot{-320}\\
-240\degrees & -280\degrees & -320\degrees \\ \hline
\end{tabular}
\caption{Rotation of paragraphs between 0 and -320 degrees} \label{fig:angles1}
\end{figure}
\end{verbatim}
\end{example} % end example


\begin{figure}
\centering
\begin{sideways}
\begin{tabular}{|c|c|c|} \hline
\prwrot{0} &\prwrot{40}&\prwrot{80}\\
0\degrees & 40\degrees & 80\degrees \\ \hline
\prwrot{120}&\prwrot{160}&\prwrot{200}\\
120\degrees & 160\degrees & 200\degrees \\ \hline
\prwrot{240}&\prwrot{280}&\prwrot{320}\\
240\degrees & 280\degrees & 320\degrees \\ \hline
\end{tabular}
\end{sideways}
\caption[Rotation of paragraphs between 0 and 320 degrees]%
        {Rotation of paragraphs between 0 and 320 degrees (with figure
         body turned sideways)}\label{fig:angles2}
\end{figure}

\begin{example}Regular caption and rotated float body.

Figure~\ref{fig:angles2} is a regular figure and caption where the figure
contents have been rotated. It was produced by the following code.
\begin{verbatim}
\begin{figure}
\centering
\begin{sideways}
\begin{tabular}{|c|c|c|} \hline
\prwrot{0} &\prwrot{40}&\prwrot{80}\\
0\degrees & 40\degrees & 80\degrees \\ \hline
\prwrot{120}&\prwrot{160}&\prwrot{200}\\
120\degrees & 160\degrees & 200\degrees \\ \hline
\prwrot{240}&\prwrot{280}&\prwrot{320}\\
240\degrees & 280\degrees & 320\degrees \\ \hline
\end{tabular}
\end{sideways}
\caption[Rotation of paragraphs between 0 and 320 degrees]%
        {Rotation of paragraphs between 0 and 320 degrees (with figure
         body turned sideways)}\label{fig:angles2}
\end{figure}
\end{verbatim}
\end{example} % end example

\begin{landscape}
\sclause{Landscaping}

    \latex{} normally prints in portrait mode. 
The \verb|landscape|\ixenv{landscape} environment
prints all the enclosed stuff in landscape mode, except for headers
and footers which are not rotated.

\begin{example} Landscaping

The source for this part of the document is:
\begin{verbatim}
\begin{landscape}
\sclause{Landscaping}

    \latex{} normally prints in portrait mode. The ...
...
... long, wide tables.
\end{landscape}
\end{verbatim}
\end{example}

    The environment starts by clearing the current page and then switches
to portrait mode. At the end of the environment the current page is cleared
and the next page is back to normal portrait mode. 

    All the other rotation commands and environments produce boxes and
\latex{} will not break a box across a page. The \verb|landscape| environemt
does not produce a box and so many pages can be printed in landscape mode
with \latex{} taking care of the page breaking for you. 

    Landscape mode is not particularly useful for normal text as the
lines are far too long for comfortable reading. Where it can be useful
is where you have a table that is too wide to fit on a portrait page, so
needs to be rotated, yet is also too long to fit on the page when it is
rotated. The \file{supertabular}\ixpack{supertabular},
the \file{longtable}\ixpack{longtable},
and the \file{xtab}\ixpack{xtab}
packages provide facilities for automatically breaking long tables across 
pages. Any of these can be used in conjunction with landscaping to both 
rotate and automatically page break long, wide tables.
\end{landscape}

\clause{The \file{xtab} package facility}

    The \file{xtab} package is fully documented in \bref{bib:xtab}.
This clause provides an overview of the package.

    The \file{xtab}\ixpack{xtab} package facility is an extension 
of the \file{supertabular}\ixpack{supertabular} package originally
developed by Johannes Braams and Theo Jurriens.
The extension provides for the specification of a header to
go on the last page of a long table.

    The principal commands available are given in \tref{tab:xtab}.
\ixenv{xtabular}
\ixenv{mpxtabular}
\ixcom{topcaption}
\ixcom{bottomcaption}
\ixcom{tablecaption}
\ixcom{tablefirsthead}
\ixcom{tablehead}
\ixcom{tablelasthead}
\ixcom{notablelasthead}
\ixcom{tabletail}
\ixcom{tablelasttail}

\topcaption{The principal xtab package commands} \label{tab:xtab}
\tablefirsthead{\hline \multicolumn{1}{|c|}{\textbf{Command}} &
                       \multicolumn{1}{c|}{\textbf{Effect}} \\ \hline }
\tablehead{\multicolumn{2}{c}%
           {{\captionsize\bfseries \tablename\ \thetable{} -- continued from previous page}} \\
           \hline \multicolumn{1}{|c|}{\textbf{Command}} &
                  \multicolumn{1}{c|}{\textbf{Effect}} \\ \hline }
\tablelasthead{\multicolumn{2}{c}%
           {{\captionsize\bfseries \tablename\ \thetable{} -- concluded from previous page}} \\ 
           \hline \multicolumn{1}{|c|}{\textbf{Command}} &
                  \multicolumn{1}{c|}{\textbf{Effect}} \\ \hline }
\tabletail{\hline \multicolumn{2}{|r|}{{Continued on next page}} \\ \hline}
\tablelasttail{\\ \hline \hline}

\begin{center}
\begin{xtabular}{|l|p{0.5\textwidth}|}
\verb|\begin{xtabular}{...}| & This is equivalent to the normal
                         \verb|\begin{tabular}{...}| environment.
                         You supply the specification of the columns
                         just as for the normal tabular environment.
                         All commands that can be used within a tabular
                         environment can also be used within
                         the xtabular environment.
 \\
 &
    Unlike the tabular environment which prevents page breaking
within the tabular, the xtabular allows page breaking, so that
tabulars can extend automatically across several pages. Xtabular
starts off with a tabular environment and checks the amount of
space left on the page as it adds each row to the tabular.
If the space left on the page is too short for another row, then
it ends the current tabular, performs a page break and starts
another tabular on the following page. This process is repeated
until all the rows have been output. 
 \\
 &
    There are special commands for captioning a xtabular as a
table, and also elements can be automatically inserted after each
(internal) \verb|\begin{tabular}| and immediately before each
\verb|\end{tabular}|.
 \\
 &
    Do not put a xtabular in a table environment, as the table
environment keeps its contents on a single page (presumably you
are using xtabular because its contents are longer than
one page).
 \\
\verb|\end{xtabular}| & End the xtabular environment. \\ \hline
\verb|\begin{mpxtabular}| & Like the xtabular environment
except that each `page' is put into a \verb|minipage| first. Thus
it is possible to have footnotes inside an \verb|mpxtabular|.
The footnote text is printed at the end of each page.
 \\
\verb|\end{mpxtabular}| & End the mpxtabular environment. \\ \hline
                          & \textbf{Note:} If any of the following commands
                            are used, then they should be placed
                            before the particular xtabular
                            environment that they apply to. \\
\verb|\topcaption{...}| & A command to provide a caption for the
                          table. The caption is placed at the top
                          of the table. \\
\verb|\bottomcaption{...}| & A command to provide a caption for the
                          table. The caption is placed at the bottom
                          of the table. \\
\verb|\tablecaption{...}| & A command to provide a caption for the
                          table. The caption is placed at the default
                          position, which is at the top
                          of the table.
 \\
 & 
                           \textbf{Notes:} You cannot use the \verb|caption|
                            command but you can put a label after
                            any of these captioning commands. If you
                            want captioning, the command must be specified
                            before the start of the xtabular environment.
  \\
  &
                            The \verb|\...caption{}| command(s) remain
                            in effect until changed by another
                            \verb|\...caption| command.
   \\
\verb|\tablefirsthead{...}| & Defines the contents of the first occurence
                         of the tabular head. The tabular head is some
                         special treatment of the first row in the table.
                         This command is optional.
                         If used, the header must be closed by the
                         end of line command for tabulars (e.g., \verb|\\|). \\
\verb|\tablehead{...}| & Defines the contents of the table head on
                         subsequent pages.
 \\
 &
                         For example, you might want to note that
                         this is a continuation of the table on
                         the previous page, as well as repeating
                         any column headings that were given
                         at the start of the xtabular by
                         \verb|\tablefirsthead|.
 \\
\verb|\tablelasthead{...}| & Defines the contents of the table head
                            on the last page of the table.
 \\
 &
    This works by writing to the \file{.aux} file the number of
pages that the xtabular extends over. When the xtabular
comes to the last tabular (which it calculates from the 
information it reads from the \file{.aux} file) it replaces the
contents of \verb|\tablehead| by the contents of \verb|\tablelasthead|.
 \\
 &
It makes no attempt to measure the amount of space used by the last
table head, so if this is more than for \verb|tablehead| then the
tabular might be a litle too long.
  \\
  &
    In any case, when using this command, the document has to
be LaTeXed at least twice, just as it has to be to resolve
references and so on.
 \\
\verb|\notablelasthead|  & Switches off the last \verb|\tablelasthead|.
                           A \verb|tablelasthead| stays in effect until
                           overwritten by a new \verb|\tablelasthead|
                           or cancelled by this command.
 \\
\verb|\tabletail{...}| & The contents of this command are inserted before
                         the \verb|\end{tabular}| on each page except
                         for the last page of the table.
 \\
 &
                         For example, you might want to note that the
                         table is continued on the next page. 
 \\
\verb|\tablelasttail{...}| & The contents of this command are inserted before
                         the final \verb|\end{tabular}| of the table.
 \\
 &
                         For example, you might want to note that
                         this is where the table ends.
\end{xtabular}
\end{center}

    As well as the \verb|xtabular|\ixenv{xtabular} and
\verb|mpxtabular|\ixenv{mpxtabular} environments there are
the corresponding starred versions 
(i.e., \verb|xtabular*|\ixenvs{xtabular} and 
        \verb|mpxtabular*|\ixenvs{mpxtabular}) for use in
two column\ixopt{twocolumn} mode where the table is meant to span 
both columns.

\begin{example} \label{ex:xtab} Table~\ref{tab:xtab} was produced by 
the following code:
\begin{verbatim}
\topcaption{The principal xtab package commands} \label{tab:xtab}
\tablefirsthead{\hline \multicolumn{1}{|c|}{\textbf{Command}} &
                       \multicolumn{1}{c|}{\textbf{Effect}} \\ \hline }
\tablehead{\multicolumn{2}{c}%
           {{\captionsize\bfseries \tablename\ \thetable{} -- 
             continued from previous page}} \\
  \hline   \multicolumn{1}{|c|}{\textbf{Command}} &
           \multicolumn{1}{c|}{\textbf{Effect}} \\ \hline }
\tablelasthead{\multicolumn{2}{c}%
           {{\captionsize\bfseries \tablename\ \thetable{} -- 
             concluded from previous page}} \\
  \hline   \multicolumn{1}{|c|}{\textbf{Command}} &
           \multicolumn{1}{c|}{\textbf{Effect}} \\ \hline }
\tabletail{\hline \multicolumn{2}{|r|}{{Continued on next page}} \\ \hline}
\tablelasttail{\\ \hline \hline}

\begin{center}
\begin{xtabular}{|l|p{0.5\textwidth}|}
\verb|\begin{xtabular}{...}| & This is equivalent to the normal
                         \verb|\begin{tabular}{...}| environment.
                         You supply the specification of the columns
                         just as for the normal tabular environment.
                         All commands that can be used within a tabular
                         environment can also be used within
                         the xtabular environment.
 \\
 &
    Unlike the tabular environment which prevents page breaking
within the tabular, the xtabular allows page breaking, so that
tabulars can extend automatically across several pages. Xtabular
... ... ...
\verb|\tablelasttail{...}| & The contents of this command are inserted before
                         the final \verb|\end{tabular}| of the table.
 \\
 &
    For example, you might want to note that  this is where 
the table ends.
\end{xtabular}
\end{center}
\end{verbatim}
\end{example}       

    The table is only broken between rows --- a row will not be split
across pages. This can lead to some bad page breaks, especially if 
there are rows with a large vertical height (like some in \tref{tab:xtab}).
It is best to keep rows not too tall.

The command
\verb|\shrinkheight{|\meta{length}\verb|}|\ixcom{shrinkheight} may be
used after the first \verb|\\| in the table to modify the allowed
height of the table on each page. A positive \meta{length} decreases
the allowed space per page and a negative \meta{length} increases
the allowed space.

\begin{example}
This example illustrates changing the natural height of the pages
in a \verb|xtabular| and its relatives.

\verb|\shrinkheight{2\baselineskip}| decreases the space per page by 
two lines.

\verb|\shrinkheight{-\baselineskip}| increases the space per page by
one line.
\end{example}

    You have to specify the font used for the 
\verb|\tablehead|\ixcom{tablehead} and
\verb|tablelasthead|\ixcom{tablelasthead} yourself. 
Within ISO documents, captions shall
be in bold font. The \file{iso}\ixclass{iso} class also provides 
a command for
setting the size of the font used in captions, namely 
\verb|\captionsize|\ixcom{captionsize}. 
Note that this is used in~\eref{ex:xtab}.
    The value of \verb|\captionsize|\ixcom{captionsize} is set by the 
class.

\sclause{Options}

    The \file{xtab} package has three options which control the amount of
information that is written to the \file{.log}\index{file!.log} file. The
options are:
\begin{enumerate}
\item The option \verb|errorshow|\ixopt{errorshow} (the default) does not
      write any extra information;
\item The option \verb|pageshow|\ixopt{pageshow} writes information
      about when and why \verb|xtabular| decides to produce a new page;
\item The option \verb|debugshow|\ixopt{debugshow}, which also includes
      \verb|pageshow|, additionally writes information about each line
      that is added to the table.
\end{enumerate}

    Under normal circumstances \file{xtab} is used without invoking any 
option. The \verb|pageshow| option may be useful when attempting to cure
a bad page break. The \verb|debugshow| option, as its name implies, is
principally of use to the \file{xtab} developer.

    Independently of the options, the command 
\verb|\sstraceon|\ixcom{sstraceon} may be used at any
point in the document to turn on printing of \verb|debugshow| data. This
can be turned off later by the \verb|\sstraceoff|\ixcom{sstraceoff}
command, which will stop all \verb|...show| printing.

\sclause{Caveats}

    The authors of the original \file{supertabular} package note that
it has the following weaknesses:
\begin{itemize}
\item Sometimes the body of the first page of a table gets moved to the 
following, leaving the caption behind;
\item Sometimes the last page of a table consists of just the head and foot
with no contents.
\end{itemize}
The weaknesses are caused by trying to guess where \tex{} will put a page 
break. The package has to guesstimate how long the next entry will be in 
the table and, if it is too long for the available space, it puts in its
own page break. If its guess is off too much in one direction, \tex{} will 
break the page unexpectadly; if its off in the other direction
\file{supertabular} will put in an unnecessary page break.

    The \file{xtab} package has reduced, but perhaps not entirely
eliminated, these weaknesses. Some hand tuning may still be required.


%%%%%%%%%
%%%%\end{document}
%%%%%%%%%



\clause{The \file{askinc} package facility} 

    The \file{askinc}\ixpack{askinc} package facility contains \latex{} 
code to enable the interactive
input of files. This functionality is a cross between the \latex{}
\verb|\include|\ixcom{include} and 
\verb|\includeonly|\ixcom{includeonly} commands, and the
\verb|\input|\ixcom{input} command.

In the body of the root source document, use the command 
\verb+\infile{+\meta{file}\verb+}+\ixcom{infile}
for each \meta{file}
that comprises the document. That is, the command \verb|\infile| is similar
to the \verb|\input|\ixcom{input} and
\verb|\include|\ixcom{include} commands.

    When \latex ing the root document, you will be asked to provide a 
comma-separated list of the \verb|\infile|s to be processed (similar to the
argument to the \verb|\includeonly|\ixcom{includeonly}
command). If you want all the files to be processed, just hit the
\verb|<RETURN>| key (or its equivalent).

    Like \verb|\include|d files, a file that is \verb|\infile|d into a 
document shall not itself contain another \verb|\infile|d file.

\begin{example}The following root file has three files that are \verb|\infile|d.
\begin{verbatim}
\documentclass[...]{...}
\usepackage{askincv}
% other preamble stuff
\begin{document}
% perhaps some stuff
\infile{file1}
\infile{file2}
\infile{file3}
% perhaps more stuff
\end{document}
\end{verbatim}
\end{example} % end example

\clause{The \file{hyphenat} package facility} \label{sec:uschyp}

    The \file{hyphenat} package is fully described in~\bref{bib:hyphenat}.
This clause provides an overview of the portions of the package that are
most relevant to typesetting ISO standards.

    In \latex{} if you need to use the underscore (\verb|_|)
character in normal text, for example when documenting identifiers
in a programming language code, you have to use the 
\verb|\_|\index{_ /@\verb?\_?} command,
as in \verb|a\_multiword\_identifier|. \latex{} normally treats
\verb|_|\index{_@\verb?_? (underscore)} as a math mode subscript command. 
Further, if you want
the possibility of hyphenation\index{hyphenation} 
at the position of an underscore
you have to use the command pairing 
\verb|\_\-|;\index{_ /@\verb?\_?}\ixcom{-} this has the side
effect of disabling normal hyphenation in any succeeding `word' in
the identifier.

    The \file{hyphenat}\ixpack{hyphenat} package facility redefines the 
\verb|\_|\index{_ /@\verb?\_?} command
so that hyphenationen is automatically enabled at the position
of the underscore and in all succeeding words in the identifier.


\begin{note}
    Using the command pair \verb|\_\-| in conjunction with this facility
disables automatic hyphenation of succeeding words, so don't do it.
\end{note}



\begin{example} \label{eg:uschyp}
This set of \latex{} source
\begin{verbatim}
Amazingly An\_excessively\_long\_multiword\_identifier%
\_demonstrating\_hyphenation

\begin{minipage}{3cm}
\begin{itemize}
\item An\_excessively\_long\_multiword\_identifier%
\_demonstrating\_hyphenation
\item Underscore in math mode: $A_n$
\item \verb|\_| command in math mode: $A\_n$
\end{itemize}
\end{minipage}
\end{verbatim}
prints as:

Amazingly An\_excessively\_long\_multiword\_identifier%
\_demonstrating\_hyphenation

\begin{minipage}{3cm}
\begin{itemize}
\item An\_excessively\_long\_multiword\_identifier%
\_demonstrating\_hyphenation
\item Underscore in math mode: $A_n$
\item \verb|\_| command in math mode: $A\_n$
\end{itemize}
\end{minipage}

\end{example}

\begin{example}
Contrast this example with \eref{eg:uschyp}.
This set of \latex{} source
\begin{verbatim}
Amazingly An\_\-excessively\_\-long\_\-multiword\_\-identifier%
\_\-demonstrating\_\-hyphenation\_\-disabling

\begin{minipage}{3cm}
\begin{itemize}
\item An\_\-excessively\_\-long\_\-multiword\_\-identifier%
\_\-demonstrating\_\-hyphenation\_\-disabling
\item Underscore in math mode: $A_n$
\item \verb|\_| command in math mode: $A\_n$
\end{itemize}
\end{minipage}
\end{verbatim}
prints as:

Amazingly An\_\-excessively\_\-long\_\-multiword\_\-identifier%
\_\-demonstrating\_\-hyphenation\_\-disabling

\begin{minipage}{3cm}
\begin{itemize}
\item An\_\-excessively\_\-long\_\-multiword\_\-identifier%
\_\-demonstrating\_\-hyphenation\_\-disabling
\item Underscore in math mode: $A_n$
\item \verb|\_| command in math mode: $A\_n$
\end{itemize}
\end{minipage}

\end{example}


    The \file{hyphenat} package also provides some other commands for
enabling hyphenation of words that include 
analphabetic\index{character!analphabetic}\footnote{An {\it analphabetic} 
character is any character
that is not alphabetic. Typically it refers to punctuation characters.}
characters. In this context, the phrase 
`breakable character'\index{character!breakable} is used
to describe an analphabetic character that enables hyphenation immediately
after it and does not prevent further hyphenation in the `word'
in which it occurs. The \verb|\_|\index{_ /@\verb?\_?} command produces 
a breakable underscore. Table~\ref{tab:breakable} lists all the commands
that generate breakable characters.

\begin{table}
\centering
\caption{Commands producing breakable characters} \label{tab:breakable}
\begin{tabular}{|l|c|} \hline
\textbf{Command}    & \textbf{Character} \\ \hline
\verb|\_|        & \_ \\
%\verb|\?|        & \? \\
\verb|\bshyp|    & \bshyp \\
\verb|\colonhyp| & \colonhyp \\
\verb|\dothyp|   & \dothyp \\
\verb|\fshyp|    & \fshyp \\ \hline
\end{tabular}
\end{table}

    The \verb|\bshyp|\ixcom{bshyp} command prodes a breakable backslash
(\verb|\|), \verb|\fshyp|\ixcom{fshp} produces a breakable forward
slash (\verb|/|), the \verb|\dothyp|\ixcom{dothyp} command produces a
breakable full stop (\verb|.|), also known in some countries as a period,
and the \verb|\colonhyp|\ixcom{colonhyp} command produces a breakable 
colon (:).

\begin{example}
This is similar to \eref{eg:uschyp} except that it demonstrates other
breakable characters.

This set of \latex{} source
\begin{verbatim}
Analphabetically an\bshyp{}excessively\fshyp{}long\dothyp{}multiword\bshyp{}identifier%
\fshyp{}demonstrating\dothyp{}hyphenation

\begin{minipage}{3cm}
Analphabetically an\bshyp{}excessively\fshyp{}long\dothyp{}multiword\bshyp{}identifier%
\fshyp{}demonstrating\dothyp{}hyphenation
\end{minipage}
\end{verbatim}
prints as:

Analphabetically an\bshyp{}excessively\fshyp{}long\dothyp{}multiword\bshyp{}identifier%
\fshyp{}demonstrating\dothyp{}hyphenation

\begin{minipage}{3cm}
Analphabetically an\bshyp{}excessively\fshyp{}long\dothyp{}multiword\bshyp{}identifier%
\fshyp{}demonstrating\dothyp{}hyphenation
\end{minipage}

\end{example}

\begin{note}
\latex{} will not hyphenate the first word in a paragraph.
\end{note}

    Just as with the \verb|\_|\index{_ /@\verb?\_?} command, 
the discretionary hyphen
command (i.e., \verb|\-|\ixcom{-}) should not be used in conjunction with 
any of the breakable character commands as it will then inhibit any
further potential hyphenation points. In general, any analphabetic
character in a word will inhibit further hyphenation.

\normannex{Additional commands} \label{anx:extraiso}

\sclause{Language configuration commands}

    There is an additional set of commands in the
\file{iso}\ixclass{iso} class
facility that
are specified here. This set has been provided to enable the style to be
easily configured for a non-English language.\index{non-English languages}

    The modified commands may be put in the document 
preamble\index{preamble} or, preferably,
placed in a separate \file{.sty}\index{file!.sty} file and 
called as a package. This latter option improves reuseability.

\ssclause{Words and phrases}

    To produce a non-English version of the \file{iso}\ixclass{iso}
class the relevant commands
from the following list may require re-definition together with
the heading commands in \tref{tab:dsa}.

\begin{note}In the listing, the default values are printed \emph{in this
font} to distinguish them from the explanatory text.
\end{note} % end note}

\begin{itemize}
\item \verb|\annexname|\ixcom{annexname}: Header for
      an annex.
      Default value is: \emph{\annexname}\index{Annex}
\item \verb|\contentsname|\ixcom{contentsname}: Header
      for table of contents listing. 
      Default value is: \emph{\contentsname}\index{Contents}
\item \verb|\copyrightname|\ixcom{copyrightname}:
      The copyright owner.
      Default value is: \emph{\copyrightname}\index{ISO}
\item \verb|\examplename|\ixcom{examplename}:
      Identification of an example.
      Default value is: \emph{\examplename}\index{EXAMPLE}
%\item \verb|\examplesname|\ixcom{examplesname}:
%      Header for a list of examples.
%      Default value is: \emph{\examplesname}\index{EXAMPLES}
\item \verb|\figurename|\ixcom{figurename}: Start of
      the caption for a figure. 
      Default value is: \emph{\figurename}\index{Figure}
\item \verb|\forewordname|\ixcom{forewordname}:
      Title of the Foreword.
      Default value is: \emph{\forewordname}\index{Foreword}
\item \verb|\indexname|\ixcom{indexname}: Header for
      the index.
      Default value is: \emph{\indexname}\index{Index}
\item \verb|\informativename|\ixcom{informativename}:
      Identification of an informative annex.
      Default value is: \emph{\informativename}\index{informative}
\item \verb|\inscopename|\ixcom{inscopename}:
      Introduction to in-scope listing.
      Default value is: \emph{\inscopename}
      \index{The following are within the scope of this}
\item \verb|\ISname|\ixcom{ISname}:
     `INTERNATIONAL STANDARD' as used in the header for an IS title page.
      Default value is: 
      \emph{\ISname}\index{INTERNATIONAL STANDARD}
\item \verb|\introductionname|\ixcom{introductionname}:
      Title of the Introduction.
      Default value is: 
      \emph{\introductionname}\index{Introduction}
\item \verb|\listannexname|\ixcom{listannexname}:
      Header for list of annexes listing.
      Default value is: \emph{\listannexname}\index{Annexes}
\item \verb|\listfigurename|\ixcom{listfigurename}:
      Header for list of figures listing.
      Default value is: \emph{\listfigurename}\index{Figures}
\item \verb|\listtablename|\ixcom{listtablename}:
      Header for list of tables listing.
      Default value is: \emph{\listtablename}\index{Tables}
\item \verb|\normativename|\ixcom{normativename}:
      Identification of a normative annex.
      Default value is: \emph{\normativename}\index{normative}
\item \verb|\normrefsname|\ixcom{normrefsname}:
      Title of normative references clause.
      Default value is: 
      \emph{\normrefsname}\index{Normative references}
\item \verb|\notename|\ixcom{notename}:
      Identification of a note.
      Default value is: \emph{\notename}\index{NOTE}
%\item \verb|\notesname|\ixcom{notesname}:
%      Header for a list of notes.
%      Default value is: \emph{\notesname}\index{NOTES}
\item \verb|\outofscopename|\ixcom{outofscopename}:
      Introduction to out-of-scope listing.
      Default value is: \emph{\outofscopename}
      \index{The following are outside the scope of this}
\item \verb|\pagename|\ixcom{pagename}:
      The word for the page header in the table of contents.
      Default value is: \emph{\pagename}\index{Page}
\item \verb|\scopename|\ixcom{scopename}:
      Title of the Scope.
      Default value is: \emph{\scopename}\index{Scope}
%\item \verb|\sectionname|\ixcom{sectionname}:
%      The word for a 'section'.
%      Default value is: \emph{\sectionname}\index{Section}
\item \verb|\tablename|\ixcom{tablename}:
      Start of the caption for a table.
      Default value is: \emph{\tablename}\index{Table}
\item \verb|\tbpname|\ixcom{tbpname}:
      Footnote text for `to be published.'.
      Default value is: 
      \emph{\tbpname}\index{To be published.}
\end{itemize}

    The following commands provide the names for referenced document elements.
\begin{itemize}
\item \verb|\annexrefname|\ixcom{annexrefname}:
      Reference to an annex.
      Default value is: \emph{\annexrefname}\index{annex}
\item \verb|\clauserefname|\ixcom{clauserefname}:
      Reference to a clause.
      Default value is: \emph{\clauserefname}\index{clause}
\item \verb|\examplerefname|\ixcom{examplerefname}:
      Reference to an example.
      Default value is: \emph{\examplerefname}\index{example}
\item \verb|\figurerefname|\ixcom{figurerefname}:
      Reference to a figure.
      Default value is: \emph{\figurerefname}\index{figure}
\item \verb|\noterefname|\ixcom{noterefname}:
      Reference to a note.
      Default value is: \emph{\noterefname}
\item \verb|\tablerefname|\ixcom{tablerefname}:
      Reference to a table.
      Default value is: \emph{\tablerefname}\index{table}
\item \verb|\pagerefname|\ixcom{pagerefname}:
      Reference to a page.
      Default value is: \emph{\pagerefname}\index{page}
\end{itemize}



\begin{note} The above commands, 
 may be changed via the \latex{} 
\verb|\renewcommand|\ixcom{renewcommand}.
\end{note} % end note
\begin{note}The \latex{} command \verb|\today|\ixcom{today}
      will probably also require modification. This is not something
      for the casual user to attempt.
\end{note} % end note


\begin{example}The following is a partial list of the commands to convert to a 
         French language\index{French language} style. 
\begin{verbatim}
\renewcommand{\annexname}{Annexe}
\renewcommand{\contentsname}{Sommaire}
\renewcommand{\examplename}{EXEMPLE}
%\renewcommand{\examplesname}{EXEMPLES}
\renewcommand{\forewordname}{Avant-propos}
\renewcommand{\ISname}{NORME INTERNATIONALE}
\renewcommand{\listtablename}{Tableaux}
\renewcommand{\scopename}{Domaine d'application}
\renewcommand{\tablename}{Tableau}
\end{verbatim}
\end{example} % end example

\ssclause{Boilerplate}

    Some commands print boilerplate text; again, the default text is in English.
For \file{iso}\ixclass{iso} there are several such commands. The first is 
\verb|\copyrightnotice|\ixcom{copyrightnotice}
 which contains the text of the copyright notice
for an International Standard. This can be changed via the 
\verb|\renewcommand|\ixcom{renewcommand} command.

    The second is \verb|\normrefbp|\ixcom{normrefbp}
which prints the boilerplate for the introduction to the Normative
references clause. Like the \verb|\copyrightnotice| command, this can
be redefined using the \verb|\renewcommand|.

    Another is the \verb|\fwdbp|\ixcom{fwdbp} command which
\verb|\input|s the boilerplate text from a file called \file{isofwdbp.tex}.
\index{isofwdbp.tex@\file{isofwdbp.tex}} For non-English text either 
modify the contents of this
file or create a new file and modify the \verb|\fwdbp| command to call in
the new file.



\begin{example}This is how \verb|\normrefbp| could be written for the 
French language and in accordance with the second edition of the 
ISO Directives.
\label{eg:frenchnfbp}
\begin{verbatim}
\renewcommand{\normrefbp}[1]{%
    Les normes suivantes contiennent des dispositions qui, par
suite de la r\'{e}f\'{e}nce qui en est faite, constituent des
dispositions valables pour la pr\'{e}sente #1. 
Au moment de la publication, les \'{e}ditions indiqu\'{e}es 
\'{e}taient en viguer. Toute norms est sujette \`{a} r\'{e}vision et
les parties prenantes des accords fond\'{e}s sur la pr\'{e}sente #1 
sont invit\'{e}es \`{a} rechercher la possibilit\'{e} d'appliquer
les \'{e}ditions les plus r\'{e}centes des normes indiqu\'{e}es
ci-apr\`{e}s. Les membres de la CEI et de l'ISO poss\`{e}dent
le registre des Normes Internationales en vigueur \`{a} un
moment donn\'{e}.
}% end renewcommand
\end{verbatim}
\end{example} % end example
\begin{example}Given that \verb|\normrefbp| has been redefined as in \eref{eg:frenchnfbp},
then the command \\
 \verb|\normrefbp{norme internationale}| will print:

\newcommand{\tempbp}[1]{%
    Les normes suivantes contiennent des dispositions qui, par
suite de la r\'{e}f\'{e}nce qui en est faite, constituent des
dispositions valables pour la pr\'{e}sente #1. 
Au moment de la publication, les \'{e}ditions indiqu\'{e}es 
\'{e}taient en viguer. Toute norms est sujette \`{a} r\'{e}vision et
les parties prenantes des accords fond\'{e}s sur la pr\'{e}sente #1 
sont invit\'{e}es \`{a} rechercher la possibilit\'{e} d'appliquer
les \'{e}ditions les plus r\'{e}centes des normes indiqu\'{e}es
ci-apr\`{e}s. Les membres de la CEI et de l'ISO poss\`{e}dent
le registre des Normes Internationales en vigueur \`{a} un
moment donn\'{e}.
}% end newcommand

\tempbp{norme internationale}
\end{example} % end example



    The \verb|\tspasfwdbp|\ixcom{tspasfwdbp} also maintains
boilerplate text in the file 
\file{tspasfwdbp.tex}\index{tspasfwdbp.tex@\file{tspasfwdbp.tex}}.

\sclause{Caption text size}

    The size of the font used for typestting the captions of figures and
tables is defined within the \file{iso}\ixclass{iso} class.
% The
%\verb|uglycaption|\ixopt{uglycaption} option resets the size to larger
%than normal for the captioning text.

    The size of the captioning font is controlled by the value of the
\verb|\captionsize|\ixcom{captionsize} command. The default definition
of \verb|\captionsize| is similar to:
\verb|\captionsize{\normalsize}| \ixcom{normalsize}\\
%The \verb|uglycaption| option resets this to (approximately): \\
%\verb|\captionsize{\large}|. \ixcom{large}

    You can reset the \verb|\captionsize| at any point in your
document to change the size of captions from then onwards.

    All the normal \latex{} font size commands are available.
\begin{example}
The following shows the effects of the font size commands. \\
\verb|{\tiny tiny text}|\ixcom{tiny} 
prints: {\tiny tiny text} \\
\verb|{\scriptsize scriptsize text}|\ixcom{scriptsize} 
prints: {\scriptsize scriptsize text} \\
\verb|{\footnotesize footnotesize text}|\ixcom{footnotesize} 
prints: {\footnotesize footnotesize text} \\
\verb|{\small small text}|\ixcom{small}
prints: {\small small text} \\
\verb|{\normalsize normalsize text}|\ixcom{normalsize} 
prints: {\normalsize normalsize text} \\
\verb|{\large large text}|\ixcom{large} 
prints: {\large large text} \\
\verb|{\Large Large text}|\ixcom{Large} 
prints {\Large Large text} \\
\verb|{\LARGE LARGE text}|\ixcom{LARGE} 
prints: {\LARGE LARGE text} \\
\verb|{\huge huge text}|\ixcom{huge} 
prints: {\huge huge text} \\
\verb|{\Huge Huge text}|\ixcom{Huge} 
prints: {\Huge Huge text}
\end{example}


\normannex{Ordering of LaTeX commands} \label{anx:lord}

    The \latex{} commands for the logical structuring of an ISO standard 
document are:
\begin{verbatim}
\documentclass[<options>]{isov2}          % for LaTeX 2e
\usepackage{<name>}                       % additional packages (LaTeX 2e)
\standard{<standard identification>}
\yearofedition{<year>}
\languageofedition{<parenthesized code letter>}
  % other preamble commands
\begin{document}
\begin{foreword}                     % start Foreword
  \fwdbp                             % boilerplate
  % other text and perhaps \fwdnopatents
\end{foreword}
\begin{introduction}                 % start Introduction
  % text and perhaps \intropatents
\end{introduction}
\title{<intro>}{<main>}{<compl>}     % the title
\scopeclause                         % The Scope clause
   \begin{inscope}{<document>}       % in scope boilerplate
   % \item list
   \end{inscope}
   % text 
   \begin{outofscope}{<document>}    % out of scope boilerplate
   % \item list
   \end{outofscope}
   % text
\normrefsclause                      % The Normative references clause
  \normrefbp{<document identifier>}  % boilerplate
  \begin{nreferences}
    % \isref{<p1>}{<p2>} and/or \disref{<p1>}{<p2>} commands
  \end{nreferences}
% definitions, symbols, abbreviation clause as appropriate
\clause{<Clause title>}
% THE BODY OF THE DOCUMENT
% ...
% \normannex{<Normative annex title>}
% ...
% \infannex{<Infomative annex title>}
% ...
\bibannex                            % optional bibliography
  % bibliography listing
% the index
\end{document}
\end{verbatim}


\infannex{Creating an index} \label{anx:indexing}

    \latex, although providing some assistance in preparing the data for an
index\index{index}, only does part of the job. Providing the command
\verb|\makeindex|\ixcom{makeindex} is put in the document's 
preamble\index{preamble}
the \latex{} command \verb|\index{text}|\ixcom{index} 
writes out \verb|text| to an \file{.idx}\index{file!.idx} file 
in the following format:\ixcom{indexentry}
\begin{verbatim}
\indexentry{text}{pg}
\end{verbatim}
where \verb|pg| is the page number in the document where the \verb|\index|
command occurred.

    The \verb|theindex|\ixenv{theindex} environment 
is used for printing an index. The format of this is:
\begin{verbatim}
\begin{theindex}
\item <text and page numbers>
  \subitem <text and page numbers>
    \subsubitem <text and page numbers>
.
.
\end{theindex}
\end{verbatim}
where \verb|\item|\ixcom{item} is a major topic entry,
\verb|\subitem|\ixcom{subitem} is a sub-topic entry, and
\verb|\subsubitem|\ixcom{subsubitem} is a sub-subtopic
entry. The command \verb|\indexspace|\ixcom{indexspace}
can be used to add space between the entries. Other text and commands can also
occur within the environment.

    \latex{} provides no help in going from an \file{.idx} file to the
\verb|theindex| environment~\bref{lamport}. The data in the file has to be
sorted, duplicate page numbers deleted, etc, and then re-written in the 
desired environment format.

    The typical process for producing a document with an index is:
\begin{enumerate}
\item Prepare the source file, say \file{fred.tex}, with the command
      \verb|\makeindex|\ixcom{makeindex} in the
      preamble\index{preamble}, and \verb|\index|\ixcom{index}
      commands within the body of the text.
\item Run \latex. As well as producing the usual output files, it will also
      produce the file \file{fred.idx}.
\item By some means produce a file, let's call it \file{fredidx.tex}, from the 
      data in \file{fred.idx} that contains the appropriate \verb|theindex|
      formatted data.
\item Run \latex{} again on \file{fred.tex} which now has to contain (either
      via \verb|\input|\ixcom{input} or \verb|\include|\ixcom{include})
      the file \file{fredidx.tex}.
\end{enumerate}

    Chen and Harrison~\bref{chen} discuss the problems of creating an index 
in their
paper \emph{Index preparation and processing} and also describe the
{\sc MakeIndex}\index{makeIndex@{\sc MakeIndex}} program. Goosens, Mittelbach
and Samarin~\bref{goosens} also describe how to use the {\sc MakeIndex} 
program for producing indexes for \latex{} documents. For users of
{\sc MakeIndex} a style file called 
\file{iso.ist}\index{iso.ist@\file{iso.ist}}\index{file!.ist} is provided as
part of this distribution.

\sclause{The index command}\ixcom{index}

    The \verb|\index| command is one of the standard \latex{} commands.
The command format is \verb|\index{|\meta{str}\verb|}|, where \meta{str}
 is any
string of characters, and it writes an entry to the \file{.idx} file in
the form \verb|\indexentry{<str>}{pg}|, where \verb|pg| is the page
number in the document where the command is called.

    Some points to note:
\begin{itemize}
\item The \verb|\index| command is, in \latex{} terminology, 
      \emph{fragile}\index{fragile}. That is, if it appears in a moving
      argument (like the caption to a table or figure) it must be preceded
      by the \latex{} \verb|\protect|\ixcom{protect}
      command.
\item Any of the ten \latex{} special characters\index{special characters}
      (i.e., \verb|#|, \verb|$|, \verb|%|, \verb|&|, \verb|~|, \verb|_|,
             \verb|^|, \verb|\|, \verb|{| and \verb|}|) may appear
     within the argument, with the proviso that it must have no unmatched
     braces (the braces in \verb|\{| and \verb|\}| are counted in the
     matching process).
\item The \verb|\index| command must not appear inside another command's
     argument (e.g., within a 
     \verb|\footnote|\ixcom{footnote} or \verb|\clause|\ixcom{clause}
     command) unless the argument of the \verb|\index| command contains
     only letters, digits, and/or punctuation characters. In particular,
     it should not contain any of the special characters.
  \begin{note}This means that the argument of the \verb|\ix|\ixcom{ix}
        command should not contain any special characters. Remember that the
       \verb|\ix| command prints its argument in the body of the text and also
       calls \verb|\index| to place its argument into the \file{.idx} file.
       \end{note} % end note
\end{itemize}

\begin{note}Under some circumstances, an \verb|\index| command appearing in another
command's argument may contain special characters, provided they are 
\verb|\protect|ed. Determination of when this is satisfactory is a process
of trial and error.
\end{note} % end note

\begin{example}The command \verb|\ix{an\protect\_underscore}| will print the
characters \emph{an\_undescore}
in the text and also write the following to the \file{.idx} file:
\begin{verbatim}
\indexentry{an\_ underscore}{pg}
\end{verbatim}
Notice that there is a space between the underscore character and the word
`underscore' in the \file{.idx} file (but there is no space in the printed
body of the document text). This extraneous space may have to be edited out 
from the final index.
\end{example} % end example


\begin{comment}

\sclause{The program GenIndex} \index{genindex@GenIndex}

    GenIndex is a C program\index{C language} that converts \file{.idx}
data to \verb|theindex|\ixenv{theindex} data. 
Source code for GenIndex is freely available from
the NIST SOLIS\index{SOLIS} system (see \ref{anx:solis}). 
The GenIndex program is not
as sophisticated as {\sc MakeIndex} but does a reasonable 
job.\footnote{These two programs are not completely compatible but do have
a common subset of commands. The common subset excludes the {\tt SeeAlso}
and {\tt See} commands. At some later time GenIndex may be rewritten to
be compatible with {\sc MakeIndex}. It is probably advisable, therefore, not
to use the GenIndex {\tt SeeAlso} and {\tt See} commands.}


\ssclause{Input} \index{genindex@GenIndex!input}

    GenIndex reads lines of data of the form:
\begin{verbatim}
\indexentry{IndexData}{Page}
\end{verbatim}

    \verb|Page| is a character string representing a page number. GenIndex only
recognizes strings that represent integer values greater than or equal to zero,
or (upper and lower case) roman numerals. For any other numbering system, the
page number is treated as zero.

    \verb|IndexData| is a string of characters and command characters that
specify the data that is to be indexed. In the simplest case, this is just the
word or phrase to appear in the index, but much more can be done than this, as
is described below.

    The general form of \verb|IndexData| is given by the following partial
grammar:
\begin{verbatim}
IndexData  = MainData ['!' SubData ['!' SubSubData ] ] .
MainData   = Data .
SubData    = Data .
SubSubData = Data .
Data       = Key [ Entry ] [ SeeAlso ] [ See ] .
Key        = any string not containing the @, !, + or * characters .
Entry      = '@' any string not containing the !, + or * characters 
                 (unless enclosed in braces {} ) .
SeeAlso    = '+' any string not containing the ! or * characters
                 (unless enclosed in braces {} ) .
See        = '*' any string not containing the ! character
                 (unless enclosed in braces {} ) .
\end{verbatim}

    Upto three levels of indexing are enabled --- a main topic entry, an optional
sub-topic, and an optional sub-subtopic. The sub-topics for an entry are
introduced by the \verb|!| character. \index{(33@{\verb?!?}}

     Apart from \verb|Key|, braces within a string must be matched. That is,
they must appear in pairs of an opening and a closing brace.
Spaces are significant within the \verb|Key| string, but not in the others.
That is, \verb*|\indexentry{A}| differs from \verb*|\indexentry{ A}| 
both of which differ from \verb*|\indexentry{A }|.


\begin{example}Here is an index entry for a simple topic:
\begin{verbatim}
\indexentry{Topic}{27}
\end{verbatim}
\end{example} % end example
\begin{example}And here is one where the key and the main entry are different, a 
sub-topic is being indexed, and the page is in roman numerals:
\begin{verbatim}
\indexentry{main@\textbf{Main}!sub topic}{xviii}
\end{verbatim}
\end{example} % end example
\begin{example}This is how the characters \verb|@|, \verb|!|, \verb|+| and
\verb|*| characters are indexed in this document.
\begin{verbatim}
\index{(64@{\verb?@?}}
\index{(33@{\verb?!?}}
\index{(43@{\verb?+?}}
\index{(42@{\verb?*?}}
\end{verbatim}
\end{example} % end example


\sssclause{Key} \index{genindex@GenIndex!input!key}

    The indexed entries are sorted alphabetically on the value of \verb|Key|.
Sorting is case-insensitive. A \verb|Key| value is required.

    More precisely, the key entries are sorted according to the
C language\index{C language} implementation collating sequence, which is usually ASCII. 
Table~\ref{tab:ascii} gives the ASCII collating sequence for the \latex\
character set. \index{ASCII}


\begin{table}
\def\vissp{\hbox{\tt\char`\ }} % visible space
\centering
\caption{The \protect\latex{} ASCII character set} \label{tab:ascii}
\begin{tabular}{|r|c|c|c|c|c|c|c|c|c|c|} \hline
  & \textbf{0} & \textbf{1} & \textbf{2} & \textbf{3} & \textbf{4} &
\textbf{5} & \textbf{6} & \textbf{7} & \textbf{8} & \textbf{9} \\ \hline\hline
\textbf{30}  &          &          & \vissp   & \verb|!| & \verb|"| &
            \verb|#| & \verb|$| & \verb|%| & \verb|&| & \verb|'|    \\
\textbf{40}  & \verb|(| & \verb|)| & \verb|*| & \verb|+| & \verb|,| &
            \verb|-| & \verb|.| & \verb|/| & \verb|0| & \verb|1|     \\
\textbf{50}  & \verb|2| & \verb|3| & \verb|4| & \verb|5| & \verb|6| &
            \verb|7| & \verb|8| & \verb|9| & \verb|:| & \verb|;|     \\
\textbf{60}  & \verb|<| & \verb|=| & \verb|>| & \verb|?| & \verb|@| &
            \verb|A| & \verb|B| & \verb|C| & \verb|D| & \verb|E|     \\
\textbf{70}  & \verb|F| & \verb|G| & \verb|H| & \verb|I| & \verb|J| &
            \verb|K| & \verb|L| & \verb|M| & \verb|N| & \verb|O|     \\
\textbf{80}  & \verb|P| & \verb|Q| & \verb|R| & \verb|S| & \verb|T| &
            \verb|U| & \verb|V| & \verb|W| & \verb|X| & \verb|Y|     \\
\textbf{90}  & \verb|Z| & \verb|[| & \verb|\| & \verb|]| & \verb|^| &
            \verb|_| & \verb|`| & \verb|a| & \verb|b| & \verb|c|     \\
\textbf{100} & \verb|d| & \verb|e| & \verb|f| & \verb|g| & \verb|h| &
            \verb|i| & \verb|j| & \verb|k| & \verb|l| & \verb|m|     \\
\textbf{110} & \verb|n| & \verb|o| & \verb|p| & \verb|q| & \verb|r| &
            \verb|s| & \verb|t| & \verb|u| & \verb|v| & \verb|w|     \\
\textbf{120} & \verb|x| & \verb|y| & \verb|z| & \verb|{| & \verb/|/ &
            \verb|}| & \verb|~| &          &          &              \\ \hline
\end{tabular}
\end{table}

\sssclause{Entry} \index{genindex@GenIndex!input!entry} 
\index{(64@{\verb?@?}}

    \verb|Entry| is optional and is introduced by the \verb|@| character. If present, this will form the indexed string.
If absent, then the value of \verb|Key| is used instead.

\sssclause{See also} \index{genindex@GenIndex!input!see also}
 \index{(43@{\verb?+?}}

    \verb|SeeAlso| is optional and is introduced by the \verb|+| character. 
Within a document, it should only be used once
per entry. If used more than once, then only the last value is taken.

    This is used to produce an entry that refers to another indexed entry in
a \emph{see also \ldots} style.

\sssclause{See} \index{genindex@GenIndex!input!see} 
\index{(42@{\verb?*?}}

    \verb|See| is optional and is introduced by the \verb|*| character. 
Within a document, it should be used only once per
entry. If used more than once, then only the last value is taken.

    This is used to produce an entry that refers to another entry in a
\emph{see \ldots} style. Use of this option prohibits printing page numbers 
for the entry.

\ssclause{Output} \index{genindex@GenIndex!output}

    The program GenIndex sorts the entries into alphabetical order (based on
the \verb|Key| values), and produces a \verb|theindex| for the input data.
Several formatting commands are included in the output file to enable
adjustment of the appearance of the printed index. These are shown in
\tref{tab:indexc}.

\begin{table}
\centering
\caption{GenIndex formatting commands} \label{tab:indexc}
\begin{tabular}{|l|l|} \hline
\textbf{Command} & \textbf{Effect} \\ \hline
\verb|\indexfill|\ixcom{indexfill} & 
    spacing style between main topic and page numbers \\
\verb|\sindexfill|\ixcom{sindexfill} & 
    spacing style between subtopic and page numbers \\
\verb|\ssindexfill|\ixcom{ssindexfill} & 
    spacing between sub-subtopic and page numbers \\
\verb|\indexsee{text}|\ixcom{indexsee} & 
    produces \verb|text| as a \emph{see} entry \\
\verb|\indexseealso{text}|\ixcom{indexseealso} & 
    produces \verb|text| as a \emph{see also} entry \\
\verb|\otherindexspace{c}|\ixcom{otherindexspace} & 
    heading for non-alphabetic entry \\
\verb|\alphaindexspace{c}|\ixcom{alphaindexspace} & 
    heading for an alphabetic entry  \\ \hline
\end{tabular}
\end{table}


\begin{note}The commands in \tref{tab:indexc} are defined in the 
      \file{iso}\ixclass{iso} class file.
\end{note}
\begin{note}If the commands are not defined in your system, 
      then you can define them
      using the \latex{} \verb|\newcommand|\ixcom{newcommand}
      command. On the other hand, if they are defined in your system, 
      you can change them using the 
      \latex{} \verb|\renewcommand|\ixcom{renewcommand} command.
\end{note} % end note


\begin{example}
This input file:
\begin{verbatim}
\indexentry{Freddy+Fred}{27}
\indexentry{Frederick*Fred}{29}
\indexentry{Fred}{42}
\indexentry{Fred}{52}
\indexentry{Fred}{43}
\end{verbatim}
will produce output like:
\begin{verbatim}
\begin{theindex}
\alphaindexspace{F}
\item Fred \indexfill 42--43, 52
\item Frederick \indexsee{Fred}
\item Freddy \indexfill 27 \indexseealso{Fred}
\end{theindex}
\end{verbatim}
\end{example} % end example

\sssclause{Indexfill commands}

    The commands \verb|\indexfill{|\meta{style}\verb|}|\ixcom{indexfill}, 
\verb|\sindexfill{|\meta{style}\verb|}|\ixcom{sindexfill} and 
\verb|\ssindexfill{|\meta{style}\verb|}|\ixcom{ssindexfill}
control the style of spacing between an indexed entry and its corresponding
page numbers.

\begin{example}
If you want to have the page numbers right justified, with lines between the
main topics and their numbers, dots between sub-topics and pages, and 
sub-subtopic numbers right justified, then you could define these commands as:
\begin{verbatim}
\newcommand{\indexfill}{\hrulefill}
\newcommand{\sindexfill}{\dotfill}
\newcommand{\ssindexfill}{\hfill}
\end{verbatim}
\end{example} % end example


\begin{anote}The usual \latex{} style is to have a small gap between the
topic and page number, the whole being set ragged right.
If this is what you want, then define the commands as:
\begin{verbatim}
\newcommand{\indexfill}{}
\newcommand{\sindexfill}{}
\newcommand{\ssindexfill}{}
\end{verbatim}
\end{anote} % end note

\sssclause{Indexsee and indexseealso commands}

    The \verb?\indexsee{?\meta{text}\verb?}?\ixcom{indexsee} and 
\verb?\indexseealso{?\meta{text}\verb?}?\ixcom{indexseealso} commands 
control the appearance
of the \emph{see} and \emph{see also} text. 


\begin{example}
These commands could be defined as:
\begin{verbatim}
\newcommand{\indexsee}[1]{\par \hspace*{2em} \emph{see} #1}
\newcommand{\indexseealso}[1]{\par \hspace*{2em} \emph{see also} #1}
\end{verbatim}
\end{example} % end example

\sssclause{Index space commands}
\ixcom{alphaindexspace}
\ixcom{otherindexspace}

    The commands 
\verb?\alphaindexspace{?\meta{c}\verb?}?\ixcom{alphaindexspace} and 
\verb?\otherindexspace{?\meta{c}\verb?}?\ixcom{otherindexspace}
control the amount of space between blocks of index entries. 
These are an extension of the
\latex{} \verb|\indexspace|\ixcom{indexspace}
command, which just inserts some vertical space into the index listing.
Both these commands take a single parameter, which is typically a single
character.

\begin{anote}
GenIndex looks at the first character (call it \verb|c|) of the \verb|Key| 
and if this changes
as it processes the ordered list of main topics, it puts that character
as the parameter for these commands. It writes \verb|\alphaindexspace{c}| if
the character is alphabetic (and \verb|c| is put into upper-case), 
otherwise it writes \verb|\otherindexspace{c}|.
\end{anote} % end note

\begin{example}
These commands could be defined as:
\begin{verbatim}
\newcommand{\alphaindexspace}[1]{\indexspace
                       {\bfseries #1}}
\newcommand{\otherindexspace}[1]{}
\end{verbatim}
which would result in the printing of a vertical space and a bold font text
for an alphabetic header, or nothing for anything else.
\end{example} % end example

\ssclause{Running GenIndex} \index{genindex@GenIndex!run}

    To run GenIndex, first obtain a copy of the program, and compile it if
necessary. Execute the program on your data.

    GenIndex takes zero, one or two file names as parameters. If no files are
given then input and output is from and to \verb|stdin| and \verb|stdout|
respectively.

    If one file is given, then input is taken from this file, and output
is to \verb|stdout|.

    If two file names are given, then input is taken from the first and output
is to the second.

\begin{example}
A typical usage would be:
\begin{verbatim}
GenIndex fred.idx fredidx.tex
\end{verbatim}
which would read the \file{fred.idx} file and write the resulting index data to
file \file{fredidx.tex}.
\end{example} % end example

\end{comment}

% sgmlannx.tex    latex and SGML

\infannex{LaTeX, the Web, and *ML} \label{anx:sgml} \index{SGML}

    ISO are becoming more interested in electronic sources for their
standards as well as the traditional camera-ready copy. Acronyms like
PDF, HTML, SGML and XML have been bandied about. Fortunately documents
written using \latex{} are well placed to be provided in a variety of 
electronic formats. A comprehensive treatment of \latex{} with respect
to this topic is provided by Goossens and Rahtz~\bref{lwebcom}.

    SGML (Standard Generalized Markup Language) is a document tagging 
language that is described in ISO~8879~\bref{sgml} and whose usage is described 
in~\bref{bryan}, among others. The principal
mover behind SGML is Charles Goldfarb from IBM, who has authored a detailed 
handbook~\bref{goldfarb} on the SGML standard.

    The concepts lying behind both \latex{} and SGML are similar, but on the face
of it they are distinctly different in both syntax and capabilities. ISO is
migrating towards electronic versions of its standard documents and, naturally, 
would prefer these to be SGML tagged. 
     Like \latex, SGML has a
concept of style files, which are termed DTDs, and both systems support
powerful macro-like capabilities. SGML provides for logical document
markup and not typesetting --- commercial SGML systems often use
\TeX{} or \latex{} as their printing engine, as does the NIST SGML
environment for ISO~10303~\bref{pandl}.



NIST have SGML tagged some ISO~10303 documents 
using manual methods, which are time consuming and expensive. 
About 1997 there was a NIST 
effort underway to develop an auto-tagger that would (semi-) automatically convert
a \latex{} tagged document to one with SGML tags. This tool assumed a
fixed set of \latex{} macros and a fixed DTD.
 The design of an auto-tagger
essentially boils down to being able to convert from a source document tagged
according to a \latex{} style file to one which is tagged according to an
SGML DTD.
    Fully automatic conversion is really only possible if the authors'
of the documents to be translated avoid using any `non-standard' macros within
their documents. There is a program called \file{ltx2x}\index{ltx2x} available
from SOLIS, which replaces \latex{} commands within a document with
user-defined text strings~\bref{ltx2x}. This can be used as a basis for
a \latex{} to whatever auto-tagger, provided the \latex{} commands are not
too exotic.

    HTML is a simple markup language, based on SGML, and is used for the
publication of many documents on the Web. XML is a subset of SGML and appears
to being taken up by every man and his dog as \emph{the} document markup
language. HTML is being recast in terms of XML instead of SGML. PDF is a page
description language that is a popular format for display of documents 
on the Web.

    \latex{} documents can be output in PDF by using pdfLaTeX. Instead
of a \file{.dvi} file being produced a \file{.pdf} file is output directly.
The best 
results are obtained when PostScript fonts rather than Knuth's cm fonts 
are used. Noting that the \file{iso} class provides an \verb|\ifpdf| command,
a general form for documents to be processed by either \latex{} or pdfLaTeX
is
\begin{verbatim}
\documentclass{isov2}
\usepackage{times}     % PostScript fonts Times, Courier, Helvetica
\ifpdf
  \pdfoutput=1         % request PDF output
  \usepackage[pdftex]{graphicx}
\else
  \usepackage{graphicx}
\fi
...
\end{verbatim}

    There are several converters available to transform a \latex{} document 
into an HTML document, but like \file{ltx2x} they generally do their own
parsing of the source file, and unlike \file{ltx2x} are typically limited
to only generating HTML. Eitan Gurari's \file{TeX4ht}\index{TeX4ht} 
suite is a notable
exception (see Chapter~4 and Appendix~B of~\bref{lwebcom}). It uses the 
\file{.dvi} file as input, so that all the parsing is done by \TeX, and can be
configured to generate a wide variety of output formats.
A set of \file{TeX4ht} configuration files are available for converting
ISO \latex{} documents into HTML\footnote{Later, configuration files for XML
output will be developed.}.

    Some points to watch when writing \latex{} documents that will assist
in translations into *ML are given below. Typically, attention to these points
will make it easier to parse the \latex{} source.

\begin{itemize}
\item Avoid using the \verb|\label|\ixcom{label} command within
      clause headings or captions. It can just as easily be placed immediately
      after these constructs.
\item Avoid using the \verb|\index|\ixcom{index} command within
      clause headings or captions. It can just as easily be placed immediately
      after these constructs.
\end{itemize}






\infannex{Obtaining LaTeX and friends} \label{anx:getstuff}

    \latex{} is a freely available document typesetting system. There are many
public domain additions to the basic system.
The information below gives pointers to where
you can obtain \latex{} etc., from the\index{Internet} Internet. 

    \latex{} runs on a wide variety of hardware, from PCs to Crays. 
Source to build a \latex{} system is freely available via anonymous 
ftp\index{ftp} from what is called CTAN\index{CTAN} 
(Comprehensive \tex{} Archive Network). 
There are three sites; pick the one nearest to you.
\begin{itemize}
\item \url{ftp.dante.de} CTAN in Germany;
\item \url{ftp.tex.ac.uk} CTAN in the UK;
\item \url{ctan.tug.org} CTAN in the USA;
\end{itemize}
The top level CTAN directory for \latex{} and friends is 
\url{/tex-archive}. CTAN contains a wide variety
of (La)TeX sources, style files, and software tools and scripts 
to assist in document processing.

\begin{anote}
    CTAN is maintained by the \tex{} Users Group (TUG). Their homepage
\isourl{http://www.tug.org} should be consulted for the current
list of CTAN sites and mirrors.
\end{anote}

\begin{comment}

\sclause{SOLIS} \index{SOLIS} \label{anx:solis}

    SOLIS is the \emph{SC4 On Line Information Service}. It contains many electronic
sources of STEP related documents. The relevant top level directory is
\url{pub/subject/sc4}.
 In particular, SOLIS contains the source for this document
and the \file{.sty} files, as well as other \latex{} related files. 
The \latex{} root directory is \url{sc4/editing/latex}. The latest
versions of the \latex{} related files are kept in the sub-directory 
\url{latex/current}.
Some \latex{} related programs are also available in the 
\url{latex/programs} sub-directory.

    SOLIS can be reached at \isourl{http://www.nist.gov/sc4}.

\end{comment}


\infannex{Changes in this release} \label{anx:changes}

    Many of the commands and environments have been redefined in order
to match the change in requirements from the the third to the fourth
edition of the ISO Directives. Usage of these is unaffected.

    The following changes have been made in this release:
\begin{itemize}
\item The \verb|cover|\ixenv{cover} environment has been added;

\item The boolean test \verb|\ifpdf|\ixcom{ifpdf} has been added;

\item The command \verb|\fwdnopatents|\ixcom{fwdnopatents} has been added
      for patent boilerplate in the Foreword.

\item The command \verb|\tpasfwdbp|\ixcom{tpasfwdbp} has been added
      for TS/PAS Foreword boilerplate.

\item The command \verb|\intropatents|\ixcom{intropatents} has been added
      for patent boilerplate in the Introduction.

\item The commands \verb|\pref|\ixcom{pref} and 
      \verb|\pagerefname|\ixcom{pagerefname} have been added.

\item The command \verb|\trwfwdbpi|\ixcom{trfwdbpi} for Foreword boilerplate
      in a Technical Report has been deleted.

\item The \verb|notes|\ixenv{notes} and \verb|examples|\ixenv{examples}
      environments have been deleted.

\item Support is provided for the \file{hyperref}\ixpack{hyperref}
      package.

\end{itemize}



\bibannex
%\label{biblio}

\begin{references}
\reference{LAMPORT, L.,}{LaTeX --- A Document Preparation System,}
            {Addison-Wesley Publishing Co., 2nd edition, 1994.} \label{lamport}
\reference{WILSON, P.R.,}{LaTeX files for typesetting ISO standards:
           Source code,}
           {NISTIR,  National Institute of Standards and Technology,
           Gaithersburg, MD 20899. June 1996.} \label{isoe}
\reference{WILSON, P.R.,}{LaTeX package files for ISO~10303: User manual,}
           {NISTIR,  National Institute of Standards and Technology,
           Gaithersburg, MD 20899. June 1996.} \label{stepsty}
\reference{RAHTZ, S., and BARROCA, L.,}{A style option for rotated
           objects in \latex,}{ TUGBoat, volume 13, number 2, pp 156--180, 
           July 1992.} \label{rahtz}
\reference{GOOSSENS, M., MITTELBACH, F. and SAMARIN, A.,}{%
           The LaTeX Companion,}
           {Addison-Wesley Publishing Co., 1994.} \label{goosens}
\reference{GOOSSENS, M., and RAHTZ, S.,}{%
           The LaTeX Web Companion --- Integrating TeX, HTML and XML,}
           {Addison-Wesley Publishing Co., 1999.} \label{lwebcom}
\reference{CHEN, P. and HARRISON, M.A.,}{Index preparation and
           processing,}{Software--Practice and Experience, 19(9):897--915,
           September 1988.} \label{chen}
%\reference{KOPKA, H. and DALY, P.W.,}{A Guide to LaTeX,}
%           {Addison-Wesley Publishing Co., 1993.} \label{kopka}
%\reference{WALSH, N.,}{Making TeX Work,}{O'Reilly \& Associates, Inc.,
%           103 Morris Street, Suite A, Sebastopol, CA 95472. 1994. } \label{walsh}
\reference{ISO 8879:1986,}{Information processing --- 
                                Text and office systems ---
           Standard Generalized Markup Language (SGML).}{} \label{sgml}
\reference{GOLDFARB, C.F.,}{The SGML Handbook,}
           {Oxford University Press, 1990.} \label{goldfarb}
\reference{BRYAN, M.,}{SGML --- An Author's Guide to the Standard Generalized
           Markup Language,}{Addison-Wesley Publishing Co., 1988. } \label{bryan}
\reference{PHILLIPS, L. and LUBELL, J.,}{An SGML Environment for STEP,}%
          {NISTIR 5515, National Institute of Standards and Technology,
           Gaithersburg, MD 20899. November 1994.} \label{pandl}
\reference{WILSON, P. R.,}{LTX2X: A LaTeX to X Auto-tagger,}%
          {NISTIR, National Institute of Standards and Technology,
           Gaithersburg, MD 20899. June 1996.} \label{ltx2x}
\begin{comment}
\reference{RESSLER, S.,}{The National PDES Testbed Mail Server User's Guide,}
           {NSTIR 4508, National Institute of Standards and Technology,
           Gaithersburg, MD 20899. January 1991.} \label{ressler}
\reference{RINAUDOT, G.R.,}{STEP On Line Information Service (SOLIS),}
          {NISTIR 5511, National Institute of Standards and Technology,
          Gaithersburg, MD 20899. October 1994. } \label{rinaudot}
\reference{KROL, E.,}{The Whole Internet --- User's Guide \& Catalog,}
           {O'Reilly \& Associates, Inc.,
           103 Morris Street, Suite A, Sebastopol, CA 95472. 1993. } \label{krol}
\end{comment}
\reference{WILSON, P.R.,}{The hyphenat package,}%
          {1999. (Available from CTAN)} \label{bib:hyphenat}
\reference{WILSON, P.R.,}{The xtab package,}%
          {1998. (Available from CTAN)} \label{bib:xtab}
\end{references}

    
%%%%%%% here is the index at the end
%%\input{isomanidx}
% isoman.tex   User guide for ISO style files for LaTeX(2e) August 2002
%
\documentclass[wd,letterpaper,copyright]{isov2}
%\documentclass[wd,letterpaper,draft]{isov2}
%\documentclass{isov2}

%%%\usepackage[isoman]{tex4ht}

\usepackage{comment}

%%%\usepackage{times}
\usepackage{isorot}
\usepackage{xtab}
\usepackage{hyphenat}

\ifpdf
  \pdfoutput=1
  \usepackage[plainpages=false,
              pdfpagelabels,
              bookmarksnumbered,
              hyperindex=true
             ]{hyperref}
\fi

\standard{LaTeX for ISO standards}
\yearofedition{2002}
\languageofedition{(E)}

%\partno{3456}
\changemarkstrue

\makeindex

%  Uncomment the following to change the Foreword heading
%\renewcommand{\forewordname}{Avant-propos} % change the Foreword title

\setcounter{tocdepth}{3} % add more levels to table of contents
%
%    remainder of preamble is some special macro definitions
\makeatletter
%   the \meta{} command
%
\begingroup
\obeyspaces%
\catcode`\^^M\active%
\gdef\meta{\begingroup\obeyspaces\catcode`\^^M\active%
\let^^M\do@space\let \do@space%
\def\-{\egroup\discretionary{-}{}{}\hbox\bgroup\it}%
\m@ta}%
\endgroup
\def\m@ta#1{\leavevmode\hbox\bgroup\texttt{<}\textit{#1}\/\texttt{>}\egroup
    \endgroup}
\def\do@space{\egroup\space
    \hbox\bgroup\it\futurelet\next\sp@ce}
\def\sp@ce{\ifx\next\do@space\expandafter\sp@@ce\fi}
\def\sp@@ce#1{\futurelet\next\sp@ce}
%
%
\makeatother
%
% the \latex command
\newcommand{\latex}{LaTeX}
\newcommand{\tex}{TeX}
%
%   the \file{} command
%
\newcommand{\file}[1]{\textsf{#1}}
%
\makeatletter
%   index a command
\newcommand{\bs}{\symbol{'134}}
\newcommand{\ixcom}[1]{\index{#1/ @{\tt \protect\bs #1}}}
%   index an environment
\newcommand{\ixenv}[1]{\index{#1 @{\tt #1} (environment)}}
%   index a starred environment
\newcommand{\ixenvs}[1]{\index{#1s @{\tt #1*} (environment)}}
%   index an option
\newcommand{\ixopt}[1]{\index{#1 @{\tt #1} (option)}}
%   index a package
\newcommand{\ixpack}[1]{\index{#1 @\file{#1} (package)}}
%   index a class
\newcommand{\ixclass}[1]{\index{#1 @\file{#1} (class)}}
%   index in typewriter font
\newcommand{\ixtt}[1]{\index{#1@{\tt #1}}}
%   index LaTeX
\newcommand{\ixltx}{\index{latex@\latex}}
%   index LaTeX 2e
\newcommand{\ixltxe}{\index{latex2e@\latex 2e}}
%   index LaTeX v2.09
\newcommand{\ixltxv}{\index{latex209@\latex{} v2.09}}
\makeatother
%
%   end of preamble
%
\begin{document}


\begin{cover}
\vspace*{4in}
\begin{center}
\Huge\bfseries LaTeX for ISO standards
\end{center}
\begin{center}
\bfseries 2002/08/10
\end{center}
\begin{center}
Peter Wilson \\
\texttt{peter.r.wilson@boeing.com}
\end{center}
\clearpage
\end{cover}
%%%\clearpage

\begin{foreword}
\fwdbp
\fwdnopatents
Annexes~\ref{anx:extraiso} and~\ref{anx:lord}
are an integral part of this document.
Annexes~\ref{anx:indexing}, \ref{anx:sgml}, \ref{anx:getstuff},
and \ref{anx:changes}
are for information only.
\end{foreword}


\begin{introduction}

    This document describes the use of the \file{isov2} \latex{} class 
file and some package files in preparing ISO standard documents.

\sclause*{Overview}

    This document describes a \latex{} class file, called \file{isov2},
 for typesetting ISO standards. It also
provides descriptions of some particular package files (e.g., \file{isorot}) 
that have been developed to support the writing of ISO standards.
The electronic source of this  document also provides an example of the
use of these files.



    The current set of files~\bref{isoe} have been developed by 
Peter Wilson (RPI, CUA and NIST, Boeing)
from files written by Kent Reed (NIST).
In turn, these were revisions of files originally created by 
Phil Spiby (CADDETC, UK), based on early work by 
Phil Kennicott (GE).\footnote{In mid 1994 \latex{} was upgraded from 
version 2.09 to what is called \latex 2e. The files described in 
earlier versions this document were compatible with both versions 
of \latex. Starting with the October 1997 release, 
support was withdrawn from any \latex{} v2.09 file versions.}


    Documents produced with the \latex{} files have been reviewed 
by the ISO Editorial Board in Geneva for conformance to their typographical
requirements. The first review was of a series of Draft International 
Standard documents. This review resulted in some changes to the style 
files. The second review was of a series of twelve International Standard 
documents (ISO 10303:1994). Likewise, this review led to changes in the 
style files to bring the documents into compliance.

    With the publication of the ISO~10303:1994 standard, the opportunity 
was taken
to provide a new baseline release of the package files.
The new baseline release was also designed to address the fact that
a major update of \latex{} to \latex 2e took place during 1994. \latex 2e
is now the officially supported version. However, some users needed time
to convert to using \latex 2e. Consequently, although the 
facilities described in the original version of this document 
were common for both 
\latex{} v2.09 and \latex 2e users, there were two different 
implementations.

    Because ISO standard documents have a very structured layout, the class
and package files have been designed to reflect the logical document structure
to a much greater extent than the `standard' \latex{} files. Further, ISO
documents are published in more than one language. The files described
here are written for the English language, but the language-specific elements
have been parameterized for easy modification for publication in other
official ISO languages, such as French.

    In 1997 ISO produced a new version of their Directives on the 
requirements for the layout of ISO documents. These were not completely
unambiguous as to their intent; the current version was released in
2001. Members of ISO TC184/SC4 have worked with
the ISO Editorial Board and have reached an agreement that more precisely
identifies the requirements. The current version of the \latex{} files 
implements that agreement.

\begin{note}
The original of this manual has been typeset using the \verb?draft?
option in order to display its effect of placing a black box at the 
end of each line that is longer than the normal width of the text.
\end{note}

\begin{note}
The author of this document may be contacted at 
\texttt{peter.r.wilson@boeing.com}.
\end{note}

\end{introduction}

\title{LaTeX for ISO standards: User manual}%
{Peter Wilson}%
{August 2002}


\scopeclause

This manual describes a set of \ixltx\latex{} files for typesetting 
documents according to the ISO Directives Part 3 (third edition, 1997), 
together with some elements from the ISO~10303 Supplementary Directives.

\begin{inscope}{manual}
\item use of \latex{} for preparing ISO standard documents.
\end{inscope}

\begin{outofscope}{manual}
\item use of \latex{} in general;
\item use of other document preparation systems.
\end{outofscope}

\textbf{IMPORTANT:} No matter whether or not there are copyright attributions
to ISO, this document is \emph{not} copyrighted by ISO. Any copyright
markings are for illustrative purposes only. This document is released under
the LaTeX Project Public Licence.

\normrefsclause \label{sec:nrefs}

\normrefbp{manual}
\begin{nreferences}

\isref{ISO/IEC Directives, Part 2}{Rules for the structure and drafting of
       International Standards. (Fourth edition, 2001)}

\isref{ISO TC 184/SC4 N1217:2001(E)}{SC4 Supplementary directives --- Rules
       for the structure and drafting of SC4
       standards for industrial data. (2001--11--01).}

\isref{ISO/IEC 8824-1:1998}{Information technology ---
       Abstract Syntax Notation One (ASN.1):
       Specification of basic notation.}

\disref{ISO}{Camera-ready copy (CRC) ---
       Submission requirements and ISO type specification.
       (Version 1.0, 1996-04-26)}

\end{nreferences}

\defabbclause
%\clause{Terms, definitions, and abbreviations}

\defsubclause
%\sclause{Terms and definitions}

    For the purposes of this manual, the following definitions
apply.

\begin{definitions}
\definition{boilerplate}{text whose wording is fixed and has been agreed
 to be present in a specific type of document} \index{boilerplate}
\definition{style file}{a set of \latex{} macros assembled into a single 
   file with an extension \file{.sty}}
            \index{style file}\ixltx\index{file!.sty}
\definition{package file}{a style file for use with \latex 2e}\ixltxe
\end{definitions}

\abbsubclause
%\sclause{Abbreviations}

    For the purposes of this manual, the following abbreviations
 apply.

\begin{symbols}
\symboldef{CD}{Committee Draft}\index{CD}
\symboldef{DIS}{Draft International Standard}\index{DIS}
\symboldef{FDIS}{Final Draft International Standard}\index{DIS}
\symboldef{IS}{International Standard}\index{IS}
\symboldef{IS-REVIEW}{The documentation style accepted by the ISO 
    Editorial Board review (September 1994) of twelve IS documents 
    (the initial release of ISO~10303) for compliance with ISO 
    typographical and layout requirements.}\index{IS-REVIEW}
\symboldef{ISOD}{ISO/IEC Directives, Part 2}\index{ISOD}\index{ISO/IEC Directives}
\symboldef{PAS}{Publicly Available Specification}\index{PAS}
\symboldef{SD}{SC4 Supplementary directives}\index{SD}\index{Supplementary directives}
\symboldef{TR}{Technical Report}\index{TR}
\symboldef{TS}{Technical Specification}\index{TS}
\symboldef{WD}{Working Draft}\index{WD}
\symboldef{CRC}{The ISO \emph{Camera-ready copy (CRC)} document}\index{CRC}
\symboldef{ToC}{table of contents}\index{ToC}
\end{symbols}

\clause{Conformance requirements}  \label{sec:iconform}

    The \latex{} macro source files shall not be modified.

    If there is a need to modify the macro definitions then the
modifications shall be defined in a separate \file{.sty}\index{file!.sty}
 file (or files), using the
\latex{} \verb|\renewcommand|\ixcom{renewcommand}
and/or the 
\verb|\renewenvironment|\ixcom{renewenvironment}
commands as appropriate. The resulting \file{.sty} file(s) shall then
be called in within the preamble\index{preamble} portion of the 
document to be typeset.

    Author specified \verb|\label{...}| commands shall not start with
the characters \verb|;i| (semicolon and `i'); definition of labels
starting with these characters is reserved for the maintainer of the
facility files.


\fcandaclause
%\clause{Fundamental concepts and assumptions}

    It is assumed that the reader of this document is familiar with the
\latex{} document preparation system.\ixltx

\begin{note}Reference~\bref{lamport} in the bibliography describes the
      \latex{} system.
\end{note}

    The reader is also assumed to be familiar with the ISO/IEC Directives 
Part~2 (ISOD\index{ISOD}). Agreements reached between the ISO Editorial Board
and ISO TC184/SC4 are documented in the SC4 Supplementary Directives
(SD\index{SD}).

    If there are any discrepancies between the layout and wording of this 
document and the requirements of the ISO/IEC Directives Part~2, 
then the requirements in that document shall be
followed for any ISO standard document.

   The \file{isov2}\ixclass{isov2} class requires the 
\file{url}\ixpack{url} package.

\begin{note}
Additional \latex{} facilities specifically designed for ISO~10303 are
defined and described elsewhere~\bref{stepsty}.
\end{note}

    Because of many revisions over the years to the files described
herein, a naming convention has been adopted for them.
The primary name of the file is suffixed by \file{v\#} or \file{\#}, where
\file{\#} is the version number of the file in question. 
All file primary names have been limited to a 
maximum of eight characters.

\begin{note}
    Table~\ref{tab:curfiles} shows the versions of the files that were
current at the time of publication.
\end{note} % end note
\begin{note}
Starting with the October 1997 release, files that were specific to 
\ixltxv\latex{} v2.09 are no
longer either supported or supplied.
\end{note}

\begin{note}
As of 1999, the \file{uschyp}\ixpack{uschyp} package is no longer supported. 
It has been replaced by the 
\file{hyphenat}\ixpack{hyphenat}\index{hyphenat.sty@\file{hyphenat.sty}}
package.
\end{note}

\ixclass{isov2}\index{isov2.cls@\file{isov2.cls}}
%             \index{isonev11.sty@\file{isonev11.sty}}
\ixopt{9pt}\index{iso9.clo@\file{iso9.clo}}
%           \index{iso9.sty@\file{iso9.sty}}
\ixopt{10pt}\index{iso10.clo@\file{iso10.clo}}
%            \index{iso10.sty@\file{iso10.sty}}
\ixopt{11pt}\index{iso11.clo@\file{iso11.clo}}
%            \index{iso11.sty@\file{iso11.sty}}
\ixpack{isorot}\index{isorot.sty@\file{isorot.sty}}
%               \index{isrotne1.sty@\file{isrotne1.sty}}
\ixpack{askinc}\index{askincv1.sty@\file{askincv1.sty}}
\ixpack{xtab}\index{xtab.sty@\file{xtab.sty}}
\ixpack{uschyp}\index{uschyp.sty@\file{uschyp.sty}}
%               \index{uschypne.sty@\file{uschypne.sty}}

\begin{table}
\centering
\caption{Current file versions} \label{tab:curfiles}
\begin{tabular}{|l|l|} \hline
\textbf{Facility} & \textbf{File}          \\ \hline\hline
\file{iso}     & \file{isov2.cls}    \\
9pt option     & \file{iso9.clo}     \\
10pt option    & \file{iso10.clo}    \\
11pt option    & \file{iso11.clo}    \\
\file{isorot}  & \file{isorot.sty}   \\
\file{askinc}  & \file{askincv1.sty} \\
\file{xtab}    & \file{xtab.sty}   \\
\file{hyphenat}  & \file{hyphenat.sty}   \\ \hline
\end{tabular}
\end{table}


\begin{anexample} At the time of 
publication of this document, any references to \file{iso.cls} should be
read as actually referring to \file{isov2.cls}, and similarly for references
to other files.
\end{anexample} % end example


\begin{note}This document is not intended for publication as a standard, 
although it has been laid out in a 
similar, but not necessarily identical, manner.\end{note} % end note


\clause{The \file{iso} class facility} 

    The \latex{} \file{isov2}\ixclass{isov2} class file
is a general file for use in preparing ISO
standard documents using the \latex{} document preparation system.

    As usual, any \latex{} document has the following structure:
\begin{verbatim}
\documentclass[<list of options>]{isov2}
% preamble goes here
\begin{document}
% document body goes here
\end{document}
\end{verbatim}

\sclause{Options}

    The \file{isov2}\ixclass{isov2} class file supports the following options: 
\begin{itemize}
\item \verb|draft|\ixopt{draft} for a draft document where 
      overfull horizontal boxes are marked, marginal notes are allowed,
      and ISO copyright text is not placed in the document;
\item \verb|final|\ixopt{final} the opposite of \verb|draft|
      (this is the default);
\item \verb|letterpaper|\ixopt{letterpaper} for printing 
      on US letter size paper;
\item \verb|a4paper|\ixopt{a4paper} for printing on A4
      size paper (this is the default);
\item \verb|twocolumn|\ixopt{twocolumn} for two column 
      formatting;
\item \verb|onecolumn|\ixopt{onecolumn} for single column
      printing (this is the default);
\item One of \verb|11pt|, \ixopt{11pt}
      \verb|10pt|, \ixopt{10pt}
      \verb|9pt|\ixopt{9pt} for 11pt, 10pt or 9pt printing
      respectively (the default is \verb|11pt|);
\item \verb|notcopyright|\ixopt{notcopyright} for 
      disabling the printing of copyright notices (this is the default);
\item \verb|copyright|\ixopt{copyright} enables printing
      of copyright notices;
\item \verb|is|\ixopt{is} for International Standard documents;
\item \verb|fdis|\ixopt{fdis} for  Final Draft
      International Standard documents; \index{FDIS}
\item \verb|dis|\ixopt{dis} for  Draft
      International Standard documents; \index{DIS}
\item \verb|cd|\ixopt{cd} for  Committee Draft
      standard documents; \index{CD}
\item \verb|wd|\ixopt{wd} for  Working Draft
      standard documents; \index{WD}
\item \verb|pas|\ixopt{pas} for Publicly Available Specification
      standard documents;
\item \verb|techrep|\ixopt{techrep} for Technical
      Report standard documents; \index{TR}
\item \verb|techspec|\ixopt{techspec} for Technical
      Spefication standard documents; \index{TS}
\item \verb|otherdoc|\ixopt{otherdoc} for documents
      that are not intended to become a standard (this is the default);
%\item \verb|uglycaption|\ixopt{uglycaption} to produce
%      an ugly style of captioning;
\item Any other facilities that are available via packages.
\end{itemize}

    When no options are specified, then the result is 11pt, single column
printing on A4 size paper, without copyright notice and a running header.
That is, by default, the options set are: 
\verb|final|\ixopt{final};
\verb|a4paper|\ixopt{a4paper}; 
\verb|onecolumn|\ixopt{onecolumn};
\verb|11pt|\ixopt{11pt};
\verb|notcopyright|\ixopt{notcopyright}; and
\verb|otherdoc|\ixopt{otherdoc}.


%\begin{note}ISOD\index{ISOD} calls for 9pt double column  printing but the 
%      SD\index{SD} calls for 11pt single column printing.
%      Using 9pt double column printing is awkward if any computer 
%      code has to
%      be typeset in the document. The IS-REVIEW\index{IS-REVIEW} accepted
%      11pt single column layout. \end{note} % end note
%\begin{note}The CRC\index{CRC} states that acceptable founts are: Univers,
%    Helvetica, and Times Roman with the body text in 10pt. 
%    The IS-REVIEW accepted camera-ready copy
%    using Computer Modern 11pt set in single column. \end{note}
\begin{note}The user of the \file{isov2}\ixclass{isov2} class is encouraged 
      to process this document using
      different combinations of the options to gain experience with
      their effects. This printing of the document is typeset using the
      \verb|draft|\ixopt{draft} option.
\end{note} % end note
\begin{note}
    The \verb|otherdoc|\ixopt{otherdoc} option 
was chosen as the default so that the
current stage of ISO standardardization has to be explicitly declared
as an option, and is therefore available to any software package that
might process the document source (e.g., a \latex{} to SGML translator).
\end{note}


\sclause{Sectioning commands}

    Table~\ref{tab:sec} gives the sectioning commands defined for \file{isov2}
class documents.
\ixcom{clause} \ixcom{normannex} \ixcom{infannex} \ixcom{repannex} 
\ixcom{sclause}
\ixcom{ssclause}
\ixcom{sssclause}
\ixcom{ssssclause}
\ixcom{sssssclause}

\begin{table}
\centering
\caption{Sectioning commands.} \label{tab:sec}
\begin{tabular}{|c|l|} \hline
\textbf{Level} & \textbf{Command} \\ \hline\hline
1   & \verb|\clause|, \verb|\normannex|, \verb|\infannex|, \verb|\repannex| \\
2           & \verb|\sclause| \\
3           & \verb|\ssclause| \\
4           & \verb|\sssclause| \\
5           & \verb|\ssssclause| \\
6           & \verb|\sssssclause| \\ \hline
\end{tabular}
\end{table}

%    The \verb|\section|\ixcom{section} command
% is similar to the normal \latex{} \verb|\part|\ixcom{part}
%command. It is only available when the \verb|sect|\ixopt{sect}
%option is specified.

    The \verb|\clause|\ixcom{clause} commands are similar to 
the normal \latex{} \verb|\section|ing 
commands. There are also starred versions of these commands 
(e.g., \verb|\clause*|).

    Three varieties of annex\index{annex} commands are available:
\begin{itemize}
\item \verb+\normannex{+\meta{title}\verb+}+\ixcom{normannex}
      for a normative annex;
\item \verb+\infannex{+\meta{title}\verb+}+\ixcom{infannex}
       for an informative annex;
\item \verb+\repannex{+\meta{title}\verb+}+\ixcom{repannex}
      for an annex that is neither normative or
     informative (e.g. an annex in a technical report).
\end{itemize}

    Only \verb|\sclause|\ixcom{sclause} and lower level 
sectioning commands can be used after
an annex\index{annex} command.


\begin{example}
    The command  \verb|\infannex{Technical discussion}|, assuming that this is the first
annex in the document, produces:
    \begin{center}
    \textbf{Annex A}\\
    (informative)

    \textbf{Technical discussion}
    \end{center}
\end{example} % end example
\begin{example}
  The command \verb|\repannex{Title of annex}|, assuming that this is the second
annex in the document, produces:

\begin{center}
  \textbf{Annex B}

  \textbf{Title of annex}
\end{center}
\end{example} % end example


\ssclause{The use of \texttt{tocdepth} and \texttt{secnumdepth}}

    In the standard \latex{} classes the
\verb|tocdepth|\ixtt{tocdepth} and 
\verb|secnumdepth|\ixtt{secnumdepth} counters
are set in the preamble\index{preamble} to respectively control the level at
which clause titles are inserted into a table of contents (ToC)\index{ToC} 
and at which clause numbering ceases.

    In the \file{isov2}\ixclass{isov2} class, 
the values of these can be changed at 
any point in the document. The change lasts until another change is
made to the value. 
\begin{example}
    Assume that in the preamble we have
\begin{verbatim}
\setcounter{secnumdepth}{3}  % number ssclauses and above
\setcounter{tocdepth}{3}     % ToC includes ssclauses and above
\end{verbatim}
and that a certain subclause has subsubclauses that should be numbered but
not put into the table of contents, then we could do:
\begin{verbatim}
...
\sclause{The certain subclause}
\setcounter{tocdepth}{2}
...
\ssclause{Numbered but not in ToC}
...
...
\setcounter{tocdepth}{3}
\sclause{Following subclause}
\end{verbatim}

\end{example}

    It can sometimes be difficult to remember which level number corresponds
to which kind of clause. Accordingly, a set of commands are provided to ease
this task. These commands can only be used after the preamble.
\begin{itemize}
\item \verb|\maxsecnumdepth{|\meta{sec}\verb|}|\ixcom{maxsecnumdepth}
      sets the level at which clauses will be numbered. This command
      should be used before the first sectioning command.
\item \verb|\maxtocdepth{|\meta{sec}\verb|}|\ixcom{maxtocdepth}
      sets the level at which clauses will be put into the ToC.
      This command should be used before the \verb|\tableofcontents| command.
\item \verb|\setsecnumdepth{|\meta{sec}\verb|}|\ixcom{setsecnumdepth}
      sets the current level at which clauses will be numbered. 
      This command can be used anywhere after the preamble.
\item \verb|\settocdepth{|\meta{sec}\verb|}|\ixcom{settocdepth}
      sets the current level at which clauses will put into the ToC. 
      This command can be used anywhere after the preamble.
\end{itemize}
The value of the argument \meta{sec} can be any of the following:
\texttt{clause}, \texttt{sclause}, \ldots, \texttt{sssssclause}.

\begin{example}
Using these commands, the previous example can also be coded as:
\begin{verbatim}
...
\begin{document}
\maxsecnumdepth{ssclause}
\maxtocdepth{ssclause}
...
\sclause{The certain subclause}
\settocdepth{sclause}
...
\ssclause{Numbered but not in ToC}
...
...
\settocdepth{ssclause}
\sclause{Following subclause}
\end{verbatim}

\end{example}



\sclause{LaTeX environments and commands}

    Many of the standard \latex{} environments and commands are available.
In particular, all the normal mathematical typesetting capabilities are
present.
However, there are some additional environments and commands defined.

\ssclause{Lists}

    The standard \latex{} \verb|itemize|,\ixenv{itemize}
\verb|enumerate|\ixenv{enumerate} and \verb|description|\ixenv{description}
environments are provided. The labels in these lists, though, differ from
those normally provided by \latex.

\begin{note}
The ISOD describes only a single level for an itemized list, being marked 
with either an em-dash or a bullet. The SD deprecates the bullet but 
provides four levels, each being marked with an em-dash. These are 
provided in the \file{isov2} class.
\end{note}
\begin{note}
The ISOD allows for two levels of enumerated lists. The SD extends this
to four levels, and these are provided in the \file{isov2} class.
\end{note}

\begin{example}
The list environments provided are shown below as:\ixenv{itemize}
\begin{verbatim}
\begin{itemize}
\item First level itemized element;
  \begin{itemize}
  \item Second level itemized element;
    \begin{itemize}
    \item Third level itemized element;
      \begin{itemize}
      \item Fourth level itemized element.
      \end{itemize}
    \end{itemize}
  \end{itemize}
\end{itemize}
\end{verbatim}

\ixenv{enumerate}
\begin{verbatim}
\begin{enumerate}
\item First level enumerated element;
  \begin{enumerate}
  \item Second level enumerated element;
    \begin{enumerate}
    \item Third level enumerated element;
      \begin{enumerate}
      \item Fourth level enumerated element.
      \end{enumerate}
    \end{enumerate}
  \end{enumerate}
\end{enumerate}
\end{verbatim}

\ixenv{description}
\begin{verbatim}
\begin{description}
\item[Description] a description element. Note that a colon is
                   automatically added to the item label.
\end{description}
\end{verbatim}
and they produce
\begin{itemize}
\item First level itemized element;
  \begin{itemize}
  \item Second level itemized element;
    \begin{itemize}
    \item Third level itemized element;
      \begin{itemize}
      \item Fourth level itemized element.
      \end{itemize}
    \end{itemize}
  \end{itemize}
\end{itemize}

\begin{enumerate}
\item First level enumerated element;
  \begin{enumerate}
  \item Second level enumerated element;
    \begin{enumerate}
    \item Third level enumerated element;
      \begin{enumerate}
      \item Fourth level enumerated element.
      \end{enumerate}
    \end{enumerate}
  \end{enumerate}
\end{enumerate}

\begin{description}
\item[Description] a description element. Note that a colon is
                   automatically added to the item label.
\end{description}

\end{example} % end example

\begin{example}
    Here is a more complex set of lists:

\begin{itemize}
\item First level itemization
  \begin{enumerate}
  \item First level enumeration
    \begin{itemize}
    \item Second level itemization
      \begin{enumerate}
      \item Second level enumeration
        \begin{itemize}
        \item Third level itemization
%          \begin{enumerate}    % At this point the list is too deeply nested.
%          \item Third level enumeration    % For the TeXnophile this is
%          \item 3rd level enumeration      % because the example environment
%         \end{enumerate}                   % is itself a list.
        \item 3rd level itemization
        \end{itemize}
      \item 2nd level enumeration
      \end{enumerate}
    \item 2nd level itemization
    \end{itemize}
  \item 1st level enumeration
  \end{enumerate}
\item 1st level itemization
\end{itemize}

\end{example} % end example

\begin{note}
On occasion, \latex{} objects to a well formed list. Typically, this happens
when a list has only one entry. \index{list!error}
 The \latex{} error message is of the form:
\begin{verbatim}
! Something's wrong--perhaps a missing \item.
\end{verbatim}
Hitting the \verb|<return>| key usually gets \latex{} to run on happily.
\end{note}% end note
\begin{note}
Similarly, you may get the message
\begin{verbatim}
! Too deeply nested
\end{verbatim}
Again, hit \verb|<return>| and processing should continue. However,
the indentation of later lists may not be correct.
\end{note} % end note


\ssclause{Notes}

    
    Two different kinds of environments are provided for typesetting notes.

\sssclause{Numbered notes}

    The environment 
\verb+\begin{note}+ \meta{text} \verb+\end{note}+\ixenv{note}
produces a numbered note whose body consists
of \meta{text}.

\begin{anexample}
The commands
\begin{verbatim}
\begin{note}Numbered note.\end{note}
\end{verbatim}
produce

\begin{note}Numbered note.\end{note}
\end{anexample}

\sssclause{Isolated notes}

    The environment 
\verb+\begin{anote}+ \meta{text} \verb+\end{anote}+\ixenv{anote}
produces an un-numbered note whose body
consists of \meta{text}.

\begin{anexample}
    The commands
\begin{verbatim}
\begin{anote}Plain note.\end{anote}
\end{verbatim}
produce

\begin{anote}Plain note.\end{anote}
\end{anexample} % end example


\ssclause{Examples} \label{ssc:ex}

    Two different kinds of environments are provided for typesetting 
examples.

\sssclause{Numbered examples} \label{cl:numex}

The environment \verb?\begin{example}? \meta{text} \verb?\end{example}?
produces a numbered example whose body consists of \meta{text}.

\begin{anexample}
The commands
\begin{verbatim}
\begin{example}Numbered example. \label{ex:exref} \end{example}
\end{verbatim}
produce

\begin{example}Numbered example. \label{ex:exref} \end{example}
\end{anexample}

\begin{anote}
Numbered examples (and numbered notes) may be identified using the
\verb?\label{...}?\ixcom{label} command, as exhibited in \eref{ex:exref}
in \ref{cl:numex}, and then referred to by the 
\verb?\ref{...}?\ixcom{ref} command.
\end{anote}

\sssclause{Isolated examples}

The environment \verb?\begin{anexample}? \meta{text} \verb?\end{anexample}?
produces an un-numbered example whose body consists of \meta{text}.
\ixenv{anexample}

\begin{anexample}
The commands
\begin{verbatim}
\begin{anexample}Isolated example.\end{anexample}
\end{verbatim}
produce

\begin{anexample}Isolated example.\end{anexample}
\end{anexample}



\ssclause{Bibliographic references}

    Two different kinds of environments are provided for bibliographic 
references. It should be noted that neither of these has anything to do with
BibTeX.\index{BibTeX}

\sssclause{Normative references}

    Normative references are listed in the 
\verb|nreferences|\ixenv{nreferences} environment.

    In this environment, bibliographic entries are denoted by either
\verb+\isref{+\meta{ref}\verb+}{+\meta{title}\verb+}+ or by 
\verb+\disref{+\meta{ref}\verb+}{+\meta{title}\verb+}+. The
\meta{ref} parameter is the number of the standard document and 
the \meta{title} parameter is the title of the standard. 

    Use the \verb|\isref|\ixcom{isref} command for published standards 
and the \verb|\disref|\ixcom{disref}
 command for documents that have not yet been 
finally approved as a standard. This latter command automatically adds a
footnote to the effect  that the document is
to be published.

\begin{example}
The \latex{} source for the references in clause~\ref{sec:nrefs} of 
this document is:
\begin{verbatim}
\begin{nreferences}
\isref{ISO/IEC Directives, Part 2}{Rules for the structure and drafting of
       International Standards. (Fourth edition, 2001)}
...
...
\disref{ISO}{Camera-ready copy (CRC) ---
       Submission requirements and ISO type specification.
       (Version 1.0, 1996-04-26)}
\end{nreferences}
\end{verbatim}
\end{example} % end example


\sssclause{Informative references}

    Informative bibliography elements are listed in the
\verb|references|\ixenv{references} environment. 
Each element in the list is specified as 
\verb+\reference{+\meta{author}\verb+}{+\meta{title}\verb+}{+\meta{publisher}\verb+}+.\ixcom{reference}

\begin{example}
The \latex{} source for the bibliography in one version of this document was:
\begin{verbatim}
\infannex{Bibliography}
\begin{references}
\reference{LAMPORT, L.,}{\latex\/ A Document Preparation System,}%
           {Addison-Wesley Publishing Co., 1986} \label{lamport}
\reference{GOOSSENS, M., MITTELBACH, F. and SAMARIN, A.,}{%
           The \latex\/ Companion,}
           Addison-Wesley Publishing Co., 1994} \label{goosens}
\reference{CHEN, P. and HARRISON, M.A.,}{Index preparation and
           processing,}{Software--Practice and Experience, 19(9):897--915,
           September 1988.} \label{chen}
....
\end{references}

\end{verbatim}
\end{example} % end example

    Informative references may be cited in the text via the \latex{}
\verb|\label|\ixcom{label} and \verb|\ref|\ixcom{ref}
mechanism. Note that \verb|\cite|\ixcom{cite} is not available
for references to bibliographic items. For the purposes of
ISO documents, the command 
\verb+\bref{+\meta{ref}\verb+}+\ixcom{bref}
is supplied which
properly formats a bibliographic reference.

\ssclause{Listing of scope items}

    The \verb|inscope|\ixenv{inscope} and 
\verb|outofscope|\ixenv{outofscope}
environments are provided for itemized listing of elements that
are within and outside the scope of the standard. Each list
element is introduced via the \verb|\item|\ixcom{item} command. 
Some boilerplate
text is also printed that introduces the scope list.

    The environments take one parameter, \meta{text}, that must read
naturally in a sentence of the form: `The following are within/outside
the scope of this \meta{text}:'. The non-parameterized part of this
sentence is specified by the commands 
|\verb|\inscopename|\ixcom{inscopename} and
\verb|\outofscopename|\ixcom{outofscopename}
 respectively for `within' and `outside'.

\begin{example}The following text was printed by the commands shown at the
end of the example.

\begin{inscope}{part of ISO~10303}
\item use of \latex{} for preparing ISO standard documents;
\item use of \latex{} for preparing ISO~10303 documents.
\end{inscope}

\begin{outofscope}{part of ISO~10303}
\item use of \latex{} in general;
\item use of other document preparation systems.
\end{outofscope}

\begin{verbatim}
\begin{inscope}{part of ISO~10303}
\item use of \latex{} for preparing ISO standard documents;
\item use of \latex{} for preparing ISO~10303 documents.
\end{inscope}

\begin{outofscope}{part of ISO~10303}
\item use of \latex{} in general;
\item use of other document preparation systems.
\end{outofscope}
\end{verbatim}

\end{example} % end example

\ssclause{Listing of definitions}

    The \verb|\begin{olddefinitions}| \ldots \verb|\end{olddefinitions}|
\ixenv{olddefinitions} environment is provided for 
listing terms that have been defined within
the normatively referenced documents. Each term in the list is specified as: \\
\verb+\olddefinition{+\meta{phrase}\verb+}{+\meta{supplement}\verb+}+\ixcom{olddefinition}.


    The \verb|\begin{definitions}| \ldots \verb|\end{definitions}|
\ixenv{definitions} environment is provided for listing 
the definitions of terms specific to the
document being written. Each term in the
list is specified as: \ixcom{definition} \\
\verb+\definition{+\meta{phrase}\verb+}{+\meta{definition text}\verb+}+.


\begin{example}A listing of terms defined elsewhere could be specified as:

\begin{verbatim}
\begin{olddefinitions}
\olddefinition{application protocol (AP)}{}
\olddefinition{integrated resource}{}
\end{olddefinitions}
\end{verbatim}
\end{example} % end example
\begin{example}The definition listing earlier in this document was produced by:

\begin{verbatim}
\begin{definitions}
\definition{boilerplate}{text whose wording is fixed ...}
\definition{style file}{a set of \latex{} macros assembled 
                        into a single file}
...
\end{definitions}
\end{verbatim}
\end{example} % end example


\ssclause{Listing of abbreviations}

    The \verb|\begin{symbols}| \ldots \verb|\end{symbols}|\ixenv{symbols}
environment is provided for listing symbols
and abbreviations. Each term (either symbol or abbreviation) in the list is
specified as: \ixcom{symboldef} \\
 \verb+\symboldef{+\meta{symbol}\verb+}{+\meta{definition text}\verb+}+.

\begin{example}The list of abbreviations earlier in this document was specified as:
\begin{verbatim}
\begin{symbols}
\symboldef{DIS}{Draft International Standard}
\symboldef{IS}{International Standard}
...
\end{symbols}
\end{verbatim}
\end{example} % end example

\sclause{Floating bodies}

    \latex{} provides the \verb|figure|\ixenv{figure}
and \verb|table|\ixenv{table} environments.
Captions (produced by the \verb|\caption|\ixcom{caption}
command) increment the figure or
table number and add the caption to the relevant contents listing file.

\index{floats} \index{floats!continuation}

    The command 
\verb+\contcaption{+\meta{text}\verb+}+\ixcom{contcaption}
may be used instead. This command neither increments the number nor adds
anything to the listing files.

\begin{example}
The following code:
\begin{verbatim}
\begin{table}[tbp]
\centering
\caption{Example table in three parts} \label{tab:cont}
\begin{tabular}{|c|c|} \hline
\textbf{Col 1} & \textbf{Col 2} \\ \hline\hline
A & 1 \\
B & 2 \\ \hline
\end{tabular}
\end{table}

\begin{table}[tbp]
\centering
\contcaption{(continued)}
\begin{tabular}{|c|c|} \hline
\textbf{Col 1} & \textbf{Col 2} \\ \hline\hline
C & 3 \\
D & 4 \\ \hline
\end{tabular}
\end{table}

\begin{table}[tbp]
\centering
\contcaption{(concluded)}
\begin{tabular}{|c|c|} \hline
\textbf{Col 1} & \textbf{Col 2} \\ \hline\hline
E & 5 \\
F & 6 \\ \hline
\end{tabular}
\end{table}
\end{verbatim}
produces the three-part \tref{tab:cont}. \end{example} %end example

\begin{table}[tbp]
\centering
\caption{Example table in three parts} \label{tab:cont}
\begin{tabular}{|c|c|} \hline
\textbf{Col 1} & \textbf{Col 2} \\ \hline\hline
A & 1 \\
B & 2 \\ \hline
\end{tabular}
\end{table}

\begin{table}[tbp]
\centering
\contcaption{(continued)}
\begin{tabular}{|c|c|} \hline
\textbf{Col 1} & \textbf{Col 2} \\ \hline\hline
C & 3 \\
D & 4 \\ \hline
\end{tabular}
\end{table}

\begin{table}[tbp]
\centering
\contcaption{(concluded)}
\begin{tabular}{|c|c|} \hline
\textbf{Col 1} & \textbf{Col 2} \\ \hline\hline
E & 5 \\
F & 6 \\ \hline
\end{tabular}
\end{table}

\sclause{Title command}


    The command to produce the title of the standard\ixcom{title}
is 
\verb+\title{+\meta{intro}\verb+}{+\meta{main}\verb+}{+\meta{compl}\verb+}+. 
The command takes three parameters
according to the three elements of the title as specified in the ISO directives.
\begin{enumerate}
\item \meta{intro} The introductory element of the title. This may be empty.
\item \meta{main} The main element of the title.
\item \meta{compl} The complementary element of the title. This may be empty.
\end{enumerate}

\begin{example}
If this were actually meant to be an ISO standard, then the \latex{} source 
for the title of this document might be:
\begin{verbatim}
\title{Industrial automation systems and integration}%
{Product data representation and exchange}%
{Part 3456 : Documentation methods: The LaTeX style
file reference manual}
\end{verbatim}
\end{example} % end example

\begin{anote}There must be a space on either side of the colon separating the
part number and the final part of the title.
\end{anote}% end of note

    Three other commands are used for setting the running header throughout the
document. These shall be placed in the preamble.\index{preamble}

    The command \verb+\standard{+\meta{number of standard}\verb+}+ is used to identify
the standard.\ixcom{standard}

    The command \verb+\yearofedition{+\meta{year}\verb+}+ is used to identify the
\ixcom{yearofedition} year of the edition.

    The command 
\verb+\languageofedition{+\meta{code}\verb+}+\ixcom{languageofedition}
 is used to identify the language of the edition.

\begin{example}
This document is identified by:
\begin{verbatim}
\standard{LaTeX for standards}
\yearofedition{1997}
\languageofedition{(E)}
\end{verbatim}
\end{example} % end of example

    As a convenience the vacuous command \verb+\extrahead+\ixcom{extrahead}
is supplied which will add it's contents, if any, to the header. It may
be used, for example, to add a document number to the header by
\verb+\renewcommand{\extrahead}{Doc number}+. If \verb+\extrahead+ is
modified it must be done in the preamble.

    The \verb|\title|\ixcom{title} command sets the page numbering 
style to be arabic,
starts a new page, numbered 1, and puts the title
at the start of the page. It also puts the appropriate header at the top
of the page, dependending on the particular combination of selected
options.

\begin{anote}
Remember that you have to use the \verb|copyright|\ixopt{copyright}
option to enable printing of copyright symbols and notices.
\end{anote}

\ssclause{The cover}

    For publication, ISO want to be supplied with a document starting
on page iii with the ToC. They supply the cover (page i) and page ii.

It can often be useful to have a cover page for documents while they
are in the process of being prepared for submittal to ISO.
    The \verb+cover+\ixenv{cover} environment is provided for that
purpose. The contents of the environment must not exceed one page and the
contents must have a \verb+\clearpage+\ixcom{clearpage} immediately before
the end. The \verb+cover+ environment also prints page ii, which has a 
copyright notice on it if the document is copyrighted.

    For example, this document starts with:
\begin{verbatim}
\begin{cover}
\vspace*{4in}
\begin{center}
\Huge\bfseries LaTeX for standards
\end{center}
\begin{center}
\bfseries 2001/07/06
\end{center}
\begin{center}
Peter Wilson \\
\texttt{peter.r.wilson@boeing.com}
\end{center}
\clearpage
\end{cover}
\end{verbatim}

\sclause{Cross referencing commands}

    The usual \latex{} \verb+\label+\ixcom{label} and
\verb+\ref+\ixcom{ref} commands are supported. The class also
provides some formatted referencing commands in addition to \verb+\ref+.

    The following commands are useful for referring to figures, clauses etc.
Each takes a parameter that is the parameter of a \latex{} \verb|\label{}|
command.

\begin{itemize}
\item The command \verb|`\aref{anx:lord}'|\ixcom{aref} prints `\aref{anx:lord}' 
      while \verb|`\ref{anx:lord}'|\ixcom{ref}
      prints `\ref{anx:lord}'
\item The command \verb|`\bref{lamport}'|\ixcom{bref} prints `\bref{lamport}'
      while \verb|`\ref{lamport}'|\ixcom{ref}
      prints `\ref{lamport}'
\item The command \verb|`\cref{sec:nrefs}'|\ixcom{cref} prints `\cref{sec:nrefs}'
      while \verb|`\ref{sec:nrefs}'|\ixcom{ref}
      prints `\ref{sec:nrefs}'
\item The command \verb|`\eref{sec:nrefs}'|\ixcom{eref} prints `\eref{sec:nrefs}'
      while \verb|`\ref{sec:nrefs}'|\ixcom{ref}
      prints `\ref{sec:nrefs}'
\item The command \verb|`\fref{sec:nrefs}'|\ixcom{fref} prints `\fref{sec:nrefs}'
      while \verb|`\ref{sec:nrefs}'|\ixcom{ref}
      prints `\ref{sec:nrefs}'
\item The command \verb|`\nref{sec:nrefs}'|\ixcom{nref} prints `\nref{sec:nrefs}'
      while \verb|`\ref{sec:nrefs}'|\ixcom{ref}
      prints `\ref{sec:nrefs}'
\item The command \verb|`\tref{sec:nrefs}'|\ixcom{tref} prints `\tref{sec:nrefs}'
      while \verb|`\ref{sec:nrefs}'|\ixcom{ref}
      prints `\ref{sec:nrefs}'
\item The command \verb|`\pref{sec:nrefs}'|\ixcom{pref} prints
      `\pref{sec:nrefs}'.
\end{itemize}




\sclause{Heading commands}

\ssclause{Foreword}

    The \verb|foreword|\ixenv{foreword} environment specifies
that a table of contents, list of
figures and list of tables be produced, and starts a new unnumbered
clause called Foreword.
Formatting is one column style only and roman page numbering is set. The 
table of contents starts on page iii.

    More precisely, the title of the clause is given by the value of
\verb|\forewordname|\ixcom{forewordname} (see \aref{anx:extraiso}).

\begin{note}
The default style only lists level 1 clauses in the table of contents. If you
need a more detailed listing, then put the command 
\verb|\setcounter{tocdepth}{<level #>}|\ixcom{setcounter}\ixtt{tocdepth}
in the preamble.\index{preamble}
For example \verb|\setcounter{tocdepth}{3}| will produce a contents
listing down to the level of \verb|\ssclause| (see \tref{tab:sec} for
the definitions of clause levels).
\end{note} %end note

    The command \verb+\fwdbp+\ixcom{fwdbp} prints the 
ISO specified boilerplate for
the initial paragraphs of a foreword.

    The boilerplate for a Technical Specification (ISO/TS) or a 
Publicly Available Specification (ISO/PAS) differs from that
for a standard. The command \verb|\tspasfwdbp|\ixcom{tspasfwdbp}
prints some of this boilerplate.
\begin{example}
In this document, \verb|\tspasfwdbp| prints:

\tspasfwdbp
\end{example}
\begin{note}
    The required paragraph immediately following this boilerplate is:

ISO/PAS [or ISO/TS] \ldots was prepared by Technical Committee ISO/TC \ldots
\end{note}

%%    The boilerplate for a Technical Report (TR) also differs from that for
%%a standard.
%%The command \verb|\trfwdbpi|\ixcom{trfwdbpi}
%%prints some of this boilerplate. 
%%\begin{example}
%%In this document, \verb|\trfwdbpi| prints:
%%
%%\trfwdbpi
%%\end{example}

    The \verb|\fwdnopatents|\ixcom{fwdnopatents} 
prints out the Foreword
boilerplate paragraph concerning potential patent rights.
\begin{example}
In this document, \verb|\fwdnopatents| prints:
\fwdnopatents 
\end{example}



\ssclause{Introduction}

    The \verb|introduction|\ixenv{introduction} environment 
starts a new unnumbered clause 
called `Introduction' with one column formatting.

    More precisely, the title of the clause is given by the value of
\verb|\introductionname|\ixcom{introductionname} (see \aref{anx:extraiso}).

    The \verb|\intropatents|\ixcom{intropatents} prints some of the 
patent boilerplate that may be required in the Introduction. 
\begin{anexample}
In this document, \verb|\intropatents| prints:
\fwdnopatents 
\end{anexample}


\ssclause{Scope clause}

    The \verb|\scopeclause|\ixcom{scopeclause} 
command starts a new numbered clause called `Scope', which is given
the label \verb|;i1|\index{;i1}.

    More precisely, the title of the clause is given by the value of
\verb|\scopename|\ixcom{scopename} (see \aref{anx:extraiso}).

\ssclause{Normative references clause}

    The \verb|\normrefsclause|\ixcom{normrefsclause}
command starts a new numbered clause called `Normative references',
which is given the label \verb|;i2|\index{;i2}.

    More precisely, the title of the clause is given by the value of
\verb|\normrefsname|\ixcom{normrefsname} (see \aref{anx:extraiso}).

    The command 
\verb+\normrefbp{+\meta{document identifier}\verb+}+\ixcom{normrefbp} 
prints the
ISO required text for the introduction to the listing of normative
references. The \meta{document identifier} parameter must be such
that it reads sensibly in a sentence of the form `\ldots constitute
provisions of this \meta{document identifier}.'.

\begin{example}Clause~\ref{sec:nrefs} in this document was started by the
commands:
\begin{verbatim}
\normrefsclause \label{sec:nrefs}

\normrefbp{manual}
\begin{nreferences}
...
\end{verbatim}
\end{example} % end example

\ssclause{Definitions, symbols and abbreviations}

    A variety of commands are provided that initiate new numbered
clauses for definitions, symbols and abbreviations. Depending on the
amount of material in these respective categories, one or more clauses
may be used. The commands and clause titles are listed in \tref{tab:dsa}.
The clause level headings are each given the label \verb|;i3|\index{;i3};
one and only one of these headings should appear in a standard document.
\ixcom{defclause} \ixcom{defname}
\ixcom{symclause} \ixcom{symname}
\ixcom{abbclause} \ixcom{abbname}
\ixcom{defsymclause} \ixcom{defsymname}
\ixcom{defabbclause} \ixcom{defabbname}
\ixcom{symabbclause} \ixcom{symabbname}
\ixcom{defsymabbclause} \ixcom{defsymabbname}
\ixcom{defsubclause}
\ixcom{symsubclause}
\ixcom{abbsubclause}
\ixcom{defsymsubclause}
\ixcom{defabbsubclause}
\ixcom{symabbsubclause}

\begin{table}
\centering
\caption{Definition, symbol and abbreviation clause commands}
\label{tab:dsa}
\begin{tabular}{|l|c|l|l|} \hline
\textbf{Command} & \textbf{Clause} & \textbf{Title} & \textbf{Default text} \\ \hline
\verb|\defclause| &       C  & \verb|\defname| & \defname{} \\
\verb|\symclause| &       C  & \verb|\symname| & \symname{}  \\
\verb|\abbclause| &       C  & \verb|\abbname| & \abbname{}  \\
\verb|\defsymclause| &    C  & \verb|\defsymname| & \defsymname{} \\
\verb|\defabbclause| &    C  & \verb|\defabbname| & \defabbname{} \\
\verb|\symabbclause| &    C  & \verb|\symabbname| & \symabbname{} \\
\verb|\defsymabbclause| & C  & \verb|\defsymabbname| & \defsymabbname{} \\
\verb|\defsubclause| &    SC & \verb|\defname| & \defname{} \\
\verb|\symsubclause| &    SC & \verb|\symname| & \symname{}  \\
\verb|\abbsubclause| &    SC & \verb|\abbname| & \abbname{}  \\
\verb|\defsymsubclause| & SC & \verb|\defsymname| & \defsymname{} \\
\verb|\defabbsubclause| & SC & \verb|\defabbname| & \defabbname{} \\
\verb|\symabbsubclause| & SC & \verb|\symabbname| & \symabbname{} \\
 \hline
\multicolumn{4}{|l|}{%
NOTE - In the table, C = clause, SC = subclause. } \\ \hline
\end{tabular}
\end{table}

\sssclause{Heading labels}

    Some of the defined clauses have associated \verb|\label|s. These
heading commands and their \verb|\label| values are listed in 
\tref{tab:clabels}.

\begin{table}
\centering
\caption{Defined clause headings with labels}
\label{tab:clabels}
\begin{tabular}{|l|c|} \hline
\textbf{Command}       &  \textbf{Label} \\ \hline
\verb|\scopeclause|    &  \texttt{;i1} \\
\verb|\normrefsclause| &  \texttt{;i2} \\
\verb|\defclause| &       \texttt{;i3} \\
\verb|\symclause| &       \texttt{;i3} \\
\verb|\abbclause| &       \texttt{;i3} \\
\verb|\defsymclause| &    \texttt{;i3} \\
\verb|\defabbclause| &    \texttt{;i3} \\
\verb|\symabbclause| &    \texttt{;i3} \\
\verb|\defsymabbclause| & \texttt{;i3} \\
 \hline
\end{tabular}
\end{table}



\ssclause{Bibliography}

    The command \verb|\bibannex|\ixcom{bibannex}
 starts an informative section of the document entitled `Bibliography'. 
Or, more precisely, 
by the value of the \verb|\bibname|\ixcom{bibname}
command.

\sclause{Urls, etc}

    The command \verb|\url{|\meta{text}\verb|}|\ixcom{url} can be used for
typesetting \meta{text} as an email address.

    The command \verb|\isourl{|\meta{text}\verb|}|\ixcom{isourl} can be 
used for
typesetting \meta{text} as an URL address.

\begin{example}
The following code
\begin{verbatim}
The Email address is \url{joe@fred.mik} or the homepage is at
\isourl{http://fred.mik/home/}.
\end{verbatim}
will be typeset as: \\
The Email address is \url{joe@fred.mik} or the homepage is at
\isourl{http://fred.mik/home/}.
\end{example}


\sclause{Version control commands}


    A set of commands are provided to assist when revising a document.
For these commands to flag the changes in the printed document the command
\verb|\changemarkstrue|\ixcom{changemarkstrue}
must be put in the preamble.\index{preamble}

    In the commands described below, the \meta{number} parameter can be used 
to correlate changes in a document
to some external (numbered) requirement for the change.

\ssclause{Editorial}

    The command \verb+\editorial{+\meta{number}\verb+}+\ixcom{editorial}
flags an editorial change to the
document text with ED$^{number}$.

\begin{anexample}

   Just to start things off, this is an original sentence, which should
take up about a line.
This example sentence contains an editorial \editorial{37}
change. The command \verb|\editorial{37}| was included in the previous
sentence.
This sentence, though, has no changes marked in it and may be
assumed to be unaltered from a prior version.
\end{anexample} % end example

\ssclause{Added}

    The command 
\verb+\added{+\meta{text}\verb+}{+\meta{number}\verb+}+\ixcom{added}
flags the
position of the additional \meta{text} and highlights it.

\begin{anexample}

   Just to start things off, this is an original sentence, which should
take up about a line.
This example sentence contains \added{some added text}{27} in
the middle of it. The command \verb|\added{some added text}{27}| was
included in the previous sentence.
This sentence, though, has no changes marked in it and may be
assumed to be unaltered from a prior version.
\end{anexample} % end example

\ssclause{Deleted}

    The command \verb+\deleted{+\meta{number}\verb+}+\ixcom{deleted}
flags the position of deleted text.

\begin{anexample}

   Just to start things off, this is an original sentence, which should
take up about a line.
Some text was deleted \deleted{34} from the middle of this sentence.
The command \verb|\deleted{34}| was included in the previous sentence.
This sentence, though, has no changes marked in it and may be
assumed to be unaltered from a prior version.
\end{anexample} % end example

\ssclause{Moved}

    The command 
\verb+\moved{+\meta{text}\verb+}{+\meta{number}\verb+}+\ixcom{moved}
flags the position of
moved \meta{text} and highlights it.

\begin{anexample}

   Just to start things off, this is an original sentence, which should
take up about a line.
This sentence contains some \moved{moved}{81} text in the middle of it.
The command \verb|\moved{moved}{81}| was included in the previous
sentence.
This sentence, though, has no changes marked in it and may be
assumed to be unaltered from a prior version.
\end{anexample} % end example

\sclause{PDF}

    The class provides a command, \verb|\ifpdf|\ixcom{ifpdf}, 
to test whether or not the document is being processed by \latex{} 
or by pdf\latex. \latex{} processing produces a \file{.dvi} file which
needs further processing, usually by \file{dvips}, to generate PostScript
for printing. pdf\latex, however, directly generates a \file{.pdf} file
which can then be printed.

\begin{anexample}
A document for processing by either \latex{} or pdf\latex{} could be 
started like:
\begin{verbatim}
\documentclass{iso}
\usepackage{times}
\ifpdf
  \pdfoutput=1
  \usepackage[pdftex]{graphicx}
\else
  \usepackage{graphicx}
\fi
....
\end{verbatim}

The \file{times} package is used in this example as PDF printers tend
to handle
PostScript fonts better than other kinds of fonts.
\end{anexample}


    The class also supports the use of the 
\file{hyperref}\ixpack{hyperref} package in conjunction with pdf\latex.
Typically bookmark processing would be specifed by:
\begin{verbatim}
\ifpdf
  \pdfoutput=1
  \usepackage[plainpages=false,
              pdfpagelabels,
              bookmarksnumbered,
              hyperindex=true
             ]{hyperref}
  ...
\else
  ...
\fi
....
\end{verbatim}
If the \verb|hyperindex=true| option to the \file{hyperref} package is used
the the index has to be processed via the
\textsc{MakeIndex}\index{makeIndex@{\sc MakeIndex}} program.

%%%%%%%%%%%
%%%\end{document}
%%%%%%%%%%%

\clause{The \file{isorot} package facility} 

   The \file{isorot}\ixpack{isorot} facility enables the rotation 
of document elements on
a page. It uses the \latex{} \verb|\special|\ixcom{special} 
command to perform its
effects, and thus can only be used
with a limited number of dvi to postscript programs. The facilities
available are summarized in \tref{tab4}.

    \file{isorot} is a modification of the \file{rotation.sty} file
created by Rahtz and Barroca~\bref{rahtz}. Further examples of the usage
of their style are given in Goosens \emph{et al}~\bref{goosens}.

\begin{note}Several examples of the effects of the commands described herein
are shown. In many cases the results are not pretty. This should act as
a warning that using rotational elements requires more care than
most other document elements.\end{note} %end note

\begin{sidewaystable}
\ixcom{rotdriver} \ixcom{clockwise} \ixcom{counterclockwise}
\ixcom{figuresright} \ixcom{figuresleft} \ixcom{rotcaption}
\ixcom{controtcaption}
\ixenv{sideways} \ixenv{turn} \ixenv{rotate} \ixenv{sidewaystable}
\ixenv{sidewaysfigure} \ixenv{landscape}
\centering
\caption{The rotation facilities} \label{tab4}
\begin{tabular}{|l|l|} \hline
\textbf{Facility} & \textbf{Effect} \\ \hline
\multicolumn{2}{|c|}{\textbf{Commands}} \\ \hline
\verb|\rotdriver{<driver>}| & 
declare the name of the dvi to Postscript translator (default {\tt dvips}) \\
\verb|\clockwise| & 
sets rotation direction clockwise for positive angles (the default) \\
\verb|\counterclockwise| &
sets rotation direction counterclockwise for positive angles \\
\verb|\figuresright| &
sets rotation direction for sideways floats counterclockwise (the default) \\
\verb|\figuresleft| &
sets rotation direction for sideways floats clockwise \\
\verb|\rotcaption| &
like the \verb|caption| command, but rotates the caption through 90 degrees \\
\verb|\controtcaption| &
like the \verb|contcaption| command, but rotates the caption through 90 degrees \\ \hline
\multicolumn{2}{|c|}{\textbf{Environments}} \\ \hline
\verb|sideways| &
rotates the contents through 90 degrees counterclockwise \\
\verb|turn| &
rotates the contents through the given angle \\
\verb|rotate| &
rotates the contents through the given angle, but no space allowed for the result\\
\verb|sidewaystable| &
like the \verb|table| environment, but rotated 90 degrees \\
\verb|sidewaystable*| &
twocolumn version of \verb|sidewaystable| \\
\verb|sidewaysfigure| &
like the \verb|figure| environment, but rotated 90 degrees \\
\verb|sidewaysfigure*| &
twocolumn version of \verb|sidewaysfigure| \\
\verb|landscape| &
prints all enclosed pages in landscape mode \\ \hline
\end{tabular}
\end{sidewaystable}

\sclause{Options}

    The \file{isorot} facility has one option, 
namely \verb|debugshow|\ixopt{debugshow}. Calling this option produces
messages on the screen and in the \file{log} file regarding the actions 
being taken.

\begin{anote} This option is principally of interest to the maintainer
of the facility. \end{anote}

\sclause{DVI drivers}

    The \file{isorot} facility supports only a limited number of
dvi to postscript translators. The default translator is \emph{dvips}.
 The following command must be put in
the preamble of the document if \emph{dvips} is not being used: 
\verb|\rotdriver{<drivername>}|,\ixcom{rotdriver} where
\verb|<drivername>| is one of the following:\footnote{I have been able to 
try the {\tt dvips} driver
but not the others. If anyone has experience with the other drivers, or has
extended the range of drivers, I would like to be given the results.} %end footnote

\begin{enumerate}
\item \verb|dvipdf| for the \emph{dvipdf} 
      translator;\ixtt{dvipdf}
\item \verb|dvips| for Tom Rockicki's \emph{dvips} 
      translator;\ixtt{dvips}
\item \verb|dvipsone| for Y\&Y's \emph{dvipsone} 
      translator;\ixtt{dvipsone}
\item \verb|dvitops| for James Clark's \emph{dvitops} 
      translator;\ixtt{dvitops}
\item \verb|dviwindo| for Y\&Y's \emph{dviwindo} 
      translator;\ixtt{dviwindo}
\item \verb|pctex32| for Personal TeX's PC TeX for 32 bit Windows 
      (\emph{pctex32})
      translator;\ixtt{pctex32}
\item \verb|pctexps| for Personal TeX's PC PTI Laser/PS (\emph{pctexps})
      translator;\ixtt{pctexps}
\item \verb|pubps| for the Arbortext's \emph{pubps} 
      translator.\ixtt{pubps}
\item \verb|textures| for Blue Sky's \emph{Textures} 
      translator;\ixtt{textures}

\end{enumerate}


\sclause{Rotational directions}

    \file{isorot} enables the textual and other elements of a document
to be rotated from their normal horizontal layout. In some cases elements
can be rotated through arbitrary angles, whereas in others only 90 degree
rotation is possible. 

    By default, a rotation through a positive number of
degrees corresponds to a clockwise rotation. The command 
\verb|\counterclockwise|\ixcom{counterclockwise}
sets the following rotations to be counterclockwise for positive angles.
The command \verb|\clockwise|\ixcom{clockwise}
sets the following rotations to be clockwise for positive angles.
These commands can be used to toggle the rotational behavior.
    

    Rotated floating environments are normally rotated so that they are
printed with a counterclockwise rotation (i.e. the original bottom of the float
is placed at the right hand side of the paper), 
which is what is normally required.
This behavior can be altered by the command
\verb|\figuresleft|,\ixcom{figuresleft}
which will give the reverse effect. The command
\verb|\figuresright|\ixcom{figuresright}
will set the behavior to the default.
These commands can be used to toggle the rotational behavior of
floats.


\sclause{Rotation of text}

    The \verb|sideways|\ixenv{sideways}
environment rotates the contents of the environment
by 90 degrees counterclockwise, and leaves space for the result.

    The \verb|\begin{turn}{|\meta{angle}\verb|}|\ixenv{turn}
environment rotates the contents by the given number
of degrees in the direction specified by the most recent of the
\verb|\clockwise|\ixcom{clockwise} or
\verb|\counterclockwise|\ixcom{counterclockwise}
commands, leaving space for the result.

    The \verb|\begin{rotate}{|\meta{angle}\verb|}|\ixenv{rotate}
environment rotates the contents by the given number
of degrees in the direction specified by the most recent of the
\verb|\clockwise|\ixcom{clockwise} or
\verb|\counterclockwise|\ixcom{counterclockwise}
commands, but no arrangements are made for leaving space for the result.


\begin{example}Some simple rotations: \label{ex:1}

This code
\begin{verbatim}
Default rotation direction: \\
A 
\begin{sideways}%
B C
\end{sideways}
D E F G H I J K L M
\begin{turn}{-90}%
Minus 90 turn
\end{turn}
N O P
\begin{rotate}{90}%
Plus 90 rotate
\end{rotate}
Q \\
and continue on with another line after rotations.
\end{verbatim}
produces the following (note how space is allowed for the \verb|turn|ed 
text, whereas the \verb|rotate|d text runs into the text below).

Default rotation direction: \\
A 
\begin{sideways}%
B C
\end{sideways}
D E F G H I J K L M
\begin{turn}{-90}%
Minus 90 turn
\end{turn}
N O P
\begin{rotate}{90}%
Plus 90 rotate
\end{rotate}
Q \\
and continue on with another line after rotations.
\end{example} % end example
\begin{example}This example shows the effect of using the 
\verb|\counterclockwise|\ixcom{counterclockwise}
command.

This code
\begin{verbatim}
Flip rotation direction: \\
\counterclockwise
A 
\begin{sideways}%
B C
\end{sideways}
D E F G H I J K L M
\begin{turn}{-90}%
Minus 90 turn
\end{turn}
N O P
\begin{rotate}{90}%
Plus 90 rotate
\end{rotate}
Q \\
Set rotation direction back to default value.
\clockwise
\end{verbatim}

produces the following, which should be compared with example~\ref{ex:1}.

Flip rotation direction: \\
\counterclockwise
A 
\begin{sideways}%
B C
\end{sideways}
D E F G H I J K L M
\begin{turn}{-90}%
Minus 90 turn
\end{turn}
N O P
\begin{rotate}{90}%
Plus 90 rotate
\end{rotate}
Q \\
Set rotation direction back to default value.
\clockwise
\end{example} % end example


    Although the examples so far have only shown the rotation of text, boxes
can also be rotated.

\begin{example}Rotating a box.

This code
\begin{verbatim}
\newsavebox{\foo}
\newlength{\fool}
\settowidth{\fool}{Hurrah for ISO.}
\savebox{\foo}{\parbox{\fool}{Hurrah for ISO. Hurrah for ISO.
                              Hurrah for ISO. Hurrah for ISO.}}
Start
\usebox{\foo}
\&
\begin{turn}{-45}\usebox{\foo}\end{turn}
\&
\begin{turn}{45}\usebox{\foo}\end{turn}
End
\end{verbatim}
produces:

\newsavebox{\foo}
\newlength{\fool}
\settowidth{\fool}{Hurrah for ISO.}
\savebox{\foo}{\parbox{\fool}{Hurrah for ISO. Hurrah for ISO.
                              Hurrah for ISO. Hurrah for ISO.}}
Start
\usebox{\foo}
\&
\begin{turn}{-45}\usebox{\foo}\end{turn}
\&
\begin{turn}{45}\usebox{\foo}\end{turn}
End

\end{example} %end example

    Elements can be rotated through arbitrary angles, and also rotated
elements can be nested inside other rotated elements.


\begin{example}Repeated rotation:

    The following example code shows that text can be rotated through any angle.
The result is shown in \fref{fig:wheel}.
\begin{verbatim}
\newcount\prwc
\newsavebox{\prwtext}
\newdimen\prwspace
\def\wheel#1#2{%
  \savebox{\prwtext}{#1\begin{sideways}#2\end{sideways}}%
  \prwspace\wd\prwtext%
  \advance\prwspace by 1cm%
  \centerline{%
  \rule{0pt}{\prwspace}%
  \rule[-\prwspace]{0pt}{\prwspace}%
  \prwc=-180\loop\ifnum\prwc<180
  \rlap{\begin{rotate}{\the\prwc}%
  \rule{1cm}{0pt}\usebox{\prwtext}\end{rotate}}%
  \advance\prwc by 20\repeat}}
\begin{figure}
\wheel{Express yourself ---}{Hooray for STEP!}
\caption{Example rotation through multiple angles}
\label{fig:wheel}
\end{figure}
\end{verbatim}

\newcount\prwc
\newsavebox{\prwtext}
\newdimen\prwspace
\def\wheel#1#2{%
  \savebox{\prwtext}{#1\begin{sideways}#2\end{sideways}}%
  \prwspace\wd\prwtext%
  \advance\prwspace by 1cm%
  \centerline{%
  \rule{0pt}{\prwspace}%
  \rule[-\prwspace]{0pt}{\prwspace}%
  \prwc=-180\loop\ifnum\prwc<180
  \rlap{\begin{rotate}{\the\prwc}%
  \rule{1cm}{0pt}\usebox{\prwtext}\end{rotate}}%
  \advance\prwc by 20\repeat}}
\begin{figure}
\vspace*{1cm}
\wheel{Express yourself ---}{Hooray for STEP!}
\vspace*{1cm}
\caption{Example rotation through multiple angles}
\label{fig:wheel}
\end{figure}

    Figures~\ref{fig:angles1} and~\ref{fig:angles2} also show rotations through a
range of angles, both positive and negative.
\end{example} %end example

\begin{example}Nested rotations. \label{ex:sidetabular}

This code
\begin{verbatim}
    Here is some text before a \verb|sideways| environment. 
And some more, and more and more garble gobble cluck
click clack clock cluck and so on and on and on.
\begin{center}
\begin{sideways}
\rule{1in}{0pt}
\begin{tabular}{|lr|}
\begin{rotate}{-45}\emph{Word}\end{rotate} & \begin{rotate}{-90}%
Occurrences\end{rotate}
\\
\hline
hello & 33 \\
goodbye & 34 \\
\hline
\end{tabular}
\end{sideways}
\end{center}
    Here is some text after a \verb|sideways| environment.
 And some more, and more and more garble gobble cluck
click clack clock cluck and so on and on and on.
\end{verbatim}
produces:

    Here is some text before a \verb|sideways| environment. 
And some more, and more and more garble gobble cluck
click clack clock cluck and so on and on and on.
\begin{center}
\begin{sideways}
%\rule{1in}{0pt}
\begin{tabular}{|lr|}
\begin{rotate}{-45}\emph{Word}\end{rotate} & \begin{rotate}{-90}%
Occurrences\end{rotate} \\ \hline
hello & 33 \\
goodbye & 34 \\ \hline
\end{tabular}
\end{sideways}
\end{center}
    Here is some text after a \verb|sideways| environment.
 And some more, and more and more garble gobble cluck
click clack clock cluck and so on and on and on.
\end{example} %end example



\sclause{Rotations of tables and figures}

    The previous examples have demonstrated the rotation of textual elements.
For instance, example~\ref{ex:sidetabular} shows that tabular material can be rotated using
the \verb|sideways|\ixenv{sideways}
environment. (Actually, any of the previously
mentioned environments may be used.)
Two further environments are provided which rotate a \latex{} float through
90 degrees. These are:
\begin{itemize}
\item \verb|sidewaystable|\ixenv{sidewaystable}, which
  corresponds to the standard \latex{} \verb|table|\ixenv{table}
  environment; and
\item \verb|sidewaysfigure|\ixenv{sidewaysfigure}, which
  corresponds to the standard \latex{} \verb|figure|\ixenv{figure}
  environment.
\end{itemize}
There are also starred versions of these, namely 
\verb|sidewaystable*|\ixenvs{sidewaystable} and
\verb|sidewaysfigure*|\ixenvs{sidewaysfigure}, for use in twocolumn mode.
However, the correspondence with the standard environments is not strictly
complete as a sideways float is alway placed on a page by itself.

    The direction of rotation may be controlled by the 
\verb|\figuresright|\ixcom{figuresright} and
\verb|\figuresleft|\ixcom{figuresleft} commands.

\begin{example}Table~\ref{tab4} is produced by the code below: \label{ex:4}

\begin{verbatim}
\begin{sidewaystable}
\centering
\caption{The rotation facilities} \label{tab4}
\begin{tabular}{|l|l|} \hline
\textbf{Facility} & \textbf{Effect} \\ \hline
\multicolumn{2}{|c|}{\textbf{Commands}} \\ \hline
\verb|\rotdriver{<driver>}| & 
declare the name of the dvi to Postscript translator (default {\tt dvips}) \\
......
\verb|sidewaysfigure| &
like the \verb|figure| environment, but rotated 90 degrees \\ \hline
\end{tabular}
\end{sidewaystable}
\end{verbatim}
\end{example} % end example



\sclause{Rotation of float captions and bodies}

    Sometimes it may be useful to rotate a caption independently of the
rotation of a figure or table. The command 
\verb|\rotcaption|\ixcom{rotcaption} is analogous
to the normal \verb|\caption|\ixcom{caption} command, 
and inserts the caption rotated
by 90~degrees. There is also the companion command 
\verb|\controtcaption|\ixcom{controtcaption}, analagous to the
\verb|\contcaption|\ixcom{contcaption} command,
for continuation captions.

%\newsavebox{\picbox}

\begin{figure}
\centering
\caption{Example figure with a standard caption.} \label{fig:nocrot}
\setlength{\unitlength}{0.2in}
\footnotesize
\begin{picture}(17,2)
\thicklines
\put(0,0){\begin{picture}(4,1)
  \put(1.5,0.5){\oval(3,1)}
  \put(1.5,0.5){\makebox(0,0){2,5 (1)}}
  \put(3,0.5){\line(1,0){1.0}}
  \put(4.25,0.5){\circle{0.5}}
  \end{picture}}

\put(4.5,0){\begin{picture}(8,1)
  \put(0,0){\dashbox{0.25}(4,1){date}}
  \put(4,0.5){\line(1,0){3.5}}
  \put(7.75,0.5){\circle{0.5}}
  \put(6,1){\makebox(0,0){A[1:3]}}
  \end{picture}}

\put(12.5,0){\begin{picture}(4,1)
  \put(0,0){\framebox(4,1){INTEGER}}
  \put(3.75,0){\line(0,1){1}}
  \end{picture}}
\end{picture}
\normalsize
\setlength{\unitlength}{1pt}
\end{figure}


\begin{example}Float with a regular caption.

Figure~\ref{fig:nocrot} is produced by the code below:
\begin{verbatim}
\begin{figure}
\centering
\caption{Example figure with a standard caption.} \label{fig:nocrot}
\setlength{\unitlength}{0.2in}
\footnotesize
\begin{picture}(17,2)
\thicklines
\put(0,0){\begin{picture}(4,1)
  \put(1.5,0.5){\oval(3,1)}
  \put(1.5,0.5){\makebox(0,0){2,5 (1)}}
  \put(3,0.5){\line(1,0){1.0}}
  \put(4.25,0.5){\circle{0.5}}
  \end{picture}}
\put(4.5,0){\begin{picture}(8,1)
  \put(0,0){\dashbox{0.25}(4,1){date}}
  \put(4,0.5){\line(1,0){3.5}}
  \put(7.75,0.5){\circle{0.5}}
  \put(6,1){\makebox(0,0){A[1:3]}}
  \end{picture}}
\put(12.5,0){\begin{picture}(4,1)
  \put(0,0){\framebox(4,1){INTEGER}}
  \put(3.75,0){\line(0,1){1}}
  \end{picture}}
\end{picture}
\normalsize
\setlength{\unitlength}{1pt}
\end{figure}
\end{verbatim}
\end{example} % end example

\begin{example}Float with a rotated caption.

Figure~\ref{fig:crot} is produced by the code below:
\begin{verbatim}
\begin{figure}
\centering
\rotcaption{Figure~\protect\ref{fig:nocrot} with a rotated caption.}
 \label{fig:crot}
\setlength{\unitlength}{0.2in}
\footnotesize
\begin{picture}(17,2)
...
\end{picture}
\normalsize
\setlength{\unitlength}{1pt}
\end{figure}
\end{verbatim}
\end{example} % end example


\begin{figure}
\centering
\rotcaption{Figure~\protect\ref{fig:nocrot} with a rotated caption.}
 \label{fig:crot}
\setlength{\unitlength}{0.2in}
\footnotesize
\begin{picture}(17,2)
\thicklines
\put(0,0){\begin{picture}(4,1)
  \put(1.5,0.5){\oval(3,1)}
  \put(1.5,0.5){\makebox(0,0){2,5 (1)}}
  \put(3,0.5){\line(1,0){1.0}}
  \put(4.25,0.5){\circle{0.5}}
  \end{picture}}
\put(4.5,0){\begin{picture}(8,1)
  \put(0,0){\dashbox{0.25}(4,1){date}}
  \put(4,0.5){\line(1,0){3.5}}
  \put(7.75,0.5){\circle{0.5}}
  \put(6,1){\makebox(0,0){A[1:3]}}
  \end{picture}}
\put(12.5,0){\begin{picture}(4,1)
  \put(0,0){\framebox(4,1){INTEGER}}
  \put(3.75,0){\line(0,1){1}}
  \end{picture}}
\end{picture}
\normalsize
\setlength{\unitlength}{1pt}
\end{figure}

    As can be seen from \fref{fig:crot} the advisability of rotating a caption
depends on the size of the body of the float. It may be better in certain
cases to leave the caption in its regular position and rotate the body of
the float instead.

\def\prwrot#1{%
\settowidth{\fool}{ISOROT}
\savebox{\foo}{\parbox{\fool}{ISOROT ISOROT ISOROT ISOROT}}%
\framebox{---\begin{turn}{#1}\framebox{\usebox{\foo}}\end{turn}---}}%
\def\degrees{{\small$^{o}$}}

\begin{figure}
\centering
\begin{tabular}{|c|c|c|} \hline
\prwrot{0} &\prwrot{-40}&\prwrot{-80}\\
0\degrees & -40\degrees & -80\degrees \\ \hline
\prwrot{-120}&\prwrot{-160}&\prwrot{-200}\\
-120\degrees & -160\degrees & -200\degrees \\ \hline
\prwrot{-240}&\prwrot{-280}&\prwrot{-320}\\
-240\degrees & -280\degrees & -320\degrees \\ \hline
\end{tabular}
\caption{Rotation of paragraphs between 0 and -320 degrees} \label{fig:angles1}
\end{figure}


\begin{example}Regular caption and float.

Figure~\ref{fig:angles1} is a regular figure and caption. It is produced by
the following code:
\begin{verbatim}
\def\prwrot#1{%
\settowidth{\fool}{ISOROT}
\savebox{\foo}{\parbox{\fool}{ISOROT ISOROT ISOROT ISOROT}}%
\framebox{---\begin{turn}{#1}\framebox{\usebox{\foo}}\end{turn}---}}%
\def\degrees{{\small$^{o}$}}
\end{verbatim}

\begin{verbatim}
\begin{figure}
\centering
\begin{tabular}{|c|c|c|} \hline
\prwrot{0} &\prwrot{-40}&\prwrot{-80}\\
0\degrees & -40\degrees & -80\degrees \\ \hline
\prwrot{-120}&\prwrot{-160}&\prwrot{-200}\\
-120\degrees & -160\degrees & -200\degrees \\ \hline
\prwrot{-240}&\prwrot{-280}&\prwrot{-320}\\
-240\degrees & -280\degrees & -320\degrees \\ \hline
\end{tabular}
\caption{Rotation of paragraphs between 0 and -320 degrees} \label{fig:angles1}
\end{figure}
\end{verbatim}
\end{example} % end example


\begin{figure}
\centering
\begin{sideways}
\begin{tabular}{|c|c|c|} \hline
\prwrot{0} &\prwrot{40}&\prwrot{80}\\
0\degrees & 40\degrees & 80\degrees \\ \hline
\prwrot{120}&\prwrot{160}&\prwrot{200}\\
120\degrees & 160\degrees & 200\degrees \\ \hline
\prwrot{240}&\prwrot{280}&\prwrot{320}\\
240\degrees & 280\degrees & 320\degrees \\ \hline
\end{tabular}
\end{sideways}
\caption[Rotation of paragraphs between 0 and 320 degrees]%
        {Rotation of paragraphs between 0 and 320 degrees (with figure
         body turned sideways)}\label{fig:angles2}
\end{figure}

\begin{example}Regular caption and rotated float body.

Figure~\ref{fig:angles2} is a regular figure and caption where the figure
contents have been rotated. It was produced by the following code.
\begin{verbatim}
\begin{figure}
\centering
\begin{sideways}
\begin{tabular}{|c|c|c|} \hline
\prwrot{0} &\prwrot{40}&\prwrot{80}\\
0\degrees & 40\degrees & 80\degrees \\ \hline
\prwrot{120}&\prwrot{160}&\prwrot{200}\\
120\degrees & 160\degrees & 200\degrees \\ \hline
\prwrot{240}&\prwrot{280}&\prwrot{320}\\
240\degrees & 280\degrees & 320\degrees \\ \hline
\end{tabular}
\end{sideways}
\caption[Rotation of paragraphs between 0 and 320 degrees]%
        {Rotation of paragraphs between 0 and 320 degrees (with figure
         body turned sideways)}\label{fig:angles2}
\end{figure}
\end{verbatim}
\end{example} % end example

\begin{landscape}
\sclause{Landscaping}

    \latex{} normally prints in portrait mode. 
The \verb|landscape|\ixenv{landscape} environment
prints all the enclosed stuff in landscape mode, except for headers
and footers which are not rotated.

\begin{example} Landscaping

The source for this part of the document is:
\begin{verbatim}
\begin{landscape}
\sclause{Landscaping}

    \latex{} normally prints in portrait mode. The ...
...
... long, wide tables.
\end{landscape}
\end{verbatim}
\end{example}

    The environment starts by clearing the current page and then switches
to portrait mode. At the end of the environment the current page is cleared
and the next page is back to normal portrait mode. 

    All the other rotation commands and environments produce boxes and
\latex{} will not break a box across a page. The \verb|landscape| environemt
does not produce a box and so many pages can be printed in landscape mode
with \latex{} taking care of the page breaking for you. 

    Landscape mode is not particularly useful for normal text as the
lines are far too long for comfortable reading. Where it can be useful
is where you have a table that is too wide to fit on a portrait page, so
needs to be rotated, yet is also too long to fit on the page when it is
rotated. The \file{supertabular}\ixpack{supertabular},
the \file{longtable}\ixpack{longtable},
and the \file{xtab}\ixpack{xtab}
packages provide facilities for automatically breaking long tables across 
pages. Any of these can be used in conjunction with landscaping to both 
rotate and automatically page break long, wide tables.
\end{landscape}

\clause{The \file{xtab} package facility}

    The \file{xtab} package is fully documented in \bref{bib:xtab}.
This clause provides an overview of the package.

    The \file{xtab}\ixpack{xtab} package facility is an extension 
of the \file{supertabular}\ixpack{supertabular} package originally
developed by Johannes Braams and Theo Jurriens.
The extension provides for the specification of a header to
go on the last page of a long table.

    The principal commands available are given in \tref{tab:xtab}.
\ixenv{xtabular}
\ixenv{mpxtabular}
\ixcom{topcaption}
\ixcom{bottomcaption}
\ixcom{tablecaption}
\ixcom{tablefirsthead}
\ixcom{tablehead}
\ixcom{tablelasthead}
\ixcom{notablelasthead}
\ixcom{tabletail}
\ixcom{tablelasttail}

\topcaption{The principal xtab package commands} \label{tab:xtab}
\tablefirsthead{\hline \multicolumn{1}{|c|}{\textbf{Command}} &
                       \multicolumn{1}{c|}{\textbf{Effect}} \\ \hline }
\tablehead{\multicolumn{2}{c}%
           {{\captionsize\bfseries \tablename\ \thetable{} -- continued from previous page}} \\
           \hline \multicolumn{1}{|c|}{\textbf{Command}} &
                  \multicolumn{1}{c|}{\textbf{Effect}} \\ \hline }
\tablelasthead{\multicolumn{2}{c}%
           {{\captionsize\bfseries \tablename\ \thetable{} -- concluded from previous page}} \\ 
           \hline \multicolumn{1}{|c|}{\textbf{Command}} &
                  \multicolumn{1}{c|}{\textbf{Effect}} \\ \hline }
\tabletail{\hline \multicolumn{2}{|r|}{{Continued on next page}} \\ \hline}
\tablelasttail{\\ \hline \hline}

\begin{center}
\begin{xtabular}{|l|p{0.5\textwidth}|}
\verb|\begin{xtabular}{...}| & This is equivalent to the normal
                         \verb|\begin{tabular}{...}| environment.
                         You supply the specification of the columns
                         just as for the normal tabular environment.
                         All commands that can be used within a tabular
                         environment can also be used within
                         the xtabular environment.
 \\
 &
    Unlike the tabular environment which prevents page breaking
within the tabular, the xtabular allows page breaking, so that
tabulars can extend automatically across several pages. Xtabular
starts off with a tabular environment and checks the amount of
space left on the page as it adds each row to the tabular.
If the space left on the page is too short for another row, then
it ends the current tabular, performs a page break and starts
another tabular on the following page. This process is repeated
until all the rows have been output. 
 \\
 &
    There are special commands for captioning a xtabular as a
table, and also elements can be automatically inserted after each
(internal) \verb|\begin{tabular}| and immediately before each
\verb|\end{tabular}|.
 \\
 &
    Do not put a xtabular in a table environment, as the table
environment keeps its contents on a single page (presumably you
are using xtabular because its contents are longer than
one page).
 \\
\verb|\end{xtabular}| & End the xtabular environment. \\ \hline
\verb|\begin{mpxtabular}| & Like the xtabular environment
except that each `page' is put into a \verb|minipage| first. Thus
it is possible to have footnotes inside an \verb|mpxtabular|.
The footnote text is printed at the end of each page.
 \\
\verb|\end{mpxtabular}| & End the mpxtabular environment. \\ \hline
                          & \textbf{Note:} If any of the following commands
                            are used, then they should be placed
                            before the particular xtabular
                            environment that they apply to. \\
\verb|\topcaption{...}| & A command to provide a caption for the
                          table. The caption is placed at the top
                          of the table. \\
\verb|\bottomcaption{...}| & A command to provide a caption for the
                          table. The caption is placed at the bottom
                          of the table. \\
\verb|\tablecaption{...}| & A command to provide a caption for the
                          table. The caption is placed at the default
                          position, which is at the top
                          of the table.
 \\
 & 
                           \textbf{Notes:} You cannot use the \verb|caption|
                            command but you can put a label after
                            any of these captioning commands. If you
                            want captioning, the command must be specified
                            before the start of the xtabular environment.
  \\
  &
                            The \verb|\...caption{}| command(s) remain
                            in effect until changed by another
                            \verb|\...caption| command.
   \\
\verb|\tablefirsthead{...}| & Defines the contents of the first occurence
                         of the tabular head. The tabular head is some
                         special treatment of the first row in the table.
                         This command is optional.
                         If used, the header must be closed by the
                         end of line command for tabulars (e.g., \verb|\\|). \\
\verb|\tablehead{...}| & Defines the contents of the table head on
                         subsequent pages.
 \\
 &
                         For example, you might want to note that
                         this is a continuation of the table on
                         the previous page, as well as repeating
                         any column headings that were given
                         at the start of the xtabular by
                         \verb|\tablefirsthead|.
 \\
\verb|\tablelasthead{...}| & Defines the contents of the table head
                            on the last page of the table.
 \\
 &
    This works by writing to the \file{.aux} file the number of
pages that the xtabular extends over. When the xtabular
comes to the last tabular (which it calculates from the 
information it reads from the \file{.aux} file) it replaces the
contents of \verb|\tablehead| by the contents of \verb|\tablelasthead|.
 \\
 &
It makes no attempt to measure the amount of space used by the last
table head, so if this is more than for \verb|tablehead| then the
tabular might be a litle too long.
  \\
  &
    In any case, when using this command, the document has to
be LaTeXed at least twice, just as it has to be to resolve
references and so on.
 \\
\verb|\notablelasthead|  & Switches off the last \verb|\tablelasthead|.
                           A \verb|tablelasthead| stays in effect until
                           overwritten by a new \verb|\tablelasthead|
                           or cancelled by this command.
 \\
\verb|\tabletail{...}| & The contents of this command are inserted before
                         the \verb|\end{tabular}| on each page except
                         for the last page of the table.
 \\
 &
                         For example, you might want to note that the
                         table is continued on the next page. 
 \\
\verb|\tablelasttail{...}| & The contents of this command are inserted before
                         the final \verb|\end{tabular}| of the table.
 \\
 &
                         For example, you might want to note that
                         this is where the table ends.
\end{xtabular}
\end{center}

    As well as the \verb|xtabular|\ixenv{xtabular} and
\verb|mpxtabular|\ixenv{mpxtabular} environments there are
the corresponding starred versions 
(i.e., \verb|xtabular*|\ixenvs{xtabular} and 
        \verb|mpxtabular*|\ixenvs{mpxtabular}) for use in
two column\ixopt{twocolumn} mode where the table is meant to span 
both columns.

\begin{example} \label{ex:xtab} Table~\ref{tab:xtab} was produced by 
the following code:
\begin{verbatim}
\topcaption{The principal xtab package commands} \label{tab:xtab}
\tablefirsthead{\hline \multicolumn{1}{|c|}{\textbf{Command}} &
                       \multicolumn{1}{c|}{\textbf{Effect}} \\ \hline }
\tablehead{\multicolumn{2}{c}%
           {{\captionsize\bfseries \tablename\ \thetable{} -- 
             continued from previous page}} \\
  \hline   \multicolumn{1}{|c|}{\textbf{Command}} &
           \multicolumn{1}{c|}{\textbf{Effect}} \\ \hline }
\tablelasthead{\multicolumn{2}{c}%
           {{\captionsize\bfseries \tablename\ \thetable{} -- 
             concluded from previous page}} \\
  \hline   \multicolumn{1}{|c|}{\textbf{Command}} &
           \multicolumn{1}{c|}{\textbf{Effect}} \\ \hline }
\tabletail{\hline \multicolumn{2}{|r|}{{Continued on next page}} \\ \hline}
\tablelasttail{\\ \hline \hline}

\begin{center}
\begin{xtabular}{|l|p{0.5\textwidth}|}
\verb|\begin{xtabular}{...}| & This is equivalent to the normal
                         \verb|\begin{tabular}{...}| environment.
                         You supply the specification of the columns
                         just as for the normal tabular environment.
                         All commands that can be used within a tabular
                         environment can also be used within
                         the xtabular environment.
 \\
 &
    Unlike the tabular environment which prevents page breaking
within the tabular, the xtabular allows page breaking, so that
tabulars can extend automatically across several pages. Xtabular
... ... ...
\verb|\tablelasttail{...}| & The contents of this command are inserted before
                         the final \verb|\end{tabular}| of the table.
 \\
 &
    For example, you might want to note that  this is where 
the table ends.
\end{xtabular}
\end{center}
\end{verbatim}
\end{example}       

    The table is only broken between rows --- a row will not be split
across pages. This can lead to some bad page breaks, especially if 
there are rows with a large vertical height (like some in \tref{tab:xtab}).
It is best to keep rows not too tall.

The command
\verb|\shrinkheight{|\meta{length}\verb|}|\ixcom{shrinkheight} may be
used after the first \verb|\\| in the table to modify the allowed
height of the table on each page. A positive \meta{length} decreases
the allowed space per page and a negative \meta{length} increases
the allowed space.

\begin{example}
This example illustrates changing the natural height of the pages
in a \verb|xtabular| and its relatives.

\verb|\shrinkheight{2\baselineskip}| decreases the space per page by 
two lines.

\verb|\shrinkheight{-\baselineskip}| increases the space per page by
one line.
\end{example}

    You have to specify the font used for the 
\verb|\tablehead|\ixcom{tablehead} and
\verb|tablelasthead|\ixcom{tablelasthead} yourself. 
Within ISO documents, captions shall
be in bold font. The \file{iso}\ixclass{iso} class also provides 
a command for
setting the size of the font used in captions, namely 
\verb|\captionsize|\ixcom{captionsize}. 
Note that this is used in~\eref{ex:xtab}.
    The value of \verb|\captionsize|\ixcom{captionsize} is set by the 
class.

\sclause{Options}

    The \file{xtab} package has three options which control the amount of
information that is written to the \file{.log}\index{file!.log} file. The
options are:
\begin{enumerate}
\item The option \verb|errorshow|\ixopt{errorshow} (the default) does not
      write any extra information;
\item The option \verb|pageshow|\ixopt{pageshow} writes information
      about when and why \verb|xtabular| decides to produce a new page;
\item The option \verb|debugshow|\ixopt{debugshow}, which also includes
      \verb|pageshow|, additionally writes information about each line
      that is added to the table.
\end{enumerate}

    Under normal circumstances \file{xtab} is used without invoking any 
option. The \verb|pageshow| option may be useful when attempting to cure
a bad page break. The \verb|debugshow| option, as its name implies, is
principally of use to the \file{xtab} developer.

    Independently of the options, the command 
\verb|\sstraceon|\ixcom{sstraceon} may be used at any
point in the document to turn on printing of \verb|debugshow| data. This
can be turned off later by the \verb|\sstraceoff|\ixcom{sstraceoff}
command, which will stop all \verb|...show| printing.

\sclause{Caveats}

    The authors of the original \file{supertabular} package note that
it has the following weaknesses:
\begin{itemize}
\item Sometimes the body of the first page of a table gets moved to the 
following, leaving the caption behind;
\item Sometimes the last page of a table consists of just the head and foot
with no contents.
\end{itemize}
The weaknesses are caused by trying to guess where \tex{} will put a page 
break. The package has to guesstimate how long the next entry will be in 
the table and, if it is too long for the available space, it puts in its
own page break. If its guess is off too much in one direction, \tex{} will 
break the page unexpectadly; if its off in the other direction
\file{supertabular} will put in an unnecessary page break.

    The \file{xtab} package has reduced, but perhaps not entirely
eliminated, these weaknesses. Some hand tuning may still be required.


%%%%%%%%%
%%%%\end{document}
%%%%%%%%%



\clause{The \file{askinc} package facility} 

    The \file{askinc}\ixpack{askinc} package facility contains \latex{} 
code to enable the interactive
input of files. This functionality is a cross between the \latex{}
\verb|\include|\ixcom{include} and 
\verb|\includeonly|\ixcom{includeonly} commands, and the
\verb|\input|\ixcom{input} command.

In the body of the root source document, use the command 
\verb+\infile{+\meta{file}\verb+}+\ixcom{infile}
for each \meta{file}
that comprises the document. That is, the command \verb|\infile| is similar
to the \verb|\input|\ixcom{input} and
\verb|\include|\ixcom{include} commands.

    When \latex ing the root document, you will be asked to provide a 
comma-separated list of the \verb|\infile|s to be processed (similar to the
argument to the \verb|\includeonly|\ixcom{includeonly}
command). If you want all the files to be processed, just hit the
\verb|<RETURN>| key (or its equivalent).

    Like \verb|\include|d files, a file that is \verb|\infile|d into a 
document shall not itself contain another \verb|\infile|d file.

\begin{example}The following root file has three files that are \verb|\infile|d.
\begin{verbatim}
\documentclass[...]{...}
\usepackage{askincv}
% other preamble stuff
\begin{document}
% perhaps some stuff
\infile{file1}
\infile{file2}
\infile{file3}
% perhaps more stuff
\end{document}
\end{verbatim}
\end{example} % end example

\clause{The \file{hyphenat} package facility} \label{sec:uschyp}

    The \file{hyphenat} package is fully described in~\bref{bib:hyphenat}.
This clause provides an overview of the portions of the package that are
most relevant to typesetting ISO standards.

    In \latex{} if you need to use the underscore (\verb|_|)
character in normal text, for example when documenting identifiers
in a programming language code, you have to use the 
\verb|\_|\index{_ /@\verb?\_?} command,
as in \verb|a\_multiword\_identifier|. \latex{} normally treats
\verb|_|\index{_@\verb?_? (underscore)} as a math mode subscript command. 
Further, if you want
the possibility of hyphenation\index{hyphenation} 
at the position of an underscore
you have to use the command pairing 
\verb|\_\-|;\index{_ /@\verb?\_?}\ixcom{-} this has the side
effect of disabling normal hyphenation in any succeeding `word' in
the identifier.

    The \file{hyphenat}\ixpack{hyphenat} package facility redefines the 
\verb|\_|\index{_ /@\verb?\_?} command
so that hyphenationen is automatically enabled at the position
of the underscore and in all succeeding words in the identifier.


\begin{note}
    Using the command pair \verb|\_\-| in conjunction with this facility
disables automatic hyphenation of succeeding words, so don't do it.
\end{note}



\begin{example} \label{eg:uschyp}
This set of \latex{} source
\begin{verbatim}
Amazingly An\_excessively\_long\_multiword\_identifier%
\_demonstrating\_hyphenation

\begin{minipage}{3cm}
\begin{itemize}
\item An\_excessively\_long\_multiword\_identifier%
\_demonstrating\_hyphenation
\item Underscore in math mode: $A_n$
\item \verb|\_| command in math mode: $A\_n$
\end{itemize}
\end{minipage}
\end{verbatim}
prints as:

Amazingly An\_excessively\_long\_multiword\_identifier%
\_demonstrating\_hyphenation

\begin{minipage}{3cm}
\begin{itemize}
\item An\_excessively\_long\_multiword\_identifier%
\_demonstrating\_hyphenation
\item Underscore in math mode: $A_n$
\item \verb|\_| command in math mode: $A\_n$
\end{itemize}
\end{minipage}

\end{example}

\begin{example}
Contrast this example with \eref{eg:uschyp}.
This set of \latex{} source
\begin{verbatim}
Amazingly An\_\-excessively\_\-long\_\-multiword\_\-identifier%
\_\-demonstrating\_\-hyphenation\_\-disabling

\begin{minipage}{3cm}
\begin{itemize}
\item An\_\-excessively\_\-long\_\-multiword\_\-identifier%
\_\-demonstrating\_\-hyphenation\_\-disabling
\item Underscore in math mode: $A_n$
\item \verb|\_| command in math mode: $A\_n$
\end{itemize}
\end{minipage}
\end{verbatim}
prints as:

Amazingly An\_\-excessively\_\-long\_\-multiword\_\-identifier%
\_\-demonstrating\_\-hyphenation\_\-disabling

\begin{minipage}{3cm}
\begin{itemize}
\item An\_\-excessively\_\-long\_\-multiword\_\-identifier%
\_\-demonstrating\_\-hyphenation\_\-disabling
\item Underscore in math mode: $A_n$
\item \verb|\_| command in math mode: $A\_n$
\end{itemize}
\end{minipage}

\end{example}


    The \file{hyphenat} package also provides some other commands for
enabling hyphenation of words that include 
analphabetic\index{character!analphabetic}\footnote{An {\it analphabetic} 
character is any character
that is not alphabetic. Typically it refers to punctuation characters.}
characters. In this context, the phrase 
`breakable character'\index{character!breakable} is used
to describe an analphabetic character that enables hyphenation immediately
after it and does not prevent further hyphenation in the `word'
in which it occurs. The \verb|\_|\index{_ /@\verb?\_?} command produces 
a breakable underscore. Table~\ref{tab:breakable} lists all the commands
that generate breakable characters.

\begin{table}
\centering
\caption{Commands producing breakable characters} \label{tab:breakable}
\begin{tabular}{|l|c|} \hline
\textbf{Command}    & \textbf{Character} \\ \hline
\verb|\_|        & \_ \\
%\verb|\?|        & \? \\
\verb|\bshyp|    & \bshyp \\
\verb|\colonhyp| & \colonhyp \\
\verb|\dothyp|   & \dothyp \\
\verb|\fshyp|    & \fshyp \\ \hline
\end{tabular}
\end{table}

    The \verb|\bshyp|\ixcom{bshyp} command prodes a breakable backslash
(\verb|\|), \verb|\fshyp|\ixcom{fshp} produces a breakable forward
slash (\verb|/|), the \verb|\dothyp|\ixcom{dothyp} command produces a
breakable full stop (\verb|.|), also known in some countries as a period,
and the \verb|\colonhyp|\ixcom{colonhyp} command produces a breakable 
colon (:).

\begin{example}
This is similar to \eref{eg:uschyp} except that it demonstrates other
breakable characters.

This set of \latex{} source
\begin{verbatim}
Analphabetically an\bshyp{}excessively\fshyp{}long\dothyp{}multiword\bshyp{}identifier%
\fshyp{}demonstrating\dothyp{}hyphenation

\begin{minipage}{3cm}
Analphabetically an\bshyp{}excessively\fshyp{}long\dothyp{}multiword\bshyp{}identifier%
\fshyp{}demonstrating\dothyp{}hyphenation
\end{minipage}
\end{verbatim}
prints as:

Analphabetically an\bshyp{}excessively\fshyp{}long\dothyp{}multiword\bshyp{}identifier%
\fshyp{}demonstrating\dothyp{}hyphenation

\begin{minipage}{3cm}
Analphabetically an\bshyp{}excessively\fshyp{}long\dothyp{}multiword\bshyp{}identifier%
\fshyp{}demonstrating\dothyp{}hyphenation
\end{minipage}

\end{example}

\begin{note}
\latex{} will not hyphenate the first word in a paragraph.
\end{note}

    Just as with the \verb|\_|\index{_ /@\verb?\_?} command, 
the discretionary hyphen
command (i.e., \verb|\-|\ixcom{-}) should not be used in conjunction with 
any of the breakable character commands as it will then inhibit any
further potential hyphenation points. In general, any analphabetic
character in a word will inhibit further hyphenation.

\normannex{Additional commands} \label{anx:extraiso}

\sclause{Language configuration commands}

    There is an additional set of commands in the
\file{iso}\ixclass{iso} class
facility that
are specified here. This set has been provided to enable the style to be
easily configured for a non-English language.\index{non-English languages}

    The modified commands may be put in the document 
preamble\index{preamble} or, preferably,
placed in a separate \file{.sty}\index{file!.sty} file and 
called as a package. This latter option improves reuseability.

\ssclause{Words and phrases}

    To produce a non-English version of the \file{iso}\ixclass{iso}
class the relevant commands
from the following list may require re-definition together with
the heading commands in \tref{tab:dsa}.

\begin{note}In the listing, the default values are printed \emph{in this
font} to distinguish them from the explanatory text.
\end{note} % end note}

\begin{itemize}
\item \verb|\annexname|\ixcom{annexname}: Header for
      an annex.
      Default value is: \emph{\annexname}\index{Annex}
\item \verb|\contentsname|\ixcom{contentsname}: Header
      for table of contents listing. 
      Default value is: \emph{\contentsname}\index{Contents}
\item \verb|\copyrightname|\ixcom{copyrightname}:
      The copyright owner.
      Default value is: \emph{\copyrightname}\index{ISO}
\item \verb|\examplename|\ixcom{examplename}:
      Identification of an example.
      Default value is: \emph{\examplename}\index{EXAMPLE}
%\item \verb|\examplesname|\ixcom{examplesname}:
%      Header for a list of examples.
%      Default value is: \emph{\examplesname}\index{EXAMPLES}
\item \verb|\figurename|\ixcom{figurename}: Start of
      the caption for a figure. 
      Default value is: \emph{\figurename}\index{Figure}
\item \verb|\forewordname|\ixcom{forewordname}:
      Title of the Foreword.
      Default value is: \emph{\forewordname}\index{Foreword}
\item \verb|\indexname|\ixcom{indexname}: Header for
      the index.
      Default value is: \emph{\indexname}\index{Index}
\item \verb|\informativename|\ixcom{informativename}:
      Identification of an informative annex.
      Default value is: \emph{\informativename}\index{informative}
\item \verb|\inscopename|\ixcom{inscopename}:
      Introduction to in-scope listing.
      Default value is: \emph{\inscopename}
      \index{The following are within the scope of this}
\item \verb|\ISname|\ixcom{ISname}:
     `INTERNATIONAL STANDARD' as used in the header for an IS title page.
      Default value is: 
      \emph{\ISname}\index{INTERNATIONAL STANDARD}
\item \verb|\introductionname|\ixcom{introductionname}:
      Title of the Introduction.
      Default value is: 
      \emph{\introductionname}\index{Introduction}
\item \verb|\listannexname|\ixcom{listannexname}:
      Header for list of annexes listing.
      Default value is: \emph{\listannexname}\index{Annexes}
\item \verb|\listfigurename|\ixcom{listfigurename}:
      Header for list of figures listing.
      Default value is: \emph{\listfigurename}\index{Figures}
\item \verb|\listtablename|\ixcom{listtablename}:
      Header for list of tables listing.
      Default value is: \emph{\listtablename}\index{Tables}
\item \verb|\normativename|\ixcom{normativename}:
      Identification of a normative annex.
      Default value is: \emph{\normativename}\index{normative}
\item \verb|\normrefsname|\ixcom{normrefsname}:
      Title of normative references clause.
      Default value is: 
      \emph{\normrefsname}\index{Normative references}
\item \verb|\notename|\ixcom{notename}:
      Identification of a note.
      Default value is: \emph{\notename}\index{NOTE}
%\item \verb|\notesname|\ixcom{notesname}:
%      Header for a list of notes.
%      Default value is: \emph{\notesname}\index{NOTES}
\item \verb|\outofscopename|\ixcom{outofscopename}:
      Introduction to out-of-scope listing.
      Default value is: \emph{\outofscopename}
      \index{The following are outside the scope of this}
\item \verb|\pagename|\ixcom{pagename}:
      The word for the page header in the table of contents.
      Default value is: \emph{\pagename}\index{Page}
\item \verb|\scopename|\ixcom{scopename}:
      Title of the Scope.
      Default value is: \emph{\scopename}\index{Scope}
%\item \verb|\sectionname|\ixcom{sectionname}:
%      The word for a 'section'.
%      Default value is: \emph{\sectionname}\index{Section}
\item \verb|\tablename|\ixcom{tablename}:
      Start of the caption for a table.
      Default value is: \emph{\tablename}\index{Table}
\item \verb|\tbpname|\ixcom{tbpname}:
      Footnote text for `to be published.'.
      Default value is: 
      \emph{\tbpname}\index{To be published.}
\end{itemize}

    The following commands provide the names for referenced document elements.
\begin{itemize}
\item \verb|\annexrefname|\ixcom{annexrefname}:
      Reference to an annex.
      Default value is: \emph{\annexrefname}\index{annex}
\item \verb|\clauserefname|\ixcom{clauserefname}:
      Reference to a clause.
      Default value is: \emph{\clauserefname}\index{clause}
\item \verb|\examplerefname|\ixcom{examplerefname}:
      Reference to an example.
      Default value is: \emph{\examplerefname}\index{example}
\item \verb|\figurerefname|\ixcom{figurerefname}:
      Reference to a figure.
      Default value is: \emph{\figurerefname}\index{figure}
\item \verb|\noterefname|\ixcom{noterefname}:
      Reference to a note.
      Default value is: \emph{\noterefname}
\item \verb|\tablerefname|\ixcom{tablerefname}:
      Reference to a table.
      Default value is: \emph{\tablerefname}\index{table}
\item \verb|\pagerefname|\ixcom{pagerefname}:
      Reference to a page.
      Default value is: \emph{\pagerefname}\index{page}
\end{itemize}



\begin{note} The above commands, 
 may be changed via the \latex{} 
\verb|\renewcommand|\ixcom{renewcommand}.
\end{note} % end note
\begin{note}The \latex{} command \verb|\today|\ixcom{today}
      will probably also require modification. This is not something
      for the casual user to attempt.
\end{note} % end note


\begin{example}The following is a partial list of the commands to convert to a 
         French language\index{French language} style. 
\begin{verbatim}
\renewcommand{\annexname}{Annexe}
\renewcommand{\contentsname}{Sommaire}
\renewcommand{\examplename}{EXEMPLE}
%\renewcommand{\examplesname}{EXEMPLES}
\renewcommand{\forewordname}{Avant-propos}
\renewcommand{\ISname}{NORME INTERNATIONALE}
\renewcommand{\listtablename}{Tableaux}
\renewcommand{\scopename}{Domaine d'application}
\renewcommand{\tablename}{Tableau}
\end{verbatim}
\end{example} % end example

\ssclause{Boilerplate}

    Some commands print boilerplate text; again, the default text is in English.
For \file{iso}\ixclass{iso} there are several such commands. The first is 
\verb|\copyrightnotice|\ixcom{copyrightnotice}
 which contains the text of the copyright notice
for an International Standard. This can be changed via the 
\verb|\renewcommand|\ixcom{renewcommand} command.

    The second is \verb|\normrefbp|\ixcom{normrefbp}
which prints the boilerplate for the introduction to the Normative
references clause. Like the \verb|\copyrightnotice| command, this can
be redefined using the \verb|\renewcommand|.

    Another is the \verb|\fwdbp|\ixcom{fwdbp} command which
\verb|\input|s the boilerplate text from a file called \file{isofwdbp.tex}.
\index{isofwdbp.tex@\file{isofwdbp.tex}} For non-English text either 
modify the contents of this
file or create a new file and modify the \verb|\fwdbp| command to call in
the new file.



\begin{example}This is how \verb|\normrefbp| could be written for the 
French language and in accordance with the second edition of the 
ISO Directives.
\label{eg:frenchnfbp}
\begin{verbatim}
\renewcommand{\normrefbp}[1]{%
    Les normes suivantes contiennent des dispositions qui, par
suite de la r\'{e}f\'{e}nce qui en est faite, constituent des
dispositions valables pour la pr\'{e}sente #1. 
Au moment de la publication, les \'{e}ditions indiqu\'{e}es 
\'{e}taient en viguer. Toute norms est sujette \`{a} r\'{e}vision et
les parties prenantes des accords fond\'{e}s sur la pr\'{e}sente #1 
sont invit\'{e}es \`{a} rechercher la possibilit\'{e} d'appliquer
les \'{e}ditions les plus r\'{e}centes des normes indiqu\'{e}es
ci-apr\`{e}s. Les membres de la CEI et de l'ISO poss\`{e}dent
le registre des Normes Internationales en vigueur \`{a} un
moment donn\'{e}.
}% end renewcommand
\end{verbatim}
\end{example} % end example
\begin{example}Given that \verb|\normrefbp| has been redefined as in \eref{eg:frenchnfbp},
then the command \\
 \verb|\normrefbp{norme internationale}| will print:

\newcommand{\tempbp}[1]{%
    Les normes suivantes contiennent des dispositions qui, par
suite de la r\'{e}f\'{e}nce qui en est faite, constituent des
dispositions valables pour la pr\'{e}sente #1. 
Au moment de la publication, les \'{e}ditions indiqu\'{e}es 
\'{e}taient en viguer. Toute norms est sujette \`{a} r\'{e}vision et
les parties prenantes des accords fond\'{e}s sur la pr\'{e}sente #1 
sont invit\'{e}es \`{a} rechercher la possibilit\'{e} d'appliquer
les \'{e}ditions les plus r\'{e}centes des normes indiqu\'{e}es
ci-apr\`{e}s. Les membres de la CEI et de l'ISO poss\`{e}dent
le registre des Normes Internationales en vigueur \`{a} un
moment donn\'{e}.
}% end newcommand

\tempbp{norme internationale}
\end{example} % end example



    The \verb|\tspasfwdbp|\ixcom{tspasfwdbp} also maintains
boilerplate text in the file 
\file{tspasfwdbp.tex}\index{tspasfwdbp.tex@\file{tspasfwdbp.tex}}.

\sclause{Caption text size}

    The size of the font used for typestting the captions of figures and
tables is defined within the \file{iso}\ixclass{iso} class.
% The
%\verb|uglycaption|\ixopt{uglycaption} option resets the size to larger
%than normal for the captioning text.

    The size of the captioning font is controlled by the value of the
\verb|\captionsize|\ixcom{captionsize} command. The default definition
of \verb|\captionsize| is similar to:
\verb|\captionsize{\normalsize}| \ixcom{normalsize}\\
%The \verb|uglycaption| option resets this to (approximately): \\
%\verb|\captionsize{\large}|. \ixcom{large}

    You can reset the \verb|\captionsize| at any point in your
document to change the size of captions from then onwards.

    All the normal \latex{} font size commands are available.
\begin{example}
The following shows the effects of the font size commands. \\
\verb|{\tiny tiny text}|\ixcom{tiny} 
prints: {\tiny tiny text} \\
\verb|{\scriptsize scriptsize text}|\ixcom{scriptsize} 
prints: {\scriptsize scriptsize text} \\
\verb|{\footnotesize footnotesize text}|\ixcom{footnotesize} 
prints: {\footnotesize footnotesize text} \\
\verb|{\small small text}|\ixcom{small}
prints: {\small small text} \\
\verb|{\normalsize normalsize text}|\ixcom{normalsize} 
prints: {\normalsize normalsize text} \\
\verb|{\large large text}|\ixcom{large} 
prints: {\large large text} \\
\verb|{\Large Large text}|\ixcom{Large} 
prints {\Large Large text} \\
\verb|{\LARGE LARGE text}|\ixcom{LARGE} 
prints: {\LARGE LARGE text} \\
\verb|{\huge huge text}|\ixcom{huge} 
prints: {\huge huge text} \\
\verb|{\Huge Huge text}|\ixcom{Huge} 
prints: {\Huge Huge text}
\end{example}


\normannex{Ordering of LaTeX commands} \label{anx:lord}

    The \latex{} commands for the logical structuring of an ISO standard 
document are:
\begin{verbatim}
\documentclass[<options>]{isov2}          % for LaTeX 2e
\usepackage{<name>}                       % additional packages (LaTeX 2e)
\standard{<standard identification>}
\yearofedition{<year>}
\languageofedition{<parenthesized code letter>}
  % other preamble commands
\begin{document}
\begin{foreword}                     % start Foreword
  \fwdbp                             % boilerplate
  % other text and perhaps \fwdnopatents
\end{foreword}
\begin{introduction}                 % start Introduction
  % text and perhaps \intropatents
\end{introduction}
\title{<intro>}{<main>}{<compl>}     % the title
\scopeclause                         % The Scope clause
   \begin{inscope}{<document>}       % in scope boilerplate
   % \item list
   \end{inscope}
   % text 
   \begin{outofscope}{<document>}    % out of scope boilerplate
   % \item list
   \end{outofscope}
   % text
\normrefsclause                      % The Normative references clause
  \normrefbp{<document identifier>}  % boilerplate
  \begin{nreferences}
    % \isref{<p1>}{<p2>} and/or \disref{<p1>}{<p2>} commands
  \end{nreferences}
% definitions, symbols, abbreviation clause as appropriate
\clause{<Clause title>}
% THE BODY OF THE DOCUMENT
% ...
% \normannex{<Normative annex title>}
% ...
% \infannex{<Infomative annex title>}
% ...
\bibannex                            % optional bibliography
  % bibliography listing
% the index
\end{document}
\end{verbatim}


\infannex{Creating an index} \label{anx:indexing}

    \latex, although providing some assistance in preparing the data for an
index\index{index}, only does part of the job. Providing the command
\verb|\makeindex|\ixcom{makeindex} is put in the document's 
preamble\index{preamble}
the \latex{} command \verb|\index{text}|\ixcom{index} 
writes out \verb|text| to an \file{.idx}\index{file!.idx} file 
in the following format:\ixcom{indexentry}
\begin{verbatim}
\indexentry{text}{pg}
\end{verbatim}
where \verb|pg| is the page number in the document where the \verb|\index|
command occurred.

    The \verb|theindex|\ixenv{theindex} environment 
is used for printing an index. The format of this is:
\begin{verbatim}
\begin{theindex}
\item <text and page numbers>
  \subitem <text and page numbers>
    \subsubitem <text and page numbers>
.
.
\end{theindex}
\end{verbatim}
where \verb|\item|\ixcom{item} is a major topic entry,
\verb|\subitem|\ixcom{subitem} is a sub-topic entry, and
\verb|\subsubitem|\ixcom{subsubitem} is a sub-subtopic
entry. The command \verb|\indexspace|\ixcom{indexspace}
can be used to add space between the entries. Other text and commands can also
occur within the environment.

    \latex{} provides no help in going from an \file{.idx} file to the
\verb|theindex| environment~\bref{lamport}. The data in the file has to be
sorted, duplicate page numbers deleted, etc, and then re-written in the 
desired environment format.

    The typical process for producing a document with an index is:
\begin{enumerate}
\item Prepare the source file, say \file{fred.tex}, with the command
      \verb|\makeindex|\ixcom{makeindex} in the
      preamble\index{preamble}, and \verb|\index|\ixcom{index}
      commands within the body of the text.
\item Run \latex. As well as producing the usual output files, it will also
      produce the file \file{fred.idx}.
\item By some means produce a file, let's call it \file{fredidx.tex}, from the 
      data in \file{fred.idx} that contains the appropriate \verb|theindex|
      formatted data.
\item Run \latex{} again on \file{fred.tex} which now has to contain (either
      via \verb|\input|\ixcom{input} or \verb|\include|\ixcom{include})
      the file \file{fredidx.tex}.
\end{enumerate}

    Chen and Harrison~\bref{chen} discuss the problems of creating an index 
in their
paper \emph{Index preparation and processing} and also describe the
{\sc MakeIndex}\index{makeIndex@{\sc MakeIndex}} program. Goosens, Mittelbach
and Samarin~\bref{goosens} also describe how to use the {\sc MakeIndex} 
program for producing indexes for \latex{} documents. For users of
{\sc MakeIndex} a style file called 
\file{iso.ist}\index{iso.ist@\file{iso.ist}}\index{file!.ist} is provided as
part of this distribution.

\sclause{The index command}\ixcom{index}

    The \verb|\index| command is one of the standard \latex{} commands.
The command format is \verb|\index{|\meta{str}\verb|}|, where \meta{str}
 is any
string of characters, and it writes an entry to the \file{.idx} file in
the form \verb|\indexentry{<str>}{pg}|, where \verb|pg| is the page
number in the document where the command is called.

    Some points to note:
\begin{itemize}
\item The \verb|\index| command is, in \latex{} terminology, 
      \emph{fragile}\index{fragile}. That is, if it appears in a moving
      argument (like the caption to a table or figure) it must be preceded
      by the \latex{} \verb|\protect|\ixcom{protect}
      command.
\item Any of the ten \latex{} special characters\index{special characters}
      (i.e., \verb|#|, \verb|$|, \verb|%|, \verb|&|, \verb|~|, \verb|_|,
             \verb|^|, \verb|\|, \verb|{| and \verb|}|) may appear
     within the argument, with the proviso that it must have no unmatched
     braces (the braces in \verb|\{| and \verb|\}| are counted in the
     matching process).
\item The \verb|\index| command must not appear inside another command's
     argument (e.g., within a 
     \verb|\footnote|\ixcom{footnote} or \verb|\clause|\ixcom{clause}
     command) unless the argument of the \verb|\index| command contains
     only letters, digits, and/or punctuation characters. In particular,
     it should not contain any of the special characters.
  \begin{note}This means that the argument of the \verb|\ix|\ixcom{ix}
        command should not contain any special characters. Remember that the
       \verb|\ix| command prints its argument in the body of the text and also
       calls \verb|\index| to place its argument into the \file{.idx} file.
       \end{note} % end note
\end{itemize}

\begin{note}Under some circumstances, an \verb|\index| command appearing in another
command's argument may contain special characters, provided they are 
\verb|\protect|ed. Determination of when this is satisfactory is a process
of trial and error.
\end{note} % end note

\begin{example}The command \verb|\ix{an\protect\_underscore}| will print the
characters \emph{an\_undescore}
in the text and also write the following to the \file{.idx} file:
\begin{verbatim}
\indexentry{an\_ underscore}{pg}
\end{verbatim}
Notice that there is a space between the underscore character and the word
`underscore' in the \file{.idx} file (but there is no space in the printed
body of the document text). This extraneous space may have to be edited out 
from the final index.
\end{example} % end example


\begin{comment}

\sclause{The program GenIndex} \index{genindex@GenIndex}

    GenIndex is a C program\index{C language} that converts \file{.idx}
data to \verb|theindex|\ixenv{theindex} data. 
Source code for GenIndex is freely available from
the NIST SOLIS\index{SOLIS} system (see \ref{anx:solis}). 
The GenIndex program is not
as sophisticated as {\sc MakeIndex} but does a reasonable 
job.\footnote{These two programs are not completely compatible but do have
a common subset of commands. The common subset excludes the {\tt SeeAlso}
and {\tt See} commands. At some later time GenIndex may be rewritten to
be compatible with {\sc MakeIndex}. It is probably advisable, therefore, not
to use the GenIndex {\tt SeeAlso} and {\tt See} commands.}


\ssclause{Input} \index{genindex@GenIndex!input}

    GenIndex reads lines of data of the form:
\begin{verbatim}
\indexentry{IndexData}{Page}
\end{verbatim}

    \verb|Page| is a character string representing a page number. GenIndex only
recognizes strings that represent integer values greater than or equal to zero,
or (upper and lower case) roman numerals. For any other numbering system, the
page number is treated as zero.

    \verb|IndexData| is a string of characters and command characters that
specify the data that is to be indexed. In the simplest case, this is just the
word or phrase to appear in the index, but much more can be done than this, as
is described below.

    The general form of \verb|IndexData| is given by the following partial
grammar:
\begin{verbatim}
IndexData  = MainData ['!' SubData ['!' SubSubData ] ] .
MainData   = Data .
SubData    = Data .
SubSubData = Data .
Data       = Key [ Entry ] [ SeeAlso ] [ See ] .
Key        = any string not containing the @, !, + or * characters .
Entry      = '@' any string not containing the !, + or * characters 
                 (unless enclosed in braces {} ) .
SeeAlso    = '+' any string not containing the ! or * characters
                 (unless enclosed in braces {} ) .
See        = '*' any string not containing the ! character
                 (unless enclosed in braces {} ) .
\end{verbatim}

    Upto three levels of indexing are enabled --- a main topic entry, an optional
sub-topic, and an optional sub-subtopic. The sub-topics for an entry are
introduced by the \verb|!| character. \index{(33@{\verb?!?}}

     Apart from \verb|Key|, braces within a string must be matched. That is,
they must appear in pairs of an opening and a closing brace.
Spaces are significant within the \verb|Key| string, but not in the others.
That is, \verb*|\indexentry{A}| differs from \verb*|\indexentry{ A}| 
both of which differ from \verb*|\indexentry{A }|.


\begin{example}Here is an index entry for a simple topic:
\begin{verbatim}
\indexentry{Topic}{27}
\end{verbatim}
\end{example} % end example
\begin{example}And here is one where the key and the main entry are different, a 
sub-topic is being indexed, and the page is in roman numerals:
\begin{verbatim}
\indexentry{main@\textbf{Main}!sub topic}{xviii}
\end{verbatim}
\end{example} % end example
\begin{example}This is how the characters \verb|@|, \verb|!|, \verb|+| and
\verb|*| characters are indexed in this document.
\begin{verbatim}
\index{(64@{\verb?@?}}
\index{(33@{\verb?!?}}
\index{(43@{\verb?+?}}
\index{(42@{\verb?*?}}
\end{verbatim}
\end{example} % end example


\sssclause{Key} \index{genindex@GenIndex!input!key}

    The indexed entries are sorted alphabetically on the value of \verb|Key|.
Sorting is case-insensitive. A \verb|Key| value is required.

    More precisely, the key entries are sorted according to the
C language\index{C language} implementation collating sequence, which is usually ASCII. 
Table~\ref{tab:ascii} gives the ASCII collating sequence for the \latex\
character set. \index{ASCII}


\begin{table}
\def\vissp{\hbox{\tt\char`\ }} % visible space
\centering
\caption{The \protect\latex{} ASCII character set} \label{tab:ascii}
\begin{tabular}{|r|c|c|c|c|c|c|c|c|c|c|} \hline
  & \textbf{0} & \textbf{1} & \textbf{2} & \textbf{3} & \textbf{4} &
\textbf{5} & \textbf{6} & \textbf{7} & \textbf{8} & \textbf{9} \\ \hline\hline
\textbf{30}  &          &          & \vissp   & \verb|!| & \verb|"| &
            \verb|#| & \verb|$| & \verb|%| & \verb|&| & \verb|'|    \\
\textbf{40}  & \verb|(| & \verb|)| & \verb|*| & \verb|+| & \verb|,| &
            \verb|-| & \verb|.| & \verb|/| & \verb|0| & \verb|1|     \\
\textbf{50}  & \verb|2| & \verb|3| & \verb|4| & \verb|5| & \verb|6| &
            \verb|7| & \verb|8| & \verb|9| & \verb|:| & \verb|;|     \\
\textbf{60}  & \verb|<| & \verb|=| & \verb|>| & \verb|?| & \verb|@| &
            \verb|A| & \verb|B| & \verb|C| & \verb|D| & \verb|E|     \\
\textbf{70}  & \verb|F| & \verb|G| & \verb|H| & \verb|I| & \verb|J| &
            \verb|K| & \verb|L| & \verb|M| & \verb|N| & \verb|O|     \\
\textbf{80}  & \verb|P| & \verb|Q| & \verb|R| & \verb|S| & \verb|T| &
            \verb|U| & \verb|V| & \verb|W| & \verb|X| & \verb|Y|     \\
\textbf{90}  & \verb|Z| & \verb|[| & \verb|\| & \verb|]| & \verb|^| &
            \verb|_| & \verb|`| & \verb|a| & \verb|b| & \verb|c|     \\
\textbf{100} & \verb|d| & \verb|e| & \verb|f| & \verb|g| & \verb|h| &
            \verb|i| & \verb|j| & \verb|k| & \verb|l| & \verb|m|     \\
\textbf{110} & \verb|n| & \verb|o| & \verb|p| & \verb|q| & \verb|r| &
            \verb|s| & \verb|t| & \verb|u| & \verb|v| & \verb|w|     \\
\textbf{120} & \verb|x| & \verb|y| & \verb|z| & \verb|{| & \verb/|/ &
            \verb|}| & \verb|~| &          &          &              \\ \hline
\end{tabular}
\end{table}

\sssclause{Entry} \index{genindex@GenIndex!input!entry} 
\index{(64@{\verb?@?}}

    \verb|Entry| is optional and is introduced by the \verb|@| character. If present, this will form the indexed string.
If absent, then the value of \verb|Key| is used instead.

\sssclause{See also} \index{genindex@GenIndex!input!see also}
 \index{(43@{\verb?+?}}

    \verb|SeeAlso| is optional and is introduced by the \verb|+| character. 
Within a document, it should only be used once
per entry. If used more than once, then only the last value is taken.

    This is used to produce an entry that refers to another indexed entry in
a \emph{see also \ldots} style.

\sssclause{See} \index{genindex@GenIndex!input!see} 
\index{(42@{\verb?*?}}

    \verb|See| is optional and is introduced by the \verb|*| character. 
Within a document, it should be used only once per
entry. If used more than once, then only the last value is taken.

    This is used to produce an entry that refers to another entry in a
\emph{see \ldots} style. Use of this option prohibits printing page numbers 
for the entry.

\ssclause{Output} \index{genindex@GenIndex!output}

    The program GenIndex sorts the entries into alphabetical order (based on
the \verb|Key| values), and produces a \verb|theindex| for the input data.
Several formatting commands are included in the output file to enable
adjustment of the appearance of the printed index. These are shown in
\tref{tab:indexc}.

\begin{table}
\centering
\caption{GenIndex formatting commands} \label{tab:indexc}
\begin{tabular}{|l|l|} \hline
\textbf{Command} & \textbf{Effect} \\ \hline
\verb|\indexfill|\ixcom{indexfill} & 
    spacing style between main topic and page numbers \\
\verb|\sindexfill|\ixcom{sindexfill} & 
    spacing style between subtopic and page numbers \\
\verb|\ssindexfill|\ixcom{ssindexfill} & 
    spacing between sub-subtopic and page numbers \\
\verb|\indexsee{text}|\ixcom{indexsee} & 
    produces \verb|text| as a \emph{see} entry \\
\verb|\indexseealso{text}|\ixcom{indexseealso} & 
    produces \verb|text| as a \emph{see also} entry \\
\verb|\otherindexspace{c}|\ixcom{otherindexspace} & 
    heading for non-alphabetic entry \\
\verb|\alphaindexspace{c}|\ixcom{alphaindexspace} & 
    heading for an alphabetic entry  \\ \hline
\end{tabular}
\end{table}


\begin{note}The commands in \tref{tab:indexc} are defined in the 
      \file{iso}\ixclass{iso} class file.
\end{note}
\begin{note}If the commands are not defined in your system, 
      then you can define them
      using the \latex{} \verb|\newcommand|\ixcom{newcommand}
      command. On the other hand, if they are defined in your system, 
      you can change them using the 
      \latex{} \verb|\renewcommand|\ixcom{renewcommand} command.
\end{note} % end note


\begin{example}
This input file:
\begin{verbatim}
\indexentry{Freddy+Fred}{27}
\indexentry{Frederick*Fred}{29}
\indexentry{Fred}{42}
\indexentry{Fred}{52}
\indexentry{Fred}{43}
\end{verbatim}
will produce output like:
\begin{verbatim}
\begin{theindex}
\alphaindexspace{F}
\item Fred \indexfill 42--43, 52
\item Frederick \indexsee{Fred}
\item Freddy \indexfill 27 \indexseealso{Fred}
\end{theindex}
\end{verbatim}
\end{example} % end example

\sssclause{Indexfill commands}

    The commands \verb|\indexfill{|\meta{style}\verb|}|\ixcom{indexfill}, 
\verb|\sindexfill{|\meta{style}\verb|}|\ixcom{sindexfill} and 
\verb|\ssindexfill{|\meta{style}\verb|}|\ixcom{ssindexfill}
control the style of spacing between an indexed entry and its corresponding
page numbers.

\begin{example}
If you want to have the page numbers right justified, with lines between the
main topics and their numbers, dots between sub-topics and pages, and 
sub-subtopic numbers right justified, then you could define these commands as:
\begin{verbatim}
\newcommand{\indexfill}{\hrulefill}
\newcommand{\sindexfill}{\dotfill}
\newcommand{\ssindexfill}{\hfill}
\end{verbatim}
\end{example} % end example


\begin{anote}The usual \latex{} style is to have a small gap between the
topic and page number, the whole being set ragged right.
If this is what you want, then define the commands as:
\begin{verbatim}
\newcommand{\indexfill}{}
\newcommand{\sindexfill}{}
\newcommand{\ssindexfill}{}
\end{verbatim}
\end{anote} % end note

\sssclause{Indexsee and indexseealso commands}

    The \verb?\indexsee{?\meta{text}\verb?}?\ixcom{indexsee} and 
\verb?\indexseealso{?\meta{text}\verb?}?\ixcom{indexseealso} commands 
control the appearance
of the \emph{see} and \emph{see also} text. 


\begin{example}
These commands could be defined as:
\begin{verbatim}
\newcommand{\indexsee}[1]{\par \hspace*{2em} \emph{see} #1}
\newcommand{\indexseealso}[1]{\par \hspace*{2em} \emph{see also} #1}
\end{verbatim}
\end{example} % end example

\sssclause{Index space commands}
\ixcom{alphaindexspace}
\ixcom{otherindexspace}

    The commands 
\verb?\alphaindexspace{?\meta{c}\verb?}?\ixcom{alphaindexspace} and 
\verb?\otherindexspace{?\meta{c}\verb?}?\ixcom{otherindexspace}
control the amount of space between blocks of index entries. 
These are an extension of the
\latex{} \verb|\indexspace|\ixcom{indexspace}
command, which just inserts some vertical space into the index listing.
Both these commands take a single parameter, which is typically a single
character.

\begin{anote}
GenIndex looks at the first character (call it \verb|c|) of the \verb|Key| 
and if this changes
as it processes the ordered list of main topics, it puts that character
as the parameter for these commands. It writes \verb|\alphaindexspace{c}| if
the character is alphabetic (and \verb|c| is put into upper-case), 
otherwise it writes \verb|\otherindexspace{c}|.
\end{anote} % end note

\begin{example}
These commands could be defined as:
\begin{verbatim}
\newcommand{\alphaindexspace}[1]{\indexspace
                       {\bfseries #1}}
\newcommand{\otherindexspace}[1]{}
\end{verbatim}
which would result in the printing of a vertical space and a bold font text
for an alphabetic header, or nothing for anything else.
\end{example} % end example

\ssclause{Running GenIndex} \index{genindex@GenIndex!run}

    To run GenIndex, first obtain a copy of the program, and compile it if
necessary. Execute the program on your data.

    GenIndex takes zero, one or two file names as parameters. If no files are
given then input and output is from and to \verb|stdin| and \verb|stdout|
respectively.

    If one file is given, then input is taken from this file, and output
is to \verb|stdout|.

    If two file names are given, then input is taken from the first and output
is to the second.

\begin{example}
A typical usage would be:
\begin{verbatim}
GenIndex fred.idx fredidx.tex
\end{verbatim}
which would read the \file{fred.idx} file and write the resulting index data to
file \file{fredidx.tex}.
\end{example} % end example

\end{comment}

% sgmlannx.tex    latex and SGML

\infannex{LaTeX, the Web, and *ML} \label{anx:sgml} \index{SGML}

    ISO are becoming more interested in electronic sources for their
standards as well as the traditional camera-ready copy. Acronyms like
PDF, HTML, SGML and XML have been bandied about. Fortunately documents
written using \latex{} are well placed to be provided in a variety of 
electronic formats. A comprehensive treatment of \latex{} with respect
to this topic is provided by Goossens and Rahtz~\bref{lwebcom}.

    SGML (Standard Generalized Markup Language) is a document tagging 
language that is described in ISO~8879~\bref{sgml} and whose usage is described 
in~\bref{bryan}, among others. The principal
mover behind SGML is Charles Goldfarb from IBM, who has authored a detailed 
handbook~\bref{goldfarb} on the SGML standard.

    The concepts lying behind both \latex{} and SGML are similar, but on the face
of it they are distinctly different in both syntax and capabilities. ISO is
migrating towards electronic versions of its standard documents and, naturally, 
would prefer these to be SGML tagged. 
     Like \latex, SGML has a
concept of style files, which are termed DTDs, and both systems support
powerful macro-like capabilities. SGML provides for logical document
markup and not typesetting --- commercial SGML systems often use
\TeX{} or \latex{} as their printing engine, as does the NIST SGML
environment for ISO~10303~\bref{pandl}.



NIST have SGML tagged some ISO~10303 documents 
using manual methods, which are time consuming and expensive. 
About 1997 there was a NIST 
effort underway to develop an auto-tagger that would (semi-) automatically convert
a \latex{} tagged document to one with SGML tags. This tool assumed a
fixed set of \latex{} macros and a fixed DTD.
 The design of an auto-tagger
essentially boils down to being able to convert from a source document tagged
according to a \latex{} style file to one which is tagged according to an
SGML DTD.
    Fully automatic conversion is really only possible if the authors'
of the documents to be translated avoid using any `non-standard' macros within
their documents. There is a program called \file{ltx2x}\index{ltx2x} available
from SOLIS, which replaces \latex{} commands within a document with
user-defined text strings~\bref{ltx2x}. This can be used as a basis for
a \latex{} to whatever auto-tagger, provided the \latex{} commands are not
too exotic.

    HTML is a simple markup language, based on SGML, and is used for the
publication of many documents on the Web. XML is a subset of SGML and appears
to being taken up by every man and his dog as \emph{the} document markup
language. HTML is being recast in terms of XML instead of SGML. PDF is a page
description language that is a popular format for display of documents 
on the Web.

    \latex{} documents can be output in PDF by using pdfLaTeX. Instead
of a \file{.dvi} file being produced a \file{.pdf} file is output directly.
The best 
results are obtained when PostScript fonts rather than Knuth's cm fonts 
are used. Noting that the \file{iso} class provides an \verb|\ifpdf| command,
a general form for documents to be processed by either \latex{} or pdfLaTeX
is
\begin{verbatim}
\documentclass{isov2}
\usepackage{times}     % PostScript fonts Times, Courier, Helvetica
\ifpdf
  \pdfoutput=1         % request PDF output
  \usepackage[pdftex]{graphicx}
\else
  \usepackage{graphicx}
\fi
...
\end{verbatim}

    There are several converters available to transform a \latex{} document 
into an HTML document, but like \file{ltx2x} they generally do their own
parsing of the source file, and unlike \file{ltx2x} are typically limited
to only generating HTML. Eitan Gurari's \file{TeX4ht}\index{TeX4ht} 
suite is a notable
exception (see Chapter~4 and Appendix~B of~\bref{lwebcom}). It uses the 
\file{.dvi} file as input, so that all the parsing is done by \TeX, and can be
configured to generate a wide variety of output formats.
A set of \file{TeX4ht} configuration files are available for converting
ISO \latex{} documents into HTML\footnote{Later, configuration files for XML
output will be developed.}.

    Some points to watch when writing \latex{} documents that will assist
in translations into *ML are given below. Typically, attention to these points
will make it easier to parse the \latex{} source.

\begin{itemize}
\item Avoid using the \verb|\label|\ixcom{label} command within
      clause headings or captions. It can just as easily be placed immediately
      after these constructs.
\item Avoid using the \verb|\index|\ixcom{index} command within
      clause headings or captions. It can just as easily be placed immediately
      after these constructs.
\end{itemize}






\infannex{Obtaining LaTeX and friends} \label{anx:getstuff}

    \latex{} is a freely available document typesetting system. There are many
public domain additions to the basic system.
The information below gives pointers to where
you can obtain \latex{} etc., from the\index{Internet} Internet. 

    \latex{} runs on a wide variety of hardware, from PCs to Crays. 
Source to build a \latex{} system is freely available via anonymous 
ftp\index{ftp} from what is called CTAN\index{CTAN} 
(Comprehensive \tex{} Archive Network). 
There are three sites; pick the one nearest to you.
\begin{itemize}
\item \url{ftp.dante.de} CTAN in Germany;
\item \url{ftp.tex.ac.uk} CTAN in the UK;
\item \url{ctan.tug.org} CTAN in the USA;
\end{itemize}
The top level CTAN directory for \latex{} and friends is 
\url{/tex-archive}. CTAN contains a wide variety
of (La)TeX sources, style files, and software tools and scripts 
to assist in document processing.

\begin{anote}
    CTAN is maintained by the \tex{} Users Group (TUG). Their homepage
\isourl{http://www.tug.org} should be consulted for the current
list of CTAN sites and mirrors.
\end{anote}

\begin{comment}

\sclause{SOLIS} \index{SOLIS} \label{anx:solis}

    SOLIS is the \emph{SC4 On Line Information Service}. It contains many electronic
sources of STEP related documents. The relevant top level directory is
\url{pub/subject/sc4}.
 In particular, SOLIS contains the source for this document
and the \file{.sty} files, as well as other \latex{} related files. 
The \latex{} root directory is \url{sc4/editing/latex}. The latest
versions of the \latex{} related files are kept in the sub-directory 
\url{latex/current}.
Some \latex{} related programs are also available in the 
\url{latex/programs} sub-directory.

    SOLIS can be reached at \isourl{http://www.nist.gov/sc4}.

\end{comment}


\infannex{Changes in this release} \label{anx:changes}

    Many of the commands and environments have been redefined in order
to match the change in requirements from the the third to the fourth
edition of the ISO Directives. Usage of these is unaffected.

    The following changes have been made in this release:
\begin{itemize}
\item The \verb|cover|\ixenv{cover} environment has been added;

\item The boolean test \verb|\ifpdf|\ixcom{ifpdf} has been added;

\item The command \verb|\fwdnopatents|\ixcom{fwdnopatents} has been added
      for patent boilerplate in the Foreword.

\item The command \verb|\tpasfwdbp|\ixcom{tpasfwdbp} has been added
      for TS/PAS Foreword boilerplate.

\item The command \verb|\intropatents|\ixcom{intropatents} has been added
      for patent boilerplate in the Introduction.

\item The commands \verb|\pref|\ixcom{pref} and 
      \verb|\pagerefname|\ixcom{pagerefname} have been added.

\item The command \verb|\trwfwdbpi|\ixcom{trfwdbpi} for Foreword boilerplate
      in a Technical Report has been deleted.

\item The \verb|notes|\ixenv{notes} and \verb|examples|\ixenv{examples}
      environments have been deleted.

\item Support is provided for the \file{hyperref}\ixpack{hyperref}
      package.

\end{itemize}



\bibannex
%\label{biblio}

\begin{references}
\reference{LAMPORT, L.,}{LaTeX --- A Document Preparation System,}
            {Addison-Wesley Publishing Co., 2nd edition, 1994.} \label{lamport}
\reference{WILSON, P.R.,}{LaTeX files for typesetting ISO standards:
           Source code,}
           {NISTIR,  National Institute of Standards and Technology,
           Gaithersburg, MD 20899. June 1996.} \label{isoe}
\reference{WILSON, P.R.,}{LaTeX package files for ISO~10303: User manual,}
           {NISTIR,  National Institute of Standards and Technology,
           Gaithersburg, MD 20899. June 1996.} \label{stepsty}
\reference{RAHTZ, S., and BARROCA, L.,}{A style option for rotated
           objects in \latex,}{ TUGBoat, volume 13, number 2, pp 156--180, 
           July 1992.} \label{rahtz}
\reference{GOOSSENS, M., MITTELBACH, F. and SAMARIN, A.,}{%
           The LaTeX Companion,}
           {Addison-Wesley Publishing Co., 1994.} \label{goosens}
\reference{GOOSSENS, M., and RAHTZ, S.,}{%
           The LaTeX Web Companion --- Integrating TeX, HTML and XML,}
           {Addison-Wesley Publishing Co., 1999.} \label{lwebcom}
\reference{CHEN, P. and HARRISON, M.A.,}{Index preparation and
           processing,}{Software--Practice and Experience, 19(9):897--915,
           September 1988.} \label{chen}
%\reference{KOPKA, H. and DALY, P.W.,}{A Guide to LaTeX,}
%           {Addison-Wesley Publishing Co., 1993.} \label{kopka}
%\reference{WALSH, N.,}{Making TeX Work,}{O'Reilly \& Associates, Inc.,
%           103 Morris Street, Suite A, Sebastopol, CA 95472. 1994. } \label{walsh}
\reference{ISO 8879:1986,}{Information processing --- 
                                Text and office systems ---
           Standard Generalized Markup Language (SGML).}{} \label{sgml}
\reference{GOLDFARB, C.F.,}{The SGML Handbook,}
           {Oxford University Press, 1990.} \label{goldfarb}
\reference{BRYAN, M.,}{SGML --- An Author's Guide to the Standard Generalized
           Markup Language,}{Addison-Wesley Publishing Co., 1988. } \label{bryan}
\reference{PHILLIPS, L. and LUBELL, J.,}{An SGML Environment for STEP,}%
          {NISTIR 5515, National Institute of Standards and Technology,
           Gaithersburg, MD 20899. November 1994.} \label{pandl}
\reference{WILSON, P. R.,}{LTX2X: A LaTeX to X Auto-tagger,}%
          {NISTIR, National Institute of Standards and Technology,
           Gaithersburg, MD 20899. June 1996.} \label{ltx2x}
\begin{comment}
\reference{RESSLER, S.,}{The National PDES Testbed Mail Server User's Guide,}
           {NSTIR 4508, National Institute of Standards and Technology,
           Gaithersburg, MD 20899. January 1991.} \label{ressler}
\reference{RINAUDOT, G.R.,}{STEP On Line Information Service (SOLIS),}
          {NISTIR 5511, National Institute of Standards and Technology,
          Gaithersburg, MD 20899. October 1994. } \label{rinaudot}
\reference{KROL, E.,}{The Whole Internet --- User's Guide \& Catalog,}
           {O'Reilly \& Associates, Inc.,
           103 Morris Street, Suite A, Sebastopol, CA 95472. 1993. } \label{krol}
\end{comment}
\reference{WILSON, P.R.,}{The hyphenat package,}%
          {1999. (Available from CTAN)} \label{bib:hyphenat}
\reference{WILSON, P.R.,}{The xtab package,}%
          {1998. (Available from CTAN)} \label{bib:xtab}
\end{references}

    
%%%%%%% here is the index at the end
%%\input{isomanidx}
% isoman.tex   User guide for ISO style files for LaTeX(2e) August 2002
%
\documentclass[wd,letterpaper,copyright]{isov2}
%\documentclass[wd,letterpaper,draft]{isov2}
%\documentclass{isov2}

%%%\usepackage[isoman]{tex4ht}

\usepackage{comment}

%%%\usepackage{times}
\usepackage{isorot}
\usepackage{xtab}
\usepackage{hyphenat}

\ifpdf
  \pdfoutput=1
  \usepackage[plainpages=false,
              pdfpagelabels,
              bookmarksnumbered,
              hyperindex=true
             ]{hyperref}
\fi

\standard{LaTeX for ISO standards}
\yearofedition{2002}
\languageofedition{(E)}

%\partno{3456}
\changemarkstrue

\makeindex

%  Uncomment the following to change the Foreword heading
%\renewcommand{\forewordname}{Avant-propos} % change the Foreword title

\setcounter{tocdepth}{3} % add more levels to table of contents
%
%    remainder of preamble is some special macro definitions
\makeatletter
%   the \meta{} command
%
\begingroup
\obeyspaces%
\catcode`\^^M\active%
\gdef\meta{\begingroup\obeyspaces\catcode`\^^M\active%
\let^^M\do@space\let \do@space%
\def\-{\egroup\discretionary{-}{}{}\hbox\bgroup\it}%
\m@ta}%
\endgroup
\def\m@ta#1{\leavevmode\hbox\bgroup\texttt{<}\textit{#1}\/\texttt{>}\egroup
    \endgroup}
\def\do@space{\egroup\space
    \hbox\bgroup\it\futurelet\next\sp@ce}
\def\sp@ce{\ifx\next\do@space\expandafter\sp@@ce\fi}
\def\sp@@ce#1{\futurelet\next\sp@ce}
%
%
\makeatother
%
% the \latex command
\newcommand{\latex}{LaTeX}
\newcommand{\tex}{TeX}
%
%   the \file{} command
%
\newcommand{\file}[1]{\textsf{#1}}
%
\makeatletter
%   index a command
\newcommand{\bs}{\symbol{'134}}
\newcommand{\ixcom}[1]{\index{#1/ @{\tt \protect\bs #1}}}
%   index an environment
\newcommand{\ixenv}[1]{\index{#1 @{\tt #1} (environment)}}
%   index a starred environment
\newcommand{\ixenvs}[1]{\index{#1s @{\tt #1*} (environment)}}
%   index an option
\newcommand{\ixopt}[1]{\index{#1 @{\tt #1} (option)}}
%   index a package
\newcommand{\ixpack}[1]{\index{#1 @\file{#1} (package)}}
%   index a class
\newcommand{\ixclass}[1]{\index{#1 @\file{#1} (class)}}
%   index in typewriter font
\newcommand{\ixtt}[1]{\index{#1@{\tt #1}}}
%   index LaTeX
\newcommand{\ixltx}{\index{latex@\latex}}
%   index LaTeX 2e
\newcommand{\ixltxe}{\index{latex2e@\latex 2e}}
%   index LaTeX v2.09
\newcommand{\ixltxv}{\index{latex209@\latex{} v2.09}}
\makeatother
%
%   end of preamble
%
\begin{document}


\begin{cover}
\vspace*{4in}
\begin{center}
\Huge\bfseries LaTeX for ISO standards
\end{center}
\begin{center}
\bfseries 2002/08/10
\end{center}
\begin{center}
Peter Wilson \\
\texttt{peter.r.wilson@boeing.com}
\end{center}
\clearpage
\end{cover}
%%%\clearpage

\begin{foreword}
\fwdbp
\fwdnopatents
Annexes~\ref{anx:extraiso} and~\ref{anx:lord}
are an integral part of this document.
Annexes~\ref{anx:indexing}, \ref{anx:sgml}, \ref{anx:getstuff},
and \ref{anx:changes}
are for information only.
\end{foreword}


\begin{introduction}

    This document describes the use of the \file{isov2} \latex{} class 
file and some package files in preparing ISO standard documents.

\sclause*{Overview}

    This document describes a \latex{} class file, called \file{isov2},
 for typesetting ISO standards. It also
provides descriptions of some particular package files (e.g., \file{isorot}) 
that have been developed to support the writing of ISO standards.
The electronic source of this  document also provides an example of the
use of these files.



    The current set of files~\bref{isoe} have been developed by 
Peter Wilson (RPI, CUA and NIST, Boeing)
from files written by Kent Reed (NIST).
In turn, these were revisions of files originally created by 
Phil Spiby (CADDETC, UK), based on early work by 
Phil Kennicott (GE).\footnote{In mid 1994 \latex{} was upgraded from 
version 2.09 to what is called \latex 2e. The files described in 
earlier versions this document were compatible with both versions 
of \latex. Starting with the October 1997 release, 
support was withdrawn from any \latex{} v2.09 file versions.}


    Documents produced with the \latex{} files have been reviewed 
by the ISO Editorial Board in Geneva for conformance to their typographical
requirements. The first review was of a series of Draft International 
Standard documents. This review resulted in some changes to the style 
files. The second review was of a series of twelve International Standard 
documents (ISO 10303:1994). Likewise, this review led to changes in the 
style files to bring the documents into compliance.

    With the publication of the ISO~10303:1994 standard, the opportunity 
was taken
to provide a new baseline release of the package files.
The new baseline release was also designed to address the fact that
a major update of \latex{} to \latex 2e took place during 1994. \latex 2e
is now the officially supported version. However, some users needed time
to convert to using \latex 2e. Consequently, although the 
facilities described in the original version of this document 
were common for both 
\latex{} v2.09 and \latex 2e users, there were two different 
implementations.

    Because ISO standard documents have a very structured layout, the class
and package files have been designed to reflect the logical document structure
to a much greater extent than the `standard' \latex{} files. Further, ISO
documents are published in more than one language. The files described
here are written for the English language, but the language-specific elements
have been parameterized for easy modification for publication in other
official ISO languages, such as French.

    In 1997 ISO produced a new version of their Directives on the 
requirements for the layout of ISO documents. These were not completely
unambiguous as to their intent; the current version was released in
2001. Members of ISO TC184/SC4 have worked with
the ISO Editorial Board and have reached an agreement that more precisely
identifies the requirements. The current version of the \latex{} files 
implements that agreement.

\begin{note}
The original of this manual has been typeset using the \verb?draft?
option in order to display its effect of placing a black box at the 
end of each line that is longer than the normal width of the text.
\end{note}

\begin{note}
The author of this document may be contacted at 
\texttt{peter.r.wilson@boeing.com}.
\end{note}

\end{introduction}

\title{LaTeX for ISO standards: User manual}%
{Peter Wilson}%
{August 2002}


\scopeclause

This manual describes a set of \ixltx\latex{} files for typesetting 
documents according to the ISO Directives Part 3 (third edition, 1997), 
together with some elements from the ISO~10303 Supplementary Directives.

\begin{inscope}{manual}
\item use of \latex{} for preparing ISO standard documents.
\end{inscope}

\begin{outofscope}{manual}
\item use of \latex{} in general;
\item use of other document preparation systems.
\end{outofscope}

\textbf{IMPORTANT:} No matter whether or not there are copyright attributions
to ISO, this document is \emph{not} copyrighted by ISO. Any copyright
markings are for illustrative purposes only. This document is released under
the LaTeX Project Public Licence.

\normrefsclause \label{sec:nrefs}

\normrefbp{manual}
\begin{nreferences}

\isref{ISO/IEC Directives, Part 2}{Rules for the structure and drafting of
       International Standards. (Fourth edition, 2001)}

\isref{ISO TC 184/SC4 N1217:2001(E)}{SC4 Supplementary directives --- Rules
       for the structure and drafting of SC4
       standards for industrial data. (2001--11--01).}

\isref{ISO/IEC 8824-1:1998}{Information technology ---
       Abstract Syntax Notation One (ASN.1):
       Specification of basic notation.}

\disref{ISO}{Camera-ready copy (CRC) ---
       Submission requirements and ISO type specification.
       (Version 1.0, 1996-04-26)}

\end{nreferences}

\defabbclause
%\clause{Terms, definitions, and abbreviations}

\defsubclause
%\sclause{Terms and definitions}

    For the purposes of this manual, the following definitions
apply.

\begin{definitions}
\definition{boilerplate}{text whose wording is fixed and has been agreed
 to be present in a specific type of document} \index{boilerplate}
\definition{style file}{a set of \latex{} macros assembled into a single 
   file with an extension \file{.sty}}
            \index{style file}\ixltx\index{file!.sty}
\definition{package file}{a style file for use with \latex 2e}\ixltxe
\end{definitions}

\abbsubclause
%\sclause{Abbreviations}

    For the purposes of this manual, the following abbreviations
 apply.

\begin{symbols}
\symboldef{CD}{Committee Draft}\index{CD}
\symboldef{DIS}{Draft International Standard}\index{DIS}
\symboldef{FDIS}{Final Draft International Standard}\index{DIS}
\symboldef{IS}{International Standard}\index{IS}
\symboldef{IS-REVIEW}{The documentation style accepted by the ISO 
    Editorial Board review (September 1994) of twelve IS documents 
    (the initial release of ISO~10303) for compliance with ISO 
    typographical and layout requirements.}\index{IS-REVIEW}
\symboldef{ISOD}{ISO/IEC Directives, Part 2}\index{ISOD}\index{ISO/IEC Directives}
\symboldef{PAS}{Publicly Available Specification}\index{PAS}
\symboldef{SD}{SC4 Supplementary directives}\index{SD}\index{Supplementary directives}
\symboldef{TR}{Technical Report}\index{TR}
\symboldef{TS}{Technical Specification}\index{TS}
\symboldef{WD}{Working Draft}\index{WD}
\symboldef{CRC}{The ISO \emph{Camera-ready copy (CRC)} document}\index{CRC}
\symboldef{ToC}{table of contents}\index{ToC}
\end{symbols}

\clause{Conformance requirements}  \label{sec:iconform}

    The \latex{} macro source files shall not be modified.

    If there is a need to modify the macro definitions then the
modifications shall be defined in a separate \file{.sty}\index{file!.sty}
 file (or files), using the
\latex{} \verb|\renewcommand|\ixcom{renewcommand}
and/or the 
\verb|\renewenvironment|\ixcom{renewenvironment}
commands as appropriate. The resulting \file{.sty} file(s) shall then
be called in within the preamble\index{preamble} portion of the 
document to be typeset.

    Author specified \verb|\label{...}| commands shall not start with
the characters \verb|;i| (semicolon and `i'); definition of labels
starting with these characters is reserved for the maintainer of the
facility files.


\fcandaclause
%\clause{Fundamental concepts and assumptions}

    It is assumed that the reader of this document is familiar with the
\latex{} document preparation system.\ixltx

\begin{note}Reference~\bref{lamport} in the bibliography describes the
      \latex{} system.
\end{note}

    The reader is also assumed to be familiar with the ISO/IEC Directives 
Part~2 (ISOD\index{ISOD}). Agreements reached between the ISO Editorial Board
and ISO TC184/SC4 are documented in the SC4 Supplementary Directives
(SD\index{SD}).

    If there are any discrepancies between the layout and wording of this 
document and the requirements of the ISO/IEC Directives Part~2, 
then the requirements in that document shall be
followed for any ISO standard document.

   The \file{isov2}\ixclass{isov2} class requires the 
\file{url}\ixpack{url} package.

\begin{note}
Additional \latex{} facilities specifically designed for ISO~10303 are
defined and described elsewhere~\bref{stepsty}.
\end{note}

    Because of many revisions over the years to the files described
herein, a naming convention has been adopted for them.
The primary name of the file is suffixed by \file{v\#} or \file{\#}, where
\file{\#} is the version number of the file in question. 
All file primary names have been limited to a 
maximum of eight characters.

\begin{note}
    Table~\ref{tab:curfiles} shows the versions of the files that were
current at the time of publication.
\end{note} % end note
\begin{note}
Starting with the October 1997 release, files that were specific to 
\ixltxv\latex{} v2.09 are no
longer either supported or supplied.
\end{note}

\begin{note}
As of 1999, the \file{uschyp}\ixpack{uschyp} package is no longer supported. 
It has been replaced by the 
\file{hyphenat}\ixpack{hyphenat}\index{hyphenat.sty@\file{hyphenat.sty}}
package.
\end{note}

\ixclass{isov2}\index{isov2.cls@\file{isov2.cls}}
%             \index{isonev11.sty@\file{isonev11.sty}}
\ixopt{9pt}\index{iso9.clo@\file{iso9.clo}}
%           \index{iso9.sty@\file{iso9.sty}}
\ixopt{10pt}\index{iso10.clo@\file{iso10.clo}}
%            \index{iso10.sty@\file{iso10.sty}}
\ixopt{11pt}\index{iso11.clo@\file{iso11.clo}}
%            \index{iso11.sty@\file{iso11.sty}}
\ixpack{isorot}\index{isorot.sty@\file{isorot.sty}}
%               \index{isrotne1.sty@\file{isrotne1.sty}}
\ixpack{askinc}\index{askincv1.sty@\file{askincv1.sty}}
\ixpack{xtab}\index{xtab.sty@\file{xtab.sty}}
\ixpack{uschyp}\index{uschyp.sty@\file{uschyp.sty}}
%               \index{uschypne.sty@\file{uschypne.sty}}

\begin{table}
\centering
\caption{Current file versions} \label{tab:curfiles}
\begin{tabular}{|l|l|} \hline
\textbf{Facility} & \textbf{File}          \\ \hline\hline
\file{iso}     & \file{isov2.cls}    \\
9pt option     & \file{iso9.clo}     \\
10pt option    & \file{iso10.clo}    \\
11pt option    & \file{iso11.clo}    \\
\file{isorot}  & \file{isorot.sty}   \\
\file{askinc}  & \file{askincv1.sty} \\
\file{xtab}    & \file{xtab.sty}   \\
\file{hyphenat}  & \file{hyphenat.sty}   \\ \hline
\end{tabular}
\end{table}


\begin{anexample} At the time of 
publication of this document, any references to \file{iso.cls} should be
read as actually referring to \file{isov2.cls}, and similarly for references
to other files.
\end{anexample} % end example


\begin{note}This document is not intended for publication as a standard, 
although it has been laid out in a 
similar, but not necessarily identical, manner.\end{note} % end note


\clause{The \file{iso} class facility} 

    The \latex{} \file{isov2}\ixclass{isov2} class file
is a general file for use in preparing ISO
standard documents using the \latex{} document preparation system.

    As usual, any \latex{} document has the following structure:
\begin{verbatim}
\documentclass[<list of options>]{isov2}
% preamble goes here
\begin{document}
% document body goes here
\end{document}
\end{verbatim}

\sclause{Options}

    The \file{isov2}\ixclass{isov2} class file supports the following options: 
\begin{itemize}
\item \verb|draft|\ixopt{draft} for a draft document where 
      overfull horizontal boxes are marked, marginal notes are allowed,
      and ISO copyright text is not placed in the document;
\item \verb|final|\ixopt{final} the opposite of \verb|draft|
      (this is the default);
\item \verb|letterpaper|\ixopt{letterpaper} for printing 
      on US letter size paper;
\item \verb|a4paper|\ixopt{a4paper} for printing on A4
      size paper (this is the default);
\item \verb|twocolumn|\ixopt{twocolumn} for two column 
      formatting;
\item \verb|onecolumn|\ixopt{onecolumn} for single column
      printing (this is the default);
\item One of \verb|11pt|, \ixopt{11pt}
      \verb|10pt|, \ixopt{10pt}
      \verb|9pt|\ixopt{9pt} for 11pt, 10pt or 9pt printing
      respectively (the default is \verb|11pt|);
\item \verb|notcopyright|\ixopt{notcopyright} for 
      disabling the printing of copyright notices (this is the default);
\item \verb|copyright|\ixopt{copyright} enables printing
      of copyright notices;
\item \verb|is|\ixopt{is} for International Standard documents;
\item \verb|fdis|\ixopt{fdis} for  Final Draft
      International Standard documents; \index{FDIS}
\item \verb|dis|\ixopt{dis} for  Draft
      International Standard documents; \index{DIS}
\item \verb|cd|\ixopt{cd} for  Committee Draft
      standard documents; \index{CD}
\item \verb|wd|\ixopt{wd} for  Working Draft
      standard documents; \index{WD}
\item \verb|pas|\ixopt{pas} for Publicly Available Specification
      standard documents;
\item \verb|techrep|\ixopt{techrep} for Technical
      Report standard documents; \index{TR}
\item \verb|techspec|\ixopt{techspec} for Technical
      Spefication standard documents; \index{TS}
\item \verb|otherdoc|\ixopt{otherdoc} for documents
      that are not intended to become a standard (this is the default);
%\item \verb|uglycaption|\ixopt{uglycaption} to produce
%      an ugly style of captioning;
\item Any other facilities that are available via packages.
\end{itemize}

    When no options are specified, then the result is 11pt, single column
printing on A4 size paper, without copyright notice and a running header.
That is, by default, the options set are: 
\verb|final|\ixopt{final};
\verb|a4paper|\ixopt{a4paper}; 
\verb|onecolumn|\ixopt{onecolumn};
\verb|11pt|\ixopt{11pt};
\verb|notcopyright|\ixopt{notcopyright}; and
\verb|otherdoc|\ixopt{otherdoc}.


%\begin{note}ISOD\index{ISOD} calls for 9pt double column  printing but the 
%      SD\index{SD} calls for 11pt single column printing.
%      Using 9pt double column printing is awkward if any computer 
%      code has to
%      be typeset in the document. The IS-REVIEW\index{IS-REVIEW} accepted
%      11pt single column layout. \end{note} % end note
%\begin{note}The CRC\index{CRC} states that acceptable founts are: Univers,
%    Helvetica, and Times Roman with the body text in 10pt. 
%    The IS-REVIEW accepted camera-ready copy
%    using Computer Modern 11pt set in single column. \end{note}
\begin{note}The user of the \file{isov2}\ixclass{isov2} class is encouraged 
      to process this document using
      different combinations of the options to gain experience with
      their effects. This printing of the document is typeset using the
      \verb|draft|\ixopt{draft} option.
\end{note} % end note
\begin{note}
    The \verb|otherdoc|\ixopt{otherdoc} option 
was chosen as the default so that the
current stage of ISO standardardization has to be explicitly declared
as an option, and is therefore available to any software package that
might process the document source (e.g., a \latex{} to SGML translator).
\end{note}


\sclause{Sectioning commands}

    Table~\ref{tab:sec} gives the sectioning commands defined for \file{isov2}
class documents.
\ixcom{clause} \ixcom{normannex} \ixcom{infannex} \ixcom{repannex} 
\ixcom{sclause}
\ixcom{ssclause}
\ixcom{sssclause}
\ixcom{ssssclause}
\ixcom{sssssclause}

\begin{table}
\centering
\caption{Sectioning commands.} \label{tab:sec}
\begin{tabular}{|c|l|} \hline
\textbf{Level} & \textbf{Command} \\ \hline\hline
1   & \verb|\clause|, \verb|\normannex|, \verb|\infannex|, \verb|\repannex| \\
2           & \verb|\sclause| \\
3           & \verb|\ssclause| \\
4           & \verb|\sssclause| \\
5           & \verb|\ssssclause| \\
6           & \verb|\sssssclause| \\ \hline
\end{tabular}
\end{table}

%    The \verb|\section|\ixcom{section} command
% is similar to the normal \latex{} \verb|\part|\ixcom{part}
%command. It is only available when the \verb|sect|\ixopt{sect}
%option is specified.

    The \verb|\clause|\ixcom{clause} commands are similar to 
the normal \latex{} \verb|\section|ing 
commands. There are also starred versions of these commands 
(e.g., \verb|\clause*|).

    Three varieties of annex\index{annex} commands are available:
\begin{itemize}
\item \verb+\normannex{+\meta{title}\verb+}+\ixcom{normannex}
      for a normative annex;
\item \verb+\infannex{+\meta{title}\verb+}+\ixcom{infannex}
       for an informative annex;
\item \verb+\repannex{+\meta{title}\verb+}+\ixcom{repannex}
      for an annex that is neither normative or
     informative (e.g. an annex in a technical report).
\end{itemize}

    Only \verb|\sclause|\ixcom{sclause} and lower level 
sectioning commands can be used after
an annex\index{annex} command.


\begin{example}
    The command  \verb|\infannex{Technical discussion}|, assuming that this is the first
annex in the document, produces:
    \begin{center}
    \textbf{Annex A}\\
    (informative)

    \textbf{Technical discussion}
    \end{center}
\end{example} % end example
\begin{example}
  The command \verb|\repannex{Title of annex}|, assuming that this is the second
annex in the document, produces:

\begin{center}
  \textbf{Annex B}

  \textbf{Title of annex}
\end{center}
\end{example} % end example


\ssclause{The use of \texttt{tocdepth} and \texttt{secnumdepth}}

    In the standard \latex{} classes the
\verb|tocdepth|\ixtt{tocdepth} and 
\verb|secnumdepth|\ixtt{secnumdepth} counters
are set in the preamble\index{preamble} to respectively control the level at
which clause titles are inserted into a table of contents (ToC)\index{ToC} 
and at which clause numbering ceases.

    In the \file{isov2}\ixclass{isov2} class, 
the values of these can be changed at 
any point in the document. The change lasts until another change is
made to the value. 
\begin{example}
    Assume that in the preamble we have
\begin{verbatim}
\setcounter{secnumdepth}{3}  % number ssclauses and above
\setcounter{tocdepth}{3}     % ToC includes ssclauses and above
\end{verbatim}
and that a certain subclause has subsubclauses that should be numbered but
not put into the table of contents, then we could do:
\begin{verbatim}
...
\sclause{The certain subclause}
\setcounter{tocdepth}{2}
...
\ssclause{Numbered but not in ToC}
...
...
\setcounter{tocdepth}{3}
\sclause{Following subclause}
\end{verbatim}

\end{example}

    It can sometimes be difficult to remember which level number corresponds
to which kind of clause. Accordingly, a set of commands are provided to ease
this task. These commands can only be used after the preamble.
\begin{itemize}
\item \verb|\maxsecnumdepth{|\meta{sec}\verb|}|\ixcom{maxsecnumdepth}
      sets the level at which clauses will be numbered. This command
      should be used before the first sectioning command.
\item \verb|\maxtocdepth{|\meta{sec}\verb|}|\ixcom{maxtocdepth}
      sets the level at which clauses will be put into the ToC.
      This command should be used before the \verb|\tableofcontents| command.
\item \verb|\setsecnumdepth{|\meta{sec}\verb|}|\ixcom{setsecnumdepth}
      sets the current level at which clauses will be numbered. 
      This command can be used anywhere after the preamble.
\item \verb|\settocdepth{|\meta{sec}\verb|}|\ixcom{settocdepth}
      sets the current level at which clauses will put into the ToC. 
      This command can be used anywhere after the preamble.
\end{itemize}
The value of the argument \meta{sec} can be any of the following:
\texttt{clause}, \texttt{sclause}, \ldots, \texttt{sssssclause}.

\begin{example}
Using these commands, the previous example can also be coded as:
\begin{verbatim}
...
\begin{document}
\maxsecnumdepth{ssclause}
\maxtocdepth{ssclause}
...
\sclause{The certain subclause}
\settocdepth{sclause}
...
\ssclause{Numbered but not in ToC}
...
...
\settocdepth{ssclause}
\sclause{Following subclause}
\end{verbatim}

\end{example}



\sclause{LaTeX environments and commands}

    Many of the standard \latex{} environments and commands are available.
In particular, all the normal mathematical typesetting capabilities are
present.
However, there are some additional environments and commands defined.

\ssclause{Lists}

    The standard \latex{} \verb|itemize|,\ixenv{itemize}
\verb|enumerate|\ixenv{enumerate} and \verb|description|\ixenv{description}
environments are provided. The labels in these lists, though, differ from
those normally provided by \latex.

\begin{note}
The ISOD describes only a single level for an itemized list, being marked 
with either an em-dash or a bullet. The SD deprecates the bullet but 
provides four levels, each being marked with an em-dash. These are 
provided in the \file{isov2} class.
\end{note}
\begin{note}
The ISOD allows for two levels of enumerated lists. The SD extends this
to four levels, and these are provided in the \file{isov2} class.
\end{note}

\begin{example}
The list environments provided are shown below as:\ixenv{itemize}
\begin{verbatim}
\begin{itemize}
\item First level itemized element;
  \begin{itemize}
  \item Second level itemized element;
    \begin{itemize}
    \item Third level itemized element;
      \begin{itemize}
      \item Fourth level itemized element.
      \end{itemize}
    \end{itemize}
  \end{itemize}
\end{itemize}
\end{verbatim}

\ixenv{enumerate}
\begin{verbatim}
\begin{enumerate}
\item First level enumerated element;
  \begin{enumerate}
  \item Second level enumerated element;
    \begin{enumerate}
    \item Third level enumerated element;
      \begin{enumerate}
      \item Fourth level enumerated element.
      \end{enumerate}
    \end{enumerate}
  \end{enumerate}
\end{enumerate}
\end{verbatim}

\ixenv{description}
\begin{verbatim}
\begin{description}
\item[Description] a description element. Note that a colon is
                   automatically added to the item label.
\end{description}
\end{verbatim}
and they produce
\begin{itemize}
\item First level itemized element;
  \begin{itemize}
  \item Second level itemized element;
    \begin{itemize}
    \item Third level itemized element;
      \begin{itemize}
      \item Fourth level itemized element.
      \end{itemize}
    \end{itemize}
  \end{itemize}
\end{itemize}

\begin{enumerate}
\item First level enumerated element;
  \begin{enumerate}
  \item Second level enumerated element;
    \begin{enumerate}
    \item Third level enumerated element;
      \begin{enumerate}
      \item Fourth level enumerated element.
      \end{enumerate}
    \end{enumerate}
  \end{enumerate}
\end{enumerate}

\begin{description}
\item[Description] a description element. Note that a colon is
                   automatically added to the item label.
\end{description}

\end{example} % end example

\begin{example}
    Here is a more complex set of lists:

\begin{itemize}
\item First level itemization
  \begin{enumerate}
  \item First level enumeration
    \begin{itemize}
    \item Second level itemization
      \begin{enumerate}
      \item Second level enumeration
        \begin{itemize}
        \item Third level itemization
%          \begin{enumerate}    % At this point the list is too deeply nested.
%          \item Third level enumeration    % For the TeXnophile this is
%          \item 3rd level enumeration      % because the example environment
%         \end{enumerate}                   % is itself a list.
        \item 3rd level itemization
        \end{itemize}
      \item 2nd level enumeration
      \end{enumerate}
    \item 2nd level itemization
    \end{itemize}
  \item 1st level enumeration
  \end{enumerate}
\item 1st level itemization
\end{itemize}

\end{example} % end example

\begin{note}
On occasion, \latex{} objects to a well formed list. Typically, this happens
when a list has only one entry. \index{list!error}
 The \latex{} error message is of the form:
\begin{verbatim}
! Something's wrong--perhaps a missing \item.
\end{verbatim}
Hitting the \verb|<return>| key usually gets \latex{} to run on happily.
\end{note}% end note
\begin{note}
Similarly, you may get the message
\begin{verbatim}
! Too deeply nested
\end{verbatim}
Again, hit \verb|<return>| and processing should continue. However,
the indentation of later lists may not be correct.
\end{note} % end note


\ssclause{Notes}

    
    Two different kinds of environments are provided for typesetting notes.

\sssclause{Numbered notes}

    The environment 
\verb+\begin{note}+ \meta{text} \verb+\end{note}+\ixenv{note}
produces a numbered note whose body consists
of \meta{text}.

\begin{anexample}
The commands
\begin{verbatim}
\begin{note}Numbered note.\end{note}
\end{verbatim}
produce

\begin{note}Numbered note.\end{note}
\end{anexample}

\sssclause{Isolated notes}

    The environment 
\verb+\begin{anote}+ \meta{text} \verb+\end{anote}+\ixenv{anote}
produces an un-numbered note whose body
consists of \meta{text}.

\begin{anexample}
    The commands
\begin{verbatim}
\begin{anote}Plain note.\end{anote}
\end{verbatim}
produce

\begin{anote}Plain note.\end{anote}
\end{anexample} % end example


\ssclause{Examples} \label{ssc:ex}

    Two different kinds of environments are provided for typesetting 
examples.

\sssclause{Numbered examples} \label{cl:numex}

The environment \verb?\begin{example}? \meta{text} \verb?\end{example}?
produces a numbered example whose body consists of \meta{text}.

\begin{anexample}
The commands
\begin{verbatim}
\begin{example}Numbered example. \label{ex:exref} \end{example}
\end{verbatim}
produce

\begin{example}Numbered example. \label{ex:exref} \end{example}
\end{anexample}

\begin{anote}
Numbered examples (and numbered notes) may be identified using the
\verb?\label{...}?\ixcom{label} command, as exhibited in \eref{ex:exref}
in \ref{cl:numex}, and then referred to by the 
\verb?\ref{...}?\ixcom{ref} command.
\end{anote}

\sssclause{Isolated examples}

The environment \verb?\begin{anexample}? \meta{text} \verb?\end{anexample}?
produces an un-numbered example whose body consists of \meta{text}.
\ixenv{anexample}

\begin{anexample}
The commands
\begin{verbatim}
\begin{anexample}Isolated example.\end{anexample}
\end{verbatim}
produce

\begin{anexample}Isolated example.\end{anexample}
\end{anexample}



\ssclause{Bibliographic references}

    Two different kinds of environments are provided for bibliographic 
references. It should be noted that neither of these has anything to do with
BibTeX.\index{BibTeX}

\sssclause{Normative references}

    Normative references are listed in the 
\verb|nreferences|\ixenv{nreferences} environment.

    In this environment, bibliographic entries are denoted by either
\verb+\isref{+\meta{ref}\verb+}{+\meta{title}\verb+}+ or by 
\verb+\disref{+\meta{ref}\verb+}{+\meta{title}\verb+}+. The
\meta{ref} parameter is the number of the standard document and 
the \meta{title} parameter is the title of the standard. 

    Use the \verb|\isref|\ixcom{isref} command for published standards 
and the \verb|\disref|\ixcom{disref}
 command for documents that have not yet been 
finally approved as a standard. This latter command automatically adds a
footnote to the effect  that the document is
to be published.

\begin{example}
The \latex{} source for the references in clause~\ref{sec:nrefs} of 
this document is:
\begin{verbatim}
\begin{nreferences}
\isref{ISO/IEC Directives, Part 2}{Rules for the structure and drafting of
       International Standards. (Fourth edition, 2001)}
...
...
\disref{ISO}{Camera-ready copy (CRC) ---
       Submission requirements and ISO type specification.
       (Version 1.0, 1996-04-26)}
\end{nreferences}
\end{verbatim}
\end{example} % end example


\sssclause{Informative references}

    Informative bibliography elements are listed in the
\verb|references|\ixenv{references} environment. 
Each element in the list is specified as 
\verb+\reference{+\meta{author}\verb+}{+\meta{title}\verb+}{+\meta{publisher}\verb+}+.\ixcom{reference}

\begin{example}
The \latex{} source for the bibliography in one version of this document was:
\begin{verbatim}
\infannex{Bibliography}
\begin{references}
\reference{LAMPORT, L.,}{\latex\/ A Document Preparation System,}%
           {Addison-Wesley Publishing Co., 1986} \label{lamport}
\reference{GOOSSENS, M., MITTELBACH, F. and SAMARIN, A.,}{%
           The \latex\/ Companion,}
           Addison-Wesley Publishing Co., 1994} \label{goosens}
\reference{CHEN, P. and HARRISON, M.A.,}{Index preparation and
           processing,}{Software--Practice and Experience, 19(9):897--915,
           September 1988.} \label{chen}
....
\end{references}

\end{verbatim}
\end{example} % end example

    Informative references may be cited in the text via the \latex{}
\verb|\label|\ixcom{label} and \verb|\ref|\ixcom{ref}
mechanism. Note that \verb|\cite|\ixcom{cite} is not available
for references to bibliographic items. For the purposes of
ISO documents, the command 
\verb+\bref{+\meta{ref}\verb+}+\ixcom{bref}
is supplied which
properly formats a bibliographic reference.

\ssclause{Listing of scope items}

    The \verb|inscope|\ixenv{inscope} and 
\verb|outofscope|\ixenv{outofscope}
environments are provided for itemized listing of elements that
are within and outside the scope of the standard. Each list
element is introduced via the \verb|\item|\ixcom{item} command. 
Some boilerplate
text is also printed that introduces the scope list.

    The environments take one parameter, \meta{text}, that must read
naturally in a sentence of the form: `The following are within/outside
the scope of this \meta{text}:'. The non-parameterized part of this
sentence is specified by the commands 
|\verb|\inscopename|\ixcom{inscopename} and
\verb|\outofscopename|\ixcom{outofscopename}
 respectively for `within' and `outside'.

\begin{example}The following text was printed by the commands shown at the
end of the example.

\begin{inscope}{part of ISO~10303}
\item use of \latex{} for preparing ISO standard documents;
\item use of \latex{} for preparing ISO~10303 documents.
\end{inscope}

\begin{outofscope}{part of ISO~10303}
\item use of \latex{} in general;
\item use of other document preparation systems.
\end{outofscope}

\begin{verbatim}
\begin{inscope}{part of ISO~10303}
\item use of \latex{} for preparing ISO standard documents;
\item use of \latex{} for preparing ISO~10303 documents.
\end{inscope}

\begin{outofscope}{part of ISO~10303}
\item use of \latex{} in general;
\item use of other document preparation systems.
\end{outofscope}
\end{verbatim}

\end{example} % end example

\ssclause{Listing of definitions}

    The \verb|\begin{olddefinitions}| \ldots \verb|\end{olddefinitions}|
\ixenv{olddefinitions} environment is provided for 
listing terms that have been defined within
the normatively referenced documents. Each term in the list is specified as: \\
\verb+\olddefinition{+\meta{phrase}\verb+}{+\meta{supplement}\verb+}+\ixcom{olddefinition}.


    The \verb|\begin{definitions}| \ldots \verb|\end{definitions}|
\ixenv{definitions} environment is provided for listing 
the definitions of terms specific to the
document being written. Each term in the
list is specified as: \ixcom{definition} \\
\verb+\definition{+\meta{phrase}\verb+}{+\meta{definition text}\verb+}+.


\begin{example}A listing of terms defined elsewhere could be specified as:

\begin{verbatim}
\begin{olddefinitions}
\olddefinition{application protocol (AP)}{}
\olddefinition{integrated resource}{}
\end{olddefinitions}
\end{verbatim}
\end{example} % end example
\begin{example}The definition listing earlier in this document was produced by:

\begin{verbatim}
\begin{definitions}
\definition{boilerplate}{text whose wording is fixed ...}
\definition{style file}{a set of \latex{} macros assembled 
                        into a single file}
...
\end{definitions}
\end{verbatim}
\end{example} % end example


\ssclause{Listing of abbreviations}

    The \verb|\begin{symbols}| \ldots \verb|\end{symbols}|\ixenv{symbols}
environment is provided for listing symbols
and abbreviations. Each term (either symbol or abbreviation) in the list is
specified as: \ixcom{symboldef} \\
 \verb+\symboldef{+\meta{symbol}\verb+}{+\meta{definition text}\verb+}+.

\begin{example}The list of abbreviations earlier in this document was specified as:
\begin{verbatim}
\begin{symbols}
\symboldef{DIS}{Draft International Standard}
\symboldef{IS}{International Standard}
...
\end{symbols}
\end{verbatim}
\end{example} % end example

\sclause{Floating bodies}

    \latex{} provides the \verb|figure|\ixenv{figure}
and \verb|table|\ixenv{table} environments.
Captions (produced by the \verb|\caption|\ixcom{caption}
command) increment the figure or
table number and add the caption to the relevant contents listing file.

\index{floats} \index{floats!continuation}

    The command 
\verb+\contcaption{+\meta{text}\verb+}+\ixcom{contcaption}
may be used instead. This command neither increments the number nor adds
anything to the listing files.

\begin{example}
The following code:
\begin{verbatim}
\begin{table}[tbp]
\centering
\caption{Example table in three parts} \label{tab:cont}
\begin{tabular}{|c|c|} \hline
\textbf{Col 1} & \textbf{Col 2} \\ \hline\hline
A & 1 \\
B & 2 \\ \hline
\end{tabular}
\end{table}

\begin{table}[tbp]
\centering
\contcaption{(continued)}
\begin{tabular}{|c|c|} \hline
\textbf{Col 1} & \textbf{Col 2} \\ \hline\hline
C & 3 \\
D & 4 \\ \hline
\end{tabular}
\end{table}

\begin{table}[tbp]
\centering
\contcaption{(concluded)}
\begin{tabular}{|c|c|} \hline
\textbf{Col 1} & \textbf{Col 2} \\ \hline\hline
E & 5 \\
F & 6 \\ \hline
\end{tabular}
\end{table}
\end{verbatim}
produces the three-part \tref{tab:cont}. \end{example} %end example

\begin{table}[tbp]
\centering
\caption{Example table in three parts} \label{tab:cont}
\begin{tabular}{|c|c|} \hline
\textbf{Col 1} & \textbf{Col 2} \\ \hline\hline
A & 1 \\
B & 2 \\ \hline
\end{tabular}
\end{table}

\begin{table}[tbp]
\centering
\contcaption{(continued)}
\begin{tabular}{|c|c|} \hline
\textbf{Col 1} & \textbf{Col 2} \\ \hline\hline
C & 3 \\
D & 4 \\ \hline
\end{tabular}
\end{table}

\begin{table}[tbp]
\centering
\contcaption{(concluded)}
\begin{tabular}{|c|c|} \hline
\textbf{Col 1} & \textbf{Col 2} \\ \hline\hline
E & 5 \\
F & 6 \\ \hline
\end{tabular}
\end{table}

\sclause{Title command}


    The command to produce the title of the standard\ixcom{title}
is 
\verb+\title{+\meta{intro}\verb+}{+\meta{main}\verb+}{+\meta{compl}\verb+}+. 
The command takes three parameters
according to the three elements of the title as specified in the ISO directives.
\begin{enumerate}
\item \meta{intro} The introductory element of the title. This may be empty.
\item \meta{main} The main element of the title.
\item \meta{compl} The complementary element of the title. This may be empty.
\end{enumerate}

\begin{example}
If this were actually meant to be an ISO standard, then the \latex{} source 
for the title of this document might be:
\begin{verbatim}
\title{Industrial automation systems and integration}%
{Product data representation and exchange}%
{Part 3456 : Documentation methods: The LaTeX style
file reference manual}
\end{verbatim}
\end{example} % end example

\begin{anote}There must be a space on either side of the colon separating the
part number and the final part of the title.
\end{anote}% end of note

    Three other commands are used for setting the running header throughout the
document. These shall be placed in the preamble.\index{preamble}

    The command \verb+\standard{+\meta{number of standard}\verb+}+ is used to identify
the standard.\ixcom{standard}

    The command \verb+\yearofedition{+\meta{year}\verb+}+ is used to identify the
\ixcom{yearofedition} year of the edition.

    The command 
\verb+\languageofedition{+\meta{code}\verb+}+\ixcom{languageofedition}
 is used to identify the language of the edition.

\begin{example}
This document is identified by:
\begin{verbatim}
\standard{LaTeX for standards}
\yearofedition{1997}
\languageofedition{(E)}
\end{verbatim}
\end{example} % end of example

    As a convenience the vacuous command \verb+\extrahead+\ixcom{extrahead}
is supplied which will add it's contents, if any, to the header. It may
be used, for example, to add a document number to the header by
\verb+\renewcommand{\extrahead}{Doc number}+. If \verb+\extrahead+ is
modified it must be done in the preamble.

    The \verb|\title|\ixcom{title} command sets the page numbering 
style to be arabic,
starts a new page, numbered 1, and puts the title
at the start of the page. It also puts the appropriate header at the top
of the page, dependending on the particular combination of selected
options.

\begin{anote}
Remember that you have to use the \verb|copyright|\ixopt{copyright}
option to enable printing of copyright symbols and notices.
\end{anote}

\ssclause{The cover}

    For publication, ISO want to be supplied with a document starting
on page iii with the ToC. They supply the cover (page i) and page ii.

It can often be useful to have a cover page for documents while they
are in the process of being prepared for submittal to ISO.
    The \verb+cover+\ixenv{cover} environment is provided for that
purpose. The contents of the environment must not exceed one page and the
contents must have a \verb+\clearpage+\ixcom{clearpage} immediately before
the end. The \verb+cover+ environment also prints page ii, which has a 
copyright notice on it if the document is copyrighted.

    For example, this document starts with:
\begin{verbatim}
\begin{cover}
\vspace*{4in}
\begin{center}
\Huge\bfseries LaTeX for standards
\end{center}
\begin{center}
\bfseries 2001/07/06
\end{center}
\begin{center}
Peter Wilson \\
\texttt{peter.r.wilson@boeing.com}
\end{center}
\clearpage
\end{cover}
\end{verbatim}

\sclause{Cross referencing commands}

    The usual \latex{} \verb+\label+\ixcom{label} and
\verb+\ref+\ixcom{ref} commands are supported. The class also
provides some formatted referencing commands in addition to \verb+\ref+.

    The following commands are useful for referring to figures, clauses etc.
Each takes a parameter that is the parameter of a \latex{} \verb|\label{}|
command.

\begin{itemize}
\item The command \verb|`\aref{anx:lord}'|\ixcom{aref} prints `\aref{anx:lord}' 
      while \verb|`\ref{anx:lord}'|\ixcom{ref}
      prints `\ref{anx:lord}'
\item The command \verb|`\bref{lamport}'|\ixcom{bref} prints `\bref{lamport}'
      while \verb|`\ref{lamport}'|\ixcom{ref}
      prints `\ref{lamport}'
\item The command \verb|`\cref{sec:nrefs}'|\ixcom{cref} prints `\cref{sec:nrefs}'
      while \verb|`\ref{sec:nrefs}'|\ixcom{ref}
      prints `\ref{sec:nrefs}'
\item The command \verb|`\eref{sec:nrefs}'|\ixcom{eref} prints `\eref{sec:nrefs}'
      while \verb|`\ref{sec:nrefs}'|\ixcom{ref}
      prints `\ref{sec:nrefs}'
\item The command \verb|`\fref{sec:nrefs}'|\ixcom{fref} prints `\fref{sec:nrefs}'
      while \verb|`\ref{sec:nrefs}'|\ixcom{ref}
      prints `\ref{sec:nrefs}'
\item The command \verb|`\nref{sec:nrefs}'|\ixcom{nref} prints `\nref{sec:nrefs}'
      while \verb|`\ref{sec:nrefs}'|\ixcom{ref}
      prints `\ref{sec:nrefs}'
\item The command \verb|`\tref{sec:nrefs}'|\ixcom{tref} prints `\tref{sec:nrefs}'
      while \verb|`\ref{sec:nrefs}'|\ixcom{ref}
      prints `\ref{sec:nrefs}'
\item The command \verb|`\pref{sec:nrefs}'|\ixcom{pref} prints
      `\pref{sec:nrefs}'.
\end{itemize}




\sclause{Heading commands}

\ssclause{Foreword}

    The \verb|foreword|\ixenv{foreword} environment specifies
that a table of contents, list of
figures and list of tables be produced, and starts a new unnumbered
clause called Foreword.
Formatting is one column style only and roman page numbering is set. The 
table of contents starts on page iii.

    More precisely, the title of the clause is given by the value of
\verb|\forewordname|\ixcom{forewordname} (see \aref{anx:extraiso}).

\begin{note}
The default style only lists level 1 clauses in the table of contents. If you
need a more detailed listing, then put the command 
\verb|\setcounter{tocdepth}{<level #>}|\ixcom{setcounter}\ixtt{tocdepth}
in the preamble.\index{preamble}
For example \verb|\setcounter{tocdepth}{3}| will produce a contents
listing down to the level of \verb|\ssclause| (see \tref{tab:sec} for
the definitions of clause levels).
\end{note} %end note

    The command \verb+\fwdbp+\ixcom{fwdbp} prints the 
ISO specified boilerplate for
the initial paragraphs of a foreword.

    The boilerplate for a Technical Specification (ISO/TS) or a 
Publicly Available Specification (ISO/PAS) differs from that
for a standard. The command \verb|\tspasfwdbp|\ixcom{tspasfwdbp}
prints some of this boilerplate.
\begin{example}
In this document, \verb|\tspasfwdbp| prints:

\tspasfwdbp
\end{example}
\begin{note}
    The required paragraph immediately following this boilerplate is:

ISO/PAS [or ISO/TS] \ldots was prepared by Technical Committee ISO/TC \ldots
\end{note}

%%    The boilerplate for a Technical Report (TR) also differs from that for
%%a standard.
%%The command \verb|\trfwdbpi|\ixcom{trfwdbpi}
%%prints some of this boilerplate. 
%%\begin{example}
%%In this document, \verb|\trfwdbpi| prints:
%%
%%\trfwdbpi
%%\end{example}

    The \verb|\fwdnopatents|\ixcom{fwdnopatents} 
prints out the Foreword
boilerplate paragraph concerning potential patent rights.
\begin{example}
In this document, \verb|\fwdnopatents| prints:
\fwdnopatents 
\end{example}



\ssclause{Introduction}

    The \verb|introduction|\ixenv{introduction} environment 
starts a new unnumbered clause 
called `Introduction' with one column formatting.

    More precisely, the title of the clause is given by the value of
\verb|\introductionname|\ixcom{introductionname} (see \aref{anx:extraiso}).

    The \verb|\intropatents|\ixcom{intropatents} prints some of the 
patent boilerplate that may be required in the Introduction. 
\begin{anexample}
In this document, \verb|\intropatents| prints:
\fwdnopatents 
\end{anexample}


\ssclause{Scope clause}

    The \verb|\scopeclause|\ixcom{scopeclause} 
command starts a new numbered clause called `Scope', which is given
the label \verb|;i1|\index{;i1}.

    More precisely, the title of the clause is given by the value of
\verb|\scopename|\ixcom{scopename} (see \aref{anx:extraiso}).

\ssclause{Normative references clause}

    The \verb|\normrefsclause|\ixcom{normrefsclause}
command starts a new numbered clause called `Normative references',
which is given the label \verb|;i2|\index{;i2}.

    More precisely, the title of the clause is given by the value of
\verb|\normrefsname|\ixcom{normrefsname} (see \aref{anx:extraiso}).

    The command 
\verb+\normrefbp{+\meta{document identifier}\verb+}+\ixcom{normrefbp} 
prints the
ISO required text for the introduction to the listing of normative
references. The \meta{document identifier} parameter must be such
that it reads sensibly in a sentence of the form `\ldots constitute
provisions of this \meta{document identifier}.'.

\begin{example}Clause~\ref{sec:nrefs} in this document was started by the
commands:
\begin{verbatim}
\normrefsclause \label{sec:nrefs}

\normrefbp{manual}
\begin{nreferences}
...
\end{verbatim}
\end{example} % end example

\ssclause{Definitions, symbols and abbreviations}

    A variety of commands are provided that initiate new numbered
clauses for definitions, symbols and abbreviations. Depending on the
amount of material in these respective categories, one or more clauses
may be used. The commands and clause titles are listed in \tref{tab:dsa}.
The clause level headings are each given the label \verb|;i3|\index{;i3};
one and only one of these headings should appear in a standard document.
\ixcom{defclause} \ixcom{defname}
\ixcom{symclause} \ixcom{symname}
\ixcom{abbclause} \ixcom{abbname}
\ixcom{defsymclause} \ixcom{defsymname}
\ixcom{defabbclause} \ixcom{defabbname}
\ixcom{symabbclause} \ixcom{symabbname}
\ixcom{defsymabbclause} \ixcom{defsymabbname}
\ixcom{defsubclause}
\ixcom{symsubclause}
\ixcom{abbsubclause}
\ixcom{defsymsubclause}
\ixcom{defabbsubclause}
\ixcom{symabbsubclause}

\begin{table}
\centering
\caption{Definition, symbol and abbreviation clause commands}
\label{tab:dsa}
\begin{tabular}{|l|c|l|l|} \hline
\textbf{Command} & \textbf{Clause} & \textbf{Title} & \textbf{Default text} \\ \hline
\verb|\defclause| &       C  & \verb|\defname| & \defname{} \\
\verb|\symclause| &       C  & \verb|\symname| & \symname{}  \\
\verb|\abbclause| &       C  & \verb|\abbname| & \abbname{}  \\
\verb|\defsymclause| &    C  & \verb|\defsymname| & \defsymname{} \\
\verb|\defabbclause| &    C  & \verb|\defabbname| & \defabbname{} \\
\verb|\symabbclause| &    C  & \verb|\symabbname| & \symabbname{} \\
\verb|\defsymabbclause| & C  & \verb|\defsymabbname| & \defsymabbname{} \\
\verb|\defsubclause| &    SC & \verb|\defname| & \defname{} \\
\verb|\symsubclause| &    SC & \verb|\symname| & \symname{}  \\
\verb|\abbsubclause| &    SC & \verb|\abbname| & \abbname{}  \\
\verb|\defsymsubclause| & SC & \verb|\defsymname| & \defsymname{} \\
\verb|\defabbsubclause| & SC & \verb|\defabbname| & \defabbname{} \\
\verb|\symabbsubclause| & SC & \verb|\symabbname| & \symabbname{} \\
 \hline
\multicolumn{4}{|l|}{%
NOTE - In the table, C = clause, SC = subclause. } \\ \hline
\end{tabular}
\end{table}

\sssclause{Heading labels}

    Some of the defined clauses have associated \verb|\label|s. These
heading commands and their \verb|\label| values are listed in 
\tref{tab:clabels}.

\begin{table}
\centering
\caption{Defined clause headings with labels}
\label{tab:clabels}
\begin{tabular}{|l|c|} \hline
\textbf{Command}       &  \textbf{Label} \\ \hline
\verb|\scopeclause|    &  \texttt{;i1} \\
\verb|\normrefsclause| &  \texttt{;i2} \\
\verb|\defclause| &       \texttt{;i3} \\
\verb|\symclause| &       \texttt{;i3} \\
\verb|\abbclause| &       \texttt{;i3} \\
\verb|\defsymclause| &    \texttt{;i3} \\
\verb|\defabbclause| &    \texttt{;i3} \\
\verb|\symabbclause| &    \texttt{;i3} \\
\verb|\defsymabbclause| & \texttt{;i3} \\
 \hline
\end{tabular}
\end{table}



\ssclause{Bibliography}

    The command \verb|\bibannex|\ixcom{bibannex}
 starts an informative section of the document entitled `Bibliography'. 
Or, more precisely, 
by the value of the \verb|\bibname|\ixcom{bibname}
command.

\sclause{Urls, etc}

    The command \verb|\url{|\meta{text}\verb|}|\ixcom{url} can be used for
typesetting \meta{text} as an email address.

    The command \verb|\isourl{|\meta{text}\verb|}|\ixcom{isourl} can be 
used for
typesetting \meta{text} as an URL address.

\begin{example}
The following code
\begin{verbatim}
The Email address is \url{joe@fred.mik} or the homepage is at
\isourl{http://fred.mik/home/}.
\end{verbatim}
will be typeset as: \\
The Email address is \url{joe@fred.mik} or the homepage is at
\isourl{http://fred.mik/home/}.
\end{example}


\sclause{Version control commands}


    A set of commands are provided to assist when revising a document.
For these commands to flag the changes in the printed document the command
\verb|\changemarkstrue|\ixcom{changemarkstrue}
must be put in the preamble.\index{preamble}

    In the commands described below, the \meta{number} parameter can be used 
to correlate changes in a document
to some external (numbered) requirement for the change.

\ssclause{Editorial}

    The command \verb+\editorial{+\meta{number}\verb+}+\ixcom{editorial}
flags an editorial change to the
document text with ED$^{number}$.

\begin{anexample}

   Just to start things off, this is an original sentence, which should
take up about a line.
This example sentence contains an editorial \editorial{37}
change. The command \verb|\editorial{37}| was included in the previous
sentence.
This sentence, though, has no changes marked in it and may be
assumed to be unaltered from a prior version.
\end{anexample} % end example

\ssclause{Added}

    The command 
\verb+\added{+\meta{text}\verb+}{+\meta{number}\verb+}+\ixcom{added}
flags the
position of the additional \meta{text} and highlights it.

\begin{anexample}

   Just to start things off, this is an original sentence, which should
take up about a line.
This example sentence contains \added{some added text}{27} in
the middle of it. The command \verb|\added{some added text}{27}| was
included in the previous sentence.
This sentence, though, has no changes marked in it and may be
assumed to be unaltered from a prior version.
\end{anexample} % end example

\ssclause{Deleted}

    The command \verb+\deleted{+\meta{number}\verb+}+\ixcom{deleted}
flags the position of deleted text.

\begin{anexample}

   Just to start things off, this is an original sentence, which should
take up about a line.
Some text was deleted \deleted{34} from the middle of this sentence.
The command \verb|\deleted{34}| was included in the previous sentence.
This sentence, though, has no changes marked in it and may be
assumed to be unaltered from a prior version.
\end{anexample} % end example

\ssclause{Moved}

    The command 
\verb+\moved{+\meta{text}\verb+}{+\meta{number}\verb+}+\ixcom{moved}
flags the position of
moved \meta{text} and highlights it.

\begin{anexample}

   Just to start things off, this is an original sentence, which should
take up about a line.
This sentence contains some \moved{moved}{81} text in the middle of it.
The command \verb|\moved{moved}{81}| was included in the previous
sentence.
This sentence, though, has no changes marked in it and may be
assumed to be unaltered from a prior version.
\end{anexample} % end example

\sclause{PDF}

    The class provides a command, \verb|\ifpdf|\ixcom{ifpdf}, 
to test whether or not the document is being processed by \latex{} 
or by pdf\latex. \latex{} processing produces a \file{.dvi} file which
needs further processing, usually by \file{dvips}, to generate PostScript
for printing. pdf\latex, however, directly generates a \file{.pdf} file
which can then be printed.

\begin{anexample}
A document for processing by either \latex{} or pdf\latex{} could be 
started like:
\begin{verbatim}
\documentclass{iso}
\usepackage{times}
\ifpdf
  \pdfoutput=1
  \usepackage[pdftex]{graphicx}
\else
  \usepackage{graphicx}
\fi
....
\end{verbatim}

The \file{times} package is used in this example as PDF printers tend
to handle
PostScript fonts better than other kinds of fonts.
\end{anexample}


    The class also supports the use of the 
\file{hyperref}\ixpack{hyperref} package in conjunction with pdf\latex.
Typically bookmark processing would be specifed by:
\begin{verbatim}
\ifpdf
  \pdfoutput=1
  \usepackage[plainpages=false,
              pdfpagelabels,
              bookmarksnumbered,
              hyperindex=true
             ]{hyperref}
  ...
\else
  ...
\fi
....
\end{verbatim}
If the \verb|hyperindex=true| option to the \file{hyperref} package is used
the the index has to be processed via the
\textsc{MakeIndex}\index{makeIndex@{\sc MakeIndex}} program.

%%%%%%%%%%%
%%%\end{document}
%%%%%%%%%%%

\clause{The \file{isorot} package facility} 

   The \file{isorot}\ixpack{isorot} facility enables the rotation 
of document elements on
a page. It uses the \latex{} \verb|\special|\ixcom{special} 
command to perform its
effects, and thus can only be used
with a limited number of dvi to postscript programs. The facilities
available are summarized in \tref{tab4}.

    \file{isorot} is a modification of the \file{rotation.sty} file
created by Rahtz and Barroca~\bref{rahtz}. Further examples of the usage
of their style are given in Goosens \emph{et al}~\bref{goosens}.

\begin{note}Several examples of the effects of the commands described herein
are shown. In many cases the results are not pretty. This should act as
a warning that using rotational elements requires more care than
most other document elements.\end{note} %end note

\begin{sidewaystable}
\ixcom{rotdriver} \ixcom{clockwise} \ixcom{counterclockwise}
\ixcom{figuresright} \ixcom{figuresleft} \ixcom{rotcaption}
\ixcom{controtcaption}
\ixenv{sideways} \ixenv{turn} \ixenv{rotate} \ixenv{sidewaystable}
\ixenv{sidewaysfigure} \ixenv{landscape}
\centering
\caption{The rotation facilities} \label{tab4}
\begin{tabular}{|l|l|} \hline
\textbf{Facility} & \textbf{Effect} \\ \hline
\multicolumn{2}{|c|}{\textbf{Commands}} \\ \hline
\verb|\rotdriver{<driver>}| & 
declare the name of the dvi to Postscript translator (default {\tt dvips}) \\
\verb|\clockwise| & 
sets rotation direction clockwise for positive angles (the default) \\
\verb|\counterclockwise| &
sets rotation direction counterclockwise for positive angles \\
\verb|\figuresright| &
sets rotation direction for sideways floats counterclockwise (the default) \\
\verb|\figuresleft| &
sets rotation direction for sideways floats clockwise \\
\verb|\rotcaption| &
like the \verb|caption| command, but rotates the caption through 90 degrees \\
\verb|\controtcaption| &
like the \verb|contcaption| command, but rotates the caption through 90 degrees \\ \hline
\multicolumn{2}{|c|}{\textbf{Environments}} \\ \hline
\verb|sideways| &
rotates the contents through 90 degrees counterclockwise \\
\verb|turn| &
rotates the contents through the given angle \\
\verb|rotate| &
rotates the contents through the given angle, but no space allowed for the result\\
\verb|sidewaystable| &
like the \verb|table| environment, but rotated 90 degrees \\
\verb|sidewaystable*| &
twocolumn version of \verb|sidewaystable| \\
\verb|sidewaysfigure| &
like the \verb|figure| environment, but rotated 90 degrees \\
\verb|sidewaysfigure*| &
twocolumn version of \verb|sidewaysfigure| \\
\verb|landscape| &
prints all enclosed pages in landscape mode \\ \hline
\end{tabular}
\end{sidewaystable}

\sclause{Options}

    The \file{isorot} facility has one option, 
namely \verb|debugshow|\ixopt{debugshow}. Calling this option produces
messages on the screen and in the \file{log} file regarding the actions 
being taken.

\begin{anote} This option is principally of interest to the maintainer
of the facility. \end{anote}

\sclause{DVI drivers}

    The \file{isorot} facility supports only a limited number of
dvi to postscript translators. The default translator is \emph{dvips}.
 The following command must be put in
the preamble of the document if \emph{dvips} is not being used: 
\verb|\rotdriver{<drivername>}|,\ixcom{rotdriver} where
\verb|<drivername>| is one of the following:\footnote{I have been able to 
try the {\tt dvips} driver
but not the others. If anyone has experience with the other drivers, or has
extended the range of drivers, I would like to be given the results.} %end footnote

\begin{enumerate}
\item \verb|dvipdf| for the \emph{dvipdf} 
      translator;\ixtt{dvipdf}
\item \verb|dvips| for Tom Rockicki's \emph{dvips} 
      translator;\ixtt{dvips}
\item \verb|dvipsone| for Y\&Y's \emph{dvipsone} 
      translator;\ixtt{dvipsone}
\item \verb|dvitops| for James Clark's \emph{dvitops} 
      translator;\ixtt{dvitops}
\item \verb|dviwindo| for Y\&Y's \emph{dviwindo} 
      translator;\ixtt{dviwindo}
\item \verb|pctex32| for Personal TeX's PC TeX for 32 bit Windows 
      (\emph{pctex32})
      translator;\ixtt{pctex32}
\item \verb|pctexps| for Personal TeX's PC PTI Laser/PS (\emph{pctexps})
      translator;\ixtt{pctexps}
\item \verb|pubps| for the Arbortext's \emph{pubps} 
      translator.\ixtt{pubps}
\item \verb|textures| for Blue Sky's \emph{Textures} 
      translator;\ixtt{textures}

\end{enumerate}


\sclause{Rotational directions}

    \file{isorot} enables the textual and other elements of a document
to be rotated from their normal horizontal layout. In some cases elements
can be rotated through arbitrary angles, whereas in others only 90 degree
rotation is possible. 

    By default, a rotation through a positive number of
degrees corresponds to a clockwise rotation. The command 
\verb|\counterclockwise|\ixcom{counterclockwise}
sets the following rotations to be counterclockwise for positive angles.
The command \verb|\clockwise|\ixcom{clockwise}
sets the following rotations to be clockwise for positive angles.
These commands can be used to toggle the rotational behavior.
    

    Rotated floating environments are normally rotated so that they are
printed with a counterclockwise rotation (i.e. the original bottom of the float
is placed at the right hand side of the paper), 
which is what is normally required.
This behavior can be altered by the command
\verb|\figuresleft|,\ixcom{figuresleft}
which will give the reverse effect. The command
\verb|\figuresright|\ixcom{figuresright}
will set the behavior to the default.
These commands can be used to toggle the rotational behavior of
floats.


\sclause{Rotation of text}

    The \verb|sideways|\ixenv{sideways}
environment rotates the contents of the environment
by 90 degrees counterclockwise, and leaves space for the result.

    The \verb|\begin{turn}{|\meta{angle}\verb|}|\ixenv{turn}
environment rotates the contents by the given number
of degrees in the direction specified by the most recent of the
\verb|\clockwise|\ixcom{clockwise} or
\verb|\counterclockwise|\ixcom{counterclockwise}
commands, leaving space for the result.

    The \verb|\begin{rotate}{|\meta{angle}\verb|}|\ixenv{rotate}
environment rotates the contents by the given number
of degrees in the direction specified by the most recent of the
\verb|\clockwise|\ixcom{clockwise} or
\verb|\counterclockwise|\ixcom{counterclockwise}
commands, but no arrangements are made for leaving space for the result.


\begin{example}Some simple rotations: \label{ex:1}

This code
\begin{verbatim}
Default rotation direction: \\
A 
\begin{sideways}%
B C
\end{sideways}
D E F G H I J K L M
\begin{turn}{-90}%
Minus 90 turn
\end{turn}
N O P
\begin{rotate}{90}%
Plus 90 rotate
\end{rotate}
Q \\
and continue on with another line after rotations.
\end{verbatim}
produces the following (note how space is allowed for the \verb|turn|ed 
text, whereas the \verb|rotate|d text runs into the text below).

Default rotation direction: \\
A 
\begin{sideways}%
B C
\end{sideways}
D E F G H I J K L M
\begin{turn}{-90}%
Minus 90 turn
\end{turn}
N O P
\begin{rotate}{90}%
Plus 90 rotate
\end{rotate}
Q \\
and continue on with another line after rotations.
\end{example} % end example
\begin{example}This example shows the effect of using the 
\verb|\counterclockwise|\ixcom{counterclockwise}
command.

This code
\begin{verbatim}
Flip rotation direction: \\
\counterclockwise
A 
\begin{sideways}%
B C
\end{sideways}
D E F G H I J K L M
\begin{turn}{-90}%
Minus 90 turn
\end{turn}
N O P
\begin{rotate}{90}%
Plus 90 rotate
\end{rotate}
Q \\
Set rotation direction back to default value.
\clockwise
\end{verbatim}

produces the following, which should be compared with example~\ref{ex:1}.

Flip rotation direction: \\
\counterclockwise
A 
\begin{sideways}%
B C
\end{sideways}
D E F G H I J K L M
\begin{turn}{-90}%
Minus 90 turn
\end{turn}
N O P
\begin{rotate}{90}%
Plus 90 rotate
\end{rotate}
Q \\
Set rotation direction back to default value.
\clockwise
\end{example} % end example


    Although the examples so far have only shown the rotation of text, boxes
can also be rotated.

\begin{example}Rotating a box.

This code
\begin{verbatim}
\newsavebox{\foo}
\newlength{\fool}
\settowidth{\fool}{Hurrah for ISO.}
\savebox{\foo}{\parbox{\fool}{Hurrah for ISO. Hurrah for ISO.
                              Hurrah for ISO. Hurrah for ISO.}}
Start
\usebox{\foo}
\&
\begin{turn}{-45}\usebox{\foo}\end{turn}
\&
\begin{turn}{45}\usebox{\foo}\end{turn}
End
\end{verbatim}
produces:

\newsavebox{\foo}
\newlength{\fool}
\settowidth{\fool}{Hurrah for ISO.}
\savebox{\foo}{\parbox{\fool}{Hurrah for ISO. Hurrah for ISO.
                              Hurrah for ISO. Hurrah for ISO.}}
Start
\usebox{\foo}
\&
\begin{turn}{-45}\usebox{\foo}\end{turn}
\&
\begin{turn}{45}\usebox{\foo}\end{turn}
End

\end{example} %end example

    Elements can be rotated through arbitrary angles, and also rotated
elements can be nested inside other rotated elements.


\begin{example}Repeated rotation:

    The following example code shows that text can be rotated through any angle.
The result is shown in \fref{fig:wheel}.
\begin{verbatim}
\newcount\prwc
\newsavebox{\prwtext}
\newdimen\prwspace
\def\wheel#1#2{%
  \savebox{\prwtext}{#1\begin{sideways}#2\end{sideways}}%
  \prwspace\wd\prwtext%
  \advance\prwspace by 1cm%
  \centerline{%
  \rule{0pt}{\prwspace}%
  \rule[-\prwspace]{0pt}{\prwspace}%
  \prwc=-180\loop\ifnum\prwc<180
  \rlap{\begin{rotate}{\the\prwc}%
  \rule{1cm}{0pt}\usebox{\prwtext}\end{rotate}}%
  \advance\prwc by 20\repeat}}
\begin{figure}
\wheel{Express yourself ---}{Hooray for STEP!}
\caption{Example rotation through multiple angles}
\label{fig:wheel}
\end{figure}
\end{verbatim}

\newcount\prwc
\newsavebox{\prwtext}
\newdimen\prwspace
\def\wheel#1#2{%
  \savebox{\prwtext}{#1\begin{sideways}#2\end{sideways}}%
  \prwspace\wd\prwtext%
  \advance\prwspace by 1cm%
  \centerline{%
  \rule{0pt}{\prwspace}%
  \rule[-\prwspace]{0pt}{\prwspace}%
  \prwc=-180\loop\ifnum\prwc<180
  \rlap{\begin{rotate}{\the\prwc}%
  \rule{1cm}{0pt}\usebox{\prwtext}\end{rotate}}%
  \advance\prwc by 20\repeat}}
\begin{figure}
\vspace*{1cm}
\wheel{Express yourself ---}{Hooray for STEP!}
\vspace*{1cm}
\caption{Example rotation through multiple angles}
\label{fig:wheel}
\end{figure}

    Figures~\ref{fig:angles1} and~\ref{fig:angles2} also show rotations through a
range of angles, both positive and negative.
\end{example} %end example

\begin{example}Nested rotations. \label{ex:sidetabular}

This code
\begin{verbatim}
    Here is some text before a \verb|sideways| environment. 
And some more, and more and more garble gobble cluck
click clack clock cluck and so on and on and on.
\begin{center}
\begin{sideways}
\rule{1in}{0pt}
\begin{tabular}{|lr|}
\begin{rotate}{-45}\emph{Word}\end{rotate} & \begin{rotate}{-90}%
Occurrences\end{rotate}
\\
\hline
hello & 33 \\
goodbye & 34 \\
\hline
\end{tabular}
\end{sideways}
\end{center}
    Here is some text after a \verb|sideways| environment.
 And some more, and more and more garble gobble cluck
click clack clock cluck and so on and on and on.
\end{verbatim}
produces:

    Here is some text before a \verb|sideways| environment. 
And some more, and more and more garble gobble cluck
click clack clock cluck and so on and on and on.
\begin{center}
\begin{sideways}
%\rule{1in}{0pt}
\begin{tabular}{|lr|}
\begin{rotate}{-45}\emph{Word}\end{rotate} & \begin{rotate}{-90}%
Occurrences\end{rotate} \\ \hline
hello & 33 \\
goodbye & 34 \\ \hline
\end{tabular}
\end{sideways}
\end{center}
    Here is some text after a \verb|sideways| environment.
 And some more, and more and more garble gobble cluck
click clack clock cluck and so on and on and on.
\end{example} %end example



\sclause{Rotations of tables and figures}

    The previous examples have demonstrated the rotation of textual elements.
For instance, example~\ref{ex:sidetabular} shows that tabular material can be rotated using
the \verb|sideways|\ixenv{sideways}
environment. (Actually, any of the previously
mentioned environments may be used.)
Two further environments are provided which rotate a \latex{} float through
90 degrees. These are:
\begin{itemize}
\item \verb|sidewaystable|\ixenv{sidewaystable}, which
  corresponds to the standard \latex{} \verb|table|\ixenv{table}
  environment; and
\item \verb|sidewaysfigure|\ixenv{sidewaysfigure}, which
  corresponds to the standard \latex{} \verb|figure|\ixenv{figure}
  environment.
\end{itemize}
There are also starred versions of these, namely 
\verb|sidewaystable*|\ixenvs{sidewaystable} and
\verb|sidewaysfigure*|\ixenvs{sidewaysfigure}, for use in twocolumn mode.
However, the correspondence with the standard environments is not strictly
complete as a sideways float is alway placed on a page by itself.

    The direction of rotation may be controlled by the 
\verb|\figuresright|\ixcom{figuresright} and
\verb|\figuresleft|\ixcom{figuresleft} commands.

\begin{example}Table~\ref{tab4} is produced by the code below: \label{ex:4}

\begin{verbatim}
\begin{sidewaystable}
\centering
\caption{The rotation facilities} \label{tab4}
\begin{tabular}{|l|l|} \hline
\textbf{Facility} & \textbf{Effect} \\ \hline
\multicolumn{2}{|c|}{\textbf{Commands}} \\ \hline
\verb|\rotdriver{<driver>}| & 
declare the name of the dvi to Postscript translator (default {\tt dvips}) \\
......
\verb|sidewaysfigure| &
like the \verb|figure| environment, but rotated 90 degrees \\ \hline
\end{tabular}
\end{sidewaystable}
\end{verbatim}
\end{example} % end example



\sclause{Rotation of float captions and bodies}

    Sometimes it may be useful to rotate a caption independently of the
rotation of a figure or table. The command 
\verb|\rotcaption|\ixcom{rotcaption} is analogous
to the normal \verb|\caption|\ixcom{caption} command, 
and inserts the caption rotated
by 90~degrees. There is also the companion command 
\verb|\controtcaption|\ixcom{controtcaption}, analagous to the
\verb|\contcaption|\ixcom{contcaption} command,
for continuation captions.

%\newsavebox{\picbox}

\begin{figure}
\centering
\caption{Example figure with a standard caption.} \label{fig:nocrot}
\setlength{\unitlength}{0.2in}
\footnotesize
\begin{picture}(17,2)
\thicklines
\put(0,0){\begin{picture}(4,1)
  \put(1.5,0.5){\oval(3,1)}
  \put(1.5,0.5){\makebox(0,0){2,5 (1)}}
  \put(3,0.5){\line(1,0){1.0}}
  \put(4.25,0.5){\circle{0.5}}
  \end{picture}}

\put(4.5,0){\begin{picture}(8,1)
  \put(0,0){\dashbox{0.25}(4,1){date}}
  \put(4,0.5){\line(1,0){3.5}}
  \put(7.75,0.5){\circle{0.5}}
  \put(6,1){\makebox(0,0){A[1:3]}}
  \end{picture}}

\put(12.5,0){\begin{picture}(4,1)
  \put(0,0){\framebox(4,1){INTEGER}}
  \put(3.75,0){\line(0,1){1}}
  \end{picture}}
\end{picture}
\normalsize
\setlength{\unitlength}{1pt}
\end{figure}


\begin{example}Float with a regular caption.

Figure~\ref{fig:nocrot} is produced by the code below:
\begin{verbatim}
\begin{figure}
\centering
\caption{Example figure with a standard caption.} \label{fig:nocrot}
\setlength{\unitlength}{0.2in}
\footnotesize
\begin{picture}(17,2)
\thicklines
\put(0,0){\begin{picture}(4,1)
  \put(1.5,0.5){\oval(3,1)}
  \put(1.5,0.5){\makebox(0,0){2,5 (1)}}
  \put(3,0.5){\line(1,0){1.0}}
  \put(4.25,0.5){\circle{0.5}}
  \end{picture}}
\put(4.5,0){\begin{picture}(8,1)
  \put(0,0){\dashbox{0.25}(4,1){date}}
  \put(4,0.5){\line(1,0){3.5}}
  \put(7.75,0.5){\circle{0.5}}
  \put(6,1){\makebox(0,0){A[1:3]}}
  \end{picture}}
\put(12.5,0){\begin{picture}(4,1)
  \put(0,0){\framebox(4,1){INTEGER}}
  \put(3.75,0){\line(0,1){1}}
  \end{picture}}
\end{picture}
\normalsize
\setlength{\unitlength}{1pt}
\end{figure}
\end{verbatim}
\end{example} % end example

\begin{example}Float with a rotated caption.

Figure~\ref{fig:crot} is produced by the code below:
\begin{verbatim}
\begin{figure}
\centering
\rotcaption{Figure~\protect\ref{fig:nocrot} with a rotated caption.}
 \label{fig:crot}
\setlength{\unitlength}{0.2in}
\footnotesize
\begin{picture}(17,2)
...
\end{picture}
\normalsize
\setlength{\unitlength}{1pt}
\end{figure}
\end{verbatim}
\end{example} % end example


\begin{figure}
\centering
\rotcaption{Figure~\protect\ref{fig:nocrot} with a rotated caption.}
 \label{fig:crot}
\setlength{\unitlength}{0.2in}
\footnotesize
\begin{picture}(17,2)
\thicklines
\put(0,0){\begin{picture}(4,1)
  \put(1.5,0.5){\oval(3,1)}
  \put(1.5,0.5){\makebox(0,0){2,5 (1)}}
  \put(3,0.5){\line(1,0){1.0}}
  \put(4.25,0.5){\circle{0.5}}
  \end{picture}}
\put(4.5,0){\begin{picture}(8,1)
  \put(0,0){\dashbox{0.25}(4,1){date}}
  \put(4,0.5){\line(1,0){3.5}}
  \put(7.75,0.5){\circle{0.5}}
  \put(6,1){\makebox(0,0){A[1:3]}}
  \end{picture}}
\put(12.5,0){\begin{picture}(4,1)
  \put(0,0){\framebox(4,1){INTEGER}}
  \put(3.75,0){\line(0,1){1}}
  \end{picture}}
\end{picture}
\normalsize
\setlength{\unitlength}{1pt}
\end{figure}

    As can be seen from \fref{fig:crot} the advisability of rotating a caption
depends on the size of the body of the float. It may be better in certain
cases to leave the caption in its regular position and rotate the body of
the float instead.

\def\prwrot#1{%
\settowidth{\fool}{ISOROT}
\savebox{\foo}{\parbox{\fool}{ISOROT ISOROT ISOROT ISOROT}}%
\framebox{---\begin{turn}{#1}\framebox{\usebox{\foo}}\end{turn}---}}%
\def\degrees{{\small$^{o}$}}

\begin{figure}
\centering
\begin{tabular}{|c|c|c|} \hline
\prwrot{0} &\prwrot{-40}&\prwrot{-80}\\
0\degrees & -40\degrees & -80\degrees \\ \hline
\prwrot{-120}&\prwrot{-160}&\prwrot{-200}\\
-120\degrees & -160\degrees & -200\degrees \\ \hline
\prwrot{-240}&\prwrot{-280}&\prwrot{-320}\\
-240\degrees & -280\degrees & -320\degrees \\ \hline
\end{tabular}
\caption{Rotation of paragraphs between 0 and -320 degrees} \label{fig:angles1}
\end{figure}


\begin{example}Regular caption and float.

Figure~\ref{fig:angles1} is a regular figure and caption. It is produced by
the following code:
\begin{verbatim}
\def\prwrot#1{%
\settowidth{\fool}{ISOROT}
\savebox{\foo}{\parbox{\fool}{ISOROT ISOROT ISOROT ISOROT}}%
\framebox{---\begin{turn}{#1}\framebox{\usebox{\foo}}\end{turn}---}}%
\def\degrees{{\small$^{o}$}}
\end{verbatim}

\begin{verbatim}
\begin{figure}
\centering
\begin{tabular}{|c|c|c|} \hline
\prwrot{0} &\prwrot{-40}&\prwrot{-80}\\
0\degrees & -40\degrees & -80\degrees \\ \hline
\prwrot{-120}&\prwrot{-160}&\prwrot{-200}\\
-120\degrees & -160\degrees & -200\degrees \\ \hline
\prwrot{-240}&\prwrot{-280}&\prwrot{-320}\\
-240\degrees & -280\degrees & -320\degrees \\ \hline
\end{tabular}
\caption{Rotation of paragraphs between 0 and -320 degrees} \label{fig:angles1}
\end{figure}
\end{verbatim}
\end{example} % end example


\begin{figure}
\centering
\begin{sideways}
\begin{tabular}{|c|c|c|} \hline
\prwrot{0} &\prwrot{40}&\prwrot{80}\\
0\degrees & 40\degrees & 80\degrees \\ \hline
\prwrot{120}&\prwrot{160}&\prwrot{200}\\
120\degrees & 160\degrees & 200\degrees \\ \hline
\prwrot{240}&\prwrot{280}&\prwrot{320}\\
240\degrees & 280\degrees & 320\degrees \\ \hline
\end{tabular}
\end{sideways}
\caption[Rotation of paragraphs between 0 and 320 degrees]%
        {Rotation of paragraphs between 0 and 320 degrees (with figure
         body turned sideways)}\label{fig:angles2}
\end{figure}

\begin{example}Regular caption and rotated float body.

Figure~\ref{fig:angles2} is a regular figure and caption where the figure
contents have been rotated. It was produced by the following code.
\begin{verbatim}
\begin{figure}
\centering
\begin{sideways}
\begin{tabular}{|c|c|c|} \hline
\prwrot{0} &\prwrot{40}&\prwrot{80}\\
0\degrees & 40\degrees & 80\degrees \\ \hline
\prwrot{120}&\prwrot{160}&\prwrot{200}\\
120\degrees & 160\degrees & 200\degrees \\ \hline
\prwrot{240}&\prwrot{280}&\prwrot{320}\\
240\degrees & 280\degrees & 320\degrees \\ \hline
\end{tabular}
\end{sideways}
\caption[Rotation of paragraphs between 0 and 320 degrees]%
        {Rotation of paragraphs between 0 and 320 degrees (with figure
         body turned sideways)}\label{fig:angles2}
\end{figure}
\end{verbatim}
\end{example} % end example

\begin{landscape}
\sclause{Landscaping}

    \latex{} normally prints in portrait mode. 
The \verb|landscape|\ixenv{landscape} environment
prints all the enclosed stuff in landscape mode, except for headers
and footers which are not rotated.

\begin{example} Landscaping

The source for this part of the document is:
\begin{verbatim}
\begin{landscape}
\sclause{Landscaping}

    \latex{} normally prints in portrait mode. The ...
...
... long, wide tables.
\end{landscape}
\end{verbatim}
\end{example}

    The environment starts by clearing the current page and then switches
to portrait mode. At the end of the environment the current page is cleared
and the next page is back to normal portrait mode. 

    All the other rotation commands and environments produce boxes and
\latex{} will not break a box across a page. The \verb|landscape| environemt
does not produce a box and so many pages can be printed in landscape mode
with \latex{} taking care of the page breaking for you. 

    Landscape mode is not particularly useful for normal text as the
lines are far too long for comfortable reading. Where it can be useful
is where you have a table that is too wide to fit on a portrait page, so
needs to be rotated, yet is also too long to fit on the page when it is
rotated. The \file{supertabular}\ixpack{supertabular},
the \file{longtable}\ixpack{longtable},
and the \file{xtab}\ixpack{xtab}
packages provide facilities for automatically breaking long tables across 
pages. Any of these can be used in conjunction with landscaping to both 
rotate and automatically page break long, wide tables.
\end{landscape}

\clause{The \file{xtab} package facility}

    The \file{xtab} package is fully documented in \bref{bib:xtab}.
This clause provides an overview of the package.

    The \file{xtab}\ixpack{xtab} package facility is an extension 
of the \file{supertabular}\ixpack{supertabular} package originally
developed by Johannes Braams and Theo Jurriens.
The extension provides for the specification of a header to
go on the last page of a long table.

    The principal commands available are given in \tref{tab:xtab}.
\ixenv{xtabular}
\ixenv{mpxtabular}
\ixcom{topcaption}
\ixcom{bottomcaption}
\ixcom{tablecaption}
\ixcom{tablefirsthead}
\ixcom{tablehead}
\ixcom{tablelasthead}
\ixcom{notablelasthead}
\ixcom{tabletail}
\ixcom{tablelasttail}

\topcaption{The principal xtab package commands} \label{tab:xtab}
\tablefirsthead{\hline \multicolumn{1}{|c|}{\textbf{Command}} &
                       \multicolumn{1}{c|}{\textbf{Effect}} \\ \hline }
\tablehead{\multicolumn{2}{c}%
           {{\captionsize\bfseries \tablename\ \thetable{} -- continued from previous page}} \\
           \hline \multicolumn{1}{|c|}{\textbf{Command}} &
                  \multicolumn{1}{c|}{\textbf{Effect}} \\ \hline }
\tablelasthead{\multicolumn{2}{c}%
           {{\captionsize\bfseries \tablename\ \thetable{} -- concluded from previous page}} \\ 
           \hline \multicolumn{1}{|c|}{\textbf{Command}} &
                  \multicolumn{1}{c|}{\textbf{Effect}} \\ \hline }
\tabletail{\hline \multicolumn{2}{|r|}{{Continued on next page}} \\ \hline}
\tablelasttail{\\ \hline \hline}

\begin{center}
\begin{xtabular}{|l|p{0.5\textwidth}|}
\verb|\begin{xtabular}{...}| & This is equivalent to the normal
                         \verb|\begin{tabular}{...}| environment.
                         You supply the specification of the columns
                         just as for the normal tabular environment.
                         All commands that can be used within a tabular
                         environment can also be used within
                         the xtabular environment.
 \\
 &
    Unlike the tabular environment which prevents page breaking
within the tabular, the xtabular allows page breaking, so that
tabulars can extend automatically across several pages. Xtabular
starts off with a tabular environment and checks the amount of
space left on the page as it adds each row to the tabular.
If the space left on the page is too short for another row, then
it ends the current tabular, performs a page break and starts
another tabular on the following page. This process is repeated
until all the rows have been output. 
 \\
 &
    There are special commands for captioning a xtabular as a
table, and also elements can be automatically inserted after each
(internal) \verb|\begin{tabular}| and immediately before each
\verb|\end{tabular}|.
 \\
 &
    Do not put a xtabular in a table environment, as the table
environment keeps its contents on a single page (presumably you
are using xtabular because its contents are longer than
one page).
 \\
\verb|\end{xtabular}| & End the xtabular environment. \\ \hline
\verb|\begin{mpxtabular}| & Like the xtabular environment
except that each `page' is put into a \verb|minipage| first. Thus
it is possible to have footnotes inside an \verb|mpxtabular|.
The footnote text is printed at the end of each page.
 \\
\verb|\end{mpxtabular}| & End the mpxtabular environment. \\ \hline
                          & \textbf{Note:} If any of the following commands
                            are used, then they should be placed
                            before the particular xtabular
                            environment that they apply to. \\
\verb|\topcaption{...}| & A command to provide a caption for the
                          table. The caption is placed at the top
                          of the table. \\
\verb|\bottomcaption{...}| & A command to provide a caption for the
                          table. The caption is placed at the bottom
                          of the table. \\
\verb|\tablecaption{...}| & A command to provide a caption for the
                          table. The caption is placed at the default
                          position, which is at the top
                          of the table.
 \\
 & 
                           \textbf{Notes:} You cannot use the \verb|caption|
                            command but you can put a label after
                            any of these captioning commands. If you
                            want captioning, the command must be specified
                            before the start of the xtabular environment.
  \\
  &
                            The \verb|\...caption{}| command(s) remain
                            in effect until changed by another
                            \verb|\...caption| command.
   \\
\verb|\tablefirsthead{...}| & Defines the contents of the first occurence
                         of the tabular head. The tabular head is some
                         special treatment of the first row in the table.
                         This command is optional.
                         If used, the header must be closed by the
                         end of line command for tabulars (e.g., \verb|\\|). \\
\verb|\tablehead{...}| & Defines the contents of the table head on
                         subsequent pages.
 \\
 &
                         For example, you might want to note that
                         this is a continuation of the table on
                         the previous page, as well as repeating
                         any column headings that were given
                         at the start of the xtabular by
                         \verb|\tablefirsthead|.
 \\
\verb|\tablelasthead{...}| & Defines the contents of the table head
                            on the last page of the table.
 \\
 &
    This works by writing to the \file{.aux} file the number of
pages that the xtabular extends over. When the xtabular
comes to the last tabular (which it calculates from the 
information it reads from the \file{.aux} file) it replaces the
contents of \verb|\tablehead| by the contents of \verb|\tablelasthead|.
 \\
 &
It makes no attempt to measure the amount of space used by the last
table head, so if this is more than for \verb|tablehead| then the
tabular might be a litle too long.
  \\
  &
    In any case, when using this command, the document has to
be LaTeXed at least twice, just as it has to be to resolve
references and so on.
 \\
\verb|\notablelasthead|  & Switches off the last \verb|\tablelasthead|.
                           A \verb|tablelasthead| stays in effect until
                           overwritten by a new \verb|\tablelasthead|
                           or cancelled by this command.
 \\
\verb|\tabletail{...}| & The contents of this command are inserted before
                         the \verb|\end{tabular}| on each page except
                         for the last page of the table.
 \\
 &
                         For example, you might want to note that the
                         table is continued on the next page. 
 \\
\verb|\tablelasttail{...}| & The contents of this command are inserted before
                         the final \verb|\end{tabular}| of the table.
 \\
 &
                         For example, you might want to note that
                         this is where the table ends.
\end{xtabular}
\end{center}

    As well as the \verb|xtabular|\ixenv{xtabular} and
\verb|mpxtabular|\ixenv{mpxtabular} environments there are
the corresponding starred versions 
(i.e., \verb|xtabular*|\ixenvs{xtabular} and 
        \verb|mpxtabular*|\ixenvs{mpxtabular}) for use in
two column\ixopt{twocolumn} mode where the table is meant to span 
both columns.

\begin{example} \label{ex:xtab} Table~\ref{tab:xtab} was produced by 
the following code:
\begin{verbatim}
\topcaption{The principal xtab package commands} \label{tab:xtab}
\tablefirsthead{\hline \multicolumn{1}{|c|}{\textbf{Command}} &
                       \multicolumn{1}{c|}{\textbf{Effect}} \\ \hline }
\tablehead{\multicolumn{2}{c}%
           {{\captionsize\bfseries \tablename\ \thetable{} -- 
             continued from previous page}} \\
  \hline   \multicolumn{1}{|c|}{\textbf{Command}} &
           \multicolumn{1}{c|}{\textbf{Effect}} \\ \hline }
\tablelasthead{\multicolumn{2}{c}%
           {{\captionsize\bfseries \tablename\ \thetable{} -- 
             concluded from previous page}} \\
  \hline   \multicolumn{1}{|c|}{\textbf{Command}} &
           \multicolumn{1}{c|}{\textbf{Effect}} \\ \hline }
\tabletail{\hline \multicolumn{2}{|r|}{{Continued on next page}} \\ \hline}
\tablelasttail{\\ \hline \hline}

\begin{center}
\begin{xtabular}{|l|p{0.5\textwidth}|}
\verb|\begin{xtabular}{...}| & This is equivalent to the normal
                         \verb|\begin{tabular}{...}| environment.
                         You supply the specification of the columns
                         just as for the normal tabular environment.
                         All commands that can be used within a tabular
                         environment can also be used within
                         the xtabular environment.
 \\
 &
    Unlike the tabular environment which prevents page breaking
within the tabular, the xtabular allows page breaking, so that
tabulars can extend automatically across several pages. Xtabular
... ... ...
\verb|\tablelasttail{...}| & The contents of this command are inserted before
                         the final \verb|\end{tabular}| of the table.
 \\
 &
    For example, you might want to note that  this is where 
the table ends.
\end{xtabular}
\end{center}
\end{verbatim}
\end{example}       

    The table is only broken between rows --- a row will not be split
across pages. This can lead to some bad page breaks, especially if 
there are rows with a large vertical height (like some in \tref{tab:xtab}).
It is best to keep rows not too tall.

The command
\verb|\shrinkheight{|\meta{length}\verb|}|\ixcom{shrinkheight} may be
used after the first \verb|\\| in the table to modify the allowed
height of the table on each page. A positive \meta{length} decreases
the allowed space per page and a negative \meta{length} increases
the allowed space.

\begin{example}
This example illustrates changing the natural height of the pages
in a \verb|xtabular| and its relatives.

\verb|\shrinkheight{2\baselineskip}| decreases the space per page by 
two lines.

\verb|\shrinkheight{-\baselineskip}| increases the space per page by
one line.
\end{example}

    You have to specify the font used for the 
\verb|\tablehead|\ixcom{tablehead} and
\verb|tablelasthead|\ixcom{tablelasthead} yourself. 
Within ISO documents, captions shall
be in bold font. The \file{iso}\ixclass{iso} class also provides 
a command for
setting the size of the font used in captions, namely 
\verb|\captionsize|\ixcom{captionsize}. 
Note that this is used in~\eref{ex:xtab}.
    The value of \verb|\captionsize|\ixcom{captionsize} is set by the 
class.

\sclause{Options}

    The \file{xtab} package has three options which control the amount of
information that is written to the \file{.log}\index{file!.log} file. The
options are:
\begin{enumerate}
\item The option \verb|errorshow|\ixopt{errorshow} (the default) does not
      write any extra information;
\item The option \verb|pageshow|\ixopt{pageshow} writes information
      about when and why \verb|xtabular| decides to produce a new page;
\item The option \verb|debugshow|\ixopt{debugshow}, which also includes
      \verb|pageshow|, additionally writes information about each line
      that is added to the table.
\end{enumerate}

    Under normal circumstances \file{xtab} is used without invoking any 
option. The \verb|pageshow| option may be useful when attempting to cure
a bad page break. The \verb|debugshow| option, as its name implies, is
principally of use to the \file{xtab} developer.

    Independently of the options, the command 
\verb|\sstraceon|\ixcom{sstraceon} may be used at any
point in the document to turn on printing of \verb|debugshow| data. This
can be turned off later by the \verb|\sstraceoff|\ixcom{sstraceoff}
command, which will stop all \verb|...show| printing.

\sclause{Caveats}

    The authors of the original \file{supertabular} package note that
it has the following weaknesses:
\begin{itemize}
\item Sometimes the body of the first page of a table gets moved to the 
following, leaving the caption behind;
\item Sometimes the last page of a table consists of just the head and foot
with no contents.
\end{itemize}
The weaknesses are caused by trying to guess where \tex{} will put a page 
break. The package has to guesstimate how long the next entry will be in 
the table and, if it is too long for the available space, it puts in its
own page break. If its guess is off too much in one direction, \tex{} will 
break the page unexpectadly; if its off in the other direction
\file{supertabular} will put in an unnecessary page break.

    The \file{xtab} package has reduced, but perhaps not entirely
eliminated, these weaknesses. Some hand tuning may still be required.


%%%%%%%%%
%%%%\end{document}
%%%%%%%%%



\clause{The \file{askinc} package facility} 

    The \file{askinc}\ixpack{askinc} package facility contains \latex{} 
code to enable the interactive
input of files. This functionality is a cross between the \latex{}
\verb|\include|\ixcom{include} and 
\verb|\includeonly|\ixcom{includeonly} commands, and the
\verb|\input|\ixcom{input} command.

In the body of the root source document, use the command 
\verb+\infile{+\meta{file}\verb+}+\ixcom{infile}
for each \meta{file}
that comprises the document. That is, the command \verb|\infile| is similar
to the \verb|\input|\ixcom{input} and
\verb|\include|\ixcom{include} commands.

    When \latex ing the root document, you will be asked to provide a 
comma-separated list of the \verb|\infile|s to be processed (similar to the
argument to the \verb|\includeonly|\ixcom{includeonly}
command). If you want all the files to be processed, just hit the
\verb|<RETURN>| key (or its equivalent).

    Like \verb|\include|d files, a file that is \verb|\infile|d into a 
document shall not itself contain another \verb|\infile|d file.

\begin{example}The following root file has three files that are \verb|\infile|d.
\begin{verbatim}
\documentclass[...]{...}
\usepackage{askincv}
% other preamble stuff
\begin{document}
% perhaps some stuff
\infile{file1}
\infile{file2}
\infile{file3}
% perhaps more stuff
\end{document}
\end{verbatim}
\end{example} % end example

\clause{The \file{hyphenat} package facility} \label{sec:uschyp}

    The \file{hyphenat} package is fully described in~\bref{bib:hyphenat}.
This clause provides an overview of the portions of the package that are
most relevant to typesetting ISO standards.

    In \latex{} if you need to use the underscore (\verb|_|)
character in normal text, for example when documenting identifiers
in a programming language code, you have to use the 
\verb|\_|\index{_ /@\verb?\_?} command,
as in \verb|a\_multiword\_identifier|. \latex{} normally treats
\verb|_|\index{_@\verb?_? (underscore)} as a math mode subscript command. 
Further, if you want
the possibility of hyphenation\index{hyphenation} 
at the position of an underscore
you have to use the command pairing 
\verb|\_\-|;\index{_ /@\verb?\_?}\ixcom{-} this has the side
effect of disabling normal hyphenation in any succeeding `word' in
the identifier.

    The \file{hyphenat}\ixpack{hyphenat} package facility redefines the 
\verb|\_|\index{_ /@\verb?\_?} command
so that hyphenationen is automatically enabled at the position
of the underscore and in all succeeding words in the identifier.


\begin{note}
    Using the command pair \verb|\_\-| in conjunction with this facility
disables automatic hyphenation of succeeding words, so don't do it.
\end{note}



\begin{example} \label{eg:uschyp}
This set of \latex{} source
\begin{verbatim}
Amazingly An\_excessively\_long\_multiword\_identifier%
\_demonstrating\_hyphenation

\begin{minipage}{3cm}
\begin{itemize}
\item An\_excessively\_long\_multiword\_identifier%
\_demonstrating\_hyphenation
\item Underscore in math mode: $A_n$
\item \verb|\_| command in math mode: $A\_n$
\end{itemize}
\end{minipage}
\end{verbatim}
prints as:

Amazingly An\_excessively\_long\_multiword\_identifier%
\_demonstrating\_hyphenation

\begin{minipage}{3cm}
\begin{itemize}
\item An\_excessively\_long\_multiword\_identifier%
\_demonstrating\_hyphenation
\item Underscore in math mode: $A_n$
\item \verb|\_| command in math mode: $A\_n$
\end{itemize}
\end{minipage}

\end{example}

\begin{example}
Contrast this example with \eref{eg:uschyp}.
This set of \latex{} source
\begin{verbatim}
Amazingly An\_\-excessively\_\-long\_\-multiword\_\-identifier%
\_\-demonstrating\_\-hyphenation\_\-disabling

\begin{minipage}{3cm}
\begin{itemize}
\item An\_\-excessively\_\-long\_\-multiword\_\-identifier%
\_\-demonstrating\_\-hyphenation\_\-disabling
\item Underscore in math mode: $A_n$
\item \verb|\_| command in math mode: $A\_n$
\end{itemize}
\end{minipage}
\end{verbatim}
prints as:

Amazingly An\_\-excessively\_\-long\_\-multiword\_\-identifier%
\_\-demonstrating\_\-hyphenation\_\-disabling

\begin{minipage}{3cm}
\begin{itemize}
\item An\_\-excessively\_\-long\_\-multiword\_\-identifier%
\_\-demonstrating\_\-hyphenation\_\-disabling
\item Underscore in math mode: $A_n$
\item \verb|\_| command in math mode: $A\_n$
\end{itemize}
\end{minipage}

\end{example}


    The \file{hyphenat} package also provides some other commands for
enabling hyphenation of words that include 
analphabetic\index{character!analphabetic}\footnote{An {\it analphabetic} 
character is any character
that is not alphabetic. Typically it refers to punctuation characters.}
characters. In this context, the phrase 
`breakable character'\index{character!breakable} is used
to describe an analphabetic character that enables hyphenation immediately
after it and does not prevent further hyphenation in the `word'
in which it occurs. The \verb|\_|\index{_ /@\verb?\_?} command produces 
a breakable underscore. Table~\ref{tab:breakable} lists all the commands
that generate breakable characters.

\begin{table}
\centering
\caption{Commands producing breakable characters} \label{tab:breakable}
\begin{tabular}{|l|c|} \hline
\textbf{Command}    & \textbf{Character} \\ \hline
\verb|\_|        & \_ \\
%\verb|\?|        & \? \\
\verb|\bshyp|    & \bshyp \\
\verb|\colonhyp| & \colonhyp \\
\verb|\dothyp|   & \dothyp \\
\verb|\fshyp|    & \fshyp \\ \hline
\end{tabular}
\end{table}

    The \verb|\bshyp|\ixcom{bshyp} command prodes a breakable backslash
(\verb|\|), \verb|\fshyp|\ixcom{fshp} produces a breakable forward
slash (\verb|/|), the \verb|\dothyp|\ixcom{dothyp} command produces a
breakable full stop (\verb|.|), also known in some countries as a period,
and the \verb|\colonhyp|\ixcom{colonhyp} command produces a breakable 
colon (:).

\begin{example}
This is similar to \eref{eg:uschyp} except that it demonstrates other
breakable characters.

This set of \latex{} source
\begin{verbatim}
Analphabetically an\bshyp{}excessively\fshyp{}long\dothyp{}multiword\bshyp{}identifier%
\fshyp{}demonstrating\dothyp{}hyphenation

\begin{minipage}{3cm}
Analphabetically an\bshyp{}excessively\fshyp{}long\dothyp{}multiword\bshyp{}identifier%
\fshyp{}demonstrating\dothyp{}hyphenation
\end{minipage}
\end{verbatim}
prints as:

Analphabetically an\bshyp{}excessively\fshyp{}long\dothyp{}multiword\bshyp{}identifier%
\fshyp{}demonstrating\dothyp{}hyphenation

\begin{minipage}{3cm}
Analphabetically an\bshyp{}excessively\fshyp{}long\dothyp{}multiword\bshyp{}identifier%
\fshyp{}demonstrating\dothyp{}hyphenation
\end{minipage}

\end{example}

\begin{note}
\latex{} will not hyphenate the first word in a paragraph.
\end{note}

    Just as with the \verb|\_|\index{_ /@\verb?\_?} command, 
the discretionary hyphen
command (i.e., \verb|\-|\ixcom{-}) should not be used in conjunction with 
any of the breakable character commands as it will then inhibit any
further potential hyphenation points. In general, any analphabetic
character in a word will inhibit further hyphenation.

\normannex{Additional commands} \label{anx:extraiso}

\sclause{Language configuration commands}

    There is an additional set of commands in the
\file{iso}\ixclass{iso} class
facility that
are specified here. This set has been provided to enable the style to be
easily configured for a non-English language.\index{non-English languages}

    The modified commands may be put in the document 
preamble\index{preamble} or, preferably,
placed in a separate \file{.sty}\index{file!.sty} file and 
called as a package. This latter option improves reuseability.

\ssclause{Words and phrases}

    To produce a non-English version of the \file{iso}\ixclass{iso}
class the relevant commands
from the following list may require re-definition together with
the heading commands in \tref{tab:dsa}.

\begin{note}In the listing, the default values are printed \emph{in this
font} to distinguish them from the explanatory text.
\end{note} % end note}

\begin{itemize}
\item \verb|\annexname|\ixcom{annexname}: Header for
      an annex.
      Default value is: \emph{\annexname}\index{Annex}
\item \verb|\contentsname|\ixcom{contentsname}: Header
      for table of contents listing. 
      Default value is: \emph{\contentsname}\index{Contents}
\item \verb|\copyrightname|\ixcom{copyrightname}:
      The copyright owner.
      Default value is: \emph{\copyrightname}\index{ISO}
\item \verb|\examplename|\ixcom{examplename}:
      Identification of an example.
      Default value is: \emph{\examplename}\index{EXAMPLE}
%\item \verb|\examplesname|\ixcom{examplesname}:
%      Header for a list of examples.
%      Default value is: \emph{\examplesname}\index{EXAMPLES}
\item \verb|\figurename|\ixcom{figurename}: Start of
      the caption for a figure. 
      Default value is: \emph{\figurename}\index{Figure}
\item \verb|\forewordname|\ixcom{forewordname}:
      Title of the Foreword.
      Default value is: \emph{\forewordname}\index{Foreword}
\item \verb|\indexname|\ixcom{indexname}: Header for
      the index.
      Default value is: \emph{\indexname}\index{Index}
\item \verb|\informativename|\ixcom{informativename}:
      Identification of an informative annex.
      Default value is: \emph{\informativename}\index{informative}
\item \verb|\inscopename|\ixcom{inscopename}:
      Introduction to in-scope listing.
      Default value is: \emph{\inscopename}
      \index{The following are within the scope of this}
\item \verb|\ISname|\ixcom{ISname}:
     `INTERNATIONAL STANDARD' as used in the header for an IS title page.
      Default value is: 
      \emph{\ISname}\index{INTERNATIONAL STANDARD}
\item \verb|\introductionname|\ixcom{introductionname}:
      Title of the Introduction.
      Default value is: 
      \emph{\introductionname}\index{Introduction}
\item \verb|\listannexname|\ixcom{listannexname}:
      Header for list of annexes listing.
      Default value is: \emph{\listannexname}\index{Annexes}
\item \verb|\listfigurename|\ixcom{listfigurename}:
      Header for list of figures listing.
      Default value is: \emph{\listfigurename}\index{Figures}
\item \verb|\listtablename|\ixcom{listtablename}:
      Header for list of tables listing.
      Default value is: \emph{\listtablename}\index{Tables}
\item \verb|\normativename|\ixcom{normativename}:
      Identification of a normative annex.
      Default value is: \emph{\normativename}\index{normative}
\item \verb|\normrefsname|\ixcom{normrefsname}:
      Title of normative references clause.
      Default value is: 
      \emph{\normrefsname}\index{Normative references}
\item \verb|\notename|\ixcom{notename}:
      Identification of a note.
      Default value is: \emph{\notename}\index{NOTE}
%\item \verb|\notesname|\ixcom{notesname}:
%      Header for a list of notes.
%      Default value is: \emph{\notesname}\index{NOTES}
\item \verb|\outofscopename|\ixcom{outofscopename}:
      Introduction to out-of-scope listing.
      Default value is: \emph{\outofscopename}
      \index{The following are outside the scope of this}
\item \verb|\pagename|\ixcom{pagename}:
      The word for the page header in the table of contents.
      Default value is: \emph{\pagename}\index{Page}
\item \verb|\scopename|\ixcom{scopename}:
      Title of the Scope.
      Default value is: \emph{\scopename}\index{Scope}
%\item \verb|\sectionname|\ixcom{sectionname}:
%      The word for a 'section'.
%      Default value is: \emph{\sectionname}\index{Section}
\item \verb|\tablename|\ixcom{tablename}:
      Start of the caption for a table.
      Default value is: \emph{\tablename}\index{Table}
\item \verb|\tbpname|\ixcom{tbpname}:
      Footnote text for `to be published.'.
      Default value is: 
      \emph{\tbpname}\index{To be published.}
\end{itemize}

    The following commands provide the names for referenced document elements.
\begin{itemize}
\item \verb|\annexrefname|\ixcom{annexrefname}:
      Reference to an annex.
      Default value is: \emph{\annexrefname}\index{annex}
\item \verb|\clauserefname|\ixcom{clauserefname}:
      Reference to a clause.
      Default value is: \emph{\clauserefname}\index{clause}
\item \verb|\examplerefname|\ixcom{examplerefname}:
      Reference to an example.
      Default value is: \emph{\examplerefname}\index{example}
\item \verb|\figurerefname|\ixcom{figurerefname}:
      Reference to a figure.
      Default value is: \emph{\figurerefname}\index{figure}
\item \verb|\noterefname|\ixcom{noterefname}:
      Reference to a note.
      Default value is: \emph{\noterefname}
\item \verb|\tablerefname|\ixcom{tablerefname}:
      Reference to a table.
      Default value is: \emph{\tablerefname}\index{table}
\item \verb|\pagerefname|\ixcom{pagerefname}:
      Reference to a page.
      Default value is: \emph{\pagerefname}\index{page}
\end{itemize}



\begin{note} The above commands, 
 may be changed via the \latex{} 
\verb|\renewcommand|\ixcom{renewcommand}.
\end{note} % end note
\begin{note}The \latex{} command \verb|\today|\ixcom{today}
      will probably also require modification. This is not something
      for the casual user to attempt.
\end{note} % end note


\begin{example}The following is a partial list of the commands to convert to a 
         French language\index{French language} style. 
\begin{verbatim}
\renewcommand{\annexname}{Annexe}
\renewcommand{\contentsname}{Sommaire}
\renewcommand{\examplename}{EXEMPLE}
%\renewcommand{\examplesname}{EXEMPLES}
\renewcommand{\forewordname}{Avant-propos}
\renewcommand{\ISname}{NORME INTERNATIONALE}
\renewcommand{\listtablename}{Tableaux}
\renewcommand{\scopename}{Domaine d'application}
\renewcommand{\tablename}{Tableau}
\end{verbatim}
\end{example} % end example

\ssclause{Boilerplate}

    Some commands print boilerplate text; again, the default text is in English.
For \file{iso}\ixclass{iso} there are several such commands. The first is 
\verb|\copyrightnotice|\ixcom{copyrightnotice}
 which contains the text of the copyright notice
for an International Standard. This can be changed via the 
\verb|\renewcommand|\ixcom{renewcommand} command.

    The second is \verb|\normrefbp|\ixcom{normrefbp}
which prints the boilerplate for the introduction to the Normative
references clause. Like the \verb|\copyrightnotice| command, this can
be redefined using the \verb|\renewcommand|.

    Another is the \verb|\fwdbp|\ixcom{fwdbp} command which
\verb|\input|s the boilerplate text from a file called \file{isofwdbp.tex}.
\index{isofwdbp.tex@\file{isofwdbp.tex}} For non-English text either 
modify the contents of this
file or create a new file and modify the \verb|\fwdbp| command to call in
the new file.



\begin{example}This is how \verb|\normrefbp| could be written for the 
French language and in accordance with the second edition of the 
ISO Directives.
\label{eg:frenchnfbp}
\begin{verbatim}
\renewcommand{\normrefbp}[1]{%
    Les normes suivantes contiennent des dispositions qui, par
suite de la r\'{e}f\'{e}nce qui en est faite, constituent des
dispositions valables pour la pr\'{e}sente #1. 
Au moment de la publication, les \'{e}ditions indiqu\'{e}es 
\'{e}taient en viguer. Toute norms est sujette \`{a} r\'{e}vision et
les parties prenantes des accords fond\'{e}s sur la pr\'{e}sente #1 
sont invit\'{e}es \`{a} rechercher la possibilit\'{e} d'appliquer
les \'{e}ditions les plus r\'{e}centes des normes indiqu\'{e}es
ci-apr\`{e}s. Les membres de la CEI et de l'ISO poss\`{e}dent
le registre des Normes Internationales en vigueur \`{a} un
moment donn\'{e}.
}% end renewcommand
\end{verbatim}
\end{example} % end example
\begin{example}Given that \verb|\normrefbp| has been redefined as in \eref{eg:frenchnfbp},
then the command \\
 \verb|\normrefbp{norme internationale}| will print:

\newcommand{\tempbp}[1]{%
    Les normes suivantes contiennent des dispositions qui, par
suite de la r\'{e}f\'{e}nce qui en est faite, constituent des
dispositions valables pour la pr\'{e}sente #1. 
Au moment de la publication, les \'{e}ditions indiqu\'{e}es 
\'{e}taient en viguer. Toute norms est sujette \`{a} r\'{e}vision et
les parties prenantes des accords fond\'{e}s sur la pr\'{e}sente #1 
sont invit\'{e}es \`{a} rechercher la possibilit\'{e} d'appliquer
les \'{e}ditions les plus r\'{e}centes des normes indiqu\'{e}es
ci-apr\`{e}s. Les membres de la CEI et de l'ISO poss\`{e}dent
le registre des Normes Internationales en vigueur \`{a} un
moment donn\'{e}.
}% end newcommand

\tempbp{norme internationale}
\end{example} % end example



    The \verb|\tspasfwdbp|\ixcom{tspasfwdbp} also maintains
boilerplate text in the file 
\file{tspasfwdbp.tex}\index{tspasfwdbp.tex@\file{tspasfwdbp.tex}}.

\sclause{Caption text size}

    The size of the font used for typestting the captions of figures and
tables is defined within the \file{iso}\ixclass{iso} class.
% The
%\verb|uglycaption|\ixopt{uglycaption} option resets the size to larger
%than normal for the captioning text.

    The size of the captioning font is controlled by the value of the
\verb|\captionsize|\ixcom{captionsize} command. The default definition
of \verb|\captionsize| is similar to:
\verb|\captionsize{\normalsize}| \ixcom{normalsize}\\
%The \verb|uglycaption| option resets this to (approximately): \\
%\verb|\captionsize{\large}|. \ixcom{large}

    You can reset the \verb|\captionsize| at any point in your
document to change the size of captions from then onwards.

    All the normal \latex{} font size commands are available.
\begin{example}
The following shows the effects of the font size commands. \\
\verb|{\tiny tiny text}|\ixcom{tiny} 
prints: {\tiny tiny text} \\
\verb|{\scriptsize scriptsize text}|\ixcom{scriptsize} 
prints: {\scriptsize scriptsize text} \\
\verb|{\footnotesize footnotesize text}|\ixcom{footnotesize} 
prints: {\footnotesize footnotesize text} \\
\verb|{\small small text}|\ixcom{small}
prints: {\small small text} \\
\verb|{\normalsize normalsize text}|\ixcom{normalsize} 
prints: {\normalsize normalsize text} \\
\verb|{\large large text}|\ixcom{large} 
prints: {\large large text} \\
\verb|{\Large Large text}|\ixcom{Large} 
prints {\Large Large text} \\
\verb|{\LARGE LARGE text}|\ixcom{LARGE} 
prints: {\LARGE LARGE text} \\
\verb|{\huge huge text}|\ixcom{huge} 
prints: {\huge huge text} \\
\verb|{\Huge Huge text}|\ixcom{Huge} 
prints: {\Huge Huge text}
\end{example}


\normannex{Ordering of LaTeX commands} \label{anx:lord}

    The \latex{} commands for the logical structuring of an ISO standard 
document are:
\begin{verbatim}
\documentclass[<options>]{isov2}          % for LaTeX 2e
\usepackage{<name>}                       % additional packages (LaTeX 2e)
\standard{<standard identification>}
\yearofedition{<year>}
\languageofedition{<parenthesized code letter>}
  % other preamble commands
\begin{document}
\begin{foreword}                     % start Foreword
  \fwdbp                             % boilerplate
  % other text and perhaps \fwdnopatents
\end{foreword}
\begin{introduction}                 % start Introduction
  % text and perhaps \intropatents
\end{introduction}
\title{<intro>}{<main>}{<compl>}     % the title
\scopeclause                         % The Scope clause
   \begin{inscope}{<document>}       % in scope boilerplate
   % \item list
   \end{inscope}
   % text 
   \begin{outofscope}{<document>}    % out of scope boilerplate
   % \item list
   \end{outofscope}
   % text
\normrefsclause                      % The Normative references clause
  \normrefbp{<document identifier>}  % boilerplate
  \begin{nreferences}
    % \isref{<p1>}{<p2>} and/or \disref{<p1>}{<p2>} commands
  \end{nreferences}
% definitions, symbols, abbreviation clause as appropriate
\clause{<Clause title>}
% THE BODY OF THE DOCUMENT
% ...
% \normannex{<Normative annex title>}
% ...
% \infannex{<Infomative annex title>}
% ...
\bibannex                            % optional bibliography
  % bibliography listing
% the index
\end{document}
\end{verbatim}


\infannex{Creating an index} \label{anx:indexing}

    \latex, although providing some assistance in preparing the data for an
index\index{index}, only does part of the job. Providing the command
\verb|\makeindex|\ixcom{makeindex} is put in the document's 
preamble\index{preamble}
the \latex{} command \verb|\index{text}|\ixcom{index} 
writes out \verb|text| to an \file{.idx}\index{file!.idx} file 
in the following format:\ixcom{indexentry}
\begin{verbatim}
\indexentry{text}{pg}
\end{verbatim}
where \verb|pg| is the page number in the document where the \verb|\index|
command occurred.

    The \verb|theindex|\ixenv{theindex} environment 
is used for printing an index. The format of this is:
\begin{verbatim}
\begin{theindex}
\item <text and page numbers>
  \subitem <text and page numbers>
    \subsubitem <text and page numbers>
.
.
\end{theindex}
\end{verbatim}
where \verb|\item|\ixcom{item} is a major topic entry,
\verb|\subitem|\ixcom{subitem} is a sub-topic entry, and
\verb|\subsubitem|\ixcom{subsubitem} is a sub-subtopic
entry. The command \verb|\indexspace|\ixcom{indexspace}
can be used to add space between the entries. Other text and commands can also
occur within the environment.

    \latex{} provides no help in going from an \file{.idx} file to the
\verb|theindex| environment~\bref{lamport}. The data in the file has to be
sorted, duplicate page numbers deleted, etc, and then re-written in the 
desired environment format.

    The typical process for producing a document with an index is:
\begin{enumerate}
\item Prepare the source file, say \file{fred.tex}, with the command
      \verb|\makeindex|\ixcom{makeindex} in the
      preamble\index{preamble}, and \verb|\index|\ixcom{index}
      commands within the body of the text.
\item Run \latex. As well as producing the usual output files, it will also
      produce the file \file{fred.idx}.
\item By some means produce a file, let's call it \file{fredidx.tex}, from the 
      data in \file{fred.idx} that contains the appropriate \verb|theindex|
      formatted data.
\item Run \latex{} again on \file{fred.tex} which now has to contain (either
      via \verb|\input|\ixcom{input} or \verb|\include|\ixcom{include})
      the file \file{fredidx.tex}.
\end{enumerate}

    Chen and Harrison~\bref{chen} discuss the problems of creating an index 
in their
paper \emph{Index preparation and processing} and also describe the
{\sc MakeIndex}\index{makeIndex@{\sc MakeIndex}} program. Goosens, Mittelbach
and Samarin~\bref{goosens} also describe how to use the {\sc MakeIndex} 
program for producing indexes for \latex{} documents. For users of
{\sc MakeIndex} a style file called 
\file{iso.ist}\index{iso.ist@\file{iso.ist}}\index{file!.ist} is provided as
part of this distribution.

\sclause{The index command}\ixcom{index}

    The \verb|\index| command is one of the standard \latex{} commands.
The command format is \verb|\index{|\meta{str}\verb|}|, where \meta{str}
 is any
string of characters, and it writes an entry to the \file{.idx} file in
the form \verb|\indexentry{<str>}{pg}|, where \verb|pg| is the page
number in the document where the command is called.

    Some points to note:
\begin{itemize}
\item The \verb|\index| command is, in \latex{} terminology, 
      \emph{fragile}\index{fragile}. That is, if it appears in a moving
      argument (like the caption to a table or figure) it must be preceded
      by the \latex{} \verb|\protect|\ixcom{protect}
      command.
\item Any of the ten \latex{} special characters\index{special characters}
      (i.e., \verb|#|, \verb|$|, \verb|%|, \verb|&|, \verb|~|, \verb|_|,
             \verb|^|, \verb|\|, \verb|{| and \verb|}|) may appear
     within the argument, with the proviso that it must have no unmatched
     braces (the braces in \verb|\{| and \verb|\}| are counted in the
     matching process).
\item The \verb|\index| command must not appear inside another command's
     argument (e.g., within a 
     \verb|\footnote|\ixcom{footnote} or \verb|\clause|\ixcom{clause}
     command) unless the argument of the \verb|\index| command contains
     only letters, digits, and/or punctuation characters. In particular,
     it should not contain any of the special characters.
  \begin{note}This means that the argument of the \verb|\ix|\ixcom{ix}
        command should not contain any special characters. Remember that the
       \verb|\ix| command prints its argument in the body of the text and also
       calls \verb|\index| to place its argument into the \file{.idx} file.
       \end{note} % end note
\end{itemize}

\begin{note}Under some circumstances, an \verb|\index| command appearing in another
command's argument may contain special characters, provided they are 
\verb|\protect|ed. Determination of when this is satisfactory is a process
of trial and error.
\end{note} % end note

\begin{example}The command \verb|\ix{an\protect\_underscore}| will print the
characters \emph{an\_undescore}
in the text and also write the following to the \file{.idx} file:
\begin{verbatim}
\indexentry{an\_ underscore}{pg}
\end{verbatim}
Notice that there is a space between the underscore character and the word
`underscore' in the \file{.idx} file (but there is no space in the printed
body of the document text). This extraneous space may have to be edited out 
from the final index.
\end{example} % end example


\begin{comment}

\sclause{The program GenIndex} \index{genindex@GenIndex}

    GenIndex is a C program\index{C language} that converts \file{.idx}
data to \verb|theindex|\ixenv{theindex} data. 
Source code for GenIndex is freely available from
the NIST SOLIS\index{SOLIS} system (see \ref{anx:solis}). 
The GenIndex program is not
as sophisticated as {\sc MakeIndex} but does a reasonable 
job.\footnote{These two programs are not completely compatible but do have
a common subset of commands. The common subset excludes the {\tt SeeAlso}
and {\tt See} commands. At some later time GenIndex may be rewritten to
be compatible with {\sc MakeIndex}. It is probably advisable, therefore, not
to use the GenIndex {\tt SeeAlso} and {\tt See} commands.}


\ssclause{Input} \index{genindex@GenIndex!input}

    GenIndex reads lines of data of the form:
\begin{verbatim}
\indexentry{IndexData}{Page}
\end{verbatim}

    \verb|Page| is a character string representing a page number. GenIndex only
recognizes strings that represent integer values greater than or equal to zero,
or (upper and lower case) roman numerals. For any other numbering system, the
page number is treated as zero.

    \verb|IndexData| is a string of characters and command characters that
specify the data that is to be indexed. In the simplest case, this is just the
word or phrase to appear in the index, but much more can be done than this, as
is described below.

    The general form of \verb|IndexData| is given by the following partial
grammar:
\begin{verbatim}
IndexData  = MainData ['!' SubData ['!' SubSubData ] ] .
MainData   = Data .
SubData    = Data .
SubSubData = Data .
Data       = Key [ Entry ] [ SeeAlso ] [ See ] .
Key        = any string not containing the @, !, + or * characters .
Entry      = '@' any string not containing the !, + or * characters 
                 (unless enclosed in braces {} ) .
SeeAlso    = '+' any string not containing the ! or * characters
                 (unless enclosed in braces {} ) .
See        = '*' any string not containing the ! character
                 (unless enclosed in braces {} ) .
\end{verbatim}

    Upto three levels of indexing are enabled --- a main topic entry, an optional
sub-topic, and an optional sub-subtopic. The sub-topics for an entry are
introduced by the \verb|!| character. \index{(33@{\verb?!?}}

     Apart from \verb|Key|, braces within a string must be matched. That is,
they must appear in pairs of an opening and a closing brace.
Spaces are significant within the \verb|Key| string, but not in the others.
That is, \verb*|\indexentry{A}| differs from \verb*|\indexentry{ A}| 
both of which differ from \verb*|\indexentry{A }|.


\begin{example}Here is an index entry for a simple topic:
\begin{verbatim}
\indexentry{Topic}{27}
\end{verbatim}
\end{example} % end example
\begin{example}And here is one where the key and the main entry are different, a 
sub-topic is being indexed, and the page is in roman numerals:
\begin{verbatim}
\indexentry{main@\textbf{Main}!sub topic}{xviii}
\end{verbatim}
\end{example} % end example
\begin{example}This is how the characters \verb|@|, \verb|!|, \verb|+| and
\verb|*| characters are indexed in this document.
\begin{verbatim}
\index{(64@{\verb?@?}}
\index{(33@{\verb?!?}}
\index{(43@{\verb?+?}}
\index{(42@{\verb?*?}}
\end{verbatim}
\end{example} % end example


\sssclause{Key} \index{genindex@GenIndex!input!key}

    The indexed entries are sorted alphabetically on the value of \verb|Key|.
Sorting is case-insensitive. A \verb|Key| value is required.

    More precisely, the key entries are sorted according to the
C language\index{C language} implementation collating sequence, which is usually ASCII. 
Table~\ref{tab:ascii} gives the ASCII collating sequence for the \latex\
character set. \index{ASCII}


\begin{table}
\def\vissp{\hbox{\tt\char`\ }} % visible space
\centering
\caption{The \protect\latex{} ASCII character set} \label{tab:ascii}
\begin{tabular}{|r|c|c|c|c|c|c|c|c|c|c|} \hline
  & \textbf{0} & \textbf{1} & \textbf{2} & \textbf{3} & \textbf{4} &
\textbf{5} & \textbf{6} & \textbf{7} & \textbf{8} & \textbf{9} \\ \hline\hline
\textbf{30}  &          &          & \vissp   & \verb|!| & \verb|"| &
            \verb|#| & \verb|$| & \verb|%| & \verb|&| & \verb|'|    \\
\textbf{40}  & \verb|(| & \verb|)| & \verb|*| & \verb|+| & \verb|,| &
            \verb|-| & \verb|.| & \verb|/| & \verb|0| & \verb|1|     \\
\textbf{50}  & \verb|2| & \verb|3| & \verb|4| & \verb|5| & \verb|6| &
            \verb|7| & \verb|8| & \verb|9| & \verb|:| & \verb|;|     \\
\textbf{60}  & \verb|<| & \verb|=| & \verb|>| & \verb|?| & \verb|@| &
            \verb|A| & \verb|B| & \verb|C| & \verb|D| & \verb|E|     \\
\textbf{70}  & \verb|F| & \verb|G| & \verb|H| & \verb|I| & \verb|J| &
            \verb|K| & \verb|L| & \verb|M| & \verb|N| & \verb|O|     \\
\textbf{80}  & \verb|P| & \verb|Q| & \verb|R| & \verb|S| & \verb|T| &
            \verb|U| & \verb|V| & \verb|W| & \verb|X| & \verb|Y|     \\
\textbf{90}  & \verb|Z| & \verb|[| & \verb|\| & \verb|]| & \verb|^| &
            \verb|_| & \verb|`| & \verb|a| & \verb|b| & \verb|c|     \\
\textbf{100} & \verb|d| & \verb|e| & \verb|f| & \verb|g| & \verb|h| &
            \verb|i| & \verb|j| & \verb|k| & \verb|l| & \verb|m|     \\
\textbf{110} & \verb|n| & \verb|o| & \verb|p| & \verb|q| & \verb|r| &
            \verb|s| & \verb|t| & \verb|u| & \verb|v| & \verb|w|     \\
\textbf{120} & \verb|x| & \verb|y| & \verb|z| & \verb|{| & \verb/|/ &
            \verb|}| & \verb|~| &          &          &              \\ \hline
\end{tabular}
\end{table}

\sssclause{Entry} \index{genindex@GenIndex!input!entry} 
\index{(64@{\verb?@?}}

    \verb|Entry| is optional and is introduced by the \verb|@| character. If present, this will form the indexed string.
If absent, then the value of \verb|Key| is used instead.

\sssclause{See also} \index{genindex@GenIndex!input!see also}
 \index{(43@{\verb?+?}}

    \verb|SeeAlso| is optional and is introduced by the \verb|+| character. 
Within a document, it should only be used once
per entry. If used more than once, then only the last value is taken.

    This is used to produce an entry that refers to another indexed entry in
a \emph{see also \ldots} style.

\sssclause{See} \index{genindex@GenIndex!input!see} 
\index{(42@{\verb?*?}}

    \verb|See| is optional and is introduced by the \verb|*| character. 
Within a document, it should be used only once per
entry. If used more than once, then only the last value is taken.

    This is used to produce an entry that refers to another entry in a
\emph{see \ldots} style. Use of this option prohibits printing page numbers 
for the entry.

\ssclause{Output} \index{genindex@GenIndex!output}

    The program GenIndex sorts the entries into alphabetical order (based on
the \verb|Key| values), and produces a \verb|theindex| for the input data.
Several formatting commands are included in the output file to enable
adjustment of the appearance of the printed index. These are shown in
\tref{tab:indexc}.

\begin{table}
\centering
\caption{GenIndex formatting commands} \label{tab:indexc}
\begin{tabular}{|l|l|} \hline
\textbf{Command} & \textbf{Effect} \\ \hline
\verb|\indexfill|\ixcom{indexfill} & 
    spacing style between main topic and page numbers \\
\verb|\sindexfill|\ixcom{sindexfill} & 
    spacing style between subtopic and page numbers \\
\verb|\ssindexfill|\ixcom{ssindexfill} & 
    spacing between sub-subtopic and page numbers \\
\verb|\indexsee{text}|\ixcom{indexsee} & 
    produces \verb|text| as a \emph{see} entry \\
\verb|\indexseealso{text}|\ixcom{indexseealso} & 
    produces \verb|text| as a \emph{see also} entry \\
\verb|\otherindexspace{c}|\ixcom{otherindexspace} & 
    heading for non-alphabetic entry \\
\verb|\alphaindexspace{c}|\ixcom{alphaindexspace} & 
    heading for an alphabetic entry  \\ \hline
\end{tabular}
\end{table}


\begin{note}The commands in \tref{tab:indexc} are defined in the 
      \file{iso}\ixclass{iso} class file.
\end{note}
\begin{note}If the commands are not defined in your system, 
      then you can define them
      using the \latex{} \verb|\newcommand|\ixcom{newcommand}
      command. On the other hand, if they are defined in your system, 
      you can change them using the 
      \latex{} \verb|\renewcommand|\ixcom{renewcommand} command.
\end{note} % end note


\begin{example}
This input file:
\begin{verbatim}
\indexentry{Freddy+Fred}{27}
\indexentry{Frederick*Fred}{29}
\indexentry{Fred}{42}
\indexentry{Fred}{52}
\indexentry{Fred}{43}
\end{verbatim}
will produce output like:
\begin{verbatim}
\begin{theindex}
\alphaindexspace{F}
\item Fred \indexfill 42--43, 52
\item Frederick \indexsee{Fred}
\item Freddy \indexfill 27 \indexseealso{Fred}
\end{theindex}
\end{verbatim}
\end{example} % end example

\sssclause{Indexfill commands}

    The commands \verb|\indexfill{|\meta{style}\verb|}|\ixcom{indexfill}, 
\verb|\sindexfill{|\meta{style}\verb|}|\ixcom{sindexfill} and 
\verb|\ssindexfill{|\meta{style}\verb|}|\ixcom{ssindexfill}
control the style of spacing between an indexed entry and its corresponding
page numbers.

\begin{example}
If you want to have the page numbers right justified, with lines between the
main topics and their numbers, dots between sub-topics and pages, and 
sub-subtopic numbers right justified, then you could define these commands as:
\begin{verbatim}
\newcommand{\indexfill}{\hrulefill}
\newcommand{\sindexfill}{\dotfill}
\newcommand{\ssindexfill}{\hfill}
\end{verbatim}
\end{example} % end example


\begin{anote}The usual \latex{} style is to have a small gap between the
topic and page number, the whole being set ragged right.
If this is what you want, then define the commands as:
\begin{verbatim}
\newcommand{\indexfill}{}
\newcommand{\sindexfill}{}
\newcommand{\ssindexfill}{}
\end{verbatim}
\end{anote} % end note

\sssclause{Indexsee and indexseealso commands}

    The \verb?\indexsee{?\meta{text}\verb?}?\ixcom{indexsee} and 
\verb?\indexseealso{?\meta{text}\verb?}?\ixcom{indexseealso} commands 
control the appearance
of the \emph{see} and \emph{see also} text. 


\begin{example}
These commands could be defined as:
\begin{verbatim}
\newcommand{\indexsee}[1]{\par \hspace*{2em} \emph{see} #1}
\newcommand{\indexseealso}[1]{\par \hspace*{2em} \emph{see also} #1}
\end{verbatim}
\end{example} % end example

\sssclause{Index space commands}
\ixcom{alphaindexspace}
\ixcom{otherindexspace}

    The commands 
\verb?\alphaindexspace{?\meta{c}\verb?}?\ixcom{alphaindexspace} and 
\verb?\otherindexspace{?\meta{c}\verb?}?\ixcom{otherindexspace}
control the amount of space between blocks of index entries. 
These are an extension of the
\latex{} \verb|\indexspace|\ixcom{indexspace}
command, which just inserts some vertical space into the index listing.
Both these commands take a single parameter, which is typically a single
character.

\begin{anote}
GenIndex looks at the first character (call it \verb|c|) of the \verb|Key| 
and if this changes
as it processes the ordered list of main topics, it puts that character
as the parameter for these commands. It writes \verb|\alphaindexspace{c}| if
the character is alphabetic (and \verb|c| is put into upper-case), 
otherwise it writes \verb|\otherindexspace{c}|.
\end{anote} % end note

\begin{example}
These commands could be defined as:
\begin{verbatim}
\newcommand{\alphaindexspace}[1]{\indexspace
                       {\bfseries #1}}
\newcommand{\otherindexspace}[1]{}
\end{verbatim}
which would result in the printing of a vertical space and a bold font text
for an alphabetic header, or nothing for anything else.
\end{example} % end example

\ssclause{Running GenIndex} \index{genindex@GenIndex!run}

    To run GenIndex, first obtain a copy of the program, and compile it if
necessary. Execute the program on your data.

    GenIndex takes zero, one or two file names as parameters. If no files are
given then input and output is from and to \verb|stdin| and \verb|stdout|
respectively.

    If one file is given, then input is taken from this file, and output
is to \verb|stdout|.

    If two file names are given, then input is taken from the first and output
is to the second.

\begin{example}
A typical usage would be:
\begin{verbatim}
GenIndex fred.idx fredidx.tex
\end{verbatim}
which would read the \file{fred.idx} file and write the resulting index data to
file \file{fredidx.tex}.
\end{example} % end example

\end{comment}

% sgmlannx.tex    latex and SGML

\infannex{LaTeX, the Web, and *ML} \label{anx:sgml} \index{SGML}

    ISO are becoming more interested in electronic sources for their
standards as well as the traditional camera-ready copy. Acronyms like
PDF, HTML, SGML and XML have been bandied about. Fortunately documents
written using \latex{} are well placed to be provided in a variety of 
electronic formats. A comprehensive treatment of \latex{} with respect
to this topic is provided by Goossens and Rahtz~\bref{lwebcom}.

    SGML (Standard Generalized Markup Language) is a document tagging 
language that is described in ISO~8879~\bref{sgml} and whose usage is described 
in~\bref{bryan}, among others. The principal
mover behind SGML is Charles Goldfarb from IBM, who has authored a detailed 
handbook~\bref{goldfarb} on the SGML standard.

    The concepts lying behind both \latex{} and SGML are similar, but on the face
of it they are distinctly different in both syntax and capabilities. ISO is
migrating towards electronic versions of its standard documents and, naturally, 
would prefer these to be SGML tagged. 
     Like \latex, SGML has a
concept of style files, which are termed DTDs, and both systems support
powerful macro-like capabilities. SGML provides for logical document
markup and not typesetting --- commercial SGML systems often use
\TeX{} or \latex{} as their printing engine, as does the NIST SGML
environment for ISO~10303~\bref{pandl}.



NIST have SGML tagged some ISO~10303 documents 
using manual methods, which are time consuming and expensive. 
About 1997 there was a NIST 
effort underway to develop an auto-tagger that would (semi-) automatically convert
a \latex{} tagged document to one with SGML tags. This tool assumed a
fixed set of \latex{} macros and a fixed DTD.
 The design of an auto-tagger
essentially boils down to being able to convert from a source document tagged
according to a \latex{} style file to one which is tagged according to an
SGML DTD.
    Fully automatic conversion is really only possible if the authors'
of the documents to be translated avoid using any `non-standard' macros within
their documents. There is a program called \file{ltx2x}\index{ltx2x} available
from SOLIS, which replaces \latex{} commands within a document with
user-defined text strings~\bref{ltx2x}. This can be used as a basis for
a \latex{} to whatever auto-tagger, provided the \latex{} commands are not
too exotic.

    HTML is a simple markup language, based on SGML, and is used for the
publication of many documents on the Web. XML is a subset of SGML and appears
to being taken up by every man and his dog as \emph{the} document markup
language. HTML is being recast in terms of XML instead of SGML. PDF is a page
description language that is a popular format for display of documents 
on the Web.

    \latex{} documents can be output in PDF by using pdfLaTeX. Instead
of a \file{.dvi} file being produced a \file{.pdf} file is output directly.
The best 
results are obtained when PostScript fonts rather than Knuth's cm fonts 
are used. Noting that the \file{iso} class provides an \verb|\ifpdf| command,
a general form for documents to be processed by either \latex{} or pdfLaTeX
is
\begin{verbatim}
\documentclass{isov2}
\usepackage{times}     % PostScript fonts Times, Courier, Helvetica
\ifpdf
  \pdfoutput=1         % request PDF output
  \usepackage[pdftex]{graphicx}
\else
  \usepackage{graphicx}
\fi
...
\end{verbatim}

    There are several converters available to transform a \latex{} document 
into an HTML document, but like \file{ltx2x} they generally do their own
parsing of the source file, and unlike \file{ltx2x} are typically limited
to only generating HTML. Eitan Gurari's \file{TeX4ht}\index{TeX4ht} 
suite is a notable
exception (see Chapter~4 and Appendix~B of~\bref{lwebcom}). It uses the 
\file{.dvi} file as input, so that all the parsing is done by \TeX, and can be
configured to generate a wide variety of output formats.
A set of \file{TeX4ht} configuration files are available for converting
ISO \latex{} documents into HTML\footnote{Later, configuration files for XML
output will be developed.}.

    Some points to watch when writing \latex{} documents that will assist
in translations into *ML are given below. Typically, attention to these points
will make it easier to parse the \latex{} source.

\begin{itemize}
\item Avoid using the \verb|\label|\ixcom{label} command within
      clause headings or captions. It can just as easily be placed immediately
      after these constructs.
\item Avoid using the \verb|\index|\ixcom{index} command within
      clause headings or captions. It can just as easily be placed immediately
      after these constructs.
\end{itemize}






\infannex{Obtaining LaTeX and friends} \label{anx:getstuff}

    \latex{} is a freely available document typesetting system. There are many
public domain additions to the basic system.
The information below gives pointers to where
you can obtain \latex{} etc., from the\index{Internet} Internet. 

    \latex{} runs on a wide variety of hardware, from PCs to Crays. 
Source to build a \latex{} system is freely available via anonymous 
ftp\index{ftp} from what is called CTAN\index{CTAN} 
(Comprehensive \tex{} Archive Network). 
There are three sites; pick the one nearest to you.
\begin{itemize}
\item \url{ftp.dante.de} CTAN in Germany;
\item \url{ftp.tex.ac.uk} CTAN in the UK;
\item \url{ctan.tug.org} CTAN in the USA;
\end{itemize}
The top level CTAN directory for \latex{} and friends is 
\url{/tex-archive}. CTAN contains a wide variety
of (La)TeX sources, style files, and software tools and scripts 
to assist in document processing.

\begin{anote}
    CTAN is maintained by the \tex{} Users Group (TUG). Their homepage
\isourl{http://www.tug.org} should be consulted for the current
list of CTAN sites and mirrors.
\end{anote}

\begin{comment}

\sclause{SOLIS} \index{SOLIS} \label{anx:solis}

    SOLIS is the \emph{SC4 On Line Information Service}. It contains many electronic
sources of STEP related documents. The relevant top level directory is
\url{pub/subject/sc4}.
 In particular, SOLIS contains the source for this document
and the \file{.sty} files, as well as other \latex{} related files. 
The \latex{} root directory is \url{sc4/editing/latex}. The latest
versions of the \latex{} related files are kept in the sub-directory 
\url{latex/current}.
Some \latex{} related programs are also available in the 
\url{latex/programs} sub-directory.

    SOLIS can be reached at \isourl{http://www.nist.gov/sc4}.

\end{comment}


\infannex{Changes in this release} \label{anx:changes}

    Many of the commands and environments have been redefined in order
to match the change in requirements from the the third to the fourth
edition of the ISO Directives. Usage of these is unaffected.

    The following changes have been made in this release:
\begin{itemize}
\item The \verb|cover|\ixenv{cover} environment has been added;

\item The boolean test \verb|\ifpdf|\ixcom{ifpdf} has been added;

\item The command \verb|\fwdnopatents|\ixcom{fwdnopatents} has been added
      for patent boilerplate in the Foreword.

\item The command \verb|\tpasfwdbp|\ixcom{tpasfwdbp} has been added
      for TS/PAS Foreword boilerplate.

\item The command \verb|\intropatents|\ixcom{intropatents} has been added
      for patent boilerplate in the Introduction.

\item The commands \verb|\pref|\ixcom{pref} and 
      \verb|\pagerefname|\ixcom{pagerefname} have been added.

\item The command \verb|\trwfwdbpi|\ixcom{trfwdbpi} for Foreword boilerplate
      in a Technical Report has been deleted.

\item The \verb|notes|\ixenv{notes} and \verb|examples|\ixenv{examples}
      environments have been deleted.

\item Support is provided for the \file{hyperref}\ixpack{hyperref}
      package.

\end{itemize}



\bibannex
%\label{biblio}

\begin{references}
\reference{LAMPORT, L.,}{LaTeX --- A Document Preparation System,}
            {Addison-Wesley Publishing Co., 2nd edition, 1994.} \label{lamport}
\reference{WILSON, P.R.,}{LaTeX files for typesetting ISO standards:
           Source code,}
           {NISTIR,  National Institute of Standards and Technology,
           Gaithersburg, MD 20899. June 1996.} \label{isoe}
\reference{WILSON, P.R.,}{LaTeX package files for ISO~10303: User manual,}
           {NISTIR,  National Institute of Standards and Technology,
           Gaithersburg, MD 20899. June 1996.} \label{stepsty}
\reference{RAHTZ, S., and BARROCA, L.,}{A style option for rotated
           objects in \latex,}{ TUGBoat, volume 13, number 2, pp 156--180, 
           July 1992.} \label{rahtz}
\reference{GOOSSENS, M., MITTELBACH, F. and SAMARIN, A.,}{%
           The LaTeX Companion,}
           {Addison-Wesley Publishing Co., 1994.} \label{goosens}
\reference{GOOSSENS, M., and RAHTZ, S.,}{%
           The LaTeX Web Companion --- Integrating TeX, HTML and XML,}
           {Addison-Wesley Publishing Co., 1999.} \label{lwebcom}
\reference{CHEN, P. and HARRISON, M.A.,}{Index preparation and
           processing,}{Software--Practice and Experience, 19(9):897--915,
           September 1988.} \label{chen}
%\reference{KOPKA, H. and DALY, P.W.,}{A Guide to LaTeX,}
%           {Addison-Wesley Publishing Co., 1993.} \label{kopka}
%\reference{WALSH, N.,}{Making TeX Work,}{O'Reilly \& Associates, Inc.,
%           103 Morris Street, Suite A, Sebastopol, CA 95472. 1994. } \label{walsh}
\reference{ISO 8879:1986,}{Information processing --- 
                                Text and office systems ---
           Standard Generalized Markup Language (SGML).}{} \label{sgml}
\reference{GOLDFARB, C.F.,}{The SGML Handbook,}
           {Oxford University Press, 1990.} \label{goldfarb}
\reference{BRYAN, M.,}{SGML --- An Author's Guide to the Standard Generalized
           Markup Language,}{Addison-Wesley Publishing Co., 1988. } \label{bryan}
\reference{PHILLIPS, L. and LUBELL, J.,}{An SGML Environment for STEP,}%
          {NISTIR 5515, National Institute of Standards and Technology,
           Gaithersburg, MD 20899. November 1994.} \label{pandl}
\reference{WILSON, P. R.,}{LTX2X: A LaTeX to X Auto-tagger,}%
          {NISTIR, National Institute of Standards and Technology,
           Gaithersburg, MD 20899. June 1996.} \label{ltx2x}
\begin{comment}
\reference{RESSLER, S.,}{The National PDES Testbed Mail Server User's Guide,}
           {NSTIR 4508, National Institute of Standards and Technology,
           Gaithersburg, MD 20899. January 1991.} \label{ressler}
\reference{RINAUDOT, G.R.,}{STEP On Line Information Service (SOLIS),}
          {NISTIR 5511, National Institute of Standards and Technology,
          Gaithersburg, MD 20899. October 1994. } \label{rinaudot}
\reference{KROL, E.,}{The Whole Internet --- User's Guide \& Catalog,}
           {O'Reilly \& Associates, Inc.,
           103 Morris Street, Suite A, Sebastopol, CA 95472. 1993. } \label{krol}
\end{comment}
\reference{WILSON, P.R.,}{The hyphenat package,}%
          {1999. (Available from CTAN)} \label{bib:hyphenat}
\reference{WILSON, P.R.,}{The xtab package,}%
          {1998. (Available from CTAN)} \label{bib:xtab}
\end{references}

    
%%%%%%% here is the index at the end
%%\input{isomanidx}
\input{isoman.ind}



\end{document}




\end{document}




\end{document}




\end{document}
