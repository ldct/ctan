  \documentclass[12pt,a4paper]{article}
   \usepackage{bibarts}   %% Reihenfolge ist gleichgueltig.
   %\notnegcorrdefabk     %% LaTeX 2.09 %%
          \usepackage{ngerman}
          %\usepackage{german}
          %\usepackage[ngermanb]{babel}
          %\usepackage[germanb]{babel}
\setlength{\footnotesep}{2ex}   %% ... wie in bibarts.sty 2.0; siehe readme.txt %%
       %\usepackage[utf8]{inputenc} %% wuerde Ein-Zeichen-Umlaute moeglich machen %%
       \usepackage[T1]{fontenc} %% Automatische Trennung von Worten mit Umlauten. %%


%%\documentstyle[12pt,german,a4,bibarts]{article}

\hyphenation{Stern-ar-gu-ment 
Voll-ein-trag 
Satz-ende 
Nach-kom-ma-stel-len
De-zi-mal-zah-len-er-ken-nung
Auf-marsch-an-wei-sun-gen
Na-mens-ar-gu-men-te
Nach-na-mens-ar-gu-men-ten
Sprach-ein-stel-lung
Stern-ar-gu-men-te
Text-edi-tor
under-sco-res
Zei-len-um-bruch
Zei-len-um-br"u-che
}

       \newcommand{\pdfko}[1]{\kern #1pt
                              %\rule[-0.4ex]{0.8pt}{2ex}\kern -0.8pt
                          \strut\ignorespaces}%
       %\newcommand{\pdfko}[1]{\strut\ignorespaces}% Falls NICHT mit pdflatex uebersetzt wird

       \newcommand{\pbs}{\string\ \unskip}
       \newcommand{\bs}{\protect\pbs}

       \title{Das \LaTeX-Paket \BibArts}
       \author{\textsc{Timo Baumann}}
       \date{\small Version 2.1, \copyright~2016.  
         \hspace{.5em}{Zur 1.3\hspace{.075em}\hy Kompatibilit"at S.\,\pageref{compabil}.}
         \hspace{.5em}\textbf{Inhalt S.\,\pageref{SectIn}}.}

\long\def\Doppelbox#1#2{\nopagebreak\vspace{2.25ex}%
       {\fbox{\parbox{.45\textwidth}{\raggedright\footnotesize\ttfamily
       \ignorespaces #1}\hfill\parbox{.45\textwidth}{\vspace{1ex}%
               \begin{minipage}{.45\textwidth}\sloppy\small
               \renewcommand{\thempfootnote}{\arabic{mpfootnote}}%
               \setcounter{mpfootnote}{\arabic{footnote}}%
               {\ignorespaces #2}%
               \setcounter{footnote}{\arabic{mpfootnote}}%
       \end{minipage}
               \vspace{1ex}}}\vspace{2.4ex}}}

 \begin{document}

 \maketitle 

 \noindent
 \BibArts\ soll \LaTeX\hy Anwender beim Schreiben geisteswissenschaftlicher
 Texte unterst"utzen (\kern-.05em\textit{arts faculty}). Der Vorspann eines deutschen 
 \LaTeX\hy Textdokuments, das das Stylefile \verb|bibarts.sty| einl"adt, 
 sieht typischerweise so aus:

  \vspace{-1.5ex}
        {\small\begin{verbatim}
   \documentclass[12pt,a4paper]{article} 
     \usepackage{bibarts}          \usepackage{ngerman}
     \usepackage[utf8]{inputenc}   \usepackage[T1]{fontenc}  %>DVI
  \end{verbatim}}
 
 \vspace{-4ex}\noindent
 Das separate Tippen von Anh"angen kann nun weitgehend entfallen. 
 Dazu werden Kopien von Literaturangaben, die sich im Haupttext oder 
 den Fu"snoten befinden, in einer Literaturliste und weiteren Listen 
 sortiert ausgedruckt.
 
 \vspace{1ex}\noindent
 Der wichtigste dieser Ausdruckbefehle, \verb|\printvli|, verh"alt sich 
 "ahnlich wie \verb|\tableofcontents| f"ur das Inhaltsverzeichnis: 
 Dazu muss im \LaTeX\hy Text ja auch
 \verb|\section{|\texttt{"Uberschriftentext}\verb|}| markiert werden. 
 Und f"ur \BibArts\ gilt:
 
 \Doppelbox
 {...\bs footnote\{Ein Beispiel f"ur Geschichtsliteratur ist
  \bs vli\{Hans-Ulrich\}\{Wehler\}\{Das Deutsche Kaiserreich, 
  \\ G"ottingen 1994\}.\}
 }
 {...\footnote{Ein Beispiel f"ur Geschichtsliteratur ist
  \vli{Hans-Ulrich}{Wehler}{Das Deutsche Kaiserreich, 
  G"ottingen 1994}.}
 }

\noindent
In \verb|\vli| l"asst sich ein sp"ater verwendeter Kurztitel 
mit \verb|\ktit| so einf"uhren:

\Doppelbox
{...\bs footnote\{Soziologie:
 \bs vli\{Niklas\} \{Luhmann\} \{\bs ktit\{Soziale Systeme\}. 
 \\ Grundri"s einer allgemeinen Theorie, 1984: Frankfurt/M.\}.\}
}
{...\footnote{Soziologie:
 \vli{Niklas} {Luhmann} {\ktit{Soziale Systeme}. Grundri"s einer 
 allgemeinen Theorie, 1984: Frankfurt/M.}.\balabel{luhmann}}
}

\vfill\noindent 
Beide Arten der Eingabe (mit und ohne \verb|\ktit|) kommen in die 
Literaturliste, die das erw"ahnte \verb|\printvli| ausdruckt.
Dies hat nichts mit \textsc{Bib}\TeX\ zu tun:\pdfko{.25}

 \printvli
  
 \vspace{1ex}\noindent
 Wie von der Erzeugung des Inhaltsverzeichnisses her bekannt, wo
 "Anderungen am "Uberschriftentext erst nach zweimaligem Start von \LaTeX\
 im Inhaltsverzeichnis zu sehen sind, so gilt auch f"ur \BibArts: Es
 sind zwei Bearbeitungen mit \LaTeX\ n"otig. Aber au"serdem muss nun zwischen 
 den beiden Bearbeitungen das Programm \verb|bibsort| gestartet
 werden, um die Literaturliste zu sortieren. Hei"st eine 
 \LaTeX\hy Textdatei \verb|meintext.tex|, ist typischerweise
 
 \vspace{-.5ex}
 \begin{verbatim} 
     bibsort -g1 -k  meintext
 \end{verbatim}

 \vspace{-3.25ex}\noindent
 in die Kommandozeile einzugeben. Dann liest \verb|bibsort| die mit
 \verb|\vli|-Li"-teraturangaben bef"ullte Datei \verb|meintext.aux| ein, 
 sortiert diese Vollzitate "uber die Option
 \verb|-g1| nach deutschen Sortierregeln und legt das Ergebnis ab in einer
 Datei \verb|meintext.vli|, die im Text mit \verb|\printvli| bei der
 zweiten \LaTeX\hy "Ubersetzung ausgedruckt werden kann. Die zus"atzlich 
 gesetzte Option \verb|-k| \hspace{.3em}sorgt
 daf"ur, dass \textit{ein} Autor (Vor- und Nachname gleich), der mit mehreren 
 Werken zitiert wird, ab seiner zweiten Nennung auf der Literaturliste 
 als $\sim$ erscheint.

 \vspace{1ex}
 Viele Texteditoren k"onnen den Start automatisieren. Falls sich die
 Datei \verb|bibsort.exe| im Verzeichnis \verb|C:\pfadangabe| befindet, dann
 gilt beispielsweise f"ur das \TeX nicCenter: Der automatisierte Start von
 \verb|bibsort| erfolgt durch Eingabe von \verb|C:\pfadangabe\bibsort.exe|
 in das Men"u \fbox{Ausgabe} $\Rightarrow$ \fbox{Ausgabeprofile definieren}
 $\Rightarrow$ \fbox{Vorbearbeitung} in die Zeile \underline{Anwendung} und
 \\[1ex]\verb|-i %tm -g1 -k|\hspace{.5em} 
 hinter \underline{Argumente} darunter.\hspace{.2em} (Eventuell versionsabh"angig.)

 \vspace{2.75ex}\noindent
 Nochmal zum \verb|\ktit|\hy Befehl: Dessen Verwendung im \verb|\vli|\hy Befehl 
 macht \textit{zus"atzlich} den Ausdruck eines Kurzzitate\hy Verzeichnisses mittels 
 \verb|\printnumvkc| m"oglich. \textsc{Wehler} fehlt nat"urlich; 
 aber vgl.\ \textsc{Luhmann} \baref[von]{luhmann}:

 \vspace{1.5ex}\label{vkc}{\footnotesize \batwocolitemdefs\printnumvkclist}

 \newpage
 \noindent
 Von den Vollzitaten wurden diejenigen, deren Kurztitel mit \verb|\ktit| markiert 
 war, ausgedruckt als Kurzzitat (Nachname plus Kurztitel). Um \verb|\printvli| zu 
 nutzen, m"ussen Sie \verb|\ktit| also nicht verwenden $-$ aber dazu,
 \verb|\printnumvkc| zu bef"ullen. Hinzu kamen noch weitere echte Kurzzitate; 
 dazu gleich unten.

 Anwender, die das [L] $-$\,hei"st: \textit{Volltitel findet sich auf der Literaturliste}\,$-$ nicht 
 wollen, k"onnen im Vorspann \verb|\notprinthints| setzen; das unterdr"uckt den Ausdruck 
 von [L] und [Q] \,(\kern-.1em\textit{Volltitel im Verzeichnis gedruckter Quellen}).



\section{Vollzitate und Kurzzitate}\label{Sect1}

 Nachdem ein Buch einmal vollzitiert wurde, kann es anschlie"send an
 weiteren Belegstellen kurzzitiert werden. Zur formatierten Eingabe von
 Literatur dienen in \BibArts\ f"ur Vollzitate die Befehle \verb|\vli| f"ur
 Literatur und \verb|\vqu| f"ur gedruckte Quellen (Quelleneditionen); f"ur
 Kurzzitate dienen \verb|\kli| und \verb|\kqu|. Letztere haben jeweils ein
 Argument weniger als die Vollangaben, weil das Vornamensargument im
 Kurzzitat wegf"allt. Der Titel wird im Kurzzitat als Kurztitel angegeben.
 Falls dieser Kurztitel im Vollzitat bereits mit \verb|\ktit| markiert
 wurde, kann \BibArts\ mitkontrollierten, ob kurzzitierte Literatur weiter
 oben in einem \LaTeX\hy Text bereits eingef"uhrt wurde. Dieser Aufgabe kommt
 das Sortierprogramm \verb|bibsort| nach, indem es Warnungen auf den
 Bildschirm ausgibt. Ohne vorausgehendes Vollzitat f"uhrt Kurzzitieren der
 Quellenedition \verb|\kqu{Clausewitz}{Vom Kriege}| zu der
 \verb|bibsort|-Warnung:

  {\scriptsize\begin{verbatim}
   %%>   Info: Short-qu-title file 1 line 143 is NOT yet introduced.
   %%      (Clausewitz)  (Vom Kriege)  
   %%       ...  Change that short-title into missing full-title (\ktit)?
  \end{verbatim}}

 \vspace{-2ex}\noindent 
 Die Kontrolle f"uhrt \BibArts\ mithilfe der Daten f"ur die \verb|.vkc|\hy Datei durch. 
 Die enth"alt alle Kurzzitate, also sowohl \verb|\ktit| in \underline{v\ko}\hy 
 Belegen als auch alle \underline{k}\hy Belege\pdfko{.25}\ 
 (\textit{\underline{c}ites}). 
 Literatur und gedruckte Quellen werden darin also parallel behandelt. Die 
 \verb|.vkc|\hy Datei wurde oben durch den Befehl \verb|\printnumvkc| ausgedruckt.

 \vspace{1ex}\noindent
 Erfolgt irrt"umlich \textit{erst} das Kurzzitat und
 \textit{weiter unten} das Vollzitat ...

\Doppelbox
{...\bs footnote\{\bs kqu\{Clausewitz\}
 \\ \hspace{2em} \{Vom Kriege\}.\} 
 \ Aber ... \\
 ...\bs footnote \b{\b{\{}}Siehe dazu weiter 
 \\ \hspace{.5em} \bs vqu \{Carl von\}
 \{Clausewitz\} 
 \\ \hspace{2.5em} \b{\{}\bs ktit\{Vom Kriege\}.
 \\ \hspace{3em} Hinterlassenes Werk, 
 3.\bs,Auf\string"|l. Frankfurt/M 1991\b{\}}.\b{\b{\}}}
}
{...\footnote{\kqu{Clausewitz} {Vom Kriege}.}
 Aber ...
 ...\footnote{Siehe dazu weiter \vqu {Carl von}{Clausewitz} {\ktit{Vom Kriege}.
 Hinterlassenes Werk, 3.\,Auf"|l.\ Frankfurt/M 1991}.}
}

 \noindent
 ... dann warnt \verb|bibsort| danach auf dem
 Bildschirm etwa (siehe Folgeseite):

 \vspace{-.75ex}
  {\scriptsize\begin{verbatim} 
   %%>   Info: Short-qu-title file 1 line 193 is NOT yet introduced.
   %%      (Clausewitz)  (Vom Kriege)  
   %%       ...  Exchange it with the full-title in file 1 line 196.
  \end{verbatim}} %% Zeilenangaben nur ausserhalb \Doppelbox brauchbar %%

 \vspace{-2.75ex}\noindent 
 Und mehrfaches identisches Vollzitieren ... \footnote{\vqu{Carl
 von}{Clausewitz}{\ktit{Vom Kriege}. Hinterlassenes Werk, 3.\,Auf"|l.\
 Frankfurt/M 1991}.} ... ergibt die \verb|bibsort|-Warnung:

  \vspace{-.75ex}
  {\scriptsize\begin{verbatim} 
   %%>   Info: Identical full-titles file 1 line 196, file 1 line 213:
   %%  !"  (Carl von)  (Clausewitz)  (\ktit {Vom Kriege}. Hinterlassenes W....)  
   %%
   %%>   This have been the first 2 (of 2) identical full-titles.
  \end{verbatim}}

 \vspace{-2.75ex}\noindent 
 \verb|!"| bedeutet, dass der Eintrag \verb|"| enth"alt \textit{und} 
 von einem Bereich herstammt, an der die \verb|"| \textit{aktiv} waren (hier: \verb+Auf"|l.+); 
 \texttt{bibsort} sieht etwa in \verb|"a| ein "a. 
 
 Falls Sie einen Autor im Text in direkt aufeinanderfolgenden Fu"snoten mit zwei Schriften 
 zitieren, gibt \BibArts\ ins \LaTeX\hy\ko\verb|.log|\hy File die Warnung aus:%
 \balabel{DERS} 

  \vspace{-.75ex}
  {\footnotesize\begin{verbatim} 
  BibArts Warning: ...vqu-cmd repeats (first) author's lastname 
     on input line 180. `{Clausewitz}'. Change to `...vqu[m,f,p]'?
  \end{verbatim}}

 \vspace{-3.25ex}\noindent 
 Diese Warnung verweist darauf, dass der Autorenname
 bei der zweiten Nennung in der direkt folgenden Fu"snote ersetzt 
 werden soll durch den in diesem Fall "ublichen Hinweis, dass er derselbe 
 ist (\textsc{ders.}). Den entsprechenden Schalter m"ussen Sie
 selbst umlegen und dabei das Geschlecht des Autors einstellen. 
 Die Schalter lassen sich zusammen mit allen v- und k-Befehlen 
 verwenden. Verf"ugbare Schalter sind
 \verb|f| (weiblich), \verb|m| (m"annlich) und \verb|p{}| (plural; S.\,\pageref{p}):

 \vspace{1.ex}{\small\noindent
\verb|  \footnote{\vqu [m] {Carl von}{Clausewitz}{\ktit{Strategie}.|\\
\verb|    Hrsg. von \vauthor{Eberhard}{Kessel}, Hamburg 1937}[58].}  => |}
{\renewcommand{\thefootnote}{{\bfseries\arabic{footnote}}}%
 \footnote{\vqu [m] {Carl von}{Clausewitz}{\ktit{Strategie}. 
 Hrsg.\ von \vauthor{Eberhard}{Kessel}, Hamburg 1937}[58].
 \hspace{.8em} $\Leftarrow$ \ \texttt{... \{Clausewitz\} \{\bs ktit\{Strategie\}. ...\}[58].}}%
}

\vspace{1.ex}\noindent 
Dabei wurde zudem eine Seitenangabe (...\texttt{\underline{\}[}58]} 
\textit{ohne} \underline{Leerzeichen}) gemacht. Falls ein folgendes Kurzzitat auf 
dieselbe Seite der Quellenedition verweist, ergibt sich ... 
{\renewcommand{\thefootnote}{{\bfseries\itshape\arabic{footnote}}}%
 \footnote{\kqu{Clausewitz}{Strategie}[58].\label{ErsterFall} 
 \hspace{5.95em} $\Leftarrow$ \ \texttt{\bs kqu\{Clausewitz\}\{Strategie\}[58].}}
} ... w"ahrend eine andere Seite (\verb|[60]|) gedruckt wird 
als~...~\footnote{\kqu{Clausewitz}{Strategie}[60]. 
\hspace{3.05em} $\Leftarrow$ \ \texttt{\bs kqu\{Clausewitz\}\{Strategie\}[60].}}

\BibArts\ druckte in Fu"snote\,\ref{ErsterFall} nur den Abk"urzungspunkt von \textsc{ebd.}, 
nicht aber den direkt folgenden Punkt am Satzende. Dies funktioniert nur, 
wenn zwischen \verb|[|\textit{Seitenzahl}\verb|]| und \verb|.|
\textit{keine Klammern oder Leerzeichen} stehen ... \footnote{\textit{\kqu{Clausewitz}{Strategie}[60]}.
\hspace{5.68em} $\Leftarrow$ \ \texttt{\bs textit\b{\{}\bs kqu\{Clausewitz\}\{Strategie\}[60]\b{\}}.
\texttt{\%Fehler}}}!

Das automatischen Ebenda-Setzen f"uhrt \BibArts\ in einer Fu"snote nicht durch, 
wenn in der vorausgehenden Fu"snote zwei verschiedene Werke angegeben sind 
(weil dies nicht eindeutig w"are): ...
\footnote{\kqu[m]{Clausewitz}{Strategie}[60] und \kqu[m]{Clausewitz}{Vom Kriege}.
\hspace{3em} \texttt{\%\% Ein Autor mit zwei Werken. \%\%}}
$\leftarrow$
\footnote{\kqu[m]{Clausewitz}{Strategie}[12]. 
\hspace{3.85em} $\Leftarrow$ \ \texttt{\bs kqu[m]\{Clausewitz\}\{Strategie\}[12].}}
... Mit \verb|\notibidemize| l"asst sich das automatische Ebenda-Setzen ausschalten
(nicht demonstriert).

In jedem Fall ist sinnvoll, im letzten Argument jedes v\fhy Befehls 
einen Teil des Volltitels mittels \verb|\ktit| als Kurztitel zu
markieren. \BibArts\ erlaubt aber\pdfko{1}\ 
dennoch das Weglassen von \verb|\ktit|, und zwar, (1)~um
Anf"anger nicht abzuschrecken, die bei ihren ersten Texten mit \BibArts\
noch nicht die k\fhy Befehle\pdfko{.5}\ 
nutzen wollen, sondern sich "uber die v-Befehle nur
ein Literaturverzeichnis (und evtl.\ ein Verzeichnis von Quelleneditionen)
automatisiert erzeugen lassen m"ochten; (2)~um dem Tippen von Texten nicht im
Wege zu stehen, in denen alle Fu"snotenbelege mittels Vollzitaten
gemacht werden sollen. 

Das Weglassen von \verb|\ktit| ist aber keine gute Methode, im Ausdruck von\pdfko{.25}\
Vollzitaten die 'im Folgenden ...'\hy Ank"undigung (wie nachfolgend 
kurzzitiert werden wird) auszuschalten.\footnote{\texttt{bibsort}
w"urde am ersten Kurzzitat vorschlagen, \textit{dieses} in ein Vollzitat zu
verwandeln.} Dazu dient \verb|\notannouncektit|. 
Es sollte besser nur im Dokumentenvorspann stehen. Das Beispiel zeigt, 
wie es lokal, also gemeinsam mit dem v\fhy Befehl geklammert, zu setzen w"are
(etwas riskant\footnote{Ein lokales Setzen eines v\fhy Befehls unter 
\texttt{\bs notannouncektit} scheint naheliegend, wenn ein Werk nur 
einmal pro Text angef"uhrt wird; falls Sie \textit{dasselbe} Werk 
sp"ater aber doch kurzzitieren, macht \BibArts\ keine Meldung, dass
\textit{der} Kurztitel nicht vorangek"undigt wurde.}):

\Doppelbox
{... \bs vli\{Niklas\}\{Luhmann\} 
  \\ \{\bs ktit\{Soziale Systeme\}. 
  \\ \ Grundri"s einer allgemeinen 
        \\ \ Theorie, 1984: Frankfurt/M.\}
  \\[1ex] 
  Unannonciert:
  \b{\b{\{}}\bs notannouncektit 
    \bs vli\{Niklas\}\{Luhmann\} \b{\{}\bs ktit\{Soziale Systeme\}. 
   \\ \ Grundri"s einer allgemeinen 
   \\ \ Theorie, 1984: Frankfurt/M.\b{\}}\b{\b{\}}}
   \\[.25ex] \%\% Zu \}. siehe unten S.\pageref{v-Ausnahme}! \%\%
}
{\vspace{.375ex}
 Annonciert: \vli{Niklas}{Luhmann} {\ktit{Soziale Systeme}. 
      Grundri"s einer allgemeinen Theorie, 1984: Frankfurt/M.}
  \\[3.875ex] Unannonciert:
  {\notannouncektit
   \vli{Niklas}{Luhmann} {\ktit{Soziale Systeme}. 
        Grundri"s einer allgemeinen 
        Theorie, 1984: Frankfurt/M.}}
}

\vspace{.75ex}\noindent
Beide \verb|\vli|-Befehle ergeben $-$\,weil beides mal \verb|\ktit| ja gesetzt
ist\,$-$ jeweils auch Eintr"age in die \verb|.vkc|-Datei (die {\small\balistnumemph\thepage} 
in der Liste S.\,\pageref{vkc} hinter {\small\printonlykli{Luhmann}{Soziale Systeme}}). 
v\fhy Befehle mit \verb|\ktit| verhalten sich also so, als st"unde ein unsichtbarer
k\fhy Befehl direkt hinter ihnen, der einen Eintrag ins
Kurzzitateverzeichnis macht. Dort in die \verb|.vkc|-Datei hinein
kommen auch Kopien aller weiteren \verb|\kli|- und \verb|\kqu|\hy Befehle. 
F"ur Historiker dient das Kurzzitateverzeichnis zus"atzlich zur Kontrolle,
ob \textit{ein} bestimmtes Werk stets als Literatur deklariert wurde (und in
sp"ateren Kurzzitaten niemals irrt"umlich als Quelle). \verb|\printvkc|
listet die verwendeten Kurztitel auf, \verb|\printnumvkc| druckt zudem
indexartig alle Seiten und ggf.\ dazu Fu"snotennummern der Zug"ange aus.

Das automatische \textsc{ebd.}-Setzen f"uhrt \BibArts\ "ubrigens nur von 
Fu"snote zu Fu"snote durch, nicht im Haupttext. Nie umgewandelt 
werden v\fhy Befehle.

Auch im speziellen Fall des Texttyps \textbf{Aufsatz} $-$\,der hat im Unterschied
zu einem Buch keine Literaturliste\,$-$ ist die Verwendung von \verb|\ktit|
sinnvoll. In solchen Texten ist es n"amlich w"unschenswert, beim Kurzzitat
Querverweise auf das Vollzitat zu setzen, um alle bibliographischen Angaben zu
finden. \BibArts\ bietet deshalb optional an, von jedem v-Befehl
(Vollzitat) eine Marke aus Autornachname und Kurztitel erzeugen zu lassen,
damit zugeh"orige k\fhy Befehle automatisiert einen Querverweis drucken k"onnen. 
Das Einschalten dieses Aufsatz\hy Modus erfolgt mit dem \BibArts-Befehl
\verb|\conferize|. Der sollte global gelten, also im Vorspann von
\LaTeX\hy Textdateien gesetzt werden.\footnote{Dies hat nichts
damit zu tun, ob f"ur den \LaTeX-Text der Dokumentenstil \texttt{\{article\}}\pdfko{.5}\ 
oder \texttt{\{book\}} gew"ahlt wird. Vielmehr sind die Auswahl des 
Dokumentenstils und das Setzen von \texttt{\bs conferize} zwei voneinander 
unabh"angige Entscheidungen. -- Studentische Hausarbeiten werden zwar oft
als Aufs"atze bezeichnet, sollen aber meist eine Literaturliste haben.} 
Ein Blick auf die Fu"snoten \ref{vz} und \ref{fz} 
im Kurzzitateverzeichnis (S.\,\pageref{vkc}) belegt, dass es sich auch im
\verb|\conferize|\hy Modus zu Kontrollzwecken ausdrucken l"asst.

\Doppelbox
{\bs conferize\ ...\bs footnote\b{\{}
   \\ \ \ Vollzitat: \bs vli\{Niklas\} 
   \\ \ \ \{Luhmann\} \{\bs ktit\{Soziale 
   \\ \ \ \ \ \ Systeme\}. Grundri"s einer 
   \\ \ \ \ \ allgemeinen Theorie, 
   \\ \ \ \ \ 1984: Frankfurt/M.\}.\b{\}} 
   \\[.5ex] ...\bs footnote\{\} \ \% kein Ebd.
   \\[.5ex] ...\bs footnote\H{\{}Kurzzitat: 
   \\ \ \ \bs kli\{Luhmann\} \ \ \{Soziale 
   \\ \ \ \ \ \ Systeme\}[23\bs f].\H{\}}
}
{\conferize ...\footnote{\label{vz}%
                  Vollzitat: \vli{Niklas}{Luhmann} {\ktit{Soziale Systeme}. 
     Grundri"s einer allgemeinen Theorie, 1984: Frankfurt/M.}.} 
  ...\footnote{}
  ...\footnote{\label{fz}Kurzzitat: \kli{Luhmann} {Soziale Systeme}[23\f].}
}


\noindent
Im Programmcode von \verb|bibarts.sty| wurde gro"ser Aufwand damit
betrieben, dass dies immer funktioniert, also auch dann, wenn sich
\LaTeX\hy Befehle in den Argumenten der 
v- und k\fhy Befehle befinden. Dazu durchsucht \BibArts\ sie und kopiert 
nur bestimmte Teile von \textit{Nachname} und \textit{Kurztitel} in das 
automatisch erzeugte \textit{Schl"usselwort} f"ur die Marke (vorz"uglich 
die Buchstaben).\footnote{\BibArts\ versucht weiter, gleiche Buchstaben 
mit verschiedenen Akzenten zu unterscheiden; das funktioniert aber nicht 
mit allen \LaTeX\hy Applikationen. \BibArts\ bildet das Schl"usselwort 
der v- und k\fhy Marke jedenfalls gleich. Hier mit \texttt{ngerman.sty} und
der Notation \texttt{\string"u}\pdfko{1.5}\ f"ur "u w"urde aus \texttt{\bs vli 
\{Peter\} \{M\string"uller\} \{Die \bs ktit\{Reise\}, Verlagsstadt 2002\}} 
im \texttt{.aux}\hy File die Marke 
\texttt{\bs newlabel\{baf.M*uller..Reise\}\{\{{\normalfont 
\textit{Fu"snote}}\}\{{\normalfont \textit{Seite}}\}\}} erzeugt.}

Sorgfalt erfordert mehr das automatische \textsc{ebd.}\hy Setzen, 
das im Buch- und im Aufsatz\hy Modus arbeitet. \BibArts\ erkennt zwei 
Argumente nur als gleich an, wenn sie zeichengleich sind. Falls Sie 
einen Namensteil mit \verb|\underline| unterstreichen und davor (wie
bei 'zerbrechlichen' Befehlen ja immer n"otig) im v\fhy Befehl 
\verb|\protect| setzen, das beim zugeh"origen k\fhy Befehl aber mal 
vergessen, gibt es zwei Eintr"age in der \verb|.vkc|-Liste und 
kein \textsc{ebd.}\hy Setzen.

Die ohnehin f"ur geisteswissenschaftliche Texte g"ultige Spielregel, dass 
jedes Kurzzitat aus Autorennachname plus Kurztitel ein bestimmtes Werk 
eindeutig bezeichnen muss, schlie"st aus, dass eine Marke absichtlich 
zweimal vorkommt (Kurztitel d"urfen sogar mehrfach gleich
sein, wenn sich nur die Nachnamen unterscheiden). Trotzdem steht 
das Befehlspaar \verb|\balabel| und\pdfko{.75}\ 
\verb|\baref| bereit, um 'von Hand' Marken setzen zu k"onnen, 
wie k\fhy Befehle es\pdfko{.25}\ 
im \verb|\conferize|\hy Stil tun (sie merken selbst, ob sie in einer 
Fu"snote stehen). In den 'von Hand' zu tippenden Schl"usselworten in den
Argumenten von \verb|\balabel| und \verb|\baref| sind Sonderzeichen 
allerdings verboten.

\Doppelbox
{M"uller \bs balabel\{Mueller\} im 
 Text.\bs footnote\b{\{}Maier in 
 \\ \ Fu"snote.\bs balabel\{Maier\}\b{\}}
 \\ ... M"uller ist nochmal erw"ahnt \bs baref\{Mueller\} 
 und Maier ebenfalls \bs baref[vgl.]\{Maier\}.
}
{M"uller \balabel{Mueller} im 
 Text.\footnote{Maier in 
 Fu"snote.\balabel{Maier}}
 ... M"uller ist nochmal erw"ahnt \baref{Mueller} 
 und Maier ebenfalls \baref[vgl.]{Maier}.
}

\noindent
Das \texttt{[}\textit{OptionalArg}\texttt{]} "uberschreibt 'siehe'
(\verb|\grefverbname|; vgl.\ unten S.\,\pageref{grefverbname}). 


\vspace{2ex}\noindent
Zur"uck zu den v\fhy Befehlen. Bei der Auswahl eines Kurztitels aus
dem Volltitel mit \verb|\ktit| wird es gelegentlich so sein, dass ein im
Volltitel klein geschriebenes Wort ausgew"ahlt werden soll. \BibArts\
erkennt die Verbindung mit sp"ater in k\fhy Befehlen gro"sgeschriebenen 
Kurztiteln mittels \verb|\onlyvoll| und \verb|\onlykurz|:

\label{Ferguson}%
\Doppelbox
{...\bs footnote\{\bs vli\{Niall\}
  \\ \ \{Ferguson\} \b{\b{\{}}Der
  \\ \ \bs ktit\b{\{}\bs onlykurz\{F\}\%
  \\ \ \ \bs onlyvoll\{f\}alsche\bs onlykurz\{r\} 
  \\ \ \ Krieg\b{\}}, M"unchen 2001\b{\b{\}}}[22].\}
 ...\bs footnote\{\bs kli\{Ferguson\}
   \\ \ \{Falscher Krieg\}[23].\}
 ...\bs footnote\{\bs clearbamem 
 \\ \ \bs kli\{Ferguson\}\{Falscher Krieg\}.\} 
}
{...\footnote{\vli{Niall}
      {Ferguson} {Der
     \ktit{\onlykurz{F}%
                 \onlyvoll{f}alsche\onlykurz{r} Krieg},
            M"unchen 2001}[22].}
 ...\footnote{\kli{Ferguson}
      {Falscher Krieg}[23].}
 ...\footnote{\clearbamem \kli{Ferguson}{Falscher Krieg}.\label{clearbamem}} 
}

\vspace{.5ex}\noindent 
Die Fu"snote\,\ref{clearbamem} soll sich auf das ganze 
Werk beziehen; nur \textsc{ebd.} wieder mittels \verb|\kli{Ferguson}{Falscher Krieg}[23]|
zu erzeugen, w"are falsch. Statt dessen l"oschte \verb|\clearbamem| die 
Zwischenspeicher. Sonst h"atte der \verb|\kli|-Befehl ohne \verb|[|\textit{Seite}\verb|]|
bei der "Ubersetzung mit \LaTeX\ diese Fehlermeldung ausgel"ost:

\vspace{0.5ex}\label{before}
{\scriptsize\begin{verbatim}
  ! Same title, before with :{p}{23}:, has now no page/folio number.
    . . . . . . . . . . . .
  \errmessage@ba ...
   \space . . . . . . . . . . . }
                                                  }
  l.461 \footnote{\kli{Ferguson}{Falscher Krieg}.}
                                                  }
\end{verbatim}}

Neben Monografien gibt es noch B"ucher, die aus mehreren Aufs"atzen bestehen. 
Es ist genug, auch \textbf{Herausgeberwerke} nur einmal vollzuzitieren. Bei der 
Ersteinf"uhrung des \textit{zweiten} Aufsatzes darf das Buch (im letzten 
Argument des '"au"seren' v\fhy Befehls) kurzzitiert sein, denn es ist ja schon 
bekannt. Es steht ein 'inneres' \textsc{ebd.}\hy Setzen an, falls Sie beide 
Aufs"atze in aufeinander folgenden Fu"snoten einf"uhren. \BibArts\ 
hat daf"ur eine zweite Speicherebene.\footnote{Falls Sie das Herausgeberwerk 
sp"ater \textit{eigenst"andig} kurzzitieren wollen (wie unten S.\,\pageref{MEhlert}:
also nicht im letzten Argument eines v- oder k\fhy Befehls) 
\textit{und} dort \textsc{[Hrsg.]} nicht mehr setzen wollen, dann m"ussen 
Sie hier \textsc{[Hrsg.]} mit \texttt{\bs onlyvoll} im inneren v\fhy Befehl und mit 
\texttt{\bs vollout} im inneren k\fhy Befehl maskieren 
(Leerzeichen so: \texttt{\{Gro"s\}\bs onlyvoll\{ [Hrsg.]\}}).}

\vspace{.75ex}
\Doppelbox
{...\bs footnote\{Innen vollzitiert: 
 \\ \bs vqu \{\} \{\} 
 \\ \b{\b{\{}}\bs ktit\{Aufmarschanweisungen 
 \\ \ \ 1912\}, abgedruckt in: 
 \\ \ \bs xvqu\{Hans\} \{Ehlert\} 
 \\{} \ \ *\b{\{}\bs midvauthor\{Michael\}
 \\ \ \ \ \ \ \{Epkenhans\} 
 \\ \ \ \ \ \bs vauthor\{Gerhard P.\} 
 \\ \ \ \ \ \ \{Gro"s\} [Hrsg.]\b{\}} 
 \\ \ \ \{Der \bs ktit\{Schlieffenplan\},
 \\ \ \ \ Paderborn 
 \\ \ \ \ 2007\}[462-466]\b{\b{\}}}*[463].\}
 \\ \
 \\ ...\bs footnote\{Innen kurz: 
 \\ \bs vqu \{\} \{\} 
 \\ \H{\{}\bs ktit\{Aufmarsch 1913/14\},
 \\ \ abgedruckt in: 
 \\ \ \bs xkqu\{Ehlert\}
 \\{}\ \ *\{\bs midkauthor\{Epkenhans\} 
 \\ \ \ \ \ \bs kauthor\{Gro"s\} [Hrsg.]\}
 \\ \ \ \{Schlieffenplan\%
 \\ \ \ \}[467-477]\H{\}}*[469].\} 
}
{\vspace{1ex}
 \fbox{\parbox{.95\textwidth}{Siehe \texttt{.vkc}-Eintr"age oben S.\,\pageref{vkc}: 
 \\[1ex] \scshape[Anonym]\\{}[Anonym]\\{}[...]\\Ehlert\baslash Epkenhans\baslash Gro"s}}\\
 \\[1ex] ...\footnote{Innen vollzitiert: \vqu {} {} 
   {\ktit{Aufmarschanweisungen 1912}, 
   abgedruckt in: \xvqu{Hans} {Ehlert}
   *{\midvauthor{Michael} {Epkenhans}
     \vauthor{Gerhard P.} {Gro"s} [Hrsg.]} 
                {Der \ktit{Schlieffenplan},
     Paderborn 2007}[462-466]}*[463].}
                
 ...\footnote{Innen kurz: \vqu {} {} 
    {\ktit{\onlyhere{\onlykurz{\,}}Aufmarsch 1913/14},
    abgedruckt in: \xkqu{Ehlert}
    *{\midkauthor{Epkenhans} \kauthor{Gro"s} [Hrsg.]}
    {Schlieffenplan%
      }[467-477]}*[469].}
}


\vspace{.75ex}\noindent
\verb|*[463]| und \verb|*[469]| ergeben 'dort: S.' zur Bezeichnung der 
zitierten Einzelseite innerhalb des zuvor genannten Seitenbereichs des 
Teiltextes. Vor \verb|*[| darf\pdfko{1}\ kein Leerzeichen stehen; vor 
\verb|[462-466]| und \verb|[467-477]| auch nicht. Setzen\pdfko{.5}\ 
von runden statt eckigen Klammern w"urde Bl.\ statt S.\ ausdrucken.

Das Beispiel f"uhrte zudem das 'Sternargument' ein, das in allen v- 
und k\hy Befehlen nach dem Nachnamensargument \verb|*{|\textit{optional}\verb|}| 
stehen darf, um Koautoren aufzunehmen. In v\fhy Befehlen sind vauthor\hy 
Formatierer und in k\fhy Befehlen kauthor\fhy Formatierer zu verwenden. 
\BibArts\ setzt \textsc{ebd.} nur dann, wenn\pdfko{.5}\  
\textit{auch} gleiche Nachnamen in den vauthor\hy\ und kauthor\hy 
Formatierern stehen. 

Dabei benennen \verb|\vauthor| und \verb|\kauthor| stets den letzten von 
jeweils mehreren Autoren. Falls $-$\,wie oben\,$-$ mehr als zwei
Autoren genannt werden, sind alle davor im Sternargument mit 
\verb|\midvauthor| bzw.\ \verb|\midkauthor| zu kennzeichnen. Die setzen 
Schr"agstriche nach dem Nachnamen. Der Schr"agstrich nach dem Erstautor 
wird von x\fhy Befehlen erzeugt. Im letzten Beispiel waren das \verb|\xvqu| 
und \verb|\xkqu|, bei Literatur sind es \verb|\xvli| und 
\verb|\xkli|. 

Auch 'normale' v- und k\fhy Befehlen d"urfen Sternargumente haben.
Nach \verb|\vli| und \verb|\vqu| k"onnen sie Attribute wie 
\verb|*{\onlyvoll{[Hrsg.]}}| aufnehmen. Das Sternargument 
des v\fhy Befehls ist hier \textit{komplett} mit \verb|\onlyvoll| maskiert, sodass sp"atere 
k\fhy Befehle kein Sternargument brauchen (\textsc{ebd.}\hy Setzung).
Statt \verb|[Hrsg.]| k"onnte auch \verb|\editor| verwendet werden
(vgl.\ unten S.\,\pageref{editor}):

\vspace{-.25ex}
\Doppelbox
{
\vspace{-.3ex}
...\bs footnote\{\bs vli\{Peter\}\{Maier\} 
\\[.4ex] \ \ \ \ \ \ *\b{\b{\{}}\bs onlyvoll\{[Hrsg.]\}\b{\b{\}}} 
\\[.3ex] \ \ \ \ \ \ \b{\{}Das \bs ktit\{Buch\}\b{\}}.\} 
\\[.7ex] Go!\bs footnote\{\bs kli\{Maier\}\{Buch\}.\} 
}
{Nicht in Listen "ubernommen.\footnote{\notktitaddtok
\printonlyvli{Peter}{Maier} *{\onlyvoll{[Hrsg.]}} {Das \ktit{Buch}}.}
Go!\footnote{\printonlykli{Maier}{Buch}.}
}


\vfill\noindent
Zur"uck zum 'inneren' \textsc{ebd.}\hy Setzen. Beachten Sie die Fu"snote
in der Mitte:

\vspace{.5ex}
\Doppelbox
{\scriptsize
 ...\bs footnote\{Innen vollzitiert: 
 \\ \bs vqu \{\} \{\} 
 \\ \b{\b{\{}}\bs ktit\{Aufmarschanweisungen 
 \\ \ \ 1912\}, abgedruckt in: 
 \\ \ \bs xvqu\{Hans\} \{Ehlert\} 
 \\{} \ \ *\b{\{}\bs midvauthor\{Michael\}
 \\ \ \ \ \ \ \{Epkenhans\} 
 \\ \ \ \ \ \bs vauthor\{Gerhard P.\} 
 \\ \ \ \ \ \ \{Gro"s\} [Hrsg.]\b{\}} 
 \\ \ \ \{Der \bs ktit\{Schlieffenplan\},
 \\ \ \ \ Paderborn 
 \\ \ \ \ 2007\}[462-466]\b{\b{\}}}*[463].\}
 \\[1ex] ...\bs footnote\{\bs kqu\{\}
 \\ \ \ \ \ \ \{Aufmarschanweisungen 
 \\ \ \ \ \ \ \ 1912\}[464].\}
 \\[1ex] ...\bs footnote\{Innen kurz: 
 \\ \bs vqu \{\} \{\} 
 \\ \H{\{}\bs ktit\{Aufmarsch 1913/14\},
 \\ \ abgedruckt in: 
 \\ \ \bs xkqu\{Ehlert\}
 \\{}\ \ *\{\bs midkauthor\{Epkenhans\} 
 \\ \ \ \ \ \bs kauthor\{Gro"s\} [Hrsg.]\}
 \\ \ \ \{Schlieffenplan\%
 \\ \ \ \}[467-477]\H{\}}*[469].\} 
}
{\vspace{9ex}\showbamem
 ...\footnote{Innen vollzitiert: \vqu {} {} 
   {\ktit{Aufmarschanweisungen 1912}, 
   abgedruckt in: \xvqu{Hans} {Ehlert}
   *{\midvauthor{Michael} {Epkenhans}
     \vauthor{Gerhard P.} {Gro"s} [Hrsg.]} 
                {Der \ktit{Schlieffenplan},
     Paderborn 2007}[462-466]}*[463].\label{InnVoll}}
                
   ...\footnote{\kqu{} {Aufmarschanweisungen 1912}[464].\label{zweite}}
   %%
   %\footnote{\kqu{}{Aufmarschanweisungen 1912!!!!!}[464].}    %% Aussen anders %%
   %%
   %\newbox\mybox
   %\setbox\mybox=\hbox{\footnotetext{\printonlyvqu{}{} {\xprintonlykqu{Ehlert} 
   %   *{\midkauthor{Epkenhans} \kauthor{Gro"s} [Hrsg.]}{Schlieffenplan}}}}

 ...\footnote{Innen kurz: \vqu {} {} 
    {\ktit{\onlyhere{\onlykurz{\,}}Aufmarsch 1913/14},
    abgedruckt in: \xkqu{Ehlert}
    *{\midkauthor{Epkenhans} \kauthor{Gro"s} [Hrsg.]}
    {Schlieffenplan%
      }[467-477]}*[469].\label{dritte}}
}


\vspace{.5ex}\noindent
Ein k\fhy Befehl, der den Eintrag der "au"seren Speicherebene wiederholt, 
l"asst die innere Ebene also unber"uhrt: In Fu"snote\,\ref{dritte} wurde 
ein inneres \textsc{ebd.} gesetzt. 

Die Zwischenspeicher lassen sich mit \verb|\showbamem| auch ansehen. 
Dies kann bei Problemen mit dem \textsc{ebd.}\hy Setzen
helfen.\footnote{\BibArts\ gibt der \LaTeX-\texttt{minipage}\hy Umgebung 
eigene Speicher; die \textsc{ebd.}\hy Setzung in \texttt{minipage}\hy
Fu"snoten erfolgt deshalb unabh"angig von Fu"snoten im "ubrigen Text.}
\BibArts\ druckt auf den Bildschirm aus (\verb|o-ref| bzw.\ \verb|i-ref| nennen 
dabei den f"ur \textsc{ebd.} gesuchten Inhalt):

{\scriptsize\vspace{3ex}\noindent
\verb|   FNT |\texttt{\ref{zweite}}
\vspace{-3ex}
\begin{verbatim}
   -- outer:  {qu}{}{}{Aufmarschanweisungen 1912} -- 
   ---------- inner:  {qu}{Ehlert}{\midkauthor {Epkenhans}
                        \kauthor {Gro"s} [Hrsg.]}{Schlieffenplan} -- 
   -- o-ref:  {qu}{}{}{Aufmarschanweisungen 1912} -- 
\end{verbatim}}


\vspace{-.5ex}\noindent
Nun h"atte in der mittleren Fu"snote alternativ auch ein ganz anderer Teil des gleichen
Herausgeberbandes kurzzitiert werden k"onnen. Rein logisch d"urfte in der letzten 
Fu"snote dann weiterhin \textsc{ebd.} stehen. Steht aber etwas anderes als 
\verb|\kqu{}{Aufmarschanweisungen 1912}| in der 
zweiten Fu"snote, unterbleibt ohne weitere Ma"snahmen das innere \textsc{ebd.}\hy Setzen in 
der dritten.\footnote{Um doch das innere \textsc{ebd.} zu kriegen:
\texttt{\bs newbox\bs mybox} im Vorspann und vor Fu"snote~\ref{dritte}:
\\[.5ex] \hspace*{.75em} \texttt{\bs setbox\bs mybox=\bs hbox\{\bs footnotetext\{\bs printonlyvqu\{\}\{\}
\\ \hspace*{1em} \{\bs xprintonlykqu\{Ehlert\} 
\\ \hspace*{1.5em} *\{\bs midkauthor\{Epkenhans\} \bs kauthor\{Gro"s\} [Hrsg.]\}
\\ \hspace*{1.5em} \{Schlieffenplan\}\}\}\}}}

\vspace{1.5ex}\noindent
Nun wird das Verzeichnis gedruckter Quellen mit \verb|\printnumvqu| gedruckt:


\balabel{printnumvqu}\printnumvqu

\noindent
Der Herausgeberband 
\textsc{Ehlert}, Hans\baslash""Michael \textsc{Epkenhans}\baslash""Gerhard P.\pdfko{.25}\  
\textsc{Gro"s} 
bekam auf der Liste einen \textit{eigenen} Volleintrag, den \BibArts\ automatisch\pdfko{1}\  
aus dem 'inneren' Vollzitat in Fu"snote~\ref{InnVoll} erzeugte 
(S.\,\pageref{InnVoll}). In den Listenpunkten "`Aufmarsch"' und 
"`Aufmarschanweisungen"' druckte \BibArts\ die 'inneren'\pdfko{1}\ 
Angaben dagegen als Kurzzitat. Damit \BibArts\ dort v\hy\ in k\fhy Angaben 
umwandeln kann, m"ussen Kurztitel in 'inneren' v\fhy Befehlen stets mit 
\verb|\ktit| markiert\pdfko{1.25}\  
sein (nur bei '"au"seren' v\fhy Befehlen macht \BibArts\ keine Fehlermeldung). 

Beim Ausdruck von v\fhy Listen ergeben Zug"ange, die auf v\fhy Befehle mit leeren 
Namensargumenten  (\verb|\vqu{}{}{...}|) zur"uckgehen, stets \printonlyvqu{}{}{...}\,.\pdfko{.75}\  
Und trotz \verb|bibsort -k| wird der zweite anonyme Autor nicht als $\sim$ gedruckt.\pdfko{.5}


\vspace{2ex}\noindent
Gelegentlich sollen Teile der Literaturangaben nur in der Liste erscheinen, jedoch 
nicht in der Fu"snote. Die Reihenangaben hier sind nur in obiger Liste:

\vspace{.75ex}
\Doppelbox
{...\bs footnote\{\bs vqu \{Karl\}\{Marx\}
 \\ \b{\b{\{}}Das \bs ktit\{Kapital\%
 \\ \ \bs onlyhere\{\string~I\}\}\%
 \\ \ \bs onlyout \{. Kritik der 
 \\ \ \ politischen "Okonomie, 
 \\ \ \ erster Band; das ist 
 \\ \ \ Bd.\bs,23 (1962) von:\}%
 \\ \ \bs onlyhere\{, in:\}
 \\ \ \bs xvqu [m]\{Karl\}\{Marx\}
 \\ \ \ *\{\bs vauthor\{Friedrich\}\{Engels\}\}
 \\ \ \ \b{\{}\bs ktit\{Werke\}, 
 \\ \ \ \ \bs onlyout \{hrsg. vom Institut 
 \\ \ \ \ \ f"ur Marxismus-Leninismus 
 \\ \ \ \ \ beim ZK der SED, 40\string~Bde.
 \\ \ \ \ \ Berlin 1958--1971\}\%
 \\ \ \ \ \bs onlyhere\{Berlin 1962\}\b{\}}\b{\b{\}}}[49].\}
 \\ \
 \\ ...\bs footnote\{\bs kqu\{Marx\} 
 \\ \ \ \{Kapital\bs onlyhere\{\string~I\}\}[49].\}
}
{...\footnote{\vqu {Karl}{Marx}
 {\onlyhere{~}Das \ktit{Kapital%
   \onlyhere{~I}}%
   \onlyout {.
   Kritik der politischen "Okonomie, erster Band;
   das ist Bd.\,23 (1962) von:}%
          \onlyhere{, in:}
    \xvqu [m]{Karl}{Marx}
     *{\vauthor{Friedrich}{Engels}}
     {\ktit{Werke}, \onlyout {hrsg.\ vom Institut f"ur
              Marxismus-Leninismus beim ZK der SED, 40~Bde.\
      Berlin 1958--1971}%
                        \onlyhere{Berlin 1962}}}[49].\balabel{ders}}
                        
 ...\footnote{\kqu{Marx} {Kapital\onlyhere{~I}}[49].\label{I}}
}

\vspace{.75ex}\noindent 
Das Argument von \verb|\onlyhere| wird nur in
Haupttext oder Fu"snote, das Argument von \verb|\onlyout|
nur in den Listen ausgedruckt. Im Beispiel steht von der
"au"seren Angabe die Nummer~("`I"') in der Fu"snote und statt
dessen eine genauere Angabe zum Band auf der Liste. Von
der inneren Angabe wurde die Institution der Herausgeber 
nur auf der Liste (ganz unten) ausgedruckt.\kern-1pt\footnote{Falls 
Fu"snote~\ref{I} \texttt{\bs kqu\{Marx\}\{Kapital\string~I\}}
enthielte, w"urde dort auch \textsc{ebd.} gesetzt; Ziel war
aber, im Kurzzitateverzeichnis S.\,\pageref{vkc} nur 
\textit{einen} Eintrag \textsc{Marx}: Kapital [Q] f"ur alle
Teilb"ande des "`Kapital"' zu bekommen.}\pdfko{.5}

Ein Vergleich mit der Liste auf der Vorseite zeigt, dass das 
\verb|[m]| nach dem\pdfko{1}\ 
inneren \verb|\xvqu|\hy Befehl 
\textsc{ders.\baslash Engels} erzeugte \baref[Eintrag von]{ders}.

Neben der Markierung von Text in \BibArts\hy Befehlen mit 
\verb|\onlyhere| und\pdfko{1}\ 
\verb|\onlyout| gibt es eine zweite M"oglichkeit, 
unterschiedliche Eintr"age in Text und Liste zu erzeugen: \BibArts\hy 
Hauptbefehle (S.\,\pageref{Hauptbefehle}) lassen sich aufsplitten in 
eine printonly- und eine addto\hy Komponente, also in die 
Aufgabenteile 'Schreibe an Ort und Stelle' und 'Schreibe in die Liste'. 
\label{printonly} \verb|\vqu| beispielsweise l"asst\pdfko{1}\ 
sich durch \verb|\printonlyvqu| plus \verb|\addtovqu| ersetzen. Die 
Syntax ist identisch.

\Doppelbox
{...\bs footnote\{
 \\ \ \bs addtovqu\{Carl von\}\{Clausewitz\}
 \\ \ \ \b{\b{\{}}\bs ktit\{Strategie\}. Hrsg. von 
 \\ \ \ \ \bs vauthor\{Eberhard\}\{Kessel\}, 
 \\ \ \ \ Hamburg 1937\b{\b{\}}}\%
 \\[.2ex] \ \bs printonlyvqu\{Carl von\}
 \\ \ \ \ \ \{Clausewitz\}
 \\ \ \ \b{\{}\bs ktit\{Strategie\}, 
 \\ \ \ \ Hamburg 1937\b{\}}[58].\}
}
{Der Herausgeber Eberhard Kessel erscheint
 nur auf der Liste der gedruckten Quellen,
 aber nicht in der Fu"snote.\footnote{
 \addtovqu{Carl von}{Clausewitz}
  {\ktit{Strategie}. Hrsg.\ von 
         \vauthor{Eberhard}{Kessel}, 
         Hamburg 1937}%
 \printonlyvqu{Carl von}{Clausewitz}
  {\ktit{Strategie}, 
         Hamburg 1937}[58].}
}

\noindent
\verb|\printonlyvqu| erzeugt zusammen mit dem
in seinem letzten Argument stehenden \verb|\ktit|
auch den Eintrag in der \hspace{-.1em}\verb|.vkc|\hy Liste (S.\,\pageref{vkc}; 
dies l"asst sich mit \verb|\notktitaddtok| unterbinden).
\verb|\printonlyvqu| und \verb|\ktit| f"ullen weiter
den Zwischenspeicher und \textit{ihre} Argumente sind 
relevant f"ur das \textsc{ebd.}\hy Setzen.\footnote
{Eine (irrt"umlich) nach der addto\hy Komponente
getippte Seitenangabe (\texttt{[58]}) ist "uberfl"ussig,
w"are aber kein Fehler, denn diese Seitenangabe w"urde 
\BibArts\ einfach 'verschlucken'.}

\verb|\addtovqu| schreibt dagegen nur in die Liste der
gedruckten Quellen. Das Tippen von \verb|\ktit| ist dort
nicht unbedingt n"otig. (Im Text \textit{hier} dient es\pdfko{1}\ 
besonders dazu, nur \textit{einen} Listeneintrag 
"`Strategie"' zu bekommen.)

Die Trennung von \BibArts\hy Befehlen in printonly- und
addto\hy Komponente bedeutet im Falle von Kurztiteln,
die im Kurzzitat gro"s und im Vollzitat klein geschrieben 
werden sollen, nicht, auf \verb|\onlyvoll| und \verb|\onlykurz| 
verzichten zu k"onnen $-$ vgl.\ Ferguson oben S.\,\pageref{vkc}
(\hspace{-.1em}\verb|.vkc|\hy Liste) und S.\,\pageref{Ferguson} (\textsc{ebd.}). 

Wenn \verb|bibsort| etwa einen Literaturlisten\hy Eintrag aus 
einem \verb|\addtovli|\hy Befehl erzeugt, wei"s es nicht, ob 
\verb|\printonlyvli| in derselben Fu"snote steht und dasselbe Werk
ausdruckt. Darauf m"ussen Sie dann selbst achten.

Die Argumente der addto\hy Befehle werden im Text nicht ausgedruckt und 
folglich auch nicht abgearbeitet. 'Innere' Komponenten "au"serer 
addto\hy Befehle m"ussen deshalb danach nochmal separat gesetzt werden. 
Denn \BibArts\pdfko{1}\ 
macht aus inneren Vollzitaten von addto\hy Befehlen keine eigenst"andigen 
Listenpunkte, druckt aber auf den v\hy Listen innere Vollzitate weiterhin 
als Kurzzitat. Das Marx\hy Zitat von eben l"asst sich somit auch so erzeugen 
(vgl.\ \textsf{\pageref{inad}$^{\ref{inad}}$}\pdfko{1}\ 
auf den num\hy Listen) wie hier umgesetzt\footnote{
 %% Text fuer Fussnote und Eintrag ins Kurzzitateverzeichnis: %%
 \printonlyvqu {Karl}{Marx} 
   {Das \ktit{Kapital\onlyhere{~I}}, in: 
  \xprintonlyvqu [m]{Karl}{Marx} *{\vauthor{Friedrich}{Engels}} 
   {\ktit{Werke}, Berlin 1962}}[50].%
 %%
    %% Eintrag des Einzeltextes ins Verzeichnis gedruckter Quellen: %%
 \addtovqu{Karl}{Marx}{Das \ktit{Kapital}. Kritik der 
   politischen "Okonomie, erster Band; das ist Bd.\,23 (1962) 
      von: \xvqu [m]{Karl}{Marx} *{\vauthor{Friedrich}{Engels}} 
            {\ktit{Werke}, hrsg.\ vom Institut f"ur 
          Marxismus-Leninismus beim ZK der SED, 40~Bde.\ 
          Berlin 1958--1971}}% %% Inneres v erscheint in v-Liste als k
 %%
 %% Eintrag der vollen Reihe ins Verzeichnis der gedruckten Quellen: %%
 \xaddtovqu         {Karl}{Marx} *{\vauthor{Friedrich}{Engels}} 
            {\ktit{Werke}, hrsg.\ vom Institut f"ur 
          Marxismus-Leninismus beim ZK der SED, 40~Bde.\ 
          Berlin 1958--1971}\label{inad}}
und auf der Folgeseite vorgemacht:                                      


\vspace{-.5ex}
{\footnotesize
  \begin{verbatim}
    \footnote{
   %% Text fuer Fussnote und Eintrag ins Kurzzitateverzeichnis: %%
     \printonlyvqu {Karl}{Marx} 
       {Das \ktit{Kapital\onlyhere{~I}}, in: 
      \xprintonlyvqu [m]{Karl}{Marx} *{\vauthor{Friedrich}{Engels}} 
       {\ktit{Werke}, Berlin 1962}}[50].%
   %% Eintrag des Einzeltextes ins Verzeichnis gedruckter Quellen: %%
     \addtovqu{Karl}{Marx}{Das \ktit{Kapital}. Kritik der 
       politischen "Okonomie, erster Band; das ist Bd.\,23 (1962) 
          von: \xvqu [m]{Karl}{Marx} *{\vauthor{Friedrich}{Engels}} 
                {\ktit{Werke}, hrsg.\ vom Institut f"ur 
              Marxismus-Leninismus beim ZK der SED, 40~Bde.\ 
              Berlin 1958--1971}}% %% Inneres v erscheint in v-Liste als k
   %% Eintrag der vollen Reihe ins Verzeichnis gedruckter Quellen: %%
     \xaddtovqu         {Karl}{Marx} *{\vauthor{Friedrich}{Engels}} 
                {\ktit{Werke}, hrsg.\ vom Institut f"ur 
              Marxismus-Leninismus beim ZK der SED, 40~Bde.\ 
              Berlin 1958--1971}}  %% aus Vorausgehendem herauskopieren
\end{verbatim}}

\noindent
Falls nur \textit{ein} Teil eines Herausgeberwerkes verwendet wird,
ist auf der Literaturliste das 'innere' Kurzzitieren und separate 
Vollangabe nicht n"otig. Alternativ kann deshalb auf 
innere v\hy\ und k\fhy Befehle ganz verzichtet werden:

\Doppelbox
{\bs vqu \{Karl\} \{Marx\} 
 \\ \b{\{}Das \bs ktit\{Kapital\}. Kritik der 
 \\ \ politischen "Okonomie, erster 
 \\ \ Band; das ist Bd.\bs,23 (1962) 
 \\ \ von: \bs midkauthor\{ders.\} 
 \\ \ \bs ntvauthor\{Friedrich\}\{Engels\}
 \\ \ Werke, hrsg. vom Institut 
 \\ \ f"ur Marxismus-Leninismus 
 \\ \ beim ZK der SED, 
 \\ \ \bs ersch\string|40\string|\{Berlin\}\{1958-{}-1971\}\b{\}}
}
{\footnotesize\notktitaddtok
 \texttt{\%HIER nicht in Listen "ubernommen\%}
 \\[.5ex] \printonlyvqu {Karl} {Marx} {Das \ktit{Kapital}. Kritik der politischen 
 "Okonomie, erster Band; das ist Bd.\,23 (1962) von: 
 \midkauthor{ders.} \ntvauthor{Friedrich}{Engels} 
 Werke, hrsg.\ vom Institut f"ur 
 Marxismus-Leninismus beim ZK der SED, 
 \ersch|40|{Berlin}{1958--1971}}
}

\vspace{1ex}\noindent 
Dies leitet "uber zur \textbf{Umstellung vorgefertigter Textelemente}. 
Schr"agstiche definiert \verb|\nsep|, das seinerseits \verb|\baslash| 
('\baslash') ausf"uhrt. mid\hy Befehle und\pdfko{1}\ 
die Sternargumente von 
x\fhy Befehlen nutzen ihn. \verb*|\renewcommand{\nsep}{, }|\pdfko{.75}\  
w"urde Komma statt Schr"agstrich zwischen Namen drucken. Dies kann 
auch lokal geschehen: Die jeweils aktuelle Definition von \verb|\nsep| reist 
\label{Ausreise} mit jedem v- und k\fhy Zugang \textit{separat} in die Listen  
und wird dort reproduziert. (\verb|\baslash|\pdfko{1.75}\ 
ist dabei unzerbrechlich, weil es \verb|\protect\pbaslash| ausf"uhrt.)

Ein weiterer Separator, \verb|\ntsep|, \label{ntsepA} der zwischen Name und 
Titel '\ntsep' druckt,\pdfko{.5}\ 
sollte dagegen nur im Dokumentenvorspann ge"andert werden. 
Ausgef"uhrt wird \verb|\ntsep| von v- und k\fhy Befehlen 
sowie \verb|\ntvauthor| und \verb|\ntkauthor|. Gelegentlich 
ist ein lokal auf den \textit{Ausdruck ganzer Listen} 
beschr"anktes "Andern\pdfko{1.125}\ 
von \verb|\ntsep| sinnvoll und k"onnte etwa 
\verb*|\renewcommand{\ntsep}{, }| lauten. 

Im letzten Beispiel wurde auch \verb!\ersch|40|{Berlin}{1958--1971}! \label{ersch}
verwendet, was ausgedruckt ergibt: \ersch|40|{Berlin}{1958--1971}.
Dabei ist \verb!|40|! optional. Ein normales Buch kann am Ende des
letztes Arguments eines v\fhy Befehls stets etwas stehen haben wie
\verb!\ersch{Berlin}{2003}! -- das ergibt: \ersch{Berlin}{2003} --; 
oder auch \verb!\ersch[2]{Berlin}{2003}!, was \ersch[2]{Berlin}{2003} 
ergibt. Und \verb!\ersch{}{}! druckt \ersch{}{} -- also: ohne Ort, ohne
Jahr. Nach \verb|\exponenteditionnumber| druckt
\verb!\ersch|5|[2]{Mainz}{2008}! aus: {\exponenteditionnumber
\ersch|5|[2]{Mainz}{2008}}, also mit $^{2}2008$. Das sonst verwendete 
'Auf{\kern.03em}l.,' ist definiert als \verb|{\teskip Auf{\kern.03em}l.,}| 
und kann ge"andert werden mittels
\verb+\renewcommand{\gerscheditionname}{\teskip Auf"|lage}+ in 'Auf"|lage'.


\vspace{1.25ex}\label{female}\noindent
Falls in den \textit{v\fhy Listen bei Autorwiederholung} \textsc{dies.} 
oder \textsc{ders.} statt $\sim$ stehen soll, k"onnen Sie \verb|\female|
bzw.\ \verb|\male| in die v-Befehle zu Anfang der Vornamensargumente
tippen. Beispiel: \verb|\vqu{\male Karl}{Marx}{|...\verb|}| Das muss $-$\,einmal
etwa f"ur die vli-Liste angefangen\,$-$ dann aber in jedem vli-Befehl stehen
(ausgenommen anonyme Autoren \verb|\vli{}{}{|...\verb|}|): Nur so wird w\kern-.1em/m von 
\hspace{.2em}\verb|bibsort -k|\hspace{.2em} richtig zugeordnet (gleiche Namen 
gelten auch dann als gleich, wenn \verb|\female| oder \verb|\male| vergessen
wird; evtl.\ gilt dann das Geschlecht der vorausgehenden Person). 
Sind dann auch alle Koautoren gleich, wird automatisch \textsc{diesn.} 
f"ur 'Dieselben' gesetzt. Falls nur die ersten von mehreren Koautoren gleich sind, 
wird f"ur \textit{die} weiterhin {\small$\sim$} oder {\small$\sim$\baslash$\sim$} 
gesetzt.\pdfko{.5}  

\vspace{.675ex}\noindent
Der \textit{Text}\hskip0pt plus 1pt\ ist mit\hskip0pt plus 1pt\ \verb|\renewcommand| an\hskip0pt plus 1pt\ 
\verb|\geademname| ({\small\verb|dies\kern -0.04em.|}\kern-.2em),\pdfko{.75}\ 
 \verb|\gidemname| ({\small\verb|ders\kern -0.04em.|}\kern-.2em) und 
\verb|\giidemname| ({\small\verb|diesn\kern -0.07em.|}\kern-.2em)\pdfko{.75}\
einstellbar (\textit{nicht die} \textsc{schrift}\textit{!}). Dies wirkt auch auf 
\verb|[f]|, \verb|[m]| und \verb|[p]| 
bei\pdfko{1.5}\ 
v- und k\fhy Befehlen (samt Ausdruck innerer v- oder k\fhy Befehle in den Listen).


\vspace{1.25ex}\noindent
Dagegen erfolgt ein \textit{Umstellen von \textsc{ebd.}} mit \label{setibidem}
\verb|\setibidem{g}{ebenda}{}| in\pdfko{1.25}\ 
\textsc{ebenda}. Die Voreinstellung ist
\verb|\setibidem{g}{ebd\kern -0.07em}{.}| in\pdfko{1.25}\  
\verb|bibarts.sty|. Das dritte
Argument kann nur entweder leer sein oder einen Punkt enthalten; es dient 
dazu, \BibArts\ mitzuteilen, wie beim automatischen\pdfko{.5}\  
\textsc{ebd.}\hy Setzen mit
einem nach dem k\fhy Befehl stehenden Punkt umgegangen werden soll (um 
\textsc{ebd.}.\ zu vermeiden). Nur hier ist \verb|\renewcommand| verboten!\pdfko{1.25}


\vspace{1.25ex}\noindent
Die Schrift, in der \textit{Autoren\hy Nachnamen} gesetzt sind, ist \verb|\authoremph|.
Mit\pdfko{1.25}\  
\verb|\renewcommand{\authoremph}{\upshape}| lie"se sich die voreingestellte
Hervorhebung von \textsc{Nachnamen} beim Ausdruck von v- und k\fhy Befehlen 
aufheben. Alternativ kann \verb|\stressing| \textit{ein} Schriftbefehl
ohne~\kern-.15em\verb|\| "ubergeben werden: \textit{Etwa} \verb|\stressing{underline}| 
\textit{initiiert 
{\itshape\notprinthints\stressing{underline}\printonlykli{Nachname}{}} auch
in kursivem Umfeld.}


\vfill\noindent{\sffamily 
Sprachabh"angig vorgefertigte Textelemente folgen in Kapitel~\ref{SprachSep} 
unten ab S.\,\pageref{SprachSep};}
\\[.25ex]{\sffamily 
einstellbare Texthervorhebungen liste ich in Kapitel~\ref{hervor} 
unten S.\,\pageref{hervor} auf;}
\\[.25ex]{\sffamily 
und der Literaturtyp \textit{Zeitschriften} kommt gleich in 
Kapitel~\ref{per} unten ab S.\,\pageref{per}.}%


\newpage 
Da das \textbf{\textsc{ders.}\hy Setzen} mit \verb|[f]|, \verb|[m]| 
oder \verb|[p]| anf"allig f"ur Fehler ist, wenn Textteile im 
Texteditor ausgeschnitten und verschoben werden, gibt es 
eine weitere \textbf{Kontrollm"oglichkeit}: "Uber den \LaTeX\hy 
Bildschirmausdruck hinaus \baref{DERS} k"onnen Sie sich testweise 
im Ausdruck selbst informieren lassen:

\vspace{1ex}
\Doppelbox
{\bs writeidemwarnings
 \\ \ \bs footnote\{\bs kqu[m]\{Clausewitz\} 
 \\ \ \ \ \{Strategie\}[61] und 
 \\ \ \ \bs kqu\{Clausewitz\}
 \\ \ \ \ \{Vom Kriege\}[62].\} 
 \\[1ex] \ \bs footnote\{\bs kqu[m]\{Clausewitz\} 
 \\ \ \ \{Strategie\}[63].\}
 \\[1ex] \ \bs footnote\{\bs kqu[m]\{Clausewitz\} 
 \\ \ \ \ \{Vom Kriege\}[64] und 
 \\ \ \ \bs kli[m]\{Luhmann\}\{Soziale 
 \\ \ \ \ Systeme\}[65].\}
 \\[1ex] \ \bs footnote\{\bs kqu[m]\{Clausewitz\} 
 \\ \ \ \{Strategie\}[66].\}
 \\[1ex] \ \bs footnote\{\bs vqu[m]\{Karl\}\{Marx\} 
 \\ \ \ \b{\b{\{}}Das 
 \\ \ \ \ \bs ktit\{Kapital\bs onlyhere\{\string~II\}\}\%
 \\ \ \ \ \bs onlyout\{. Kritik der 
 \\ \ \ \ \ politischen "Okonomie, 
 \\ \ \ \ \ zweiter Band; das ist 
 \\ \ \ \ \ Bd.\bs,24 (1962) 
 \\ \ \ \ \ von:\}\bs onlyhere\{, in:\} 
 \\ \ \ \ \bs xvqu \{Karl\}\{Marx\}
 \\ \ \ \ \ *\{\bs vauthor
 \\ \ \ \ \ \ \ \ \{Friedrich\}\{Engels\}\} 
 \\ \ \ \ \ \b{\{}\bs ktit\{Werke\}, 
 \\ \ \ \ \ \ \bs onlyout\{hrsg.\ vom Institut 
 \\ \ \ \ \ \ \ f"ur Marxismus-Leninismus 
 \\ \ \ \ \ \ \ beim ZK der SED, 40\string~Bde. 
 \\ \ \ \ \ \ \ Berlin 1958-{}-1971\}\%
 \\ \ \ \ \ \ \bs onlyhere\{Berlin 
 \\ \ \ \ \ \ \ 1962\}\b{\}}\b{\b{\}}}[67].\}
}
{\writeidemwarnings
 \footnote{\kqu[m]{Clausewitz} {Strategie}[61] und 
      \kqu{Clausewitz} {Vom Kriege}[62].} 
 \\[8.25ex]
 \footnote{\kqu[m]{Clausewitz} {Strategie}[63].}
 \\[3ex]
 \footnote{\kqu[m]{Clausewitz} {Vom Kriege}[64] und 
      \kli[m]{Luhmann}{Soziale Systeme}[65].}
 \\[8.25ex]
 \footnote{\kqu[m]{Clausewitz} {Strategie}[66].}
 \\[3ex]
 \footnote{\vqu[m]{Karl}{Marx} {Das 
  \ktit{Kapital\onlyhere{~II}}%
                     \onlyout{. Kritik der politischen 
  "Okonomie, zweiter Band; das ist Bd.\,24 (1962) 
  von:}\onlyhere{, in:} \xvqu {Karl}{Marx}
  *{\vauthor{Friedrich}{Engels}} {\ktit{Werke}, \onlyout{hrsg.\ 
  vom Institut f"ur Marxismus-Leninismus beim ZK der SED, 
  40~Bde.\ Berlin 1958--1971}%
        \onlyhere{Berlin 1962}}}[67].
        \\
        \texttt{\% Vgl.\ DERS. in der Liste \baref[]{printnumvqu}}
        }
}

\vspace{1ex}\noindent
Nach Setzen von \verb|\writeidemwarnings| druckten v- und k\fhy Befehle 
dabei in Klammern deren Nachnamensargument hinter folgenden Symbolen aus:

 \vspace{2ex}\noindent
 {\small\begin{tabular}{cl}%
 $\bullet$    & \textsc{\footnotesize DERS.} \sffamily fehlt m"oglicherweise (gleiche Nachnamen registriert).              \\
 $\heartsuit$ & \textsc{\footnotesize DERS.} \sffamily ist offenbar richtig gesetzt (gleiche Nachnamen registriert).       \\
 $\nabla$     & \textsc{\footnotesize DERS.} \sffamily wegen fehlender Autoren in vorausgehender Fu"snote unberechtigt.    \\
 $\spadesuit$ & \textsc{\footnotesize DERS.} \sffamily "uberschreibt einen Namen, der nicht der vorausgehende ist.          \\
 $\clubsuit$  & \textsc{\footnotesize DERS.} \sffamily steht irref"uhrenderweise nach einer Fu"snote mit mehreren Autoren. \\
 \end{tabular}}

\vspace{2ex}\noindent
\BibArts\ kontrolliert niemals Koautoren. Falls auch die in
aufeinanderfolgenden Fu"snoten gleich sind, lassen sie sich zwar durch 
Ersatzworte ersetzen, wozu \BibArts\ bei Fehlern aber nicht warnt. 
Hier ein Beispiel ohne Fehler:

\Doppelbox
{\bs footnote\{.... \bs xkli\{Maier\} 
  \\ \ *\H{\{}\bs midkauthor\{M"uller\} 
        \\ \ \ \ \bs kauthor\{Huber\}\H{\}} \{Geld\}[i].\}
 \\[1ex] \bs footnote\{\bs xkli[p\{\}]\{Maier\} 
  \\ \  *\{\bs midkauthor\{M"uller\} 
        \\ \ \ \ \bs kauthor\{Huber\}\} \{Haus\}[ii].\}
 \\[1ex] \bs footnote\{\bs xkli[p\b{\b{\{}} ersten beiden 
 \\ \ \ \ und \bs kauthor\{Schmidt\}\b{\b{\}}}]
 \\ \ \ \{Maier\} *\b{\{}\bs midkauthor\{M"uller\} 
 \\ \ \ \ \bs kauthor\{Schmidt\}\b{\}} 
 \\ \ \ \{Vorsorge\}[iii].\}
}
{\texttt{\% HIER nicht in den Listen \%}
 \\[3ex]
 \footnote{Lokal erst \xprintonlykli{Maier} 
    *{\midkauthor{M"uller} \kauthor{Huber}} {Geld}[i].}
                
 \footnote{\xprintonlykli[p{}]{Maier} 
    *{\midkauthor{M"uller} \kauthor{Huber}} {Haus}[ii].}
                
 \footnote{\xprintonlykli[p{ ersten beiden und \kauthor{Schmidt}}] {Maier} 
    *{\midkauthor{M"uller} \kauthor{Schmidt}} {Vorsorge}[iii].}
}


\noindent\label{p}%
Nur mit \verb|[p{}]| oder \verb|[p{|\textit{xx}\verb|}]| werden alle 
Namen mit \textsc{diesn.} "uberschrieben (und nicht wie mit \verb|[p]| 
nur der erste).\footnote{\texttt{[f\{\}]} und \texttt{[m\{\}]} existieren aus 
Symmetriegr"unden, sind aber "uberfl"ussig.} Wenn mehrere, aber eben 
nicht alle Autoren \textit{dieselben} sind, m"ussen Sie die zuviel mit
\textsc{diesn.} "uberschriebenen wie gerade gezeigt in \textit{xx}
wieder nennen. 

Um \textit{richtig sortierte Listen} zu erzeugen, sollten in v- 
und k\fhy Befehlen die 'regul"aren' Namensargumente
\textit{in jedem Fall} vollst"andig bef"ullt sein (obwohl \textit{die} 
in der Fu"snote gar nicht gedruckt werden, sondern \textsc{ders.} etc.).

Falls Sie \textsc{dies.}, \textsc{ders.} und \textsc{diesn.} nicht
verwenden wollen, k"onnen Sie alle diesbez"uglichen Warnungen auch 
mit \verb|\notwarnsamename| im Vorspann Ihres \LaTeX\hy Textes
ausschalten. Das unterbindet bei der \LaTeX\hy "Ubersetzung die 
Bildschirmwarnung {\small\texttt{cmd repeats (first) author's 
lastname}} samt allen weiteren Meldungen f"ur die eben 
aufgelisteten Fehlertypen. \textit{Zus"atzlich} wird 
\verb|\writeidemwarnings| unwirksam; \BibArts\ druckt also nicht 
mehr {\small$\bullet\heartsuit\nabla\spadesuit\clubsuit$}\,.

\vspace{1.25ex}\noindent
Dass bei \textit{inneren} v- und k\fhy Befehlen gesetzte Attribute \verb|[f]|, 
\verb|[m]| und \verb|[p]| in\pdfko{1}\ 
die Listen "ubernommen werden, wurde oben 
beim Ausdruck des Verzeichnisses der gedruckten Quellen anhand des ersten 
Bandes von Marx' Kapital demonstriert. Beim zweiten Band dagegen ist 
der 'innere' Marx nicht mit \verb|[m]| versehen; beim "Ubersetzen der 
Fu"snote, aus der der Eintrag herstammt, erscheint die Warnung 
{\footnotesize\texttt{Inner ...vqu-cmd repeats author's lastname}}. 
Falls Sie\pdfko{1}\ nur dort kein \textsc{ders.} haben wollten, k"onnen
Sie vor den inneren v- oder k\fhy Befehl \verb|\notwarnsamename| 
setzen, um die Warnung lokal auszuschalten.\footnote{Bei Ausdruck der 
v\fhy Listen erfolgt nie eine Warnung, falls der 
innere und der "au"sere Autor gleich sind (und innen \texttt{[f]}, 
\texttt{[m]} oder \texttt{[p\{\}]} bzw.\ \texttt{[p\{}\textit{xx}\texttt{\}]} fehlt).}

\newpage\noindent
Zum \textbf{Zitieren mehrb"andiger Werke} gibt es einen Speicher 
f"ur Bandnummern. Das optionale Argument \verb+|+\textit{Band}\verb+|+
zur \textsc{ebd}.\hy Setzung steht ohne Leerzeichen vor dem 
Seitenargument (\verb+\ersch|+\textit{Band}\verb+|+... bef"ullt den Speicher nicht):

\label{Reinhard}%
\Doppelbox
{\bs footnote\{... \bs vli\{Wolfgang\} 
 \\ \ \{Reinhard\} \b{\b{\{}}Geschichte der
 \\ \ \ \bs ktit\b{\{}\bs onlyvoll\{e\}\%
 \\ \ \ \ \ \ \ \ \ \bs onlykurz\{E\}urop"aische\%
 \\ \ \ \ \ \ \ \ \ \bs onlyvoll\{n\} Expansion\b{\}}, 
 \\ \ \ \bs ersch\string|4\string|\{Stuttgart\}
 \\ \ \ \ \ \ \ \ \ \ \ \{1983-{}-1990\}\b{\b{\}}}\string|2\string|[98].\}
 \\[1ex] \bs footnote\{\bs kli \{Reinhard\} \{Europ"aische Expansion\}\string|2\string|[98].\}
 \\[1ex] \bs footnote\{\bs kli \{Reinhard\} \{Europ"aische Expansion\}\string|3\string|[1].\}
}
{\vspace{1.5ex}%
 \footnote{Band aus Reihe: \notktitaddtok\printonlyvli{Wolfgang} {Reinhard} {Geschichte der
 \ktit{\onlyvoll{e}%
   \onlykurz{E}urop"aische%
         \onlyvoll{n} Expansion},
 \ersch|4|{Stuttgart} {1983--1990}}|2|[98].\label{Ranf}}
 
 \footnote{\printonlykli {Reinhard} {Europ"aische Expansion}|2|[98].}
 
 \footnote{\printonlykli {Reinhard} {Europ"aische Expansion}|3|[1].\label{Rend}}
}

\noindent
St"unden hier \verb|\vli| und \verb|\kli| statt der
tats"achlich verwendeten printonly-Befehle, ginge ins Kurzzitateverzeichnis  
\textsc{Reinhard}: Europ"aische Expansion [L] \hspace{.5em} 
\textsf{\pageref{Ranf}$^{\ref{Ranf}-\ref{Rend}}$} 
und ins Literaturverzeichnis: 
\textsc{Reinhard}, Wolfgang: Geschichte der europ"aischen Expansion, 
4 Bde., Stuttgart 1983--1990.

\vspace{.1ex}
W"urde die Bandangabe \verb+|3|+ \hspace{.1em}(oder Nummer \hspace{-.3em}\verb+_n_+)\hspace{.1em} in 
Fu"snote~\ref{Rend} fehlen, dann erschiene w"ahrend der \LaTeX\hy "Ubersetzung die 
Fehlermeldung:\footnote{Eine Warnung bei 'innerer' \textsc{ebd.}\hy Setzung erfolgt nur,
wenn \textit{innen} Band- bzw.\ Seitenangaben fehlen
\textit{und} zuvor (auch) entsprechende '"au"sere' Angaben standen. Sie erhalten\pdfko{.5}\
also \textsc{ebd.}, aber u.\,U. keine Warnung, wenn Sie innere 
Bandangaben zu tippen vergessen!}

\vspace{.5ex}\noindent 
\label{pervol}%
{\small\texttt{ !~Same title, before with :\{pervol\}\{2\}:, has now no no./vol number.}}

\vspace{.75ex}\noindent
Dagegen d"urfte die \verb|[1]| nach der \verb+|3|+ wegbleiben, weil es 
sich um einen anderen Band als in der vorausgehenden 
Fu"snote handelt. $-$ Im Falle von \textbf{mehrb"andigen Herausgeberwerken}
sind auch 'innere' Bandangaben erlaubt:

\vspace{-.2ex}%
\Doppelbox
{\vspace{1.75ex}\bs footnote\{\bs vli \{\}\{\}
 \\ \ \b{\{}\bs ktit\{Au"sen~1\}, in:
 \\ \ \ \bs vli \{\}\{\} \b{\b{\{}}\bs ktit\{Innen\}\b{\b{\}}}\%
 \\ \ \ \ \ \ \ \string|12\string|[100-199]\b{\}}*[111].\}
 \\[.5ex] \bs footnote\{\bs kli\{\}\{Au"sen~1\}[111].\}
 \\[.5ex] \bs footnote\{... \bs vli\{\}\{\}
 \\ \ \H{\{}\bs ktit\{Au"sen~2\}, in: \bs kli \{\} 
 \\ \ \ \ \{Innen\}\string|12\string|[200-299]\H{\}}*[222].\}
}
{\footnote{\notktitaddtok\printonlyvli {}{}%
  {\ktit{Au"sen~1}, in:
    \printonlyvli {}{}
        {\ktit{Innen}}%
          |12|[100-199]}*[111].}
 \footnote{\printonlykli{}{Au"sen~1}[111].}
 \footnote{Gleiche Reihe: \notktitaddtok\printonlyvli{}{}
        {\ktit {Au"sen~2}, in:
    \printonlykli {} {Innen}|12|[200-299]}*[222].}
}

\vspace{-.2ex}\noindent
\BibArts\ pr"uft nicht, ob \verb|*[|\textit{Seite}\verb|]| innerhalb des 
genannten Seitenintervalls liegt.

\newpage\noindent
Wie sind \textbf{Werke} in die Listen aufzunehmen, \textbf{die im Text nie 
verwendet wurden}, die Sie aber trotzdem im Anhang auf"|listen m"ochten? 
Solche Werke d"urfen auf den num\hy Listen nicht mit den Seiten\fhy/""Fu"snotennummern 
der Stelle gedruckt werden, an der die Angaben im \hspace{-.15em}\verb|.tex|\hy 
File stehen! Die addto\hy Befehle sind also ungeeignet. Statt dessen gibt es die 
\verb|{unused}|\hy Umgebung:\balabel{unused}

\vspace{-.25ex}
{\small\begin{verbatim}
  \begin{unused}
    \vli{James M.}{McPherson}{\ktit{Battle Cry of Freedom}. The 
       American Civil War, Oxford 1988}[vi] 
  \end{unused}     %% Beispiel HIER nicht in den Listen %%
\end{verbatim}}

\vspace{-.25ex}
%\begin{unused}
%\vli{James M.}{McPherson}{\ktit{Battle Cry of Freedom}. The 
%American Civil War, Oxford 1988}[vi]
%\end{unused}
%%
\noindent Seitenzahlen wie hier \verb|[vi]| 
werden ignoriert. Damit lassen sich Werke w"ahrend des Schreibens 
einfacher von einer Fu"snote in die \verb|{unused}|\hy Umgebung 
verschieben. Dort sind sogar addto\hy Befehle 
erlaubt (tats"achlich sieht \BibArts\ dort \verb|\vli|\hy Befehle als
\verb|\addtovli|\hy Befehle unbestimmter Herkunft).\pdfko{.125}\ 
\verb|\printonlyvli| hat in \verb|{unused}|\hy Umgebungen nat"urlich nichts 
zu suchen. Aber sonst d"urfen Sie alle \BibArts\hy Hauptbefehle 
wie \verb|\vli| und \verb|\vqu| sowie die\pdfko{1}\ 
unten Seite~\pageref{per}
und \pageref{archivquellen} eingef"uhrten Befehle \verb|\per| und \verb|\arq| 
nutzen; dazwischen d"urfen Leerzeichen und \textit{einfache} Zeilenumbr"uche
stehen. Wenn Sie sich an diese Regeln halten, haben Sie eine im ausgedruckten 
Text unsichtbare Spielwiese. Ganz am Ende Ihrer\hspace{-.2em} \verb|.tex|\hy Datei 
$-$ dort insbesondere nach Ende einer \verb|twocolum|\fhy Umgebung oder nach 
einem \verb|\newpage|\hy Befehl $-$ haben\pdfko{.5}\ 
\verb|{unused}|\hy Umgebungen 
nichts zu suchen, da sie dort nicht mehr umgesetzt werden und deshalb in 
die Listen nichts geschrieben w"urde. Ein guter Platz zum Sammeln ist 
dagegen \textit{vor} dem zugeh"origen Listenausdruckbefehl.

\vspace{2ex}\noindent
In \verb|{unused}|\hy Umgebungen gilt "ahnliches, wie f"ur addto\fhy 
Befehle au"serhalb: 

\vspace{1ex}\noindent
\parbox{1.8em}{(1)}Die dort in v\fhy Befehlen
mit \verb|\ktit| markierten Kurztitel bewirken keinen Eintrag ins 
Kurzzitateverzeichnis. Falls gew"unscht, sind dazu 
\verb|\kli| oder \verb|\kqu|\pdfko{1.25}\ 
in separaten Eintr"agen in die \verb|{unused}|\hy 
Umgebung einzuf"ugen. 

\vspace{1ex}\noindent
\parbox{1.8em}{(2)}Innere v\fhy Befehle erzeugen \textit{nicht 
automatisch einen eigenen Volleintrag in\pdfko{.75}\ 
den v\fhy Listen}. Dort werden sie aber als Kurzzitat
ausgedruckt. Innere v\fhy Befehle m"ussen in \verb|{unused}|\hy Umgebungen 
deshalb kopiert und danach nochmals separat in die Umgebung eingef"ugt 
werden. (Eigentlich sind in \verb|{unused}|\hy Umgebungen innere
v\fhy Befehle unn"otig und somit innere k\fhy Befehle ausreichend.)


\vfill\noindent
\textsf{Bevor ich Zeitschriftenbelege vorstelle, kommt nun erst das w"ortliche 
Zitieren.}


\newpage
\section{W"ortliche Zitate in verschiedenen Sprachen}\label{Sect2}

Bei l"angeren w"ortlichen Zitaten ist "ublich, diese zur besseren
Erkennbarkeit vom restlichen Text deutlich abzusetzen. \BibArts\
stellt eine Umgebung bereit:\pdfko{1}\vspace{-.5ex}

\Doppelbox
{\ \ ... das Zitat auch:
 \\[.4ex] \bs begin\{originalquote\}
 \\[.35ex] \ \ \string"\string`Der Krieg entsteht  
 \\ \ \ nicht urpl"otzlich; seine 
 \\ \ \ Verbreitung ist nicht das 
 \\ \ \ Werk eines Augenblicks, 
 \\[.125ex] \ \ [...].\string"\string'\bs footnote \b{\{}
 \\[.125ex] \ \ \ \ \bs kqu\{Clausewitz\} 
 \\[.15ex] \ \ \ \ \ \ \ \ \{Vom Kriege\}[22].\b{\}}
 \\[.25ex] \bs end\{originalquote\}
}
{\renewcommand{\originalquotetype}{\footnotesize}%
 Der umgebende Text hat den deutschen Trennsatz, und das Zitat auch:
 \begin{originalquote}
 "`Der Krieg entsteht nicht urpl"otzlich; seine Verbreitung 
 ist nicht das Werk eines Augen"-blicks, 
 [...]."'\footnote {
 \kqu{Clausewitz} {Vom Kriege}[22].}
 \end{originalquote}
}

\vspace{.25ex}\noindent
Gr"unde f"ur die neue Umgebung: Die \LaTeX\hy Umgebung \verb|{quote}| setzt den
vertikalen Abstand zum Fu"snotenbereich viel zu klein, wenn mitten in den
"ubersetzten Zitatblock ein Seitenumbruch f"allt. Damit die
\verb|{originalquote}|\hy Umgebung reagieren kann, versieht \verb|bibarts.sty| 
den bestehenden \LaTeX\hy Befehl\pdfko{1.25}\ 
\verb|\footnoterule| mit einem Zusatz.\footnote{
Au"serdem wird der \textit{Fu"snotenbereich} an den Fu"s der Seite geschoben 
durch Einf"ugen von zus"atzlichem vertikalem Zwischenraum. 
Deshalb sollten Sie zusammen mit \BibArts\ den
\LaTeX\hy Befehl \texttt{\bs flushbottom} 
nicht verwenden. $-$ \BibArts\ setzt in Version 2.1 \texttt{\bs footnotesep} 
nicht mehr auf \texttt{2ex}, um den \textit{Abstand zwischen Fu"snoten} zu vergr"o"sern.
Falls Sie einen Text mit \BibArts~2.0 begannen,
m"ussen Sie nun \texttt{\bs setlength\{\bs footnotesep\}\{2ex\}} setzen.} 
Wenn Sie \verb|\footnoterule| anschlie"send\pdfko{1}\ 
einfach umdefinieren, dann schalten Sie diese Eigenschaft
aus.\footnote{Um die Dicke des Strichs vor dem Fu"snotenbereich zu
"andern, m"ussen Sie unter \BibArts\pdfko{1.25}\ 
den Befehl \texttt{\bs fnrbasave} statt 
lehrbuchgem"a"s \texttt{\bs footnoterule} "andern, beispielsweise: \\[.25ex]
\hspace*{2em}{\ttfamily
 \bs renewcommand\{\bs fnrbasave\}\{\bs noindent\bs rule\{5cm\}\{0.5mm\}\bs vspace\{1ex\}\} }}

Ein zweiter Grund f"ur die \verb|{originalquote}|\hy Umgebung betrifft
den Zeilenumbruch. Geistes- und
SozialwissenschaftlerInnen zitieren oft in einer von der Basissprache
ihres Textes abweichenden Sprache. Dann muss zur richtigen Worttrennung 
aber \textit{nur} der Trennsatz umgestellt werden. 
\verb|\selectlanguage|\pdfko{1.25}\ 
aus \texttt{ngerman.sty} bewirkt aber gleichzeitig, 
dass ein Kapitel nach Setzen von\pdfko{.5}\ 
\verb|{english}| \textit{Chapter} 
hei"st und eine Seite pl"otzlich \textit{Page}. \BibArts\ separiert beides
(vgl.\ sprachabh"angige Textelemente unten S.\,\pageref{SprachSep}). 
Zur Einstellung des\pdfko{.25}\  
Trennsatzes mit \BibArts\hy Befehlen dienen 
dieselben Schl"usselbegriffe, die Sie\pdfko{.5}\  
auch als Argument f"ur \verb|\selectlanguage| verwenden. 
Folgendes Beispiel ist\pdfko{.75}\ 
englisch und druckt das Zitat in der Gr"o"se der umgebenden Schrift aus: 


{\notktitaddtok\renewcommand{\originalquotetype}{}
  \begin{originalquote}[english]
     "`Virginia brought crucial resources to the Confederacy. 
     Her population was the South's largest. Her industrial 
     capacity was nearly as great as that of the seven original 
     Confederate states combined."'\footnote{Auch englische 
     Trennung: \printonlyvli{James M.}{McPherson}{\ktit{Battle 
     Cry of Freedom}. The American Civil War, Oxford 1988}.}
  \end{originalquote}}

\vspace{-.5ex}\noindent
Dieses w"ortliche Zitat wurde mit folgendem \LaTeX\hy Code erzeugt:

\vspace{-.75ex}{\small\begin{verbatim}
  {\renewcommand{\originalquotetype}{}     %% Statt \small
   \begin{originalquote}[english]
      "`Virginia brought crucial resources to the Confederacy. 
      Her population was the South's largest. Her industrial 
      capacity was nearly as great as that of the seven original 
      Confederate states combined."'\footnote{Auch englische 
      Trennung: \vli{James M.}{McPherson}{\ktit{Battle Cry of 
      Freedom}. The American Civil War, Oxford 1988}.}
   \end{originalquote}}
\end{verbatim}}


\vspace{-.5ex}\noindent
Falls dort \verb|\begin{originalquote}[eglihs]| st"unde, w"are die 
Fehlermeldung bei der \LaTeX-"Ubersetzung:

\vspace{-.5ex}{\scriptsize\begin{verbatim}
     ** Arg(s) of BibArts' sethyphenation-command: Error around line 1371!
        You've called \begin{originalquote}[eglihs].
     <H><return>  for immediate help, 
     <return>     to continue.
   ! Language-name `eglihs' is undefined. (Old VALUE remains valid: 43).
     . . . . . . . . . . . .
   \errmessage@ba ...
    \space . . . . . . . . . . . }
                                                  }
   l.1145 \begin{originalquote}[eglihs]
\end{verbatim}}

\vspace{-.75ex}\noindent
Da oben tats"achlich ein \verb|\printonlyvli|\hy Befehl
steht, ist eine Besonderheit von \BibArts\ nur 
S.\,\pageref{Trennbeispiel} bei "`Zum Schluss ..."' zu sehen:  
\verb|bibsort| reproduziert den am Zugang g"ultigen Trennsatz 
\textit{beim Listenausdruck}. Bei der \LaTeX\hy "Ubersetzung 
der \textit{Datei mit der erzeugten Liste} kommen Bildschirm\hy 
Meldungen:\label{hyphenation}%

\vspace{-0.5ex}
{\scriptsize\begin{verbatim}
   [bibsort] Reproduce hyphenation 0 in line 1226 of BibArts file. 
   [bibsort] Reproduce hyphenation 44 in line 1230 of BibArts file.
\end{verbatim}}

\vspace{-0.75ex}\noindent 
Das ist die Trennsatz\hy Umschaltung \textit{vor} dem englischen Listenpunkt 
und das Zur"uckschalten ins Deutsche \textit{dahinter} (44 f"ur deutsch ist
versionsabh"angig).

\vspace{1.25ex}\noindent
Um Trenns"atze $-$ und nur die $-$ auch au"serhalb von \verb|{originalquote}| 
einzustellen, bietet \BibArts\ den weiteren Befehl \verb|\sethyphenation| an. 
Ein deutschsprachiges Wort\footnote{In einem Zitat aus \kqu{Clausewitz}
{Vom Kriege}[75 (I.6)].} ist unten f"alschlicherweise franz"osisch getrennt. 
Falls Sie diesen Text mit \LaTeX\ "ubersetzen und nicht widers-prechend 
getrennt wird, verf"ugt Ihre \LaTeX\hy Version entweder "uber keinen 
franz"osischen Trennsatz oder reagiert auf Umschaltungen \textit{in} 
Abs"atzen nicht (sondern nur am Absatzkopf):

    \vspace{2ex}%
    \noindent\hspace{1.9em}\parbox{12.5cm}{\small\ttfamily
    \string"\string`Ein gro"ser Teil der Nachrichten, die man im Kriege \\
    bekommt, ist \{\bs sethyphenation\{french\} widersprechend\}, \\
    ein noch gr"o"serer ist falsch und bei weitem der \\
    gr"o"ste einer ziemlichen Ungewi"sheit unterworfen.\string"\string'
    }

    \vspace{2ex}%
    \noindent\hspace{1.9em}\parbox{12.5cm}{\small
    "`Ein gro"ser Teil der Nachrichten, die man im Kriege 
    bekommt, ist {\sethyphenation{french} widersprechend}, 
    ein noch gr"o"serer ist falsch und bei weitem der 
    gr"o"ste einer ziemlichen Ungewi"sheit unterworfen."'
    }

\vspace{2.5ex}\noindent
Die verschiedenen Befehle zur Trennsatz\hy Einstellung sind
kombinierbar. Falls in einer \verb|{originalquote}|\hy Umgebung 
der Titel des zitierten Werkes eine andere Sprache als das 
w"ortliche Zitat hat, darf \verb|\sethyphenation| am Kopf\pdfko{1.5}\ 
der Fu"snote stehen. Falls Sie \verb|\sethyphenation| oder 
\verb|\selectlanguage| zudem \textit{in} den 
\BibArts\hy Argumenten verwenden, ist dies (samt Argument) f"ur die
Sortierreihenfolge unerheblich. Speziell aber im Nachnamensargument
von \verb|\vli| und \verb|\kli| sollten Sie solche Befehle wegen der
\textsc{ebd.}\hy Setzung vermeiden. Wenn Sie stattdessen \textit{Trennhilfen} 
bei Autornamen nutzen, sollten die \textit{bei allen 
v- und k\fhy Nennungen eines Werkes} einheitlich gesetzt
sein.\footnote{Mehrere (ansonsten) zeichengleiche Listenzug"ange, 
bei denen \texttt{\bs sethyphenation} oder\pdfko{.5}\ 
\texttt{\bs selectlanguage} mal gesetzt und 
mal vergessen (oder mit verschiedenen Sprachen besetzt) 
wurde, ergeben mehrere Listeneintr"age; uneinheitliche 
Trennhilfen \texttt{\bs-} und \texttt{\dq-} auch.
\texttt{bibsort\hspace{.3em}-k} \hspace{.2em}setzt $\sim$ 
bei wechselnden Trennhilfen, nicht aber bei vergessenen 
set\hy Befehlen.} 

\vspace{1.5ex}\noindent
Nebenbei: Die \verb|{originalquote}|\hy Umgebung und der
\verb|\sethyphenation|\hy Befehl "andern absichtlich auch das 
\textit{spacing} nicht, weil dies in einem Text durchgehend gleich sein sollte.
Vgl. unten Kap.\,\ref{nonfrenchspacing} ab S.\,\pageref{nonfrenchspacing} und 
Kap.\,\ref{Punkte} ab S.\,\pageref{Punkte}.

\vspace{1.5ex}\noindent
\BibArts\ reproduziert dar"uber hinaus den \verb|german.sty|\hy\ bzw.\ 
\verb|ngerman.sty|\hy Befehl \verb|\originalTeX| beim Listenausdruck, 
falls ein Eintrag aus einem Umfeld mit ver"andertem \textit{catcode} f"ur \verb|"| 
herstammt. Von den beiden \verb|"a| unten S.\,\pageref{originaltex}
ist nur eines als "a einsortiert. \verb|\originalTeX| schaltet zudem
den englischen Trennsatz ein. W"ahrend der "Ubersetzung einer 
\BibArts\hy Liste wird gemeldet:

\vspace{-.75ex}{\scriptsize\begin{verbatim}
     [bibsort] Set \baoriginalTeX in line 51 of BibArts file. 
     [bibsort] Reproduce hyphenation 0 in line 52 of BibArts file. 
     [bibsort] Set \bagermanTeX in line 61 of BibArts file.  (new)
     [bibsort] Reproduce hyphenation 44 in line 62 of BibArts file.
\end{verbatim}}

\vspace{-1ex}\noindent
\verb|\baoriginalTeX| f"uhrt \verb|\originalTeX| aus, \verb|\bagermanTeX| 
f"uhrt selbst"andig \verb|\germanTeX| oder \verb|\ngermanTeX|
(mit Meldungen \texttt{\footnotesize (old)} oder \texttt{\footnotesize (new)}) aus, 
je nach dem, ob Sie \verb|german.sty|
oder \verb|ngerman.sty| geladen haben.\footnote{Die Zwischenstufe mit
\texttt{\bs baoriginalTeX} bzw.\ \texttt{\bs bagermanTeX} dient dazu, dass
Sie mit \texttt{\bs renewcommand} beide Definitionen ausschalten k"onnen,
falls es in Ihrem Text eine ganz andere Bedeutung hat, wenn sich der 
\textit{catcode} des Doppelanf"uhrungszeichens "andert.}
Ganz allgemein kommt \verb|bibarts.sty| damit klar, falls die
Zeichen \verb|~":;!?'`<>| \textit{aktiv} sein sollten, doch "Anderungen
\label{catcode}
des \textit{catcode} reproduziert \verb|bibsort| nur bez"uglich~\verb|"|.


\section{Formatierungs- und Editionshilfen}\label{Sect3}

Um Datumsangaben gutformatiert drucken zu k"onnen, verf"ugt 
\BibArts\ f"ur das\pdfko{1}\ 
Deutsche "uber den Befehl \verb|\te|, der einen Punkt und ein 
kurzes Leerzeichen (ohne Zeilenumbrucherlaubnis) druckt:
\hspace{.4em}\verb|Der 1\te April| \hspace{.4em}\verb|=>| \hspace{.4em}Der 1\te April.

\vspace{1ex}\noindent
F"ur englische Texte stellt das Paket 
\hspace{.1em}\verb|\eordinal{|\textit{arabische Zahl}\verb|}| 
\hspace{.2em}bereit:

\vspace{.75ex}{\small\noindent
\verb|     \eordinal{1} Assistant  => | \eordinal{1} Assistant. \\
\verb|     \eordinal{2} Assistant  => | \eordinal{2} Assistant. \\
\verb|     \eordinal{3} Assistant  => | \eordinal{3} Assistant. \\
\verb|     \eordinal{4} Assistant  => | \eordinal{4} Assistant. \\
\verb|    \eordinal{11} Assistant  => | \eordinal{11} Assistant. \\
\verb|    \eordinal{21} Assistant  => | \eordinal{21} Assistant.} 

\vspace{1ex}\noindent
Im Franz"osischen ergibt sich bei \verb|{1}| ein geschlechtsspezifischer Unterschied:

\vspace{.75ex}{\small\noindent
\verb|    Le \fordinalm{1} homme   => | Le \fordinalm{1} homme. \\
\verb|    La \fordinalf{1} femme   => | La \fordinalf{1} femme. \\
\verb|    Le \fordinalm{2} homme   => | Le \fordinalm{2} homme. \\
\verb|    La \fordinalf{2} femme   => | La \fordinalf{2} femme.}  

\vspace{1ex}\noindent
Die ordinal\hy Befehle dienen auch als Hilfsbefehle f"ur den Befehl 
\verb|\ersch| (oben S.\,\pageref{ersch}). \verb|\ersch| nutzt verschiedene 
ordinal\hy Befehle, wenn  \verb|\bacaptionsgerman|, 
\verb|\bacaptionsenglish| oder \verb|\bacaptionsfrench| gilt 
(vgl.\ Kapitel~\ref{SprachSep} unten ab S.\,\pageref{SprachSep}). 
\textit{Auf"|l.} und \textit{edition} lassen sich direkt "andern 
(\verb|\gerscheditionname| und \verb|\eerscheditionname| unten 
S.\,\pageref{erscheditionname}). Weil aber \verb|\ferscheditionname|
das feminine Wort \textit{\'edi\-tion} druckt, setzt \BibArts\ 
\verb|\fordinalf| in \verb|\ersch| ein.\footnote{\texttt{\bs ersch} 
nutzt unter \texttt{\bs bacaptionsgerman} statt \texttt{\bs te} den 
reinen Hilfsbefehl \texttt{\bs gordinal}.} Bei Wechsel zu einem maskulinen Wort 
m"ussten Sie zudem \verb|\fordinal| anpassen:\label{fordinalf}%

\vspace{.75ex}{\small\noindent
\verb|   \bacaptionsfrench| \\
\verb|     \ersch[1]{Paris}{1976}   => | {\bacaptionsfrench\ersch[1]{Paris}{1976} \\
\verb|       \renewcommand{\ferscheditionname}{\fupskip classement}| \\[-.25ex]
\verb|       \renewcommand{\fordinal}{\fordinalm}| \\[.1ex]
\verb|     \ersch[1]{Paris}{1976}   => | \renewcommand{\fordinal}{\fordinalm}%
\renewcommand{\ferscheditionname}{\fupskip classement,}%
\ersch[1]{Paris}{1976}}}

\vspace{1.25ex}\noindent
Zum Hochstellen von freien Texteingaben dient 
\verb|\fup{|\textit{\kern-.05em Text}\verb|}| 
\hspace{.2em}(\kern-.025em\textit{F}\kern-.05em rench \textit{up}). In\pdfko{.25}\  
schr"aggestelltem Umfeld wird automatisch eine \textit{italics}\hy
Korrektur gesetzt. Die l"asst sich nach \hspace{-.1em}\verb|.| mit \verb|\bahasdot| 
unterbinden (vgl.\ Kapitel~\ref{Punkte} ab 
S.\,\pageref{Punkte}):\footnote{Ist \texttt{\bs fup} bereits 
besetzt, etwa von \texttt{french.sty}, 
"uberschreibt \texttt{bibarts.sty} es \textit{nicht}!} 

\vspace{1ex}{\small\noindent
\verb|          S\fup{te} Claire             => |         S\fup{te} Claire \\
\verb|  \textit{S\fup{te} Claire}            => | \textit{S\fup{te} Claire} \\
\verb|  \textit{S.\fup{te} Claire}           => | \textit{S.\fup{te} Claire} \\
\verb|  \textit{S.\bahasdot\fup{te} Claire}  => | \textit{S.\bahasdot\fup{te} Claire}}


\newpage\noindent
F"ur \textit{Editionsarbeiten} (w"ortliches Zitieren) stellt \BibArts\ 
\verb|\abra{|\textit{Symbol}\verb|}| und\pdfko{1}\
\verb|\fabra{|\textit{Symbol}\verb|}| 
bereit. Als \textit{Symbol} lassen sich i.\,O. vergessene Satzzeichen 
nachtragen, die dann in eckigen Klammern (\kern-.1em\textit{a}ngular \textit{bra}ckets) 
ausgedruckt werden, um sie als \textit{editorische Zus"atze} zu kennzeichnen. 
Der Fixier\hy Befehl \verb|\fabra| verbietet einen Zeilenumbruch direkt
\textit{nach} dem \textit{Symbol}. 

Besonderheit der beiden Befehle ist, dass sie etliche kleine Symbole 
automatisch in \textit{h"ohenangepassten} Klammern ausdrucken. 'Unbekannte' 
Zeichen werden in ein normales eckiges Klammerpaar gesetzt. Bekannte Symbole sind:

\vspace{1.75ex}\noindent
   \verb|   \abra{,}      => | Rot\abra{,} blau und gr"un \\[-.5ex]
   \verb|   \abra{.}      => | waren die Farben\abra{.} Und \\[-.5ex]
   \verb|   \abra{...}    => | \abra{...} waren weitere \\[-.5ex]
   \verb|   \abra{\dots}  => | sowie \abra{\dots} mit \\[-.5ex]
   \verb|   \abra{$-$}    => | \abra{$-$} sagen wir \abra{$-$} \\[-.5ex]
   \verb|   \abra{-}      => | gr"un\abra{-} und gelb\abra{-}stichigen \\[-.5ex]
   \verb|   \abra{--}     => | Punkten \abra{--} wenn man so will.\\[-.5ex]
   \verb|   \abra{---}    => | Englischer\abra{---}Gedankenstrich.\\[-.5ex]
   \verb|  \fabra{`}      => | \hbox to 3em{\fabra{`}So,\hfill} \verb| % \glq und \grq werden| \\[-.5ex]
   \verb|  \fabra{'}      => | \hbox to 3em{so\abra{'}.\hfill}  \verb| %    auch bearbeitet|\\[-.5ex]
   \verb|  \fabra{\glqq}  => | Er sagte: \fabra{\glqq}Das \\[-.5ex]
   \verb|   \abra{\grqq}  => | kann nicht sein.\abra{\grqq} \\[-.5ex]
   \verb|  \fabra{"`}     => | Nochmal: \fabra{"`}Das \\[-.5ex]
   \verb|   \abra{"'}     => | kann nicht sein.\abra{"'} \\[-.5ex]
   \verb|   \abra{``}     => | Dann meiner er: ,,Gut!\abra{``} \\[-.5ex]
   \verb|   \abra{''}     => | ''Good?\abra{''} \\
   \verb|  \fabra{g}gf.   => | \hbox to 3em{\fabra{g}gf.\hfill} \verb| % unbekannt => 'grosse' Klammern|


\vspace{2ex}\noindent
Damit \BibArts\ die \textit{Symbole} erkennen kann, m"ussen sie genau "ubereinstimmen,
d"urfen also auch keine Leerzeichen enthalten. F"ur das \verb|"| in
\verb|"`| und \verb|"'| ist zudem Voraussetzung, dass es einen \textit{catcode} 
von 13 (\textit{aktiv}) hat, wie es nach Laden von \verb|german.sty| oder 
\verb|ngerman.sty| der Fall ist. \BibArts~2.1 arbeitet dabei zuverl"assiger als 
Version~2.0. Im \LaTeX\hy Original\hy US\hy Englisch ist nun 
\verb|\abra{"}| bzw.\ \verb|\fabra{"}| m"oglich ({\originalTeX\fabra{"}}).
Das Ersatzzeichen \verb|\abra{\dq}| ist im Deutschen ungeeignet; es expandiert 
zum dort verbotenen\hspace{-.1em} \verb|"|\hspace{.1em} ('zerbricht').

\vspace{1ex}\vfill\noindent
In den 'kleinen' Klammern der abra-Befehle setzt \BibArts\ 
die \textit{Symbole} aufrecht,
weil die sonst in einigen schr"aggestellten Schriften 
schlecht zentriert in den Klammern erscheinen w"urden. 
\verb|\abra| und \verb|\fabra| machen eine 
\textit{italics}\hy Korrektur. Ein direkt davor getipptes 
\verb|\bahasdot| unterbindet diese wiederum:

\vspace{1.75ex}\noindent
\verb|    \fabra{"`}Haus\abra{"'}         => | {\fabra{"`}Haus\abra{"'}} \\
\verb|  \itshape | \\
\verb|    \fabra{"`}Haus\abra{"'}         => | {\itshape \fabra{"`}Haus\abra{"'}} \\
\verb|    \fabra{"`}H.\abra{"'}           => | {\itshape \fabra{"`}H.\abra{"'}} \\
\verb|    \fabra{"`}H.\bahasdot\abra{"'}  => | {\itshape \fabra{"`}H.\bahasdot\abra{"'}}

\vspace{2ex}\noindent
Weil normale \textit{Minuszeichen} in Worten die Silbentrennung ausschalten,
stellt \BibArts\ zudem \verb|\hy| und \verb|\fhy| bereit. 
\verb|\hy| erlaubt die Trennung direkt nach dem gedruckten Minuszeichen 
(\verb|Haber\hy Bosch\hy Verfahren =>| Haber\hy Bosch\hy Verfahren),
w"ahrend \verb|\fhy| ein Minuszeichen druckt, das fest am
Folgewort klebt: \verb|Truppenaufmarsch und \fhy abzug =>| 
Truppenaufmarsch und \fhy abzug. Gegebenenfalls w"urde auch ab-zug getrennt
(anders als nach~\verb|"~|).

\vspace{1ex}\noindent
\verb|\hy| machte im Beispiel oben auch ein \textit{kerning} zum V\ko,
das es nach direkt angetippten Minuszeichen nicht gibt:
\hspace{.3em}\verb|Haber-Bosch-Verfahren| 
\hspace{.3em}\verb|=>| \hspace{.2em}Haber-Bosch-Verfahren. 
Das \textit{kerning} erfolgt vor A, T\ko, v, V\ko, w, W\ko, x, X, y und Y\ko, sowie vor 
\verb|`|, \verb|'|, \verb|\glq|, \verb|)|, \verb|]| und \verb|\}| automatisch.
Es funktioniert auch dann, wenn der Buchstabe \textit{einen} Akzent hat
(\kern-.1em\textit{aktives} \hspace{-.1em}\verb|"|, \verb|\"|, 
\verb|\.|, \verb|\=|, \verb|\^|, \verb|\'|, \verb|\`|, 
\verb|\~|,\pdfko{.75}\ 
\verb*|\accent |\textit{num}\verb*| |,
\verb|\b|, \verb|\c|, \verb|\d|, \verb|\H|, \verb|\k|, 
\verb|\r|, \verb|\u| oder \verb|\v|; nur \verb|\t| 
funktioniert nicht).\pdfko{.5}

\vspace{1.25ex}\noindent
\verb|  -Yser        => | -Yser \\
\verb|  \hy Yser     => | \hy Yser \\
\verb|  \hy\"Yser    => | \hy \"Yser \\
\verb|  \hy\"{Y}ser  => | \hy \"Yser 

\vspace{1.5ex}\noindent
Dieses automatische \textit{kerning} l"asst sich durch \verb|\nothyko| 
ausschalten (Wiedereinschalten mit \verb|\hyko|). Setzen von 
\verb|\hy{}|\hspace{-.15em}\textit{Wort} bzw.\ 
\verb|\fhy{}|\hspace{-.15em}\textit{Wort} unterbindet es ebenfalls. 
Das folgende Wort kann dann immer noch getrennt werden. 
In einem \texttt{typewriter}\hy Umfeld sollten Sie weiterhin 
Minuszeichen '\verb|-|' tippen.

\vspace{1.5ex}\noindent
\hspace*{-.16em}\textit{Vor} \verb|\hy| oder \verb|\fhy| kann 
$-$ falls ein penibler Textsatz gew"unscht ist $-$ kein 
automatisches \textit{kerning} durchgef"uhrt werden. \BibArts\ 
stellt den Korrekturbefehl \verb|\ko| bereit. Die Kosmetik 
ist (\kern-.05em\textit{wenn "uberhaupt!}) n"otig vor Gro"sbuchstaben, die sehr 
weit vom Minuszeichen entfernt sind: T\ko, V\ko, W und Y\ko. 

\vspace{1.25ex}\noindent
\verb|  T\hy Zacke     => | T\hy Zacke    \\
\verb|  T\ko\hy Zacke  => | T\ko\hy Zacke \\
\verb|  V\hy Form      => | V\hy Form     \\
\verb|  V\ko\hy Form   => | V\ko\hy Form

\vspace{1.5ex}\noindent
Die Definition von \verb|\ko| kann Ihnen als Beispiel f"ur "ahnliche Befehle dienen

\vspace{1.25ex}{\small\noindent
\verb|  \newcommand{\pko}{\ifhmode\nobreak\hskip -0.07em plus 0em\fi}| \\
\verb|  \newcommand{\ko}{\protect\pko}|\label{ProtectBeispiel}}

\vspace{1.5ex}\noindent
falls Sie die Korrektur zwischen V und Punkt oder Komma zu klein finden:

\vspace{1.25ex}\noindent
\verb|  V\te Armee     => | V\te Armee    \\
\verb|  V\ko\te Armee  => | V\ko\te Armee

\vspace{1.5ex}\noindent
Sicher w"are der Abstand von V und \hspace{-.1em}\verb|.| 
aber besser in den Ligaturtabellen definiert (worauf
\verb|\te| reagiert: \verb| P\te I und P{}\te I => | 
P\te I und P{}\te I).




\newpage
\section{Abk"urzungen}\label{Sect4}

\BibArts\ stellt Instrumente zur Verwaltung von Abk"urzungen zur Verf"ugung.
Dies betrifft nicht den Abstand zwischen Buchstaben, oder Buchstaben und
Punkten. Vielmehr k"onnen Sie Abk"urzungen in Ihrem Text weiterhin so
schreiben, wie Sie das wollen; Sie k"onnen aber \BibArts-Befehle nutzen, um
sich ein Abk"urzungsverzeichnis ausdrucken zu lassen und werden von
\verb|bibsort| darauf hingewiesen, ob Sie eine verwendete Abk"urzung bereits
f"ur Ihren Leser definierten. Spielregeln sind: Falls eine Abk"urzung f"ur
den Leser in einer Fu"snote bereits aufgel"ost wurde, darf sie \textit{in weiteren
Fu"snoten} ohne neuerliche Erkl"arung verwendet werden; erfolgte die
Definition der Bedeutung im Haupttext, darf die Abk"urzung danach "uberall
verwendet werden. Das Abk"urzungsverzeichnis wird in jedem Fall mit
Abk"urzung und Auf"|l"osung gef"uttert; \verb|bibsort| warnt, falls mehrfache
Auf"|l"osungen voneinander abweichen.

Abk"urzungen sind also zun"achst zu definieren. Dabei ist wahlfrei, ob erst
die Abk"urzung und dann ihre Auf"|l"osung gesetzt wird oder umgekehrt: 

\Doppelbox
{... eine
 \bs abkdef\{OHG\}\{Offene 
 \\ \ \ \ Handelsgesellschaft\}.
 \\ Oder: \bs defabk\{Offene 
 \\ \ Handelsgesellschaft\}\{OHG\}.
 Nun d"urfen Sie \bs abk\{OHG\} benutzen.
}
{Das Unternehmen ist eine 
 \abkdef{OHG}{Offene 
 Handelsgesellschaft}.
 Oder: \defabk{Offene 
 Handelsgesellschaft}{OHG}.
 Nun d"urfen Sie \abk{OHG} benutzen.
}\label{defabk}
 

\noindent
Falls Sie die weitere Abk"urzung \abk{GmbH} mit \verb|\abk{GmbH}| setzen,
aber nie definieren, wird sie nicht ins Abk"urzungsverzeichnis "ubernommen; 
stattdessen druckt \verb|bibsort| folgende Warnung auf den Bildschirm: 

\vspace{-.5ex}
{\scriptsize\begin{verbatim}
  %%>   Warning: Abbreviation "GmbH" is NEVER defined!
  %%     The entry (file 1 line 1528) is rejected. Use \abkdef?
\end{verbatim}}

\vspace{-.5ex}\noindent
Falls Sie die Abk"urzung mit \verb|\abkdef| oder
\verb|\defabk| definieren, dies im Texteditor aber in einer 
Zeile \textit{nach} \verb|\abk{GmbH}| tun, kommt sie ins 
Abk"urzungsverzeichnis. \verb|bibsort| warnt in seinem 
Bildschirmausdruck aber:

%\abk{GmbH}
%\abkdef{GmbH}{Gemeinschaft mit beschr"ankter Haftung}

\vspace{-.5ex}
{\scriptsize\begin{verbatim}
  %%>   Warning: Abbreviation "GmbH" is used in
  %%     file 1 line 1552 and def in file 1 line 1553!
\end{verbatim}}

\vspace{-.5ex}\noindent
Wie erw"ahnt sollen Abk"urzungen, die \textit{nur} in Fu"snoten
aufgel"ost werden, anschlie"send nicht im Haupttext
verwendet werden. Falls Sie tippen ...

\Doppelbox
{...\bs footnote\b{\{}Ein \bs abkdef\{e.\bs,V.\} 
 \\ \ \ \ \ \ \ \{eingetragener Verein\} hat 
 \\ \ mehrere Mitglieder.\b{\}} Der Verein hat \bs abk\{e.\bs,V.\} als Form.
}
{...\footnote{Ein \abkdef{e.\,V.} 
 {eingetragener Verein} hat 
 mehrere Mitglieder.} Der Verein hat \abk{e.\,V.} als Form.
}

\noindent
... erscheinen Abk"urzung und zugeh"orige Auf"|l"osung zwar im
Abk"urzungsverzeichnis, aber \verb|bibsort| macht die 
Bildschirm\hy Meldung:

{\scriptsize\begin{verbatim}
  %%>   Warning: Abbreviation "e.\,V." is used in
  %%     file 1 line 1560 and def in A FNT file 1 line 1561!
\end{verbatim}}

\noindent
Durch eine Eigenart von \LaTeX2e\ nennt die Meldung
die Zeilennummer, in der die Fu"snote endet,\footnote{In \LaTeX~2.09 evtl.\ auch bez"uglich der Zeile,
in der sie anf"angt.} w"ahrend \verb|\abkdef| im Beispiel sich tats"achlich
in einer vorausgehenden Editorzeile befand. Unabh"angig davon taucht \abk{e.\,V.} im 
Abk"urzungsverzeichnis auf, denn die Definition ist ja da. 

Das Verzeichnis
wird nun mit \verb|\printnumabklist| gedruckt. (Die Befehle \verb|\printabk|
und \verb|\printnumabk| erg"aben einen doppelspaltigen Ausdruck in 
\verb|\footnotesize| beginnend auf einer neuen Seite unter der "Uberschrift 
\textbf{Abk"urzungen}, was ich hier aus Platzgr"unden unterlasse).  

{\batwocolitemdefs\printnumabklist}\balabel{abklist}

\noindent
Die K"opfe der Listenpunkte wurden dabei in \verb|\abklistemph|
ausgedruckt, das\pdfko{1.25}\ 
defaultm"a"sig \verb|\bfseries| ausf"uhrt 
(\textbf{fett}). Die Seiten, von der die Definitionen\pdfko{.75}\  
herstammen, sind 
in der Auf"|listung von Seitenzahlen nicht hervorgehoben. Eine
Hervorhebung einzelner Seitenzahlen sieht \texttt{bibsort} 2.1 nie vor.

F"ur Abk"urzungen wie \printonlyabk{u.\,a.}, die Allgemeingut sind und deshalb
vielleicht nicht ins Abk"urzungsverzeichnis sollen, kann 
\verb|\printonlyabk{u.\,a.}| genutzt werden, um das 
Argument einheitlich in der Schrift aller Abk"urzungen 
ausgedruckt zu bekommen. Die Kontrolle durch \verb|bibsort| entf"allt 
dann. Radikaler k"onnen Sie insofern im Vorspann mittels 
\verb|\renewcommand{\abkemph}{}|
die Hervorhebung von Abk"urzungen ausschalten und dann \verb|u.\,a.| tippen.

Falls ein Eintrag ins Abk"urzungsverzeichnis soll, man sich die 
Auf"|l"osung im Text aber sparen will, hilft die bereits erw"ahnte 
\verb|{unused}|\hy Umgebung:

\vspace{-1.25ex}
{\small
\begin{verbatim}
  \begin{unused}
    \abkdef{u.\,a.}{unter anderem}   %HIER in Listen umgesetzt%
  \end{unused}
\end{verbatim}}
%
  \begin{unused}
    \abkdef{u.\,a.}{unter anderem}  %ist umgesetzt%
  \end{unused}

\vspace{-1ex}\noindent
Solche Definitionen kommen ohne Seiten\fhy/""Fu"snotennummer in die 
num\hy Liste. Zudem kann \verb|\abk{u.\,a.}| dann "uberall im Text 
(also auch davor) verwendet werden, ohne dass \verb|bibsort| das 
Fehlen der Auf"|l"osung bem"akelt. 

%Wenn das oben nur in einer Fu"snote erkl"arte \verb|\abk{e.\,V.}| jetzt  
%nochmal im Haupttext -- anders als oben: \textit{auf einer anderen Seite 
%als die Definition} -- verwendet wird (hier: \abk{e.\,V.}), meldet \verb|bibsort| 
%diesen Fehler "ubrigens anders:
%
%\vspace{-1ex}
%{\scriptsize\begin{verbatim}
%  %%>   Warning: Abbreviation "e.\,V." is used
%  %%     in TEXT file 1 line 1384, whereas all foregoing defs were in FNTs!
%\end{verbatim}}

\noindent
Mehrfach verwendete Abk"urzungen sowie mehrfach verwendete Auf"|l"osungen
m"ussen zeichengleich sein, um von \verb|bibsort| als gleich erkannt zu
werden. Wird das bereits oben aufgel"oste \abk{OHG}
nochmals erkl"art (vielleicht wollen Sie die Bedeutung einiger bereits definierter
Abk"urzungen am Anfang eines neuen Gro"skapitels nochmal erkl"aren),
wird dies akzeptiert. Wenn Sie dann aber \verb|\abkdef{OHG}{Offene Handelsgschaft}| 
tippen, meldet \verb|bibsort|:\addtoabkdef{OHG}{Offene Handelsgschaft}

\vspace{-.5ex}
{\scriptsize\begin{verbatim}
  %%>   Warning: Different defs for abbreviation "OHG":
  %%     *Accept file 1 line 1831 "Offene Handelsgesellschaft";
  %%     *Reject file 1 line 1960 "Offene Handelsgschaft".
\end{verbatim}}

\vspace{-1ex}\noindent
... und im Abk"urzungsverzeichnis erscheint nur die akzeptierte Variante.


\vspace{1ex}\noindent
Falls die Auf"|l"osung einer Abk"urzung im Abk"urzungsverzeichnis anders sein
soll als im Text, lassen sich die Befehle \verb|\abkdef| und \verb|\defabk|
aufsplitten in\pdfko{1.25}\ 
ihre Teilkomponenten. Vergleichen Sie "`(Kochsalz)"' hier und in der Liste:

\Doppelbox
{\vspace{.25ex}
 Das ist \bs addtoabkdef\{NaCl\} 
 \\ \ \{Natriumchlorid (Kochsalz)\}
 \bs printonlyabkdef\{NaCl\} 
 \\ \ \{Natriumchlorid\}.
 \vspace{.25ex}
}
{Das ist \addtoabkdef{NaCl} 
 {Natriumchlorid (Kochsalz)}
 \printonlyabkdef{NaCl} 
 {Natriumchlorid}.
}

\noindent
Dasselbe l"asst sich erreichen durch

\Doppelbox
{\vspace{.25ex}
 Das ist \bs abkdef\{NaCl\} 
 \\ \ \b{\{}Natriumchlorid\%
 \\ \ \ \ \bs onlyout\{ (Kochsalz)\}\b{\}}.
}
{\vspace{1.5ex}
 Das ist \abkdef{NaCl} 
  {Natriumchlorid%
         \onlyout{ (Kochsalz)}}.
}


\noindent
\verb|\abk| l"asst sich in \verb|\addtoabk| und 
\verb|\printonlyabk| aufspalten.
Es gibt damit zwei Arten f"ur eine in Text und Liste 
abweichende Gro"s\fhy/""Kleinschreibung:

\Doppelbox
{\vspace{.25ex}
 \bs printonlyabk\{E.\bs,V.\}\string's \bs addtoabk\{e.\bs,V.\} 
 sind beim Amtsgericht anzumelden. 
 \\[2ex]
 \bs abk\b{\{}\bs onlyhere\{E\}\%
  \\ \ \ \bs onlyout\{e\}.\bs,V.\b{\}} kann auch
 \\ alternativ so notiert sein.
 \vspace{.25ex}
}
{\printonlyabk{E.\,V.}'s \addtoabk{e.\,V.} 
 sind beim Amtsgericht anzumelden. 
 
 \vspace{1.1ex}
 \abk{\onlyhere{E}%
  \onlyout{e}.\,V.} kann auch
         alternativ so notiert sein.
}

\vspace{.25ex}\noindent
Wie angedeutet, werden das Argument von \verb|\abk| sowie
die Abk"urzungen in \verb|\abkdef| und \verb|\defabk| \textit{im Text} 
in der Schrift \verb|\abkemph| gedruckt; der Befehl f"uhrt
defaultm"a"sig \verb|\sffamily| aus ({\sffamily{sans serif}}). 
\verb|\renewcommand{\abkemph}{}| druckt Abk"urzungen in
Umfeldschrift aus. Sogar \verb|\itshape| oder \verb|\slshape| w"aren erlaubt; 
nur Befehle der Art \verb|\textbf| oder \verb|\textit| sind verboten.

\vfill\noindent
\textsf{Zu vorgefertigten Elementen im Listenausdruck siehe unten S.\,\pageref{Erklaerung}.}


\newpage
\section{\texttt{\bs abk\{X.X.X.\}} unter \texttt{\bs nonfrenchspacing}}\label{Sect5}\label{nonfrenchspacing}

\parbox{1.8em}{(1)}Falls Sie \verb|\nonfrenchspacing| einschalten (originaler 
\LaTeX\hy Textsatz mit\pdfko{1.25}\
vergr"o"serten Leerzeichen am Satzende), gilt in \LaTeX\ \textit{normalerweise} eine\pdfko{1.75}\ 
Vorschrift f"ur Abk"urzungen, die mit einem Kleinbuchstaben und einem
Punkt enden: \textbf{Wenn der Satz danach weiter geht}, ist 
\verb*|.\ |\hspace{.2em} zu tippen. 

Im Argument von \verb|\abk| ist dagegen egal, ob der letzte Buchstabe klein oder gro"s ist. 
\BibArts\ pr"uft, ob \textit{nach} dem Argument ein Punkt steht; falls nein, geht es davon aus, 
dass ein Leerzeichen mit 'normaler' L"ange zu setzen ist:\footnote{Falls
\texttt{ \}? \}! \}: \}; \}, }folgen, stellen die die Leerzeichenl"ange stets eigenst"andig 
ein.}\pdfko{1}

\vspace{1ex}
{\sffamily\nonfrenchspacing\noindent
\hbox to 10em{\texttt{ \ \ \ \ \ Dr. Maier }\hfill}\verb| => |\hbox to 6em{Dr. Maier\hfill} {\footnotesize\verb| %% falsch in US-Voreinstellung|} \\
\hbox to 10em{\texttt{ \ \ \ \ \ Dr.\bs\ Maier}\hfill}\verb| => |\hbox to 6em{Dr.\ Maier\hfill} {\footnotesize\verb| %% ueblich in US-Voreinstellung|} \\
\hbox to 10em{\texttt{ \bs abk\{Dr.\} Maier}\hfill}\verb| => |\hbox to 6em{\printonlyabk{Dr.} Maier\hfill}  {\footnotesize\verb| %% ausreichend in BibArts|} 
}


\vspace{2ex}\noindent
\parbox{1.8em}{(2)}Wenn dagegen eine \textbf{Abk"urzung am Satzende} steht, ist im 
\LaTeX\hy Standard '\verb|\nonfrenchspacing|' nur dann etwas zu
unternehmen, falls die Abk"urzung mit einem Gro"sbuchstaben
endet\hspace{.1em} (danach ist \verb|\@.|\hspace{.05em} statt\hspace{-.05em} \verb|.| zu setzen). 

Nach \verb|\abk| m"ussen Sie \textit{am Satzende} dagegen etwas 
unternehmen, \textit{wenn\pdfko{1.5}\ der Punkt zur Abk"urzung geh"ort}. 
Bei \verb|\abk{NASA.}| statt \verb|\abk{NASA}| etwa:

\vspace{1.25ex}
{\sffamily\nonfrenchspacing\noindent
\hbox to 10em{\texttt{ \ \ \ \ \ NASA\bs @.\ Next}\hfill}\verb| => |\hbox to 6em{NASA\@. Next\hfill} {\footnotesize\verb| %% US-Voreinstellung richtig|} \\
\hbox to 10em{\texttt{ \bs abk\{NASA\}.\ \ Next}\hfill}\verb| => |\hbox to 6em{\printonlyabk{NASA}. Next\hfill} {\footnotesize\verb| %% BibArts Typ 1 richtig|} \\
\hbox to 10em{\texttt{ \bs abk\{NASA.\} \ Next}\hfill}\verb| => |\hbox to 6em{\printonlyabk{NASA.} Next\hfill} {\footnotesize\verb| %% BibArts Typ 2 falsch!|} \\
\hbox to 10em{\texttt{ \bs abk\{NASA.\}.\ Next}\hfill}\verb| => |\hbox to 6em{\printonlyabk{NASA.}. Next\hfill} {\footnotesize\verb| %% BibArts Typ 2 richtig|}
}

\vspace{1.5ex}\noindent
Sie d"urfen den 'Satzende-Punkt' sprachunabh"angig zus"atzlich setzen.
Er wird 'verschluckt', wenn die Abk"urzung selbst
schon mit einem Punkt endet; ein folgendes Leerzeichen wird nur im
\verb|\nonfrenchspacing| verl"angert.\footnote{\BibArts\ 
pr"uft erst, ob \texttt{.} einen \texttt{\bs sfcode} von 3000 hat (gilt unter 
\texttt{\bs nonfrenchspacing}); falls das nicht gilt, 'verl"angert' es 
keine Leerzeichen. Unter \texttt{\bs frenchspacing} hat der Punkt einen
\texttt{\bs sfcode} von 1000; falls Sie einen dritten Wert verwenden, 
k"onnen Sie in einer Kopie von \texttt{bibarts.sty} alle 
3000er\hy Stellen gegen Ihre Zahl austauschten und die Kopie nutzen.}

In jedem Fall sollten Sie direkt nach dem letzten Argument eines \BibArts\hy Befehls
nie \verb|\@| setzen. In \texttt{bibarts.sty} ist die Behandlung von \verb|\@.| 
nie vorgesehen, auch nicht beim automatischen Setzen von \textit{italics}\hy 
Korrekturen!

\vspace{2ex}\noindent
Die hier genannten Spielregeln f"ur das \textit{spacing} gelten
auch f"ur andere \BibArts\hy Befehle (vgl.\ unten ab S.\,\pageref{Punkte}). 
So viel jetzt schon: Unter \verb|\frenchspacing| (gilt 
nach Laden von \verb|german.sty| oder \verb|ngerman.sty|) ist 
beim Schreiben an nichts zu denken, weil im deutschen
Textsatz alle Leerzeichen gleich gro"s sind. Sie m"ussen bei 
\verb|Typ 2| also nicht \verb*|.}. | setzen; und wenn Sie es doch 
t"aten, w"urde nur ein Punkt gedruckt (ohne Einfluss auf die 
folgende Leerzeichenl"ange).


\section{Zeitschriften und allgemein Bandangaben}\label{Sect6}\label{per}

\BibArts\ stellt zum Zitieren gedruckter Literatur als weitere Klasse
\textit{Zeitschriften}\pdfko{.5}\ bereit. Die kommen ins Argument von \verb|\per| 
(\textit{periodical}). Typischerweise steht\pdfko{.5}\ \verb|\per| im letzten 
Argument von \verb|\vli|, um Aufs"atze in Zeitschriften anzugeben. 
Nach dem Argument von \verb|\per| k"onnen \verb|_|\textit{Nummer und 
Erscheinungsjahr}\ko\verb|_|\pdfko{1}\ zwischen \textit{underscores} stehen. 
Vor \verb|_| darf kein Leerzeichen sein.

\Doppelbox
{
...\bs footnote\{\bs vqu \{John 
\\ \ \ Frederick Charles\} \{Fuller\} 
\\ \ \H{\{}Gold Medal (Military) 
\\ \ \ \bs ktit\{Prize Essay\} for 1919, 
\\ \ \ in: \bs per\{Journal of the
\\ \ \ \ \ \ \ \ Royal United Service 
\\ \ \ \ \ \ \ \ Institution\}\string_458 
\\ \ \ \ \ \ \ \ \ (1920)\string_[239-274]\H{\}}*[240].\}
\\[1.5ex]
...\bs footnote\{\bs kqu \{Fuller\} 
\\ \ \ \ \ \ \ \ \ \ \ \ \{Prize Essay\}[241].\}
\\[1.5ex]
...\bs footnote\{\bs vqu\{R[ichard]\} 
\\ \ \{Chevenix Trench\} 
\\ \ \{Gold Medal (Military) 
\\ \ \ \bs ktit\{Prize Essay\} for 1922, 
\\ \ \ in: \bs per\{Journal of the 
\\ \ \ \ \ \ \ \ Royal United Service 
\\ \ \ \ \ \ \ \ Institution\}\string_470 
\\ \ \ \ \ \ \ \ \ (1923)\string_[199-227]\}*[200].\}
}
{\notktitaddtok
\fbox{Als Beispiel gedruckte Quellen:}
\\[1.5ex]
...\footnote{\printonlyvqu {John\, Frederick\, Charles\,} {Fuller} 
{Gold Medal (Military) \ktit{Prize Essay} for 1919, 
in: \per{Journal of the Royal United Service 
Institution}_458 (1920)_[239-274]}*[240].\label{noNr}\label{Fuller1}}

...\footnote{\printonlykqu {Fuller} 
{Prize Essay}[241].\label{Fuller2}}

...\footnote{\printonlyvqu{R[ichard]\ \ } {Chevenix\ \ Trench} 
{Gold Medal (Military) \ktit{Prize Essay} for 1922, 
in: \per{Journal of the Royal United Service 
Institution}_470 (1923)_[199-227]}*[200].\balabel{Nr}\label{Trench}}
}

\noindent
Die innere und "au"sere Wiederholung ergab \textsc{ebd.};
und \verb|\printnumvkc| druckt

\vspace{-0.4ex}
{\small\begin{description}\parsep 0ex \itemsep -.5ex
\item \textsc{Chevenix Trench}: Prize Essay [Q] \ \textsf{\pageref{Trench}$^{\ref{Trench}}$}
\item \textsc{Fuller}: Prize Essay [Q] \ \textsf{\pageref{Fuller1}$^{\ref{Fuller1},\ \ref{Fuller2}}$}
\end{description}}

\vspace{-0.25ex}\noindent
In die Liste \verb|\printvqu| kommt (in \verb|bibarts.vqu| tats"achlich nicht umgesetzt):

\vspace{-0.4ex}
{\small
\begin{description}\parsep 0ex \itemsep -.5ex
\item \textsc{Chevenix Trench}, R[ichard]: Gold Medal (Military) Prize Essay for
1922, in:\pdfko{.5}\ 
\textsc{Journal of the Royal United Service Institution} 470 (1923), S.\,199-227.
\item \textsc{Fuller}, John Frederick Charles: Gold Medal (Military) Prize Essay for
1919, in: \textsc{Journal of the Royal United Service Institution} 458 (1920), S.\,239-274.
\end{description}}

\vspace{-0.25ex}\noindent
Au"serdem lassen sich die verwendeten Zeitschriften in einer separaten Liste drucken. 
M"oglich ist, dabei an einzelne Eintr"age \textit{Zusatztext} anzuh"angen: 

\vspace{.4ex}%
{\footnotesize
\hspace{1em}\verb|\fillper{Journal of the Royal United Service Institution}| \\[-.5ex]
\hspace*{6.95em}\verb|{Zeitschrift gegr|"u\verb|ndet 1857}     %% ist umgesetzt %%|
}
\fillper{Journal of the Royal United Service Institution}
{Zeitschrift gegr"undet 1857}%

\vspace{.75ex}\vfill\noindent
\verb|\printnumper| druckt die Liste der Zeitschriften (das \verb|.per|-File) dann so:

\printnumper\label{printnumper}

\noindent
Der nur einmal gesetzte fill\hy Befehl diente dazu, einen Zusatz anzuh"angen, 
der zur Vereinfachung nicht bei jedem Zitat aus der Zeitschrift getippt
werden soll. Zu den Gedankenstrichen vor den fill\hy Eintr"agen 
siehe \verb|$-$| unten S.\,\pageref{perlistopen}. 

\vspace{1ex}\noindent
Wie nach allen \BibArts\hy Befehlen (vgl.\ S.\,\pageref{Hauptbefehle}) 
sind Sie frei, \verb+|+\ko\textit{Bandangaben}\ko\verb+|+ \textbf{oder}\pdfko{.75}\
\verb+_+\textit{Heftnummern}\verb+_+ zu setzen. Beide drucken jeweils 
eigene vorgefertigte Textelemente (\textit{captions}). Im Text hier 
wurden f"ur Zeitschriften die \textit{underscores}\pdfko{.1}\ gew"ahlt, die 
mit der Nennung der \textit{Heftnummer} beginnen. In der letzten Fu"snote \baref{Nr} 
stand nach \textsc{ebd.} zus"atzlich {\footnotesize Nr.}, was
fehlt, wenn\pdfko{.75}\ keine \textsc{ebd.}\hy Setzung erfolgt (Anm.\,\ref{noNr}). 
Dies geht zur"uck auf die Definitionen:

\vspace{-1ex}
{\small
\begin{verbatim}
  \gpername   =>  {\ifbaibidem{, Nr.\,}{\pernosep}}    % _X_
  \gperpname  =>  {\ifbaibidem{, Nr.\,}{\pernosep}}    % _X, Y_
\end{verbatim}}\label{pernosep}

\vspace{-.75ex}\noindent
worin \verb|\ifbaibidem| sein erstes Argument im Ebenda-Fall und
sonst sein zweites Argument ausf"uhrt (das ein Leerzeichen druckt). 
\verb|\gperpname| $-$\,Plural\,$-$ f"uhrt
\BibArts\ statt \verb|\gpername| dann aus, wenn im Argument zwischen
den \textit{underscores} sich ein Minuszeichen, ein Komma, \verb|\f|
oder \verb|\ff| findet, also eine\pdfko{.75}\ 
Auf"|listung von mehreren Zeitschriftennummern enthalten ist. 

Dies gilt "aquivalent f"ur \verb+|+\ko\textit{Bandangaben}\ko\verb+|+, 
die besonders nach dem letzten\pdfko{.25}\ Pflichtargument von 
v- oder k\fhy Befehlen stehen d"urfen (vgl.\ oben S.\,\pageref{Reinhard}):

\vspace{-1ex}
{\small
\begin{verbatim}
  \gvolname   =>  {, Bd.\,}                            % |X|
  \gvolpname  =>  {, Bde.\,}                           % |X, Y|
\end{verbatim}}

\vspace{-.75ex}\noindent
wobei Singular und Plural erkennbar unterschiedliche Separatoren drucken:

\vfill
\Doppelbox
{\vspace{.75ex}%
  \bs footnote\{Wieder \bs xkqu \{Marx\} 
\\[-1pt] \ \ \ *\{\bs kauthor\{Engels\}\} 
\\[-1pt] \ \ \ \{Werke\}\string|11-13\string|.\}
\\[3pt] \bs footnote\{\bs xkqu \{Marx\} 
\\[-1pt] \ \ \ *\{\bs kauthor\{Engels\}\} 
\\[-1pt] \ \ \ \{Werke\}\string|14\string|.\}
\\[3pt] \bs footnote\{\bs xkqu \{Marx\} 
\\[-1pt] \ \ \ *\{\bs kauthor\{Engels\}\} 
\\[-1pt] \ \ \ \{Werke\}\string|15\bs f\string|.\}
\\[3pt] \bs footnote\{\bs xkqu \{Marx\} 
\\[-1pt] \ \ \ *\{\bs kauthor\{Engels\}\} 
\\[-1pt] \ \ \ \{Werke\}\string|17, 18\string|.\}
\vspace{.75ex}%
}
{
\footnote{Wieder \xkqu {Marx} *{\kauthor{Engels}} {Werke}|11-13|.}

\footnote{\xkqu {Marx} *{\kauthor{Engels}} {Werke}|14|.}

\footnote{\xkqu {Marx} *{\kauthor{Engels}} {Werke}|15\f|.}

\footnote{\xkqu {Marx} *{\kauthor{Engels}} {Werke}|17, 18|.}
}

\noindent 
\verb|\gpername|, \verb|\gperpname| sowie \verb|\gvolname| 
und \verb|\gvolpname| lassen sich etwa\pdfko{.5}\ 
mittels \verb|\renewcommand{\gpername}{, Heft }| ver"andern
(Beispiel ohne \textit{if}):

\vspace{.7ex}
\verb+\per{ZfG.}_5_.+\hspace{1em} \verb|=>|\hspace{.5em} 
{\renewcommand{\gpername}{, Heft }\per{ZfG.}_5_.}

\vspace{.9ex}\noindent
Falls \BibArts\ nach v\fhy, k- oder per\hy Befehlen im Eintrag zwischen den 
\textit{senkrechten\pdfko{.5}\ Strichen} bzw.\ zwischen den \textit{underscores} 
Singular und Plural nicht richtig erkennt, l"asst sich mit 
\verb|\basingular| bzw.\ \verb|\baplural| \textit{am Ende} 
nachjustieren:\label{baplural}

\vspace{-.25ex}
\Doppelbox
{
\bs footnote\{\bs per \{ZfG.\}\string|11 u. 
\\ \ \ 13\string|.\}
\\[1ex]
\bs footnote\{\bs per \{ZfG.\}\string|11 u. 
\\ \ \ 13\bs baplural\string| (erzwungen).\}
\\[2ex]
\bs footnote\{\bs per \{ZfG.\}\string|17, 
\\ \ \ 18 oder 19\string|.\}
\\[1ex]
\bs footnote\{\bs per \{ZfG.\}\string|17, 
\\ \ \ 18 oder 19\bs basingular\string| (dito).\}
\\[1ex]
\bs footnote\{\bs per \{ZfG.\}\string_17, 
\\ \ \ 18 oder 19\bs basingular\string_.\}
\\[1ex]
\bs footnote\{\bs per \{ZfG.\}\string_17, 
\\ \ \ 18 oder 19\string_.\}
}
{
\footnote{\per {ZfG.}|11 u. 13|.}

\footnote{\per {ZfG.}|11 u. 13\baplural| (erzwungen).}

\footnote{\per {ZfG.}|17, 18 oder 19|.}

\footnote{\per {ZfG.}|17, 18 oder 19\basingular| (dito).}

\footnote{\per {ZfG.}_17, 18 oder 19\basingular_. 
   \texttt{ \ \% Wechsel auf \string_...\string_ \%}\label{Wechsel}}

\footnote{\per {ZfG.}_17, 18 oder 19_.}
}

\vspace{-.25ex}\noindent
\verb+|+...\verb+|+ und \verb+_+...\verb+_+ f"ullen also denselben Speicher. 
Falls Sie zwischen beiden unbeabsichtigt wechseln (vgl.\ Anm.\,\ref{Wechsel}), 
erhalten Sie keine Warnung. (\BibArts\pdfko{1}\ macht auch hier nur die oben 
S.\,\pageref{before} und \pageref{pervol} beschrieben Fehlermeldungen.)

\vspace{1ex}\noindent
Wenn Zeitschriften abgek"urzt werden \textit{und} die Abk"urzung 
zus"atzlich im Abk"urzungsverzeichnis erscheinen soll, vereinfacht
dies \verb|\abkper|: Das f"uhrt \verb|\per|\pdfko{1}\ aus 
(Liste S.\,\pageref{printnumper}) und zus"atzlich
\verb|\addtoabk| f"ur das Abk"urzungsverzeichnis\pdfko{.5}\  
\baref[siehe {\abklistemph{ZfG.}}]{abklist}. Die Abk"urzung 
erscheint dort nur, wenn sie definiert ist:

\vspace{-.75ex}
\Doppelbox
{\vspace{.5ex}
 Die \bs abkper \{ZfG.\} \bs addtoabkdef\{ZfG.\}\{Zeitschrift 
 \\ \ \ \ \ f"ur Geschichtswissenschaft\} ist ... Satzende:
 \bs abkper\{ZfG.\}.
 \vspace{.25ex}
}
{\vspace{.75ex}%
 Die \abkper {ZfG.} \addtoabkdef{ZfG.}{Zeitschrift f"ur 
 Geschichtswissenschaft} ist eine wissenschaftliche Zeitschrift.
 Am Satzende: \abkper{ZfG.}.
}

\vspace{-.5ex}\noindent
Hinter dem Hauptargument von \verb|\abkper| d"urfen Angaben zu
Heftnummern und Seitenzahlen stehen wie nach jedem \BibArts\hy Befehl.
%<<<

\vspace{1.25ex}\noindent
\verb|\per{|\textsc{Argument}\verb|}| und \verb|\abkper{|\textsc{Argument}\verb|}| 
werden in \verb|\peremph| ausgedruckt. 
Dessen Definition darf nicht leer sein; zumindest \verb|\upshape| 
sollte\pdfko{.1}\ darin stehen $-$ denn mit \verb|\renewcommand{\peremph}{}| 
allein w"urden alle per\hy Befehle, die in schr"aggestelltem 
Schriftumfeld stehen, etwas melden wie:

\vspace{.5ex}{\scriptsize\begin{verbatim}
   BibArts Warning: Add \upshape to \peremph on input line 1696.
\end{verbatim}}

%>>>

\noindent
Die Aufgaben von \verb|\per| lassen sich in
\verb|\addtoper| und \verb|\printonlyper| teilen: 

\vspace{-.25ex}
\Doppelbox
{
Die \bs printonlyper\{Zeitschrift
\\ \ \ \ f"ur Geschichtswissenschaft\} 
\bs addtoper\{ZfG.\} 
\\ soll als Abk"urzung ins Zeitschriftenverzeichnis. 
\\[.25ex] Alternativ gibt auch 
\bs per\b{\{}Z\bs onlyhere\{eitschrift \}\%
\\ \ \ \ \ \ f\bs onlyhere\{"ur \}\%
\\ \ \ \ \ \ G\bs onlyhere
\\ \ \ \ \ \ \ \{eschichtswissenschaft\}\%
\\ \ \bs onlyout\{.\}\b{\}} nur einen Eintrag.
}
{
Die \printonlyper{Zeitschrift f"ur Geschichtswissenschaft} 
\addtoper{ZfG.} 
soll als Abk"urzung ins Zeitschriftenverzeichnis.
Alternativ gibt auch 
\per{Z\onlyhere{eitschrift }%
f\onlyhere{"ur }%
G\onlyhere{eschichtswissenschaft}%
\onlyout{.}} nur einen Eintrag.
\\[.75ex]
\fbox{\parbox{.95\textwidth}{Vergleichen Sie dazu die Angabe 
 der Seite \label{hier}{\balistnumemph{\pageref{hier}}} hier
 nach {\perlistemph{ZfG.}}\ im Zeitschriftenverzeichnis 
 oben S.\,{\pageref{printnumper}}.}}\vspace{-.75ex}
}

\vspace{.5ex}\noindent
Manche Verlage wollen vor der \verb+|+\textit{Nummer}\verb+|+ eines
Bandes oder eines Jahrgangs immer nur ein Leerzeichen haben (statt "`Bd."').
Deshalb l"asst sich das voreingestellte \verb|\printlongpervol|
auf \verb|\notprintlongpervol| umstellen. Der Ausdruck von Zeitschriften 
und Literatur wird "ahnlicher (nach [L] bleibt das Komma aber stehen;
in \verb+_+\textit{Heftnummern}\verb+_+ f"allt bei \textsc{ebd.}\hy Setzung  
die {\footnotesize Nr.} weg):

\Doppelbox
{
Voreinstellung.\bs footnote\{ 
\\ \ \ \bs kqu\{Marx\}\{Kapital\}\string|1\string|[2].\}
\\[1ex] \bs footnote\{Kein Ebenda.\}
\\[4ex] \bs notprintlongpervol
\\[1ex] \bs footnote\{ 
\\ \ \ \bs kqu\{Marx\}\{Kapital\}\string|1\string|[3].\}
\\[1ex] \bs footnote\{ 
\\ \ \ \bs kqu\{Marx\}\{Kapital\}\string|1\string|[4].\}
}
{
Voreinstellung.\footnote{
\kqu{Marx}{Kapital}|1|[2].}

\footnote{Kein Ebenda.}

\

\notprintlongpervol
\footnote{
\kqu{Marx}{Kapital}|1|[3].}

\footnote{
\kqu{Marx}{Kapital}|1|[4].}
}\label{notprintlongpagefolio}

\vfill\noindent
Soll vor der Seitenzahl ein Doppelpunkt statt "`S."' stehen, 
kann au"serdem \verb|\notprintlongpagefolio| gesetzt werden:

\Doppelbox
{
 \bs notprintlongpagefolio
 \\[.5ex] \bs footnote\{ 
 \\ \ \ \bs per\{ShortMagazine\}\string_25\string_[4].\}
 \\[.2ex] \bs footnote\{ 
 \\ \ \ \bs per\{ShortMagazine\}\string_25\string_[5].\}
 \\[.2ex] \bs footnote\{ 
 \\ \ \ \bs per\{ShortMagazine\}\string_26\string_[6].\}
 \\[1ex] \bs notprintlongpervol
 \\[.5ex] \bs footnote\{
 \\ \ \ \bs per\{ShortMagazine\}\string_27\string_[7].\}
}
{
 \notprintlongpagefolio
 \footnote{\per {ShortMagazine}_25_[4].}

 \footnote{\per {ShortMagazine}_25_[5].}

 \footnote{\per{ShortMagazine}_26_[6].}

 \notprintlongpervol
 \footnote{\per{ShortMagazine}_27_[7].}
}

\noindent
\verb|\notprintlongpagefolio| allein ergibt (dann druckt
\verb|(|\textit{n}\verb|)| auch nicht "`Bl."'):

\vspace{-.5ex}
\Doppelbox
{
\bs notprintlongpagefolio
\\[1ex] \bs footnote\{ 
\\ \ \ \bs kqu\{Marx\}\{Kapital\}\string|25\string|[4].\}
\\[1ex] \bs footnote\{ 
\\ \ \ \bs kqu\{Marx\}\{Kapital\}\string|25\string|[5].\}
}
{
\notprintlongpagefolio
\footnote{
\kqu{Marx}{Kapital}|25|[4].}

\footnote{
\kqu{Marx}{Kapital}|25|[5].}
}

\vspace{-.5ex}\noindent
Das folgende Beispiel zeigt (unten), was Setzen von
\verb|\notprintlongpervol| und \verb|\notprintlongpagefolio| 
zusammen mit \verb|\notannouncektit| ergibt:

\vspace{-.25ex}
\Doppelbox
{
\bs notprintlongpagefolio
\\[1ex]
...\bs footnote\{\bs vqu \{John 
\\ \ \ Frederick Charles\} \{Fuller\} 
\\ \ \H{\{}Gold Medal (Military) 
\\ \ \ \bs ktit\{Prize Essay\} for 1919, 
\\ \ \ in: \bs per\{Journal of the
\\ \ \ \ \ \ \ \ Royal United Service 
\\ \ \ \ \ \ \ \ Institution\}\string_458 
\\ \ \ \ \ \ \ \ \ (1920)\string_[239-274]\H{\}}*[240].\}
\\[1.5ex]
...\bs footnote\{\bs kqu \{Fuller\} 
\\ \ \ \ \ \ \ \ \ \ \ \ \{Prize Essay\}[241].\}
\\[1.5ex]
\bs notannouncektit
\bs notprintlongpervol
\\[1ex]
...\bs footnote\{\bs vqu\{R[ichard]\} 
\\ \ \{Chevenix Trench\} 
\\ \ \{Gold Medal (Military) 
\\ \ \ \bs ktit\{Prize Essay\} for 1922, 
\\ \ \ in: \bs per\{Journal of the 
\\ \ \ \ \ \ \ \ Royal United Service 
\\ \ \ \ \ \ \ \ Institution\}\string_470 
\\ \ \ \ \ \ \ \ \ (1923)\string_[199-227]\}*[200].\}
}
{\notktitaddtok
\notprintlongpagefolio
\texttt{\% Solche Befehle sollten nur}\\
\texttt{\% global gesetzt werden; hier}\\
\texttt{\% geht es darum, die Konse-}\\
\texttt{\% quenzen zu demonstrieren.}
\\[6ex]
...\footnote{\printonlyvqu {John\, Frederick\, Charles\,} {Fuller} 
{Gold Medal (Military) \ktit{Prize Essay} for 1919, 
in: \per{Journal of the Royal United Service 
Institution}_458 (1920)_[239-274]}*[240].\label{erste}}

...\footnote{\printonlykqu {Fuller} 
{Prize Essay}[241].}

\notannouncektit
\notprintlongpervol
...\footnote{\printonlyvqu{R[ichard]\ \ } {Chevenix\ \ Trench} 
{Gold Medal (Military) \ktit{Prize Essay} for 1922, 
in: \per{Journal of the Royal United Service 
Institution}_470 (1923)_[199-227]}*[200].}
}

\noindent
\verb|\notprintlongpervol| ordnete an, dass in der letzten Fu"snote
nach dem inneren \textsc{ebd.} (vor {\footnotesize 470}) kein {\footnotesize Nr.}\ 
ausgedruckt wurde. In Fu"snote~\ref{erste} fehlte {\footnotesize Nr.}\pdfko{.5}\
bereits, weil dort kein \textsc{ebd.} gesetzt ist und
in \verb|\gpername| das zweite Argument von \verb|\ifbaibidem| dann 
\verb|\pernosep| ausdruckt, ein gesch"utztes Leerzeichen.\footnote{\label{pernosep2}
\texttt{\bs renewcommand\{\bs pernosep\}\{\bs ifbashortcite\{\bs bastrut\bs bakxxcorr\}\{\}\bs\ \}}
\\ w"urde stattdessen einen Zeilenumbruch am Leerzeichen erlauben.}\pdfko{.5}

\vspace{1ex}\noindent
Die Befehle sind auch auf Listen anwendbar. Ausgedruckt
werden w"urde beispielsweise ein Eintrag unter 
\verb|{\notprintlongpagefolio\printvqu}| so:

\vspace{-1ex}
{\footnotesize
\begin{description}\parsep 0ex \itemsep -.5ex
\item \textsc{[Anonym]}: Aufmarschanweisungen 1912, 
abgedruckt in: \textsc{Ehlert}\baslash \textsc{Epkenhans}\baslash \textsc{Gro"s}\pdfko{1}\
[Hrsg.]: Schlieffenplan [Q]: 462-466.
\end{description}}


\newpage
\section{Archivquellen}\label{Sect7}\label{archivquellen}

Historische Forschungsliteratur weist h"aufig ein separates Verzeichnis 
f"ur \textit{ungedruckte Quellen} auf, die \BibArts\ wiederum aus 
Haupttext oder Fu"snoten gewinnen kann. Zudem ist eine korrekte 
\textsc{ebd.}\hy Setzung in Fu"snoten n"otig. Beides bew"altigt der Befehl 
\verb|\arq| mittels zwei Pflicht- und zwei optionalen Argumenten. 
Das erste Pflichtargument nennt ein Schriftst"uck und das zweite eine 
Archivsignatur (eventuell samt Eigennamen des Quellenbestandes). Die 
\textsc{ebd.}\hy Setzung kann mal Schriftst"uck \textit{und} Signatur betreffen, 
mal nur die\pdfko{.25}\ 
Signatur (wenn Sie ein anderes Schriftst"uck aus demselben 
Bestand zitieren). Nur das zweite Pflichtargument kommt ins Verzeichnis 
ungedruckter Quellen.

Wenn in aufeinanderfolgenden Fu"snoten verschiedene Mappen eines 
Bestandes mit gleicher Hauptsignatur stehen, kann \BibArts\ daraus etwas 
wie\pdfko{1}\ 
{\footnotesize\textsc{Ebd.}, Bd.\,2} machen. Bandangaben sind dazu in 
senkrechte Striche nach dem\pdfko{.5}\  
zweiten Pflichtargument zu setzen. Und wenn die Schriftst"ucke Nummern von 
einem Paginierungsstempel haben, k"onnen sie abschlie"send in 
runden Klammern stehen. Vor {\small\verb+|+\textit{Band}\verb+|+} und 
vor {\small\verb+(+\textit{Blatt}\verb+)+} darf kein Leerzeichen stehen:

\Doppelbox
{\strut\\[-2ex]
...\bs footnote\{\bs arq\{Haber 
\\ \ am 17.12.1914 an 
Kultusminister\} \{GStAPK, HA\bs ,1, Rep\string~76\string~Vc, 
\\ \ Sekt\string~1, 
Tit\string~23, Litt\string~A, 
\\ \ Nr.\bs ,108\}\string|2\string|(223\bs f).\}
\\[1ex]
...\bs footnote\{\bs arq\{Setsuro Tamaru 
\\ \ \ \ \ am 24.12.1914 an 
Clara Haber\} \{GStAPK, HA\bs ,1, Rep\string~76\string~Vc, 
\\ \ Sekt\string~1, 
Tit\string~23, Litt\string~A, 
\\ \ Nr.\bs ,108\}\string|2\string|(226-231).\}
\\[1ex]
...\bs footnote\{\bs arq\{Setsuro Tamaru 
\\ \ \ \ \ am 24.12.1914 an 
Clara Haber\} \{GStAPK, HA\bs ,1, Rep\string~76\string~Vc, 
\\ \ Sekt\string~1, 
Tit\string~23, Litt\string~A, 
\\ \ Nr.\bs ,108\}\string|2\string|(226-231).\}
\\[1ex]
...\bs footnote\{\bs arq\{Valentini am 
\\ \ \ \ 13.3.1911 an Schmidt\} \{GStAPK, 
HA\bs ,1, Rep\string~76\string~Vc, Sekt\string~1, Tit\string~23, Litt\string~A, 
Nr.\bs ,108\}\string|1\string|(47).\}
\\[-2.25ex]\strut
}
{
...\footnote{\arq{Haber am 17.12.1914 an 
Kultusminister} {GStAPK, HA\,1, Rep~76~Vc, Sekt~1, 
Tit~23, Litt~A, Nr.\,108}|2|(223\f).}

...\footnote{\arq{Setsuro Tamaru am 24.12.1914 an 
Clara Haber} {GStAPK, HA\,1, Rep~76~Vc, Sekt~1, 
Tit~23, Litt~A, Nr.\,108}|2|(226-231).}

...\footnote{\arq{Setsuro Tamaru am 24.12.1914 an 
Clara Haber} {GStAPK, HA\,1, Rep~76~Vc, Sekt~1, 
Tit~23, Litt~A, Nr.\,108}|2|(226-231).}

...\footnote{\arq{Valentini am 13.3.1911 an Schmidt} 
{GStAPK, HA\,1, Rep~76~Vc, Sekt~1, Tit~23, Litt~A, 
Nr.\,108}|1|(47).} 
}

\vspace{-.25ex}\noindent
Blattnummern werden (deutsch) im Singular und Plural gleich angek"undigt:

\vspace{-.5ex}
{\small
\begin{verbatim}
  \gisonfolioname  => {, Bl.\,}
  \gisonfoliopname => {, Bl.\,}
\end{verbatim}}

\vspace{-.25ex}\vfill\noindent
Falls Sie sowohl den Blattbereich eines mehrseitigen Schreibens als auch 
das darin zitierte Blatt angeben m"ochten, sollte beides in die runden Klammern:

\newpage
\Doppelbox
{\strut\\[-2ex]
\bs footnote\{Mehrseitig: 
\\ \bs arq\{Haber am 3.5.1913 an Kr"uss\} \{GStAPK, HA\bs ,1, 
Rep\string~76\string~Vc, Sekt\string~1, Tit\string~23, Litt\string~A, 
Nr.\bs ,108\}\string|2\string|(94\bs ,a-e: a).\}
\\[1.5ex]
\bs footnote\{
\\ \bs arq\{Haber am 3.5.1913 an Kr"uss\} \{GStAPK, HA\bs ,1, 
Rep\string~76\string~Vc, Sekt\string~1, Tit\string~23, Litt\string~A, 
Nr.\bs ,108\}\string|2\string|(94\bs ,a-e: b).\}
\\[-2.25ex]\strut
}
{
\footnote{Mehrseitig: 
\arq{Haber am 3.5.1913 an Kr"uss} {GStAPK, HA\,1, 
Rep~76~Vc, Sekt~1, Tit~23, Litt~A, Nr.\,108}|2|(94\,a-e: a).}

\footnote{
\arq{Haber am 3.5.1913 an Kr"uss} {GStAPK, HA\,1, 
Rep~76~Vc, Sekt~1, Tit~23, Litt~A, Nr.\,108}|2|(94\,a-e: b).}
}

\noindent
Eine Alternative w"are, \verb|(94\,a-e)| nur beim Verweis auf das 
ganze Schreiben zu tippen und zum Beleg einer speziellen Stelle
nur z.\,B. \verb|(94\,c)| anzugeben.

\vspace{1ex}\noindent
Zum \textbf{Ausdruck des Archivquellenverzeichnisses} lassen sich 
optional "Uberschriften hinzuf"ugen. Zur richtigen Einsortierung 
in der Liste m"ussen sie mit \textit{den ersten Buchstaben
der "uberschriebenen Signaturen} beginnen:

\vspace{1.25ex}
{\small
\verb|  \arqsection{GStAPK}{Geheimes Staatsarchiv|\\
\verb|                         Preu|\texttt{"s}\verb|ischer Kulturbesitz}|\\
\verb|  \arqsection{BA} {Bundesarchiv}    %% Arg 1 bestimmt Einsortierung|
}
  \arqsection{GStAPK}{Geheimes Staatsarchiv 
                      Preu"sischer Kulturbesitz}      
  \arqsection{BA} {Bundesarchiv}    %% Arg 1 bestimmt Einsortierung

\vspace{1.5ex}\noindent
\verb|\arqsubsection| erzeugt eine Unter\fhy,
\verb|\arqsubsubsection| eine Unterunter\hy "Uberschrift
(und muss je in \underline{mehr} Zeichen mit den "uberschriebenen 
Signaturen "ubereinstimmen). 
\verb|\arqsubsection {GStAPK|\underline{\texttt{, HA}}\verb|} {Hauptabteilung}|
\arqsubsection {GStAPK, HA} {Hauptabteilung} wurde hier verwendet. 
Und die Zahl der B"ande in Bestand Nr.\,108 wird mit diesem
unsichtbaren fill\hy Befehl an den zugeh"origen \verb|\arq|\hy Zugang angeh"angt:\pdfko{1}

\vspace{-.25ex}\noindent
{\small
\begin{verbatim}
  \fillarq{GStAPK, HA\,1, Rep~76~Vc, Sekt~1, 
                     Tit~23, Litt~A, Nr.\,108} {2\,Bde.}
\end{verbatim}}
  \fillarq{GStAPK, HA\,1, Rep~76~Vc, Sekt~1, 
                     Tit~23, Litt~A, Nr.\,108} {2\,Bde.}


\vspace{-.25ex}\noindent
\verb|\printarq| druckt die von \verb|bibsort| erzeugte 
\verb|.arq|\hy Datei so aus (vgl.\ S.\,\pageref{newpage}):

\vfill
\printarq


\newpage
Falls Sie \verb|\arq| in \verb|\printonlyarq| und \verb|\addtoarq| 
separieren, behalten beide die zwei Pflichtargumente. 
Bei \verb|\addtoarq| scheint dies "uberfl"ussig zu sein, weil in 
der Liste nur die Hauptsignatur gedruckt wird. Allerdings wird 
das erste Argument von \verb|\arq| genauso wie hier dasjenige von 
\verb|\addtoarq|~...

\Doppelbox
{
...\bs footnote\b{\{} \bs printonlyarq\{\% 
\\ \ \ Gesellschaftsvertrag der KCAG\} 
\\ \ \{BA R\bs ,8729\string~4\}(94). 
\\[.25ex] \ \bs addtoarq\{Gesellschaftsvertrag 
\\ \ \ der KCAG\}\{BA Zwischenarchiv 
\\ \ \ Dahlwitz\bs hy Hoppegarten 
\\ \ \ R\bs ,8729\string~4\}(94)
Die ...\b{\}}
}
{
...\footnote{ \printonlyarq{% 
Gesellschaftsvertrag der KCAG} 
{BA R\,8729~4}(94). 
\addtoarq{Gesellschaftsvertrag 
der KCAG}{BA Zwischenarchiv 
Dahlwitz\hy Hoppegarten 
R\,8729~4}(94)
Die \texttt{(94)} des addto\hy Befehls verschluckt der Fu"snotenausdruck;
dort darf also eine Blattnummer stehen. \textsc{ebd.} 'sucht' die
printonly\hy Komponente.}
}

\noindent
... zur Nachvollziehbarkeit als \hspace{.1em} 
{\small\verb|%%| \textit{Kommentar}\verb| %%|} 
\hspace{.1em} ins \verb|.aux|\hy File "ubertragen:

\vspace{-.5ex}
{\footnotesize\begin{verbatim}
  %\archqentry{BA Zwischenarchiv Dahlwitz\hy Hoppegarten 
    R\,8729~4}{}{{}{}{}{}}{{36}{101}{@}}[13][43](line 2726)
       %% (mpf)  Gesellschaftsvertrag der KCAG %%
\end{verbatim}}

\vspace{-.5ex}\noindent 
Die Eintr"age {\small\verb|(f)|} bzw.\ {\small\verb|(mpf)|} 
dokumentieren, ob der Zugang aus einer normalen oder 
\verb|minipage|\hy Fu"snote herstammt (oder leer: nicht aus 
einer Fu"snote). 

\vspace{1ex}\noindent
Im \verb|.arq|\hy File erscheinen die gesammelten BA\hy Eintr"age 
nach ihrer "Uberschrift:

\vspace{-.5ex}
{\scriptsize\begin{verbatim}
   \archqentry{BA}{Bundesarchiv}{{\pbastrut \ \balistcorr 
     $-$ }{\bahasdot }{1}{}}{{}{}{-}}[13][43](line 2666)
       %%  <- List-internal heading (class 1).

   \archqentry{BA Zwischenarchiv Dahlwitz\hy Hoppegarten 
     R\,8729~4}{}{{}{}{}{}}{{36}{}{}}[13][43](line 2782)  %%  %%
    \first@baidx{36, 36$^{101}$, 47, 52$^{110}$}
\end{verbatim}}

\vspace{-.5ex}\noindent
Dabei ist die Notation ...~\verb|R\,8729~4}| zudem ein Beispiel, wie eine 
einzelne Band\hy\ oder Mappennummer $-$\,4\,$-$ alternativ in die Hauptsignatur und damit 
direkt ins Verzeichnis ungedruckter Quellen aufgenommen werden kann.

\vspace{1ex}\noindent
Falls Sie nur darauf hinweisen wollen, dass es einen Bestand oder 
eine Akte gibt (also kein bestimmtes Schriftst"uck daraus zitieren),
kann das erste Argument von arq\hy Befehlen alternativ auch einfach 
leer bleiben:

\Doppelbox
{
In \bs arq\{\}\{BA Zwischenarchiv 
\\ \ \ \ \ \ \ Dahlwitz\bs hy Hoppegarten 
\\ \ \ \ \ \ \ R\bs ,8729\string~4\} findet sich ...
}
{
In \arq{}{BA Zwischenarchiv 
Dahlwitz\hy Hoppegarten 
R\,8729~4} findet sich ...
}

\vfill\noindent
{\sffamily Zur Schrifteinstellung mit }\verb|\arqemph|{\sffamily\ 
und }\verb|\arqlistemph|{\sffamily\ unten S.\,\pageref{arqemph1} und \pageref{arqemph2}. \\
Und S.\,\pageref{Gedankenstriche} wird erkl"art, wie die 
fill\hy\ und section\hy Separatoren einzustellen sind.}


\newpage
\section{Orts\fhy, Sach\hy\ und Personenregister}\label{Sect8}

\BibArts\ stellt drei Register zur Verf"ugung. Dies hat \textbf{nichts} 
mit \textsc{MakeIndex}\pdfko{1}\ 
zu tun. \verb|bibsort| nutzt die gezeigte F"ahigkeit,
auch die Fu"snotennummern zu verarbeiten. Bef"ullt werden die Register mit den Argumenten von
\verb|\addtogrr|\pdfko{1.25}\ 
(Ortsregister), \verb|\addtosrr| (Sachregister) und \verb|\addtoprr|
(Personenregister).\pdfko{.75}\ 
Ein vielfach verwendetes Stichwort \textit{kann} zudem mittels fill\hy 
Befehl einen ausf"uhr"-lich{\small(}\hspace{-.05em}er\hspace{-.05em}{\small)}en 
Zusatz erhalten, der nur einmal getippt zu werden braucht. 
fill\hy Befehle haben ein benutztes addto\hy Stichwort als 
erstes und eine Erg"anzung dazu als zweites Argument.
Alle diese Befehle sind im Text unsichtbar:

\Doppelbox
{
\bs fillgrr\{Rom\}\{Stadt in Italien\}
\\[0.4ex] Hier geht es ums Ortsregister (Geografie).\bs footnote\b{\{}Rom 
\\[0.1ex] \ \bs \underbar{addtogrr\{Rom\}} ist ein Ort.\b{\}} 
\\[0.6ex] Nero lebte in Rom.\bs \underbar{addtogrr\{Rom\}} 
\\[0.6ex] \bs printnumgrrlist
}
{
\fillgrr{Rom}{Stadt in Italien}
Hier geht es ums Ortsregister (Geografie).\footnote{Rom 
\addtogrr{Rom} ist ein Ort.} 
Nero lebte in Rom.\addtogrr{Rom} 
 %{\batwocolitemdefs
\printnumgrrlist
 %}
}

\vspace{-.25ex}\noindent
\verb|\printnumgrr| erg"abe einen \verb|\twocolumn|\hy 
Ausdruck mit der "Uberschrift \textit{\ggrrtitlename}; 
\verb|\printnumprr| und \verb|\printnumsrr| drucken die anderen 
Register.\footnote{Falls Eintr"age in den zahlen\hy losen Ausgaben  
\texttt{\bs printgrr}, \texttt{\bs printsrr} und \texttt{\bs printprr}\pdfko{1.5}\ 
mit einem Punkt enden sollen: 
\texttt{\bs renewcommand\{\bs fromnopagexrrsep\}\{\bs bapoint\}}}
Listenk"opfe wie "`Rom"' werden in \verb|\xrrlistemph| ausgedruckt, das \label{xrr}
sich mit \verb|\itshape| etc.\ belegen l"asst; evtl.\ vorhandene fills 
behalten die Umfeldschrift.

\vspace{1ex}\noindent
Neben \verb|\fillgrr| existieren \verb|\fillsrr| und \verb|\fillprr|. 
Falls Sie ein Stichwort zweimal und dann mit unterschiedlichen 
Zus"atzen bef"ullen, warnt \verb|bibsort|:

\vspace{-.75ex}
{\scriptsize\begin{verbatim}
    %%>   Warning: Different fills for head "Rom":
    %%     *Accept file 1 line 2669 "Stadt in Italien";
    %%     *Reject file 1 line 2678 "Stadt in Mittel-Italien".
\end{verbatim}}

\vspace{-1ex}\noindent
Den Registern lassen sich Querverweise der Art "`Roma $\rightarrow$ Rom"' 
hinzuf"ugen:

\vspace{-1ex}{\small
\begin{verbatim}
  {\renewcommand{\xrrlistopen}{\bastrut\ \balistcorr$\rightarrow$ }%
   \renewcommand{\xrrlistclose}{}%
   \fillgrr{Roma}{Rom}}%           %\addtogrr{Roma} nicht verwenden
\end{verbatim}}\label{xrrlistclose}%

\vspace{-.25ex}\noindent
Serien von fill\hy Befehlen sollten in \verb|{unused}|\hy 
Umgebungen \baref[vgl.]{unused}; au"serhalb w"urden Leerzeichen 
erzeugt, wenn Sie mehrere \textit{fills} zeilenweise antippen.

\vspace{1.5ex}\noindent
\verb|bibsort| kann keine Unterstichworte erzeugen, sondern erzeugt nur 
Hauptstichworte. Es gibt keine Sonderzeichen, wie sie
f"ur den \LaTeX\hy Befehl \verb|\index|\pdfko{1.75}\  
vorgesehen sind. 
Argumente werden so eingetippt, dass sie \LaTeX\ auch direkt\pdfko{1}\ 
drucken w"urde; nur zerbrechliche Befehle sollten
Sie mit \verb|\protect| sch"utzen.\pdfko{1}

In den Listen bestimmen allein die Stichworte die Sortierreihenfolge;
die F"ullungen haben kein Gewicht. Falls Sie die Zusatzf"ullungen nicht in
runden Klammern gedruckt haben wollen, k"onnen Sie \verb|\xrrlistopen| 
etwa in \verb|{, }| und \verb|\xrrlistclose| in \verb|{}| 
"andern.\footnote{Im Beispiel sind wegen des 'niederen' Zeichens 
am Kopf von \texttt{\bs xrrlistopen} weder \texttt{\bs bastrut} noch die 
\textit{italics}\hy Korrektur \texttt{\bs balistcorr} n"otig; vgl. 
unten das Kap.\,\ref{schraegkap} ab S.\,\pageref{schraegkap}.} Dies 
bietet sich insbesondere an, wenn Sie in Ihrem Text beispielsweise nur
eine Person \verb|Churchill| haben, aber\pdfko{.75}\ 
mehrere \verb|Maier| mit unterschiedlichen
Vornamen. Dann k"onnen Sie in Ihrem Text jeweils kurz \verb|\addtoprr{Churchill}| 
sowie \verb|\addtoprr{Maier, Hans}| und \verb|\addtoprr{Maier, Peter}| setzen 
$-$ und an einer Stelle des Textes alle\pdfko{.75}\  
\textit{fehlenden} Vornamen erkl"aren (die \texttt{\%}
vermeiden die Erzeugung von Leerraum):

{\small\begin{verbatim}
      %% Verschiedene Stellen mit Namen im Text:
       ... Winston Churchill \addtoprr{Churchill} ...
             ... Hans Maier \addtoprr{Maier, Hans} ...
                 ... Peter Maier \addtoprr{Maier, Peter} ...
  ... Theobald von Bethmann-Hollweg \addtoprr{Bethmann-Hollweg} ...
 
       %% Eine Stelle zum Sammeln der optionalen Zusatzfuellungen:
  {\renewcommand{\xrrlistopen}{, }%
   \renewcommand{\xrrlistclose}{}%
   \fillprr{Churchill}{Winston (1874-1965)}%
   \fillprr{Bethmann-Hollweg}{Theobald von (1856-1921)}%
  }%
       %% .... und hier gilt wieder die Default-Klammerung:
   \fillsrr{Maier, Peter}{1887-\protect\framebox{????}}%

  \printnumprr
\end{verbatim}}

\vspace{.25ex}\noindent
Falls \verb|\printnumprr| \textit{ganz am Ende Ihres Textes} steht, sollte immer 
\textit{vorher}\pdfko{1}\
gesammelt werden $-$ dahinter werden die fill\hy Befehle nicht mehr ausgef"uhrt.

\vspace{1.5ex}\noindent
Die lokalen Umdefinitionen von \verb|\xrrlistopen| und
\verb|\xrrlistclose| reisen mit \verb|Churchill| und \verb|Bethmann-Hollweg|
ins \hspace{-.2em}\verb|.prr|\hy File. \verb|\printnumprr| druckt die Liste aus. 
Die Eintr"age auf der von \texttt{bibsort} erzeugten Liste ergeben etwa:

\vspace{-0.25ex}
\begin{description}\itemsep 0pt \parskip 0pt \lineskip 0pt \small
\item Bethmann-Hollweg, Theobald von \\ (1856-1921)\hspace{1.2em}\textsf{35}
\item Churchill, Winston (1874-1965)\hspace{1.2em}\textsf{35}
\item Maier, Hans\hspace{1.2em}\textsf{35}
\item Maier, Peter (1887-\framebox{????})\hspace{1.2em}\textsf{35}
\end{description}


\newpage
\section{\texttt{\bs protect} und zerbrechliche Befehle}\label{Sect9}\label{zerbrechen}

Ein \LaTeX\hy Befehl $-$\,etwa mit 
\verb|\newcommand{|\textit{Befehlsname}\verb|}{|\textit{Deklaration}\verb|}| 
definiert\,$-$ arbeitet bei der Ausf"uhrung seine Deklaration ab. Die
besteht oft aus mehreren schon vorhandenen \LaTeX\hy Befehlen. Falls
Sie den neuen Befehl\pdfko{.75}\ 
\textit{in das Argument} eines \BibArts\hy Befehls wie etwa
\verb|\vli| tippen, wird eine Kopie dieses Eintrags an Ort und Stelle
ausgedruckt und eine zweite Kopie in das \verb|.aux|\hy File geschrieben.
Ist der neue Befehl nicht gesch"utzt, wird er dabei von \LaTeX\ 
allerdings teilweise ausgef"uhrt: Enth"alt die Deklaration Befehle,
die ihrerseits gesch"utzt sind, wird \textit{die Deklaration} 
ins \verb|.aux|\hy File kopiert; sind\pdfko{1}\ deren Befehle aber ungesch"utzt, 
wiederum deren Deklarationen $-$ u.\,s.\,w.

Wie weit ein Befehl in diesem Sinne 'zerbricht', ist also unklar. Ist
ein neuer Befehl ungesch"utzt, droht zumindest, dass \verb|bibsort| Ihre
Eintr"age nicht richtig sortiert. Schlimmstenfalls wird beim Schreiben 
ins \verb|.aux|\hy File oder beim Drucken der daraus erzeugten Liste die 
\TeX\hy Kapazit"at "uberschritten und die \LaTeX\hy "Ubersetzung Ihres 
Textes abgebrochen.

Seit \LaTeX2e\ ist letzteres kaum noch ein Problem, da fast alle wichtigen
Befehle gesch"utzt definiert sind. Allerdings bleibt das Risiko, dass
Titel mit Ihren eigenen Neudefinitionen falsch einsortiert werden. Wenn Sie etwa
\verb|\newcommand{\meinspace}{{\hskip 3cm}}| definieren und
\verb|\meinspace| in das Argument eines \verb|\vli|\hy Befehls tippen, 
wird dies bei der \LaTeX\hy "Ubersetzung den Eintrag 
\verb|{\hskip 3cm}| im \verb|.aux|\hy File ergeben und Ihr
Literaturtitel von \verb|bibsort| im \verb|.vli|\hy File 
entsprechend der Zeichenfolge \verb|3cm| einsortiert.

\underbar{Gegenma"snahme:} Durch Tippen von \verb|\protect\meinspace|
in solche Argumente ist der Befehl gesch"utzt; es wird \verb|\meinspace| 
ins \verb|.aux|\hy File kopiert.

Dabei muss \verb|\protect| also nicht von \verb|\onlyout| maskiert werden!
Vielmehr arbeitet \verb|\protect| in der addto- und der printonly\hy Komponente
des \verb|\vli|\hy Befehls unterschiedliche Deklarationen ab; beim Ausdrucken 
tut es nichts.

Falls Sie eine Eigendefinition sehr oft benutzen, k"onnen Sie $-$\,wie
oben S.\,\pageref{ProtectBeispiel} f"ur \verb|\ko| vorgemacht\,$-$ den 
Schutz in eine Doppel\hy Definition einf"ugen.

Da die \BibArts\hy Befehle meist nur Text aufnehmen sollen, stellt sich das
Problem selten. In \verb|bibarts.sty| habe ich versucht, alle Befehle zu 
sichern, die Buchstaben ausdrucken, aber zumindest nicht auf
allen \LaTeX\hy Versionen optimal gesch"utzt sind. (Dieser Schutz erstreckt
sich aber nicht auf die Argumente von \LaTeX\hy Befehlen wie 
\verb|\section| oder \verb|\index|!) Nirgendwo gesichert ist "ubrigens 
\verb|\underline{X}|, von dem (ohne \verb|\protect| davor)
im \verb|.aux|\hy File \verb|\relax $\@@underline {\hbox {X}}\mathsurround \z@ $\relax| 
o.\,"a.\ ankommt. Das wird von \verb|bibsort| zwar zwischen \verb|W|
und \verb|Xa| einsortiert; trotzdem sollten Sie ein neues \verb|.aux|\hy File
bzw.\ die von \verb|bibsort| daraus erzeugten Dateien immer durchsehen,
nachdem Sie einen Befehl in ein \BibArts\hy Argument setzten,
"uber dessen Zerbrechlichkeit/""Unzerbrechlichkeit Sie nichts wissen.


%% %?%!

\newpage
\section{Punkte, \,\texttt{\bs bahasdot} \,und \,\texttt{\bs banotdot}}\label{Sect10}\label{Punkte}

Die \BibArts\hy Befehle, die statt ihrer Argumente auch \textsc{ebd.} 
ausdrucken k"onnen,\pdfko{.5}\ 
d"urfen den Ausdruck eines unmittelbar nach ihnen 
getippten Punktes eigenst"andig unterbinden. Sonst w"urden 
am Satzende oft zwei Punkte gedruckt (\kern-.5pt\textsc{ebd.}.). 
Diese Befehle sind \verb|\kli|, \verb|\kqu|, \verb|\per| 
und \verb|\abkper|, sowie \verb|\arq|. 

\vspace{1ex}\noindent
In englischen Texten (unter \verb|\nonfrenchspacing|) sorgen
Punkte nach \BibArts\hy Befehlen zudem f"ur das richtige
\textit{spacing}: \hspace{.2em}\verb*|}. | am Ende eines der gerade aufgez"ahlten
Befehle bezeichnet ein Satzende und verl"angert das 
Leerzeichen~(\kern-.02em\verb*| |\kern.02em).\pdfko{.75}\
Auch \verb*|\abk{X.X.X.}. | bezeichnet ein Satzende und druckt 
\textit{immer} \hspace{.1em}\abk{X.X.X.}.

Auch im \verb|\frenchspacing| (dt.\ oder frz.\ Texte mit stets gleicher Leerzeichenl"ange) 
sollten Sie hinter \verb|\kli| und sogar unter \verb|\notprinthints| tippen:

\vspace{1ex}{\notprinthints\noindent\small
\verb| [deutsch/franz.:] \kli{Maier}{D.\,D.\,R.}. N =>| \printonlykli{Maier}{D.\,D.\,R.}. N \\
\verb| \nonfrenchspacing \kli{Maier}{D.\,D.\,R.}. N =>| {\nonfrenchspacing\printonlykli{Maier}{D.\,D.\,R.}. N}}

\vspace{1.25ex}\noindent
... denn nur dann k"onnen Sie sp"ater wieder auf \verb|\printhints| zur"uckschalten:\pdfko{1}\

\vspace{1ex}{\printhints\noindent\small
\verb| [deutsch/franz.:] \kli{Maier}{D.\,D.\,R.}. N =>| \printonlykli{Maier}{D.\,D.\,R.}. N \\
\verb| \nonfrenchspacing \kli{Maier}{D.\,D.\,R.}. N =>| {\nonfrenchspacing\printonlykli{Maier}{D.\,D.\,R.}. N}}

\vspace{1.75ex}\noindent
\BibArts\ durchsucht viele Argumente nach Punkten und verhindert ..\ 
eigenst"andig. Falls es doch ..\ druckt, 'sieht' es den Punkt 
\textit{am Ende des Arguments}\pdfko{.5}\ 
nicht. Mit  \hspace{.2em}\verb|{|...\verb|\bahasdot}.| \hspace{-.1em}k"onnen
Sie befehlen, den Punkt zu 'verschlucken'.

\vspace{1ex}\noindent
Nicht automatisch ist die Punktl"oschung nach optionalen Zusatzargumenten 
f"ur Band- oder Seitenangaben. Deshalb sollten Sie etwa 
\verb|f.\bahasdot].| \hspace{-.1em}ans\pdfko{1.75}\  
Ende setzen, oder einfach \verb|\f].| oder \verb|\sq].|  
(f"ur \textit{folgende} oder \textit{sequentes}), die\pdfko{1.75}\ 
\textit{beide} im Englischen und Deutschen f.\ drucken:

\vspace{1.25ex}\noindent
{\small
\verb+  \per{ZfG.}[2 f.].           =>+ \per{ZfG.}[2 f.]. \hspace{1cm} \verb|%% falsch!| \\
\verb+  \per{ZfG.}[2 f.\bahasdot].  =>+ \per{ZfG.}[2 f.\bahasdot]. \\
\verb+  \per{ZfG.}[2\f].            =>+ \per{ZfG.}[2\f].
}

\vspace{1.5ex}\noindent
Ein Sonderfall: \verb|\bahasdot| darf \textit{nicht} 
nach \verb|!| oder \verb|?| ans Argumenten\hy Ende\pdfko{1}\ 
gesetzt werden, weil dann ggf.\ notwendige \textit{italics}\hy Korrekturen unterbleiben:

\vspace{1.25ex}\noindent
{\small
{\renewcommand{\kxxemph}{\itshape}\notprinthints \showbacorr 
 \verb|   \renewcommand{\kxxemph}{\itshape} \notprinthints \showbacorr| \\[.25ex]
 \verb+ (\kli{Kingsley}{Westward Ho!\bahasdot})  =>+ (\printonlykli{Kingsley}{Westward Ho!\bahasdot}) \\
 \verb+ (\kqu{Sienkiewicz}{Quo vadis?\bahasdot}) =>+ (\printonlykqu{Sienkiewicz}{Quo vadis?\bahasdot})
}}

\vspace{1.5ex}\noindent
Vielmehr ist es n"otig, nur das Setzen des nachfolgenden Punktes zu unterbinden,
die \textit{italics}\hy Korrektur aber zu belassen. \verb|\banotdot| ist zu verwenden:\pdfko{1}

\vspace{1.5ex}\noindent
{\small
{\renewcommand{\kxxemph}{\itshape}\showbacorr\notprinthints 
\verb+ (\kli{Kingsley}{Westward Ho!\banotdot})  =>+ (\printonlykli{Kingsley}{Westward Ho!\banotdot}) \\
\verb+ (\kqu{Sienkiewicz}{Quo vadis?\banotdot}) =>+ (\printonlykqu{Sienkiewicz}{Quo vadis?\banotdot})
}}

\vspace{1.75ex}\noindent
Damit \verb|bibsort| stets zeichengleiche Eintr"age bekommt, 
muss ein einmal begonnenes Setzen von \verb|\banotdot| beim 
jeweiligen Titel immer erfolgen:

\vspace{1ex}\noindent
{\footnotesize
{\renewcommand{\kxxemph}{\itshape}\notprinthints \showbacorr 
 \verb|  \renewcommand{\kxxemph}{\itshape} \notprinthints \showbacorr| \\[.25ex]
 \verb+ (\kli{Kingsley}{Westward Ho!\banotdot}.)  =>+ (\printonlykli{Kingsley}{Westward Ho!\banotdot}.) \\
 \verb+ \kli{Kingsley}{Westward Ho!\banotdot}[3]. =>+ \printonlykli{Kingsley}{Westward Ho!\banotdot}[3]. \\
 \verb+ \kqu{Sienkiewicz}{Quo vadis?\banotdot} in =>+ \printonlykqu{Sienkiewicz}{Quo vadis?\banotdot} in \\
 \verb+ \kqu{Sienkiewicz}{Quo vadis?\banotdot}|2| =>+ \printonlykqu{Sienkiewicz}{Quo vadis?\banotdot}|2|%
}}\label{kxxB}

\vspace{2.5ex}\noindent
Nicht automatisch bew"altigt wird \,\verb|!\banotdot}| \,vor \textit{mehreren} Punkten.
Zur\pdfko{1}\ 
L"osung dieses sehr seltenen Problems kann \verb|\strut| nach \verb|}| gesetzt
werden: \verb|\kli{N.}{XX!\banotdot}...| kann zur falschen \textit{italics}\hy Korrektur
{\renewcommand{\kxxemph}{\itshape}\notprinthints \showbacorr 
 \printonlykli{N.}{XX!\banotdot}...\pdfko{.5}\ 
f"uhren, w"ahrend 
\,\verb|\kli{N.}{XX!\banotdot}\strut...| \,zu\, 
\printonlykli{N.}{XX!\banotdot}\strut...} \,f"uhrt.\pdfko{.5}\ 
 
\vspace{2ex}\noindent
\fbox{\parbox{.98\textwidth}{\hfil\sffamily 
Einfacher ist es sicher, wenn Sie sich Kurztitel ohne Satzzeichen aussuchen.}}


\vspace{4ex}\noindent
Nach dem letzten Argument von \verb|\vli| oder \verb|\vqu| l"oscht 
\BibArts\ einen Punkt\pdfko{1}\  
\textit{im Text} \underline{nie} automatisch, 
weil es \textit{dort} das letzte Argument nicht durchsucht. 
Falls das letzte Argument eines v\fhy Befehls mit einem Punkt
enden sollte, ist die Verwendung von \verb|\bahasdot| in jeder
Sprache sinnvoll, denn nur dann ist ein sp"aterer Wechsel zwischen
\verb|\announcektit| und \verb|\notannouncektit|\pdfko{1.5}\  
m"oglich (die Ank"undigung der sp"ateren Kurzzitierweise). 

\Doppelbox
{
\bs notannouncektit
\\
Vers.\string~1: \bs vli\{Niklas\}\{Luhmann\} \{\bs ktit\{Soziale Systeme\}. Grundri"s
einer allgemeinen Theorie, 1984: Frankfurt/M.\}. \ \ \%\%~FALSCH
\\[1ex]
\% Nicht in die Listen umgesetzt:
\\[.25ex]
Vers.\string~2: \bs vli\{Niklas\}\{Luhmann\} \{\bs ktit\{Soziale Systeme\}. Grundri"s
einer allgemeinen Theorie, 1984: Frankfurt/M.\bs bahasdot\}. Das ...
}
{\vspace{1.1ex}
\notannouncektit
Vers.~1: \vli{Niklas}{Luhmann} {\ktit{Soziale Systeme}. Grundri"s
einer allgemeinen Theorie, 1984: Frankfurt/M.}. 
\\[2.375ex]
Vers.~2: \printonlyvli{Niklas}{Luhmann} {\ktit{Soziale Systeme}. Grundri"s
einer allgemeinen Theorie, 1984: Frankfurt/M.\bahasdot}. 
Das ist auch unter \texttt{\bs frenchspacing} besser!
}%
\label{v-Ausnahme}%

\vspace{.25ex}\noindent
Beim Drucken der \textit{Listen} wird im letzten Argument von v\fhy Befehlen 
aber nach 'Punkt' gesucht; Frankfurt/M.. ist so in \verb|\printvli|
und \verb|\printvqu|\pdfko{.5}\  
ausgeschlossen (solange nicht etwas wie \verb|.{}}| 
am Ende steht). Vers.\,2 ist in die Listen nicht umgesetzt,
um dort zwei Luhmann\hy Eintr"age zu vermeiden.

\vfill\noindent
Zusammengefasst gibt es eine Ausnahme bei v\fhy Befehlen,
nachdem der Schalter \verb|\notannouncektit| gesetzt wurde.
Falls Sie in deutschen Texten darauf (\texttt{ngerman.sty} setzt 
\verb|\frenchspacing|) und auf \verb|\notprinthints|
verzichten, brauchen Sie \verb|\banotdot| und \verb|\bahasdot| 
nicht unbedingt zu kennen. 


\newpage\noindent
F"ur die \textbf{Definition der Textelemente}, die f"ur den Ausdruck zwischen den
Argumenten von \BibArts\hy Befehlen vorgefertigt sind ('Separatoren'), 
dient der Befehl \verb|\bapoint| zum Drucken eines Punktes am Separatoren"-\textit{kopf}. 
\hspace{.1em}\verb|\bapoint| reagiert auf die Suche nach einem Punkt \textit{am Endes des 
Arguments davor}\pdfko{.25}\  
(bzw.\ auf Ihr \verb|\bahasdot| oder \verb|\banotdot|) und druckt dann 
\textit{keinen} Punkt. 

Falls Sie im Text in \verb|\arq| zwischen Schriftst"uck und Signatur 
einen Punkt statt ein Komma haben wollen, m"ussen Sie \verb|\arqsep| 
umdefinieren. Sie sollten nicht \verb*|{. }|\kern.1em\ zuweisen:\kern.2em\
\verb|\renewcommand{\arqsep}{\bapoint\newsentence}| reagiert automatisch
und druckt keinen Punkt, wenn die sp"atere Eingabe des Schriftst"ucks 
bereits selbst mit einem Punkt endet.\footnote{
\texttt{\,\bs renewcommand\{\bs arqsep\}\{\bs bapoint\bs newsentence\} 
\ \% fuer beide spacings!}
\\[.75ex]
%\nonfrenchspacing  %%Testen Sie!%%
\renewcommand{\arqsep}{\bapoint\newsentence}
\strut\texttt{ \ \ \ \bs arq\{Gesellschaftsvertrag der KCAG\}\{BA} ... \texttt{ =>} \\[.5ex]
\arq{Gesellschaftsvertrag der KCAG} {BA Zwischenarchiv Dahlwitz\hy Hoppegarten R\,8729~4}. 
\\[1ex]
\strut\texttt{ \ \ \ \bs arq\{Test!\bs banotdot\}\{BA} ... \texttt{ =>} \\[.5ex]
\clearbamem\arq{Test!\banotdot} {BA Zwischenarchiv Dahlwitz\hy Hoppegarten R\,8729~4}. 
\\[1ex]
\strut\texttt{ \ \ \ \bs arq\{Abk.\}\{BA} ... \texttt{ =>} \\[.5ex] 
\clearbamem\arq{Abk.} {BA Zwischenarchiv Dahlwitz\hy Hoppegarten R\,8729~4}.} 

Beim \textit{Ausdruck der Listen} wird \verb|\bapoint| am Ende jedes Listenpunkts
ausgef"uhrt von \verb|\printvli| und \verb|\printvqu| 
(durch \verb|\fromnopagevxxsep|) sowie von \verb|\printarq| 
(durch \verb|\fromnopagearqsep|) und von
\verb|\printper| (durch \verb|\fromnopagepersep|). Falls Sie die
einzelnen Listenpunkte in \verb|\printvkc| und \verb|\printabk| ebenfalls 
hinterpunktet haben wollen, m"ussen Sie einfach 
\verb|\renewcommand{\fromnopagevkcsep}{\bapoint}| befehlen 
und im gleichen Stil \verb|\fromnopageabksep| umdefinieren.


\vspace{1.5ex}\noindent
Obwohl \BibArts\ \textit{im Text} das letzte Argument des v\fhy Befehls nicht 
nach Punkten durchsucht, druckt ein dort ans Ende gesetztes 
\verb|\ersch{|\textit{Ort}\verb|}{}| mit leerer\pdfko{1}\ 
Jahresangabe \ersch{\textit{Ort}}{} mit \textit{einem} Punkt aus. Sie k"onnen am 
Satzende also\pdfko{.4}\  
intuitiv vorgehen und den \underline{Punkt\kern-1.5pt} einfach 
hinter die Literaturangabe setzen:

\vspace{.5ex}\noindent
{\small
 \verb|   \vli{}{}{Titel, \ersch{Bonn}{}}|\underline{\texttt{.}}\verb|  => | 
 \printonlyvli{}{}{Titel, \ersch{Bonn}{}}.}

\vspace{.5ex}\noindent
Nicht gedruckt wird \oJ. deshalb, weil das leere \verb|\ersch|\hy Argument 
\verb|\oJ| ausf"uhrt, das seinerseits ganz am 
Ende \verb|\bahasdot| setzt. Da \verb|\oD|, \verb|\oO| und \verb|\oJ|\pdfko{1.25}\
zun"achst \verb|\protect|\hy gesch"utzt \verb|\poD|, \verb|\poO| und \verb|\poJ|
ausf"uhren, sollte \textit{an diesen} eine \label{poJ}
Umdefinition von \oD, \oO\ und \oJ\ ansetzen
(ggf.\ mit \verb|\bahasdot|\pdfko{.75}\ 
am Ende). \verb|\ersch| verwendet \verb|\oO| 
und \verb|\oJ| nur in deutschen Texten; deren\pdfko{1}\  
Umdefinition "andert \verb|\ersch| nur unter 
\verb|\bacaptionsgerman| (vgl.\ S.\,\pageref{SprachSep}, \pageref{gerschnoyearname}). 

Die bereits erw"ahnten Befehle \verb|\f| und \verb|\sq| setzen
\verb|\bahasdot| ebenfalls.\pdfko{.75}\ 
\textit{Beide} f"uhren von
der Spracheinstellung abh"angig entweder \verb|\gfolpagename|\pdfko{1.25}\ 
oder \verb|\efolpagename| oder \verb|\ffolpagename| aus und 
drucken f.\ im Deutschen\pdfko{1}\ 
und Englischen, aber sq.\ im Franz"osischen.
Es gibt auch \verb|\ff| (und \verb|\sqq|).

%%%>>>>>


\newpage
\section{\textit{Italics}\hy Korrekturen und Separatoren}%
\label{schraegkap}\label{Sect11}

\verb|\showbacorr| macht die Stellen testweise sichtbar, an 
denen \BibArts\ \textit{italics}\hy Korrekturen durchf"uhrt. 
Da v-, k-, per- und arq\hy Befehle am Anfang stets
in aufrechte Schrift umschalten, ist \textit{in schr"aggestelltem
Umfeld} immer eine \textbf{Kopf"|korrektur} n"otig. Anders als andere
\BibArts\hy Korrekturen machen Kopf"|korrekturen nichts 
\textit{nach Leerzeichen}; im Beispiel deshalb runde
Klammern:

\vspace{1.25ex}\noindent\hspace{.5em}
{\small\itshape\showbacorr
\begin{tabular}{ll}
& \verb| \itshape \showbacorr| \\
(\printonlyvli{}{}{Rest}) & \verb|(\vli{}{}{Rest})| \\
(\printonlyvli{}{Nachname}{Rest}) & \verb|(\vli{}{Nachname}{Rest})| \\
(\printonlyvli{Vorname}{Nachname}{Rest}) & \verb|(\vli{Vorname}{Nachname}{Rest})| \\
(\printonlykli{}{Kurztitel}) & \verb|(\kli{}{Kurztitel})| \\
(\printonlykli{Nachname}{Kurztitel}) & \verb|(\kli{Nachname}{Kurztitel})| \\
(\printonlyper{Zeitschrift}) & \verb|(\per{Zeitschrift})| \\
(\printonlyarq{}{Bestand}) & \verb|(\arq{}{Bestand})| \\
(\printonlyarq{Dokument}{Bestand}) & \verb|(\arq{Dokument}{Bestand})| \\
(\printonlyabkdef{Initialen}{Erkl"arung}) & \verb|(\abkdef{Initialen}{Erkl|{\upshape\texttt{"a}}\verb|rung})| \\
(\printonlyabk{Initialen}) & \verb|(\abk{Initialen})| \\
& \verb| \renewcommand{\abkemph}{\upshape}| \\ 
 \renewcommand{\abkemph}{\upshape}%
(\printonlyabkdef{Initialen}{Erkl"arung}) & \verb|(\abkdef{Initialen}{Erkl|{\upshape\texttt{"a}}\verb|rung})| \\
 \renewcommand{\abkemph}{\upshape}%
(\printonlyabk{Initialen}) & \verb|(\abk{Initialen})| \\
\end{tabular}}\label{abkA}

\vspace{1.75ex}\noindent
Dieser Typ Korrektur ist \textit{nicht} in einem \textit{einstellbaren 
Separator} festgelegt. Vielmehr schaltet \verb|\notbafrontcorr| diesen 
speziellen Korrekturtyp aus.\footnote{Die Kopf"|korrektur ist 
\texttt{\bs/} \hspace{.2em}(f"ur andere \textit{italics}\hy Korrekturen setzt \BibArts\ 
\texttt{\bs kern 0.1em}).} Den Befehl k"onnen Sie lokal setzen (oder global 
im Vorspann Ihres \LaTeX\hy Textes, um bei Bedarf etwa \verb|\baupcorr| 
selbst vor \BibArts\hy Befehle zu tippen):

\vspace{-.75ex}
\Doppelbox
{
\bs itshape \bs showbacorr
          \\ \ (\bs arq\{.Dokument\}\{Bestand\})
\\[.25ex] \{\bs notbafrontcorr 
          \\ \ (\bs arq\{.Dokument\}\{Bestand\})\}
}
{\vspace{2.55ex}
\itshape \showbacorr
   (\printonlyarq{.Dokument}{Bestand})
\\[2.4ex] {\notbafrontcorr (\printonlyarq{.Dokument}{Bestand})}
}

\vspace{-.5ex}\noindent
\verb|\vauthor|, \verb|\midvauthor|, \verb|\kauthor| und 
\verb|\mitkauthor| korrigieren ebenfalls (in Voreinstellung \verb|\bafrontcorr|) 
'am Kopf', falls sie eingeklammert sind. 

\vspace{1.25ex}\noindent
\textbf{Endkorrekturen} nach
\verb|\abk|, \verb|\kli| und \verb|\kqu| lassen sich nicht abschalten. Bei Abk"urzungen 
korrigiert \BibArts\ nach \verb|\renewcommand{\abkemph}{\itshape}| 
in aufrechtem Umfeld, wenn der Text im Argument von \verb|\abk| 
\textit{nicht} mit einem '\texttt{.}' endet. \verb|\abk| 'sieht' 
\textit{auch das Zeichen nach seinem Argument} und reagiert:

\vspace{.6ex}\noindent
{\small\verb|          {\showbacorr \abk{GmbH},   \abk{GmbH}!}   =>|} 
{\renewcommand{\abkemph}{\itshape}\showbacorr \abk{GmbH}, \abk{GmbH}!}
\\
{\small\verb|          {\showbacorr \abk{e.\,V.}, \abk{e.\,V.}!} =>|} 
{\renewcommand{\abkemph}{\itshape}\showbacorr \abk{e.\,V.}, \abk{e.\,V.}!}

\vspace{1.125ex}\noindent
Dasselbe gilt f"ur \verb|\renewcommand{\kxxemph}{\itshape}|, \label{kxxA}
mit dem der Kurztitel in \verb|\kli| und \verb|\kqu| \textit{kursiv} 
gesetzt wird.\footnote{\texttt{\bs kxxemph} wirkt sich au"serdem noch auf die 
Vorank"undigung der Kurzzitate in den v\fhy Befehlen aus;
\renewcommand{\kxxemph}{\bfseries\itshape}\showbacorr
\texttt{\bs renewcommand\{\bs kxxemph\}\{\bs bfseries\bs itshape\}} \texttt{\bs showbacorr}
bewirkt:\pdfko{.5}\ \vli{Niklas} {Luhmann}{\ktit{Soziale Systeme}. Grundri"s 
einer allgemeinen Theorie, 1984: Frankfurt/M.}[123].} Im Fall von 
\verb|\notprinthints|, das den Ausdruck von {\small [L]} und {\small [Q]} 
unterbindet, wird automatisch korrigiert:

\vspace{1ex}\noindent
{\small\verb|          {\showbacorr \kli{N}{K},   \kli{N}{K}!}   =>|}
{\renewcommand{\kxxemph}{\itshape}\notprinthints\showbacorr\printonlykli{N}{K}, \printonlykli{N}{K}!}
\\
{\small\verb|          {\showbacorr \kli{N}{K.},  \kli{N}{K.}!}  =>|}
{\renewcommand{\kxxemph}{\itshape}\notprinthints\showbacorr\printonlykli{N}{K.}, \printonlykli{N}{K.}!}

\vspace{1.25ex}\noindent
Auch im schr"aggestellten Umfeld verhalten sich beide Befehle weiter richtig:

\vspace{1.1ex}\noindent
{\small\verb|  {\itshape\showbacorr \abk{GmbH},   \abk{GmbH}!}   =>|} 
{\itshape\renewcommand{\abkemph}{\itshape}\showbacorr\abk{GmbH}, \abk{GmbH}!}\\
{\small\verb|  {\itshape\showbacorr \kli{N}{K},   \kli{N}{K}!}   =>|}
{\itshape\renewcommand{\kxxemph}{\itshape}\notprinthints\showbacorr\printonlykli{N}{K}, \printonlykli{N}{K}!}

\vspace{4ex}\noindent
Nun zu \textbf{Separatoren}. Das sind vorgefertigte Textelemente, 
die \textit{zwischen} den Argumenten von \BibArts\hy Befehlen 
ausgedruckt werden. Bei der \textit{Anwendung}\pdfko{.1}\
sollten Sie keine eigenen Korrekturen ans Ende von \BibArts\hy Argumenten tippen 
und den \verb|\|...\verb|emph|\hy Befehlen f"ur die Schrift von \BibArts\hy 
Argumenten nie etwas wie \verb|\textbf| oder \verb|\textit| zuweisen, sondern 
\verb|\bfseries| oder \verb|\itshape|.\footnote{Am Ende des Arguments darf auch 
keine \texttt{\{} mehr stehen, \texttt{\}} schon.} 

Zur \textit{Ver"anderung von Separatoren} l"asst sich \verb|\renewcommand| 
verwenden.\pdfko{.75}\ 
Befehle f"ur ggf.\ n"otige \textit{italics}\hy Korrekturen m"ussen Sie 
dabei selbst platzieren. Die sind meist n"otig, falls ein Separator etwas 
anderes als '\texttt{,}' oder '\texttt{.}' enth"alt:\pdfko{.75}

\Doppelbox
{\vspace{.25ex}
\bs itshape\bs showbacorr
\\[1.3ex]
\bs renewcommand\{\bs nsep\}\underline{\{,\vphantom{p}} \}   
\\[.5ex] ... ... \bs xkqu\{Ehlert\} 
\\ \ \string*\b{\{}\bs midkauthor\{Epkenhans\} 
\\ \ \ \ \bs kauthor\{Gro"s\} [Hrsg.]\b{\}}
\\ \ \{Schlieffenplan\}[468].
\\[1.4ex] 
\bs renewcommand\{\bs nsep\}\{\underline{/\bs baupcorr\}}
\\[.5ex] ... ... \bs xkqu\{Ehlert\} 
\\ \ \string*\{\bs midkauthor\{Epkenhans\} 
\\ \ \ \ \bs kauthor\{Gro"s\} [Hrsg.]\}
\\ \ \{Schlieffenplan\}[469].
}
{\vspace{.5ex}
\itshape\showbacorr\noindent
Ge"anderte \textup{\texttt{\bs nsep}} sind defaultm"a"sig in
Umfeldschrift; formatierte Nachnamen
sind immer aufrecht:

\vspace{.255cm}
\renewcommand{\nsep}{, }    %% nach Komma keine Korrektur %%
... ... \xkqu{Ehlert} *{\midkauthor{Epkenhans} \kauthor{Gro"s} [Hrsg.]}
   {Schlieffenplan}[468]. 

\vspace{.255cm}
\renewcommand{\nsep}{/\baupcorr}    %% nach + schon %%
... ... \xkqu{Ehlert} *{\midkauthor{Epkenhans} \kauthor{Gro"s} [Hrsg.]}
   {Schlieffenplan}[469].
\vspace{1.05cm}\strut
}%
\label{MEhlert}%


\vspace{.5ex}\noindent
\verb|\baupcorr| korrigiert immer dann, wenn es in
in einem schr"aggestellten Umfeld steht. F"ur den \textit{Kopf einer
Separator\hy Definition} ist es ungeeignet, denn es hat keine Information
dar"uber, ob Sie in das \textit{davorstehende Argument des\pdfko{1}\  
\BibArts\hy Befehls} bei der Anwendung Text tippten, der mit einem Punkt endet. 


\newpage \BibArts\ durchsucht Argumente deshalb
und stellt ortsabh"angige \textit{italics}\hy Korrekturen bereit,
die nach einem Punkt nichts machen. \verb|\balistcorr| ist ein
Beispiel. Wie alle ortsabh"angigen Korrekturen kommt es nur am
Kopf des zugeh"origen Separators zu Einsatz, in diesem Fall von
\verb|\frompagesep|. \label{frompagesep} Der definiert in den 
num\hy Listenausdrucken, was vor den Indexzahlen steht:

\vspace{.25ex}
{\small
\begin{verbatim}
  {\renewcommand{\frompagesep}{\balistcorr ; }  % HOCH mit Korrektur
   \itshape \showbacorr \printnumvlilist }
\end{verbatim}}

\vspace{-3ex}
  {\renewcommand{\frompagesep}{\balistcorr ; }  % HOCH mit Korrektur
   \itshape \showbacorr \batwocolitemdefs\printnumvlilist }


\vspace{-.5ex}{\small
\begin{verbatim}
  {\renewcommand{\frompagesep}{, }          % NIEDRIG ohne Korrektur
   \renewcommand{\ntsep}{\upshape , }     % Komma zw. Name und Titel
   \itshape \showbacorr \printnumvlilist }
\end{verbatim}}

\vspace{-3ex}
  {\renewcommand{\frompagesep}{, }              % NIEDRIG ohne Korr.
   \renewcommand{\ntsep}{\upshape , }     % Komma zw. Name und Titel
   \itshape \showbacorr \baonecolitemdefs\printnumvlilist }


\vspace{1.5ex}\noindent
Die Indexzahlen der num-Listen druckt \BibArts\ in 
\verb|\balistnumemph| aus, f"ur das \verb|\sffamily| voreingestellt 
ist. \label{listnum} Wegen der ggf.\ zu druckenden Exponenten f"uhrt \BibArts\ 
dort Umstellungen auf schr"aggestellte Schriften nicht aus. Und das Argument 
von \verb|\frompagesep| wird ausgedruckt wie die Indexzahlen.

\vspace{2ex}\noindent
\textit{Italics}\hy Korrekturen sind in Separatoren stets n"otig, 
falls dort ein schr"aggesteller Textbereich auf einen aufrecht gedruckten
Textbereich treffen k"onnte und sich dazwischen kein Punkt oder Komma befindet.
Fallunterscheidungen sind m"oglich: Der Separator \verb|\ntsep| f"uhrt 
in v- und k\fhy Befehlen zwischen Name und Titel 
\verb|{: \ifbashortcite{\bakntsepcorr}{}}| \label{ntsepB} 
aus, korrigiert also\pdfko{1.25}\ 
nur in k\fhy Befehlen.
Mit \verb|\renewcommand{\kxxemph}{\upshape}| k"onnen Sie\pdfko{1.25}\  
eine stets aufrechte Schrift f"ur den Kurztitel in 
\verb|\kli| und \verb|\kqu| einstellen: 

\vspace{-.325ex}
\Doppelbox
{
   \bs renewcommand\{\bs kxxemph\}\{\bs upshape\}
\\ \bs showbacorr \bs itshape
\\ \bs kli\{Ferguson\}\{Falscher Krieg\}
}
{
\vspace{5ex}
\renewcommand{\kxxemph}{\upshape}
\showbacorr \itshape
\kli{Ferguson}{Falscher Krieg}
}

\vfill\noindent
Dasselbe \verb|\ntsep| korrigiert $-$\,wegen der if\fhy Abfrage\,$-$ in v\fhy Befehlen \textit{nicht}.

\newpage
\noindent
\BibArts' \textbf{if\hy Befehle} haben zwei Argumente, von denen \BibArts\ das erste
bei Ja und das zweite bei Nein umsetzt. \verb|\ifbaibidem| etwa steht nach
\textsc{ebd.} bereit:

\vspace{.875ex}{\small\noindent
\verb|  \ifbashortcite  | \verb|{|\textit{falls k-Befehl}\verb|}| \verb|{|\textit{sonst (falls v-, per oder arq-Befehl)}\verb|}| \\
\verb|  \ifbaperiodical | \verb|{|\textit{falls per-Befehl}\verb|}| \verb|{|\textit{sonst (falls v-, k- oder arq-Befehl)}\verb|}| \\
\verb|  \ifbaprinthints | \verb|{|\textit{unter Voreinstellung}\verb|}| \verb|{|\textit{falls }\verb|\notprinthints|\textit{ gilt}\verb|}| \\
\verb|  \ifbaibidem     | \verb|{|\textit{falls k-, per- oder arq-Befehl als }\textsc{ebd.}\textit{ gedruckt}\verb|}| \verb|{|\textit{sonst}\verb|}| \\
\verb|  \ifbahaspervol  | \verb|{|\textit{hat }\verb+|n|+ \verb+_n_+\textit{ und }\verb|\notprintlongpagefolio|\textit{ gilt}\verb|}| \verb|{|\textit{sonst}\verb|}| \\
\verb|  \ifbahasdot     | \verb|{|\textit{am Separatorenkopf: falls Arg davor mit} 
 \hspace{-.15em}\verb|.| \hspace{-.1em}\textit{endet}\verb|}| \verb|{|\textit{sonst}\verb|}|}


\vspace{1.25ex}\noindent
Die wichtigsten Korrekturbefehle, die \BibArts\ f"ur Separatoren bereitstellt, sind: 

\vspace{.875ex}{\small\noindent 
\verb|  \balistcorr     | in \verb|\frompagesep| (vgl.\ oben S.\,\pageref{frompagesep}) \\
\verb|  \bakntsepcorr   | in \verb|\ntsep| vor k-Titeln (vgl.\ oben S.\,\pageref{ntsepA}, \pageref{ntsepB}) \\
\verb|  \bakxxcorr      | in \verb|\pagefolioshortsep| (s.\,u); intern nach k-Titeln \\
\verb|  \baabkcorr      | in \verb|\abkdefopen|, \verb|\defabkopen| und \verb|\defabkclose|} 


\vspace{1.325ex}\noindent 
Die Definition von \verb|\pagefolioshortsep| k"onnen Sie in \verb|bibarts.sty|
suchen: \label{bakxxcorr} \verb|\bakxxcorr| wird unter \verb|\notprinthints| plus 
\verb|\notprintlongpagefolio| unter bestimmten Bedingungen nach 
\verb|\kli| oder \verb|\kqu| ausgef"uhrt (vgl.\ 
S.\,\pageref{notprintlongpagefolio}).\footnote{\texttt{\scriptsize\bs notprintlongpagefolio 
\bs renewcommand\{\bs kxxemph\}\{\bs itshape\} \bs notprinthints \bs showbacorr} 
\\ \hspace*{1cm}\texttt{\scriptsize
\bs kli\{Luhmann\}\{Soziale Systeme\}[23].  => } 
\clearbamem\notprintlongpagefolio 
\renewcommand{\kxxemph}{\itshape}\notprinthints \showbacorr 
\kli{Luhmann}{Soziale Systeme}[23].}


\vspace{1.5ex}\noindent
F"ur \textbf{Abk"urzungsdefinitionen} sind die Klammersymbole einstellbar. 
Die Separatoren \verb|\abkdefopen|, \verb|\defabkopen| und 
\verb|\defabkclose| sollten diese Klammern \textit{und} \verb|\baabkcorr| 
enthalten (in \verb|\abkdefopen| und \verb|\defabkclose| davor, in 
\verb|\defabkopen| danach). Stehen \verb|\abkdef| oder \verb|\defabk| 
(oben S.\,\pageref{defabk}) in Textbereichen mit aufrechter Schrift, 
f"uhrt \verb|\baabkcorr| dann eine \textit{italics}\hy Korrektur aus, wenn 
f"ur \verb|\abkemph| eine schr"aggestellte Schrift gilt:\footnote{Oder 
wenn in schr"aggestelltem Umfeld \texttt{\bs upshape} f"ur 
\texttt{\bs abkemph} gilt.}

\vspace{-.5ex}
\Doppelbox
{\vspace{.75ex}
\bs renewcommand\{\bs abkemph\}\{\bs itshape\}
\\[.25ex] \ \ \bs showbacorr
\\[.5ex] \b{\{}\bs renewcommand\{\bs abkdefopen\}
\\[.3ex] \ \ \ \ \ \ \{\bs baabkcorr\bs\ [\kern.1em\} 
\\[.25ex] \ \bs renewcommand\{\bs abkdefclose\}\{\kern.1em]\}
\\[.5ex] \ \bs abkdef\{OHG\}\{Offene 
\\ \ \ \ \ \ Handelsgesellschaft\}\b{\}} \ u.\bs
\\[.5ex]
 \b{\b{\{}}\bs renewcommand\{\bs defabkopen\}
\\ \ \ \ \ \ \ \{\bs bastrut\bs\ \string"\string<\bs baabkcorr\}\%
\\[.5ex] \ \bs renewcommand\{\bs defabkclose\}
\\ \ \ \ \ \ \ \{\bs baabkcorr \string"\string>\}\%
\\[.05ex] \ \bs defabk\{Offene 
\\ \ \ \ \ \ Handelsgesellschaft\}\{OHG\}.\b{\b{\}}}
}
{\vspace{1ex}
 \fbox{\parbox{.95\textwidth}{\sffamily In \texttt{\bs abkdefopen} oder
             \texttt{\bs defabkclose} steht \texttt{\bs baabkcorr} \textit{vor} Leerzeichen.}}
 \\[5.75ex]
   \small\renewcommand{\abkemph}{\itshape} \showbacorr 
 {\renewcommand{\abkdefopen} {\baabkcorr\ [}%
  \renewcommand{\abkdefclose}{]}%
   \abkdef{OHG}{Offene Handelsgesellschaft}}  u.\
 {\renewcommand{\defabkopen}{\bastrut\ "<\baabkcorr}%
  \renewcommand{\defabkclose}{\baabkcorr ">}%
   \defabk{Offene Handelsgesellschaft}{OHG}.}
 \\[2.75ex]
 \fbox{\parbox{.95\textwidth}{\sffamily \texttt{\bs bastrut\bs\ }\,stellt sicher, 
         dass Zeilenumbr"uche am \texttt{\bs\ }\,stattfinden k"onnen.}}
}


\noindent
Wird statt \verb|{OHG}| alternativ \verb|{e.\,V.}| eingesetzt, unterbleibt die 
Korrektur:

\vspace{.5ex}
{\small\renewcommand{\abkemph}{\itshape}\showbacorr 
{\renewcommand{\abkdefopen} {\baabkcorr\ [}%
 \renewcommand{\abkdefclose}{]}%
  \abkdef{e.\,V.}{eingetragener Verein}} und
{\renewcommand{\defabkopen}{\bastrut\ "<\baabkcorr}%
 \renewcommand{\defabkclose}{\baabkcorr ">}%
  \defabk{eingetragener Verein}{e.\,V.}.}
}



\vspace{1ex}\noindent
\BibArts\ macht nach \verb|\defabkopen| (hier nach "<) 
einen in \LaTeX2e von \verb|\itshape|\pdfko{1}\ 
ausgedruckten horizontalen 
Abstand r"uckg"angig. \LaTeX\,2.09 macht die Korrektur nicht;
die \BibArts\hy Gegenkorrektur kann im Dokumentenvorspann mit 
\verb|\notnegcorrdefabk| ausgeschaltet werden.  


\vspace{1ex}\noindent
Das zweite Argument von \verb|\defabk| ist die
Abk"urzung, die in einstellbarer\pdfko{.5}\ 
Schrift gedruckt wird. 
\BibArts\ 'sieht' es das \textit{folgende} Zeichen~'!' und korrigiert:


\Doppelbox
{
    \bs renewcommand\{\bs abkemph\}\{\bs em\} 
 \\[.5ex] \bs showbacorr
 \\[.5ex] \bs renewcommand\{\bs defabkopen\}
 \\ \ \ \ \ \ \ \underline{\{\bs}ifbahasdot\{\bs bastrut\bs\ \}
 \\ \ \ \ \ \ \ \ \ \ \ \ \ \ \ \ \ \ \ \{ \bs baabkcorr\underline{\}\}}
 \\[.5ex] \bs renewcommand\{\bs defabkclose\}\{\}
 \\ \ \ \ \%\%=\{\bs defabkclose\}\{\bs baabkcorr\}
\\[2.5ex]
 Ein \bs defabk\{eingetragener 
\\ \ \ Verein\} \{e.\bs,V.\}!
\\[1.5ex]
 \bs defabk \{Offene 
\\ \ \ Handelsgesellschaft\}\{OHG\}!
\\[1.5ex]
   \bs itshape 
\\[.5ex] Noch ein \bs defabk\{eingetragener 
\\ \ \ Verein\}\{e.\bs,V.\}!
\\[1.5ex]
 \bs defabk\{Offene Handelsges.\}
\\ \ \ \{OHG\}!
}
{
 \vspace{19.4ex}
 \renewcommand{\abkemph}{\em} 
 \showbacorr
 \renewcommand{\defabkopen}
  {\ifbahasdot{\bastrut\ }
           { \baabkcorr}}
 \renewcommand{\defabkclose}{}   %=%{\defabkclose}{\baabkcorr}

 Ein \defabk {eingetragener Verein} {e.\,V.}!
 \\[3.5ex]
 \defabk {Offene Handelsgesellschaft}{OHG}!
 \\[6.6ex]
   \itshape 
 Noch ein \defabk{eingetragener Verein}{e.\,V.}!
 \\[3.6ex]
 \printonlydefabk{Offene Handelsges.} {OHG}!
}\label{abkB}


\noindent
Nur im obigen Fall bei \verb|\defabkopen| wird
\verb|\ifbahasdot{|\textit{ja}\verb|}{|\textit{nein}\verb|}| 
ben"otigt: \verb|\baabkcorr| steht in der Voreinstellung 
\textit{nach} einer '\verb|(|' $-$ aber hier, wo
es am Kopf des Separators steht, ist die \textit{italics}\hy 
Korrektur nicht immer n"otig. Und im \textit{nein}\hy Fall 
steht \verb|\baabkcorr| \textit{nach} dem Leerzeichen:
Wie die meisten corr\hy Befehle w"urde \verb*|\baabkcorr\ | 
keinen Zeilenumbruch erlauben (vgl.\ S.\,\pageref{Systematik}).

\vspace{1ex}\noindent
In den Beispielen oben wurden die Klammersymbole lokal angepasst.
Dies wirkt sich \textit{nicht} auf das \textbf{Abk"urzungsverzeichnis} 
aus, denn f"ur dessen Ausdruck gelten eigene 
Separatoren: \verb|\abklistopen| und \verb|\abklistclose| 
legen fest, was dort vor und nach der \textit{Erkl"arung}
\label{Erklaerung} stehen soll. Im Abk"urzungsverzeichnis 
steht die Abk"urzung immer links und die Erkl"arung immer rechts.

Allerdings reisen die Definitionen von
\verb|\abklistopen| und \verb|\abklistclose| 
("ahnlich der Definition von \verb|\nsep|\hspace{.075em}: siehe
oben S.\,\pageref{Ausreise}) mit ins \hspace{-.15em}
\verb|.aux|-File. Falls eine bestimmte Abk"urzung also
im Abk"urzungsverzeichnis ihre eigene Klammerung haben soll,
m"ussen \verb|\abklistopen| und \verb|\abklistclose| lokal
angepasst werden. Beim Umdefinieren sind zerbrechliche 
Befehle mit \verb|\protect| zu sch"utzen. Gelten f"ur
mehrere Zug"ange \textit{einer} Abk"urzung unterschiedliche Definitionen
der Listenseparatoren, warnt \texttt{bibsort} mittels
Bildschirmmeldung. Die jeweils erste Definition setzt es f"ur
den Ausdruck der Liste ein.

\verb|bibarts.sty| 
legt f"ur \verb|\abklistopen| zun"achst \verb|{\protect\pabklo}| fest; 
das f"uhrt \verb|{\bastrut\hskip 1.2em minus 0.3em\balistcorr}| aus. 
"Aquivalent h"angen \verb|\abklistclose|
und \verb|\pabklc| zusammen (das nichts tut: \verb|{}|). 

Zur "Anderung der Listenseparatoren k"onnen Sie vor dem Befehl zum Ausdruck
des Abk"urzungsverzeichnisses die Befehle \verb|\pabklo| oder
\verb|\pabklc| "andern. Dies wirkt sich aus auf alle Eintr"age,
an deren Stellen im Text die Voreinstellungen f"ur
\verb|\abklistopen| und \verb|\abklistclose| nicht ver"andert wurden:

\vspace{-.25ex}{\small
\begin{verbatim}
  {\renewcommand{\pabklo}{\bastrut\ \balistcorr =\ }
   \renewcommand{\pabklc}{!} %% ^^ \bastrut\ erlaubt Umbruch nach .
   \renewcommand{\abklistemph}{\itshape}
   \showbacorr \small \printnumabklist}
\end{verbatim}}

\vspace{-2.5ex}
  {\renewcommand{\pabklo}{\bastrut\ \balistcorr =\ }
   \renewcommand{\pabklc}{!}
   \renewcommand{\abklistemph}{\itshape}
   \showbacorr \small \batwocolitemdefs\printnumabklist}

\noindent
Ein Umdefinieren der beim Listenausdruck 
gesetzten Klammerung muss also nicht global im Vorspann Ihres 
\LaTeX\hy Textes erfolgen. Falls Sie dennoch \verb|\abklistopen| 
oder \verb|\abklistclose| im Vorspann umdefinieren,
funktioniert die vorgef"uhrte Anpassung von \verb|\pabklo| 
und \verb|\pabklc| nicht. Dann m"ussen\pdfko{.25}\ Sie an Ihren 
Definitionen von \verb|\abklistopen| und \verb|\abklistclose| 
ansetzen.

\vspace{1ex}\noindent
So lassen sich Klammer\hy Separatoren f"ur einzelne 
Listeneintr"age "andern ...

\vspace{-.2ex}
\Doppelbox
{\vspace{.3ex}
 Der 
 \\ \b{\{}\bs renewcommand\{\bs abklistopen\}\{, \}\%
 \\ \ \bs renewcommand\{\bs abklistclose\}
 \\ \ \ \ \ \{ [Erkl"arung am Zugangsort]\}\%
 \\ \ \bs abkdef\{S\}\{Sonderfall\}\b{\}} 
 \\ in der Liste.
 \vspace{.3ex}
}
{
 \ \ Der {\renewcommand{\abklistopen}{, }%
  \renewcommand{\abklistclose}{ [Erkl"arung am Zugangsort]}%
 \abkdef{S}{Sonderfall}} in der Liste.
}

\vspace{-.2ex}\noindent
... wobei die Definition von \verb|\abklistopen| der "Ubersichtlichkeit 
halber mit einem 'niederen' Zeichen beginnt, 
vor dem keine Korrektur n"otig ist; und \verb|\abklistclose| 
braucht nie eine. Geschrieben wird ins \verb|.aux|-File etwas wie:

\vspace{0.325ex}
{\scriptsize\begin{verbatim}
    %\abkrzentry{OHG}{Offene Handelsgesellschaft}{{\pabklo }{\pabklc }{}}...(line 2922)
    ...
    %\abkrzentry{S}{Sonderfall}{{, }{ [Erkl\IeC {\"a}rung am Zugangsort]}{}}...(line 3006)
\end{verbatim}}


\vspace{2ex}\noindent
Zum \textbf{Drucken von Archivquellenangaben} in Text oder Fu"snoten: F"ur das 
Argument von \verb|\arqsep| gibt es keinen corr-Befehl zur Korrektur 
zwischen Dokument und Bestandsangabe, da \BibArts\ f"ur beide Argumente
eine aufrechte Schrift \verb|\arqemph| erzwingt. Z.\,B. \verb|\renewcommand{\arqsep}{: }| 
reicht aus. Hier steht ein solcher \verb|\arq|\hy Befehl
{\showbacorr\itshape in schr"aggestelltem Umfeld: 
\renewcommand{\arqsep}{: }\arq{Gesellschaftsvertrag der 
KCAG}{BA Zwischenarchiv Dahlwitz\hy Hoppegarten R\,8729~4}(94)}. 

Eine Abweichung ergibt sich f"ur das \textbf{Drucken des Archivverzeichnisses}: Da 
es kein (aufrechtes) \textsc{ebd.} im Archivquellenverzeichnis gibt, darf dort 
f"ur \verb|\arqlistemph| (oder \verb|\arqemph|, siehe S.\,\pageref{arqemph2}) 
\textit{kursiv} eingestellt werden:

\vspace{-.25ex}
{\footnotesize\begin{verbatim}
 {\renewcommand{\arqlistemph}{\itshape}\showbacorr \printnumarq}
\end{verbatim}}\label{arqemph1}

\vspace{-2.5ex}
 {\renewcommand{\arqlistemph}{\itshape}\showbacorr \printnumarq}


\vspace{2.5ex}\noindent
Auch der im Text verbotene Kursivdruck von Zeitschriftentiteln ist 
im \textbf{Zeitschriftenverzeichnis} erlaubt. Falls Sie 
\verb|\frompagesep| umdefinieren wollten, um den
num-Listenausdruck zu modifizieren,
sollten Sie an den Anfang \verb|\balistcorr| 
setzen, wenn ein 'hohes' Zeichen wie $\rightarrow$ am Anfang steht:

\vspace{-.5ex}
{\footnotesize\begin{verbatim}
 {\renewcommand{\perlistemph}{\itshape}%
  \renewcommand{\frompagesep}{\bastrut\hskip0pt\balistcorr$\rightarrow$}%
  \showbacorr \printnumper} % ^^^^^^^^^^^^^^^^^ Trennung
\end{verbatim}}

\vspace{-2.5ex}
 {\renewcommand{\perlistemph}{\itshape}%
  \renewcommand{\frompagesep}{\bastrut\hskip0pt\balistcorr$\rightarrow$}%
  \showbacorr \printnumper} % ^^^^^^^^^^^^^^^^^ Trennung

\vspace{1ex}\noindent
Die "offnenden und schlie"senden Separatoren f"ur die per\hy Liste
hei"sen (symmetrisch zu den abk\hy Befehlen) \label{perlistopen}
\verb|\perlistopen| und \verb|\perlistclose|. Sie f"uhren
\verb|\protect\pperlo| und \verb|\protect\pperlc| aus, um
beim Schreiben ins \verb|.aux|\hy File nicht zu zerbrechen.
Beim Ausdrucken der per\hy Liste f"uhren \verb|\pperlo| und\pdfko{.5}\ 
\verb|\pperlc| gem"a"s Voreinstellung 
\verb|{\bastrut\ \balistcorr $-$ }| und \verb|{}| aus. Daran k"onnen
Sie wiederum im Umfeld des Listenausdruckbefehls ansetzen:

\vspace{-.75ex}
{\small\begin{verbatim}
  {\renewcommand{\pperlo}{\bastrut\ \balistcorr ((}
   \renewcommand{\pperlc}{))}
   \renewcommand{\perlistemph}{\itshape} \showbacorr \printper}
\end{verbatim}}

\vspace{-3.75ex}
  {\renewcommand{\pperlo}{\bastrut\ \balistcorr ((}
   \renewcommand{\pperlc}{))}
   \renewcommand{\perlistemph}{\itshape} \showbacorr \printper}


\vspace{.5ex}\noindent
F"ur das Archivquellenverzeichnis dienen 
\verb|\arqlistopen| und \verb|\arqlistclose| als Separatoren.
\label{Gedankenstriche} Sie f"uhren \verb|{\protect\parqlo}| 
und \verb|{\protect\parqlc}| aus und expandieren
zu \verb|{\bastrut\hskip 1em minus 0.3em\balistcorr}| und \verb|{}|.
Die "Uberschriften, die im Verzeichnis ungedruckter Quellen
existieren k"onnen, enthalten die oben gezeigten Gedankenstriche,
die von den Befehlen \verb|\arqsectionopen|, \verb|\arqsubsectionopen|
und \verb|\arqsubsubsectionopen|\pdfko{1}\ 
initiiert werden. Alle f"uhren direkt \verb|{\bastrut\ \balistcorr $-$ }| aus.

\vspace{1ex}\noindent
Weiter existieren \verb|\xrrlistopen| und \verb|\xrrlistclose|.
Sie schreiben f"ur die drei \BibArts\hy Register \verb|\protect|\hy gesch"utzt 
\verb|\pxrrlo| und \verb|\pxrrlc| ins \verb|.aux|\hy File und expandieren 
zu \verb|{\bastrut\ \balistcorr(}| bzw.\ \verb|{)}| \,(vgl.\ oben S.\,\pageref{xrrlistclose}).


\vfil
\fbox{\parbox{.9\textwidth}{\footnotesize \vspace{.675ex}
Die mit \texttt{\bs usepackage[T1]\{fontenc\}} eingeladene
Schriftkodierung gibt \LaTeX2e\ die gegen"uber \LaTeX\,2.09 neue F"ahigkeit, 
Worte mit deutschen Sonderzeichen eigenst"andig trennen zu k"onnen.
In \texttt{OT1}\hy Schriftkodierung dagegen kann \LaTeX2e\ Silben, die etwa 
Umlaute enthalten, auch weiterhin oft nicht umbrechen. Der Vorteil ist 
gewaltig. Allerdings kann speziell PDF\LaTeX\ unter \texttt{T1} offenbar die L"ange 
von Zeilen nicht mehr genau bestimmen, wenn ein Wort etwa \textit{schr"ag} 
hervorgehoben ist. Jedenfalls macht in PDF\hy Dateien die rechte Seite des
Zeilenblocks dann oft Schlangenlinien. Und weil \BibArts\ Dieselbe, Derselbe 
und Ebenda in \textsc{small\,caps} setzt, werden solche Zeilen in 
PDF\ko\hy\texttt{T1}\hy Dateien bis zu 1,5 Punkte l"anger als andere
ausgedruckt. Dies trat nicht auf in \texttt{ba-short.pdf}, dessen
Quellfile \texttt{ba-short.tex} die defaultm"a"sig geladene Schriftkodierung 
\texttt{OT1} aufruft. Das Problem existiert in DVI\hy Files 
nach meiner Beobachtung
"uberhaupt nicht (egal, ob mit \texttt{OT1} oder \texttt{T1})!\vspace{.3ex}}}


\newpage\noindent
Nun zu \textbf{Kurztiteln in vli- und vqu\hy Befehlen}:
In Voreinstellung setzt \BibArts\ $-$\,anders als 
etwa \verb|\textit|\,$-$ zwischen seinen Argumenten auch an Leerzeichen 
\textit{italics}\hy Korrekturen. So wird in v\fhy Befehlen nach\hspace{.1em} 
\textit{im Folgenden}\verb*| |\pdfko{1.25}\ korrigiert (die Definition
von \verb|\gannouncektitname| endet mit \verb|\baupcorr|):

\Doppelbox
{
\bs showbacorr \bs itshape ...: 
\\[.1ex] \bs vqu\{Carl von\} \{Clausewitz\} 
\\ \ \string*\b{\{}(\bs vauthor\{Eberhard\}\{Kessel\}\% 
\\ \ \ \ \ \ \ \bs onlyvoll\{ \bs editor\})\b{\}} 
\\ \ \{\bs ktit\{Strategie\}, Hanseatische 
\\ \ \ \ \ \ \ Verlagsanstalt 1937\}
}
{
\notktitaddtok
\showbacorr \itshape
...: \printonlyvqu{Carl von} {Clausewitz} 
*{(\vauthor{Eberhard}{Kessel}% 
\onlyvoll{ \editor})} 
{\ktit{Strategie}, Hanseatische Verlagsanstalt 1937}\footnotemark 
}\label{editor}%
\refstepcounter{footnote}\footnotetext{Das dabei verwendete 
\texttt{\bs editor} hat kein Sortiergewicht
(Beispiel nicht in der Liste gedruckter Quellen). 
F"ur mehrere Herausgeber existiert \texttt{\bs editors}; im
Deutschen sind sowohl \texttt{\bs geditorname} als auch \texttt{\bs geditorpname}
mit {\editor} belegt. Wie viele Textelemente, die in 
Umfeldschrift gedruckt werden, kommen die beiden ohne \textit{italics}\hy 
Korrekturen aus.}

\vspace{2ex}\noindent
\textsf{Sprachabh"angige Separatoren werden 
unten in Kapitel~\ref{SprachSep} ab S.\,\pageref{SprachSep} 
behandelt.}

\vspace{3ex}\noindent
Es gibt keinen Befehl, um die Schrift \textit{des gesamten 'Rest' 
von v\fhy Befehlen}\pdfko{.25}\ einzustellen. Auch der Volltitel darin l"asst 
sich nur eingeschr"ankt hervorheben: Falls Sie \verb|\ktit{|...\verb|}| 
innerhalb des letzten Arguments n"amlich einfach\pdfko{1}\
einklammern, kann \BibArts\ den Kurztitel nicht mehr 'sehen'! 
Da '"au"sere' v\fhy Befehle kein \verb|\ktit| haben 
\textit{m"ussen}, f"allt \BibArts\ diese 'Ausblendung' nicht auf 
und\pdfko{1}\ 
es macht keine Fehlermeldung. Mit den beiden \LaTeX2e\hy 
Befehlen \verb|\itshape|\pdfko{1.25}\ 
und \verb|\upshape| gibt es eine L"osung ohne Klammern. Sie 
nutzt aus, dass zwischen Titel und Erscheinungsort/""\fhy jahr 
ein Komma steht: 

\Doppelbox
{\vspace{-.8ex}
Vers.\string~1: \bs vli\{Niklas\}\{Luhmann\} \b{\{}\bs textit\b{\b{\{}}\bs ktit\{Soziale Systeme\}. 
\\ \ \ Grundri"s einer allgemeinen 
\\ \ \ Theorie\b{\b{\}}}, 1984: Frankfurt/M.\b{\}}.
\\[2.8ex]
Vers.\string~2: \bs vli\{Niklas\}\{Luhmann\} \H{\{}\bs\underbar{itshape}\bs ktit\{Soziale Systeme\}. 
\\ \ Grundri"s einer allgemeinen 
\\ \ Theorie\bs\underbar{upshape}, 1984: 
\\ \ Frankfurt/M.\H{\}}. \ \%\% RICHTIG \%\%
}
{\vspace{1ex}
Vers.~1: \printonlyvli{Niklas}{Luhmann} {\textit{\ktit{Soziale Systeme}. 
Grundri"s einer allgemeinen 
Theorie}, 1984: Frankfurt/M.}. 
\\[4.1ex]
Vers.~2: \printonlyvli{Niklas}{Luhmann} {\itshape\ktit{Soziale Systeme}. 
Grundri"s einer allgemeinen 
Theorie\upshape, 1984: Frankfurt/M.}.
}%


\vfill\noindent
\textsf{Zur Vermeidung von\hspace{.1em} ..\hspace{.1em} 
nach v\fhy Befehlen im Allgemeinen siehe oben S.\,\pageref{v-Ausnahme}.}
\vfill

%%%<<<<<

%?%!

\newpage\noindent
\textbf{Wiederholung: Anwenderfreie corr-Befehle, }\texttt{\bs bapoint}\textbf{ und }\texttt{\bs bastrut}

\begin{itemize}
\item Ganz am Anfang der Definition eines Separators k"onnen \verb|\bapoint|
      oder\pdfko{1.5}\ 
      \verb|\bastrut| stehen. Sie schlie"sen sich gegenseitig aus; es
      d"urfen nicht beide hintereinander stehen. Falls Sie ganz an den
      Anfang eines Separators ein 'echtes' Zeichen setzen (kein
      Leerzeichen), sind beide "uberfl"ussig.
\item \verb|\bapoint| tut nichts, wenn das \textit{im \BibArts\hy Befehl
      direkt zuvor gesetzte Argument} mit einem Punkt oder \verb|\banotdot|
      oder \verb|\bahasdot| endet. Sonst druckt \verb|\bapoint| einen Punkt.
\item \verb|\bastrut| steht vor Leerzeichen (\verb*|\ |) oder
      \verb|\hskip|\hy\ oder \verb|\hspace|\hy Befehlen, um ihnen einen
      Zeilenumbruch zu erlauben. (Falls das vorausgehend gesetzte Argument
      eines \BibArts\hy Befehls mit einem Punkt endet, k"onnte dies sonst
      einen Zeilenumbruch oft verbieten.) \verb|\bastrut| ist freilich vor
      \verb|~| und anderen gesch"utzten Leerzeichen nicht n"otig.
\item Ist das erste 'echte' Zeichen eines Separators ein hohes Zeichen,
      sollten Sie dann einen corr-Befehl davorsetzen, wenn das vorausgehend
      gesetzte Argument schr"aggestellt sein k"onnte (vgl.\ unten S.\,\pageref{hervor}). 
      Vor Punkt, Komma oder \verb|\bapoint| ist nie ein corr\hy Befehl n"otig. 
\item W"ahrend die \LaTeX\hy eigene \textit{italics}\hy Korrektur \verb|\/|
      vor oder nach Leerzeichen nichts tut, treten \BibArts\hy
      corr\hy Befehle immer in Aktion. In den Voreinstellungen sind corr\hy
      Befehle auch an Leerzeichen gesetzt. Ein Grund ist, dass Sie sich
      anhand der Vorfertigungen eines Separators dar"uber informieren
      k"onnen, wie der zugeh"orige corr\hy Befehl hei"st. Ob Sie dies in
      Ihren Neudefinitionen auch machen, ist freilich wahlfrei.
\item Steht ein corr\hy Befehl vor einem Leerzeichen, unterbindet dies einen 
      Zeilenumbruch; in umgekehrter Reihenfolge ist einer erlaubt. Weiter
      soll \verb|\bastrut| \textbf{nie} \textit{nach} einem corr\hy Befehl
      stehen. Um einen Zeilenumbruch zu erlauben, gilt die Reihenfolge
      \verb|\bastrut| $-$ Leerzeichen $-$ corr\hy Befehl. Es gibt zwei
      Ausnahmen: \verb|\baabkcorr|, wenn es in den Separatoren
      \verb|\abkdefopen| oder \verb|\defabkclose| steht; 
      \verb|\bakxxcorr| immer (S.\,\pageref{bakxxcorr}): 
      Die m"ussen \textit{vor} Leerzeichen stehen!
      Das gilt nach den Argumenten, f"ur die Schr"agschriften einstellbar
      sind $-$ also nur f"ur abk- und k\hy Befehle.\label{Systematik}
\item Wenn ein Separator von mehreren \BibArts\hy Befehlen benutzt wird,
      m"ussen manche corr-Befehle in \BibArts\hy if\hy Argumenten stehen.
      \verb|\ifbashortcite| etwa trennt k\fhy Befehle von allen anderen
      \BibArts\hy Befehlen. Weiter kann eine Unterscheidung mit
      \verb|\ifbaibidem| n"otig werden, weil ein f"ur einen ganzen
      \BibArts\hy Befehl eingesetztes \textsc{ebd.} immer aufrecht ist.
      \verb|\pernosep| etwa wird nur \textit{sonst} abgearbeitet
                        (siehe S.\,\pageref{pernosep} und S.\,\pageref{pernosep2} samt 
                        Anm.\,\ref{pernosep2}).
\end{itemize}



\newpage
\section{Sprachabh"angige Separatoren (\kern-.02em\textit{captions})}\label{Sect12}\label{SprachSep}

Wenn Sie mit dem \BibArts-Befehl \verb|\sethyphenation{|\textit{Sprache}\verb|}| 
oder alternativ f"ur Zitatbl"ocke mit
\verb|\begin{originalquote}[|\textit{Sprache}\verb|]| eine bestimmte\pdfko{1.125}\ 
Sprache einstellen, wird nur bestimmt, wie Worte getrennt werden (Trennsatz).
Andere Schalter stellen die Basissprache des Textes ein, bestimmen also,
in welcher Sprache Text\hy Separatoren (\kern-.1em\textit{captions}) zu drucken 
sind $-$\pdfko{1.25}\  
ob beispielsweise die Abk"urzung f"ur Seite \textit{S.} oder \textit{p.} lautet. 
W"ahrend der\pdfko{1.25}\ 
Trennsatz oft mehrfach in einem Text f"ur fremdsprachige w"ortliche Zitate 
jeweils angepasst wird, bleibt die Sprache der Text\hy Separatoren in einem 
Text meist durchgehend gleich. \BibArts\hy \textit{captions} werden also meist 
im Vorspann des\pdfko{1.25}\  
\LaTeX\hy Textes festgelegt; \verb|\bacaptionsgerman| 
ist voreingestellt. \BibArts\ stellt\pdfko{1.5}\  
gegenw"artig zudem \verb|\bacaptionsenglish| 
und \verb|\bacaptionsfrench| bereit.\pdfko{1}\  
Falls Sie \verb|ngerman.sty| nutzen, 
m"ussen Sie ggf.\ die dort bereitgestellten Befehle \verb|\captionsenglish| 
oder \verb|\captionsfrench| zus"atzlich setzen, denn die\pdfko{1.25}\  
\verb|\bacaptions|...\hy Befehle "andern die Voreinstellungen f"ur 
\LaTeX\hy \textit{captions} wie\pdfko{1.75}\  
etwa \textit{\chaptername} (\verb|\chaptername|) nicht. 
\verb|\bacaptions|...\hy Befehle "andern nur die\pdfko{1.5}\  
Voreinstellungen f"ur \BibArts\hy Befehle; dementsprechend bewirkt etwa

\vspace{.75ex}{\small\noindent
\verb|  \bacaptionsenglish \vli{}{}{Text}[20]  => |{\bacaptionsenglish \printonlyvli{}{}{Text}[20]} \\
\verb|  \bacaptionsgerman  \vli{}{}{Text}[20]  => |{\bacaptionsgerman  \printonlyvli{}{}{Text}[20]}}

\vspace{1.325ex}\noindent
Bei der \textit{Einstellung} sprachabh"angiger Separatoren gibt es einen
Sonderfall: Wie oben Seite~\pageref{setibidem}
beschrieben, wird \textsc{ebd.}\ mit \verb|\setibidem{g}{ebenda}{}|\pdfko{1}\ 
in \textsc{ebenda} umgestellt; im letzten geschweifte Klammerpaar
kann alternativ ein Punkt stehen, falls eine Abk"urzung gedruckt werden
soll (der Punkt, mit dem die Abk"urzung endet). In den beiden anderen
Sprachen lauten die Voreinstellungen \textsc{ibid.} im Franz"osischen
und \textsc{ibidem} im Englischen. Dies\pdfko{1}\  
l"asst sich "andern, etwa vertauschen.\footnote{
\texttt{\bs setibidem\{e\}\{ibid\bs kern -0.07em\}\{.\}} \setibidem{e}{ibid\kern -0.07em}{.} \\
\hspace*{5.5mm} \texttt{\bs setibidem\{f\}\{ibidem\}\{\}} \setibidem{f}{ibidem}{} \\
\hspace*{5.5mm} \texttt{\ \ \ \ \ \ \ \ \ \ \ \ \ \ \ \ \ \ \ \ \ \ \ \ \bs kli\{\}\{Text\}[20]. \ => } \bacaptionsgerman \printonlykli{}{Text}[20]. \\
\hspace*{5.5mm} \texttt{\bs bacaptionsenglish \ \ \ \ \ \bs kli\{\}\{Text\}[20]. \ => } \bacaptionsenglish \printonlykli{}{Text}[20]. \\
\hspace*{5.5mm} \texttt{\bs bacaptionsfrench \ \ \ \ \ \ \bs kli\{\}\{Text\}[20]. \ => } \bacaptionsfrench \printonlykli{}{Text}[20].}

\vspace{1ex}\noindent
Alle anderen \textit{captions} d"urfen Sie mit \verb|\renewcommand| "andern, 
wenn Ihnen\pdfko{1}\ 
die Voreinstellungen von \BibArts\ nicht gefallen. Z.\,B.
\verb|\vli{}{}{Text}| w"urde nach \verb|\renewcommand{\ganonymousname}{[?]}| 
ausdrucken: {\renewcommand{\ganonymousname}{[?]} \printonlyvli{}{}{Text}}.\pdfko{.5}\
Die deutschen \textit{captions} beginnen mit \verb|\g|..., die englischen mit 
\verb|\e|...\ und die\pdfko{1}\ 
franz"osischen mit \verb|\f|...\,. Ich liste nun die 
in \verb|bibarts.sty| definierten Voreinstellungen auf. Nach Silben mit 
Sonderzeichen (wie "s oder \'e) ist jeweils die Trennhilfe \verb|\-| 
eingesetzt; da diese die erste erlaubte Trennstelle in einem\pdfko{.5}\ 
Wort definiert, 
sind ggf.\ weitere \verb|\-| im davorstehenden Wortteil zu setzen. 

\vspace{1ex}\noindent{\small
\verb|\ganonymousname => {[Anonym]}| \\[-.8ex]
\verb|\eanonymousname => {[Anonymous]}| \\[-.8ex]
\verb|\fanonymousname => {[Anonyme]}| \\[-.2ex]
Kein Autor in v- oder k-Befehlen eingetippt: \verb|\kli{}{T} =>| \printonlykli{}{T}.
\\[.8ex]%
\verb|\geditorname => {[\kern 0.04em Hrsg.]\kern 0.02em}| \\[-.8ex]
\verb|\eeditorname => {(\kern -0.03em ed.\kern -0.06em)\kern 0.02em}| \\[-.8ex]
\verb|\feditorname => {(\kern -0.03em \'ed.\kern -0.06em)\kern 0.02em}| \\[-.2ex]
Text f"ur Befehl \verb|\editor =>| \editor\ \ (siehe oben S.\,\pageref{editor}).
\\[.8ex]%
\verb|\geditorpname => {[\kern 0.04em Hrsg.]\kern 0.02em}| \\[-.8ex]
\verb|\eeditorpname => {(\kern -0.03em eds.\kern -0.08em)\kern 0.02em}| \\[-.8ex]
\verb|\feditorpname => {(\kern -0.03em \'ed.\kern -0.06em)\kern 0.02em}| \\[-.2ex]
Text f"ur Befehl \verb|\editors| (mehrere Herausgeber); mit \verb|\bacaptionsenglish|: {\bacaptionsenglish\editors}.
\\[.8ex]%
\verb|\gidemname => {ders\kern -0.04em.}| \\[-.8ex]
\verb|\eidemname => {idem}| \\[-.8ex]
\verb|\fidemname => {le m\^e\-me}| \\[-.2ex]
Derselbe Autor wird direkt hintereinander mit verschiedenen Werken zitiert: \\[-.2ex]
\verb|[m]| direkt nach v- und k-Befehlen: \verb|\kli[m]{N}{T} =>| \printonlykli[m]{N}{T}.
\\[.8ex]%
\verb|\geademname => {dies\kern -0.04em.}| \\[-.8ex]
\verb|\eeademname => {eadem}| \\[-.8ex]
\verb|\feademname => {la m\^e\-me}| \\[-.2ex]
\verb|[f]| direkt nach v- und k-Befehlen: \verb|\kli[f]{N}{T} =>| \printonlykli[f]{N}{T}.
\\[.8ex]%
\verb|\giidemname => {diesn\kern -0.07em.}| \\[-.8ex]
\verb|\eiidemname => {iidem}| \\[-.8ex]
\verb|\fiidemname => {les m\^e\-mes}| \\[-.2ex]
\verb|[p]| direkt nach v- und k-Befehlen: \verb|\kli[p{}]{N1}*{N2}{T} =>| \printonlykli[p{}]{N1}*{N2}{T}.
\\[.8ex]%
\verb|\gvolname => {, Bd.\,}| \\[-.8ex]
\verb|\evolname => {, vol.\,}| \\[-.8ex]
\verb|\fvolname => {, vol.\,}| \\[-.2ex]
Bandangabe nach v-, k-, arq- und per-Befehlen: \verb+\per{ZfG.}|2| =>+ \per{ZfG.}|2|.
\\[.8ex]%
\verb|\gvolpname => {, Bde.\,}| \\[-.8ex]
\verb|\evolpname => {, vols.\,}| \\[-.8ex]
\verb|\fvolpname => {, vol.\,}| \\[-.2ex]
Nach v-, k-, arq- und per-Befehlen: 
\verb+\per{ZfG.}|2-3| =>+ \per{ZfG.}|2-3|.\footnote{\BibArts\ ermittelt
 einen vorliegenden Plural selbst"andig, indem es das Argument nach 
 \texttt{-}, [Komma], \texttt{\bs hy}, \texttt{\bs fhy}, \texttt{\bs f}, 
 \texttt{\bs ff}, \texttt{\bs sq}, und \texttt{\bs sqq} durchsucht,
 oder setzt die Plural\hy \textit{caption}\pdfko{.75}\ 
 ein, wenn Sie \texttt{\bs baplural} setzen; vgl.\ oben S.\,\pageref{baplural}.}
\\[.8ex]%
\verb|\gpername => {\ifbaibidem{, Nr.\,}{\pernosep}}| \\[-.8ex]
\verb|\epername => {\ifbaibidem{, no.\,}{\pernosep}}| \\[-.8ex]
\verb|\fpername => {\ifbaibidem{, n\fup{o}\,}{\pernosep}}| \\[-.2ex]
Heftangaben im Singular (mit/ohne \textsc{ebd.}) in v-, k-, arq- und per-Befehlen.\footnote{
 \texttt{\bs per\{ZfG.\}\_5\_ und \bs per\{ZfG.\}\_6\_ \ \ \ \ => } 
 \per{ZfG.}_5_ und \per{ZfG.}_6_.}
\\[.8ex]%
\verb|\gperpname => {\ifbaibidem{, Nr.\,}{\pernosep}}| \\[-.8ex]
\verb|\eperpname => {\ifbaibidem{, no.\,}{\pernosep}}| \\[-.8ex]
\verb|\fperpname => {\ifbaibidem{, n\fup{os}\,}{\pernosep}}| \\[-.2ex]
Heftangaben im Plural (mit/ohne \textsc{ebd.}) in v-, k-, arq- und per-Befehlen. \\[-.2ex]
Beispiel unter \verb|\bacaptionsfrench| in der Fu"snote.\footnote{\bacaptionsfrench
 \texttt{\bs per\{Jour\}\_4-5\_ et \bs per\{Jour\}\_6-7\_ \ => }
 \printonlyper{Jour}_4-5_ et \printonlyper{Jour}_6-7_.}
\newpage\noindent%\\[.8ex]%
\verb|\gisonfolioname => {, Bl.\,}| \\[-.8ex]
\verb|\eisonfolioname => {, folio\nobreak \ }| \\[-.8ex]
\verb|\fisonfolioname => {, folio\nobreak \ }| \\[-.2ex]
Blattangabe nach v-, k-, arq- und per-Befehlen: \verb|\arq{}{PRO}(2) =>| \printonlyarq{}{PRO}(2).
\\[.8ex]%
\verb|\gisonfoliopname => {, Bl.\,}| \\[-.8ex]
\verb|\eisonfoliopname => {, folii\nobreak \ }| \\[-.8ex]
\verb|\fisonfoliopname => {, folii\nobreak \ }| \\[-.2ex]
Sichtbar in {\bacaptionsenglish\verb|\bacaptionsenglish| \verb|\arq{}{PRO}(2-3) =>| \printonlyarq{}{PRO}(2-3).}
\\[.8ex]%
\verb|\gisonxfolioname => {, dort: Bl.\,}| \\[-.8ex]
\verb|\eisonxfolioname => {, there: Folio\nobreak \ }| \\[-.8ex]
\verb|\fisonxfolioname => {, l\`a: Folio\nobreak \ }| \\[-.2ex]
\verb|*|-Blatt nach v-, k-, arq- und per-Befehlen: \verb|\arq{}{PRO}*(2) =>| \printonlyarq{}{PRO}*(2).
\\[.8ex]%
\verb|\gisonxfoliopname => {, dort: Bl.\,}| \\[-.8ex]
\verb|\eisonxfoliopname => {, there: Folii\nobreak \ }| \\[-.8ex]
\verb|\fisonxfoliopname => {, l\`a: Folii\nobreak \ }| \\[-.2ex]
Sichtbar in {\bacaptionsenglish\verb|\bacaptionsenglish| \verb|\arq{}{PRO}*(2-3) =>| \printonlyarq{}{PRO}*(2-3).}
\\[.8ex]%
\verb|\gisonpagename => {, S.\,}| \\[-.8ex]
\verb|\eisonpagename => {, p.\,}| \\[-.8ex]
\verb|\fisonpagename => {, p.\,}| \\[-.2ex]
Seitenangabe nach v-, k-, arq- und per-Befehlen: \verb|\kli{N}{T}[2] =>| \printonlykli{N}{T}[2].
\\[.8ex]%
\verb|\gisonpagepname => {, S.\,}| \\[-.8ex]
\verb|\eisonpagepname => {, pp.\,}| \\[-.8ex]
\verb|\fisonpagepname => {, p.\,}| \\[-.2ex]
Sichtbar in {\bacaptionsenglish\verb|\bacaptionsenglish| \verb|\kli{N}{T}[2-3] =>| \printonlykli{N}{T}[2-3].}
\\[.8ex]%
\verb|\gisonxpagename => {, dort: S.\,}| \\[-.8ex]
\verb|\eisonxpagename => {, there: p.\,}| \\[-.8ex]
\verb|\fisonxpagename => {, l\`a: p.\,}| \\[-.2ex]
\verb|*|-Seite nach v-, k-, arq- und per-Befehlen: \verb|\kli{N}{T}*[2] =>| \printonlykli{N}{T}*[2].
\\[.8ex]%
\verb|\gisonxpagepname => {, dort: S.\,}| \\[-.8ex]
\verb|\eisonxpagepname => {, there: pp.\,}| \\[-.8ex]
\verb|\fisonxpagepname => {, l\`a: p.\,}| \\[-.2ex]
Sichtbar in {\bacaptionsenglish\verb|\bacaptionsenglish| \verb|\kli{N}{T}*[2-3] =>| \printonlykli{N}{T}*[2-3].}
\\[.8ex]%
\verb|\gbibtitlename => {Quellen und Literatur}| \\[-.8ex]
\verb|\ebibtitlename => {Bibliography}| \\[-.8ex]
\verb|\fbibtitlename => {Bibliographie}| \\[-.2ex]
Titel gesamter Belegapparat ("Uberschrift \BibArts-Anhang) \verb|\printbibtitle|.
\\[.8ex]%
\verb|\gabktitlename => {Ab\-k\"ur\-zungen}| \\[-.8ex]
\verb|\eabktitlename => {Abbreviations}| \\[-.8ex]
\verb|\fabktitlename => {Ab\-r\'e\-viations}| \\[-.2ex]
Titel Abk"urzungsverzeichnis \verb|\printabk| und \verb|\printnumabk| bzw.\ \verb|\printabktitle|.
\\[.8ex]%
\verb|\gvlititlename => {Literatur}| \\[-.8ex]
\verb|\evlititlename => {Literature}| \\[-.8ex]
\verb|\fvlititlename => {Travaux}| \\[-.2ex]
Titel Literaturliste \verb|\printvli| und \verb|\printnumvli| bzw.\ \verb|\printvlititle|.
\\[.8ex]%
\verb|\ghinttovliname => {[L]}          %\| \\[-.8ex]
\verb|\ehinttovliname => {[L]}          % )  Alle ohne italics-Korrektur!| \\[-.8ex]
\verb|\fhinttovliname => {[T]}          %/| \\[-.2ex]
Hinweis auf Liste mit vollen Literaturangaben: \verb|\kli{N}{T} =>| \printonlykli{N}{T}.
\\[.8ex]%
\verb|\gvqutitlename => {Gedruckte Quellen}| \\[-.8ex]
\verb|\evqutitlename => {Published Documents}| \\[-.8ex]
\verb|\fvqutitlename => {Sources im\-pri\-m\'ees}| \\[-.2ex]
Titel Verzeichnis gedruckter Quellen \verb|\printvqu| (plus -\verb|num|-) bzw.\ \verb|\printvqutitle|.
\\[.8ex]%
\verb|\ghinttovquname => {[Q]}          %\| \\[-.8ex]
\verb|\ehinttovquname => {[D]}          % )  Alle ohne italics-Korrektur!| \\[-.8ex]
\verb|\fhinttovquname => {[S]}          %/| \\[-.2ex]
Hinweis auf Verzeichnis mit vollen Quellenangaben: \verb|\kqu{N}{T} =>| \printonlykqu{N}{T}.
\\[.8ex]%
\verb|\gpertitlename => {Zeitschriften}| \\[-.8ex]
\verb|\epertitlename => {Periodicals}| \\[-.8ex]
\verb|\fpertitlename => {P\'e\-riodiques}| \\[-.2ex]
Titel Zeitschriftenverzeichnis \verb|\printper| und \verb|\printnumper| bzw.\ \verb|\printpertitle|.
\\[.8ex]%
\verb|\garqtitlename => {Ungedruckte Quellen}| \\[-.8ex]
\verb|\earqtitlename => {Unpublished Documents}| \\[-.8ex]
\verb|\farqtitlename => {Sources in\-\'edi\-tes}| \\[-.2ex]
Titel Archivquellenverzeichnis \verb|\printarq|,  \verb|\printnumarq| bzw.\ \verb|\printarqtitle|.
\\[.8ex]%
\verb|\gvkctitlename => {Verwendete Kurztitel}| \\[-.8ex]
\verb|\evkctitlename => {Shortened References}| \\[-.8ex]
\verb|\fvkctitlename => {Titres ab\-r\'e\-g\'ees}| \\[-.2ex]
Titel Kurzzitateverzeichnis \verb|\printnumvkc| und \verb|\printvkc| bzw.\ \verb|\printvkctitle|.
\\[.8ex]%
\verb|\ggrrtitlename => {Ortsregister}| \\[-.8ex]
\verb|\egrrtitlename => {Geographical index}| \\[-.8ex]
\verb|\fgrrtitlename => {Registre g\'eo\-graphique}| \\[-.2ex]
Titel Ortsregister f"ur Liste \verb|\printnumgrr| und \verb|\printgrr| bzw.\ \verb|\printgrrtitle|.
\\[.8ex]%
\verb|\gprrtitlename => {Personenregister}| \\[-.8ex]
\verb|\eprrtitlename => {Person index}| \\[-.8ex]
\verb|\fprrtitlename => {Registre des personnes}| \\[-.2ex]
Titel Personenreg.\ f"ur Liste \verb|\printnumprr| und \verb|\printprr| bzw.\ \verb|\printprrtitle|.
\\[.8ex]%
\verb|\gsrrtitlename => {Sachregister}| \\[-.8ex]
\verb|\esrrtitlename => {Subject index}| \\[-.8ex]
\verb|\fsrrtitlename => {Registre des sujets}| \\[-.2ex]
Titel Sachregister f"ur Liste \verb|\printnumsrr| und \verb|\printsrr| bzw.\ \verb|\printsrrtitle|.
\\[.8ex]%
\verb|\gfolpagename => {\badelspacebefore\,f\kern -0.1pt.\bahasdot}| \\[-.8ex]
\verb|\efolpagename => {\badelspacebefore\,f\kern -0.1pt.\bahasdot}| \\[-.8ex]
\verb|\ffolpagename => {\badelspacebefore\ sq.\bahasdot}| \\[-.2ex]
Abk"urzung 'folgende' (\verb|\f|\,=\,\verb|\sq|): \verb|\per{ZfG.}_2\sq_[3 \f] =>| \per{ZfG.}_2\sq_[3 \f].
\\[.8ex]%
\verb|\gxfolpagename => {\badelspacebefore\,ff\kern -0.1pt.\bahasdot}| \\[-.8ex]
\verb|\exfolpagename => {\badelspacebefore\,ff\kern -0.1pt.\bahasdot}| \\[-.8ex]
\verb|\fxfolpagename => {\badelspacebefore\ sqq.\bahasdot}| \\[-.2ex]
Abk"urzung 'mehrere folgende': \verb|\per{ZfG.}_2\ff_[3\sqq] =>| \per{ZfG.}_2\ff_[3\sqq].
\\[2.5ex]%
\verb|\gannouncektitname => |{\footnotesize\verb|{\bastrut\ (\kern 0.015em im Folgenden \baupcorr}|} \\[-.4ex]
\hspace*{5cm} \verb|\gannouncekendname => {)}| \\[-.4ex]
\verb|\eannouncektitname => |{\footnotesize\verb|{\bastrut\ (\kern -0.02em cited as \baupcorr}|} \\[-.4ex]
\hspace*{5cm} \verb|\eannouncekendname => {)}| \\[-.4ex]
\verb|\fannouncektitname => |{\footnotesize\verb|{\bastrut\ (\kern 0.02em par la suite \baupcorr}|} \\[-.4ex]
\hspace*{5cm} \verb|\fannouncekendname => {)}| \\[.2ex]
\verb|\ktit|""-Ank"undigung: \verb|\vli{V}{N}{\ktit{T}} =>| {\notktitaddtok\printonlyvli{V}{N}{\ktit{T}}}.
\newpage\noindent%\\[.8ex]%
\verb|\grefvbegname => {(}| \\[-.8ex]
\hspace*{1cm} \verb|\grefvendname => {\barefcorr)}| \\[-.8ex]
\verb|\erefvbegname => {[\nobreak \hskip 1pt plus 0pt}| \\[-.8ex]
\hspace*{1cm} \verb|\erefvendname => {\nobreak \hskip 1pt plus 0pt\barefcorr]}| \\[-.8ex]
\verb|\frefvbegname => {(}| \\[-.8ex]
\hspace*{1cm} \verb|\frefvendname => {\barefcorr)}| \\[-.2ex]
Klammern in \verb|\conferize| f"ur k-Befehle: \verb|\kli{Luhmann}{Soziale Systeme} =>| \\[-.5ex]
{\conferize \kli{Luhmann}{Soziale Systeme}}. \\[-.5ex]
Die \textit{italics}\hy Korrektur \verb|\barefcorr| ist f"ur Verweise bes.\ auf \verb|{minipage}|-Fu"snoten.
\\[.8ex]%
\verb|\gconfername => {\kern -0.03em wie}| \\[-.8ex]
\verb|\econfername => {\kern -0.05em cf.\bahasdot}| \\[-.8ex]
\verb|\fconfername => {\kern -0.03em op.\ cit.\bahasdot}| \\[-.2ex]
Bezugsworte im Querverweis des eben genannten \verb|\conferize|-Stils.
\\[.8ex]%
\verb|\grefvpagname => {S.\,}| \\[-.8ex]
\verb|\erefvpagname => {p.\,}| \\[-.8ex]
\verb|\frefvpagname => {p.\,}| \\[-.2ex]
Seitenabk"urzung im Querverweis des eben genannten \verb|\conferize|-Stils; und \\[-.5ex]
ebenso f"ur \BibArts-Querverweise: \verb|\baref{Mueller} =>| \baref{Mueller}, wozu auch \\[-.5ex]
die oben unter \verb|\grefvbegname| ... genannten Klammersymbole benutzt werden.
\\[.8ex]%
\verb|\grefverbname => {siehe}| \label{grefverbname} \\[-.8ex]
\verb|\erefverbname => {see}| \\[-.8ex]
\verb|\frefverbname => {voir}| \\[-.2ex]
Bezugswort im eben genannten \verb|\baref|-Querverweis.
\\[.8ex]%
\verb|\grefvfntname => {, Anm.\,}| \\[-.8ex]
\verb|\erefvfntname => {, n.\,}| \\[-.8ex]
\verb|\frefvfntname => {, n.\,}| \\[-.2ex]
Abk"urzung f"ur '\kern-.05em Anmerkung' oder 'Fu"snote' in den \verb|\conferize|-k-Befehle oben \\[-.5ex]
und f"ur \BibArts\hy Querverweise: \verb|\baref{XX} =>| \baref{XX}.\footnote{
\balabel{XX}\texttt{\bs balabel\{XX\}}.
\BibArts\ bemerkt automatisch, ob dies in einer Fu"snote steht.}
\\[.8ex]%
\label{erscheditionname}%
\verb|\gerscheditionname => {\teskip Auf{\kern.03em}l.,}| \\[-.8ex]
\verb|\eerscheditionname => {\fupskip edition,}| \\[-.8ex]
\verb|\ferscheditionname => {\fupskip \'edi\-tion,}  %%| 
       \texttt{Vgl. oben S.\pageref{fordinalf}.} \\[-.2ex]
Auf"|lage-Abk"urzung in \verb|\ersch[4]{Stuttgart}{1899} =>| \ersch[4]{Stuttgart}{1899}.
\\[.8ex]%
\verb|\gerschvolumename => {Bd.,}    \gerschvolumepname => {Bde.,}| \\[-.8ex]
\verb|\eerschvolumename => {vol.,}   \eerschvolumepname => {vols.,}| \\[-.8ex]
\verb|\ferschvolumename => {vol.,}   \ferschvolumepname => {vol.,}| \\[-.2ex]
Band-Abk"urzung in \verb+\ersch|3|{Stuttgart}{1899} =>+ \ersch|3|{Stuttgart}{1899}.
\\[.8ex]%
\verb|\gerschnohousename => {\oO,}| {\footnotesize\verb|=> {o.\kern 0.1em O\kern -0.08em.\bahasdot}|}\\[-.8ex]
\verb|\eerschnohousename => {n.\kern 0.15em p.,}            | ('no place') \\[-.8ex]
\verb|\ferschnohousename => {s.\kern 0.15em l\kern 0.02em.,}| \ ('sans lieu') \\[-.2ex]
Kein Erscheinungsort getippt in \verb|\ersch{}{1899} =>| \ersch{}{1899} \ ('ohne Ort').
\\[.8ex]%
\verb|\gerschnoyearname => {\oJ}  | {\footnotesize\verb|=> {o.\kern 0.1em J\kern -0.09em.\bahasdot}|} \\[-.8ex] \label{gerschnoyearname}
\verb|\eerschnoyearname => {n.\kern 0.13em d.\bahasdot}| \ ('no date') \\[-.8ex]
\verb|\ferschnoyearname => {s.\kern 0.13em d.\bahasdot}| \ ('sans date') \\[-.2ex]
Kein Jahr in \verb|\vli{}{Sam}{Titel, \ersch{Paris}{}}. =>| 
{\notktitaddtok\printonlyvli{}{Sam}{Titel, \ersch{Paris}{}}.}
}

\noindent
\textbf{Separater Ausdruck von vorgefertigten Textelementen}\\[1ex]
Wenn Befehle, die \verb|\bahasdot| oder \verb|\banotdot| nutzen, nicht am 
Ende des Arguments eines \BibArts\fhy Hauptbefehls stehen, kann dies 
einen Zeilenumbruch verhindern. An \verb*|\oJ, | wird umgebrochen, aber ein 
\verb|\oJ| \textit{direkt} folgendes Leerzeichen ist gesch"utzt. Dann k"onnen 
Sie \verb|\strut| einf"ugen. Im freien Text ist also \verb*|\oJ\strut\ | 
statt \verb*|\oJ\ | zu tippen. Das gilt auch, wenn ein Befehl \verb|\oJ| ausf"uhrt: 
\verb|\ersch{Stuttgart}{}\strut\ next =>| \ersch{Stuttgart}{}\strut\ next.
Am Ende von Hauptbefehlen macht \BibArts\ dies f"ur Sie eigenst"andig: 
\verb|\vli{}{N.}{...., \ersch{Stuttgart}{}} next =>| 
\printonlyvli{}{N.}{...., \ersch{Stuttgart}{}} next. Auch ist es der
Hauptbefehl, der Ihren Satzende\hy Punkt nach einer
automatisch eingef"ugten Abk"urzung 'verschluckt'; Beispiel
ist der Hauptbefehl \verb|\vli{}{N.}{...., \ersch{Stuttgart}{}}. =>|
\printonlyvli{}{N.}{...., \ersch{Stuttgart}{}}. 
Oder \verb|\vli{}{N.}{...., \oJ}. =>| \printonlyvli{}{N.}{...., \oJ}.
Dagegen ergibt im freien Text \verb|\oJ. =>| \oJ. Und
\verb|\ersch{Stuttgart}{}. =>| \ersch{Stuttgart}{}. Auch stellen
nur \BibArts\hy Hauptbefehle das \textit{spacing} wie
oben S.\,\pageref{nonfrenchspacing} geschildert ein.
Sonst ist \verb|\ersch{Stuttgart}{}\strut\newsentence Next| stets m"oglich,
bei 'englischem' \textit{spacing} sogar n"otig:

\vspace{.75ex}
{\small
{\frenchspacing \verb|\frenchspacing     |
\ersch{Stuttgart}{}\strut\newsentence Next
}

{\nonfrenchspacing \verb|\nonfrenchspacing  |
\ersch{Stuttgart}{}\strut\newsentence Next
}
}

\vspace{1ex}\noindent
"Ahnliches gilt \textit{innerhalb} von \BibArts\hy Hauptbefehlen.
Am besten sollten Sie nach \verb|\oO| ein Komma setzen: 
\verb|\vli{}{N.}{...., \oO, \oJ} =>|
\printonlyvli{}{N.}{...., \oO, \oJ}.
Damit umschifft auch \verb|\ersch{}{2014}| die genannten Probleme: 
\ersch{}{2014}\hspace{.2em} (\verb|\oO| steht im Beispiel 
\textit{vor} dem Ende des Arguments eines v\fhy Befehls).

\vspace{1.5ex}\noindent
\fbox{\parbox{.98\textwidth}{\sffamily 
Diese Umbruchprobleme k"onnen auftreten, 
falls Sie\hspace{-.05em} \texttt{\bs ersch\{}{\normalfont\textit{xxx}}\texttt{\}\{\}} 
oder \texttt{\bs oO} und \texttt{\bs oJ} au"serhalb
der \BibArts\hy Hauptbefehle verwenden $-$ um "uberall dasselbe Druckbild 
der Abk"urzung zu erhalten, oder, um sp"ater einheitlich "andern zu k"onnen,
was \texttt{\bs oJ} druckt, z.\,B.: \texttt{\bs renewcommand\{\bs poJ\}\{ohne Jahr\}}
\hspace{.3em} (S.\,\pageref{poJ}). $-$\,Es ist nur dann relevant, falls 
Sie nicht einfach von Hand \texttt{o.\bs,J.} tippen!\,$-$}}


\vfill\noindent
Einige vorgefertigte Textelemente wurden gegen"uber Version 2.0 \textit{ge"andert}. 
Die \textit{alten} Definitionen waren (Wiederherstellen mit \verb|\renewcommand| m"oglich):

\vspace{.5ex}\noindent{\small
\verb|   \gbibmarkname      => {im folgenden: }| \\[-.5ex]
\verb|   \eannouncektitname => { (\kern -0.02em cited as \baupcorr}| \\[-.5ex]
\verb|   \fannouncektitname => { (\kern 0.02em par la suite \baupcorr}| \\[-.5ex]
\verb|   \gannouncektitname => { (\kern 0.015em im folgenden \baupcorr}| \\[-.5ex]
\verb|   \eerschnohousename => {no publishing house,}| \\[-.5ex]
\verb|   \ferschnohousename => {sans maison d'{}\'edi\-tion,}| \\[-.5ex]
\verb|   \eerschnoyearname  => {no exact year\kern -0.04em}| \\[-.5ex]
\verb|   \ferschnoyearname  => {sans an\-n\'ee}| \\[-.5ex]
\verb|   \evkctitlename     => {Short Titles}| 
}


\newpage
\section{Die \BibArts\hy Hauptbefehle}\label{Sect13}\label{Hauptbefehle}

Hier sind die Befehle zur Belegeingabe nochmals zusammengestellt:

\vspace{4ex}
 \begin{tabular}{rrrc}%
  \bf Basis      & \verb|=| \bf addto-Teil & \verb|+| \bf printonly-Teil & \bf Zusatzf"ullung \\ \hline
                 &                       &                           &                 \\
  \verb|\vli|    & \verb|= \addtovli|    & \verb|+ \printonlyvli|    &                 \\
  \verb|\vqu|    & \verb|= \addtovqu|    & \verb|+ \printonlyvqu|    &                 \\
  \verb|\kli|    & \verb|= \addtokli|    & \verb|+ \printonlykli|    &                 \\
  \verb|\kqu|    & \verb|= \addtokqu|    & \verb|+ \printonlykqu|    &                 \\
                 &                       &                           &                 \\
  \verb|\xvli|   & \verb|= \xaddtovli|   & \verb|+ \xprintonlyvli|   &                 \\
  \verb|\xvqu|   & \verb|= \xaddtovqu|   & \verb|+ \xprintonlyvqu|   &                 \\
  \verb|\xkli|   & \verb|= \xaddtokli|   & \verb|+ \xprintonlykli|   &                 \\
  \verb|\xkqu|   & \verb|= \xaddtokqu|   & \verb|+ \xprintonlykqu|   &                 \\
                 &                       &                           &                 \\
  \verb|\per|    & \verb|= \addtoper|    & \verb|+ \printonlyper|    & \verb|\fillper| \\
  \verb|\arq|    & \verb|= \addtoarq|    & \verb|+ \printonlyarq|    & \verb|\fillarq| \\
                 &                       &                           &                 \\
  \verb|\abkdef| & \verb|= \addtoabkdef| & \verb|+ \printonlyabkdef| &                 \\
  \verb|\defabk| & \verb|= \addtodefabk| & \verb|+ \printonlydefabk| &                 \\
  \verb|\abk|    & \verb|= \addtoabk|    & \verb|+ \printonlyabk|    &                 \\
                 &                       &                           &                 \\
                 & \verb|\addtogrr|      &                           & \verb|\fillgrr| \\
                 & \verb|\addtoprr|      &                           & \verb|\fillprr| \\
                 & \verb|\addtosrr|      &                           & \verb|\fillsrr| \\
 \end{tabular}

\vspace{5.5ex}\noindent 
\textbf{Spielregeln:}

\begin{itemize}
\item k\hy Beleg ('Kurzzitat') erst nach Einf"uhrung eines Werks mittels v\hy Beleg
\item abk erst nach Einf"uhrung durch abkdef oder defabk
\item per hat ein Argument und arq zwei (Schriftst"uck plus Signatur)
\item fill-Befehle k"onnen im zweiten Argument einmal an zentraler Stelle
      umf"anglichen Zusatztext aufnehmen, um das Stichwort im ersten Argument
      zu erkl"aren; das 'Stichwort' entspricht beim arq-Befehl dem zweiten Argument
      (der Signatur), bei per- und rr-Befehlen \textit{dem} Argument\pdfko{1.5}
\end{itemize}

\vspace{1.5ex}\noindent 
Au"serdem existiert noch \verb|\abkper|, das \verb|\abk + \per| ausf"uhrt.



\newpage
\section{Hervorhebung von \BibArts\hy Argumenten}\label{Sect14}\label{hervor}

Hier eine Zusammenstellung der Befehle f"ur die Einstellung von Schriften: 

\vspace{2.5ex}\noindent
\begin{tabular}{lll}
\textbf{Befehl}       & \textbf{Voreinstellung}              & \textbf{Alternative}     \\[1ex]
\verb|\authoremph|    & \verb|{\normalfont\scshape}|         & \verb|{\upshape\|\abra{...}\verb|}|        \\
\verb|\kxxemph|       & \verb|{}| \ (\verb|\kli|- und \verb|\kqu|\hy Titel)& alles (S.\,\pageref{kxxA}, \pageref{ntsepB}, \pageref{kxxB}) \\
\verb|\edibidemph|    & \verb|{\scshape}| \ (\textsc{ebd.}, \textsc{ders.}\ko)& \textbf{KEINE!} \\
\verb|\abkemph|       & \verb|{\sffamily}|                   & alles (S.\,\pageref{abkA}, \pageref{abkB}) \\
\verb|\abklistemph|   & \verb|{\bfseries}|                   & alles; \verb|{\abkemph}| \\
\verb|\arqemph|       & \verb|{\normalfont\sffamily}|        & \verb|{\upshape\|\abra{...}\verb|}|        \\
\verb|\arqlistemph|   & \verb|{\arqemph\relax\normalsize}|   & alles (vgl.\ unten)      \\
\verb|\peremph|       & \verb|{\normalfont\scshape}|         & \verb|{\upshape\|\abra{...}\verb|}|        \\
\verb|\perlistemph|   & \verb|{\peremph}|                    & alles (siehe\ unten)     \\
\verb|\xrrlistemph|   & \verb|{}| \ (Register-Stichworte)    & alles (S.\,\pageref{xrr})\\
\verb|\balistnumemph| & \verb|{\sffamily}|                   & \verb|{}| (auto-up: S.\,\pageref{listnum}) \\
\end{tabular}\label{arqemph2}

\vspace{2ex}\noindent
"Anderungen an diesen Befehlen lassen sich mit \verb|\renewcommand|
durchf"uhren.\pdfko{1}\ 
\verb|\authoremph|, \verb|\edibidemph|, \verb|\arqemph| 
und \verb|\peremph| l"asst sich auch etwa\pdfko{1.5}\  
\verb|\bfseries| zuweisen, 
aber nur \textit{nach} \verb|\upshape| oder besser \verb|\normalfont|
(in\pdfko{1.5}\ 
schr"aggestelltem "au"seren Umfeld w"urde \BibArts\ sonst jedes Mal 
warnen).\footnote{Mit 'alles' sind oben Standardschriften gemeint, keine
negativ geneigten Schriften!}
 
\vspace{.25ex}
Zudem sollten f"ur \verb|\edibidemph| nur \textsc{kleine kapitelle} 
als Basis verwendet
werden, denn nur das umgeht das Problem der 
Klein-/""Gro"sschreibung (\textsc{ebd.} und \textsc{ders.} m"ussen ja 
nicht immer am Anfang eines Satzes stehen)!

\vspace{.075ex}
\verb|\arqlistemph| und \verb|\perlistemph| sind so voreingestellt, dass 
sie (im wesentlichen) die Einstellungen von \verb|\arqemph| und 
\verb|\peremph| f"ur den Listenausdruck "ubernehmen. 
F"ur listemph\hy Befehle gilt "ubertragbar:

\vspace{.25ex}
{\footnotesize
\begin{verbatim}
   {\renewcommand{\perlistemph}{\slshape}%
    Auf den Ausdruck von \per{ShortMagazine} wirkt sich dies nicht aus!
    \renewcommand{\balistnumemph}{}    %% Zahlen nicht in sans serif %%
    \printnumper}
\end{verbatim}}

\vspace{.25ex}
{\renewcommand{\perlistemph}{\slshape}%
 \hspace{.2em}Auf den Ausdruck von \per{ShortMagazine} wirkt sich dies nicht aus!
 \renewcommand{\balistnumemph}{}    %% Zahlen nicht in sans serif %%
\vspace{-1.75ex}
 \printnumper}


\newcommand{\mynwarrow}{\hspace{1cm}\raisebox{1.1ex}{{\footnotesize$\nwarrow$}}\hspace{.3em}}%
\newpage
\section{\BibArts\hy Ein\fhy\ko/\ko\ko Ausschalter (bes.\ \ko f"ur \ko\ko Vorspann)}\label{Sect15}

\vspace{1ex}\

\hspace*{7.5mm}
\begin{tabular}{ll}
 \bf Voreinstellung "andern\hspace{1cm}  & \bf $\sim$ wiederherstellen\hspace{1.1em}(1/2) \\ \hline
   \multicolumn{2}{l}{} \\[1ex]
         %%
 \verb|\notannouncektit|          & \verb|   \announcektit|          \\
   \multicolumn{2}{l}{\mynwarrow {\footnotesize\sffamily v\hy Befehl druckt den sp"ater verwendeten Kurztitel nicht aus}} \\[3ex]
         %%
 \verb|\notbafrontcorr|           & \verb|   \bafrontcorr|           \\
   \multicolumn{2}{l}{\mynwarrow {\footnotesize\sffamily {\normalfont\footnotesize\textit{Italics}\hy}Korrektur am Kopf von \BibArts\hy Befehlen unterlassen}} \\[3ex]
         %%
 \verb|   \baonecolitemdefs|      & \verb|\notbaitemdefs|            \\
   \multicolumn{2}{l}{\mynwarrow {\footnotesize\sffamily \texttt{list}\hy Befehle bekommen item\hy Abst"ande wie \texttt{\bs printvli} etc.\ (lokal)}} \\[3ex]
         %%
 \verb|   \batwocolitemdefs|      & \verb|\notbaitemdefs|            \\
   \multicolumn{2}{l}{\mynwarrow {\footnotesize\sffamily \texttt{list}\hy Befehle bekommen item\hy Abst"ande wie \texttt{\bs printvkc} etc.\ (lokal)}} \\[3ex]
         %%
 \verb|   \conferize|             & \verb|\notconferize|             \\    
   \multicolumn{2}{l}{\mynwarrow {\footnotesize\sffamily Verweis vom k\hy Befehl auf Stelle des zugeh"origen v\hy Befehls}} \\[3ex]
         %%
 \verb|   \exponenteditionnumber| & \verb|\notexponenteditionnumber| \\
   \multicolumn{2}{l}{\mynwarrow {\footnotesize\sffamily{\normalfont\footnotesize\texttt{\bs ersch}}\hy Befehl druckt Nummer der {\normalfont\footnotesize\texttt{[}}{\normalfont\footnotesize\textit{Auf"|lage}}{\normalfont\footnotesize\texttt{]}} als Exponent}} \\[3ex]
         %%
 \verb|\nothyko|                  & \verb|   \hyko|             \\
   \multicolumn{2}{l}{\mynwarrow {\footnotesize\sffamily Automatisches 
           {\normalfont\footnotesize\textit{kerning}} nach {\normalfont\footnotesize\texttt{\bs hy}} 
                 und {\normalfont\footnotesize\texttt{\bs fhy}} ausschalten: \hyko\fhy Y\ \nothyko\fhy Y}} \\[3ex]
         %%
 \verb|\notibidemize|             & \verb|   \ibidemize|             \\
   \multicolumn{2}{l}{\mynwarrow {\footnotesize\sffamily Automatisches {\normalfont\footnotesize\textsc{ebd.}}\hy Setzen von Fu"snote zu Fu"snote ausschalten}} \\[3ex]
         %%
 \verb|\notktitaddtok|            & \verb|   \ktitaddtok|            \\
   \multicolumn{2}{l}{\mynwarrow {\footnotesize\sffamily{\normalfont\footnotesize\texttt{\bs ktit}} in v\fhy Befehl erzeugt keinen \kern-.2em{\normalfont\footnotesize\texttt{.vkc}}\hy Eintrag wie ein k\fhy Befehl}} \\[3ex]
         %%
 \verb|\notkurzaddtoarq|          & \verb|   \kurzaddtoarq|            \\
   \multicolumn{2}{l}{\mynwarrow {\footnotesize\sffamily{\normalfont\footnotesize\texttt{\bs kurz}} 
            (\kern-.075em {\normalfont\footnotesize\textit{Vorl"aufer}} von 
                         {\normalfont\footnotesize\texttt{\bs ktit}}) erzeugt keinen 
                         \hspace{-.2em}{\normalfont\footnotesize\texttt{.arq}}\hy Eintrag}} \\[3ex]
\end{tabular}


\newpage
\section*{\hspace{2em}\BibArts\hy Ein\fhy\ko/\ko\ko Ausschalter (bes.\ \ko f"ur \ko\ko Vorspann)}

\vspace{1ex}\

\hspace*{7.5mm}
\begin{tabular}{ll}
 \bf Voreinstellung "andern\hspace{1cm}  & \bf $\sim$ wiederherstellen\hspace{1.1em}\hfill (2/2) \\ \hline
   \multicolumn{2}{l}{} \\[1ex]
         %%
 \verb|\notnegcorrdefabk|         & \verb|   \negcorrdefabk|         \\
   \multicolumn{2}{l}{\mynwarrow {\footnotesize\sffamily Kein negativer Abstand nach Klammer-Auf in Abk"urzungen}} \\[3ex]
         %%
 \verb|\notprinthints|            & \verb|   \printhints|            \\
   \multicolumn{2}{l}{\mynwarrow {\footnotesize\sffamily k\hy Befehle sollen [L]- bzw\ko.\ [Q]\hy Hinweise nicht drucken}} \\[3ex]
         %%
 \verb|\notprintlongpagefolio|    & \verb|   \printlongpagefolio|    \\
   \multicolumn{2}{l}{\mynwarrow {\footnotesize\sffamily 'S.' bei {\normalfont\footnotesize\texttt{[p]}} bzw\ko.\ 'Bl.' bei {\normalfont\footnotesize\texttt{(p)}} nicht drucken}} \\[3ex]
         %%
 \verb|\notprintlongpervol|       & \verb|   \printlongpervol|       \\
   \multicolumn{2}{l}{\mynwarrow {\footnotesize\sffamily 'Bd.' bei {\normalfont\footnotesize\texttt{\string|n\string|}} bzw\ko.\ 'Nr.' bei {\normalfont\footnotesize\texttt{\string_n\string_}} nicht drucken}} \\[3ex]
         %%
 \verb|   \bibsortheads|          & \verb|\notbibsortheads|        \\
   \multicolumn{2}{l}{\mynwarrow {\footnotesize\sffamily Listen: Initialen vor Eintr"agen mit neuem Anfangsbuchstaben}} \\[3ex]
         %%
 \verb|   \bibsortspaces|         & \verb|\notbibsortspaces|       \\
   \multicolumn{2}{l}{\mynwarrow {\footnotesize\sffamily Listen: Abstand zw\ko.\ Eintr"agen mit versch.\ Anfangsbuchstaben}} \\[3ex]
         %%
 \verb|   \showbacorr|            & \verb|\notshowbacorr|            \\
   \multicolumn{2}{l}{\mynwarrow {\footnotesize\sffamily Stelle mit \BibArts{\normalfont\footnotesize\hy\ko\textit{Italics}\hy}Korrektur im Ausdruck markieren}} \\[3ex]
         %%
 \verb|   \showbamem|             & \verb|\notshowbamem|             \\
   \multicolumn{2}{l}{\mynwarrow {\footnotesize\sffamily \BibArts\hy Zwischenspeicher auf Bildschirm drucken (\ko{\normalfont\footnotesize\textsc{ebd.}}\hy Setzung)}} \\[3ex]
         %%
 \verb|\notwarnsamename|          & \verb|   \warnsamename|          \\
   \multicolumn{2}{l}{\mynwarrow {\footnotesize\sffamily Bildschirmwarnung bei Wiederholung von Autornachnamen aus}} \\[3ex]
         %%
 \verb|   \writeidemwarnings|     & \verb|\notwriteidemwarnings|     \\
   \multicolumn{2}{l}{\mynwarrow {\footnotesize\sffamily{\normalfont\footnotesize\textsc{ders.}}\hy Setzung im Ausdruck testhalber mit {\small$\bullet\heartsuit\nabla\spadesuit\clubsuit$} markieren}} \\
         %%
\end{tabular}


\newpage
\section{\BibArts\hy\hspace{-.025em}1.3\hspace{.075em}\hy Texte unter \BibArts~2.1}\label{Sect16}\label{compabil}

\BibArts~2.x hat so viele Neuerungen, dass ein Text in Version~1.3
vor der \LaTeX\hy Bearbeitung "uberarbeitet werden m"usste. An
den Befehlen \verb|\schrift| (f"ur ganze v\fhy Befehle), 
\verb|\barschrift| und \verb|\indschrift| mit 
\verb|\renewcommand| ansetzende "Anderungen sind heute
\textit{wirkungslos}.\footnote{\texttt{\bs frompagesep} 
(oben S.\,\pageref{frompagesep}) ersetzt zudem \texttt{\bs verw}; \kern 1pt und 
\texttt{\bs ntsep} (S.\,\pageref{ntsepA}) \texttt{\bs punctuation}\hspace{.05em}.\kern1pt} 
Lesen Sie \verb|readme.txt|.

\vspace{.05ex}%
\textbf{Behalten Sie zur "Ubersetzung alter \BibArts\hy Texte
die Programmdateien Ihrer \BibArts\hy 1.x\hy Version zur"uck!} 
... Falls Sie dies vers"aumten:

\vspace{.075ex}%
\BibArts~1.3 hatte keine automatische \textsc{ebd.}\hy Setzung. 
Dort konnte \verb|\kurz|\pdfko{1}\ 
ganz am Ende des letzten Arguments 
eines v\fhy Befehls stehen; es druckte sein Argument nach 
\textsf{im folgenden} (und in v\fhy Listen in eckigen Klammern)
einfach\pdfko{.25}\  
aus. \BibArts~2.1 erkennt alte 
\hspace*{-.15em}\texttt{.tex}\hy Dateien und startet eine 
Emulation.\footnote{Wird vom alten Vorspannbefehl 
\texttt{\bs makebar} eingeschaltet (stehen lassen!) und redefiniert auch 
\hspace{-.1em}\texttt{\bs printvli}, das in 1.3 keine "Uberschrift
druckte. Sonst wird \texttt{\bs makebar} nicht\pdfko{1}\
mehr ben"otigt. Es gibt kein \hspace*{-.15em}\texttt{.bar}\hy File mehr: 
\BibArts\ nutzt nun \hspace{-.25em}\texttt{.aux}\hy Files 
(\textbf{dazu Kap.\,\ref{bibsort}}).}
Kopien der Argumente von \verb|\kurz| sowie der alten 
\verb|\bib|\hy Befehle gehen heute\pdfko{1.125}\  
ins \hspace*{-.15em}\texttt{.arq}\hy 
Verzeichnis, das es in \BibArts~1.3 nicht gab; eine 
\verb|\printind|\hy Emulation druckt alles aus $-$ nach Bearbeitung 
mit \texttt{bibsort}. F"ur \textsc{MakeIndex}\pdfko{1.25}\  
gedachte Steuerzeichen werden jetzt also ausgedruckt 
(\kern -.05em vgl.\ unten S.\,\pageref{subitem})!

\vspace{-.625ex}
\Doppelbox
{      \bs notkurzaddtoarq \% (jetzt nicht)
    \\ Fast wie 1.3: \bs vli\{Norbert\} 
    \\ \ \{Schwarz\} \b{\{}Einf"uhrung in 
    \\ \ \ \ \bs protect\bs TeX, Bonn
    \\ \ \ \ 1988 \bs kurz\{Schwarz\}\b{\}}
}
{\notkurzaddtoarq
  \texttt{ \footnotesize \%\% \bs kurz druckt in \bs arqemph \%\%} \\[.2ex]
    Fast wie 1.3: \printonlyvli{Norbert} 
                {Schwarz} {Einf"uhrung in 
                \protect\TeX, Bonn
    1988 \kurz{Schwarz}}
}

\vspace{-.125ex}\noindent
Einige Befehle sind auch in 2.1\hy Texten
brauchbar. \verb|\stressing{underline}| stellt wie in 1.3 die 
Autorenhervorhebung ein und ist heute Alternative f"ur 
\verb|\renewcommand{\authoremph}{\upshape\underline}|. Auch 
das Paar\pdfko{1.125}\  
\verb|\bibmark| und \verb|\bibref| existiert weiter 
(die x\fhy Befehle sind nun unn"otig):

\vspace{-.875ex}
\Doppelbox
{
 Text.\bs footnote\b{\{}Albert Lecl\bs\string`erc: 
        \\[.25ex] \ \ Der Sommerregen, Paris 1985 
        \\[.29ex] \ \ (\bs bibmark\{Lecl\bs\string`erc\}).\b{\}}
        \\[.47ex] Schon Lecl\bs\string`erc wollte 
        \\[.25ex] freie Eingaben.\bs footnote\b{\b{\{}}
        \\[.16ex] \ \ \bs bibref\{\bs scshape Lecl\bs\string`erc\}.\b{\b{\}}}
}
{
 Text.\footnote{Albert Lecl\`erc: Der Sommerregen, Paris 1985 (\bibmark{Lecl\`erc}).}
        Schon Lecl\`erc wollte freie Eingaben.\footnote{
        \bibref{\scshape Lecl\`erc}.}
}

\vspace{-.325ex}\noindent
\verb|\bibref| passt sich an, wenn \verb|\bibmark| in 
keiner Fu"snote war. Neu sind dazu \textit{captions} 
\verb|\gbibmarkname| ('{im Folgenden: }'), 
\verb|\fbibmarkname| ('{par la suite: }')
und \verb|\ebibmarkname| ('{cited as: }'),
deren Definitionen mit Leerzeichen enden.



\newpage
\section{Listenausdruck (\BibArts-Belegapparat)}\label{Sect17}

Wie die von \verb|bibsort| erzeugten Dateien (\kern -.05em vgl.\ unten 
ab S.\,\pageref{bibsort}) auszudrucken
sind, wurde in den jeweiligen Kapiteln bereits fallweise abgehandelt:
\verb|bibarts.sty| stellt dazu print- und printnum\hy Befehle bereit $-$ wobei
die\pdfko{1.875}\ 
print\hy Befehle die Zug"ange als Liste und die printnum\hy Befehle
zus"atzlich hinter jeden Listenpunkt die Zugangsstellen indexartig drucken.
Bei beiden Befehlsklassen enth"alt das Befehlswort zum Ausdruck der
jeweiligen Liste dieselben drei Buchstaben, die auch der Befehl zum
F"ullen der Liste aufweist. Auch das Dateinamen\hy Suffix der von
\verb|bibsort| erzeugten Liste hat diese Zeichen: Eintr"age des 
\BibArts\hy Befehls \verb|\vli| kommen in eine Datei \verb|.vli|, 
die Sie\pdfko{1}\  
mit \verb|\printvli| oder \verb|\printnumvli| im Anhang 
Ihres Textes ausdrucken k"onnen. Entsprechendes gilt f"ur \verb|\vqu|, 
\verb|\arq| und \verb|\per|. Ausnahme ist das Kurzzitateverzeichnis 
\verb|.vkc|, das die Zug"ange der \verb|\kli|- \textit{und} 
\verb|\kqu|\hy Eintr"age erh"alt (sowie der Zug"ange, die \BibArts\ 
aus den Argumenten von \verb|\ktit| und den Nachnamensargumenten 
der v\fhy Befehle \textit{automatisch} erzeugt); das Kurzzitateverzeichnis
wird mit \verb|\printvkc| oder \verb|\printnumvkc| ausgedruckt. Und 
f"ur das\pdfko{.75}\ 
Abk"urzungsverzeichnis, das mit \verb|\printabk| oder 
\verb|\printnumabk| ausgedruckt\pdfko{.5}\ 
wird, bef"ullen die \BibArts\hy Befehle 
\verb|\abkdef| oder \verb|\defabk| eine von \verb|bibsort| erzeugte 
Datei \verb|.abk|; f"ur so eingef"uhrte Abk"urzungen liefern 
\verb|\abk|\hy Befehle weitere Seitenzahlen und ggf.\ Fu"snotennummern, 
die \verb|\printnumabk| ausdruckt. 

F"ur alle diese Listen liest \verb|bibsort| das\slash die \verb|.aux|\hy 
File{\small(}s{\small)} Ihres \LaTeX\hy Textes ein und erzeugt daraus 
die genannten Dateien. Das Namens\hy Pr"afix ist dasjenige des \LaTeX\hy 
Haupttextes (die Literaturliste \textit{hier} ist \verb|bibarts.vli|). 

Das Orts\fhy, Personen und Sachregister wird jeweils nur mit addto\hy 
Befehlen bef"ullt, etwa \verb|\addtogrr|. Das sind Befehle, die nichts 
an Ort und Stelle\pdfko{.75}\ 
drucken. Verwechseln Sie die print- und printnum\hy 
Befehle nicht mit Befehlen wie \verb|\printonlyvli|, die \textit{nur} 
an Ort und Stelle drucken (vgl.\ oben S.\,\pageref{printonly}). 

Die Basis\hy Ausdruckbefehle f"ur die drei Register
sowie das Abk"urzungs\hy\ und das Kurzzitateverzeichnis 
stellen eine fixe Schriftgr"o"se und zweispaltigen 
Seitenausdruck f"ur den Listenausdruck ein. Beides ist bei 
\verb|\printvli|, \verb|\printvqu|, \verb|\printper| und 
\verb|\printarq| samt \verb|num|\hy Varianten nicht der Fall.

Die Listen werden defaultm"a"sig unter "Uberschriften ausgedruckt, 
deren vorgefertigter Text Kapitel~\ref{SprachSep}
auf"|listete. "Anderungen dieser "Uberschriftentexte k"onnen Sie mit 
\verb|\renewcommand| an den titlename\hy Befehlen durchf"uhren. 

Weiter l"asst sich der Ausdruck von "Uberschrift und Liste 
separieren. Die vli\hy "Uberschrift etwa k"onnen Sie mit 
\verb|\printvlititle| drucken. Wie auch bei\pdfko{.75}\ 
\verb|\printvli| oder \verb|\printnumvli| kommt der 
Titel ohne Kapitelnummer ins Inhaltsverzeichnis. Alternativ 
k"onnen Sie etwa \verb|\subsection{|\kern -.1em\textit{"Uberschrift}\verb|}| 
tippen, falls Sie dort Kapitelnummern haben wollen.
Die Liste l"asst sich darunter in beiden F"allen mit \verb|\printvlilist| 
oder \verb|\printnumvlilist| ausdrucken. 


\noindent
\verb|\printbibtitle| ist "Uberschrift f"ur den gesamten Belegapparat, 
defaultm"a"sig in section\hy Gr"o"se. Die anderen title\hy Befehle 
verwenden eine Gr"o"se kleiner:

\vspace{1.25ex}{\small\noindent
\verb|    |\hbox to 14em{\textit{Beide drucken "Uberschrift}\hfill}%
\verb|   |\hbox to 8.6em{\textit{mit Text im dt.}\hfill}%
\verb|   |\hbox to 6em{\textit{Default}\hfill} \\[-.25ex]
\verb|             \printbibtitle  =>  \gbibtitlename  =>  section| \\[-.5ex]
\verb|   \printvli \printvlititle  =>  \gvlititlename  =>  subsection| \\[-.5ex]
\verb|   \printvqu \printvqutitle  =>  \gvqutitlename  =>  subsection| \\[-.5ex]
\verb|   \printabk \printabktitle  =>  \gabktitlename  =>  subsection| \\[-.5ex]
\verb|   \printper \printpertitle  =>  \gpertitlename  =>  subsection| \\[-.5ex]
\verb|   \printarq \printarqtitle  =>  \garqtitlename  =>  subsection| \\[-.5ex]
\verb|   \printvkc \printvkctitle  =>  \gvkctitlename  =>  subsection| \\[-.5ex]
\verb|   \printgrr \printgrrtitle  =>  \ggrrtitlename  =>  subsection| \\[-.5ex]
\verb|   \printprr \printprrtitle  =>  \gprrtitlename  =>  subsection| \\[-.5ex]
\verb|   \printsrr \printsrrtitle  =>  \gsrrtitlename  =>  subsection|}

\vspace{1.25ex}\noindent
Hinter print\fhy, printnum\hy\ und title\hy Befehlen kann ein 
optionales Argument die "Uberschriftengr"o"se "andern, 
\verb|\printvli[section]| etwa. Einzusetzen ist ein
"Uberschriftenbefehl ohne \textit{backslash}. Bei Befehlen, die
Listen zweispaltig drucken (unten), ist \verb|[chapter]| 
verboten. Die "Uberschrift kommt stets ins Inhaltsverzeichnis 
(\verb|\tableofcontents|) und unter \verb|\pagestyle{headings}| 
in die Kopfzeile; Befehle, die zwei Spalten anordnen, setzen 
die Anfangsseite \texttt{plain}. \hspace{.1em}(Die list\hy 
Befehle setzen nichts in Kopfzeile oder Inhaltsverzeichnis.)

\vspace{1ex}\noindent
Die normalen print\hy Befehle drucken "Uberschrift \hspace{-.1em}\textit{und} 
nachformatierte Listen:

\vspace{.75ex}{\small\noindent
\verb|   \printvli  =  \printvlititle + \printvlilist |in \textit{Umfeldschrift} \\[-.5ex]
\verb|   \printvqu  =  \printvqutitle + \printvqulist |in \textit{Umfeldschrift} \\[-.5ex]
\verb|   \printabk  =  \printabktitle + \printabklist |in \verb|\twocolumn| und \\[-.5ex]
\verb|                                                     \footnotesize| \\[-.5ex]
\verb|   \printper  =  \printpertitle + \printperlist |in \textit{Umfeldschrift} \\[-.5ex]
\verb|   \printarq  =  \printarqtitle + \printarqlist |in \textit{Umfeldschrift} \\[-.5ex]
\verb|   \printvkc  =  \printvkctitle + \printvkclist |in \verb|\twocolumn\small| \\[-.5ex]
\verb|   \printgrr  =  \printgrrtitle + \printgrrlist |in \verb|\twocolumn\small| \\[-.5ex]
\verb|   \printprr  =  \printprrtitle + \printprrlist |in \verb|\twocolumn\small| \\[-.5ex]
\verb|   \printsrr  =  \printsrrtitle + \printsrrlist |in \verb|\twocolumn\small|}

\vspace{1.25ex}\noindent
Die printnum\hy Befehle verhalten sich beim Ausdruck ebenso wie die genannten
print\hy Befehle, nur f"uhren sie stattdessen \verb|\printnum|...\verb|list|\hy 
Befehle aus. 

\vspace{1ex}\noindent
Eine "Uberschrift mit Nummerierung \textbf{A} w"are 
(nicht umgesetzt):\footnote{
...\,\texttt{\bs pagestyle\{headings\}}\,\textit{"Uberschrift} 
\texttt{\bs pagestyle\{myheadings\}}\,...\,\texttt{\bs end\{appendix\}} \\
druckt die Kopfzeile von \textit{"Uberschrift} im \textit{ganzen} Appendix
(dann kein \texttt{\bs markboth} setzen!).\hspace*{.2em}}

\vspace{-1.25ex}
{\footnotesize
\begin{verbatim}
   \clearpage \begin{appendix} \pagestyle{headings}
   \section{Belegapparat und Register}\thispagestyle{plain}\vspace{7mm}
   {\small \printarq \newpage \printvqu \printvli \newpage}\printnumgrr
   \end{appendix} %Einspaltige Bereiche enden mit \newpage (Kopfzeile!)
\end{verbatim}}

\vspace{-1ex}\vfill\noindent
\verb|\print|...\verb|list|- und \verb|\printnum|...\verb|list|\hy Befehle 
schalten \textit{nie} \verb|\twocolumn| ein:

\newpage
\noindent{\small
\verb| \clearpage {\pagestyle{headings}\small \printbibtitle \printvqu| \\
\verb| \printvli \printvkctitle\baonecolitemdefs\printnumvkclist \newpage}|}

\vspace*{1.5ex}
                 {\pagestyle{headings}\small \printbibtitle \printvqu
      \printvli \printvkctitle\baonecolitemdefs\printnumvkclist \newpage}



\twocolumn[\subsubsection*{Auf"|listung der print-, printnum-, title-, list- und num...list-Befehle}
Hier \texttt{[}\kern-.05em\textit{OptArg}\texttt{]}'s f"ur "Uberschriften, die eine Stufe gr"o"ser als der Default sind. Die jeweils ersten zwei Befehle sind in den title- und einen list-Befehl teilbar. \\ 
\strut \\
\texttt{\bs printbibtitle\lbrack chapter\rbrack} \hfill
{\footnotesize Dokumentenklasse \texttt{\{report\}}\ }\vspace{1.75ex}]
\noindent
%%
\textbf{\gvlititlename} \\[0.25ex]
\verb|\printvli[section]| \\
\verb|\printnumvli[section]| \\[0.5ex]
\verb|\printvlititle[section]| \\
\verb|\printvlilist| \\
\verb|\printnumvlilist| \\[1.75ex]
%%
\textbf{\gvqutitlename} \\[0.25ex]
\verb|\printvqu[section]| \\
\verb|\printnumvqu[section]| \\[0.5ex]
\verb|\printvqutitle[section]| \\
\verb|\printvqulist| \\
\verb|\printnumvqulist| \\[1.75ex]
%%
\textbf{\gvkctitlename} \\[0.25ex]
\verb|\printvkc[section]|~$^{(t, s)}$ \\
\verb|\printnumvkc[section]|~$^{(t, s)}$ \\[0.5ex]
\verb|\printvkctitle[section]| \\
\verb|\printvkclist| \\
\verb|\printnumvkclist| \\[1.75ex]
%%
\textbf{\gpertitlename} \\[0.25ex]
\verb|\printper[section]| \\
\verb|\printnumper[section]| \\[0.5ex]
\verb|\printpertitle[section]| \\
\verb|\printperlist| \\
\verb|\printnumperlist| \\[1.75ex]
%%
\textbf{\garqtitlename} \\[0.25ex]
\verb|\printarq[section]| \\
\verb|\printnumarq[section]| \\[0.5ex]
\verb|\printarqtitle[section]| \\
\verb|\printarqlist| \\
\verb|\printnumarqlist| \\[1.75ex]
%%
\textbf{\gabktitlename} \\[0.25ex]
\verb|\printabk[section]|~$^{(t, f)}$ \\
\verb|\printnumabk[section]|~$^{(t, f)}$ \\[0.5ex]
\verb|\printabktitle[section]| \\
\verb|\printabklist| \\
\verb|\printnumabklist| \\[1.75ex]
%%
\textbf{\ggrrtitlename} \\[0.25ex]
\verb|\printgrr[section]|~$^{(t, s)}$ \\
\verb|\printnumgrr[section]|~$^{(t, s)}$ \\[0.5ex]
\verb|\printgrrtitle[section]| \\
\verb|\printgrrlist| \\
\verb|\printnumgrrlist| \\[1.75ex]
%%
\textbf{\gprrtitlename} \\[0.25ex]
\verb|\printprr[section]|~$^{(t, s)}$ \\
\verb|\printnumprr[section]|~$^{(t, s)}$ \\[0.5ex]
\verb|\printprrtitle[section]| \\
\verb|\printprrlist| \\
\verb|\printnumprrlist| \\[1.75ex]
%%
\textbf{\gsrrtitlename} \\[0.25ex]
\verb|\printsrr[section]|~$^{(t, s)}$ \\
\verb|\printnumsrr[section]|~$^{(t, s)}$ \\[0.5ex]
\verb|\printsrrtitle[section]| \\
\verb|\printsrrlist| \\
\verb|\printnumsrrlist|\\[1.75ex]
%%
\textbf{Legende} \\[0.25ex]
{\footnotesize
\hbox to 2em{$^{(t, f)}$\hfill} \verb|\twocolumn \footnotesize| \\[-.5ex]
\hbox to 2em{$^{(t, s)}$\hfill} \verb|\twocolumn \small| \\[-.25ex]
Befehle, die \verb|[|\textit{Arg}\verb|]| annehmen, erzeugen \\[-.5ex]
$-$ "Uberschrift in \textit{Default}\fhy/\kern-.1em\textit{Arg}\fhy Gr"o"se \\[-.65ex]
$-$ Inhaltsverzeichnis-Eintrag (\kern-.05em\textit{dito}) \\[-.65ex]
$-$ Kopfzeilen-Eintrag unter \texttt{headings}}
\onecolumn


\newpage\noindent
\verb|bibsort| bereitet f"ur den Ausdruck der Listen vor, 
den Wechsel von Eintr"agen mit unterschiedlichen Anfangsbuchstaben zu betonen.
Es gibt vergr"o"serte Abst"ande und Buchstaben: 
\verb|{\bibsortspaces\printnumvkc}| und\pdfko{1.5}\ 
\verb|{\bibsortheads\printnumvkc}| erg"aben tats"achlich jeweils eigenen Seiten:

\vspace{-1ex}\noindent\hspace{1em}%
\parbox[t]{.45\textwidth}
{\subsection*{\gvkctitlename}\vspace{\batwocoltopskip}\bibsortspaces\batwocolitemdefs\small\printnumvkclist}
\hspace{.5em}%
{\makeatletter\def\@baitemdefs{\parsep 0pt \itemsep 0pt \parskip 0pt \lineskip 0pt \rightskip 1cm minus 1cm}\makeatother
\parbox[t]{.45\textwidth}
{\subsection*{\gvkctitlename}\vspace{\batwocoltopskip}\bibsortheads\small\printnumvkclist}}%
\label{head}%

\vfill\noindent
print\hy Befehle, die \textit{selbst} zweispaltig drucken,
setzen strikte Vorgaben f"ur Abst"ande um, etwa 
\verb|\itemsep| \verb|0pt|. Dagegen gelten \textit{f"ur list\hy Befehle}
nur die Vorgaben der \verb|{description}|\hy Liste.
\verb|\batwocolitemdefs| stellt dazu Abst"ande f"ur zweispaltigen Ausdruck 
ein; es wird wie \verb|\small| vor list\hy Befehle gesetzt.

\vspace{1ex}\noindent
F"ur list\hy Befehle in einspaltigem Umfeld dient 
\verb|\baonecolitemdefs|, das viel weniger Vorgaben
macht und Spielr"aume l"asst. Eigene Definitionen
legt etwa 
\verb|{\bamyitemdefs{\rightskip| \verb|1cm minus| \verb|1cm}\printvkclist}|
fest. Alle f"ur \verb|\print|...\verb|list| oder 
\verb|\printnum|...\verb|list| eventuell gemachten itemdef\hy Vorgaben 
schaltet \verb|\notbaitemdefs| aus (es stellt die
Voreinstellung wieder her).

\vspace{1.5ex}\noindent
\verb|\printvkclist| l"asst sich \textit{mit Zus"atzen} 
genauso wie \verb|\printvkc| ausdrucken
(weil die {\small\texttt{[}...\texttt{]}} abschirmen, w"are 
eine Kopfzeile danach nochmal zu definieren):

\vspace{1.ex}{\small\noindent
\verb|   \twocolumn[\printvkctitle\vspace{\batwocoltopskip}] %\markboth|...\\
\verb|   {\small\bibsortheads \batwocolitemdefs\printvkclist}\onecolumn|}

\vspace{1.25ex}\noindent
\verb|\batwocoltopskip| wird eigenst"andig von Befehlen, die 
zweispaltigen Druck anordnen, gesetzt, und nur, wenn 
\verb|\bibsortspaces| oder \verb|\bibsortheads| gilt. 
Und das \verb|\batwocolitemdefs| w"urden list\hy Befehle unter 
"au"serem \verb|\twocolumn| bei \textit{gleichzeitigem} 
\verb|\bibsortheads| sogar selbst setzen (dann kann 
nur noch z.\,B. \verb|\renewcommand{\baselinestretch}{1.1}| 
die Zeilenabst"ande "andern).

\vspace{1.5ex}\noindent
\verb|\bibsortspaces| und \verb|\bibsortheads| schalten sich 
gegenseitig ab: Automatisch gilt also immer nur eines von beiden. 
Zudem l"ost \verb|\notbibsortheads| auch \verb|\notbibsortspaces| 
aus $-$ und umgekehrt.

\vspace{1.5ex}\noindent
Beim Archivquellenverzeichnis kann ein Konflikt auftreten:
Sie sollten sich entscheiden, ob Sie \verb|\bibsortspaces| 
bzw.\ \hspace{-.05em}\verb|\bibsortheads| aktivieren
m"ochten, \textit{oder} \verb|\arqsection|,
\verb|\arqsubsection| und \verb|\arqsubsubsection| nutzen.
Und nur im Archivquellenverzeichnis k"onnen Sie \textit{innerhalb 
einer Liste} in den Seitenumbruch eingreifen. Umbruchbefehle
wie \verb|\newpage| k"onnen direkt nach \verb|\arqsection|, 
\verb|\arqsubsection| oder \verb|\arqsubsubsection| optional 
"ubergeben werden:\label{newpage} 
\verb|\arqsection[\newpage]{BA}{Bundesarchiv}| \hspace{.1em}ist ein Beispiel.
Zerbrechliche Befehle sollten mit \verb|\protect| gesch"utzt werden.
Befehle mit Argumenten in eckigen Klammern wie \verb|\rule[2ex]{1cm}{1cm}| lassen
sich so einsetzen: 
...\verb|[\protect\rule\lbrack 2ex\rbrack{1cm}{1cm}]|...\footnote{Ist 
eine arq\hy "Uberschrift der allererste Listeneintrag, dann sind \textit{nur}
Abstandsbefehle wie etwa \texttt{\bs vspace} im optionalen Argument 
erlaubt, denn \texttt{bibsort} setzt die optionalen Argumente innerhalb 
des \texttt{.arq}\hy Files in eine Zeile \textit{vor} die arq\hy section; 
in einer \LaTeX\hy Liste darf jedoch zu druckender Text oder 
\texttt{\bs rule} erst \textit{nach} dem ersten \texttt{\bs item}\hy Befehl stehen.} 
Einfach l"asst sich \textit{zus"atzlicher Abstand zum vorausgehenden 
Listenpunkt} etwa mit \verb|\arqsection| \verb|[\vspace{2ex}]| \verb|{B}{Bund}| 
einstellen.\footnote{\texttt{\bs arqsectionbegin}, 
\texttt{\bs arqsubsectionbegin} und \texttt{\bs arqsubsubsectionbegin} 
legen den Basisabstand fest; sie werden von \texttt{[}...\texttt{]} 
nicht "uberschrieben, sondern \textit{danach} ausgef"uhrt. Diese Befehle sind 
untereinander austariert; Anf"anger sollten sie unver"andert lassen.}  

\vspace{1ex}\vfill\noindent
{\sffamily Die list\hy Befehle m"ussen Sie nicht verwenden; sie dienen nur f"ur Sonderw"unsche.}


\newpage
\section{\texttt{bibsort} samt Erweiterungen gegen"uber 2.0}\label{Sect18}\label{bibsort}

\verb|bibsort| ist das Sortierprogramm von \BibArts\kern.1em. Eine
Datei \verb|bibsort.exe| liegt dem Paket neben \verb|bibarts.sty| bei. 
Beide zusammen sollen den Anhang Ihres \LaTeX\hy Textes erzeugen. 
Anwender, bei denen \verb|bibsort.exe| nicht startet, sollten  
\verb|bibsort.c| mit einem f"ur ihr Betriebssystem
geeigneten C\fhy Compiler selbst in eine Bin"ardatei "ubersetzen und 
dann diese einsetzen. Der Quellcode von \verb|bibsort.c| setzt kein 
bestimmtes Betriebssystem voraus. Bei mir machte der DEVCPP\hy Editor 
Schwierigkeiten, der \verb|gcc| pers"onlich nie.\footnote{\texttt{bibsort.c} ist in 
ANSI~C; die Kommandozeilen \texttt{gcc -c bibsort.c -o bibsort.o}
und \texttt{gcc bibsort.o -o bibsort.exe} ergaben bei mir eine 
Bin"ardatei (\texttt{Dev-Cpp\_5.4.0}).}

\BibArts\ kommt heute (Version 2.1) ohne \textsc{MakeIndex} aus.  
\verb|bibsort| verarbeitet keine Steuerzeichen, hat kein 
Maskierungszeichen f"ur Steuerzeichen und kein
Steuerfile.\footnote{\BibArts\,1.3 brauchte \textsc{MakeIndex}, um einen
Belegstellenindex zu erzeugen (siehe S.\,\pageref{compabil}).}
Anders als \textsc{MakeIndex} erzeugt es keine 
\verb|\subitem|\kern.05em s.\label{subitem}% 

Ganz oben wurde bereits erkl"art, wie \verb|bibsort.exe| zusammen mit
speziellen \LaTeX\hy Editoren benutzt wird. Ansonsten kann es durch Antippen
von \verb|bibsort| plus Dateinamens\hy Pr"afix und Optionen in der
Eingabeauf"|forderung des Betriebssystems gestartet werden. Bei 
\LaTeX\hy Texten, die aus mehreren Dateien bestehen, ist das Namenspr"afix der Hauptdatei 
anzutippen. \verb|bibsort| liest die zugeh"orige \hspace{-.1em}\texttt{.aux}\hy Datei ein; 
die ggf.\ enthaltene \verb|\@include|\hy Liste wird abgearbeitet, sodass auch 
bei sequenzieller "Ubersetzung eines \LaTeX\hy Textes (wenn \verb|\includeonly|
nicht alle Dateien nennt) f"ur den \BibArts\hy Anhang immer vollst"andige Listen 
erzeugt werden. F"ur \verb|bibarts.tex| hier ist 
\verb|bibarts| das Namenspr"afix. Die aus
\verb|bibarts.aux| erzeugte Datei \verb|bibarts.vli| enth"alt die
Literaturliste, \verb|bibarts.abk| das Abk"urzungsverzeichnis, etc.

\verb|bibsort| liest aus einer \hspace{-.1em}\texttt{.aux}\hy Datei nur die Zeilen ein, die
mit \BibArts\ erzeugt wurden. In einem ersten Schritt sortiert es 
diese Zeilen klein\fhy\slash gro"s\hy schreibungs\hy invariant. Das Programm sortiert 
\textit{in Grundeinstellung} "a, "o und~"u als a, o und u; weiter sind \verb|\ss| und 
\verb|\3| (sowie \verb|"s|, falls \verb|"| \textit{aktiv} ist\footnote{\BibArts\
"ubergibt jedem Eintrag ins \hspace{-.1em}\texttt{.aux}\hy File den an der
entsprechenden Stelle g"ultigen \textit{catcode} von \string" und die
zur Worttrennung eingestellte Sprache. Beides wird beim Ausdruck
des entsprechenden Listenpunkts reproduziert (vgl.\ oben S.\,\pageref{hyphenation}\,f.).
Dies "andert ggf.\ Zeilenumbr"uche in den Listen und daneben Ausdruck und 
Sortierreihenfolge etwa von~\texttt{\string"a}.}) gleich~\verb|s|. 
Wird \texttt{bibsort} die Option \ko\verb|-g2|~(Wortliste) "ubergeben,
sortiert es "s als \verb|ss|; mit \ko\verb|-g1|~(Namensliste) 
gelten \textit{zudem} die Umlaute als \verb|ae|, \verb|oe| und \verb|ue| 
(letzteres\pdfko{.25}\ entspricht DIN\,5007\fhy 2). Zahlen werden \textit{in Grundeinstellung} 
vor Buchstaben sortiert; mit \ko\verb|-g1| oder \ko\verb|-g2| 
ist es umgekehrt. Nur, wenn Zeilen anhand der enthaltenen Zahlen sowie 
gro"s\fhy\slash klein\hy invarianten Buchstaben keinen Unterschied gegen"uber
anderen Zeilen aufweisen, wird die Gro"s\fhy\slash Klein\hy Schreibung beachtet,
danach etwaige Akzente auf den Buchstaben, zuletzt Satzzeichen.

Etliche Einstellungen sind nicht ver"anderbar. Etwa wird \verb|\o|~(\o) im 
ersten Schritt immer als \verb|o| einsortiert, dann vor allen \verb|o|'s mit 
'aufgesetzten' Akzenten. Ebenso nicht\hy einstellbar ist, dass 
\verb|$\alpha$| als \verb|a| einsortiert wird. Um im Einzelfall abweichende 
Sortierreihenfolgen zu erzwingen, k"onnen Sie den \BibArts\hy Befehl 
\verb|\sort{|\textit{Zeichenfolge}\verb|}| \balabel{sort} verwenden (dessen Argument
sortiert, aber nicht ausgedruckt wird). Die meisten weiteren \LaTeX\hy Befehle 
ignoriert\pdfko{1}\ 
\verb|bibsort| einfach. Bei anderen wie etwa \verb|\parbox| werden 
$-$\,in der Routine\pdfko{1}\ 
\verb|transformtable| in \verb|bibsort.c| definiert\,$-$ 
die L"angen\hy\ und Positionierungs\hy Angaben ignoriert. Au"serdem wird
\verb|\diskretionary{A}{B}{C}| wie \verb|C| sortiert, \verb|"ck| bei
\textit{aktivem} \verb|"| wie \verb|ck|. Und in
\verb|\protect\pageref{XX}| wird das Schl"usselwort
\verb|XX| als nicht zu druckender Text erkannt (vgl.\ \verb|\pageref{X1}|).

\vspace{.75ex}\noindent
Nur \verb|bibsort| in der Eingabeauf"|forderung getippt druckt auf den Bildschirm: 

{\scriptsize
\begin{verbatim}
   %%>  This is bibsort 2.1  (for help:  bibsort -?)
   %%      bibsort 2.1 is part of BibArts 2.1    (C) Timo Baumann  2016.
   %%   I read a LaTeX .aux file (follow \@input), and I write my output in files
   %%     .vli  Literature     .vqu  Published sources     .grr  Geographic index
   %%     .vkc  Short titles   .arq  Unpublished sources   .prr  Person index
   %%     .per  Periodicals    .abk  Abbreviations         .srr  Subject index
   %%
   %%  bibsort <LaTeXFile> [-o <OutFile>] [-g1/-g2] [-x] [-l] [-p] [-k] [-d]
   %%
   %%         DefaultSort:  0, .., 9, (A a), (B b), .., (s \ss S), .., (Z z)  and
   %%         (a [\.\'\`\^\"\=\~]a \aa=\r{a} [\b\c\k\d\H\t\u\v]{a} \ae),  b, ..
   %%   -g1/2 GermanSort:   (a A), .., (z Z), 0, .., 9;   and (\" or active "):
   %%         -g1  GermanTelefonebookStyle:  "a = \"a = ae, ..., "s = \ss = ss;
   %%         -g2  ModernGermanDictionary:   "a = \"a = a,  ..., "s = \ss = ss.
   %%     -x  DoNotExpectgerman.sty: Active "-characters do NOT produce umlauts.
   %%     -l  Ignore spaces (leer).       -p  Sort "P.S." before "Peter" (point).
   %%     -k  Idemize multiple used authors in the .vli and .vqu lists (kill).
   %%     -d  Change your '/' with '\' in paths on my .aux files task list (dos).
   %%    -??  More options:  -s1 -f1  -s2 xxxx  -f2 xxxx  -c -t1 -i=j -m
   %%     -r  Read informations about my license and documentation files.
   %%
   %%>  I give up my job, because I get no <FileNamePrefix> of an auxiliary file.
\end{verbatim}}

\noindent
Dabei ist die ganze Serie der neun Hilfsdateien aufgelistet, die \verb|bibsort| 
erzeugen kann $-$ \textit{und l"oschen darf}, falls keine Eintr"age da 
sind!\footnote{Wenn Sie z.\,B. 
\ko\texttt{\bs vli} in Ihrem Text verwenden, erzeugt \texttt{bibsort} eine Datei 
\ko\texttt{.vli}; falls\pdfko{1}\ 
Sie \texttt{\bs vli} sp"ater wieder l"oschen und Ihr mit 
\LaTeX\ "ubersetzter Text derartige Literaturangaben dann nicht mehr enth"alt, 
l"oscht ein weiterer Start von \texttt{bibsort} die \ko\texttt{.vli}\hy Datei.
$-$\pdfko{1}\  
Unter \texttt{\bs nofiles} bleiben die \ko\texttt{.aux}\hy Files unver"andert
stehen und \texttt{bibsort} erzeugt die Datei\pdfko{1}\  
\texttt{.vli} immer wieder gleich, d.h.: ohne neue/""ver"anderte 
\texttt{\bs vli}\hy Befehle im Text aufzunehmen.\pdfko{.325}}

\vspace{.075ex}
\verb|-d| dient dazu, dass \verb|bibsort| Dateien findet, wenn in 
\verb|\include|\hy Argumenten Pfadangaben stehen; dort m"ussen Sie \verb|/| 
verwenden (was aber \textit{einige\pdfko{1}\ 
Betriebssysteme} nicht akzeptieren): 
\hspace{.1em} \verb|bibsort -d| \hspace{.1em} ruft Dateien mit \verb|\| auf.

\vspace{.175ex}
Die Option \ko\verb|-k| zum Drucken von $\sim$ auf der Literaturliste 
f"ur mehrfach genannte Autoren wurde schon erkl"art. Vgl.\ 
\verb|\female| und \verb|\male| oben S.\,\pageref{female}.

\vspace{.75ex}\noindent
Nun ein Beispiel f"ur die Sortierreihenfolge (\verb|\printprr| ohne Seitenverweise): 


 \newcommand{\demotext}[1]{\nosort{\protect\printdemotext{#1}}}%
{\newcommand{\printdemotext}[1]{{\hskip 0.1cm plus 0.5cm\footnotesize\texttt{#1}}}%
 \renewcommand{\gprrtitlename}{Das zweckentfremdete Personenregister als Beispiel}%
 \printprr}

\addtoprr{\"y\demotext{ \bs\protect\string"y}}
\addtoprr{ya}
\addtoprr{DiFabio}
\addtoprr{Di Niro}
\addtoprr{ba}
\addtoprr{ a}
\addtoprr{5.000}
\addtoprr{0,5}
\addtoprr{0.5}
\addtoprr{0.a6}
\addtoprr{0.26}
\addtoprr{0.25}
\addtoprr{0,251}
\addtoprr{0.251}
\addtoprr{1.500}
\addtoprr{1,500}
\addtoprr{1501}
\addtoprr{1,5}
\addtoprr{1,1}
\addtoprr{1,125}
\addtoprr{1,45}
\addtoprr{1.750}
\addtoprr{1,75}
\addtoprr{$\frac{2}{2}$\demotext{ \$\bs frac\{2\}\{2\}\$}}
\addtoprr{$\frac{2} {2}$\demotext{ \$\bs frac\{2\} \{2\}\$}}
\addtoprr{$\frac{10}{2}$}
\addtoprr{$\frac{13}{1}$}
\addtoprr{$\frac{3}{4}$}
\addtoprr{$\frac{5}{2}$}
\addtoprr{$\frac{4}{3}$}
%\addtoprr{$\frac{3}{3}$}
\addtoprr{$\frac{1}{3}$}
\addtoprr{\sort{9,5}$\frac{19}{2}$\demotext{ \bs sort\{9,5\}\$\bs frac\{19\}\{2\}\$}}
\addtoprr{$\frac{b}{a}$\demotext{ \$\bs frac\{b\}\{a\}\$}}
\addtoprr{1}
\addtoprr{14\te Auf"|l.\demotext{ 14\bs te Auf\string"\string|l.}}
\addtoprr{15}
\addtoprr{22}
\addtoprr{2}
\addtoprr{\protect\underline{2}\demotext{ \bs protect\bs underline\{2\}}}
\addtoprr{b \sort{b}c\demotext{ \ \ \ b \bs sort\{b\}c}}
\addtoprr{b\sort{b} c\demotext{ b\bs sort\{b\} c}}
\addtoprr{b\sort{b}b\demotext{ b\bs sort\{b\}b}}
\addtoprr{b\sort{b}a\demotext{ b\bs sort\{b\}a}}
\addtoprr{b\sort{b}\demotext{ \ b\bs sort\{b\}}}
\addtoprr{*b\demotext{ \protect\string*b}}
\addtoprr{bb}
\addtoprr{bc}
\addtoprr{$b$\demotext{ \ \$b\$}}
\addtoprr{b$^{2}$\demotext{ b\$\protect\string^\{2\}\$}}
\addtoprr{b$_{2}$\demotext{ b\$\protect\string_\{2\}\$}}
\addtoprr{b\fup{2}\demotext{ \ b\bs fup\{2\}}}
\addtoprr{b\fup{b}\demotext{ \ \ b\bs fup\{b\}}}
\addtoprr{b1}
\addtoprr{b2}
\addtoprr{b 2}
\addtoprr{b3}
\addtoprr{b20}
\addtoprr{b98}
\addtoprr{b100}
\addtoprr{b\fup{\baRomannum{99}}\demotext{ b\bs fup\{\bs baRomannum\{99\}\}}}
\addtoprr{{c}\demotext{ \{c\}}}
\addtoprr{{}c\demotext{ \{\}c}}
\addtoprr{c{}\demotext{ c\{\}}}
\addtoprr{"a\demotext{ \protect\string"a}}
\addtoprr{\"a\demotext{ \bs\protect\string"a}}
\addtoprr{"A\demotext{ \protect\string"A}}
\addtoprr{\"A\demotext{ \bs\protect\string"A}}
\addtoprr{ae}
\addtoprr{\c{\"a}\demotext{ \bs c\{\bs\protect\string"a\}}}
\addtoprr{af}
\addtoprr{sr}
\addtoprr{st}
\addtoprr{ss}
\addtoprr{\3\demotext{ \ \bs 3}}
\addtoprr{\ss\demotext{ \ \bs ss}}
\addtoprr{"s\demotext{ \ \protect\string"s}}
\addtoprr{\sz\demotext{ \bs sz \ \%neu in 2.1}}
\addtoprr{{\scshape \ss}\demotext{ \{\bs scshape \bs ss\}}}
\addtoprr{"z\demotext{ \ \protect\string"z}}
\addtoprr{Stra"se\demotext{ \ \ Stra\protect\string"se}}
\addtoprr{Stra{\ss}e\demotext{ \ \ Stra\{\bs ss\}e}}
\addtoprr{Stra"sburg\demotext{ Stra\protect\string"sburg}}
\addtoprr{Stra{\ss}burg\demotext{ Stra\{\bs ss\}burg}}

 \addtoprr{Zum Schluss die Worttrennung american im \label{Trennbeispiel} deutschen Trennsatz}
{\sethyphenation{english}
 \addtoprr{Zum Schluss die Worttrennung american im englischen Trennsatz}}

\addtoprr{A}
\addtoprr{B}
\addtoprr{C}
\addtoprr{D}
\addtoprr{E}
\addtoprr{F}
\addtoprr{G}
\addtoprr{H}
\addtoprr{I}
\addtoprr{J}
\addtoprr{K}
\addtoprr{L}
\addtoprr{M}
\addtoprr{N}
\addtoprr{O}
\addtoprr{P}
\addtoprr{Q}
\addtoprr{R}
\addtoprr{S}
\addtoprr{T}

\addtoprr{W}
\addtoprr{X}
\addtoprr{Y}
\addtoprr{Z}
\addtoprr{Ba}
\addtoprr{Ca}
\addtoprr{Ci}
\addtoprr{Da}
%\addtoprr{Ea}
%\addtoprr{Fa}
\addtoprr{Ga}
%\addtoprr{Ha}
\addtoprr{Ia}
\addtoprr{Ja}
\addtoprr{Ka}
\addtoprr{La}
\addtoprr{Ma}
\addtoprr{Na}
\addtoprr{Oa}
\addtoprr{"o\demotext{ \protect\string"o}}
\addtoprr{Ra}
\addtoprr{Sa}
\addtoprr{Ta}
\addtoprr{Ti}

\addtoprr{Xa}
\addtoprr{Ya}
\addtoprr{Za}

\addtoprr{a}
\addtoprr{b}
\addtoprr{c}
\addtoprr{d}
\addtoprr{e}
\addtoprr{f}
\addtoprr{g}
\addtoprr{h}
\addtoprr{i}
\addtoprr{j}
\addtoprr{k}
\addtoprr{l}
\addtoprr{m}
\addtoprr{n}
\addtoprr{o}
\addtoprr{p}
\addtoprr{q}
\addtoprr{r}
\addtoprr{s}
\addtoprr{t}
\addtoprr{u}
\addtoprr{v}
\addtoprr{w}
\addtoprr{x}
\addtoprr{y}
\addtoprr{z}
\addtoprr{aa}
\addtoprr{ba}
\addtoprr{ca}
\addtoprr{ci}
\addtoprr{da}
%\addtoprr{ea}
%\addtoprr{fa}
\addtoprr{ga}
%\addtoprr{ha}
\addtoprr{ia}
\addtoprr{ja}
\addtoprr{ka}
\addtoprr{la}
\addtoprr{na}
\addtoprr{oa}
\addtoprr{pa}
\addtoprr{ra}
\addtoprr{sa}
\addtoprr{ta}
\addtoprr{ti}

\addtoprr{xa}
\addtoprr{ya}
\addtoprr{za}

\addtoprr{ph}
\addtoprr{pi}
\addtoprr{ps}
\addtoprr{pt}
\addtoprr{Ph}
\addtoprr{Pi}
\addtoprr{Ps}
\addtoprr{Pt}

\addtoprr{b a}
\addtoprr{B a}

\addtoprr{0000000000}
\addtoprr{100000000}
\addtoprr{2~100~000,7\demotext{ 2\protect\string~100\protect\string~000,7}}
\addtoprr{2\,100\,000,65\demotext{ 2\bs,100\bs,000,65}}
\addtoprr{2.099.999}
\addtoprr{2.100.001}
\addtoprr{\sort{2.100.000}2,1\,Mio.\demotext{ \ \bs sort\{2.100.000\}2,1\bs,Mio.}}
\addtoprr{\baromannum{14}\te Bd.\demotext{ \bs baromannum\{14\}\bs te Bd.}}
\addtoprr{\baRomannum{15}\te Bd.\demotext{ \bs baRomannum\{15\}\bs te Bd.}}
\addtoprr{400000}
\addtoprr{6000}
\addtoprr{700}
\addtoprr{9}
\addtoprr{.}
\addtoprr{,}
\addtoprr{;}
\addtoprr{:}
\addtoprr{?}
\addtoprr{!}
\addtoprr{?`\demotext{ \protect\string?{}\protect\string`}}
\addtoprr{!`\demotext{ \protect\string!{}\protect\string`}}
\addtoprr{*\demotext{ *}}
\addtoprr{\$\demotext{ \bs\protect\$}}
\addtoprr{\#\demotext{ \bs\protect\#}}
\addtoprr{"`a"'\demotext{ \protect\string"\protect\string`a\protect\string"\protect\string'}}
\addtoprr{"<a">\demotext{ \protect\string"\protect\string<a\protect\string"\protect\string>}}
\addtoprr{\nosort{'}a'\demotext{ \bs nosort\{{\protect\string'}\}a'}}
\addtoprr{'a'\demotext{ {\protect\string'}a{\protect\string'}}}
\addtoprr{a'a\demotext{ a\protect\string'a \ \ \%\% vgl. \bs nosort\{\protect\string'\}a}}
{\originalTeX \addtoprr{"a\demotext{ \{\bs originalTeX \bs addtoprr\{\protect\string"a\}\}\protect\label{originaltex}}}}
\addtoprr{\glqq a\grqq\demotext{ \bs glqq a\bs grqq}}
\addtoprr{b\$2\demotext{ b\bs\$2}}
\addtoprr{b\&2\demotext{ b\bs\&2}}
\addtoprr{b/2}
\addtoprr{$b\;2$\demotext{ \$b\bs;2\$}}
\addtoprr{$b\:2$\demotext{ \$b\bs:2\$}}
\addtoprr{(b)}
\addtoprr{[b]}
\addtoprr{\{b\demotext{ \bs\{b}}
\addtoprr{\%\demotext{ \bs\%}}
\addtoprr{b\_2\demotext{ b\bs\protect\string_2}}
\addtoprr{b=2}
\addtoprr{b<2}
\addtoprr{b>2}
\addtoprr{b@2}
\addtoprr{b|2\demotext{ b\protect\string|2}}
\addtoprr{b+2}
\addtoprr{b-2}
\addtoprr{b--2\demotext{ \ b{\protect\string-}{\protect\string-}2}}
\addtoprr{b---2\demotext{ b{\protect\string-}{\protect\string-}{\protect\string-}2}}
\addtoprr{Ae}
\addtoprr{Ac}
\addtoprr{Af}
\addtoprr{\.{b}\demotext{ \bs.\{b\}}}
\addtoprr{$\dot{b}$\demotext{ \$\bs dot\{b\}\$}}
\addtoprr{\'{b}\demotext{ \bs\protect\string'\{b\}}}
\addtoprr{$\acute{b}$\demotext{ \$\bs acute\{b\}\$}}
\addtoprr{\`{b}\demotext{ \bs\protect\string`\{b\}}}
\addtoprr{$\grave{b}$\demotext{ \$\bs grave\{b\}\$}}
\addtoprr{\^{b}\demotext{ \bs\protect\string^\{b\}}}
\addtoprr{$\hat{b}$\demotext{ \$\bs hat\{b\}\$}}
\addtoprr{$\ddot{b}$\demotext{ \$\bs ddot\{b\}\$}}
\addtoprr{\={b}\demotext{ \bs=\{b\}}}
\addtoprr{$\bar{b}$\demotext{ \$\bs bar\{b\}\$}}
\addtoprr{$\vec{b}$\demotext{ \$\bs vec\{b\}\$}}
\addtoprr{\~{b}\demotext{ \bs\protect\string~\{b\}}}
\addtoprr{$\tilde{b}$\demotext{ \$\bs tilde\{b\}\$}}
\addtoprr{\r{a}\demotext{ \bs r\{a\}}}
\addtoprr{\r{A}\demotext{ \bs r\{A\}}}
\addtoprr{\aa\demotext{ \bs aa}}
\addtoprr{\AA\demotext{ \bs AA}}
\addtoprr{\r{b}\demotext{ \bs r\{b\}}}
\addtoprr{$\mathring{b}$\demotext{ \$\bs mathring\{b\}\$ \ \%neu in 2.1}}
\addtoprr{\b{b}\demotext{ \bs b\{b\}}}
\addtoprr{\c{b}\demotext{ \bs c\{b\}}}
\addtoprr{\k{b}\demotext{ \bs k\{b\}}}
\addtoprr{\d{b}\demotext{ \bs d\{b\}}}
\addtoprr{\H{b}\demotext{ \bs H\{b\}}}
\addtoprr{\t{b}\demotext{ \bs t\{b\}}}
\addtoprr{\u{b}\demotext{ \bs u\{b\}}}
\addtoprr{$\breve{b}$\demotext{ \$\bs breve\{b\}\$}}
\addtoprr{\v{b}\demotext{ \bs v\{b\}}}
\addtoprr{$\check{b}$\demotext{ \$\bs check\{b\}\$}}
\addtoprr{\protect\underline{b}\demotext{ \bs protect\bs underline\{b\}}}
\addtoprr{\protect\underbar{b}\demotext{ \bs protect\bs underbar\{b\}}}
\addtoprr{ab}
\addtoprr{a b}
\addtoprr{a\,b\demotext{ a\bs,b}}
\addtoprr{a\protect\space b\demotext{ a\bs protect\bs space b}}
\addtoprr{a\relax b\demotext{ a\bs relax b}}
\addtoprr{a!b\demotext{ a\string!b}}
\addtoprr{aa}
\addtoprr{ab}
\addtoprr{a"-b\demotext{ a\protect\string"{\protect\string-}b}}
\addtoprr{a\-b\demotext{ a\bs{\protect\string-}b}}
\addtoprr{ac}
\addtoprr{\i\demotext{ \bs i}}
\addtoprr{\'\i\demotext{ \bs\protect\string'\bs i}}
\addtoprr{\^\i\demotext{ \bs\protect\string^\bs i}}
\addtoprr{\`\i\demotext{ \bs\protect\string`\bs i}}
\addtoprr{$\imath$\demotext{ \$\bs imath\$}}
\addtoprr{\j\demotext{ \bs j}}
\addtoprr{$\jmath$\demotext{ \$\bs jmath\$}}
\addtoprr{\ae\demotext{ \bs ae}}
\addtoprr{\AE\demotext{ \bs AE}}
\addtoprr{\copyright\demotext{ \bs copyright}}
\addtoprr{\dh\demotext{ \bs dh}}
\addtoprr{\dj\demotext{ \bs dj}}
\addtoprr{\DH\demotext{ \bs DH}}
\addtoprr{\DJ\demotext{ \bs DJ}}
\addtoprr{\l\demotext{ \bs l}}
\addtoprr{$\ell$\demotext{ \$\bs ell\$}}
\addtoprr{\ij\demotext{ \bs ij wird niederl.\ als i sortiert}}
\addtoprr{\IJ\demotext{ \bs IJ wird prim"ar als I sortiert}}
\addtoprr{$\Im$\demotext{ \$\bs Im\$}}
\addtoprr{\L\demotext{ \bs L}}
\addtoprr{\LaTeX\demotext{ \bs LaTeX}}
\addtoprr{LATEX}
\addtoprr{LaTeX}
\addtoprr{\ng\demotext{ \bs ng}}
\addtoprr{\NG\demotext{ \bs NG}}
\addtoprr{\o\demotext{ \bs o}}
\addtoprr{\O\demotext{ \bs O}}
\addtoprr{\oe\demotext{ \bs oe}}
\addtoprr{\OE\demotext{ \bs OE}}
\addtoprr{\textsterling\demotext{ \bs textsterling}}
\addtoprr{\pounds\demotext{ \bs pounds}}
\addtoprr{\textregistered\demotext{ \bs textregistered}}
\addtoprr{$\Re$\demotext{ \$\bs Re\$}}
\addtoprr{\th\demotext{ \bs th}}
\addtoprr{\TH\demotext{ \bs TH}}
\addtoprr{\protect\TeX\demotext{ \bs protect\bs TeX}}
\addtoprr{TEX\demotext{ TEX}}
\addtoprr{TeX\demotext{ TeX}}
\addtoprr{$\alpha$\demotext{ \$\bs alpha\$}}
\addtoprr{$\beta$\demotext{ \$\bs beta\$}}
\addtoprr{$\chi$\demotext{ \$\bs chi\$}}
\addtoprr{$\Chi$\demotext{ \$\bs Chi\$ \ \%neu in 2.1}}
\addtoprr{$\delta$\demotext{ \$\bs delta\$}}
\addtoprr{$\Delta$\demotext{ \$\bs Delta\$}}
\addtoprr{$\epsilon$\demotext{ \$\bs epsilon\$}}
\addtoprr{$\varepsilon$\demotext{ \$\bs varepsilon\$}}
\addtoprr{$\eta$\demotext{ \$\bs eta\$}}
\addtoprr{$\gamma$\demotext{ \$\bs gamma\$}}
\addtoprr{$\Gamma$\demotext{ \$\bs Gamma\$}}
\addtoprr{$\iota$\demotext{ \$\bs iota\$}}
\addtoprr{$\kappa$\demotext{ \$\bs kappa\$}}
\addtoprr{$\lambda$\demotext{ \$\bs lambda\$}}
\addtoprr{$\Lambda$\demotext{ \$\bs Lambda\$}}
\addtoprr{$\mu$\demotext{ \$\bs mu\$}}
\addtoprr{$\nu$\demotext{ \$\bs nu\$}}
\addtoprr{\d{$\tilde\nu$}\demotext{ \bs d\{\$\bs tilde\bs nu\$\}}}
\addtoprr{$\tilde\nu$\demotext{ \$\bs tilde\bs nu\$}}
\addtoprr{$\breve\nu$\demotext{ \$\bs breve\bs nu\$}}
\addtoprr{$\omicron$\demotext{ \$\bs omicron\$ (bibarts.sty)}}   %% existiert in LaTeX nicht
\addtoprr{$\omega$\demotext{ \$\bs omega\$}}
\addtoprr{$\Omega$\demotext{ \$\bs Omega\$}}
\addtoprr{$\pi$\demotext{ \$\bs pi\$}}
\addtoprr{$\Pi$\demotext{ \$\bs Pi\$}}
\addtoprr{$\varpi$\demotext{ \$\bs varpi\$}}
\addtoprr{$\phi$\demotext{ \$\bs phi\$}}
\addtoprr{$\Phi$\demotext{ \$\bs Phi\$}}
\addtoprr{$\varphi$\demotext{ \$\bs varphi\$}}
\addtoprr{$\psi$\demotext{ \$\bs psi\$}}
\addtoprr{$\Psi$\demotext{ \$\bs Psi\$}}
\addtoprr{$\rho$\demotext{ \$\bs rho\$}}
\addtoprr{$\varrho$\demotext{ \$\bs varrho\$}}
\addtoprr{$\sigma$\demotext{ \$\bs sigma\$}}
\addtoprr{$\Sigma$\demotext{ \$\bs Sigma\$}}
\addtoprr{$\varsigma$\demotext{ \$\bs varsigma\$}}
\addtoprr{$\tau$\demotext{ \$\bs tau\$}}
\addtoprr{$\theta$\demotext{ \ \$\bs theta\$}}
\addtoprr{$\Theta$\demotext{ \$\bs Theta\$}}
\addtoprr{$\vartheta$\demotext{ \$\bs vartheta\$}}
\addtoprr{$\xi$\demotext{ \$\bs xi\$}}
\addtoprr{$\Xi$\demotext{ \$\bs Xi\$}}
\addtoprr{$\upsilon$\demotext{ \$\bs upsilon\$}}
\addtoprr{$\Upsilon$\demotext{ \$\bs Upsilon\$}}
\addtoprr{$\zeta$\demotext{ \$\bs zeta\$}}
\addtoprr{$\Rho \acute\omicron \delta \omicron \varsigma$ 
   \demotext{ \$\bs Rho \bs acute\bs omicron \bs delta \bs omicron \bs varsigma\$}}
\addtoprr{$\Eta \rho \alpha$ (Hera) 
   \demotext{ \$\bs Eta \bs rho \bs alpha\$}}

\addtoprr{b\\[-2mm]b\demotext{ b\bs\bs[-2mm]b}}
\addtoprr{b\vspace{2mm}b\demotext{ b\bs vspace\{2mm\}b}}
\addtoprr{b\hspace{2mm}b\demotext{ \ b\bs hspace\{2mm\}b}}
\addtoprr{b\index{X}b\demotext{ \ \ b\bs index\{X\}b}}
\addtoprr{b \index{X}b\demotext{ \ \ b \bs index\{X\}b}}
\addtoprr{b \index{X} b\demotext{ \ \ b \bs index\{X\} b}}
\addtoprr{b\glossary{X}b\demotext{ \ b\bs glossary\{X\}b}}
\addtoprr{b \label{X1} b\demotext{ \ b \bs label\{X1\} b}}
\addtoprr{b \pageref{X1} b\demotext{ b \bs pageref\{X1\} b}}
\addtoprr{b \baref{X2} b\demotext{ b \bs baref\{X2\} b}}
\addtoprr{b \balabel{X2} b\demotext{ \ b \bs balabel\{X2\} b}}
\addtoprr{b \protect\pageref{X1} b\demotext{ b \bs protect\bs pageref\{X1\} b}}
\addtoprr{b\protect\linebreak[1]b\demotext{ \ \ \ b\bs protect\bs linebreak[1]b}}
\addtoprr{b\protect\nolinebreak[1]b\demotext{ b\bs protect\bs nolinebreak[1]b}}
\addtoprr{b\protect\pagebreak[2]b\demotext{ b\bs protect\bs pagebreak[2]b}}
\addtoprr{b\protect\nopagebreak[2]b\demotext{ b\bs protect\bs nopagebreak[2]b}}
\addtoprr{b\protect\footnote[144]{X} b\demotext{ b\bs protect\bs footnote[144]\{X\} b}}
\addtoprr{b\protect\footnotetext[145]{Y} b\demotext{ b\bs protect\bs footnotetext[145]\{Y\} b}}
\addtoprr{b\protect\footnotemark[146] b\demotext{ b\bs protect\bs footnotemark[146] b}}
\addtoprr{b\protect\framebox [1cm][l]{b}\demotext{ b\bs protect\bs framebox [1cm][l]\{b\}}}
\addtoprr{b\protect\makebox [1cm][r]{b}\demotext{ b\bs protect\bs makebox [1cm][r]\{b\}}}
\addtoprr{b\protect\parbox[t]{5mm}{b}\demotext{ b\bs protect\bs parbox[t]\{5mm\}\{b\}}}
\addtoprr{b\protect\raisebox{0.5ex} [3mm] [3mm] {b}\demotext{ b\bs protect\bs raisebox\{0.5ex\} [3mm] [3mm] \{b\}}}
\addtoprr{b\protect\rule[1mm] {1mm} {2mm}b\demotext{ b\bs protect\bs rule[1mm] \{1mm\} \{2mm\}b}}
%
\addtoprr{b Nach Titel\demotext{ \ \ b Nach Titel}}
\addtoprr{b \vli {Vor} {Nach} {Titel}\demotext{ \ \ b \bs vli \{Vor\} \{{\bf Nach}\} \{Titel\}}}
\addtoprr{b \vli {Vor} {Nach}{Titel}\demotext{ \ b \bs vli \{Vor\} \{{\bf Nach}\}\{Titel\}}}
\addtoprr{b \kli {Nach} {Titel}\demotext{ \ b \bs kli \{{\bf Nach}\} \{Titel\}}}
\addtoprr{b \kli [f]{Nach} {Titel}\demotext{ \ b \bs kli [f]\{{\bf Nach}\} \{Titel\}}}
\addtoprr{b \vli[m] {Vor} {Nach} {Titel}\demotext{ \ \ \ b \bs vli[m] \{Vor\} \{{\bf Nach}\} \{Titel\}}}
\addtoprr{b Nach\demotext{ \ b Nach}}
\addtoprr{b Nach Vor\demotext{ \ \ \ \ b Nach Vor}}
\addtoprr{b Nach Vor U\demotext{ \ \ \ \ b Nach Vor U}}
\addtoprr{b Nach Titel-a\demotext{ \ \ \ b Nach Titel-a}}
\addtoprr{b \vauthor{Vor}{Nach}\demotext{ b \bs vauthor\{Vor\}\{{\bf Nach}\}}}
\addtoprr{b \kauthor{Nach}\demotext{ \ \ b \bs kauthor\{{\bf Nach}\}}}
%
\addtoprr{\protect\begin{large}b\protect \end{large}b\demotext{ \bs protect\bs begin\{large\}b\bs protect \bs end\{large\}b}}
\addtoprr{b\protect\typeout{9}b\demotext{ b\bs protect\bs typeout\{9\}b}}
\addtoprr{b\message{9}b\demotext{ b\bs message\{9\}b}}
\addtoprr{b\mathhexbox{1}{2}{3}b\demotext{ b\bs mathhexbox\{1\}\{2\}\{3\}b}}
\addtoprr{b\protect\phantom{X}b\demotext{ \ b\bs protect\bs phantom\{X\}b}}
\addtoprr{b\protect\vphantom{X}b\demotext{ \ b\bs protect\bs vphantom\{X\}b}}
\addtoprr{b\protect\hphantom{X}b\demotext{ \ b\bs protect\bs hphantom\{X\}b}}
\addtoprr{b\sethyphenation{french}b\demotext{ b\bs sethyphenation\{french\}b}}
\addtoprr{b\selectlanguage{french}b\demotext{ b\bs selectlanguage\{french\}b}}
\addtoprr{b\discretionary{a-}{c} {b}\demotext{ b\bs discretionary\{a-\}\{c\} \{b\}}}

\noindent
Zur Erzeugung der vorausgehenden Liste bekam \verb|bibsort| als
\textit{Sortier}\hy Optio"-n(en) \texttt{\bibsortargs} "ubergeben 
(\verb|\bibsortargs| steht hier vor '"ubergeben', um die 
zuletzt verwendeten Optionen auszudrucken\footnote{Neu:
\texttt{\bs bibsortargs} listet ein gesetztes \texttt{-k} 
immer auf, obwohl es nur v\fhy Listen betrifft. Die Optionen 
\texttt{-d} und \texttt{-m} werden nie ausgedruckt; 
sie beeinflussen das Druckergebnis nicht.}). 

Die vorausgehenden Seiten zeigten auch viele \LaTeX\hy Befehle,
die \verb|bibsort| verarbeitet. In \verb|bibsort.c| k"onnen Sie in
\verb|weighttable| sehen, welche Befehle einen Sortierwert erhalten. 
\LaTeX\hy Befehlsnamen \textit{aus Buchstaben} m"ussen in dieser Liste mit \verb|\t| 
enden, \textit{aus einzelnen Zeichen} bestehende Befehle wie \verb|\"| d"urfen das nicht. Au"ser den 
eben vorgef"uhrten Befehlen wird noch \verb|\ul| wie eine Unterstreichung 
sortiert; auch zwei in \LaTeX\,2.09 auftretende Expansionsstufen 
von \copyright\ und \pounds\ (\verb|\pcopyright| und \verb|\ppounds|) 
werden ber"ucksichtigt. 

'Unbekannte' Befehle sortiert \verb|bibsort| "ahnlich wie Satzzeichen,
gewichtet sie also nur, wenn Zeilen \textit{sonst} nur 
\textit{gleiche Buchstaben und Zahlen} enthalten.\pdfko{1}

Fu"snotenexponenten (\verb|\footnotemark[146]|) werden absichtlich nicht gewichtet, 
mathematischen Exponenten(\verb|$^{2}$|) schon. 
\verb|bibsort| macht auch Unterschiede zwischen der Ausdruckreihenfolge
und der Sortierreihenfolge:
Die Argumente von \verb|\vauthor| und \verb|\midvauthor| sowie  \verb|\ntvauthor| 
sortiert \verb|bibsort| zuerst nach Nachnamen, und nur bei gleichen Nachnamen 
\textit{anschlie"send} nach Vornamen. 
Und Argumente nach etlichen Befehlen werden 'ausgerichtet' sortiert;
insbesondere in 'inneren' v- und k\hy Befehlen ist es deshalb egal, ob Sie
\verb|{A}{B}| oder \verb*|{A} {B}| tippen: sortiert wird in beiden F"allen
\verb*|A B|. 


\vspace{1ex}\noindent
Die Spracheinstellung bestimmt, worin \verb|bibsort| eine 
\textbf{Dezimalzahl} sieht: Mit \verb|-g1| oder \verb|-g2| kommt 
\verb|0,251| \textit{vor} \verb|0,5| (deutsche 'Nachkommastellen'), 
\textit{sonst}\pdfko{0}\ zwischen 22 und 700 (englisch gelesen 
nulltausendzweihunderteinundf"unfzig). 

\textit{Strukturierungszeichen zum besseren 
Lesen gro"ser Zahlen} werden im Englischen und Deutschen erkannt:  
\verb|1.000| wird mit \ko\verb|-g1| oder \ko\verb|-g2| als tausend verstanden, 
sonst als 1 und danach 0, denn englisch stellt \verb|1,000| die Zahl Tausend dar. 
Mit \ko\verb|-g1| oder \ko\verb|-g2| wird \verb|0.251| als 251 verstanden 
$-$ nur bei \textit{drei} 'Nachpunktstellen' ist '.' Strukturierungszeichen:
\verb|Bd.\,3.1| gilt als 'drei-Punkt-eins' und wird vor \verb|Bd.\,10.2| 
einsortiert, aber \verb|Bd.\,3.100| danach.

Bei Punkt und Komma l"asst sich die f"ur einen Text einmal gew"ahlte 
Sprachkonvention sp"ater also nur noch schwer "andern. Setzen der Punkte oder
Kommata in geschweifte Klammern schaltet jedoch die Dezimalzahlenerkennung aus;
\texttt{100\{.\}200} gelten als zwei Zahlen 100 und 200 hintereinander.

Sprachunabh"angig gelten einzelne Leerzeichen, einzelne \verb|\,| oder 
\verb|~| \textit{vor\pdfko{1}\ 
Dreierkolonnen von Zahlzeichen} nicht als 
Unterbrechung einer Zahl. Dementsprechend steht \verb|1\,000\,000| 
immer f"ur eine Million. Und \verb|1000| ist 
freilich immer tausend. Die Zahlenerkennung funktioniert bis zu
999.999.999.999.999 vor dem 'Komma' (deutsch) und bis zu zus"atzlich
16 Stellen nach dem 'Komma' (wobei dort keine Strukturierungszeichen enthalten
sein d"urfen: 0,0000000000000001 ist die kleinste korrekt sortierbare Zahl).
Ziffern nach der 15ten bzw.\ 16ten Stelle werden ignoriert von den 'h"oheren' 
Sortierschichten (die zuerst die Reihenfolge bestimmen) 
und als neue Zahlen begriffen.

Negative Zahlen werden f"alschlicherweise nach ihrem Betrag sortiert. 
F"ur ein Buch, das 700 v.\,Chr.\ erschien, k"onnen Sie 1/700 im  
Taschenrechner bestimmen und unter \verb|-g1| oder \verb|-g2| dann 
\verb|\sort{0,001429}700 v.\,Chr.| im\pdfko{1}\ 
Text setzen;
entsprechend lassen sich alle 'negativen' Jahre vor das 
Jahr \verb|+1|\pdfko{1}\ 
einsortieren (das Jahr \verb|0| kommt aber 
weiterhin vor \textit{allen} anderen Zahlen).

Etwas anderes als Punkte zwischen Zahlen sind Punkte nach
Buchstaben, n"amlich Abk"urzungen. Die Option \ko\verb|-p| sorgt daf"ur, 
dass dort ein \ko\verb|.| im\pdfko{1.25}\ Unterschied zu anderen 
Satzzeichen "ahnlich wie ein Buchstabe z"ahlt. 

\verb|-l| bringt \verb|bibsort| dazu, Leerzeichen nicht zu gewichten. 
Dann wird \verb|DiFabio| vor \verb|Di Niro| einsortiert (entgegen der
Grundeinstellung). \verb|-l|~wirkt sich allerdings nur auf Leerzeichen 
\textit{in} Argumenten der \BibArts\hy Befehle aus; falls Sie 
\verb|\vli{Di}{Niro}{|...\verb|}| tippen, hat es keine Auswirkung.


%%%%%%%%%%

\vspace{2ex}\noindent
Es gibt auch Optionsschalter f"ur \verb|bibsort| zum
\textbf{Sortieren der Argumente}:

\vspace{.15ex}
\verb|-t1| stellt einen Versuch dar, Zeichen aus der zweiten
H"alfte der ASCII\hy Codetabelle auszudrucken; das kommt aber
anscheinend nur vor, wenn Sie kein \verb|{inputenc}| setzen
und Zeichen des erweiterten Teils trotzdem tippen. In was
umgewandelt wird, ist in \verb|bibsort.c| in \verb|teinzerw| fix 
definiert. 

\vspace{.05ex}
\verb|-i=j| sortiert \verb|i| unter \verb|j|; Zug"ange mit  
\textit{beiden} Anfangsbuchstaben bilden in den Listen also 
\textit{einen} Block (wie in alten deutschen Zettelkatalogen).


\vspace{1.5ex}\noindent
Neben \verb|\sort| \baref{sort} l"asst sich die Sortierreihenfolge
mit dem Befehl \verb|\nosort| steuern. Sein Argument wird gedruckt, 
aber zur Bestimmung der Sortierreihenfolge weitgehend ausgeblendet. 
\verb|\nosort| wird gebraucht, wenn zwei gleiche Buchstabenfolgen 
(oder Zahlen) einmal geklammert und einmal ungeklammert auftauchen 
und die 'nackten' Buchstaben zuerst einsortiert werden sollen. (Das 
ist defaultm"a"sig umgekehrt wegen der Spielregel 'Zeichen zuerst':
die oberen Sortierschichten finden zwischen \verb|'a'| und \verb|a| 
keinen Unterschied, die unteren vergleichen danach \verb|'| mit 
\verb|a|.) Wie in der Liste vorgemacht, w"are \verb|\nosort{'}a'|
zu setzen. $-$ Sie k"onnen alternativ auch mit

\vspace{-1.5ex}{\small
\begin{verbatim}
   \newcommand{\Acmd}{'}
   \newcommand{\cmd}{\protect\Acmd}
   \addtoprr{\cmd a'}
\end{verbatim}}

\vspace{-1.5ex}\noindent
erzwingen, dass \verb|'a'| nach \verb|a| und vor \verb|\glqq a\grqq| 
(\glqq a\grqq) einsortiert wird $-$
wegen des A in \verb|\Acmd| gegen"uber g in \verb|\grqq|.
(Dies wirkt ebenfalls nur bei Eintr"agen mit sonst gleichen 
Zeichenfolgen aus Buchstaben und Zahlen!)

\vspace{1ex}\noindent
\verb|bibsort| schreibt keine Protokolldatei, sondern setzt seine
Fehlermeldungen als Kommentarzeilen in die erzeugten Dateien. 
Wenn anders sortiert wird als erwartet, k"onnen Sie \ko\verb|-m| 
setzen; dann f"ugt \verb|bibsort| zu jedem Eintrag als\pdfko{1.25}\   
\LaTeX\hy 
Kommentar seine zum Sortieren genutzten Meta\hy Zeilen hinzu.
(Die\pdfko{.25}\    
Sonderzeichen zur Nachbewertung bilden manche Editoren 
nur teilweise ab!)

\vspace{1ex}\noindent
Wie oben S.\,\pageref{head} vorgef"uhrt, bewirkt \verb|\bibsortheads|,
dass Bl"ocke mit gleichen Anfangsbuchstaben in den Listen mit 
"Uberschriftenbuchstaben versehen werden; \verb|\bibsortspaces|
setzt an diesen Stellen nur vergr"o"serte vertikale Abst"ande.
Die Buchstaben stellt \verb|bibsort| in allen neun
Dateien immer bereit;\pdfko{.5}\ 
\verb|\bibsortheads| ordnet nur an, dies nicht mehr auszublenden. 
Ein Umstellen der Schrift zum Drucken der "Uberschriftenbuchstaben 
ist nicht vorgesehen.

%%%%%%%%%%

\vspace{1.5ex}\noindent
Nun zu \textbf{Seiten- und Fu"snotennummern in num\hy Ausdruckbefehlen} (wie \verb|\printnumvkc|):
Die drucken hinter den Text des Listenpunktes die Seitenzahlen und eventuell Fu"snotennummern aus,
von denen mehrere textgleiche Zug"ange herstammen. Die Reihenfolge,
in der Zahlentypen ausgedruckt werden, hat eine \textbf{Grundeinstellung}.
(Auf der n"achsten Seite beschreibe ich anwenderdefinierte Einstellungen.)
Die Grundeinstellung ist (interne Typen):

\vspace{1.5ex}{\small\sffamily
\hspace{.5cm}\begin{tabular}{rl}
 T4   & \verb|\fnsymbol|,\footnotemark\ also \hspace{.5em} {\small 
 $* \hspace{1em} \dagger \hspace{1em} \ddagger \hspace{1em} \mathchar "278 \hspace{1em} \mathchar "27B \hspace{1em} \delimiter "026B30D \hspace{1em} ** \hspace{1em} \dagger \dagger \hspace{1em} \ddagger \ddagger$} \\
 T5,6 & r"omische Zahlen aus \,\verb|i v x l c d m|\,, dann aus \,\verb|I V X L C D M| \\
 T7   & arabische Zahlen aus \,\verb|0| \,bis \,\verb|9| \\
 T8,9 & Buchstaben\hy Z"ahler aus \,\verb|a| \,bis \,\verb|z|\,, dann aus \,\verb|A| \,bis \,\verb|Z| \\
 T10  & Zeichenfolgen, die nicht als Zahl (an)erkannt werden \\
\end{tabular}}\footnotetext{\texttt{\bs mathchar \string"278}
\,sowie \texttt{\bs ensuremath \{\bs mathsection \}} 
werden als $\mathchar "278$~(Symbolz"ahlerstand~4) akzeptiert,
\texttt{\bs mathchar \string"27B} 
\,sowie \texttt{\bs ensuremath \{\bs mathparagraph \}} 
als $\mathchar "27B$~(5). $-$ \texttt{bibsort}~2.1 akzeptiert zudem 
\texttt{\bs TextOrMath\{}\textit{Textmodus}\texttt{\}\{}\textit{Mathematikmodus}\texttt{\}} 
aus der \LaTeX\hy Distribution 2015/10/01
und zieht zur Bewertung das zweite Argument heran.} 

\vspace{1.8ex}\noindent
Ein Unterschied zwischen der Seiten- und Fu"snoten-Nummerierung ergibt sich
trotz der einheitlichen Reihenfolge, in der die Zahlentypen ausgedruckt
werden, "uber die Reihenfolge, in der \verb|bibsort| seine Instrumente 
anwendet:\pdfko{.25}\ 
Bei den Seitenzahlen \textit{pr"uft} es erst auf kleine r"omische Zahlen 
und dann auf\pdfko{.5}\ kleine Buchstaben. Das bedeutet, dass ein \verb|c| 
defaultm"a"sig als 100 \textit{gilt}, egal,\pdfko{.5}\
ob es von \verb|\pagenumbering{roman}| 
oder von \verb|\pagenumbering{alph}| herstammt (und Drei bedeuten \textit{soll}). 
Doch ein Instrumententausch ist m"oglich: Falls Sie in Ihrer Einleitung gro"se 
r"omische Seitenzahlen wollen, k"onnen Sie \verb|-s1| setzen. In diesem 
Fall d"urfen Sie zudem \verb|\pagenumbering{alph}| in Ihrem\pdfko{.3}\ 
Appendix setzen (ohne \verb|-s1| steht Ihnen \verb|\pagenumbering{Alph}| frei).

\vspace{.2ex}
Beim Auslesen der Fu"snotennummern pr"uft \verb|bibsort| defaultm"a"sig 
erst\pdfko{1.25}\  
auf gro"se r"omische Zahlen und dann auf gro"se Buchstabenz"ahler, weiter  
erst auf kleine Buchstabenz"ahler und dann auf kleine r"omische Zahlen. 
In dieser Bewertungsreihenfolge werden also die Fu"snotenexponenten
von \texttt{minipages} richtig verarbeitet (kleine Buchstaben). 
\ko\verb|-f1| dreht die Bewertungsreihenfolge um: Dann sind kleine r"omische Zahlen
und Gro"sbuchstaben als Fu"snotennummern m"oglich (neben Symbolen
und arabischen Zahlen).\footnote
{Falls Sie \texttt{\bs renewcommand\{\bs thefootnote\}\{\bs Alph\{footnote\}\}}
ohne \texttt{-f1} verwenden, wird \texttt{bibsort} 
drei Zug"ange \texttt{\bs per\{ZfG.\}} aus den Fu"snoten 
{\renewcommand{\thefootnote}{\Alph{footnote}}\footnotemark[1] \footnotemark[2] \footnotemark[3]}
nicht indexartig als $^{A-C}$ zusammenfassen k"onnen, sondern 
die Zahlentypen falsch erkennen: \texttt{\bs printnumper} w"urde
dann hinter den Listenpunkt ZfG.\ die Folge 
\textit{Seitenzahl}\kern1pt$^{C}$$^{,}$ $^{A}$$^{,}$ $^{B}$ ausdrucken.
$-$ "Uberhaupt ist ein Test, ob \texttt{bibsort} Zug"ange von drei 
aufeinanderfolgenden Seiten oder Fu"snoten 
etwa zu 1--3 zusammenfassen kann, das Mittel der Wahl, um zu pr"ufen, ob 
es Ihre Angaben verstanden hat! $-$ Falls Sie 
\texttt{\bs renewcommand\{\bs thepage\}\{\bs fnsymbol\{page\}\}}
anordneten, k"onnen Sie etwa mit 
\texttt{\bs renewcommand\{\bs balistnumemph\}\{\bs rmfamily\}} 
reagieren, wenn beim Listenausdruck die \LaTeX\hy Warnung kommt, 
der Font \texttt{OMS/cmsy/m/n} existiere nicht.}

\vspace{1ex}\noindent
Die wichtigste \textbf{Neuerung} von \BibArts~2.1 gegen"uber 2.0 ist,
dass die Reihenfolge von \textsf{T4} bis \textsf{T9} nun 
optional einstellbar ist: mit \verb|bibsort|~\verb|-s2|~\textit{xxxx}\pdfko{.5}\ 
die Seitenzahlen und separat mit \verb|-f2|~\textit{xxxx}
die Fu"snotennummern. \textit{xxxx} muss\pdfko{.75}\ 
vier der sechs Buchstaben \verb|a|, \verb|A|, \verb|r|, 
\verb|R|, \verb|n| und \verb|s| enthalten, wobei 
\verb|a|~alph, \verb|A|~Alph, \verb|n|~arabic, 
\verb|r|~roman, \verb|R|~Roman und \verb|s|~fnsymbol bedeutet;
in dieser Reihenfolge wird dann der Zahlenindex in den
num\hy Listen gedruckt. Es stehen vier statt alle sechs
Attribute zur Auswahl, weil etwa die Seitennummern c oder C
weiterhin sowohl einer Buchstabenz"ahlung wie auch einer r"omischen 
Z"ahlung entstammen k"onnten: Was gemeint ist, m"ussen Sie 
\texttt{bibsort} sagen. Falls Sie\pdfko{.5}\  
in \verb|bibsort|~\verb|-s2|~\textit{xxxx} 
das \verb|A| setzen, ist deshalb \verb|R| verboten (und umgekehrt), 
und falls Sie \verb|a| setzen, ist \verb|r| verboten (und umgekehrt).
\verb|A| und \verb|a| k"onnen Sie\pdfko{.6}\ 
also in beliebiger Reihenfolge setzen, 
alternativ auch \verb|A| und \verb|r|, \verb|R| und \verb|r|, oder\pdfko{1}\  
\verb|R| und \verb|a|. Das bedeutet gleichzeitig, dass Sie 
beispielsweise Roman in Ihrem Text dann als Seitenz"ahler nicht verwenden 
d"urfen, falls Sie dazu bereits Alph nutzen. Auf in \textit{xxxx} ungenannte Z"ahler wird 
in diesem Modus nicht mehr gepr"uft; sie gelten (wie unbewertbare Nummern) 
als \textsf{T10}! Unver"andert\pdfko{.75}\ 
gilt f"ur Alph und alph, dass \texttt{bibsort} nach z\texttt{=}26 
weiterz"ahlt und aa\texttt{=}27 ist,\pdfko{.5}\  
ab\texttt{=}28, usw. Stets m"ussen \verb|n| und \verb|s| in \textit{xxxx} 
getippt werden (auch wenn Sie beispielsweise fnsymbol im Text gar nicht 
nutzen), um auf die stets geforderten vier Buchstaben zu kommen. 
Durch mehrere Starts von \texttt{bibsort} erzeugte ich\pdfko{.75}\ 
aus einer dazwischen unver"anderten \LaTeX\hy Datei 
\fabra{...}\texttt{.tex} diese Kolonnen:


\vspace{.75ex}\noindent
\verb| bibsort -s2 sArn -f2 snRa |\abra{...}\verb|    |\textsf{A$^{I}$--C$^{III}$$^{,}$\ $^{d-f}$, i$^{*}$--v$^{\mathparagraph }$, 1$^{1}$--5$^{5}$}

\vspace{.25ex}\noindent
\verb| bibsort -s2 sArn -f2 asnR |\abra{...}\verb|    |\textsf{A$^{I}$,\ B$^{II}$, C$^{d-f}$$^{,}$\ $^{III}$, i$^{*}$--v$^{\mathparagraph }$, 1$^{1}$--5$^{5}$}

\vspace{.25ex}\noindent
\verb| bibsort -s2 nsrR          |\abra{...}\verb|    |\textsf{1$^{1}$--5$^{5}$, i$^{*}$--v$^{\mathparagraph }$, C$^{III}$$^{,}$\ $^{d-f}$, \{{A\/}\}$^{I}$, \{{B\/}\}$^{II}$}


\vspace{1.25ex}\noindent
\verb|bibsort| akzeptiert in den Zahlenargumenten die 
"ublichen Befehle zur Einstellung des Schriftgrades. 
Beispielsweise akzeptiert \BibArts\ Ihre Eingabe: 

\vspace{.675ex}\noindent{\small
\verb|  \renewcommand{\thempfootnote}{{\itshape\Alph{mpfootnote}}}|}

\vspace{.75ex}\noindent
Entsprechendes gilt f"ur \verb|\thepage| und \verb|\thefootnote|.
Schriftgr"o"sen\hy Befehle wie \verb|\large| weist \verb|bibsort| dagegen 
zur Index\hy Zahlenverarbeitung zur"uck und wertet solche Nummern als 
\textsf{T10} (TEXT). Es gibt aber \verb|\bapageframe| und\pdfko{.75}\ 
\verb|\bafootnoteframe|, um Befehlscode oder Text vor \verb|bibsort| zu
verbergen:

\vspace{.75ex}\noindent{\small
\verb|  \renewcommand{\thepage}{{\bapageframe{\roman{page}}}}| \\
\verb|  \renewcommand{\thefootnote}{{\bafootnoteframe{\arabic{footnote}}}}|}

\vspace{1ex}\noindent
Die drucken gem"a"s Voreinstellung die Seitenzahl (auf der Seite!)
und die Fu"snotenexponenten in Schr"agstrichen aus. \verb|bibsort| 
druckt diese 'Klammern' nicht aus, erkennt aber den Wert des Z"ahlers. 
(Andere Programme wie \textsc{MakeIndex} akzeptieren derart ver"anderte 
Z"ahler jedoch nicht mehr!) 

Falls Sie andere Symbole ausgedruckt haben wollen, hier ein Beispiel:

\vspace{1ex}\noindent{\small
\verb|  \renewcommand{\pbapageframe}[1]{\{#1\}}| \\
\verb|  \renewcommand{\pbafootnoteframe}[1]{(#1)}| \\
\verb|  \renewcommand{\thepage}{{\bapageframe{\roman{page}}}}| \\
\verb|  \renewcommand{\thefootnote}| \\
\verb|      {{\bfseries\bafootnoteframe{\arabic{footnote}}}}|}

\vspace{1.5ex}\noindent
Das w"urde die Seitenzahl in geschweiften Klammern und die 
Fu"snotennummer fett in fetten runden Klammern ausdrucken 
(sowie auch \LaTeX- \ko und\pdfko{1.25}\ 
\BibArts\hy Querverweise, die auf 
solche Seiten oder Fu"snoten zeigen).

\vspace{.1ex}
Als unbewertbar bewertete 'Zahlen' (\textsf{T10}) druckt \verb|bibsort| 
alphabetisch sortiert als \textsf{$^{\{{\,A\/}\}}$$^{,}$ $^{\{{\,B\/}\}}$$^{,}$ $^{\{{\,C\/}\}}$}
aus. Es gibt nie eine Zusammenfassung zu Gruppen wie A--C.
Sie k"onnen mit \verb|-c| den Ausdruck der geschweiften Klammern 
unterdr"ucken. Leere Z"ahlerstandsausdrucke\footnote{Tritt auf bei 
'r"omische Seite Null'.} gibt \verb|bibsort| als \verb|[]| wieder; 
und Ausdrucke von Z"ahler"uberl"aufen\footnote{Wenn Z"ahlerst"ande 
gr"o"ser 26 in Buchstaben ausgedruckt werden sollen. Die Eigenschaft
von \texttt{bibsort}, aa dort als 27 einzustufen, ist mit dem
\LaTeX\hy alph nicht ausnutzbar.} als \verb|()|; beides l"asst sich 
nicht ausschalten.

\vspace{1ex}\noindent
Eine \textbf{"Anderung} von \BibArts~2.1 gegen"uber Version 2.0 ist,
dass ein vom Pr"afix des Input\hy Files \textit{abweichender} Name f"ur
das Pr"afix der ganzen Familie der von \texttt{bibsort} erzeugten
Dateien ab jetzt mit \verb|-o| angek"undigt werden \textit{muss}:

\vspace{.4ex}
\verb|bibsort |\textit{infile}\verb| -o |\textit{outfile}

\vspace{.6ex}\noindent
Weiter \textit{kann} das Input\hy File mit einer Option, 
n"amlich \verb|-i|~\textit{infile}, nun explizit gekennzeichnet 
werden. So lassen sich etwa auch Dateien bearbeiten, deren 
Name wie eine Option mit einem Minuszeichen beginnt.



%%

\vspace{1ex}\noindent
\verb|bibsort| sortiert \verb|\fnsymbol| im Grundeinstellung
deswegen zuerst ein, weil \LaTeX\ diese Marken f"ur Fu"snoten in 
seiner Titelkonstruktion vorsieht. Innerhalb des Arguments
von \verb|\title| separierte ich \verb|\footnote| in 
\verb|\footnotemark|\pdfko{.5}\ und \verb|\footnotetext| 
in folgendem Beispiel, sonst droht ein Speicher"uberlauf:


\newpage

\vspace{.5cm}\noindent
\begin{verbatim}
\documentclass[12pt,a4paper]{article}
  \usepackage{ngerman} \usepackage{bibarts}
          
   \author{Peter Maier}
    \title{Aufsatz\footnotemark[1]}


\begin{document}

{\renewcommand{\thefootnote}{\fnsymbol{footnote}}
 \footnotetext[1]{Vgl.\ dazu \vli{Niall}{Ferguson}{Der
    \ktit{\onlykurz{F}\onlyvoll{f}alsche\onlykurz{r} Krieg}, 
    M"unchen 2001}[22].}}

\maketitle

\noindent
Der erste Satz.\footnote{\kli{Ferguson}{Falscher Krieg}[23].}

\end{document}
\end{verbatim}

\vfill
\begin{minipage}{.9\textwidth}
\vspace{1.5cm}\noindent
{\renewcommand{\thempfootnote}{\fnsymbol{mpfootnote}}%
\begin{center}
{\LARGE \hspace*{.4em}Aufsatz$^{*}$}\\[4.05ex]
{\large Peter Maier}\\[2.65ex]
{\large \today}\\
\end{center}
 \footnotetext[1]{Vgl.\ dazu \vli{Niall}{Ferguson}{Der
 \ktit{\onlykurz{F}\onlyvoll{f}alsche\onlykurz{r} Krieg}, 
       M"unchen 2001}[22].}%
}%
{\renewcommand{\thempfootnote}{\arabic{mpfootnote}}%
\vspace{4ex}\noindent
Der erste Satz.\footnote{\kli{Ferguson}{Falscher Krieg}[23].}}
\vspace{2cm}
\end{minipage}



\newpage
\thispagestyle{empty}
\section*{Inhaltsbeschreibung}\label{SectIn}
\newcommand\tocline[2]{\hbox to .5cm{\hfill\bfseries{\ref{#1}}} \ {#2} \dotfill\hbox to .5cm {\hfill\ttfamily\pageref{#1}}}

\strut \\[-1ex]
\hspace*{2cm}\textsf{Zun"achst werden die zentralen \BibArts\hy Befehle erkl"art:} \\[.875ex]
\tocline{Sect1} {Vollzitate und Kurzzitate} \\[.25ex]
\tocline{Sect2} {W"ortliche Zitate in verschiedenen Sprachen} \\[.25ex]
\tocline{Sect3}{Formatierungs- und Editionshilfen} \\[.25ex]
\tocline{Sect4} {Abk"urzungen} \\[.25ex]
\tocline{Sect5} {\texttt{\bs abk\{X.X.X.\}} unter \texttt{\bs nonfrenchspacing}} \\[.25ex]
\tocline{Sect6} {Zeitschriften und allgemein Bandangaben} \\[.25ex]
\tocline{Sect7} {Archivquellen} \\[.25ex]
\tocline{Sect8} {Orts\fhy, Sach\hy\ und Personenregister} \\[1.5ex]
\hspace*{2cm}\textsf{Dann beschreibe ich Sonderf"alle und Hintergrundbefehle:} \\[.875ex]
\tocline{Sect9} {\texttt{\bs protect} und zerbrechliche Befehle} \\[.25ex]
\tocline{Sect10}{Punkte, \,\texttt{\bs bahasdot} \,und \,\texttt{\bs banotdot}} \\[.25ex]
\tocline{Sect11} {\textit{Italics}\hy Korrekturen und Separatoren} \\[.25ex]
\tocline{Sect12}{Sprachabh"angige Separatoren (\kern-.02em\textit{captions})} \\[1.5ex]
\hspace*{2cm}\textsf{Hier kommen Zusammenstellungen nach Aufgabentyp:} \\[.875ex]
\tocline{Sect13}{Die \BibArts\hy Hauptbefehle} \\[.25ex]
\tocline{Sect14}{Hervorhebung von \BibArts\hy Argumenten} \\[.25ex]
\tocline{Sect15}{\BibArts\hy Ein\fhy\ko/\ko\ko Ausschalter (bes.\ \ko f"ur \ko\ko Vorspann)} \\[.25ex]
\tocline{Sect16}{\BibArts\hy\hspace{-.025em}1.3\hspace{.075em}\hy Texte unter \BibArts~2.1} \\[.25ex]
\tocline{Sect17}{Listenausdruck (\BibArts-Belegapparat)} \\[1.5ex]
\hspace*{2cm}\textsf{Und zuletzt Sortierprogramm und Sortierreihenfolge:} \\[.875ex]
\tocline{Sect18}{\texttt{bibsort} samt Erweiterungen gegen"uber 2.0} \\ \strut



\vfill
\hbox{\parbox{7.7cm}{\footnotesize\noindent
\textbf{\BibArts~2.1 (9 Dateien, 8 vom 19.03.2016):} \\[.85ex]
 \begin{tabular}{ll}%
 \texttt{readme.txt}   & Versionsgeschichte seit 1.3        \\[-1.75pt]
 \texttt{bibarts.sty}  & Das \LaTeX-Style-File              \\[-1.75pt]
 \texttt{bibarts.pdf}  & Diese Dokumentation hier           \\[-1.75pt]
 \texttt{bibarts.tex}  & Quellcode von \texttt{bibarts.pdf} \\[-1.75pt]
 \texttt{ba-short.pdf} & Englische Kurzdokumentation        \\[-1.75pt]
 \texttt{ba-short.tex} & Quellcode von \texttt{ba-short.pdf}\\[-1.75pt]
 \texttt{bibsort.exe}  & Bin"ardatei f"ur die Listen        \\[-1.75pt]
 \texttt{bibsort.c}    & Quellcode von \texttt{bibsort.exe} \\[-1.75pt]
 \texttt{COPYING}      & Lizenz (vom 28.11.1993)                \\
 \end{tabular}} 
\\
\hspace*{2.4mm}
\parbox{5.6cm}{\vspace{1.56ex}\sethyphenation{english}\tiny\sffamily
    This program is free software; you can redistribute it and/or modify
    it under the terms of the GNU General Public License as published by
    the Free Software Foundation; either version 2 of the License, or
    (at your option) any later version.

    This program is distributed in the hope that it will be useful,
    but WITHOUT ANY WARRANTY; without even the implied warranty of
    MERCHANTABILITY or FITNESS FOR A PARTICULAR PURPOSE.  See the
    GNU General Public License for more details.

    You should have received a copy of the GNU General Public License
    along with this program; if not, write to the Free Software
    Foundation, Inc., 675 Mass Ave, Cambridge, MA 02139, USA.
}}

{\footnotesize\vspace{2ex}\noindent \BibArts\ ist kostenlos. Bitte
dokumentieren Sie "Anderungen vor der Weitergabe. 
Zur Diskussion schreiben Sie Emails an 
\texttt{bibarts}\kern.075em(\kern-.05em at\kern-.075em)\kern.075em\texttt{gmx.de}\kern.05em; 
ich werde nach M"oglichkeit antworten.}


%%%%%%%%%%%%%%%%%%%%%%%%%%%%%%%%%%%%


%\tableofcontents

\end{document}
