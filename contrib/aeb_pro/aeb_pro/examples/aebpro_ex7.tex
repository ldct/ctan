% use useacrobat option with pdftex and xetex if you have acrobat
% use nopro if you don't have acrobat (the attachsource silently fails)
\documentclass{article}
\usepackage[
    gopro,
    web={designv,nodirectory,usesf},
    eforms, % only needed for \toggleAttachmentsPanel
    attachsource={tex}
]{aeb_pro}
\usepackage{array}
\newcommand{\cs}[1]{\texttt{\char`\\#1}}


\begin{comment}
    The layoutmag key has values:
        navitab=UseNone,UseOutlines,UseThumbs,UseOC,UseAttachments
        pagelayout=SinglePage,OneColumn,TwoColumnLeft,TwoColumnRight,TwoPageLeft,TwoPageRight
        mag=ActualSize,FitPage,FitWidth,FitHeight,FitVisible, or positive number (e.g., mag = 50)
        openatpage = page number (base 1)

    The windowoptions key has values:
        fit,center,showtitle,fullscreen

    The uioptions key has values:
        hidetoolbar,hidemenubar,hidewindowui
\end{comment}

\DeclareInitView
{%
    layoutmag={openatpage=2,mag=ActualSize,pagelayout=TwoPageRight},
    windowoptions={fit,center,showtitle},
    uioptions={hidetoolbar,hidemenubar,hidewindowui}
}
\DeclareDocInfo
{%
    title=AcroTeX Fun with Initial View,
    university=Acro\negthinspace\TeX.Net,
    author=D. P. Story,
    email=dpstory@acrotex.net,
    subject=Testing total control,
    talksite=\url{www.acrotex.net},
    version=1.0,
    keywords={Initial View tab, Document Properties}
}
\nocopyright

\optionalPageMatter
{%
    \par\bigskip
    \begin{center}
    \toggleAttachmentsPanel{red}{View the attachment}
    \end{center}
}



\begin{document}

\maketitle

\noindent The \cs{DeclareInitView} command is a companion to
\cs{DeclareDocInfo}, each of these fills a tab of the
\textsf{Document Properties} dialog box.

\begin{verbatim}
\DeclareInitView
{%
    layoutmag={openatpage=2,mag=ActualSize,pagelayout=TwoPageRight},
    windowoptions={fit,center,showtitle},
    uioptions={hidetoolbar,hidemenubar,hidewindowui}
}
\DeclareDocInfo
{%
    title=AcroTeX Fun with Initial View,
    university=Acro\negthinspace\TeX.Net,
    author=D. P. Story,
    email=dpstory@acrotex.net,
    subject=Testing total control,
    talksite=\url{www.acrotex.net},
    version=1.0,
    keywords={Initial View tab, Document Properties}
}
\end{verbatim}
Use this document to experiment with the various
properties of set by \cs{DeclareInitView}.

\newpage

\begin{itemize}

    \item \texttt{layoutmag}: This key sets the initial page layout and magnification
    of the document. The values of this key are themselves key-values:

    \begin{small}\setlength{\extrarowheight}{3pt}
    \begin{tabular}{|>{\ttfamily}l>{\raggedright}p{1.85in}p{2.15in}<{\raggedright}|}\hline
    \multicolumn{1}{|l}{Key}         &\multicolumn{1}{l}{Value(s)} & Description \\\hline
    navitab     & \texttt{UseNone}, \texttt{UseOutlines}, \texttt{UseThumbs},
                  \texttt{UseOC}, \texttt{UseAttachments}
                & The UI for these are Page Only, Bookmarks Panel
                  and Page, Pages Panel and Page, Layers Panel and
                  Page, Attachments Panel and Page, respectively. The
                  default is \texttt{UseNone}\\
    pagelayout  & \texttt{SinglePage}, \texttt{OneColumn}, \texttt{TwoPageLeft},
                  \texttt{TwoColumnLeft}, \texttt{TwoPageRight}, \texttt{TwoColumnRight}
                & The UI for these are Single Page, Single Page
                  Continuous, Two-Up (Facing), Two-Up Continuous
                  (Facing), Two-Up (Cover Page), Two-Up Continuous
                  (Cover Page), respectively. The default is user's
                  preferences.\\
    mag         & \texttt{ActualSize}, \texttt{FitPage}, \texttt{FitWidth},
                  \texttt{FitHeight}, \texttt{FitVisible}, or \texttt{<positive number>}
                & The UI for these are Actual Size, Fit Page, Fit
                  Width, Fit Height, Fit Visible, respectively. If a
                  positive number is provided, this is interpreted as
                  a magnification percentage. The default is to use
                  user's preferences.  \\
    openatpage  & \texttt{<positive number>}
                & The page number (base 1) to open the document at. Default is page 1.\\\hline
    \end{tabular}
    \end{small}%

\newpage

    \item \texttt{windowoptions}: The Window Options region of the
    Initial View tab consists of a series of check boxes, which when
    checked modifies the initial state of the document window. These are
    not really Boolean keys. If the key is present, the
    corresponding box in the UI will be checked, otherwise, the box
    remains cleared.

    \begin{tabular}{|>{\ttfamily}lp{2in}|}\hline
    Key         & Description \\\hline
    fit         & Resize window to initial page\\
    center      & Center window on screen \\
    fullscreen  & Open in Full Screen mode\\
    showtitle   & Show document title in the title bar\\\hline
    \end{tabular}

    Note that you can open the document in Full Screen mode using the
    \texttt{fullscreen} key above, or by using the \texttt{fullscreen} key
    of the \cs{setDefaultFS}. Either will work.

\newpage

    \item \texttt{uioptions}: The User Interface Options region of
    the Initial View tab consists of a series of check boxes, which
    when checked hides a UI control. These are not really Boolean
    keys. If the key is present, the corresponding box in the UI
    will be checked, otherwise, the box remains cleared.

        \begin{tabular}{|>{\ttfamily}lp{2in}|}\hline
    Key             & Description \\\hline
    hidemenubar     & Hide menu bar\\
    hidetoolbar     & Hide tool bars\\
    hidewindowui    & Hide window controls\\\hline
    \end{tabular}

\end{itemize}


\end{document}
