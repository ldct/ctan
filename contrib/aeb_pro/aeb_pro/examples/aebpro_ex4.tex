% dvips/Distiller workflow only
\documentclass{article}
\usepackage{amsmath}
\usepackage[%
    web={
        pro,
        designv,
        tight,
%        forcolorpaper,
        centertitlepage,
        dvipsnames,
        usesf
    },
    uselayers,
    eforms,
    aebxmp,
]{aeb_pro}
\usepackage{graphicx}
\usepackage[nomessages]{fp}
%
% The versions of pstricks-add and pstricks should be
% fairly recent.
%
\usepackage{pstricks-add}
\usepackage[absolute,overlay]{textpos}

\DeclareDocInfo
{
    title=The AeB Pro Package\texorpdfstring{\\[1ex]}{: }Layers and Rollovers,
    author=D. P. Story,
    university=Acro\negthinspace\TeX.Net,
    email=dpstory@acrotex.net,
    subject=Test file for the AeB Pro package,
    keywords={Adobe Acrobat, JavaScript},
    talksite=http://www.acrotex.net,
    talkdate={January 12, 2007},
    copyrightStatus=True,
    copyrightNotice={Copyright (C) \the\year, D. P. Story},
    copyrightInfoURL=http://www.acrotex.net
}
\talkdateLabel{Published:}

\newcommand{\cs}[1]{\texttt{\char`\\#1}}
\newcommand\newtopic{\par\ifdim\lastskip>0pt\relax\vskip-\lastskip\fi
\vskip\medskipamount\noindent}
\newenvironment{sverbatim}
{\par\footnotesize\verbatim}{\endverbatim}
\def\AcroTeX{Acro\negthinspace\TeX}

\begin{document}

\maketitle

\section{Layers}

When the \texttt{uselayers} option is taken, the necessary code is
input to produce layers (Optional Content Groups). The
\textcolor{blue}{{\AcroTeX} Presentation Bundle} (APB), which has a
very sophisticated method of control over layers, by comparison, the
\textcolor{blue}{AeB Pro} layer support is very primitive indeed. As
a rule, after you create a layer, you will need a control of that
layer. This could be a button or a link that executes JavaScript.

The basic command for creating a layer is \cs{xBld}. The content of
the layer is set off by the \cs{xBld}/\cs{eBld} pair.  The command
\cs{xBld} takes two parameters: (1) the first is optional,
\texttt{true} if the layer is initially visible or \texttt{false},
the default, if the layer is hidden initially; (2) the name of the
layer, this is used to reference the layer, to make it visible or
hidden.

The creation and control of layers are illustrated in the two
subsequent sections.

\newtopic When constructing a layer, there are two possible scenarios:
\begin{enumerate}
    \item The layer takes up ``tex'' space
    \item The layer does not take up ``tex'' space
\end{enumerate}
Let's look at each of these in turn.

\subsection{Layer takes up space}

A layer that takes up space in tex is the easiest case.  For
example, guess what I'm thinking thinking about: I'm thinking about
my formerly favorite number, \xBld{mythoughts}the number is the
value of the integral $\int_0^4 4x + \frac14\,dx$. \eBld Want a
hint? \setLink[%
\A{\JS{%
    var oLayer = getxBld("mythoughts");
    if ( oLayer != null )
        oLayer.state = !oLayer.state;
    }}
]{\textcolor{red}{Click here}} I hope that hint worked for you.
Click on the link to hide the layer again.

\newtopic For those viewing this document in PDF, the relevant code
code is seen below.

First the layer is enclosed in a \cs{xBld}/\cs{eBld} pair, the
default state is hidden, so we don't supply the optional parameter.
The require (second) parameter is the name used to refer to the
layer.

The link text has a JavaScript action. First we get the OCG object
for this layer by calling the \texttt{getxBld} function (this is
part of the AeB Pro JavaScript) then if non-null (you may not have
spelled the name correctly) we toggle the current state,
\texttt{oLayer.state = !oLayer.state}.
\begin{sverbatim}
\xBld{mythoughts}the number is the value of the integral
$\int_0^4 4x + \frac14\,dx$. \eBld Want a hint?
\setLink[\A{\JS{%
    var oLayer = getxBld("mythoughts");
    if ( oLayer != null )
        oLayer.state = !oLayer.state;
    }}
]{\textcolor{red}{Click here}}
\end{sverbatim}

An advantage of the layers approach is that the content of the
layers are latexed as part of the content of the tex file;
consequently, you can include virtually anything in your layer that
tex can handle, math, figures, PSTricks, etc. Acrobat Pro 7.0 (and
distiller) or later is required to build the layers, but only Adobe
Reader 7.0 or later is needed to view the document, once
constructed.

\NewPage
\subsection{Layer takes no space}\label{nospace}

Perhaps the easiest way of creating layers that do not take up any tex space is to use a
package such as \textsf{textpos}. In the preamble of this document, I've placed the following
command:

\begin{sverbatim}
\usepackage{pstricks-add,pstricks}
\usepackage[absolute,overlay]{textpos}
\end{sverbatim}

\begin{textblock*}{.67\linewidth}[.5,.5](.5\paperwidth,.5\paperheight)
\xBld{crossinggraphcs}\psshadowbox[framesep=0pt]{\fcolorbox{red}{cornsilk}{%
\parbox{\linewidth}
{%
\begin{center}
\begin{pspicture}(-0.5,-1,0)(7,5)
  \pscustom[linewidth=1pt,fillstyle=hlines,hatchcolor=red]{
    \psplot[linewidth=1pt]{2}{5}{%
     x x mul neg x 6 mul add 5 sub}
    \psplot[linewidth=1pt]{5}{2}{%
     x x mul neg 3 div x 4 mul 3 div add 5 3 div add}
  }
  \pscustom[linewidth=1pt,fillstyle=vlines,hatchcolor=blue]{
    \psline(0,1.666667)
    \psplot[linewidth=1pt]{0}{2}{%
       x x mul neg 3 div x 4 mul 3 div add 5 3 div add}
    \psplot[linewidth=1pt]{2}{1}{%
     x x mul neg x 6 mul add 5 sub}
  }
  \psaxes{->}(0,0)(-0.5,-0.5)(7,5)
  \uput[-90](7,0){$x$}
  \uput[0](0,5){$y$}
  \psplot[linewidth=1pt]{0.75}{5.25}{%
     x x mul neg x 6 mul add 5 sub}
  \psplot[linewidth=1pt]{0}{5.25}{%
     x x mul neg 3 div x 4 mul 3 div add 5 3 div add}
  \rput(5,2){$f$}
  \rput(3,2){$g$}
\end{pspicture}
\end{center}
}}}\eBld
\end{textblock*}

Now we can create a layer that takes up no space, but is overlaid on top of the tex content.
We use the \textsf{textpos} commands to place the layer. Once created and placed, we must
have a way of showing it and hiding it. This time, we'll use a button:
\pushButton[
    \CA{Push}
\A{\JS{%
    toggleSetThisLayer("crossinggraphcs");
}}]{crossinggraphcsBtn}{}{12bp}

\newtopic As you can see from this example, you must be very careful in the placement of your
layers vis-\`{a}-vis the button.  Form fields are laid on top of
content, and a layer is considered part of the content. This particular layer could be move to the
upper left corner, for example, Let's see how a link works, shall we? \setLink[%
\A{\JS{%
    toggleSetThisLayer("crossinggraphcs");
}}]{\textcolor{red}{Click here}}! The link is active even through the layer, you just have to find it!
I could have set this link to that it has a visible bounding box, in which case the border of the
bounding box shows through the layer, but hey, at least you can find the link!

One thing you need to keep in mind is the placement of the code,
within the \texttt{textblock} environment.  The overlaid material
will appear on whatever page the \texttt{textblock} ends up on.
Should you add more material before the \texttt{textblock}, the
controls may drop off to the next page. When the user presses the
control, nothing happens, actually, the layer is appearing on the previous page.
Sometimes it makes sense to insert a \cs{newpage} to ensure the proper placement
of the overlay relative to the placement of the controls.

\NewPage % this is \newpage when for screen, does nothing when forpaper, defined in web

\defineRC{help1}
{%
\psset{llx=-12pt, lly=-12pt,urx =12pt,ury =12pt,xAxisLabel=$x$,yAxisLabel=$y$}
\begin{psgraph*}[arrows=->,trigLabels=true,trigLabelBase=2,dx=\psPiH](0,0)(-.5,-1.5)(6.75,1.5){.75\linewidth}{.25\linewidth}
    \psset{algebraic=true}%
    \rput(4,1){$y=\sin(x)$}%
    \psplot[linecolor=blue]{0}{TwoPi}{sin(x)}
\end{psgraph*}
}

\begin{rollover}
\begin{textblock*}{.4\linewidth}[.5,.5](.5\paperwidth,.5\paperheight)
\xBld{help1}\psshadowbox[framesep=0pt]{\fcolorbox{red}{cornsilk}{%
\parbox{\linewidth}{\insertRC{help1}}}}\eBld
\end{textblock*}
\end{rollover}
\begin{printRollover}
\definePR{help1}{\parbox{.4\linewidth}{\insertRC{help1}}}\insertPR{help1}
\end{printRollover}

\section{Rollovers}

The AeB Pro package offers you two rollovers, which ostensibly provides
help to the document consumer.

\subsection{\texorpdfstring{\protect\cs{texHelp}}{\textbackslash{texHelp}} uses Layers}

The rollover is a special case of the techniques discussed in
\hyperref[nospace]{section~\ref*{nospace}}. For a rollover, you
create a form field, enclosing a word, when the user rolls across
the word, a help box opens. For example, recall the graph of the
\texHelp{help1}{$\sin(x)$,} isn't it pretty? Try rolling over the
word that is tagged with the `\texHelpIndicator{}' symbol.

You will have to look at the source file to see what goes on here,
suffice it to say that the command \cs{texHelp} was used to create
the rollover, the symbol that appears is defined in the
\textsf{aeb\_pro} package by \cs{texHelpIndicator}, this command can
be re-defined to have a different symbol.

\subsection{\texorpdfstring{\protect\cs{pdfHelp}}{\textbackslash pdfHelp} does not}

There is another type of rollover, \cs{pdfHelp}. The command takes three arguments,
the first one is the name of the button field that is enclose around the third argument,
the second argument is the help text. The help text is not part of the {\LaTeX} content
of the document is is not compiled by the {\TeX} compiler, so it should be text. For example,
can you remember the inventor of \pdfHelp{knuth}{Donald Knuth}{\TeX?}

As with \cs{texHelp}, the symbol that tags the word as a help word
is defined in \textsf{aeb\_pro.dtx} by the command
\cs{pdf\-Help\-Indicator}. This command can be re-defined as needed.

\section{Layers and Animation}

Let's see if we can conjure up a little animation, shall we?

\noindent\begin{minipage}{.65\linewidth}\centering
\DeclareAnime{sinegraph}{10}{40}
\def\thisframe{\animeBld\psplot[linecolor=red]{0}{\xi}{sin(x)}\eBld}
\psset{llx =-12pt,lly=-12pt,urx =12pt,ury =12pt} % ,trigLabels=true,labelFontSize=\small
\begin{psgraph*}[arrows=->,trigLabels=true,trigLabelBase=2,dx=\psPiH](0,0)(-.5,-1.5)(6.75,1.5){164pt}{70pt}
    \psset{algebraic=true}%
    \rput(4,1){$y=\sin(x)$}
    \FPdiv{\myDelta}{\psPiTwo}{\nFrames}%
    \def\xi{0}%
    \multido{\i=1+1}{\nFrames}{\FPadd{\xi}{\xi}{\myDelta}\thisframe}
\end{psgraph*}
\end{minipage}\hfill
\begin{minipage}{.3\linewidth}%\centering
\backAnimeBtn{24bp}{12bp}\kern1bp\clearAnimeBtn{24bp}{12bp}\kern1bp\forwardAnimeBtn{24bp}{12bp}
\end{minipage}

\newtopic It is possible to combine the techniques of
\hyperref[nospace]{section~\ref*{nospace}} with those of animation, to create
an anime that pops up. We leave this as an exercise for the reader, that's you.

\NewPage

\newtopic\xBld{dialog1}This page illustrates how you can create a series of
steps, all controlled by a button. \eBld\xBld{dialog2}Let us begin by
drawing the $\sin(x)$ curve. \eBld\xBld{dialog3}Do you think that I can do it?
\eBld\xBld{dialog4} Well, here goes!\eBld

\begin{center}
\DeclareAnime{graph}{10}{40}
\def\frame{\animeBld\psplot[linecolor=red]{0}{\xi}{sin(x)}\eBld}
\psset{llx=-12pt, lly=-12pt,urx =12pt,ury =12pt,xAxisLabel=$x$,yAxisLabel=$y$}
\begin{psgraph*}[arrows=->,trigLabels=true,trigLabelBase=2,dx=\psPiH](0,0)(-.5,-1.5)(6.75,1.5){.5\linewidth}{.25\linewidth}
    \psset{algebraic=true}%
    \rput(4,1){$y=\sin(x)$}%
    \FPdiv{\myDelta}{\psPiTwo}{\nFrames}%
    \def\xi{0}%
    \multido{\i=1+1}{\nFrames}{\FPadd{\xi}{\xi}{\myDelta}\frame}
    \xBld{step1}\psplot[linecolor=blue]{0}{Pi 2 div}{sin(x)}\eBld
    \xBld{step2}\psplot[linecolor=webgreen]{Pi 2 div}{Pi}{sin(x)}\eBld
    \xBld{step3}\psplot[linecolor=Plum]{Pi}{3 Pi mul 2 div}{sin(x)}\eBld
    \xBld{step4}\psplot[linecolor=yellow]{3 Pi mul 2 div}{TwoPi}{sin(x)}\eBld
\end{psgraph*}
\end{center}

\newtopic\xBld{step1}Notice that the function increases from $[\,0, \pi/2\,]$
and attains it maximum value of $1$ at $ x = \pi/2$. \eBld\xBld{step2}The
sine function then decreases to zero over the interval $[\,\pi/2, \pi\,]$.
\eBld \xBld{step3}Continuing its sinusoidal path it decreases on the
interval $[\,\pi,3\pi/2\,]$ down to its minimum value of $-1$ at $ x =
3\pi/2$. \eBld \xBld{step4}Finally, it increases up to $0$ over the
interval $[\,3\pi/2, 2\pi\,]$ completing the first cycle of its
graph.\eBld

\newtopic\xBld{wrapup}Try \texttt{shift-click} too. dps\eBld

\begin{defineJS}[\catcode`*=0]{\stepwise}
if ( typeof stepcounter == "undefined" )  stepcounter = 0;
if ( !event.shift ) stepcounter++;
if ( stepcounter > 10 ) stepcounter = 11;
if ( stepcounter < 1 ) stepcounter = 0;
switch (stepcounter)
{
    case 1:  toggleSetThisLayer("dialog1", !event.shift); break;
    case 2:  toggleSetThisLayer("dialog2", !event.shift); break;
    case 3:  toggleSetThisLayer("dialog3", !event.shift); break;
    case 4:  toggleSetThisLayer("dialog4", !event.shift); break;
    case 5:  if (event.shift) aebAnimeLayersBackward(*animSpeed,*nFrames,"*animBaseName");
             else aebAnimeLayersForward(*animSpeed,*nFrames,"*animBaseName");
             break;
    case 6:  toggleSetThisLayer("step1", !event.shift); break;
    case 7:  toggleSetThisLayer("step2", !event.shift); break;
    case 8:  toggleSetThisLayer("step3", !event.shift); break;
    case 9:  toggleSetThisLayer("step4", !event.shift); break;
    case 10: toggleSetThisLayer("wrapup",!event.shift); break;
    default:
}
if ( event.shift ) stepcounter--;
\end{defineJS}

\medskip
\begin{center}
    \pushButton[\CA{Step Through}\TU{Step Through}\A{\JS{\stepwise}}]%
    {stepthru}{}{12bp}
\end{center}


\end{document}
