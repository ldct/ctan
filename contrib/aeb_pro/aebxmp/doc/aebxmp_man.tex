% makeindex < aebpro_man.idx > aebpro_man.ind
\documentclass{article}
\usepackage[fleqn]{amsmath}
\usepackage[
    web={centertitlepage,designv,
        forcolorpaper,tight*,latextoc,pro},
    eforms,aebxmp
]{aeb_pro}
\usepackage{aeb_mlink}
\usepackage{graphicx,array}
%\usepackage{myriadpro}
%\usepackage[usecmtt]{myriadpro}
\usepackage[altbullet]{lucidbry}

%\usepackage{makeidx}
%\makeindex
\usepackage{acroman}
\usepackage[active]{srcltx}

\def\anglemeta#1{$\langle\textit{\texttt{#1}}\rangle$}
\def\meta#1{\textit{\texttt{#1}}}
\def\darg#1{\texttt{\{#1\}}}
\def\takeMeasure{\bgroup\obeyspaces\takeMeasurei}
\def\takeMeasurei#1{\global\setbox\webtempboxi\hbox{\ttfamily#1}\egroup}
\def\bxSize{\wd\webtempboxi+2\fboxsep+2\fboxrule}
\let\pkg\textsf
\let\env\texttt
\let\opt\texttt
\let\app\textsf
\let\uif\textsf

\edef\amtIndent{\the\parindent} %\leftmargini
\def\SUB#1{${}_{\text{#1}}$}

\newdimen\aebdimen \aebdimen\abovedisplayskip
\newcommand\bVerb[1][]{\begingroup#1\vskip\aebdimen\parindent0pt}%
\def\eVerb{\vskip\aebdimen\endgroup\noindent}

\def\cmdtitle#1{\texorpdfstring{\protect\cs{#1}}{\textbackslash{#1}}}

%\margins{.25in}{.25in}{24pt}{.25in} % left,right,top, bottom
%\screensize{6.5in}{5in} % height, width

\university{\AcroTeX.Net}
\title{The aebXMP Package\texorpdfstring{\\[1em]}{: }Updating XMP using E4X and {\LaTeX}}
\author{D. P. Story}
\Authors{{D. P. Story}{J\u00FCrgen Gilg}}
\subject{Documentation of the AeBXMP Package}
\Keywords{XMP,E4X,Adobe Acrobat,JavaScript}
\email{dpstory@acrotex.net}
\version{2.5a, 2017/02/17}
\talksite{\url{www.acrotex.net}}
\copyrightStatus{True}
\copyrightNotice{Copyright (C) 2006-\the\year, D. P. Story}
\copyrightInfoURL{http://www.acrotex.net}
\authortitle{Programming and Development, AcroTeX.Net}
\descriptionwriter{Testing and Promotions Department, AcroTeX.Net}
\customProperties
{%
    {name=Developer,value={D. P. Story, Esq.}}
    {name=Motivator,value=J\u00FCrgen Gilg}
}

\DeclareInitView{windowoptions=showtitle}

\def\dps{$\hbox{$\mathfrak D$\kern-.3em\hbox{$\mathfrak P$}%
   \kern-.6em \hbox{$\mathcal S$}}$}

\def\AcroTeX{Acro\negthinspace\TeX}

\makeatletter
\@mparswitchfalse\reversemarginpar
\makeatother

\makeatletter
\renewcommand{\paragraph}
    {\@startsection{paragraph}{4}{0pt}{6pt}{-3pt}{\bfseries}}
\renewcommand*\l@subsection{\@dottedtocline{2}{1.5em}{2.5em}}
\renewcommand*\descriptionlabel[1]{\hspace\labelsep
    \normalfont #1}
\newcommand{\aebDescriptionlabel}[1]{%
    \setlength\dimen@{\amtIndent+\labelsep}%
    {\hspace*{\dimen@}#1}}
\makeatother
\newenvironment{aebDescript}
    {\begin{list}{}{\setlength{\labelwidth}{0pt}%
        \setlength{\leftmargin}{\leftmargin}%
        \setlength{\leftmargin}{\leftmargin+\amtIndent}%
        \setlength\itemindent{-\leftmargin}%
        \let\makelabel\aebDescriptionlabel
    }}{\end{list}}

%\pagestyle{empty}
%\parindent0pt\parskip\medskipamount


\definePath\bgPath{"C:/Users/Public/Documents/ManualBGs/Manual_BG_Print_AeB.pdf"}
\begin{docassembly}
\addWatermarkFromFile({
    bOnTop:false,
    cDIPath:\bgPath
});
\executeSave();
\end{docassembly}


\begin{document}

\maketitle

\changelinkcolorto{black}

\tableofcontents

\changelinkcolorto{webgreen}


\section{Introduction}

The motivation for the development of this package came from
Herr~J\"{u}rgen Gilg, who had a need to fill in the metadata fields
beyond those normally populated by using \pkg{hyperref}:
\textsf{Title}, \textsf{Author}, \textsf{Subject} and
\textsf{Keywords}. Of particular interest to him were the metadata
fields \textsf{Copyright Status}, \textsf{Copyright Notice} and
\textsf{Copyright Info URL}.

After doing some research on the CTAN archives, I came across the
\textsf{hyperxmp} package by Scott Pakin.\begin{NoHyper}\footnote{The reader is
invited to read the documentation of the \textsf{hyperxmp}, as
contained therein is a good discussion of XMP (eXtensible Metadata
Platform).}\end{NoHyper} The package works well with \textsf{pdftex} and
\textsf{dvipdfm}, but has a bit of a problem when using the
distiller. For this reason, I sought my own solution to the problem.

As a beta tester of \app{Acrobat~8 Professional}, I had the opportunity to
use some of the new capabilities of the JavaScript interpreter as an
alternate approach to the one used by Mr.~Patkin. The JavaScript
version~1.6 interpreter, the one used in version 8, comes with E4X,
an XML parser, built in.  I could see that E4X could be exploited to
manipulate the XMP data, and this was my approach.

\section{Requirements}

The techniques used by the \textsf{aebxmp} package to update the XMP data
require the \textcolor{blue}{{\AcroTeX} eDucation Bundle} (AeB), freely
available from \url{www.acrotex.net}.  Because E4X is used, we also require
\app{Acrobat~8 Pro} or later, and, since you have \textsf{Acrobat~8} or
later, my \LaTeX ing friend, this package will work for you with all
workflows: \app{dvips/Distiller}, \app{pdflatex} (including \app{lualatex}),
and \app{xelatex}. To emphasize, for non-\app{Distiller} workflow, the full
\app{Acrobat} application is still required to be your default {\PDF} viewer
on your own computer.

\section{The Test File}

The package \textsf{aebxmp} has a simple test file,
\textsf{aebxmp\_test.tex}, which is found in the \texttt{examples} folder.
After you build the {\PDF} and open the document in \app{Acrobat} for the first time,
the new metadata is imported; \marginpar{\raggedleft save the document}\emph{it is important to save the document} after
the data is imported.

To use this package, you must have, in addition to Acrobat~8 Pro (or later),
installed on your computer a standard {\TeX} system, including the
latest version of AeB.\footnote{AeB can be downloaded from any CTAN
server, from \url{www.math.uakron.edu/~dpstory/webeq.html}, or from
\url{www.acrotex.net}.}

\section{Metadata}

The \pkg{hyperref} package provides basic metadata support, providing
a mechanism for providing the title, author, subject, and keywords. Beyond
these, we can include additional metadata.

\subsection{Author, Author}

The \pkg{hyperref} package allows you to specify an author or authors;
the \textsf{Web} package uses this through the \cs{title} command, which
passes its argument on to \pkg{hyperref} of inclusion in the PDF Info
dictionary. The authors names are not individually assessable through
JavaScript. In Acrobat~9, I believe, the JavaScript API includes the
\texttt{Doc.info.Authors} property, its value is an array of authors,
We can access each author using array notation: First author is
\texttt{this.info.Authors[0]}, second author is
\texttt{this.info.Authors[1]}, and so on. The number of authors is
obtained using the length property of arrays, \texttt{this.info.Authors.length}.

The \pkg{aebxmp} interface to this is through the \cs{Authors} command. It takes
a list of authors, each enclosed in braces, no commas (,) between the
authors. Like so:
\begin{Verbatim}[xleftmargin=\amtIndent,fontsize=\fontsize{9}{11}\selectfont]
\Authors{{D. P. Story}{J\u00FCrgen Gilg}}
\end{Verbatim}
Using \cs{Authors} will overwrite the author(s) named in the \cs{author}
command (of \pkg{web}), or more generally, passed to
\cs{hypersetup\darg{pdfauthor=\darg{\anglemeta{author}}}} of
\pkg{hyperref}.\begin{NoHyper}\footnote{This may be to your advantage when
using \textsf{Web}; the value of \cs{author} is used to display the author(s)
of the document, you may want to present the names one way on the title page,
for example, and another way in the \textsf{Description} tab of the
\textsf{Document Properties} dialog box.}\end{NoHyper}

If you do use \cs{Authors} and overwrite the author(s) as passed through
\pkg{hyperref}, the first author listed will be the one returned by
\texttt{this.info.Author} (a JavaScript property) in the \uif{Document
Properties} dialog box (\texttt{Ctrl+D}), all authors are listed in a
semi-colon delimited list.


\subsection{The \texorpdfstring{\cs{Keywords}}{\textbackslash{Keywords}} Command}

The \cs{Authors} command utilizes \texttt{Doc.info.Authors}, which takes
an array of authors names. There is no such property available for
\texttt{Doc.info.Keywords}, the value of this property takes only a string
of keywords.  The keywords are stored by Acrobat in three ways,
(1) \texttt{pdf:Keywords}; (2) in the \textbf{Info} dictionary; and (3) in
\texttt{dc:subject}. In the latter case, \texttt{dc:subject} takes a \texttt{Bag}
of subjects (keywords). A \texttt{Bag}, in XMP parlance, is an unordered array.

The command \cs{Keywords} takes a comma delimited list of keywords. Notice
the word is capitalized to distinguish it from \cs{keywords}, which is
defined in the \textsf{Web} package as the interface to inserting the
keywords, via \pkg{hyperref}, into the \textbf{Info} dictionary.
\begin{Verbatim}[xleftmargin=\amtIndent,fontsize=\fontsize{9}{11}\selectfont]
\Keywords{AcroTeX.Net,XMP,E4X,Adobe Acrobat,JavaScript}
\end{Verbatim}
The command takes each comma-delimited list of keywords and
inserts each into the \texttt{dc:subject} part of the metadata.
For this document, the keywords appear in the XML metadata as
\begin{Verbatim}[xleftmargin=\amtIndent,fontsize=\fontsize{9}{11}\selectfont]
<dc:subject>
   <rdf:Bag>
      <rdf:li>AcroTeX.Net</rdf:li>
      <rdf:li>XMP</rdf:li>
      <rdf:li>E4X</rdf:li>
      <rdf:li>Adobe Acrobat</rdf:li>
      <rdf:li>JavaScript</rdf:li>
   </rdf:Bag>
</dc:subject>
\end{Verbatim}
When the document is first opened, the following command is executed
\begin{Verbatim}[xleftmargin=\amtIndent,fontsize=\fontsize{9}{11}\selectfont]
this.info.Keywords="AcroTeX.Net; XMP; E4X; Adobe Acrobat; JavaScript"
\end{Verbatim}
Here, I have broken the string across lines for readability. This inserts
the same list of keywords into the \textbf{Info} dictionary.

When you executed \texttt{this.info.Keywords} in the console, you'll get
%
\begin{Verbatim}[xleftmargin=\amtIndent,fontsize=\fontsize{9}{11}\selectfont]
AcroTeX.Net; XMP; E4X; Adobe Acrobat; JavaScript
\end{Verbatim}
as expected.

To access the individual keywords, I've defined an array of keywords,
\texttt{aKeywords}. It takes as its argument the index of the
keyword you want to get; For example, executing \texttt{aKeywords[0]} in
the console returns a value of \texttt{"AcroTeX.Net"}, while
\texttt{aKeywords[4]} returns a value of \texttt{"JavaScript"}. If you
execute \texttt{aKeywords[5]}, a value of \texttt{undefined} is returned.
The number of keywords is \texttt{aKeywords.length}.

Listing the keywords may be of interest to someone, it is an exercise to
me.\begin{NoHyper}\footnote{Actually, this is the way \textbf{Acrobat} handles a
comma-delimited list of keywords when the words are entered through
the user interface; it puts them in a \texttt{Bag}.}\end{NoHyper}

The command \cs{xmpDoNotInsKWScript}, when expanded in the preamble, will
turn off the creation of the array \texttt{aKeywords}.

\subsection{Additional Metadata}

As mentioned previously, the \pkg{aebxmp} package addresses three areas of
interest: Setting the \uif{Copyright Status}, \uif{Copyright Notice}, and the
\textsf{Copyright Info URL}. Obviously, other elements of the XMP can be
addressed. To that end, the \pkg{aebxmp} package defines five new {\LaTeX}
commands to populate the values of the five metadata fields \textsf{Copyright
Status}, \uif{Copyright Notice}, \uif{Copyright Info URL}, \uif{Author
Title}, and \uif{Writer Description}. Values for the arguments of these
commands are documented below.
\begin{description}
\item[\cs{copyrightStatus\darg{True|False}}:] If \texttt{True},
    \uif{Copyright Status} is set to \texttt{Copyrighted}; if
    \texttt{False}, \textsf{Copyright Status} is set to \uif{Public
    Domain}. If left empty, the status is set to \uif{Unknown}.

    Unless you've executed \cs{xmpDoNotInsKWScript} in the preamble, the
    \pkg{aebxmp} defined JavaScript function \texttt{getCopyrightStatus()}
    is available. The function returns the copyright status:
    \texttt{Copyrighted}, \texttt{Public Domain}, and \texttt{Unknown}.

\item[\cs{copyrightNotice\darg{\anglemeta{text}}}:] The \anglemeta{text} of
    the \textsf{Copyright Notice} is defined

\item[\cs{copyrightInfoURL\darg{\anglemeta{URL}}}:] The \anglemeta{URL} to
    the copyright information

    Unless you've executed \cs{xmpDoNotInsKWScript} in the preamble, the
    \pkg{aebxmp} defined JavaScript function \texttt{getCopyrightInfoURL()}
    is available. The function returns the copyrightinfo URL (\anglemeta{URL}).

\item[\cs{authortitle\darg{\anglemeta{text}}}:] The \anglemeta{text} appears
    in the \textsf{Author Title} line on the \textsf{Additional Metadata}
    dialog box. This is a \uif{Photoshop} property. (See the \uif{Advanced}
    category in the left panel.)

    Unless you've executed \cs{xmpDoNotInsKWScript} in the preamble, the
    \pkg{aebxmp} defined JavaScript function \texttt{getAuthorTitle()}
    is available. The function returns the authortitle (\anglemeta{text}).

\item[\cs{descriptionwriter\darg{\anglemeta{text}}}:] The \anglemeta{text}
    appears in the \textsf{Description Writer} line on the \uif{Additional
    Metadata} dialog box. This is a \uif{Photoshop} property. (See the
    \uif{Advanced} category in the left panel.)

    Unless you've executed \cs{xmpDoNotInsKWScript} in the preamble, the
    \pkg{aebxmp} defined JavaScript function \texttt{getDescriptionWriter()}
    is available. The function returns the descriptionwriter (\anglemeta{text}).

\end{description}

\newtopic\noindent
For example, for this document, we have in the preamble,
\begin{Verbatim}[xleftmargin=\amtIndent,fontsize=\fontsize{9}{11}\selectfont]
\copyrightStatus{True}
\copyrightNotice{Copyright (C) 2006-\the\year, D. P. Story}
\copyrightInfoURL{http://www.acrotex.net}
\authortitle{Programming and Development, AcroTeX.Net}
\descriptionwriter{Testing and Promotions Department, AcroTeX.Net}
\end{Verbatim}
Enter unicode (\cs{uXXXX}) directly into the \cs{copyrightNotice} and \cs{copyrightInfoURL} fields; for example,
\begin{Verbatim}[xleftmargin=\amtIndent,fontsize=\fontsize{9}{11}\selectfont]
\copyrightNotice{Copyright (C) 2006-\the\year, J\u00FCrgen Gilg}
\end{Verbatim}
Unicode can be used all the metadata commands discussed in the manual.

The \cs{copyrightNotice} can take multiple arguments, one for each
language. The syntax is
\begin{Verbatim}[xleftmargin=\amtIndent,fontsize=\fontsize{9}{11}\selectfont,commandchars=!()]
\copyrightNotice{%
    {[!anglemeta(lang!SUB(1))] copyright in this language}
    {[!anglemeta(lang!SUB(2))] copyright in this language}
    ...
}
\end{Verbatim}
The first copyright item listed is also counted as the default language
and will be marked as \texttt{x-default} as the value of the
\texttt{xml:lang} attribute. Contrary to {\LaTeX} custom, the brackets do
not indicate optional arguments, they are required except for the first
item in the list. Each group, which are enclosed in braces
(\texttt{\{\}}), represents a copyright notice; the part enclosed on
brackets (\texttt{[]}) contains the language designator. This is a
two-letter code to indicate the language; it can also have a sub-tag to
indicate a country (see \texttt{en-US} in example below). See the ISO
639-1 standard, and the RFC 3066 standard, referenced at the end of the
manual, for more information on language codes.

For example,
\begin{Verbatim}[xleftmargin=\amtIndent,fontsize=\fontsize{9}{11}\selectfont]
\copyrightNotice{%
    {[en-US]Copyright (C) \the\year, D. P. Story}
    {[fr]Copyright (C) \the\year, D. P. Story}
    {[de]Copyright (C) \the\year, D. P. Story}
}
\end{Verbatim}

There is also a \cs{sourceFile} command that takes one argument. If
\cs{sourceFile} does not appear in the preamble, the \cs{jobname.tex} is
written to the metadata (as part of the Dublin Core Properties). If
\verb!\sourceFile{}! is expanded in the preamble, no source file data will
be inserted into the metadata. Finally, executing the command
\verb!\sourceFile{hw01_1100.tex}! causes the string
\texttt{hw01\_1100.tex} to be written as the value of the
\texttt{dc:source} key.


\subsection{\texorpdfstring{\cs{Title}, \cs{Subject}, and \cs{metaLang}}
    {\textbackslash{Title}, \textbackslash{Subject}, \textbackslash{metaLang}}}

The Title and Subject keys can also be recorded with alternate languages;
for this reason, \textsf{aebxmp} defines \cs{Title} and \cs{Subject}.
The syntax of these two are similar to \cs{copyrightNotice}, described
above
\begin{Verbatim}[xleftmargin=\amtIndent,commandchars=!(),fontsize=\fontsize{9}{11}\selectfont]
\Title{%
    {[!anglemeta(lang!SUB(1))] title in this language}
    {[!anglemeta(lang!SUB(2))] title in this language}
    ...
}
\Subject{%
    {[!anglemeta(lang!SUB(1))] subject in this language}
    {[!anglemeta(lang!SUB(2))] subject in this language}
    ...
}
\end{Verbatim}
The first one listed is also designated as the default language, marked
with \texttt{x-default}.

For example,
\begin{Verbatim}[xleftmargin=\amtIndent,fontsize=\fontsize{9}{11}\selectfont]
\Title{%
    {[en-US]Testing the aebxmp Package}
    {[fr]Test du paquet aebxmp}
    {[de]Testen des aebxmp Pakets}
}
\Subject{%
    {[en-US]Test file for using E4X to update the XMP Data Model}
    {[fr]Fichier de test utilisant E4X pour mettre � jour
            le mod�le de donn�es XMP}
    {[de]Testdatei f�r die Verwendung von E4X, um das XMP Daten
            Modell zu aktualisieren}
}
\end{Verbatim}
Note that literal characters such as \texttt{\"{u}} are also
recognized so that unicode is not needed here.

The arrays \texttt{aTitle} and \texttt{aSubject} are defined in the
document, unless the command \cs{xmpDoNotInsKWScript} is executed in the
preamble. For example, if you executed \texttt{aTitle[0]} in the console
(or part of a JavaScript action of a button), the array element is seen to
be \texttt{"[x-default]:\,Testing the aebxmp Package"}, while
\texttt{aTitle[1]} is \texttt{"[en-US]:\,Testing the aebxmp Package"}. The
array \texttt{aSubject} behaves similarly.

The data passed by \cs{Title} and \cs{Subject} overrides the data passed
by the \textsf{web} commands \cs{title} and \cs{subject}, and overrides
the data passed by the \pkg{hyperref} keys \texttt{pdftitle} and
\texttt{pdfsubject}.

Special characters need to be entered using unicode (\cs{uXXXX}), not the
octal or unicode generated by \pkg{hyperref}. Do not use {\LaTeX}
markup that expands to special characters inside the arguments of any of
the commands defined in this package.

The \cs{metaLang} command allows you to document the languages used in
the metadata. Multiple languages may be specified.
\begin{Verbatim}[xleftmargin=\amtIndent,commandchars=!(),fontsize=\fontsize{9}{11}\selectfont]
\metaLang{!anglemeta(lang!SUB(1)),!anglemeta(lang!SUB(2)),..,!anglemeta(lang!SUB(n))}
\end{Verbatim}
For example
\begin{Verbatim}[xleftmargin=\amtIndent,fontsize=\fontsize{9}{11}\selectfont]
\metaLang{en,en-US,fr,de}
\end{Verbatim}


\subsection{Custom Document Properties}
You can define custom properties using the \cs{customProperties} command.
\bVerb
\begin{minipage}{.5\linewidth}
\def\1{\textbf{Standard Syntax}}%
\begin{Verbatim}[xleftmargin=\amtIndent,commandchars=!(),fontsize=\fontsize{9}{11}\selectfont]
!1
\customProperties
{%
    {name=!anglemeta(name!SUB(1)),value=!anglemeta(value!SUB(1))}
    {name=!anglemeta(name!SUB(2)),value=!anglemeta(value!SUB(2))}
}
\end{Verbatim}
\end{minipage}\hfil
\begin{minipage}{.5\linewidth}
\def\1{\textbf{Colon Syntax}}%
\begin{Verbatim}[xleftmargin=\amtIndent,commandchars=!(),fontsize=\fontsize{9}{11}\selectfont]
!1
\customProperties
{%
    {name:!anglemeta(name!SUB(1)),value:!anglemeta(value!SUB(1))}
    {name:!anglemeta(name!SUB(2)),value:!anglemeta(value!SUB(2))}
}
\end{Verbatim}  
\end{minipage}
\eVerb
The `colon' syntax is also recognized, but do not mix the two syntaxes together, use either all equal signs or all colons.

The value of the \texttt{name} key requires a unique name, which must not
be one of the standard property names \texttt{Title}, \texttt{Author}, \texttt{Subject}, \texttt{Keywords},
\texttt{Creator}, \texttt{Producer}, \texttt{CreationDate}, \texttt{ModDate}, and
\texttt{Trapped}. The value of \texttt{name} shall consist of the
characters \texttt{A--Z}, \texttt{a--z}, and \texttt{0--9}, and beginning with a
letter. The value may contain unicode, for example, in the preamble of
this document we have,
\begin{Verbatim}[xleftmargin=\amtIndent,fontsize=\fontsize{9}{11}\selectfont]
\customProperties
{%
    {name=Developer,value={D. P. Story, Esq.}}
    {name=Motivator,value=J\u00FCrgen Gilg}
}
\end{Verbatim}
Instead of unicode, this same set of custom properties can be defined as
follows:
\begin{Verbatim}[xleftmargin=\amtIndent,fontsize=\fontsize{9}{11}\selectfont]
\customProperties
{%
    {name=Developer,value={D. P. Story, Esq.}}
    {name=Motivator,value=J�rgen Gilg}
}
\end{Verbatim}
That is, using literal characters, if your editor supports it.

The properties may be accessed through the \texttt{info} property of the
\texttt{Doc} object. The button (on the left) \marginpar{\hfill\pushButton[\TU{Press to see the document
properties}\CA{Info}
\A{\JS{%
    console.show();\r
    console.clear();\r
    for (var o in this.info)\r\t
        console.println("info."+o+"="+this.info[o]);\r
    if (aKeywords!=undefined) {\r\t
    	console.println("List Keywords:");\r\t
    	for (var i=0; i< aKeywords.length; i++){\r\t\t
        	console.println(" "+aKeywords[i]);\r\t
    	}\r
    }
}}]{info}{}{11bp}}
opens the console debugger window and displays all the document
properties.


The custom properties may be viewed using the user interface; press
\texttt{Ctrl+D} and choose the \textsf{Custom} tab.

For more information on this topic, see
\href{http://www.adobe.com/devnet/xmp.html}{%
Part 3, Storage in Files}, section 3.2 on \textbf{Native metadata in PDF files}, in particular,
see section~3.2.1 concerning user-defined document properties.


\section{Checking for validity}

While looking at this file in Acrobat (or Adobe Reader), press
\texttt{Ctrl+D} to get the \textsf{Document Properties} dialog box.
Select the \textsf{Description} tab and click \textsf{Additional
Metadata}. Since this document was built using the \textsf{aebxmp}
package, with the declarations

\begin{Verbatim}[xleftmargin=\amtIndent,fontsize=\fontsize{9}{12}\selectfont]
\copyrightStatus{True}
\copyrightNotice{Copyright (C) 2006-\the\year, D. P. Story}
\copyrightInfoURL{http://www.acrotex.net}
\authortitle{Programming and Development, AcroTeX.Net}
\descriptionwriter{Testing and Promotions Department, AcroTeX.Net}
\end{Verbatim}

\newtopic In the Advanced Metadata dialog box, you should see,
\begin{itemize}
    \item[] \textsf{Copyright Status}: \texttt{Copyrighted}
    \item[] \textsf{Copyright Notice}: \texttt{Copyright (C) 2006-\the\year, D. P. Story}
    \item[] \textsf{Copyright Info URL}: \texttt{http://www.acrotex.net}
    \item[] \textsf{Author Title}: \texttt{Programming and Development,
    AcroTeX.Net}
    \item[] \textsf{Description Writer}: \texttt{Testing and Promotions Department,
    AcroTeX.Net}
\end{itemize}
in addition to the usual Document Title, Author, Description, and
Keywords. I promise you that I did not enter these values through the user interface. \texttt{:-)}
\section{Resources}

\newtopic The resources for the development of this package are
\begin{itemize}
    \item \textsl{Standard ECMA-357: ECMAScript for XML (E4X) Specification},\\
        {\small\url{http://www.ecma-international.org/publications/standards/Ecma-357.htm}}
    \item \textsl{XMP Specification}, \url{http://www.adobe.com/devnet/xmp.html}
    \item \textsl{Acrobat JavaScript Scripting Reference}, Version 8.0\\
    \url{http://www.adobe.com/go/acrobat_developer}
    \item \textsf{hyperxmp} package by Scott Pakin,
    \url{http://ctan.org/pkg/hyperxmp}
    \item The {\AcroTeX} System Tools, available for free download at \url{www.acrotex.net}. This is
        a {\LaTeX}-based system.
    \item ISO 639-1 two-letter abbreviation.\\
    \url{http://www.loc.gov/standards/iso639-2/php/English_list.php}
    \item IETF RFC 3066\\
    \url{http://www.ietf.org/rfc/rfc3066.txt}

% http://www.iana.org/assignments/language-subtag-registry

\end{itemize}

\newtopic\noindent
Now, I simply must get back to my retirement. \dps

\end{document}
