%%%%%%%%%%%%%%%%%%%%%%%%%%%%%%%%%%%%%%%%%%%%%%%%%%%%%%%%%%%%%%%%%%%%%%%%%
%
%  TEXT FIELD LATEX FILE EXAMPLE: USES QUESTIONANDRESPONSES COMMAND, 7 ARGUMENTS
%       1ST ARGUMENT INDICATES THIS IS "FIELD" TYPE QUESTION
%       2ND, 3RD, 4TH ARGUMENTS GIVE NAMES OF PREVIOUS, INDEX, NEXT LINKING FILES
%       5TH ARGUMENT GIVES QUESTION, WITH UP TO 5
%           BLANK TEXT FIELDS FOR QUESTION TAKER (USER); APPEAR ON SCREEN
%           ANSWER TEXT FIELDS FOR QUESTION GIVER; DO NOT APPEAR ON SCREEN
%       6TH, 7TH ARGUMENTS GIVE RESPONSES IF USER ANSWER CORRECT OR INCORRECT
%
%%%%%%%%%%%%%%%%%%%%%%%%%%%%%%%%%%%%%%%%%%%%%%%%%%%%%%%%%%%%%%%%%%%%%%%%%
\documentclass[dvips]{article}
\usepackage{interactiveworkbook} % put in style directory; cannot appear in any other directory

\begin{document}

\questionandresponses{field} % tells style file choose fieldclear, fieldsubmit buttons
{popup.pdf}{ndex.pdf}{radio.pdf} % previous, index and next files are 2nd, 3rd, 4th arguments
{ % question and responses has *seven* arguments
    Question 3. $\;$ A multiple text field question.
    % question is 5th argument
    \begin{center}
    % "field..." shows blank text fields on screen; "answerfield..." not shown on screen
    % question taker (user) types answer in "field..."; question giver uses "answer..."
    % question taker types answer text "one" into argument of \answerfieldone{}
    one \fieldone \answerfieldone{one} \\
    two \fieldtwo \answerfieldtwo{two} \\ % question giver gives answer "two" in this case
    three \fieldthree \answerfieldthree{three} \\
    four \fieldfour \answerfieldfour{four} \\
    five \fieldfive \answerfieldfive{five} % all five text fields do not have to be used
    \end{center}
} % end of first argument; "}" *must* appear after last line of question, with no blank lines
{Yes, correct.} % correct response hint is 6th argument of questionandresponses command
{No, at least one of the answers you typed in is not exactly the same
as the preset question giver answers.} % incorrect response hint is 7th argument



\end{document}
