%%%%%%%%%%%%%%%%%%%%%%%%%%%%%%%%%%%%%%%%%%%%%%%%%%%%%%%%%%%%%%%%%% \iffalse %%%%
%                                                                              %
%  Copyright (c) 2016 - Michiel Helvensteijn   (www.mhelvens.net)              %
%                                                                              %
%  This work may be distributed and/or modified under the conditions           %
%  of the LaTeX Project Public License, either version 1.3 of this             %
%  license or (at your option) any later version. The latest version           %
%  of this license is in     http://www.latex-project.org/lppl.txt             %
%  and version 1.3 or later is part of all distributions of LaTeX              %
%  version 2005/12/01 or later.                                                %
%                                                                              %
%  This work has the LPPL maintenance status `maintained'.                     %
%                                                                              %
%  The Current Maintainer of this work is Michiel Helvensteijn.                %
%                                                                              %
%  This work consists of the files withargs.tex and withargs.sty.              %
%                                                                              %
%%%%%%%%%%%%%%%%%%%%%%%%%%%%%%%%%%%%%%%%%%%%%%%%%%%%%%%%%%%%%%%%%%%%%%% \fi %%%%

\documentclass[a4paper]{withargs-packagedoc}

%%%%%%%%%%%%%%%%%%%%%%%%%%%%%%%%%%%%%%%%%%%%%%%%%%%%%%%%%%%%%%%%%%%%%%%%%%%%%%%%
% Setup                                                                        %
%%%%%%%%%%%%%%%%%%%%%%%%%%%%%%%%%%%%%%%%%%%%%%%%%%%%%%%%%%%%%%%%%%%%%%%%%%%%%%%%

\moretexcs{
	withargs,foo,bar,bas,expectedResult,
	actualResult,MyPost,A,uniquecsname,NewPost,
	cs_generate_variant:Nn,
	withargs:nn,withargs:cn
}

%%%%%%%%%%%%%%%%%%%%%%%%%%%%%%%%%%%%%%%%%%%%%%%%%%%%%%%%%%%%%%%%%%%%%%%%%%%%%%%%
% Global Changes                                                               %
%%%%%%%%%%%%%%%%%%%%%%%%%%%%%%%%%%%%%%%%%%%%%%%%%%%%%%%%%%%%%%%%%%%%%%%%%%%%%%%%

\changes{0.0.1}{2013/10/01}
  {initial version}
  
\changes{0.0.2}{2013/10/11}
  {renamed package to \texttt{withargs}}
  
\changes{0.1.0}{2015/09/07}
  {adjusted code to new version of expl3}

\changes{0.2.0}{2016/12/19}
  {adjusted code to new version of expl3}

%%%%%%%%%%%%%%%%%%%%%%%%%%%%%%%%%%%%%%%%%%%%%%%%%%%%%%%%%%%%%%%%%%%%%%%%%%%%%%%%
\begin{document}                                                               %
%%%%%%%%%%%%%%%%%%%%%%%%%%%%%%%%%%%%%%%%%%%%%%%%%%%%%%%%%%%%%%%%%%%%%%%%%%%%%%%%

\maketitle

%%%%%%%%%%%%%%%%%%%%%%%%%%%%%%%%%%%%%%%%%%%%%%%%%%%%%%%%%%%%%%%%%%%%%%%%%%%%%%%%
\section{Introduction}                                                         %
%%%%%%%%%%%%%%%%%%%%%%%%%%%%%%%%%%%%%%%%%%%%%%%%%%%%%%%%%%%%%%%%%%%%%%%%%%%%%%%%

An example is worth a thousand words:

\begin{latex-example-show}
\withargs [!] [Hello] [world] { #2 #3#1 }
\end{latex-example-show}

A quick explanation: we're passing three pieces of \LaTeX\ code in the form
of optional arguments to the final argument, which forms the output. We're
defining a new anonymous macro and invoking it right away.
Up to seven optional arguments are supported.



\subsection{Why is this useful?} %%%%%%%%%%%%%%%%%%%%%%%%%%%%%%%%%%%%%%%%%%%%%%%

One of the main use-cases for this package is to insert the expanded
result of a computation into the middle of a big
blob of code that \emph{shouldn't} be expanded. This is possible because
argument substitution bypasses the expansion process:

\begin{latex-example-show}
\def \expectedResult {\bar}
\def \actualResult   {\bas}
\withargs (oo) [\actualResult] [\expectedResult] {
    \texttt{\detokenize{
        The \foo variable resulted in `#1' instead of `#2'!
    }}
}
\end{latex-example-show}

`|(oo)|' indicates that both arguments should be expanded `once'.
Any optional \LaTeX3 style argument specification between parentheses
may be specified to control expansion.

\needspace{10\baselineskip}
You can define your own commands accepting `templates'
with \TeX-style parameters:

\begin{latex-example-show}
\newcommand{\NewPost}[4]{
    % #1: author     % #2: title
    % #3: content    % #4: template
    \withargs (xnn) [#1] [#2] [#3] {#4}
}

\def\MyPost{ \NewPost{Author}{Title}
    {Boy, do I have some important stuff to say!} }

\begin{tabular}{p{3.5cm}p{3.5cm}p{3.5cm}}
    \MyPost{\textbf{#2\hfill#1}\vskip1mm\hrule\vskip1mm\textit{#3}} &
    \MyPost{\fbox{#1: ``#2''}\par#3\vskip1mm\hrule}                 &
    \MyPost{#2: #3\hfill\textit{(#1)}}
\end{tabular}
\end{latex-example-show}

Generally, if speed is not a concern, |\withargs| can be used
to make \LaTeX\ code more readable in a variety of situations.



%\vskip-.4\baselineskip
%%%%%%%%%%%%%%%%%%%%%%%%%%%%%%%%%%%%%%%%%%%%%%%%%%%%%%%%%%%%%%%%%%%%%%%%%%%%%%%%
\section{Document Level Commands}                                              %
%%%%%%%%%%%%%%%%%%%%%%%%%%%%%%%%%%%%%%%%%%%%%%%%%%%%%%%%%%%%%%%%%%%%%%%%%%%%%%%%
%\vskip-.5\baselineskip

\def\mystrut{\rule{0pt}{4mm}}

\describemacro{\withargs}{
	\texttt{\bfseries(}%
		\(\overbrace{\mystrut\hbox{\meta{argument specification}}}^x\)%
	\texttt{\bfseries)}
	\(\overbrace{\mystrut\hbox{\oarg{1} \oarg{2} \oarg{3} \oarg{4} \oarg{5} \oarg{6} \oarg{7}}}^x\)
	\marg{body}
}

Leaves \meta{body} in the output with |#1|\ldots|#7| replaced
by optional arguments \meta{1}\ldots\!\meta{7}.

An optional \LaTeX3 style \meta{argument specification} between parentheses
can be provided, in which case the arguments undergo expansion as specified
before being placed in the \meta{body}.
%%
Here is an example demonstrating the possible expansion types:

\begin{latex-example-show}
\def\foo{\bar}   \def\s{s}   \def\bar{ba\s}   \def\bas{FOO-BAR-BAS}

\withargs (n     o     f     x     c     v   )
          [\foo][\foo][\foo][\foo][\foo][\foo] {
    \begin{tabular}{llll}
        n: & \texttt{\detokenize{#1}} &   % no expansion
        o: & \texttt{\detokenize{#2}} \\  % expand once
        f: & \texttt{\detokenize{#3}} &   % expand until unexpandable
        x: & \texttt{\detokenize{#4}} \\  % exhaustive expansion
        c: & \texttt{\detokenize{#5}} &   % \csname conversion
        v: & \texttt{\detokenize{#6}}     % `c', then `o'
    \end{tabular}
}
\end{latex-example-show}

Note that all spaces in the \meta{argument specification} are ignored.
For details about \LaTeX3 argument specifications, have a look at the
following documentation:

\begin{center}
	\url{http://www.ctan.org/pkg/expl3}
\end{center}


\describemacro{\uniquecsname}{}\vskip-2\baselineskip

Perhaps a bit misleading, |\uniquecsname| is not actually defined as a command,
but |\withargs| recognizes it as a special marker. If an optional |\withargs|
argument consists entirely of |\uniquecsname|, it is replaced by a
command sequence name which is guaranteed to be unique\ldots unless
you look at the source code (Section~\ref{sec:implementation})
and intentionally replicate the naming scheme.

\begin{latex-example-show}
\withargs (ccc) [\uniquecsname][\uniquecsname][\uniquecsname] {
    \def#1{A}  \let\A#1
    \def#2{B}
    \def#3{C}
    #3#2#1%
}%
(\A)
\end{latex-example-show}

\vskip5mm


%%%%%%%%%%%%%%%%%%%%%%%%%%%%%%%%%%%%%%%%%%%%%%%%%%%%%%%%%%%%%%%%%%%%%%%%%%%%%%%%
\section{\LaTeX3 Functions}                                                    %
%%%%%%%%%%%%%%%%%%%%%%%%%%%%%%%%%%%%%%%%%%%%%%%%%%%%%%%%%%%%%%%%%%%%%%%%%%%%%%%%

The \LaTeX3 functions underlying the functionality of |\withargs| are
also available as is.

\describeanymacro{\texttt{\textbackslash withargs:\smash{\(\overbrace{\mystrut\hbox{nnnnnnnn}\!}^x\,\)}n}}{
	\(\overbrace{\mystrut\hbox{\marg{1} \marg{2} \marg{3} \marg{4} \marg{5} \marg{6} \marg{7} \marg{8}}}^x\)
	\marg{body}
}

These functions do pretty much the same thing as the main |\withargs| macro,
except that the argument specification is embedded in the function name
as per \LaTeX3 coding convention, so a parameter slot is freed up for
custom use. They are slightly faster than |\withargs|, as there is no
need to gather optional arguments or do any error checking.
The downside is that the |\uniquecsname| marker doesn't work for these.

To use a specific expansion scheme, you have to define a variant:

\begin{latex-example-show}
\ExplSyntaxOn
    \cs_generate_variant:Nn \withargs:nn {cn}
    \withargs:cn {LaTeX} {#1}
\ExplSyntaxOff
\end{latex-example-show}

Read the \LaTeX3 documentation for details.



%%%%%%%%%%%%%%%%%%%%%%%%%%%%%%%%%%%%%%%%%%%%%%%%%%%%%%%%%%%%%%%%%%%%%%%%%%%%%%%%
\end{document}                                                                 %
%%%%%%%%%%%%%%%%%%%%%%%%%%%%%%%%%%%%%%%%%%%%%%%%%%%%%%%%%%%%%%%%%%%%%%%%%%%%%%%%
