\documentclass[a4paper]{article}
\usepackage{minimum,tableau,ifthen} 
\usepackage{verbatim}

\setlength{\TabTitreL}{1cm}

\begin{document}

\begin{figure}[h]
\begin{center}
\begin{MonTableau}{2}{9}{1.2} 

\TabNewCol{0} 
\rTabPut{Br}{-.15}{.25}{$x$}
\rTabPut{Br}{-.15}{.40}{$f'(x)$}
\rTabPut{Br}{-.15}{.40}{$f(x)$} 

\TabNewCol{0} 
\rTabPut{Bl}{.15}{.25}{$0$} 
\rTabPut{Bl}{.35}{.30}{}
\rTabPut{Bl}{.35}{.75}{$+\infty$} 

\psline[style=TabDblBarre](.025,0)(.025,2)

\TabNewCol{.3}
\rTabPut{B}{0}{.25}{$\frac{1}{2k}$} 
\rTabPut{B}{0}{.40}{$0$}
\rTabPut{B}{0}{.2}{$\frac{\ln2k}{2}$} 

\TabNewCol{1} 
\rTabPut{Br}{-.15}{.30}{$+\infty$}
\rTabPut{Br}{-.15}{.50}{} 
\rTabPut{Br}{-.15}{.75}{$+\infty$}

\TabFleche{B2}{C2}
\TabFleche{C2}{D2}

\rput(.15,.5){$-$}\rput(.65,.5){$+$}

\end{MonTableau}
\end{center}
\end{figure}

\begin{figure}[h]
\begin{center}
\begin{MonTableau}{2}{9}{1.2} 

\TabShowLabelOn              % ce qui change toute la pr�sentation

\TabNewCol{0} 
\rTabPut{Br}{-.15}{.25}{$x$}
\rTabPut{Br}{-.15}{.40}{$f'(x)$}
\rTabPut{Br}{-.15}{.40}{$f(x)$} 

\TabNewCol{0} 
\rTabPut{Bl}{.15}{.25}{$0$} 
\rTabPut{Bl}{.35}{.30}{}
\rTabPut{Bl}{.35}{.75}{$+\infty$} 

\psline[style=TabDblBarre](.025,0)(.025,2)

\TabNewCol{.3}
\rTabPut{B}{0}{.25}{$\frac{1}{2k}$} 
\rTabPut{B}{0}{.40}{$0$}
\rTabPut{B}{0}{.2}{$\frac{\ln2k}{2}$} 

\TabNewCol{1} 
\rTabPut{Br}{-.15}{.30}{$+\infty$}
\rTabPut{Br}{-.15}{.50}{} 
\rTabPut{Br}{-.15}{.75}{$+\infty$}

\TabFleche{B2}{C2}
\TabFleche{C2}{D2}

\rput(.15,.5){$-$}\rput(.65,.5){$+$}

\end{MonTableau}
\end{center}
\end{figure}


\begin{verbatim}

%%%%%%%%%%%%%%%%%%%%%%%%%%%%%%%%%%%%%%%%%%
%%%%%%       D�but du tableau       %%%%%%
%%%%%%%%%%%%%%%%%%%%%%%%%%%%%%%%%%%%%%%%%%

\setlength{\TabTitreL}{1cm}             % r�gle la largeur de la 
                                        % colonne de gauche
										
\begin{MonTableau}{2}{9}{1.2}           % 2 lignes
                                        % 9cm largeur utile
                                        % 1.2cm hauteur utile par case
										
\TabNewCol{0}                           % Colonne centr�e sur la barre de gauche
\rTabPut{Br}{-.15}{.25}{$x$}            % Br comme rput
\rTabPut{Br}{-.15}{.40}{$f'(x)$}        % -.15 d�calage horizontal en cm
\rTabPut{Br}{-.15}{.40}{$f(x)$}         % .25 ou .40 d�calage verticale en %
                                        % dans la case du tableau
										
\TabNewCol{0}                           % Colonne centr�e sur la barre de gauche
\rTabPut{Bl}{.15}{.25}{$0$}             % +.15 et +.35 d�calage horizontal en cm
\rTabPut{Bl}{.35}{.30}{}                % .25 et .30 d�calage verticale en %
                                        % dans la case du tableau
                                        % On met les lignes vides pour que les
\rTabPut{Bl}{.35}{.80}{$+\infty$}       % nodes s'incr�mentent

                                        % Une double barre
\psline[style=TabDblBarre](.025,0)(.025,2)


\TabNewCol{.3}                          % Colonne en .3
\rTabPut{B}{0}{.25}{$\frac{1}{2k}$}     % Valeur de x
\rTabPut{B}{0}{.40}{$0$}                % Annule la d�riv�e
\rTabPut{B}{0}{.2}{$\frac{\ln2k}{2}$}   % Valeur du minimum
                                        % d�calage vertical .2 permet de monter
                                        % et descendre l'�tiquette dans la case

\TabNewCol{1}                           % Colonne centr�e sur la barre de droite
\rTabPut{Br}{-.15}{.30}{$+\infty$}      % Je n'insiste pas vous avez certainement
\rTabPut{Br}{-.15}{.50}{}               % compris
\rTabPut{Br}{-.15}{.80}{$+\infty$}

\TabFleche{B2}{C2}                      % une fl�che entre les colonnes C et E
                                        % concernant la ligne 2
\TabFleche{C2}{D2}                      % une fl�che entre les colonnes E et G
                                        % concernant la ligne 2

\rput(.15,.5){$-$}\rput(.65,.5){$+$}    % On met le + et le -

\end{MonTableau}

\end{verbatim}

\newpage
\verbatiminput{Latex/tableau.sty}

\end{document}
