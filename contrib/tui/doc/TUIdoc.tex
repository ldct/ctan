% !TEX TS-program = pdflatex
% !TEX encoding = UTF-8 Unicode
%
\documentclass[11pt,
              article,
              oneside
              ]{memoir}

\usepackage{amssymb}
\usepackage[utf8]{inputenc}
\usepackage[T1]{fontenc}
\usepackage[spanish]{babel}
\usepackage[svgnames]{xcolor}

\usepackage[babel=true,
  verbose=false,
  tracking=true,
  expansion=true,
  protrusion=true
  ]{microtype}
\SetTracking{ encoding = *, shape = sc }{ 40 }
\SetTracking[context = notracking, ]{encoding = *}{0}

%--------------------------------------------------------------------------------
\clubpenalty=9996
\widowpenalty=9999
\brokenpenalty=4991
\predisplaypenalty=10000
\postdisplaypenalty=1549
\displaywidowpenalty=1602
\flushbottom\raggedbottom
\parindent=13pt

\usepackage[autostyle=false, style=usenglish]{csquotes}
\usepackage{xspace}
\usepackage[nott]{kpfonts}
\usepackage{graphicx}
\usepackage{verbatim}
\usepackage{hyperref}
\usepackage{paralist}

\renewcommand\thesection{\arabic{section}}



% Section style -----------------------------------
\setsecnumformat{\upshape\csname the#1\endcsname.\quad}
\setsecheadstyle{%
  \exhyphenpenalty=100000\hyphenpenalty=100000%
  \large\bfseries\scshape}
\setbeforesecskip{-2\onelineskip}
\setaftersecskip{.5\onelineskip}

% Subsection style -----------------------------------
\setsubsecheadstyle{%
  \exhyphenpenalty=100000\hyphenpenalty=100000%
  \noindent\normalsize\bfseries}
\setbeforesubsecskip{-2\onelineskip}
\setaftersubsecskip{.2\onelineskip}

% Subsubsection style -----------------------------------
\setsubsubsecheadstyle{%
  \exhyphenpenalty=100000\hyphenpenalty=100000%
  \noindent\normalsize\em\hspace*{13pt}}
\setbeforesubsecskip{-2\onelineskip}
\setaftersubsecskip{.2\onelineskip}

\setsecnumdepth{subsubsection}
\maxsecnumdepth{subsubsection}
\maxtocdepth{section}


\newcommand*{\elipsis}{\,.\,.\,.\,}
\newenvironment{cita}%
 {\begin{list}{}%
              {\setlength\rightmargin{0pt}%
              \setlength\leftmargin{\parindent}}%
  \item[]\small\ignorespaces}
 {\unskip\end{list}}


\newcommand\CTANurl[1]{\url{#1}}

\newcommand*{\comando}[1]{\texttt{#1}}

\newcommand*{\paquete}[1]{\texttt{\color{NavyBlue}#1}\xspace}
\newcommand*{\tui}{{\normalfont\paquete{tui}}\xspace}
\newcommand*{\MEMOIR}{\paquete{memoir}}

\newcommand*{\XeLaTeX}{X\hspace*{-2pt}\raisebox{-2pt}{\rotatebox[origin=c]{-180}{E}}\LaTeX\xspace}

\title{\LARGE\scshape La clase\,\ \tui\\[6pt]
  \large\itshape Tesis de la Facultad de Ingeniería\\
  \large de la Universidad de los Andes\\[12pt]
 \normalsize\normalfont\scshape Manual del usuario}
\author{\normalsize\normalfont\scshape Nicolás Vaughan\\%
  \normalsize\normalfont\scshape Balliol College, Oxford\\%
  \small\url{nivaca@fastmail.net}}
\date{\normalsize\normalfont v.1.9, 2012/07/27}

\frenchspacing


%===============================================================
\begin{document}
\vspace*{-2cm}\maketitle
\vspace*{-1cm}
\chapterstyle{article}
\renewcommand*{\printtoctitle}[1]{\large\bfseries\scshape #1}
\tableofcontents*


\section{Introducción}
\noindent
La clase \tui está diseñada para ser usada como base para las tesis de la Facultad de Ingeniería de la Universidad de los Andes escritas en \LaTeX. La clase está montada a su vez sobre la clase \MEMOIR~\cite{MEMOIR}, una de las más refinadas, potentes y actualizadas de \LaTeX\ y  \XeLaTeX. Se recomienda, por tanto, que el usuario de \tui tenga a mano el manual de \MEMOIR disponible sin costo alguno en línea.\footnote{\url{http://mirror.ctan.org/macros/latex/contrib/memoir/memman.pdf}} La versatilidad de la clase \tui, heredada de la de la clase \MEMOIR, se desprende del hecho de que ésta proporciona una buena cantidad de macros que antes eran proporcionados por distintos paquetes o clases, y que por ende debían ser cargados adicionalmente a la clase principal (\paquete{article}, \paquete{book}, \paquete{report}, etc.) Como es sabido por quien trabaja sobre la plataforma \TeX, muchas veces esos paquetes y clases son incompatibles entre sí, y el usuario debe entonces hallar complicados mecanismos para evitar dichas incompatibilidad, si ello fuere posible. La clase \MEMOIR evita ese problema. Entre los paquetes más usados en \LaTeX\ cuya funcionalidad es proporcionada por \MEMOIR están los siguientes: \paquete{fancyhdr}~\cite{FANCYHDR}, para modificar y dar formato a los encabezados y pies de página; \paquete{crop}~\cite{CROP}, para crear y dar formato a marcas de corte para impresión; \paquete{geometry}~\cite{GEOMETRY}, para cambiar el formato y tamaño de la página y de los márgenes; \paquete{sidecap}~\cite{SIDECAP}, para incluir las etiquetas de figuras y tablas hacia los lados; y \paquete{subfigure}~\cite{SUBFIGURE}, para proveer soporte para subfiguras y subtablas. La lista completa de paquetes y clases cuya funcionalidad ha sido proveída o \enquote{simulada} por \MEMOIR puede consultarse en el manual de usuario de dicha clase.

Así las cosas, el usuario de nuestra clase \tui \textbf{no} requerirá cargar esos paquetes en el preámbulo del archivo \texttt{.tex} principal de su tesis; asimismo, no tendrá que preocuparse por especificar los detalles \enquote{formales} del documento (márgenes, tamaño de página, estilo de capítulos y sección, estilos de los encabezados, etc.), por cuanto éstos ya están definidos internamente por \tui.

Por otro lado, sugerimos el uso de una fuente tipográfica para \LaTeX---la fuente Kepler, proveída por el paquete \paquete{kpfonts}, del Johannes Kepler Project~\cite{KPFONTS}, la cual, además de ser altamente legible y estéticamente atractiva, se integra perfectamente a \LaTeX. El usuario no tendrá que instalarla---proceso tremendamente dispendioso en \LaTeX---ya que toda distribución reciente de \TeX\ la debe incluir.\footnote{En \texttt{MikTeX} será instalada automáticamente por el administrador de paquetes, en caso de que no esté ya en el sistema local.} En caso de que exista alguna incompatibilidad entre \paquete{kpfonts} y alguno de los paquetes que el usuario requiera (e.g., paquetes de símbolos matemáticos), puede desactivarse con la opción \comando{kpfonts=false}, como se detallará más adelante. O también podrá escoger, con la opción correspondiente al cargar la clase \tui, la fuente Times, provista por el paquete \paquete{mathptmx}~\cite{TIMES} o, en su defecto, con la fuente de Computer Modern (CM) de Donald Knuth, característica de \TeX. (No se recomienda esta opción, ya que la fuente CM no provee \textsc{\bfseries versalitas en negrita}, usadas en los títulos de sección, etc.)

Nuestra clase está diseñada para funcionar con cualquier distribución reciente (actualizada) de \LaTeX\ (i.e., \paquete{MikTeX}, \paquete{TeXLive} o \paquete{MacTeX}).



\section{Clases y paquetes requeridos}
\noindent
La clase \tui requiere los siguientes paquetes y clases, todos disponibles en \paquete{CTAN}%
\footnote{\url{http://www.ctan.org}}
  e incluidos en cualquier distribución reciente de \TeX:
\medskip  
\begin{compactenum}
  \item Clase \tui (provista aquí)
  \item Clase \MEMOIR [2011/03/06]
  \item \paquete{amsmath} [2000/07/18] --- soporte matemático de \LaTeX
  \item \paquete{amssymb} [2009/06/22] --- símbolos matemáticos básicos
  \item \paquete{amsthm} [2004/08/06] --- redefinición de teoremas
  \item \paquete{hyperref} [2011/03/09] --- manejo de hipervínculos
  \item \paquete{makeidx} --- índices analíticos, etc.
  \item \paquete{xcolor} [2007/01/21] --- soporte avanzado para color
  \item \paquete{graphicx} [1999/02/1] --- soporte para gráficas
  \item \paquete{makeidx} --- para creación de índices (analíticos, etc.)
  \item \paquete{kvoptions} [2010/02/22] --- para opciones avanzadas de la clase
\end{compactenum}



\section{Paquetes opcionales}
\noindent
Opcionalmente, la clase \tui puede llamar a uno o varios de los siguientes paquetes, todos disponibles también en \paquete{CTAN} e incluidos en cualquier distribución reciente de \TeX:
\medskip  
\begin{compactenum}
  \item \paquete{MnSymbol} [2007/01/21] --- uso de la fuente Minion Symbol en modo matemático
  \item \paquete{kpfonts} [2010/08/20] --- uso de la fuente Kepler
  \item \paquete{mathptmx} [2005/04/1]  --- uso de la fuente Times
  \item \paquete{mathpazo} [2005/04/12]  --- uso de la fuente Palatino
\end{compactenum}



\section{Opciones de clase}
\renewcommand*{\tablename}{Tabla}
\noindent
La tabla \ref{tab:opciones} detalla las opciones provista por la clase \tui. Las opciones chuleadas son seleccionadas por defecto.
\begin{table}
  \begin{center}
    \begin{tabular}{r|c}
      \textit{Opción} & \textit{Por defecto}\\
      \hline
    \texttt{noblancas} &  \\
    \texttt{spanish} & $\checkmark$ \\
    \texttt{english} &  \\
    \texttt{draft} &  \\
    \texttt{publish} &  \\
    \texttt{microtype} & $\checkmark$ \\
    \texttt{kpfonts} & $\checkmark$ \\
    \texttt{times} &  \\
    \texttt{grande} &  \\
    \texttt{mnsymbol} & $\checkmark$ \\
      \hline
    \end{tabular}
    \caption{Opciones de la clase \tui}\label{tab:opciones}
  \end{center}
\end{table}

\medskip\noindent
Sintaxis: \verb+\documentclass[+%
  \textit{\color{DarkRed}opción\textsubscript{1}\,\color{NavyBlue}, \color{DarkRed}opción\textsubscript{2}\color{NavyBlue}\,, .\,.\,.\,}%
  \verb+]{tui}+

\medskip  

\begin{compactenum}
  \item \texttt{noblancas} --- Elimina las hojas blancas entre partes del documento.
  \item \texttt{spanish} --- Selecciona español como el idioma del documento (por defecto).
  \item \texttt{english} --- Selecciona inglés como el idioma del documento.
  \item \texttt{draft} --- Opción borrador (muestra particiones problemáticas, no compone imágenes, etc.).
  \item \texttt{publish} --- Incluye marcas de corte y la sección de \enquote{Descripción de la colección}.
  \item \texttt{microtype} --- Monta el paquete \texttt{microtype} para ajustes tipográficos avanzados (por defecto).
  \item \texttt{kpfonts} --- Carga el paquete de fuentes \texttt{kpfonts} (por defecto).
  \item \texttt{times} --- Carga el paquete de fuentes \texttt{mathptmx} para la fuente Times New Roman.
  \item \texttt{grande} --- Determina el tamaño de la fuente de 11pt (en lugar de 10pt), para tesis de menos de 150 páginas.
  \item \texttt{mnsymbol} --- Carga el paquete de símbolos matemáticos \texttt{mnsymbol}.
\end{compactenum}

\medskip\noindent
Por ejemplo, si quisiéramos hacer una tesis en inglés, lista para publicación (con marcas de corte, etc.), pero sin cargar la fuente Minion Symbol, deberíamos usar el siguiente comando en el preámbulo del archivo principal:

\bigskip\noindent
\verb+ \documentclass[english, publish, mnsymbol=false]{tui} +

\bigskip\noindent
O si quisiéramos hacer una tesis en español, para no ser publicada (digamos, para lectura de los jurados, etc.), usando la fuente CM, deberíamos usar el siguiente comando en el preámbulo del archivo principal:

\bigskip\noindent
\verb+ \documentclass[nofonts=true]{tui} +

\bigskip\noindent
Si nuestra tesis es de menos de 150 páginas, deberemos escoger una fuente de tamaño un poco más grande. Para ello definimos así el preámbulo de nuestro archivo principal:%
	\footnote{Buscando uniformidad, las opciones 10pt, 11pt, 12pt, etc., \emph{no} están soportadas por nuestra clase. Únicamente ofrecemos la opción \texttt{grande} (o equivalentemente \texttt{grande=true}) y \texttt{grande=false}, que respectivamente determinan un tamaño de fuente principal de 11pt y 10pt.}

\bigskip\noindent
\verb+ \documentclass[grande]{tui} +


\bigskip\noindent
Y para hacer una tesis en español, para no ser publicada, usando las fuentes Kepler, la fuente Minion Symbol y las mejoras microtipográficas de \paquete{microtype}, simplemente cargamos la clase así:

\bigskip\noindent
\verb+ \documentclass{tui} +

\bigskip\noindent
Finalmente, notemos que las opciones \texttt{kpfonts}, \texttt{times}, y \texttt{palatino} son incompatibles entre sí, por cuanto en \LaTeX\ no es posible usar más de una fuente principal para los documentos.\footnote{Ello sí es posible, empero, usando  \XeLaTeX, pero la clase \tui no ha sido diseñada teniendo en cuenta esa posibilidad.}



\section[Algunas recomendaciones y sugerencias]{Algunas recomendaciones y sugerencias\\ para el uso de la clase \tui}
\begin{compactenum}
  \item El archivo de la clase \tui (\comando{tui.cls}) debe estar ubicado donde \LaTeX\ pueda encontrarlo. Lo más sencillo es dejarlo en el mismo directorio donde están los archivos \comando{.tex} de la tesis.
  \item La clase \tui es provista a los usuarios con una \textit{plantilla}, i.e., una serie de archivos \comando{.tex} que conforman la estructura de una tesis. Se sugiere expresamente que el usuario trabaje sobre esos archivos.
  \item La clase \tui provee el formato de todos los elementos de la tesis, desde el tamaño y formato de página, hasta los el estilo de los encabezados, títulos de capítulo, sección, etc. Por tal razón, el usuario \textbf{no} deberá cambiar nada del aspecto formal de la tesis en su documento. Es decir, no deberá incluir fuentes adicionales (salvo las que incluyan símbolos matemáticos o similares, que sean necesarios para el contenido de la tesis), ni especificar características formales propias de la plantilla. La única excepción, sin embargo, concierne a elementos propios de la tesis que no hayan sido previstos por la plantilla.  E.g., una tabla de listados (en una tesis de ingeniería de sistemas). La inclusión de tal tabla debe seguir los estándares de la clase \tui (usando el formato de la misma mediante los comandos provistos por la clase \MEMOIR). En caso de duda, se sugiere que el usuario contacte a la Sra. Carolina Mazo de Ediciones Uniandes (\url{cmazo@uniandes.edu.co}).
  \item La opción \comando{draft} de la clase \tui no sólo acelera la compilación (el montaje tipográfico) de los archivos \LaTeX\ (por cuanto no monta gráficas, etc.), sino que muestra con un rectángulo negro los lugares en que hay problemas de partición de palabras al final de los renglones. \LaTeX\ tiene una serie elaborada de algoritmos para la justificación de párrafos y la partición de palabras que, sin embargo, con frecuencia no funciona como uno esperaría. (El \enquote{log} de compilación nos muestra mensajes de advertencia de \comando{Overflow hbox}.) Para ello hay que analizar caso por caso, y decidir si hay que introducir una variante de partición de palabra (con el comando \verb+\hyphenate{...}+ en el preámbulo del archivo principal, o en el archivo \texttt{hyphenation.tex} usado en nuestra plantilla) o si hay que reformular la redacción.
  \item Si la tesis está pensada para publicación, y en general como buena práctica, las imágenes y gráficas que se importen en \LaTeX\ (usando el comando \verb+\includegraphics{...}+) deben tener mínimamente la siguiente resolución: 300\,dpi para imágenes a color y 600\,dpi para imágenes en escala de grises o en blanco y negro.
  \item Tenga en cuenta que la impresión para publicación que hace la Universidad no es en colores (salvo para la carátula del libro). Eso implica que las gráficas e imágenes que hayan sido incluidas en color serán impresas en escala de grises. Ello puede llevar a confusiones muchas veces en la interpretación de las gráficas (e.g., una línea verde se ve igual que una línea azul, cuando son impresas en escala de grises). El usuario debe considerar si es posible buscar otros medios para etiquetar las gráficas, de tal suerte que no use colores. O, si son inevitables, una opción es hacer disponibles las gráficas e imágenes en color, u otro material de soporte, en algún sitio de internet (e.g., la página de la Facultad, una página web personal, un servidor gratuito de imágenes como \url{http://www.flickr.com/}) al que la el texto de la tesis refiera en su momento.
  \item Asimismo, cuando requiera especificar tonos de grices, esto debe hacerse escogiendo o bien el espacio de color CMYK, o bien el espacio de color Gray. (Para los detalles, consulte el manual del paquete \paquete{xcolor} \cite{XCOLOR}.)
  \item Por defecto (y en aras de la universalidad), la codificación de caracteres de los archivos de la tesis es Latin 1 (ISO-8859-1). La clase \tui determina esa codificación mediante el paquete \paquete{inputenc}.%
	\footnote{El comando es \texttt{\textbackslash{}RequirePackage[latin1]{inputenc}[2008/03/30]}.}
	Si fuere necesario, es posible usar otras codificaciones (e.g., Unicode UTF8), para lo cual hay que modificar el archivo de la plantilla.%
	\footnote{Se comenta con el signo\ \ \% el comando \texttt{\textbackslash{}RequirePackage[latin1]{inputenc} [2008/03/30]}, y se le quita el signo\ \ \% a los comandos \texttt{\textbackslash{}RequirePackage[utf8]{inputenc} [2008/03/30]} y \texttt{\textbackslash{}RequirePackage[T1]{fontenc}[2005/09/27]}.}
	En todo caso, la codificación de todos los archivos de texto de la tesis debe ser la misma (Latin 1, UTF8, etc.).
  \item Cualquier editor de texto plano (\textit{plain text}) es más que suficiente para trabajar con archivos de \LaTeX. Existen editores específicos para \TeX\ y sus \enquote{hijos} (\LaTeX,  \XeLaTeX, etc.), que ayudan enormemente en el proceso de edición, al incluir una serie de herramientas y macros diseñadas para \TeX. Entre los más famosos están WinEdt (sólo para Windows, propietario), TeXShop (sólo para Mac OS X, sin costo), TeXworks (Windows, Linux y Mac OS X, sin costo), TeXnicCenter (Windows, sin costo) y TexMakerX (Windows, Linux y Mac OS X, sin costo). Asimismo, los editores de texto para programadores generalmente ofrecen soporte para editar inteligentemente archivos de \LaTeX, por ejemplo Emacs (con el paquete Auctex), Vi, gedit, Notepad++, Bluefish, BBEdit, EmEditor, TextMate y jEdit. Cualquiera de ellos está en capacidad de manejar los archivos que el usuario requerirá para la escritura de su tesis.
  \item No obstante lo anterior, los archivos \comando{.tex} \textbf{no} pueden ser editados con procesadores de palabras como Word, Writer, Pages etc. \textbf{Tampoco} pueden editarse con editores \textsc{wysiwym} \textit{basados} en \LaTeX, tales como Lyx o Scientific WorkPlace.
  \item La persona encargada de la corrección de estilo de su tesis (en Ediciones Uniandes) le indicará---entre otras cosas---errores typográficos, tales como viudas y huérfanas.%
	\footnote{Una \emph{viuda} es la última línea de un párrafo que ha quedado sola empezando una página. Análogamente, una \emph{huérfana} es la primera línea de un párrafo que ha quedado sola al final de una página.}
Estos errores deben corregirse \emph{luego} de haber realizado todos los otros cambios, ya que la paginación del documento final seguramente cambiará. La manera más fácil de solucionar problemas de viudas y/o huérfanas es mediante el comando \verb+\enlargethispage{+\emph{tamaño}\verb+}+, proveido por la clase \MEMOIR. Lo que hace es agrandar la página actual según el tamaño indicado en el argumento. E.g., 
\verb+\enlargethispage{\baselineskip}+ agrandará la página lo suficiente para que quepa una línea completa adicional de texto normal. (Para detalles sobre su utilización, cf. \cite{MEMOIR}, \S\,3.5, pp.\, 50ss.)
	\item Otro error tipográfico común consiste en líneas que se salen de \hbox{la caja de texto} (como este ejemplo). Esto sucede por muchas razones, aunque la más común es que \LaTeX---o más exactamente \comando{babel}---no tenga en su base de datos de división de palabras la palabra en cuestión. Para ello puede incluir un `guión opcional' en la palabra (e.g., \comando{tex\-to}), donde se requiera la partición. Con todo, se recomienda incluirla en el archivo \comando{hyphenation.tex} (suministrada con la plantilla de la clase \tui), siguiendo la sintaxis requerida, y de acuerdo con las reglas ortográficas de idioma en uso (e.g., \comando{tex-to).} \LaTeX{} indicada este tipo de error mediante un mensaje de advertencia (\emph{warning}) en compilación como el siguiente:\\[-6pt]
\begin{small}
\begin{verbatim}
Overfull \hbox (21.5pt too wide) in paragraph at lines 256--257
\end{verbatim}
\end{small}~\\[-6pt]
%
Para facilitar la detección de este error, \LaTeX{} puede indicarlo mediante un rectángulo negro (como este \rule[-3pt]{4mm}{4mm}\,) al final de la línea en cuestión. Para ello debe seleccionar la opción \comando{draft} cuando cargue la clase \tui (e.g., \verb+\documentclass[publish,english,draft]{tui}+).
	\item Muchas veces, sin embargo, la solución de la partición de palabras no es suficiente para resolver el problema. Otra posible solución consiste en relajar las restricciones de justificación de párrafos de \LaTeX. Esto solo debe hacerse localmente, i.e., párrafo por párrafo. \MEMOIR (y por tanto \tui) provee los ambientes \comando{sloppypar} y \comando{midsloppypar} para ello. (Para ello, cf. \cite{MEMOIR}, \S\,3.4, p.\,88.) La mayoría de errores de este tipo pueden solucionarse en este modo.
  \item La clase \tui, montada sobre la clase \MEMOIR, provee una serie de comandos robustos para las diferentes divisiones de la tesis (capítulos, secciones, subsecciones, partes, etc.). A veces el título de alguna división es bastante largo, lo cual hace que no quepa completo en el encabezado de la página. Para ello existe la siguiente sintaxis avanzada:\\[6pt]
      {\small%
      \verb+\section+%
        \verb+[+%
        \itshape{título para el índice general}%
        \verb+][+%
        \itshape{título para el encabezado}%
        \verb+]{+%
        \itshape{título para el cuerpo del texto}%
        \verb+}+%
       }
       \ \\[6pt]
       \noindent Los primeros dos parámetros son opcionales, el tercero es obligatorio. Pero si se usa el primero, es necesario incluir el segundo, y viceversa. Esto es sumamente útil para trabajar con títulos de capítulos o secciones muy largos, por ejemplo:\\[6pt]
       {\small%
       \verb+\chapter+%
        \verb+[+%
        {Probabilistic fatigue analysis by heterogeneity in materials and the fatigue phenomena}%
        \verb+][+%
        {Probabilistic fatigue analysis\,.\,.\,.}%
        \verb+]{+%
        {Probabilistic fatigue analysis by heterogeneity in materials and the fatigue phenomena}%
        \verb+}+%
       }
       \ \\[6pt]
       \noindent En este caso el encabezado de página tendrá una versión abreviada del título, mientras que el índice general (\textsc{toc}) y el cuerpo del texto tendrán la versión completa. (Una explicación detallada de la sintaxis de estos comandos la puede encontrar el usuario en la \S\ 6.2 de \cite{MEMOIR}.)
	\item Como ya se indicó arriba, la clase \MEMOIR provee la funcionalidad de numerosas otras clases y paquetes, e.g., \paquete{ccaption}, \paquete{tocloft}, \paquete{fancyhdr}, etc. El autor y el administrador de \MEMOIR se han esforzado en evitar incompatibilidades con la mayoría de paquetes importantes de \LaTeX{}; con todo, es imposible prever todos los conflictos. Sin embargo, cuando aparece un choque con algún paquete, la solución del problema suele ser bastante simple (e.g., cargar antes o después el paquete, redefinir un comando, etc.). Se recomienda al usuario revisar la documentación de la clase \MEMOIR para cualquier eventualidad. 
	\item De igual manera, los paquetes de fuentes \paquete{kpfonts} y \paquete{MnSymbol}, cargados por defecto por la clase \tui, pueden ser incompatibles con otras fuentes de símbolos que el usuario esté utilizando en su tesis. En caso tal, el usuario debe revisar la documentación de dichos paquetes para establecer si el símbolo o los símbolos en conflicto están ya definidos por ellos, en cuyo caso se hace innecesaria la utilización del paquete en conflicto. Si esto no es posible, puede probarse cargando el paquete de símbolos \emph{antes} de cargar los paquetes \paquete{kpfonts} y \paquete{MnSymbol}, editando el archivo de la clase \tui. Si esto no funcionara, es posible deshabilitar uno o ambos paquetes mediante las opciones proveídas por la clase. (Asimismo, en \cite{SYMBOLS} el usuario encontrará consejos sobre cómo lidiar con dichas incompatibilidades.)
	\item El autor de la clase \tui sugiere los siguientes títulos como referencia útil en el manejo de \LaTeX: \cite{COMPANION}, \cite{GOOSSENS94}, \cite{KOPKA} y especialmente \cite{UNIVERSO}.
\end{compactenum}



% Referencias -----------------------------------------
\renewcommand*{\bibname}{Referencias}
\begin{thebibliography}{GMS94A}
\small

\bibitem[Wil11]{MEMOIR}
  Peter Wilson.
  \newblock \emph{The memoir class}.
  \newblock Marzo de 2011.
  \newblock (Disponible en CTAN vía
             \CTANurl{/macros/latex/contrib/memoir})

\bibitem[Oos05]{FANCYHDR}
  Piet van Oostrum.
  \newblock \emph{Page Layout in LaTeX}.
  \newblock Marzo de 2005.
  \newblock (Disponible en CTAN vía
             \CTANurl{/macros/latex/contrib/fancyhdr})

\bibitem[Fra05]{CROP}
  Melchior Franz.
  \newblock \emph{The Crop Package}.
  \newblock Mayo de 2003.
  \newblock (Disponible en CTAN vía
             \CTANurl{/macros/latex/contrib/fancyhdr})

\bibitem[Ume99]{GEOMETRY}
  Hideo Umeki.
  \newblock \emph{The geometry package}.
  \newblock Noviembre de 1999.
  \newblock (Disponible en CTAN vía
            \CTANurl{/macros/latex/contrib/geometry/})

\bibitem[Bez99]{TITLESEC}
  Javier Bezos.
  \newblock \emph{The titlesec and titletoc packages}.
  \newblock Noviembre de 1999.
  \newblock (Disponible en CTAN vía
             \CTANurl{/macros/latex/contrib/titlesec/})

\bibitem[NG98]{SIDECAP}
  Rolf Niespraschk and Hubert G\"{a}\ss{}lein.
  \newblock \emph{The sidecap package}.
  \newblock Junio de 1998.
  \newblock (Disponible en CTAN vía
            \CTANurl{/macros/latex/contrib/sidecap/})

\bibitem[Coc02]{SUBFIGURE}
  Steven Douglas Cochran.
  \newblock \emph{The subfigure package}.
  \newblock Marzo de 2002.
  \newblock (Disponible en CTAN vía
             \CTANurl{/macros/latex/contrib/subfigure})

\bibitem[Uwe07]{XCOLOR}
  \newblock \emph{The Xcolor Package}.
  \newblock Enero de 2007.
  \newblock (Disponible en CTAN vía
             \CTANurl{/macros/latex/contrib/xcolor/})

\bibitem[Cai11]{KPFONTS}
  Christophe Caignaert.
  \newblock \emph{The subfigure package}.
  \newblock Marzo de 2011.
  \newblock (Disponible en CTAN vía
             \CTANurl{/fonts/kpfonts})

\bibitem[SPQ05]{TIMES}
  \newblock \emph{Times w/ Math, improved (SPQR, WaS)}.
  \newblock Abril de 2005.
  \newblock (Disponible en CTAN vía
             \CTANurl{/fonts/psfonts/psnfss-source/mathptmx})

\bibitem[Pug05]{PALATINO}
  \newblock \emph{Palatino w/ Pazo Math (D.Puga, WaS)}.
  \newblock Abril de 2005.
  \newblock (Disponible en CTAN vía
             \CTANurl{/fonts/psfonts/psnfss-source/mathpazo})

\bibitem[Pak09]{SYMBOLS}
	Scott Pakin.
  \newblock \emph{The Comprehensive \LaTeX Symbol List}.
  \newblock Noviembre de 2009.
  \newblock (Disponible en CTAN vía
             \CTANurl{/info/symbols/comprehensive/})

\bibitem[GMS94]{GOOSSENS94}
  Michel Goossens, Frank Mittelbach and Alexander Samarin.
  \newblock \emph{The \LaTeX\ Companion}.
  \newblock Addison-Wesley Publishing Company, 1994
  \newblock ISBN 0201541998.

\bibitem[MG\textsuperscript{+}04]{COMPANION}
  Frank Mittelbach, Michael Goossens, et al.
  \newblock \emph{The \LaTeX\ Companion: Second Edition}.
  \newblock Addison-Wesley, 2004.
  \newblock ISBN 0201362996.

\bibitem[Kop2004]{KOPKA}
  Helmut Kopka and Patrick W. Daly
  \newblock \emph{A Guide to \LaTeX: Tools and Technologies for Computer Typesetting. Fourth Edition}.
  \newblock Addison-Wesley, 2004.
  \newblock ISBN 9780321173850.

\bibitem[Dec2003]{UNIVERSO}
  Rodrigo De Castro Korgi
  \newblock \emph{El universo \LaTeX. Segunda edición}.
  \newblock Bogotá: Universidad Nacional, 2003.
  \newblock ISBN 9780321173850

\end{thebibliography}

\end{document}% =============================================================
\endinput 