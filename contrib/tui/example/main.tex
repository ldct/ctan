% !TEX root = main.tex
% !TEX encoding = Windows Latin 1
% !TEX TS-program = pdflatex
% arara: pdflatexmk
%
% Plantilla para las tesis de ingenieria de la Universidad de los Andes
% 2012/07/27, v.1.9

% Para las opciones de clase, ver la documentacion
\documentclass[publish,english]{tui}
\checklanguage

%\include{personal} % Incluya aqui el archivo con su comandos personales y los paquetes adicionales que quiera cargar.

%
% bgteubner class bundle
%
% hyphenation.tex
% Copyright 2003--2012 Harald Harders
%
% This program may be distributed and/or modified under the
% conditions of the LaTeX Project Public License, either version 1.3
% of this license or (at your opinion) any later version.
% The latest version of this license is in
%    http://www.latex-project.org/lppl.txt
% and version 1.3 or later is part of all distributions of LaTeX
% version 1999/12/01 or later.
%
% This program consists of all files listed in manifest.txt.
\hyphenation{Unix Micro-soft Da-tei-endung}
 % Excepciones de guiones (editar archivo)


% --------------------------------------------
% En caso de NO usar indice analitico (no recomendado),
% comentar esta seccion:
\makeatletter
  \iftui@draft\relax
    \else\makeindex % indice analitico
  \fi
\makeatother

%----------------------------------------------

\begin{document}% =========================================
\setupeverything
\frontmatter % Los preliminares de la tesis
\pagestyle{empty}

% !TEX root = main.tex
% !TEX encoding = Windows Latin 1
% !TEX TS-program = pdflatex
%
% portada.tex (portadilla y portada de la tesis)

% +++++++++++++++++++++++ Portadilla +++++++++++++++++++++++++++++
%
\pagestyle{empty}
\dosenblanco

% Portadilla
\begin{center}
  \scshape%
  % Aqui va el titulo de la tesis----------------------------------
  Este es el t\'{i}tulo \\
  de la tesis doctoral \\
  % ---------------------------------------------------------------
\end{center}

\cleardoublepage

% +++++++++++++++++++++++ Portada +++++++++++++++++++++++++++++
%
\begin{center}
{%
  \LARGE\scshape\bfseries%
  % Aqui va el titulo de la tesis ---------------------------
  Este es el t\'{i}tulo \\ [6pt]
  de la tesis doctoral \\ [6pt]
  % ---------------------------------------------------------
}

\vfill

{
 \large\scshape%
  A dissertation\\
  by\\
  % Aqui va el autor de la tesis-----------------------------
  Este es el autor de la tesis
  % ---------------------------------------------------------
}

\vfill

{%
  % Datos de la Universidad, el titulo, etc. ------------------
  {\normalfont Submitted to the School of Engineering of the}\\[3pt]
  {\large\scshape Universidad de los Andes}\\[3pt]
  {\normalfont in partial fulfillment for the requirements for the Degree of}\\[3pt]
  {\large\scshape Doctor in Engineering}
  % -------------------------------------------------------------
}

\vfill

%----------------------- No tocar este codigo ----------------------
\makeatletter
  \iftui@spanish
    {\scshape Aprobada por:}
  \else
    {\scshape Approved by:}
  \fi
\makeatother
% ------------------------------------------------------------------


% Llenar esta informacion ------------------------------------------
\begin{tabbing}
  \hspace{6.5cm}\=\kill
  \hspace{1cm}Committee Chair:    \> {\itshape Dr. . . . } \\
  \hspace{1cm}Committee Members:  \> {\itshape Prof. . . . } \\
  			    											 \> {\itshape Prof. . . . } \\
                                  \> {\itshape Prof. Dr. . . . } \\
                                  \> {\itshape Prof. Dr. . . . } \\
                                  \> {\itshape Prof. Dr. . . . } \\
                                  \> {\itshape Prof. Dr. . . . } \\
                                  \> {\itshape Prof. Dr. . . . } \\
                                  
  \hspace{1cm}Dean School of Engineering: \> {\itshape Dr. Alain Gauthier} \\
  \hspace{1cm}Assistant Dean: \> {\itshape Dr. Rubby Casallas}
\end{tabbing}
% --------------------------------------------------------------------

\vfill

{%
  \normalsize\scshape
  % Fecha y area --------------------------------------------------
  February 2011\\
  {\scshape%
  Field: Ingenier\'{i}a de Sistemas y Computaci\'{o}n
  % ---------------------------------------------------------------
  }
}

\end{center}



% Fin de portada.tex
\endinput  % Portadilla y portada (editar archivo)

\makeatletter
\iftui@publish
  % !TEX root = main.tex
% !TEX encoding = Windows Latin 1
% !TEX TS-program = pdflatex
%
% plegal.tex (Pagina legal de la tesis)


\hbox{}\thispagestyle{empty}

\vspace*{-2cm}
\begin{center}
\noindent
\fbox{\parbox{\textwidth}{%
\scriptsize
 \vspace*{3cm}

 [Informaci{\'o}n p{\'a}gina legal]

 \vspace*{3cm}
}}

\end{center}

\vfill


{%
\parindent=0pt
\scriptsize

Primera edici{\'o}n:  Febrero de 2010\\

\copyright\ [Nombre del autor]\\
\url{email@del.autor}

\medskip

\copyright\ Universidad de los Andes, Facultad de Ingenier\'{\i}a\\
\phantom{\copyright\ }Departamento de Ingenier\'{\i}a * * * * * * * * * * * * \\

\bigskip

Ediciones Uniandes\\ Carrera 1\textsuperscript{a} No. 19-27. Edificio AU 6\\
Bogot{\'a} \textsc{DC}, Colombia\\
Tel{\'e}fono: 339 49 49 / 339 49 99. Ext: 2133. Fax: ext. 2158\\
\url{http://libreria.uniandes.edu.co}\\
\url{infeduni@uniandes.edu.co}

\bigskip

ISBN: [N{\'u}mero del ISBN]

\bigskip

Correcci{\'o}n de estilo en ingl{\'e}s: xxxxxxxxxxxxxx

\bigskip

Dise{\~n}o de car{\'a}tula: xxxxxxxxxxxxxx

\bigskip

Dise{\~n}o de colecci{\'o}n: Nicol{\'a}s Vaughan

\bigskip

Impresi{\'o}n: [Nombre editorial] \\ \
[Direcci{\'o}n] \\ \
Bogot{\'a} \textsc{DC}, Colombia

}

\vfill

\begin{tiny}
\begin{flushleft}
\noindent  Impreso en Colombia -- Printed in Colombia\\

\noindent Todos los derechos reservados. Esta publicaci{\'o}n no puede
ser reproducida ni en su todo ni en sus partes, ni registrada en o
trasmitida por un sistema de recuperaci{\'o}n de informaci{\'o}n, en ninguna
forma ni por ning{\'u}n medio sea mec{\'a}nico, fotoqu\'{\i}mico, electr{\'o}nico,
magn{\'e}tico, electro-{\'o}ptico, por fotocopia o cualquier otro, sin el
permiso previo por escrito de la editorial.
\end{flushleft}
\end{tiny}

% Fin de plegal.tex
\endinput  % Pagina legal (editar archivo)
  % !TEX root = main.tex
% !TEX encoding = Windows Latin 1
% !TEX TS-program = pdflatex
%
% Archivo: coleccion.tex (Descripcion de la coleccion de tesis)

% ------------  Importante: No editar este archivo ----------------------------

\selectlanguage{spanish}
\chapter*{Descripci{\'o}n de la colecci{\'o}n}
\thispagestyle{empty}
\noindent
Esta colecci{\'o}n re{\'u}ne los mejores trabajos de grado de maestr\'{\i}a y de doctorado de la Facultad de Ingenier\'{\i}a de la Universidad de los Andes. Con el {\'a}nimo de divulgar estos resultados de nuestros grupos de investigaci{\'o}n, la Facultad los pone a disposici{\'o}n de la comunidad acad{\'e}mica.

\bigskip

\noindent
Decano, Alain Gauthier Sellier; Vicedecana de Posgrado e Investigaci{\'o}n, Rubby Casallas Guti{\'e}rrez; Vicedecano de Pregrado, Rafael G{\'o}mez D\'{\i}az; Vicedecano para el Sector Externo, Gonzalo Torres Cadena; Secretaria General, Claudia C{\'a}rdenas Guti{\'e}rrez; Directores de Departamento: de Ingenier\'{\i}a Civil y Ambiental, Arcesio Lizcano Pel{\'a}ez; de El{\'e}ctrica y Electr{\'o}nica, Roberto Bustamante Miller; de Industrial, Roberto Zarama Urdaneta; de Mec{\'a}ni\-ca, {\'E}dgar Alejandro Mara{\~n}{\'o}n Le{\'o}n; de Qu\'{\i}mica, {\'O}scar {\'A}lvarez Solano; de Sistemas y Computaci{\'o}n, Jorge Alberto Villalobos Salcedo.



\checklanguage
% Fin archivo coleccion.tex
\endinput 
 % Descripcion coleccion tesis (no editar archivo)
  \else\relax\fi
\makeatother

\cleardoublepage

% !TEX root = main.tex
% !TEX encoding = Windows Latin 1
% !TEX TS-program = pdflatex
% 
% agradec.tex (Agradecimientos de la tesis)

\thispagestyle{empty}

\ 

\vspace*{2cm}

{
\flushright\itshape

To me,\\
to myself,\\
and to no-one else.\\

}


\vfill




% Fin de dedicat.tex
\endinput  % Dedicatoria (opcional) (editar archivo)
% !TEX root = main.tex
% !TEX encoding = Windows Latin 1
% !TEX TS-program = pdflatex
% 
% agradec.tex (Agradecimientos de la tesis)

\Agradecimientos
\noindent








% Fin de agradec.tex
\endinput  % Agradecimientos (editar archivo)
%%% abstract
%%% ---------------------------------------------------------------------------
\begingroup

%%% english
%%% ...........................................................................
\addchap{Abstract}

\blindtext


%%% german
%%% ...........................................................................
\otherlanguage{ngerman}
\addchap{Kurzfassung}

\blindtext

\endgroup % Resumen ingles (editar archivo)
% !TEX root = main.tex
% !TEX encoding = Windows Latin 1
% !TEX TS-program = pdflatex
%
% Archivo: resumen.tex (en espanol)


\chapter{Resumen} % No cambiar el titulo
\selectlanguage{spanish}
\noindent
Duis tristique sollicitudin leo nec consequat. Praesent et dui convallis velit tincidunt fermentum. Mauris cursus purus at sem viverra sed imperdiet sapien imperdiet. Aliquam mattis, elit eget rutrum vulputate, tortor sem pulvinar justo, sit amet mollis felis sem at nibh. Donec malesuada, neque id interdum eleifend, arcu augue porta elit, nec tristique libero metus at massa. Fusce fringilla laoreet rhoncus. Suspendisse potenti. Phasellus dignissim sodales mauris at pharetra. Donec gravida fringilla velit ac rutrum.

Curabitur ornare lectus id diam molestie eu imperdiet nulla tempus. Maecenas vestibulum enim et dui ornare blandit. Vivamus fermentum faucibus viverra. Maecenas at justo sapien. Aenean rhoncus augue mattis purus rhoncus venenatis. Suspendisse metus felis, porttitor in varius in, vulputate at tortor. Aliquam molestie, turpis et malesuada porta, tortor sapien pharetra sapien, ac rhoncus quam dolor a sapien. Pellentesque varius laoreet enim ut auctor. Nullam nec ultricies nisi. Nullam porta lectus et ante consectetur posuere.


Curabitur ornare lectus id diam molestie eu imperdiet nulla tempus. Maecenas vestibulum enim et dui ornare blandit. Vivamus fermentum faucibus viverra. Maecenas at justo sapien. Aenean rhoncus augue mattis purus rhoncus venenatis. Suspendisse metus felis, porttitor in varius in, vulputate at tortor. Aliquam molestie, turpis et malesuada porta, tortor sapien pharetra sapien, ac rhoncus quam dolor a sapien. Pellentesque varius laoreet enim ut auctor. Nullam nec ultricies nisi. Nullam porta lectus et ante consectetur posuere.


\bigskip
\noindent
\textit{Palabras clave:} palabra uno; palabra dos; palabra tres.

\checklanguage
% Fin archivo resumen.tex
\endinput  % Resumen espanol (editar archivo)


% Tabla de contenido, dependiendo del idioma
\tuitableofcontents

\cleardoublepage

\listoftables  % En caso de que haya tablas --------------
% (comentar en caso contrario)

\cleardoublepage

\listoffigures % En caso de que haya figuras --------------
% (comentar en caso contrario)

\cleardoublepage

%% !TEX root = main.tex
% !TEX encoding = Windows Latin 1
% !TEX TS-program = pdflatex
% 
% Archivo: listofsymbols.tex (Lista de simbolos)


\ListofSymbols
~
% ------------- Editar tabla -----------------------
\begin{center}
\begin{tabular}{rl}
  & \\
\end{tabular}
\end{center}

% ----------------------------------------------------


% Fin archivo listofsymbols.tex
\endinput % (editar archivo)--------------
% (comentar si no es usado)

\cleardoublepage

\chapter{The introduction}
Modern research on lexical access began in the 1950's (though
\cite{ColeRudnicky1983} note very similar research performed in the 1890's by
William Chandler Bagley).  Several statistical properties of the mental
lexicon have consistently been found to influence how humans process speech.
One of the earliest and most robust findings was that lexical frequency has a
strong influence on lexical access. Repeated research has shown that high
frequency words elicit quicker and more accurate responses than low
frequency words in a large variety of experimental conditions
(e.g.\ \cite{Broadbent1967, Taft1979, BenkiJASA}). Another factor which has
been reliably shown to affect lexical access is neighborhood density.
Neighborhood density is a metric of similarity, roughly defined as the degree
to which a word is similar to others (both phonological and orthographical
measures have been used). Words which have many similar words are said to be
in dense neighborhoods, whereas words which have few similar words are said to
be in sparse neighborhoods. In contrast to lexical frequency, which
facilitates the activation of a word in the brain, neighborhood density has
been found to inhibit activation (e.g.\ \cite{Luce1986, Luce1998, BenkiJASA,
Imai2005}). Of course these are not the only factors which affect language
processing, but they are the most frequently cited, and will be referred to
again in the following sections.
\begin{table*}[!htb]
  \centering
  \caption[Basic Predictions]{Basic Predictions: Predicted results are marked
  with a checkmark, and a relative effect size is also given.}
  \label{T:predictions}
  \begin{tabularx}{\textwidth}{%
    >{\setlength{\hsize}{1.5\hsize}\raggedright\arraybackslash}X%
    *{2}{>{\setlength{\hsize}{.7\hsize}\raggedright\arraybackslash}X}%
    *{2}{>{\setlength{\hsize}{1.05\hsize}\raggedright\arraybackslash}X}}
  \hline\hline
  \rule{0em}{1.1em}& English native listeners& German native listeners & 
       English non-native listeners & German non-native listeners\\[.3em]
  \cline{2-5}
  \rule{0em}{1.1em}lexical status & \checkmark robust& \checkmark
  robust& \checkmark less than native listeners& \checkmark less than native
  listeners\\
  morphology & marginal & more than English & less than L1& less than L1\\
  lexical frequency & \checkmark robust& \checkmark
  robust& \checkmark less than native listeners& \checkmark less than native
  listeners\\
  neighborhood density & \checkmark robust & \checkmark robust & \checkmark less
  than L1& \checkmark less than L1\\[.3em]
  \hline\hline
  \end{tabularx}
\end{table*}
 % Introduccion de la tesis (editar archivo)

% -----------------------------------------------------------
\mainmatter % Cuerpo de la tesis
\pagestyle{tui}

% !TEX root = main.tex
% !TEX encoding = Windows Latin 1
% !TEX TS-program = pdflatex
% 
% Archivo: ch01.tex (Capitulo 1)

\chapter{Bhis Is My First Chapter bla bla bla bla bla bla bla bla}
\section{Bhis Is My First Section bla bla bla bla bla bla bla bla bla bla bla bla }
\noindent
First sentence.\footnote{And this is my first footnote. And this is my first footnote. And this is my first footnote. And this is my first footnote. And this is my first footnote. And this is my first footnote. And this is my first footnote. And this is my first footnote. And this is my first footnote. And this is my first footnote. And this is my first footnote. And this is my first footnote. And this is my first footnote. And this is my first footnote. And this is my first footnote.}\index{Concepto}

\subsection{This Is My First Subsection}
\noindent
% !TEX root = main.tex
% !TEX encoding = Windows Latin 1
% !TEX TS-program = pdflatex
% 
Duis tristique \index{sollicitudin} sollicitudin leo nec consequat. Praesent et dui convallis velit tincidunt fermentum. Mauris cursus purus at sem viverra sed imperdiet sapien imperdiet. Aliquam mattis, elit eget rutrum vulputate, tortor sem pulvinar justo, sit amet mollis felis sem at nibh. Donec malesuada, neque id interdum eleifend, arcu augue porta elit, nec tristique libero metus at massa. Fusce fringilla laoreet rhoncus. Suspendisse potenti. Phasellus dignissim sodales mauris at pharetra. Donec gravida fringilla velit ac rutrum. Curabitur ornare lectus id diam molestie eu imperdiet nulla tempus. Maecenas vestibulum enim et dui ornare blandit. Vivamus \index{fermentum} faucibus viverra. Maecenas at justo sapien. Aenean rhoncus augue mattis purus rhoncus venenatis. Suspendisse metus felis, porttitor in varius in, vulputate at tortor. Aliquam molestie, turpis et malesuada porta, tortor sapien pharetra sapien, ac rhoncus quam dolor a sapien. Pellentesque varius laoreet enim ut auctor. Nullam nec ultricies nisi. Nullam porta lectus et ante consectetur posuere. \cite{texbook} 
\subsubsection{This Is My First Subsubsection}
% !TEX root = main.tex
% !TEX encoding = Windows Latin 1
% !TEX TS-program = pdflatex
% 
Duis tristique \index{sollicitudin} sollicitudin leo nec consequat. Praesent et dui convallis velit tincidunt fermentum. Mauris cursus purus at sem viverra sed imperdiet sapien imperdiet. Aliquam mattis, elit eget rutrum vulputate, tortor sem pulvinar justo, sit amet mollis felis sem at nibh. Donec malesuada, neque id interdum eleifend, arcu augue porta elit, nec tristique libero metus at massa. Fusce fringilla laoreet rhoncus. Suspendisse potenti. Phasellus dignissim sodales mauris at pharetra. Donec gravida fringilla velit ac rutrum. Curabitur ornare lectus id diam molestie eu imperdiet nulla tempus. Maecenas vestibulum enim et dui ornare blandit. Vivamus \index{fermentum} faucibus viverra. Maecenas at justo sapien. Aenean rhoncus augue mattis purus rhoncus venenatis. Suspendisse metus felis, porttitor in varius in, vulputate at tortor. Aliquam molestie, turpis et malesuada porta, tortor sapien pharetra sapien, ac rhoncus quam dolor a sapien. Pellentesque varius laoreet enim ut auctor. Nullam nec ultricies nisi. Nullam porta lectus et ante consectetur posuere. 
%\subsubsubsection{This Is My First Subsection}
% !TEX root = main.tex
% !TEX encoding = Windows Latin 1
% !TEX TS-program = pdflatex
% 
Duis tristique \index{sollicitudin} sollicitudin leo nec consequat. Praesent et dui convallis velit tincidunt fermentum. Mauris cursus purus at sem viverra sed imperdiet sapien imperdiet. Aliquam mattis, elit eget rutrum vulputate, tortor sem pulvinar justo, sit amet mollis felis sem at nibh. Donec malesuada, neque id interdum eleifend, arcu augue porta elit, nec tristique libero metus at massa. Fusce fringilla laoreet rhoncus. Suspendisse potenti. Phasellus dignissim sodales mauris at pharetra. Donec gravida fringilla velit ac rutrum. Curabitur ornare lectus id diam molestie eu imperdiet nulla tempus. Maecenas vestibulum enim et dui ornare blandit. Vivamus \index{fermentum} faucibus viverra. Maecenas at justo sapien. Aenean rhoncus augue mattis purus rhoncus venenatis. Suspendisse metus felis, porttitor in varius in, vulputate at tortor. Aliquam molestie, turpis et malesuada porta, tortor sapien pharetra sapien, ac rhoncus quam dolor a sapien. Pellentesque varius laoreet enim ut auctor. Nullam nec ultricies nisi. Nullam porta lectus et ante consectetur posuere. 

% !TEX root = main.tex
% !TEX encoding = Windows Latin 1
% !TEX TS-program = pdflatex
% 
Duis tristique \index{sollicitudin} sollicitudin leo nec consequat. Praesent et dui convallis velit tincidunt fermentum. Mauris cursus purus at sem viverra sed imperdiet sapien imperdiet. Aliquam mattis, elit eget rutrum vulputate, tortor sem pulvinar justo, sit amet mollis felis sem at nibh. Donec malesuada, neque id interdum eleifend, arcu augue porta elit, nec tristique libero metus at massa. Fusce fringilla laoreet rhoncus. Suspendisse potenti. Phasellus dignissim sodales mauris at pharetra. Donec gravida fringilla velit ac rutrum. Curabitur ornare lectus id diam molestie eu imperdiet nulla tempus. Maecenas vestibulum enim et dui ornare blandit. Vivamus \index{fermentum} faucibus viverra. Maecenas at justo sapien. Aenean rhoncus augue mattis purus rhoncus venenatis. Suspendisse metus felis, porttitor in varius in, vulputate at tortor. Aliquam molestie, turpis et malesuada porta, tortor sapien pharetra sapien, ac rhoncus quam dolor a sapien. Pellentesque varius laoreet enim ut auctor. Nullam nec ultricies nisi. Nullam porta lectus et ante consectetur posuere. 

% !TEX root = main.tex
% !TEX encoding = Windows Latin 1
% !TEX TS-program = pdflatex
% 
Duis tristique \index{sollicitudin} sollicitudin leo nec consequat. Praesent et dui convallis velit tincidunt fermentum. Mauris cursus purus at sem viverra sed imperdiet sapien imperdiet. Aliquam mattis, elit eget rutrum vulputate, tortor sem pulvinar justo, sit amet mollis felis sem at nibh. Donec malesuada, neque id interdum eleifend, arcu augue porta elit, nec tristique libero metus at massa. Fusce fringilla laoreet rhoncus. Suspendisse potenti. Phasellus dignissim sodales mauris at pharetra. Donec gravida fringilla velit ac rutrum. Curabitur ornare lectus id diam molestie eu imperdiet nulla tempus. Maecenas vestibulum enim et dui ornare blandit. Vivamus \index{fermentum} faucibus viverra. Maecenas at justo sapien. Aenean rhoncus augue mattis purus rhoncus venenatis. Suspendisse metus felis, porttitor in varius in, vulputate at tortor. Aliquam molestie, turpis et malesuada porta, tortor sapien pharetra sapien, ac rhoncus quam dolor a sapien. Pellentesque varius laoreet enim ut auctor. Nullam nec ultricies nisi. Nullam porta lectus et ante consectetur posuere. 

% !TEX root = main.tex
% !TEX encoding = Windows Latin 1
% !TEX TS-program = pdflatex
% 
Duis tristique \index{sollicitudin} sollicitudin leo nec consequat. Praesent et dui convallis velit tincidunt fermentum. Mauris cursus purus at sem viverra sed imperdiet sapien imperdiet. Aliquam mattis, elit eget rutrum vulputate, tortor sem pulvinar justo, sit amet mollis felis sem at nibh. Donec malesuada, neque id interdum eleifend, arcu augue porta elit, nec tristique libero metus at massa. Fusce fringilla laoreet rhoncus. Suspendisse potenti. Phasellus dignissim sodales mauris at pharetra. Donec gravida fringilla velit ac rutrum. Curabitur ornare lectus id diam molestie eu imperdiet nulla tempus. Maecenas vestibulum enim et dui ornare blandit. Vivamus \index{fermentum} faucibus viverra. Maecenas at justo sapien. Aenean rhoncus augue mattis purus rhoncus venenatis. Suspendisse metus felis, porttitor in varius in, vulputate at tortor. Aliquam molestie, turpis et malesuada porta, tortor sapien pharetra sapien, ac rhoncus quam dolor a sapien. Pellentesque varius laoreet enim ut auctor. Nullam nec ultricies nisi. Nullam porta lectus et ante consectetur posuere. 

% !TEX root = main.tex
% !TEX encoding = Windows Latin 1
% !TEX TS-program = pdflatex
% 
Duis tristique \index{sollicitudin} sollicitudin leo nec consequat. Praesent et dui convallis velit tincidunt fermentum. Mauris cursus purus at sem viverra sed imperdiet sapien imperdiet. Aliquam mattis, elit eget rutrum vulputate, tortor sem pulvinar justo, sit amet mollis felis sem at nibh. Donec malesuada, neque id interdum eleifend, arcu augue porta elit, nec tristique libero metus at massa. Fusce fringilla laoreet rhoncus. Suspendisse potenti. Phasellus dignissim sodales mauris at pharetra. Donec gravida fringilla velit ac rutrum. Curabitur ornare lectus id diam molestie eu imperdiet nulla tempus. Maecenas vestibulum enim et dui ornare blandit. Vivamus \index{fermentum} faucibus viverra. Maecenas at justo sapien. Aenean rhoncus augue mattis purus rhoncus venenatis. Suspendisse metus felis, porttitor in varius in, vulputate at tortor. Aliquam molestie, turpis et malesuada porta, tortor sapien pharetra sapien, ac rhoncus quam dolor a sapien. Pellentesque varius laoreet enim ut auctor. Nullam nec ultricies nisi. Nullam porta lectus et ante consectetur posuere. 

\section{This Is My Second Section}
\noindent
% !TEX root = main.tex
% !TEX encoding = Windows Latin 1
% !TEX TS-program = pdflatex
% 
Duis tristique \index{sollicitudin} sollicitudin leo nec consequat. Praesent et dui convallis velit tincidunt fermentum. Mauris cursus purus at sem viverra sed imperdiet sapien imperdiet. Aliquam mattis, elit eget rutrum vulputate, tortor sem pulvinar justo, sit amet mollis felis sem at nibh. Donec malesuada, neque id interdum eleifend, arcu augue porta elit, nec tristique libero metus at massa. Fusce fringilla laoreet rhoncus. Suspendisse potenti. Phasellus dignissim sodales mauris at pharetra. Donec gravida fringilla velit ac rutrum. Curabitur ornare lectus id diam molestie eu imperdiet nulla tempus. Maecenas vestibulum enim et dui ornare blandit. Vivamus \index{fermentum} faucibus viverra. Maecenas at justo sapien. Aenean rhoncus augue mattis purus rhoncus venenatis. Suspendisse metus felis, porttitor in varius in, vulputate at tortor. Aliquam molestie, turpis et malesuada porta, tortor sapien pharetra sapien, ac rhoncus quam dolor a sapien. Pellentesque varius laoreet enim ut auctor. Nullam nec ultricies nisi. Nullam porta lectus et ante consectetur posuere. \footnote{See also \cite{Aup91,Dou72,Hal82}.}


% !TEX root = main.tex
% !TEX encoding = Windows Latin 1
% !TEX TS-program = pdflatex
% 
Duis tristique \index{sollicitudin} sollicitudin leo nec consequat. Praesent et dui convallis velit tincidunt fermentum. Mauris cursus purus at sem viverra sed imperdiet sapien imperdiet. Aliquam mattis, elit eget rutrum vulputate, tortor sem pulvinar justo, sit amet mollis felis sem at nibh. Donec malesuada, neque id interdum eleifend, arcu augue porta elit, nec tristique libero metus at massa. Fusce fringilla laoreet rhoncus. Suspendisse potenti. Phasellus dignissim sodales mauris at pharetra. Donec gravida fringilla velit ac rutrum. Curabitur ornare lectus id diam molestie eu imperdiet nulla tempus. Maecenas vestibulum enim et dui ornare blandit. Vivamus \index{fermentum} faucibus viverra. Maecenas at justo sapien. Aenean rhoncus augue mattis purus rhoncus venenatis. Suspendisse metus felis, porttitor in varius in, vulputate at tortor. Aliquam molestie, turpis et malesuada porta, tortor sapien pharetra sapien, ac rhoncus quam dolor a sapien. Pellentesque varius laoreet enim ut auctor. Nullam nec ultricies nisi. Nullam porta lectus et ante consectetur posuere. 

% !TEX root = main.tex
% !TEX encoding = Windows Latin 1
% !TEX TS-program = pdflatex
% 
Duis tristique \index{sollicitudin} sollicitudin leo nec consequat. Praesent et dui convallis velit tincidunt fermentum. Mauris cursus purus at sem viverra sed imperdiet sapien imperdiet. Aliquam mattis, elit eget rutrum vulputate, tortor sem pulvinar justo, sit amet mollis felis sem at nibh. Donec malesuada, neque id interdum eleifend, arcu augue porta elit, nec tristique libero metus at massa. Fusce fringilla laoreet rhoncus. Suspendisse potenti. Phasellus dignissim sodales mauris at pharetra. Donec gravida fringilla velit ac rutrum. Curabitur ornare lectus id diam molestie eu imperdiet nulla tempus. Maecenas vestibulum enim et dui ornare blandit. Vivamus \index{fermentum} faucibus viverra. Maecenas at justo sapien. Aenean rhoncus augue mattis purus rhoncus venenatis. Suspendisse metus felis, porttitor in varius in, vulputate at tortor. Aliquam molestie, turpis et malesuada porta, tortor sapien pharetra sapien, ac rhoncus quam dolor a sapien. Pellentesque varius laoreet enim ut auctor. Nullam nec ultricies nisi. Nullam porta lectus et ante consectetur posuere. 

% !TEX root = main.tex
% !TEX encoding = Windows Latin 1
% !TEX TS-program = pdflatex
% 
Duis tristique \index{sollicitudin} sollicitudin leo nec consequat. Praesent et dui convallis velit tincidunt fermentum. Mauris cursus purus at sem viverra sed imperdiet sapien imperdiet. Aliquam mattis, elit eget rutrum vulputate, tortor sem pulvinar justo, sit amet mollis felis sem at nibh. Donec malesuada, neque id interdum eleifend, arcu augue porta elit, nec tristique libero metus at massa. Fusce fringilla laoreet rhoncus. Suspendisse potenti. Phasellus dignissim sodales mauris at pharetra. Donec gravida fringilla velit ac rutrum. Curabitur ornare lectus id diam molestie eu imperdiet nulla tempus. Maecenas vestibulum enim et dui ornare blandit. Vivamus \index{fermentum} faucibus viverra. Maecenas at justo sapien. Aenean rhoncus augue mattis purus rhoncus venenatis. Suspendisse metus felis, porttitor in varius in, vulputate at tortor. Aliquam molestie, turpis et malesuada porta, tortor sapien pharetra sapien, ac rhoncus quam dolor a sapien. Pellentesque varius laoreet enim ut auctor. Nullam nec ultricies nisi. Nullam porta lectus et ante consectetur posuere. 

\begin{table}[h]
  \begin{center}
\begin{tabular}{|r|l|}
  \hline
  7C0 & hexadecimal \\
  3700 & octal \\ \cline{2-2}
  11111000000 & binary \\
  \hline \hline
  1984 & decimal \\
  \hline
\end{tabular}
\caption{This is my first table}
\end{center}
\end{table}

\section{This Is My Third Section}
\noindent
% !TEX root = main.tex
% !TEX encoding = Windows Latin 1
% !TEX TS-program = pdflatex
% 
Duis tristique \index{sollicitudin} sollicitudin leo nec consequat. Praesent et dui convallis velit tincidunt fermentum. Mauris cursus purus at sem viverra sed imperdiet sapien imperdiet. Aliquam mattis, elit eget rutrum vulputate, tortor sem pulvinar justo, sit amet mollis felis sem at nibh. Donec malesuada, neque id interdum eleifend, arcu augue porta elit, nec tristique libero metus at massa. Fusce fringilla laoreet rhoncus. Suspendisse potenti. Phasellus dignissim sodales mauris at pharetra. Donec gravida fringilla velit ac rutrum. Curabitur ornare lectus id diam molestie eu imperdiet nulla tempus. Maecenas vestibulum enim et dui ornare blandit. Vivamus \index{fermentum} faucibus viverra. Maecenas at justo sapien. Aenean rhoncus augue mattis purus rhoncus venenatis. Suspendisse metus felis, porttitor in varius in, vulputate at tortor. Aliquam molestie, turpis et malesuada porta, tortor sapien pharetra sapien, ac rhoncus quam dolor a sapien. Pellentesque varius laoreet enim ut auctor. Nullam nec ultricies nisi. Nullam porta lectus et ante consectetur posuere. 
\begin{figure}[h]
  \begin{center}
    \includegraphics{imagen}
    \caption{This is my first figure}
  \end{center}
\end{figure}


% !TEX root = main.tex
% !TEX encoding = Windows Latin 1
% !TEX TS-program = pdflatex
% 
Duis tristique \index{sollicitudin} sollicitudin leo nec consequat. Praesent et dui convallis velit tincidunt fermentum. Mauris cursus purus at sem viverra sed imperdiet sapien imperdiet. Aliquam mattis, elit eget rutrum vulputate, tortor sem pulvinar justo, sit amet mollis felis sem at nibh. Donec malesuada, neque id interdum eleifend, arcu augue porta elit, nec tristique libero metus at massa. Fusce fringilla laoreet rhoncus. Suspendisse potenti. Phasellus dignissim sodales mauris at pharetra. Donec gravida fringilla velit ac rutrum. Curabitur ornare lectus id diam molestie eu imperdiet nulla tempus. Maecenas vestibulum enim et dui ornare blandit. Vivamus \index{fermentum} faucibus viverra. Maecenas at justo sapien. Aenean rhoncus augue mattis purus rhoncus venenatis. Suspendisse metus felis, porttitor in varius in, vulputate at tortor. Aliquam molestie, turpis et malesuada porta, tortor sapien pharetra sapien, ac rhoncus quam dolor a sapien. Pellentesque varius laoreet enim ut auctor. Nullam nec ultricies nisi. Nullam porta lectus et ante consectetur posuere. 

\begin{figure}[h]
  \begin{center}
    \includegraphics{imagen}
    \caption{This is my second figure}
  \end{center}
\end{figure}

% !TEX root = main.tex
% !TEX encoding = Windows Latin 1
% !TEX TS-program = pdflatex
% 
Duis tristique \index{sollicitudin} sollicitudin leo nec consequat. Praesent et dui convallis velit tincidunt fermentum. Mauris cursus purus at sem viverra sed imperdiet sapien imperdiet. Aliquam mattis, elit eget rutrum vulputate, tortor sem pulvinar justo, sit amet mollis felis sem at nibh. Donec malesuada, neque id interdum eleifend, arcu augue porta elit, nec tristique libero metus at massa. Fusce fringilla laoreet rhoncus. Suspendisse potenti. Phasellus dignissim sodales mauris at pharetra. Donec gravida fringilla velit ac rutrum. Curabitur ornare lectus id diam molestie eu imperdiet nulla tempus. Maecenas vestibulum enim et dui ornare blandit. Vivamus \index{fermentum} faucibus viverra. Maecenas at justo sapien. Aenean rhoncus augue mattis purus rhoncus venenatis. Suspendisse metus felis, porttitor in varius in, vulputate at tortor. Aliquam molestie, turpis et malesuada porta, tortor sapien pharetra sapien, ac rhoncus quam dolor a sapien. Pellentesque varius laoreet enim ut auctor. Nullam nec ultricies nisi. Nullam porta lectus et ante consectetur posuere. 

% !TEX root = main.tex
% !TEX encoding = Windows Latin 1
% !TEX TS-program = pdflatex
% 
Duis tristique \index{sollicitudin} sollicitudin leo nec consequat. Praesent et dui convallis velit tincidunt fermentum. Mauris cursus purus at sem viverra sed imperdiet sapien imperdiet. Aliquam mattis, elit eget rutrum vulputate, tortor sem pulvinar justo, sit amet mollis felis sem at nibh. Donec malesuada, neque id interdum eleifend, arcu augue porta elit, nec tristique libero metus at massa. Fusce fringilla laoreet rhoncus. Suspendisse potenti. Phasellus dignissim sodales mauris at pharetra. Donec gravida fringilla velit ac rutrum. Curabitur ornare lectus id diam molestie eu imperdiet nulla tempus. Maecenas vestibulum enim et dui ornare blandit. Vivamus \index{fermentum} faucibus viverra. Maecenas at justo sapien. Aenean rhoncus augue mattis purus rhoncus venenatis. Suspendisse metus felis, porttitor in varius in, vulputate at tortor. Aliquam molestie, turpis et malesuada porta, tortor sapien pharetra sapien, ac rhoncus quam dolor a sapien. Pellentesque varius laoreet enim ut auctor. Nullam nec ultricies nisi. Nullam porta lectus et ante consectetur posuere. 

% !TEX root = main.tex
% !TEX encoding = Windows Latin 1
% !TEX TS-program = pdflatex
% 
Duis tristique \index{sollicitudin} sollicitudin leo nec consequat. Praesent et dui convallis velit tincidunt fermentum. Mauris cursus purus at sem viverra sed imperdiet sapien imperdiet. Aliquam mattis, elit eget rutrum vulputate, tortor sem pulvinar justo, sit amet mollis felis sem at nibh. Donec malesuada, neque id interdum eleifend, arcu augue porta elit, nec tristique libero metus at massa. Fusce fringilla laoreet rhoncus. Suspendisse potenti. Phasellus dignissim sodales mauris at pharetra. Donec gravida fringilla velit ac rutrum. Curabitur ornare lectus id diam molestie eu imperdiet nulla tempus. Maecenas vestibulum enim et dui ornare blandit. Vivamus \index{fermentum} faucibus viverra. Maecenas at justo sapien. Aenean rhoncus augue mattis purus rhoncus venenatis. Suspendisse metus felis, porttitor in varius in, vulputate at tortor. Aliquam molestie, turpis et malesuada porta, tortor sapien pharetra sapien, ac rhoncus quam dolor a sapien. Pellentesque varius laoreet enim ut auctor. Nullam nec ultricies nisi. Nullam porta lectus et ante consectetur posuere. 

% !TEX root = main.tex
% !TEX encoding = Windows Latin 1
% !TEX TS-program = pdflatex
% 
Duis tristique \index{sollicitudin} sollicitudin leo nec consequat. Praesent et dui convallis velit tincidunt fermentum. Mauris cursus purus at sem viverra sed imperdiet sapien imperdiet. Aliquam mattis, elit eget rutrum vulputate, tortor sem pulvinar justo, sit amet mollis felis sem at nibh. Donec malesuada, neque id interdum eleifend, arcu augue porta elit, nec tristique libero metus at massa. Fusce fringilla laoreet rhoncus. Suspendisse potenti. Phasellus dignissim sodales mauris at pharetra. Donec gravida fringilla velit ac rutrum. Curabitur ornare lectus id diam molestie eu imperdiet nulla tempus. Maecenas vestibulum enim et dui ornare blandit. Vivamus \index{fermentum} faucibus viverra. Maecenas at justo sapien. Aenean rhoncus augue mattis purus rhoncus venenatis. Suspendisse metus felis, porttitor in varius in, vulputate at tortor. Aliquam molestie, turpis et malesuada porta, tortor sapien pharetra sapien, ac rhoncus quam dolor a sapien. Pellentesque varius laoreet enim ut auctor. Nullam nec ultricies nisi. Nullam porta lectus et ante consectetur posuere. 

% !TEX root = main.tex
% !TEX encoding = Windows Latin 1
% !TEX TS-program = pdflatex
% 
Duis tristique \index{sollicitudin} sollicitudin leo nec consequat. Praesent et dui convallis velit tincidunt fermentum. Mauris cursus purus at sem viverra sed imperdiet sapien imperdiet. Aliquam mattis, elit eget rutrum vulputate, tortor sem pulvinar justo, sit amet mollis felis sem at nibh. Donec malesuada, neque id interdum eleifend, arcu augue porta elit, nec tristique libero metus at massa. Fusce fringilla laoreet rhoncus. Suspendisse potenti. Phasellus dignissim sodales mauris at pharetra. Donec gravida fringilla velit ac rutrum. Curabitur ornare lectus id diam molestie eu imperdiet nulla tempus. Maecenas vestibulum enim et dui ornare blandit. Vivamus \index{fermentum} faucibus viverra. Maecenas at justo sapien. Aenean rhoncus augue mattis purus rhoncus venenatis. Suspendisse metus felis, porttitor in varius in, vulputate at tortor. Aliquam molestie, turpis et malesuada porta, tortor sapien pharetra sapien, ac rhoncus quam dolor a sapien. Pellentesque varius laoreet enim ut auctor. Nullam nec ultricies nisi. Nullam porta lectus et ante consectetur posuere. 

% Fin archivo ch01.tex
\endinput  % Capitulo 1 (editar archivo)
% !TEX root = main.tex
% !TEX encoding = Windows Latin 1
% !TEX TS-program = pdflatex
% 
% Archivo: ch02.tex (Capitulo 2)

\chapter{*******}
\section{*******}




% Fin archivo ch02.tex
\endinput  % Capitulo 2 (editar archivo)
% !TEX root = main.tex
% !TEX encoding = Windows Latin 1
% !TEX TS-program = pdflatex
% 
% Archivo: ch03.tex (Capitulo 3)

\chapter{*******}
\section{*******}




% Fin archivo ch03.tex
\endinput  % Capitulo 3 (editar archivo)
%\include{ch04} % etc...
% Anadir los que sean necesarios.

\appendix % Apendices de la tesis
% !TEX root = main.tex
% !TEX encoding = Windows Latin 1
% !TEX TS-program = pdflatex
% 
% Archivo: ap01.tex (Apendice 1)

\chapter{This Is My First Appendix bla bla bla bla bla bla bla bla}
\section{This Is My First Section bla bla bla bla bla bla bla bla bla bla bla bla }
\noindent
First sentence.\footnote{And this is my first footnote. And this is my first footnote. And this is my first footnote. And this is my first footnote. And this is my first footnote. And this is my first footnote. And this is my first footnote. And this is my first footnote. And this is my first footnote. And this is my first footnote. And this is my first footnote. And this is my first footnote. And this is my first footnote. And this is my first footnote. And this is my first footnote.}\index{Concepto}

\subsection{This Is My First Subsection}
\noindent
% !TEX root = main.tex
% !TEX encoding = Windows Latin 1
% !TEX TS-program = pdflatex
% 
Duis tristique \index{sollicitudin} sollicitudin leo nec consequat. Praesent et dui convallis velit tincidunt fermentum. Mauris cursus purus at sem viverra sed imperdiet sapien imperdiet. Aliquam mattis, elit eget rutrum vulputate, tortor sem pulvinar justo, sit amet mollis felis sem at nibh. Donec malesuada, neque id interdum eleifend, arcu augue porta elit, nec tristique libero metus at massa. Fusce fringilla laoreet rhoncus. Suspendisse potenti. Phasellus dignissim sodales mauris at pharetra. Donec gravida fringilla velit ac rutrum. Curabitur ornare lectus id diam molestie eu imperdiet nulla tempus. Maecenas vestibulum enim et dui ornare blandit. Vivamus \index{fermentum} faucibus viverra. Maecenas at justo sapien. Aenean rhoncus augue mattis purus rhoncus venenatis. Suspendisse metus felis, porttitor in varius in, vulputate at tortor. Aliquam molestie, turpis et malesuada porta, tortor sapien pharetra sapien, ac rhoncus quam dolor a sapien. Pellentesque varius laoreet enim ut auctor. Nullam nec ultricies nisi. Nullam porta lectus et ante consectetur posuere. \cite{texbook} 

% !TEX root = main.tex
% !TEX encoding = Windows Latin 1
% !TEX TS-program = pdflatex
% 
Duis tristique \index{sollicitudin} sollicitudin leo nec consequat. Praesent et dui convallis velit tincidunt fermentum. Mauris cursus purus at sem viverra sed imperdiet sapien imperdiet. Aliquam mattis, elit eget rutrum vulputate, tortor sem pulvinar justo, sit amet mollis felis sem at nibh. Donec malesuada, neque id interdum eleifend, arcu augue porta elit, nec tristique libero metus at massa. Fusce fringilla laoreet rhoncus. Suspendisse potenti. Phasellus dignissim sodales mauris at pharetra. Donec gravida fringilla velit ac rutrum. Curabitur ornare lectus id diam molestie eu imperdiet nulla tempus. Maecenas vestibulum enim et dui ornare blandit. Vivamus \index{fermentum} faucibus viverra. Maecenas at justo sapien. Aenean rhoncus augue mattis purus rhoncus venenatis. Suspendisse metus felis, porttitor in varius in, vulputate at tortor. Aliquam molestie, turpis et malesuada porta, tortor sapien pharetra sapien, ac rhoncus quam dolor a sapien. Pellentesque varius laoreet enim ut auctor. Nullam nec ultricies nisi. Nullam porta lectus et ante consectetur posuere. 

% !TEX root = main.tex
% !TEX encoding = Windows Latin 1
% !TEX TS-program = pdflatex
% 
Duis tristique \index{sollicitudin} sollicitudin leo nec consequat. Praesent et dui convallis velit tincidunt fermentum. Mauris cursus purus at sem viverra sed imperdiet sapien imperdiet. Aliquam mattis, elit eget rutrum vulputate, tortor sem pulvinar justo, sit amet mollis felis sem at nibh. Donec malesuada, neque id interdum eleifend, arcu augue porta elit, nec tristique libero metus at massa. Fusce fringilla laoreet rhoncus. Suspendisse potenti. Phasellus dignissim sodales mauris at pharetra. Donec gravida fringilla velit ac rutrum. Curabitur ornare lectus id diam molestie eu imperdiet nulla tempus. Maecenas vestibulum enim et dui ornare blandit. Vivamus \index{fermentum} faucibus viverra. Maecenas at justo sapien. Aenean rhoncus augue mattis purus rhoncus venenatis. Suspendisse metus felis, porttitor in varius in, vulputate at tortor. Aliquam molestie, turpis et malesuada porta, tortor sapien pharetra sapien, ac rhoncus quam dolor a sapien. Pellentesque varius laoreet enim ut auctor. Nullam nec ultricies nisi. Nullam porta lectus et ante consectetur posuere. 

% !TEX root = main.tex
% !TEX encoding = Windows Latin 1
% !TEX TS-program = pdflatex
% 
Duis tristique \index{sollicitudin} sollicitudin leo nec consequat. Praesent et dui convallis velit tincidunt fermentum. Mauris cursus purus at sem viverra sed imperdiet sapien imperdiet. Aliquam mattis, elit eget rutrum vulputate, tortor sem pulvinar justo, sit amet mollis felis sem at nibh. Donec malesuada, neque id interdum eleifend, arcu augue porta elit, nec tristique libero metus at massa. Fusce fringilla laoreet rhoncus. Suspendisse potenti. Phasellus dignissim sodales mauris at pharetra. Donec gravida fringilla velit ac rutrum. Curabitur ornare lectus id diam molestie eu imperdiet nulla tempus. Maecenas vestibulum enim et dui ornare blandit. Vivamus \index{fermentum} faucibus viverra. Maecenas at justo sapien. Aenean rhoncus augue mattis purus rhoncus venenatis. Suspendisse metus felis, porttitor in varius in, vulputate at tortor. Aliquam molestie, turpis et malesuada porta, tortor sapien pharetra sapien, ac rhoncus quam dolor a sapien. Pellentesque varius laoreet enim ut auctor. Nullam nec ultricies nisi. Nullam porta lectus et ante consectetur posuere. 

\subsubsection{This Is My First Subsubsection}
\noindent
% !TEX root = main.tex
% !TEX encoding = Windows Latin 1
% !TEX TS-program = pdflatex
% 
Duis tristique \index{sollicitudin} sollicitudin leo nec consequat. Praesent et dui convallis velit tincidunt fermentum. Mauris cursus purus at sem viverra sed imperdiet sapien imperdiet. Aliquam mattis, elit eget rutrum vulputate, tortor sem pulvinar justo, sit amet mollis felis sem at nibh. Donec malesuada, neque id interdum eleifend, arcu augue porta elit, nec tristique libero metus at massa. Fusce fringilla laoreet rhoncus. Suspendisse potenti. Phasellus dignissim sodales mauris at pharetra. Donec gravida fringilla velit ac rutrum. Curabitur ornare lectus id diam molestie eu imperdiet nulla tempus. Maecenas vestibulum enim et dui ornare blandit. Vivamus \index{fermentum} faucibus viverra. Maecenas at justo sapien. Aenean rhoncus augue mattis purus rhoncus venenatis. Suspendisse metus felis, porttitor in varius in, vulputate at tortor. Aliquam molestie, turpis et malesuada porta, tortor sapien pharetra sapien, ac rhoncus quam dolor a sapien. Pellentesque varius laoreet enim ut auctor. Nullam nec ultricies nisi. Nullam porta lectus et ante consectetur posuere. \cite{texbook}

% !TEX root = main.tex
% !TEX encoding = Windows Latin 1
% !TEX TS-program = pdflatex
% 
Duis tristique \index{sollicitudin} sollicitudin leo nec consequat. Praesent et dui convallis velit tincidunt fermentum. Mauris cursus purus at sem viverra sed imperdiet sapien imperdiet. Aliquam mattis, elit eget rutrum vulputate, tortor sem pulvinar justo, sit amet mollis felis sem at nibh. Donec malesuada, neque id interdum eleifend, arcu augue porta elit, nec tristique libero metus at massa. Fusce fringilla laoreet rhoncus. Suspendisse potenti. Phasellus dignissim sodales mauris at pharetra. Donec gravida fringilla velit ac rutrum. Curabitur ornare lectus id diam molestie eu imperdiet nulla tempus. Maecenas vestibulum enim et dui ornare blandit. Vivamus \index{fermentum} faucibus viverra. Maecenas at justo sapien. Aenean rhoncus augue mattis purus rhoncus venenatis. Suspendisse metus felis, porttitor in varius in, vulputate at tortor. Aliquam molestie, turpis et malesuada porta, tortor sapien pharetra sapien, ac rhoncus quam dolor a sapien. Pellentesque varius laoreet enim ut auctor. Nullam nec ultricies nisi. Nullam porta lectus et ante consectetur posuere. 

% !TEX root = main.tex
% !TEX encoding = Windows Latin 1
% !TEX TS-program = pdflatex
% 
Duis tristique \index{sollicitudin} sollicitudin leo nec consequat. Praesent et dui convallis velit tincidunt fermentum. Mauris cursus purus at sem viverra sed imperdiet sapien imperdiet. Aliquam mattis, elit eget rutrum vulputate, tortor sem pulvinar justo, sit amet mollis felis sem at nibh. Donec malesuada, neque id interdum eleifend, arcu augue porta elit, nec tristique libero metus at massa. Fusce fringilla laoreet rhoncus. Suspendisse potenti. Phasellus dignissim sodales mauris at pharetra. Donec gravida fringilla velit ac rutrum. Curabitur ornare lectus id diam molestie eu imperdiet nulla tempus. Maecenas vestibulum enim et dui ornare blandit. Vivamus \index{fermentum} faucibus viverra. Maecenas at justo sapien. Aenean rhoncus augue mattis purus rhoncus venenatis. Suspendisse metus felis, porttitor in varius in, vulputate at tortor. Aliquam molestie, turpis et malesuada porta, tortor sapien pharetra sapien, ac rhoncus quam dolor a sapien. Pellentesque varius laoreet enim ut auctor. Nullam nec ultricies nisi. Nullam porta lectus et ante consectetur posuere. 

% !TEX root = main.tex
% !TEX encoding = Windows Latin 1
% !TEX TS-program = pdflatex
% 
Duis tristique \index{sollicitudin} sollicitudin leo nec consequat. Praesent et dui convallis velit tincidunt fermentum. Mauris cursus purus at sem viverra sed imperdiet sapien imperdiet. Aliquam mattis, elit eget rutrum vulputate, tortor sem pulvinar justo, sit amet mollis felis sem at nibh. Donec malesuada, neque id interdum eleifend, arcu augue porta elit, nec tristique libero metus at massa. Fusce fringilla laoreet rhoncus. Suspendisse potenti. Phasellus dignissim sodales mauris at pharetra. Donec gravida fringilla velit ac rutrum. Curabitur ornare lectus id diam molestie eu imperdiet nulla tempus. Maecenas vestibulum enim et dui ornare blandit. Vivamus \index{fermentum} faucibus viverra. Maecenas at justo sapien. Aenean rhoncus augue mattis purus rhoncus venenatis. Suspendisse metus felis, porttitor in varius in, vulputate at tortor. Aliquam molestie, turpis et malesuada porta, tortor sapien pharetra sapien, ac rhoncus quam dolor a sapien. Pellentesque varius laoreet enim ut auctor. Nullam nec ultricies nisi. Nullam porta lectus et ante consectetur posuere. 

\section{This Is My Second Section}
\noindent
% !TEX root = main.tex
% !TEX encoding = Windows Latin 1
% !TEX TS-program = pdflatex
% 
Duis tristique \index{sollicitudin} sollicitudin leo nec consequat. Praesent et dui convallis velit tincidunt fermentum. Mauris cursus purus at sem viverra sed imperdiet sapien imperdiet. Aliquam mattis, elit eget rutrum vulputate, tortor sem pulvinar justo, sit amet mollis felis sem at nibh. Donec malesuada, neque id interdum eleifend, arcu augue porta elit, nec tristique libero metus at massa. Fusce fringilla laoreet rhoncus. Suspendisse potenti. Phasellus dignissim sodales mauris at pharetra. Donec gravida fringilla velit ac rutrum. Curabitur ornare lectus id diam molestie eu imperdiet nulla tempus. Maecenas vestibulum enim et dui ornare blandit. Vivamus \index{fermentum} faucibus viverra. Maecenas at justo sapien. Aenean rhoncus augue mattis purus rhoncus venenatis. Suspendisse metus felis, porttitor in varius in, vulputate at tortor. Aliquam molestie, turpis et malesuada porta, tortor sapien pharetra sapien, ac rhoncus quam dolor a sapien. Pellentesque varius laoreet enim ut auctor. Nullam nec ultricies nisi. Nullam porta lectus et ante consectetur posuere. \footnote{See also \cite{Aup91,Dou72,Hal82}.}


% !TEX root = main.tex
% !TEX encoding = Windows Latin 1
% !TEX TS-program = pdflatex
% 
Duis tristique \index{sollicitudin} sollicitudin leo nec consequat. Praesent et dui convallis velit tincidunt fermentum. Mauris cursus purus at sem viverra sed imperdiet sapien imperdiet. Aliquam mattis, elit eget rutrum vulputate, tortor sem pulvinar justo, sit amet mollis felis sem at nibh. Donec malesuada, neque id interdum eleifend, arcu augue porta elit, nec tristique libero metus at massa. Fusce fringilla laoreet rhoncus. Suspendisse potenti. Phasellus dignissim sodales mauris at pharetra. Donec gravida fringilla velit ac rutrum. Curabitur ornare lectus id diam molestie eu imperdiet nulla tempus. Maecenas vestibulum enim et dui ornare blandit. Vivamus \index{fermentum} faucibus viverra. Maecenas at justo sapien. Aenean rhoncus augue mattis purus rhoncus venenatis. Suspendisse metus felis, porttitor in varius in, vulputate at tortor. Aliquam molestie, turpis et malesuada porta, tortor sapien pharetra sapien, ac rhoncus quam dolor a sapien. Pellentesque varius laoreet enim ut auctor. Nullam nec ultricies nisi. Nullam porta lectus et ante consectetur posuere. 

% !TEX root = main.tex
% !TEX encoding = Windows Latin 1
% !TEX TS-program = pdflatex
% 
Duis tristique \index{sollicitudin} sollicitudin leo nec consequat. Praesent et dui convallis velit tincidunt fermentum. Mauris cursus purus at sem viverra sed imperdiet sapien imperdiet. Aliquam mattis, elit eget rutrum vulputate, tortor sem pulvinar justo, sit amet mollis felis sem at nibh. Donec malesuada, neque id interdum eleifend, arcu augue porta elit, nec tristique libero metus at massa. Fusce fringilla laoreet rhoncus. Suspendisse potenti. Phasellus dignissim sodales mauris at pharetra. Donec gravida fringilla velit ac rutrum. Curabitur ornare lectus id diam molestie eu imperdiet nulla tempus. Maecenas vestibulum enim et dui ornare blandit. Vivamus \index{fermentum} faucibus viverra. Maecenas at justo sapien. Aenean rhoncus augue mattis purus rhoncus venenatis. Suspendisse metus felis, porttitor in varius in, vulputate at tortor. Aliquam molestie, turpis et malesuada porta, tortor sapien pharetra sapien, ac rhoncus quam dolor a sapien. Pellentesque varius laoreet enim ut auctor. Nullam nec ultricies nisi. Nullam porta lectus et ante consectetur posuere. 

% !TEX root = main.tex
% !TEX encoding = Windows Latin 1
% !TEX TS-program = pdflatex
% 
Duis tristique \index{sollicitudin} sollicitudin leo nec consequat. Praesent et dui convallis velit tincidunt fermentum. Mauris cursus purus at sem viverra sed imperdiet sapien imperdiet. Aliquam mattis, elit eget rutrum vulputate, tortor sem pulvinar justo, sit amet mollis felis sem at nibh. Donec malesuada, neque id interdum eleifend, arcu augue porta elit, nec tristique libero metus at massa. Fusce fringilla laoreet rhoncus. Suspendisse potenti. Phasellus dignissim sodales mauris at pharetra. Donec gravida fringilla velit ac rutrum. Curabitur ornare lectus id diam molestie eu imperdiet nulla tempus. Maecenas vestibulum enim et dui ornare blandit. Vivamus \index{fermentum} faucibus viverra. Maecenas at justo sapien. Aenean rhoncus augue mattis purus rhoncus venenatis. Suspendisse metus felis, porttitor in varius in, vulputate at tortor. Aliquam molestie, turpis et malesuada porta, tortor sapien pharetra sapien, ac rhoncus quam dolor a sapien. Pellentesque varius laoreet enim ut auctor. Nullam nec ultricies nisi. Nullam porta lectus et ante consectetur posuere. 

\begin{table}[h]
  \begin{center}
\begin{tabular}{|r|l|}
  \hline
  7C0 & hexadecimal \\
  3700 & octal \\ \cline{2-2}
  11111000000 & binary \\
  \hline \hline
  1984 & decimal \\
  \hline
\end{tabular}
\caption{This is my first table}
\end{center}
\end{table}

\section{This Is My Third Section}
\noindent
% !TEX root = main.tex
% !TEX encoding = Windows Latin 1
% !TEX TS-program = pdflatex
% 
Duis tristique \index{sollicitudin} sollicitudin leo nec consequat. Praesent et dui convallis velit tincidunt fermentum. Mauris cursus purus at sem viverra sed imperdiet sapien imperdiet. Aliquam mattis, elit eget rutrum vulputate, tortor sem pulvinar justo, sit amet mollis felis sem at nibh. Donec malesuada, neque id interdum eleifend, arcu augue porta elit, nec tristique libero metus at massa. Fusce fringilla laoreet rhoncus. Suspendisse potenti. Phasellus dignissim sodales mauris at pharetra. Donec gravida fringilla velit ac rutrum. Curabitur ornare lectus id diam molestie eu imperdiet nulla tempus. Maecenas vestibulum enim et dui ornare blandit. Vivamus \index{fermentum} faucibus viverra. Maecenas at justo sapien. Aenean rhoncus augue mattis purus rhoncus venenatis. Suspendisse metus felis, porttitor in varius in, vulputate at tortor. Aliquam molestie, turpis et malesuada porta, tortor sapien pharetra sapien, ac rhoncus quam dolor a sapien. Pellentesque varius laoreet enim ut auctor. Nullam nec ultricies nisi. Nullam porta lectus et ante consectetur posuere. 
\begin{figure}[h]
  \begin{center}
    \includegraphics{imagen}
    \caption{This is my first figure}
  \end{center}
\end{figure}


% !TEX root = main.tex
% !TEX encoding = Windows Latin 1
% !TEX TS-program = pdflatex
% 
Duis tristique \index{sollicitudin} sollicitudin leo nec consequat. Praesent et dui convallis velit tincidunt fermentum. Mauris cursus purus at sem viverra sed imperdiet sapien imperdiet. Aliquam mattis, elit eget rutrum vulputate, tortor sem pulvinar justo, sit amet mollis felis sem at nibh. Donec malesuada, neque id interdum eleifend, arcu augue porta elit, nec tristique libero metus at massa. Fusce fringilla laoreet rhoncus. Suspendisse potenti. Phasellus dignissim sodales mauris at pharetra. Donec gravida fringilla velit ac rutrum. Curabitur ornare lectus id diam molestie eu imperdiet nulla tempus. Maecenas vestibulum enim et dui ornare blandit. Vivamus \index{fermentum} faucibus viverra. Maecenas at justo sapien. Aenean rhoncus augue mattis purus rhoncus venenatis. Suspendisse metus felis, porttitor in varius in, vulputate at tortor. Aliquam molestie, turpis et malesuada porta, tortor sapien pharetra sapien, ac rhoncus quam dolor a sapien. Pellentesque varius laoreet enim ut auctor. Nullam nec ultricies nisi. Nullam porta lectus et ante consectetur posuere. 

\begin{figure}[h]
  \begin{center}
    \includegraphics{imagen}
    \caption{This is my second figure}
  \end{center}
\end{figure}

% !TEX root = main.tex
% !TEX encoding = Windows Latin 1
% !TEX TS-program = pdflatex
% 
Duis tristique \index{sollicitudin} sollicitudin leo nec consequat. Praesent et dui convallis velit tincidunt fermentum. Mauris cursus purus at sem viverra sed imperdiet sapien imperdiet. Aliquam mattis, elit eget rutrum vulputate, tortor sem pulvinar justo, sit amet mollis felis sem at nibh. Donec malesuada, neque id interdum eleifend, arcu augue porta elit, nec tristique libero metus at massa. Fusce fringilla laoreet rhoncus. Suspendisse potenti. Phasellus dignissim sodales mauris at pharetra. Donec gravida fringilla velit ac rutrum. Curabitur ornare lectus id diam molestie eu imperdiet nulla tempus. Maecenas vestibulum enim et dui ornare blandit. Vivamus \index{fermentum} faucibus viverra. Maecenas at justo sapien. Aenean rhoncus augue mattis purus rhoncus venenatis. Suspendisse metus felis, porttitor in varius in, vulputate at tortor. Aliquam molestie, turpis et malesuada porta, tortor sapien pharetra sapien, ac rhoncus quam dolor a sapien. Pellentesque varius laoreet enim ut auctor. Nullam nec ultricies nisi. Nullam porta lectus et ante consectetur posuere. 

% !TEX root = main.tex
% !TEX encoding = Windows Latin 1
% !TEX TS-program = pdflatex
% 
Duis tristique \index{sollicitudin} sollicitudin leo nec consequat. Praesent et dui convallis velit tincidunt fermentum. Mauris cursus purus at sem viverra sed imperdiet sapien imperdiet. Aliquam mattis, elit eget rutrum vulputate, tortor sem pulvinar justo, sit amet mollis felis sem at nibh. Donec malesuada, neque id interdum eleifend, arcu augue porta elit, nec tristique libero metus at massa. Fusce fringilla laoreet rhoncus. Suspendisse potenti. Phasellus dignissim sodales mauris at pharetra. Donec gravida fringilla velit ac rutrum. Curabitur ornare lectus id diam molestie eu imperdiet nulla tempus. Maecenas vestibulum enim et dui ornare blandit. Vivamus \index{fermentum} faucibus viverra. Maecenas at justo sapien. Aenean rhoncus augue mattis purus rhoncus venenatis. Suspendisse metus felis, porttitor in varius in, vulputate at tortor. Aliquam molestie, turpis et malesuada porta, tortor sapien pharetra sapien, ac rhoncus quam dolor a sapien. Pellentesque varius laoreet enim ut auctor. Nullam nec ultricies nisi. Nullam porta lectus et ante consectetur posuere. 

% !TEX root = main.tex
% !TEX encoding = Windows Latin 1
% !TEX TS-program = pdflatex
% 
Duis tristique \index{sollicitudin} sollicitudin leo nec consequat. Praesent et dui convallis velit tincidunt fermentum. Mauris cursus purus at sem viverra sed imperdiet sapien imperdiet. Aliquam mattis, elit eget rutrum vulputate, tortor sem pulvinar justo, sit amet mollis felis sem at nibh. Donec malesuada, neque id interdum eleifend, arcu augue porta elit, nec tristique libero metus at massa. Fusce fringilla laoreet rhoncus. Suspendisse potenti. Phasellus dignissim sodales mauris at pharetra. Donec gravida fringilla velit ac rutrum. Curabitur ornare lectus id diam molestie eu imperdiet nulla tempus. Maecenas vestibulum enim et dui ornare blandit. Vivamus \index{fermentum} faucibus viverra. Maecenas at justo sapien. Aenean rhoncus augue mattis purus rhoncus venenatis. Suspendisse metus felis, porttitor in varius in, vulputate at tortor. Aliquam molestie, turpis et malesuada porta, tortor sapien pharetra sapien, ac rhoncus quam dolor a sapien. Pellentesque varius laoreet enim ut auctor. Nullam nec ultricies nisi. Nullam porta lectus et ante consectetur posuere. 

% !TEX root = main.tex
% !TEX encoding = Windows Latin 1
% !TEX TS-program = pdflatex
% 
Duis tristique \index{sollicitudin} sollicitudin leo nec consequat. Praesent et dui convallis velit tincidunt fermentum. Mauris cursus purus at sem viverra sed imperdiet sapien imperdiet. Aliquam mattis, elit eget rutrum vulputate, tortor sem pulvinar justo, sit amet mollis felis sem at nibh. Donec malesuada, neque id interdum eleifend, arcu augue porta elit, nec tristique libero metus at massa. Fusce fringilla laoreet rhoncus. Suspendisse potenti. Phasellus dignissim sodales mauris at pharetra. Donec gravida fringilla velit ac rutrum. Curabitur ornare lectus id diam molestie eu imperdiet nulla tempus. Maecenas vestibulum enim et dui ornare blandit. Vivamus \index{fermentum} faucibus viverra. Maecenas at justo sapien. Aenean rhoncus augue mattis purus rhoncus venenatis. Suspendisse metus felis, porttitor in varius in, vulputate at tortor. Aliquam molestie, turpis et malesuada porta, tortor sapien pharetra sapien, ac rhoncus quam dolor a sapien. Pellentesque varius laoreet enim ut auctor. Nullam nec ultricies nisi. Nullam porta lectus et ante consectetur posuere. 

% !TEX root = main.tex
% !TEX encoding = Windows Latin 1
% !TEX TS-program = pdflatex
% 
Duis tristique \index{sollicitudin} sollicitudin leo nec consequat. Praesent et dui convallis velit tincidunt fermentum. Mauris cursus purus at sem viverra sed imperdiet sapien imperdiet. Aliquam mattis, elit eget rutrum vulputate, tortor sem pulvinar justo, sit amet mollis felis sem at nibh. Donec malesuada, neque id interdum eleifend, arcu augue porta elit, nec tristique libero metus at massa. Fusce fringilla laoreet rhoncus. Suspendisse potenti. Phasellus dignissim sodales mauris at pharetra. Donec gravida fringilla velit ac rutrum. Curabitur ornare lectus id diam molestie eu imperdiet nulla tempus. Maecenas vestibulum enim et dui ornare blandit. Vivamus \index{fermentum} faucibus viverra. Maecenas at justo sapien. Aenean rhoncus augue mattis purus rhoncus venenatis. Suspendisse metus felis, porttitor in varius in, vulputate at tortor. Aliquam molestie, turpis et malesuada porta, tortor sapien pharetra sapien, ac rhoncus quam dolor a sapien. Pellentesque varius laoreet enim ut auctor. Nullam nec ultricies nisi. Nullam porta lectus et ante consectetur posuere. 

% Fin archivo ap01.tex
\endinput  % Apendice 1  (editar archivo)
%\include{ap02} % Apendice 2 (editar archivo)
%\include{ap03} % etc...
% Anadir los que sean necesarios.

%----------------------------------------------------------
\backmatter % Partes finales de la tesis
\renewcommand*{\afterchapskip}{2\onelineskip}

\bibliography{bibliotesis} % Usar el propio o editar este
% (recuerde correr bibtex luego de la primera compilacion)

\CrearIndice % En caso de que haya indice analitico
% (comentar en caso contrario)
% (recuerde correr makeindex luego de la primera compilacion)





\end{document}% ================================================
% end of the main file: main.tex
\typeout{End main.tex}
\endinput