% !TEX root = main.tex
% !TEX encoding = Windows Latin 1
% !TEX TS-program = pdflatex
% 
% Archivo: ch01.tex (Capitulo 1)

\chapter{Bhis Is My First Chapter bla bla bla bla bla bla bla bla}
\section{Bhis Is My First Section bla bla bla bla bla bla bla bla bla bla bla bla }
\noindent
First sentence.\footnote{And this is my first footnote. And this is my first footnote. And this is my first footnote. And this is my first footnote. And this is my first footnote. And this is my first footnote. And this is my first footnote. And this is my first footnote. And this is my first footnote. And this is my first footnote. And this is my first footnote. And this is my first footnote. And this is my first footnote. And this is my first footnote. And this is my first footnote.}\index{Concepto}

\subsection{This Is My First Subsection}
\noindent
% !TEX root = main.tex
% !TEX encoding = Windows Latin 1
% !TEX TS-program = pdflatex
% 
Duis tristique \index{sollicitudin} sollicitudin leo nec consequat. Praesent et dui convallis velit tincidunt fermentum. Mauris cursus purus at sem viverra sed imperdiet sapien imperdiet. Aliquam mattis, elit eget rutrum vulputate, tortor sem pulvinar justo, sit amet mollis felis sem at nibh. Donec malesuada, neque id interdum eleifend, arcu augue porta elit, nec tristique libero metus at massa. Fusce fringilla laoreet rhoncus. Suspendisse potenti. Phasellus dignissim sodales mauris at pharetra. Donec gravida fringilla velit ac rutrum. Curabitur ornare lectus id diam molestie eu imperdiet nulla tempus. Maecenas vestibulum enim et dui ornare blandit. Vivamus \index{fermentum} faucibus viverra. Maecenas at justo sapien. Aenean rhoncus augue mattis purus rhoncus venenatis. Suspendisse metus felis, porttitor in varius in, vulputate at tortor. Aliquam molestie, turpis et malesuada porta, tortor sapien pharetra sapien, ac rhoncus quam dolor a sapien. Pellentesque varius laoreet enim ut auctor. Nullam nec ultricies nisi. Nullam porta lectus et ante consectetur posuere. \cite{texbook} 
\subsubsection{This Is My First Subsubsection}
% !TEX root = main.tex
% !TEX encoding = Windows Latin 1
% !TEX TS-program = pdflatex
% 
Duis tristique \index{sollicitudin} sollicitudin leo nec consequat. Praesent et dui convallis velit tincidunt fermentum. Mauris cursus purus at sem viverra sed imperdiet sapien imperdiet. Aliquam mattis, elit eget rutrum vulputate, tortor sem pulvinar justo, sit amet mollis felis sem at nibh. Donec malesuada, neque id interdum eleifend, arcu augue porta elit, nec tristique libero metus at massa. Fusce fringilla laoreet rhoncus. Suspendisse potenti. Phasellus dignissim sodales mauris at pharetra. Donec gravida fringilla velit ac rutrum. Curabitur ornare lectus id diam molestie eu imperdiet nulla tempus. Maecenas vestibulum enim et dui ornare blandit. Vivamus \index{fermentum} faucibus viverra. Maecenas at justo sapien. Aenean rhoncus augue mattis purus rhoncus venenatis. Suspendisse metus felis, porttitor in varius in, vulputate at tortor. Aliquam molestie, turpis et malesuada porta, tortor sapien pharetra sapien, ac rhoncus quam dolor a sapien. Pellentesque varius laoreet enim ut auctor. Nullam nec ultricies nisi. Nullam porta lectus et ante consectetur posuere. 
%\subsubsubsection{This Is My First Subsection}
% !TEX root = main.tex
% !TEX encoding = Windows Latin 1
% !TEX TS-program = pdflatex
% 
Duis tristique \index{sollicitudin} sollicitudin leo nec consequat. Praesent et dui convallis velit tincidunt fermentum. Mauris cursus purus at sem viverra sed imperdiet sapien imperdiet. Aliquam mattis, elit eget rutrum vulputate, tortor sem pulvinar justo, sit amet mollis felis sem at nibh. Donec malesuada, neque id interdum eleifend, arcu augue porta elit, nec tristique libero metus at massa. Fusce fringilla laoreet rhoncus. Suspendisse potenti. Phasellus dignissim sodales mauris at pharetra. Donec gravida fringilla velit ac rutrum. Curabitur ornare lectus id diam molestie eu imperdiet nulla tempus. Maecenas vestibulum enim et dui ornare blandit. Vivamus \index{fermentum} faucibus viverra. Maecenas at justo sapien. Aenean rhoncus augue mattis purus rhoncus venenatis. Suspendisse metus felis, porttitor in varius in, vulputate at tortor. Aliquam molestie, turpis et malesuada porta, tortor sapien pharetra sapien, ac rhoncus quam dolor a sapien. Pellentesque varius laoreet enim ut auctor. Nullam nec ultricies nisi. Nullam porta lectus et ante consectetur posuere. 

% !TEX root = main.tex
% !TEX encoding = Windows Latin 1
% !TEX TS-program = pdflatex
% 
Duis tristique \index{sollicitudin} sollicitudin leo nec consequat. Praesent et dui convallis velit tincidunt fermentum. Mauris cursus purus at sem viverra sed imperdiet sapien imperdiet. Aliquam mattis, elit eget rutrum vulputate, tortor sem pulvinar justo, sit amet mollis felis sem at nibh. Donec malesuada, neque id interdum eleifend, arcu augue porta elit, nec tristique libero metus at massa. Fusce fringilla laoreet rhoncus. Suspendisse potenti. Phasellus dignissim sodales mauris at pharetra. Donec gravida fringilla velit ac rutrum. Curabitur ornare lectus id diam molestie eu imperdiet nulla tempus. Maecenas vestibulum enim et dui ornare blandit. Vivamus \index{fermentum} faucibus viverra. Maecenas at justo sapien. Aenean rhoncus augue mattis purus rhoncus venenatis. Suspendisse metus felis, porttitor in varius in, vulputate at tortor. Aliquam molestie, turpis et malesuada porta, tortor sapien pharetra sapien, ac rhoncus quam dolor a sapien. Pellentesque varius laoreet enim ut auctor. Nullam nec ultricies nisi. Nullam porta lectus et ante consectetur posuere. 

% !TEX root = main.tex
% !TEX encoding = Windows Latin 1
% !TEX TS-program = pdflatex
% 
Duis tristique \index{sollicitudin} sollicitudin leo nec consequat. Praesent et dui convallis velit tincidunt fermentum. Mauris cursus purus at sem viverra sed imperdiet sapien imperdiet. Aliquam mattis, elit eget rutrum vulputate, tortor sem pulvinar justo, sit amet mollis felis sem at nibh. Donec malesuada, neque id interdum eleifend, arcu augue porta elit, nec tristique libero metus at massa. Fusce fringilla laoreet rhoncus. Suspendisse potenti. Phasellus dignissim sodales mauris at pharetra. Donec gravida fringilla velit ac rutrum. Curabitur ornare lectus id diam molestie eu imperdiet nulla tempus. Maecenas vestibulum enim et dui ornare blandit. Vivamus \index{fermentum} faucibus viverra. Maecenas at justo sapien. Aenean rhoncus augue mattis purus rhoncus venenatis. Suspendisse metus felis, porttitor in varius in, vulputate at tortor. Aliquam molestie, turpis et malesuada porta, tortor sapien pharetra sapien, ac rhoncus quam dolor a sapien. Pellentesque varius laoreet enim ut auctor. Nullam nec ultricies nisi. Nullam porta lectus et ante consectetur posuere. 

% !TEX root = main.tex
% !TEX encoding = Windows Latin 1
% !TEX TS-program = pdflatex
% 
Duis tristique \index{sollicitudin} sollicitudin leo nec consequat. Praesent et dui convallis velit tincidunt fermentum. Mauris cursus purus at sem viverra sed imperdiet sapien imperdiet. Aliquam mattis, elit eget rutrum vulputate, tortor sem pulvinar justo, sit amet mollis felis sem at nibh. Donec malesuada, neque id interdum eleifend, arcu augue porta elit, nec tristique libero metus at massa. Fusce fringilla laoreet rhoncus. Suspendisse potenti. Phasellus dignissim sodales mauris at pharetra. Donec gravida fringilla velit ac rutrum. Curabitur ornare lectus id diam molestie eu imperdiet nulla tempus. Maecenas vestibulum enim et dui ornare blandit. Vivamus \index{fermentum} faucibus viverra. Maecenas at justo sapien. Aenean rhoncus augue mattis purus rhoncus venenatis. Suspendisse metus felis, porttitor in varius in, vulputate at tortor. Aliquam molestie, turpis et malesuada porta, tortor sapien pharetra sapien, ac rhoncus quam dolor a sapien. Pellentesque varius laoreet enim ut auctor. Nullam nec ultricies nisi. Nullam porta lectus et ante consectetur posuere. 

% !TEX root = main.tex
% !TEX encoding = Windows Latin 1
% !TEX TS-program = pdflatex
% 
Duis tristique \index{sollicitudin} sollicitudin leo nec consequat. Praesent et dui convallis velit tincidunt fermentum. Mauris cursus purus at sem viverra sed imperdiet sapien imperdiet. Aliquam mattis, elit eget rutrum vulputate, tortor sem pulvinar justo, sit amet mollis felis sem at nibh. Donec malesuada, neque id interdum eleifend, arcu augue porta elit, nec tristique libero metus at massa. Fusce fringilla laoreet rhoncus. Suspendisse potenti. Phasellus dignissim sodales mauris at pharetra. Donec gravida fringilla velit ac rutrum. Curabitur ornare lectus id diam molestie eu imperdiet nulla tempus. Maecenas vestibulum enim et dui ornare blandit. Vivamus \index{fermentum} faucibus viverra. Maecenas at justo sapien. Aenean rhoncus augue mattis purus rhoncus venenatis. Suspendisse metus felis, porttitor in varius in, vulputate at tortor. Aliquam molestie, turpis et malesuada porta, tortor sapien pharetra sapien, ac rhoncus quam dolor a sapien. Pellentesque varius laoreet enim ut auctor. Nullam nec ultricies nisi. Nullam porta lectus et ante consectetur posuere. 

\section{This Is My Second Section}
\noindent
% !TEX root = main.tex
% !TEX encoding = Windows Latin 1
% !TEX TS-program = pdflatex
% 
Duis tristique \index{sollicitudin} sollicitudin leo nec consequat. Praesent et dui convallis velit tincidunt fermentum. Mauris cursus purus at sem viverra sed imperdiet sapien imperdiet. Aliquam mattis, elit eget rutrum vulputate, tortor sem pulvinar justo, sit amet mollis felis sem at nibh. Donec malesuada, neque id interdum eleifend, arcu augue porta elit, nec tristique libero metus at massa. Fusce fringilla laoreet rhoncus. Suspendisse potenti. Phasellus dignissim sodales mauris at pharetra. Donec gravida fringilla velit ac rutrum. Curabitur ornare lectus id diam molestie eu imperdiet nulla tempus. Maecenas vestibulum enim et dui ornare blandit. Vivamus \index{fermentum} faucibus viverra. Maecenas at justo sapien. Aenean rhoncus augue mattis purus rhoncus venenatis. Suspendisse metus felis, porttitor in varius in, vulputate at tortor. Aliquam molestie, turpis et malesuada porta, tortor sapien pharetra sapien, ac rhoncus quam dolor a sapien. Pellentesque varius laoreet enim ut auctor. Nullam nec ultricies nisi. Nullam porta lectus et ante consectetur posuere. \footnote{See also \cite{Aup91,Dou72,Hal82}.}


% !TEX root = main.tex
% !TEX encoding = Windows Latin 1
% !TEX TS-program = pdflatex
% 
Duis tristique \index{sollicitudin} sollicitudin leo nec consequat. Praesent et dui convallis velit tincidunt fermentum. Mauris cursus purus at sem viverra sed imperdiet sapien imperdiet. Aliquam mattis, elit eget rutrum vulputate, tortor sem pulvinar justo, sit amet mollis felis sem at nibh. Donec malesuada, neque id interdum eleifend, arcu augue porta elit, nec tristique libero metus at massa. Fusce fringilla laoreet rhoncus. Suspendisse potenti. Phasellus dignissim sodales mauris at pharetra. Donec gravida fringilla velit ac rutrum. Curabitur ornare lectus id diam molestie eu imperdiet nulla tempus. Maecenas vestibulum enim et dui ornare blandit. Vivamus \index{fermentum} faucibus viverra. Maecenas at justo sapien. Aenean rhoncus augue mattis purus rhoncus venenatis. Suspendisse metus felis, porttitor in varius in, vulputate at tortor. Aliquam molestie, turpis et malesuada porta, tortor sapien pharetra sapien, ac rhoncus quam dolor a sapien. Pellentesque varius laoreet enim ut auctor. Nullam nec ultricies nisi. Nullam porta lectus et ante consectetur posuere. 

% !TEX root = main.tex
% !TEX encoding = Windows Latin 1
% !TEX TS-program = pdflatex
% 
Duis tristique \index{sollicitudin} sollicitudin leo nec consequat. Praesent et dui convallis velit tincidunt fermentum. Mauris cursus purus at sem viverra sed imperdiet sapien imperdiet. Aliquam mattis, elit eget rutrum vulputate, tortor sem pulvinar justo, sit amet mollis felis sem at nibh. Donec malesuada, neque id interdum eleifend, arcu augue porta elit, nec tristique libero metus at massa. Fusce fringilla laoreet rhoncus. Suspendisse potenti. Phasellus dignissim sodales mauris at pharetra. Donec gravida fringilla velit ac rutrum. Curabitur ornare lectus id diam molestie eu imperdiet nulla tempus. Maecenas vestibulum enim et dui ornare blandit. Vivamus \index{fermentum} faucibus viverra. Maecenas at justo sapien. Aenean rhoncus augue mattis purus rhoncus venenatis. Suspendisse metus felis, porttitor in varius in, vulputate at tortor. Aliquam molestie, turpis et malesuada porta, tortor sapien pharetra sapien, ac rhoncus quam dolor a sapien. Pellentesque varius laoreet enim ut auctor. Nullam nec ultricies nisi. Nullam porta lectus et ante consectetur posuere. 

% !TEX root = main.tex
% !TEX encoding = Windows Latin 1
% !TEX TS-program = pdflatex
% 
Duis tristique \index{sollicitudin} sollicitudin leo nec consequat. Praesent et dui convallis velit tincidunt fermentum. Mauris cursus purus at sem viverra sed imperdiet sapien imperdiet. Aliquam mattis, elit eget rutrum vulputate, tortor sem pulvinar justo, sit amet mollis felis sem at nibh. Donec malesuada, neque id interdum eleifend, arcu augue porta elit, nec tristique libero metus at massa. Fusce fringilla laoreet rhoncus. Suspendisse potenti. Phasellus dignissim sodales mauris at pharetra. Donec gravida fringilla velit ac rutrum. Curabitur ornare lectus id diam molestie eu imperdiet nulla tempus. Maecenas vestibulum enim et dui ornare blandit. Vivamus \index{fermentum} faucibus viverra. Maecenas at justo sapien. Aenean rhoncus augue mattis purus rhoncus venenatis. Suspendisse metus felis, porttitor in varius in, vulputate at tortor. Aliquam molestie, turpis et malesuada porta, tortor sapien pharetra sapien, ac rhoncus quam dolor a sapien. Pellentesque varius laoreet enim ut auctor. Nullam nec ultricies nisi. Nullam porta lectus et ante consectetur posuere. 

\begin{table}[h]
  \begin{center}
\begin{tabular}{|r|l|}
  \hline
  7C0 & hexadecimal \\
  3700 & octal \\ \cline{2-2}
  11111000000 & binary \\
  \hline \hline
  1984 & decimal \\
  \hline
\end{tabular}
\caption{This is my first table}
\end{center}
\end{table}

\section{This Is My Third Section}
\noindent
% !TEX root = main.tex
% !TEX encoding = Windows Latin 1
% !TEX TS-program = pdflatex
% 
Duis tristique \index{sollicitudin} sollicitudin leo nec consequat. Praesent et dui convallis velit tincidunt fermentum. Mauris cursus purus at sem viverra sed imperdiet sapien imperdiet. Aliquam mattis, elit eget rutrum vulputate, tortor sem pulvinar justo, sit amet mollis felis sem at nibh. Donec malesuada, neque id interdum eleifend, arcu augue porta elit, nec tristique libero metus at massa. Fusce fringilla laoreet rhoncus. Suspendisse potenti. Phasellus dignissim sodales mauris at pharetra. Donec gravida fringilla velit ac rutrum. Curabitur ornare lectus id diam molestie eu imperdiet nulla tempus. Maecenas vestibulum enim et dui ornare blandit. Vivamus \index{fermentum} faucibus viverra. Maecenas at justo sapien. Aenean rhoncus augue mattis purus rhoncus venenatis. Suspendisse metus felis, porttitor in varius in, vulputate at tortor. Aliquam molestie, turpis et malesuada porta, tortor sapien pharetra sapien, ac rhoncus quam dolor a sapien. Pellentesque varius laoreet enim ut auctor. Nullam nec ultricies nisi. Nullam porta lectus et ante consectetur posuere. 
\begin{figure}[h]
  \begin{center}
    \includegraphics{imagen}
    \caption{This is my first figure}
  \end{center}
\end{figure}


% !TEX root = main.tex
% !TEX encoding = Windows Latin 1
% !TEX TS-program = pdflatex
% 
Duis tristique \index{sollicitudin} sollicitudin leo nec consequat. Praesent et dui convallis velit tincidunt fermentum. Mauris cursus purus at sem viverra sed imperdiet sapien imperdiet. Aliquam mattis, elit eget rutrum vulputate, tortor sem pulvinar justo, sit amet mollis felis sem at nibh. Donec malesuada, neque id interdum eleifend, arcu augue porta elit, nec tristique libero metus at massa. Fusce fringilla laoreet rhoncus. Suspendisse potenti. Phasellus dignissim sodales mauris at pharetra. Donec gravida fringilla velit ac rutrum. Curabitur ornare lectus id diam molestie eu imperdiet nulla tempus. Maecenas vestibulum enim et dui ornare blandit. Vivamus \index{fermentum} faucibus viverra. Maecenas at justo sapien. Aenean rhoncus augue mattis purus rhoncus venenatis. Suspendisse metus felis, porttitor in varius in, vulputate at tortor. Aliquam molestie, turpis et malesuada porta, tortor sapien pharetra sapien, ac rhoncus quam dolor a sapien. Pellentesque varius laoreet enim ut auctor. Nullam nec ultricies nisi. Nullam porta lectus et ante consectetur posuere. 

\begin{figure}[h]
  \begin{center}
    \includegraphics{imagen}
    \caption{This is my second figure}
  \end{center}
\end{figure}

% !TEX root = main.tex
% !TEX encoding = Windows Latin 1
% !TEX TS-program = pdflatex
% 
Duis tristique \index{sollicitudin} sollicitudin leo nec consequat. Praesent et dui convallis velit tincidunt fermentum. Mauris cursus purus at sem viverra sed imperdiet sapien imperdiet. Aliquam mattis, elit eget rutrum vulputate, tortor sem pulvinar justo, sit amet mollis felis sem at nibh. Donec malesuada, neque id interdum eleifend, arcu augue porta elit, nec tristique libero metus at massa. Fusce fringilla laoreet rhoncus. Suspendisse potenti. Phasellus dignissim sodales mauris at pharetra. Donec gravida fringilla velit ac rutrum. Curabitur ornare lectus id diam molestie eu imperdiet nulla tempus. Maecenas vestibulum enim et dui ornare blandit. Vivamus \index{fermentum} faucibus viverra. Maecenas at justo sapien. Aenean rhoncus augue mattis purus rhoncus venenatis. Suspendisse metus felis, porttitor in varius in, vulputate at tortor. Aliquam molestie, turpis et malesuada porta, tortor sapien pharetra sapien, ac rhoncus quam dolor a sapien. Pellentesque varius laoreet enim ut auctor. Nullam nec ultricies nisi. Nullam porta lectus et ante consectetur posuere. 

% !TEX root = main.tex
% !TEX encoding = Windows Latin 1
% !TEX TS-program = pdflatex
% 
Duis tristique \index{sollicitudin} sollicitudin leo nec consequat. Praesent et dui convallis velit tincidunt fermentum. Mauris cursus purus at sem viverra sed imperdiet sapien imperdiet. Aliquam mattis, elit eget rutrum vulputate, tortor sem pulvinar justo, sit amet mollis felis sem at nibh. Donec malesuada, neque id interdum eleifend, arcu augue porta elit, nec tristique libero metus at massa. Fusce fringilla laoreet rhoncus. Suspendisse potenti. Phasellus dignissim sodales mauris at pharetra. Donec gravida fringilla velit ac rutrum. Curabitur ornare lectus id diam molestie eu imperdiet nulla tempus. Maecenas vestibulum enim et dui ornare blandit. Vivamus \index{fermentum} faucibus viverra. Maecenas at justo sapien. Aenean rhoncus augue mattis purus rhoncus venenatis. Suspendisse metus felis, porttitor in varius in, vulputate at tortor. Aliquam molestie, turpis et malesuada porta, tortor sapien pharetra sapien, ac rhoncus quam dolor a sapien. Pellentesque varius laoreet enim ut auctor. Nullam nec ultricies nisi. Nullam porta lectus et ante consectetur posuere. 

% !TEX root = main.tex
% !TEX encoding = Windows Latin 1
% !TEX TS-program = pdflatex
% 
Duis tristique \index{sollicitudin} sollicitudin leo nec consequat. Praesent et dui convallis velit tincidunt fermentum. Mauris cursus purus at sem viverra sed imperdiet sapien imperdiet. Aliquam mattis, elit eget rutrum vulputate, tortor sem pulvinar justo, sit amet mollis felis sem at nibh. Donec malesuada, neque id interdum eleifend, arcu augue porta elit, nec tristique libero metus at massa. Fusce fringilla laoreet rhoncus. Suspendisse potenti. Phasellus dignissim sodales mauris at pharetra. Donec gravida fringilla velit ac rutrum. Curabitur ornare lectus id diam molestie eu imperdiet nulla tempus. Maecenas vestibulum enim et dui ornare blandit. Vivamus \index{fermentum} faucibus viverra. Maecenas at justo sapien. Aenean rhoncus augue mattis purus rhoncus venenatis. Suspendisse metus felis, porttitor in varius in, vulputate at tortor. Aliquam molestie, turpis et malesuada porta, tortor sapien pharetra sapien, ac rhoncus quam dolor a sapien. Pellentesque varius laoreet enim ut auctor. Nullam nec ultricies nisi. Nullam porta lectus et ante consectetur posuere. 

% !TEX root = main.tex
% !TEX encoding = Windows Latin 1
% !TEX TS-program = pdflatex
% 
Duis tristique \index{sollicitudin} sollicitudin leo nec consequat. Praesent et dui convallis velit tincidunt fermentum. Mauris cursus purus at sem viverra sed imperdiet sapien imperdiet. Aliquam mattis, elit eget rutrum vulputate, tortor sem pulvinar justo, sit amet mollis felis sem at nibh. Donec malesuada, neque id interdum eleifend, arcu augue porta elit, nec tristique libero metus at massa. Fusce fringilla laoreet rhoncus. Suspendisse potenti. Phasellus dignissim sodales mauris at pharetra. Donec gravida fringilla velit ac rutrum. Curabitur ornare lectus id diam molestie eu imperdiet nulla tempus. Maecenas vestibulum enim et dui ornare blandit. Vivamus \index{fermentum} faucibus viverra. Maecenas at justo sapien. Aenean rhoncus augue mattis purus rhoncus venenatis. Suspendisse metus felis, porttitor in varius in, vulputate at tortor. Aliquam molestie, turpis et malesuada porta, tortor sapien pharetra sapien, ac rhoncus quam dolor a sapien. Pellentesque varius laoreet enim ut auctor. Nullam nec ultricies nisi. Nullam porta lectus et ante consectetur posuere. 

% !TEX root = main.tex
% !TEX encoding = Windows Latin 1
% !TEX TS-program = pdflatex
% 
Duis tristique \index{sollicitudin} sollicitudin leo nec consequat. Praesent et dui convallis velit tincidunt fermentum. Mauris cursus purus at sem viverra sed imperdiet sapien imperdiet. Aliquam mattis, elit eget rutrum vulputate, tortor sem pulvinar justo, sit amet mollis felis sem at nibh. Donec malesuada, neque id interdum eleifend, arcu augue porta elit, nec tristique libero metus at massa. Fusce fringilla laoreet rhoncus. Suspendisse potenti. Phasellus dignissim sodales mauris at pharetra. Donec gravida fringilla velit ac rutrum. Curabitur ornare lectus id diam molestie eu imperdiet nulla tempus. Maecenas vestibulum enim et dui ornare blandit. Vivamus \index{fermentum} faucibus viverra. Maecenas at justo sapien. Aenean rhoncus augue mattis purus rhoncus venenatis. Suspendisse metus felis, porttitor in varius in, vulputate at tortor. Aliquam molestie, turpis et malesuada porta, tortor sapien pharetra sapien, ac rhoncus quam dolor a sapien. Pellentesque varius laoreet enim ut auctor. Nullam nec ultricies nisi. Nullam porta lectus et ante consectetur posuere. 

% Fin archivo ch01.tex
\endinput 