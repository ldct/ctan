%\iffalse
%<+thesis>\def\filename{thesis}
%<+thema>\def\filename{thema}
%\fi
\def\fileversion{1.0g}
\def\filedate{1996/25/01}
\def\docdate{1995/25/01}
% \CheckSum{2689}
%
% \iffalse    This is a META-COMMENT
%
% Copyright (C) 1991, 1996 by Wenzel Matiaske, mati1831@perform.ww.tu-berlin.de
%
% This file is to be used with the LaTeX2e system.
% ------------------------------------------------
%
% This macro is free software; you can redistribute it and/or modify it
% under the terms of the GNU General Public License as published by the
% Free Software Foundation; either version 1, or (at your option) any
% later version.
%
% The macros and the documentation are distributed in the hope that they
% will be useful, but WITHOUT ANY WARRANTY; without even the implied
% warranty of MERCHANTABILITY or FITNESS FOR A PARTICULAR PURPOSE.  See
% the GNU General Public License for more details.
%
% You should have received a copy of the GNU General Public License
% along with this program; if not, write to the Free Software
% Foundation, Inc., 675 Mass Ave, Cambridge, MA 02139, USA.
%
% There are undoubtably bugs in the macros or the documentation. Should
% you make improvements, bug fixes, etc., however, I ask you to send 
% improvements back to me for incorporation into the macro for the 
% rest of us. 
%
% Updates are available via anonymous ftp to host `perform.ww.tu-berlin.de'.
%
%                                ___      
%   wenzel matiaske  |           / /_/-Berlin
%                    |  mail:   Technical University Berlin
%                    |          Dept. of Economics, WW6
%                    |          Uhlandstr. 4-5, D-10623 Berlin
%                    |  phone:  +49 30 314-22574
%                    |  email:  mati1831@perform.ww.tu-berlin.de
%
%  \fi
%
% \MakeShortVerb{\|}
%
%    \ifsolodoc
%      \title{Die \LaTeX-Stile \texttt{thesis} und \texttt{thema}}
%      \author{Wenzel Matiaske\thanks{TU-Berlin, FB 14, WW 6,
%      Uhlandstr. 4--5, 10623~Berlin, Tel.~030-314\,225\,74, email:
%      mati1831@perform.ww.tu-berlin.de.}} 
%      \date{\docdate}
%      \maketitle
%      \selectlanguage{\english}
%      \def\localin{\par}
%      \begin{small}
%      \begin{center}\small\textbf{Abstract}\end{center}
%      \MakePercentComment % local01.tex -- updated for thesis1e 15 Apr 1996
%                updated for LaTeX2e 11 Dec 1994
%                first released 1 Sept 1993
%
% Copyright (C) 1993, 1996 by Wenzel Matiaske, mati1831@perform.ww.tu-berlin.de
%
% As input to the local LaTeX-guide "`local.tex"'.
%
% For distribution of this document see the copyright notice in the
% original sources mentioned below.
%

\makeatletter
\newif\iflocalin
\newif\ifappendixin
\@ifundefined{localin}{\localintrue}{\localinfalse}
\@ifundefined{appendixin}{\appendixinfalse}{\appendixintrue}
\@ifundefined{docdir}{\def\docdir{\dots /emtex/doc/}}{}
\makeatother

\iflocalin
   \subsubsection{The style files \texttt{thesis} and \texttt{thema}}
\fi
\ifappendixin
  \subsection{The style files \texttt{thesis} and \texttt{thema}}
\fi

The style files \verb|thesis| and \verb|thema| base on the standard
layout \verb|report|. In contrast to this standard document style the
outlook of these styles is more European and more flexible. The
layout may easily be changed via options like (defaults are underlined):

\begin{description}
\item[\underline{\texttt{indent}} or \texttt{noindent}] (no) indent of
  paragraphs and footnotes,

\item[\texttt{itemization} or \underline{\texttt{noitemization}}] (no)
  bullets, stars etc. in the \texttt{itemize} environ\-ment,

\item[\underline{\texttt{enumeration}} or \texttt{noenumeration}]
  alphanumeric or numeric \texttt{enumerate} environ\-ment,

\item[\underline{\texttt{headline}} or \texttt{noheadline},
  \texttt{headcount} or \underline{\texttt{noheadcount}}] change the
  layout of the headings,

\item[\underline{\texttt{center}} or \texttt{nocenter},
  \underline{\texttt{upper}} or \texttt{noupper}] change the style of the
  sectioning,

\item[\underline{\texttt{slanted}} or \texttt{bold} or
  \texttt{sfbold}] change the fonts used in chapters, titles or
  headings,

\item[\texttt{crosshair}] marks empty pages with a crosshair, which is
  useful for camera reproducing of the output, and

\item[\texttt{chapterbib}] for chapter bibliographies with \BibTeX.

\end{description}


For default, the text ``Chapter'' is not printed in front of a chapter. 
If you want a more standard \LaTeX-layout define \verb|\chapapp{|{\em
Chapter}\verb|}| in the preamble.

The style defines an advanced section of commands for the title page. The
commands \verb|\subtitle{|\emph{text}\verb|}| and
\verb|\translator{|\emph{text}\verb|}| take some text for the
title page; \verb|\dedication{|\emph{text}\verb|}| produce a separate
dedication page, and notes on the backside of the title page
\verb|\uppertitleback{|\emph{text}\verb|}|, \verb|\middletitleback|,
and \verb|\lowertitleback| may be used.

The style file \texttt{thema} defines some more commands, which may be
useful in collections. The command \verb|\chapterauthors{|{\em
authors}\verb|}| takes the authors as an argument, which are printed
in the chapter title and in the table of contents. If you want to put the
author into the headline of the odd side pages the command
\verb|\shortauthor{|\emph{text}\verb|}| may be used. This command must
be specified before the corresponding \verb|\chapter|. The environment
\texttt{chapterabstract} prints the abstract of the chapter. Especially
with \texttt{thema} the style option \texttt{chapterbib} may be useful. This
option allows \BibTeX{} to produce a bibliography for every file which
is read by the command \verb|\include{|\emph{file}\verb|}| in a control
file.

The fonts used in \texttt{thesis} and \texttt{thema} may be changed
with the following commands which take a font declaration as
argument:

\begin{center}\small
\begin{tabular}{@{}ll@{}}
\verb|\partfont{|\emph{decl}\verb|}|            &\verb|\theorembodyfont{|\emph{decl}\verb|}|  \\
\verb|\chapterfont{|\emph{decl}\verb|}|         &\verb|\itemfont{|\emph{decl}\verb|}|         \\  
\verb|\chapterauthorfont{|\emph{decl}\verb|}|   &\verb|\examplefont{|\emph{decl}\verb|}|      \\  
\verb|\sectionfont{|\emph{decl}\verb|}|         &\verb|\headingstextfont{|\emph{decl}\verb|}|\\   
\verb|\subsectionfont{|\emph{decl}\verb|}|      &\verb|\pagenumberfont{|\emph{decl}\verb|}|\\     
\verb|\subsubsectionfont{|\emph{decl}\verb|}|   &\verb|\captionheaderfont{|\emph{decl}\verb|}|\\  
\verb|\paragraphfont{|\emph{decl}\verb|}|       &\verb|\captionbodyfont{|\emph{decl}\verb|}|\\    
\verb|\subparagraphfont{|\emph{decl}\verb|}|    &\verb|\figurefont{|\emph{decl}\verb|}|\\         
\verb|\titlefont{|\emph{decl}\verb|}|           &\verb|\tablefont{|\emph{decl}\verb|}|\\          
\verb|\subtitlefont{|\emph{decl}\verb|}|        &\verb|\indexsize{|\emph{decl}\verb|}|\\       
\verb|\authorfont{|\emph{decl}\verb|}|          &\verb|\bibsize{|\emph{decl}\verb|}|\\          
\verb|\translatorfont{|\emph{decl}\verb|}|      &\verb|\theoremheaderfont{|\emph{decl}\verb|}|\\
\verb|\institutionfont{|\emph{decl}\verb|}|     &\\
\end{tabular}
\end{center}

\iflocalin
For more details see the German documentation
\file{\docdir thesis}.
\fi


 \MakePercentIgnore
%      \end{small}
%      \newpage
%      \section{Einleitung}
%      \selectlanguage{\german}
%    \else 
%      \section{B\"ucher und Sammelwerke}
%    \fi
% 
%    Die hier vorgestellten
%    \LaTeX-Stile\footnote{Version\fileversion\ vom
%    \filedate. Dokumentation vom \docdate.} basieren auf dem
%    Grundstil \texttt{report.cls} und sind somit zum ein- und
%    doppelseitigen Satz von l\"angeren Arbeiten wie B\"uchern
%    geeignet. Gemeinsam ist den Formaten, da\ss{} sie einige Optionen
%    zur flexibleren Gestaltung des Layouts beinhalten. Im Unterschied
%    zur Stilart \texttt{thesis} stellt \texttt{thema} zus\"atzlich
%    einige Befehle zur Verf\"ugung, die bei der Erstellung von
%    Sammelwerken n\"utzlich sind.  
%
%   \begin{figure}[h!]
%   \begin{verbatim}
%\documentclass[12pt,nocenter,sfbold]{thema}
%\chapapp{Chapter}
%\begin{document}
% 
%\title{Fighting Fire with Fire} \subtitle{Festooning French Phrases}
%\translator{PhD Dissertation}  \author{F. Phidias Phony-Baloney}
%\institution{Fanstord University}
%\maketitle
%...
%\end{document}
%   \end{verbatim}
%   \caption{\label{thetest}Anwendungsbeispiel der Stilart \texttt{thema}}
%   \end{figure}
%
%
%   \ifsolodoc 
%        \section{B\"ucher mit \texttt{thesis}} 
%   \else
%       \subsection{Die Stilart \texttt{thesis}} 
%   \fi
%
%   \ifsolodoc 
%        \subsection{Zus\"atzliche Stiloptionen} 
%   \else
%       \subsubsection{Zus\"atzliche Stiloptionen} 
%   \fi
% 
%   Die wichtigsten Ver\"anderungen gegen\"uber dem Grundstil
%   \texttt{report.cls} betreffen das Layout und sind
%   benutzerfreundlich zu Optionen geb\"undelt. Sie betreffen die
%   Absatzformatierung, die Gestaltung von Listenumgebungen und
%   Kopfzeilen sowie das Layout der \"Uberschriften. Die neuen
%   Optionen sind in Tabelle \ref{theoptions} zusammengestellt. 
%   Die jeweiligen Voreinstellungen sind in der \"Ubersicht
%   unterstrichen. Werden beispielsweise linksb\"undige
%   \"Uberschriften in fetten serifenlosen Typen gew\"unscht, sind die
%   Optionen \texttt{nocenter} und \texttt{sfbold} im optionalen Argument
%   des Kommandos  |\documentstyle[|\emph{options}|]{|\emph{stil}|}|
%   zu spezifizieren, wie Anwendungsbeispiel \ref{thetest} zeigt. 
%
%   \begin{table}[ht] \small \centering
%      \caption{\label{theoptions}Zus\"atzliche Stiloptionen}
%      \begin{tabular}{lp{20em}}
%      \hline
%      \underline{|indent|} & Einzug zu Beginn von Abs\"atzen oder
%                             Fu\ss{}notenabs\"atzen (\LaTeX-Standard). \\
%      |noindent|           & Kein Einzug derselben.              \\
%      |itemize|            & Hervorhebung der Staffelung in 
%                             \texttt{itemize} Umgebungen durch 
%                             verschiedene Zeichen wie z.~B. |\bullet|,
%                             |\star| etc. (\LaTeX-Standard).      \\
% \underline{noitemize}      &   Keine Hervorhebung der "`itemization"'.  \\
% \underline{enumerate}      &   Alphanumerische Z\"ahlung in
%                               \texttt{enumerate} Umgebungen
%                               (\LaTeX-Standard).      \\
%   |noenumerate|            &   Dekadische Z\"ahlung.  \\
% \underline{|headline|}     &   Kopfzeilen werden unterstrichen.      \\
%  |noheadline|              &   Keine Unterstreichung derselben.       \\
%  |headcount|               &   Kapitel und Abschnittsnumerierung
%                                werden in Kopfzeilen ausgegeben.       \\
% \underline{|noheadcount|}  &   Kopfzeilen ohne diese Z\"ahler.        \\
% \underline{|center|}       &   \"Uberschriften und Kopfzeilen zentriert. \\
% |nocenter|                 &   \"Uberschriften und Kopfzeilen
%                                linksb\"undig.\\
% \underline{|upper|}        &   Kapitel\"uberschriften in
%                                Gro\ss{}buchstaben.  \\
%  |noupper|                 &   Keine Umwandlung der Kapitel\"uberschriften 
%                                in Gro\ss{}buchstaben.      \\
% \underline{|slanted|}      &   Gr\"o\ss{}ere/geneigte Typen 
%                                in \"Uberschriften,
%                                Kopf"-zei"-len, Li"-sten und Titelei. \\
% |bold|                     &   Gr\"o\ss{}ere/fette Typen. \\
% |sfbold|                   &   Gr\"o\ss{}ere/fette serifenlose
%                                Typen. \\ 
% |chapterbib|               &   Kapitelweise
%                                Literaturverzeichnisse 
%                                in Zusammen"-arbeit
%                                mit \BibTeX. \\
% |crosshair|                &  Markierung am oberen Seitenrand.   \\
% \hline
% \end{tabular}
% \end{table}
%
%  
% \DescribeMacro{\chapapp}
% \DescribeMacro{\appendix}
%    Die Kapitel\"uberschriften werden hier standardm\"a\ss{}ig ohne
%    die Angabe des Kapitelnamens "`Chapter"' oder "`Kapitel"'
%    gesetzt. Wird die Ausgabe der Bezeichnung gew\"unscht, kann
%    dies durch Spezifikation des neuen Befehls
%    |\chapapp{|\emph{text}|}| erreicht werden;
%    z.~B. |\chapapp{\chaptername}| im Vorspann des Dokumentes. 
%
%    Dies hat auch Auswirkung auf die Formatierung des Anhanges.
%    Per Voreinstellung wird der Anhang durch eine Teil\"uberschrift,
%    d.~h. eine eigene Seite, vom \"ubrigen Text abgesetzt. Wird
%    |\chapapp| dagegen spezifiziert, unterbleibt die
%    Teil\"uberschrift und stattdessen wird wie \"ublich der
%    Kapitelname "`Appendix"' bzw. "`Anhang"' ausgegeben.
%
% \pagebreak[4]
%   \ifsolodoc 
%        \subsection{Erweiterungen der Titelei} 
%    \else
%       \subsubsection{Erweiterungen der Titelei} 
%    \fi
%
% \DescribeMacro{\subtitle}
% \DescribeMacro{\translator}
% \DescribeMacro{\institution}
% \DescribeMacro{\dedication}
%     Die Titelei ist um einige Kommandos erweitert worden. Der Angabe
%     eines Untertitels dient der Befehl |\subtitle{|\emph{text}|}|.
%     Der Untertitel wird in kleineren Typen unterhalb des eigentlichen 
%     |\title{|\emph{text}|}| gesetzt. Der den Kommandos
%     |\translator{|\emph{text}|}| bzw. 
%     |\institution{|\emph{text}|}|
%     \"ubergebene Text wird auf der Titelseite vertikal zentriert
%     bzw. an deren Ende ausgegeben. Eine Widmung, die auf einer
%     separaten Seite plaziert wird, erzeugt der Befehl
%     |\dedication{|\emph{text}|}|. Diese Kommandos behalten sowohl
%     bei einseitigem als auch bei zweiseitigem Druck ihre Wirkung. 
%
% \DescribeMacro{\uppertitleback}
% \DescribeMacro{\middletitleback}
% \DescribeMacro{\lowertitleback}
%     Bei zweiseitigem Druck werden nicht nur die \"ubliche Titelseite
%     und ggf. eine Widmung, sondern zus\"atzlich eine Umschlagsseite 
%     und ein sogenannter Schmutztitel erzeugt. Ausschlie\ss{}lich
%     f\"ur zweiseitigen Druck sind auch die  Kommandos
%     (|\uppertitleback{|\emph{text}|}|,
%     |\middletitleback{|\emph{text}|}| und 
%     |\lowertitleback{|\emph{text}|}| konstruiert, 
%     die dem \texttt{script.sty} entnommen sind \cite{neukam:92}.
%     Sie dienen dem Satz von Verlags-, Copyright- oder ISBN-Angaben
%     auf der R\"uckseite des Titels. Bei zweiseitigem Druck beginnt die
%     Seitenz\"ahlung nach der Titelei standardm\"a\ss{}ig mit der f\"unften
%     Seite.
%
%     \ifsolodoc 
%          \subsection{Geringf\"ugige \"Anderungen} 
%     \else
%        \subsubsection{Geringf\"ugige \"Anderungen} 
%     \fi
%
%     Ferner wurden einige geringf\"ugige Ver\"anderungen vorgenommen.
%     Diese betreffen insbesondere die Option  \texttt{draft}, die nun
%     eine besondere Kopfzeile erzeugt, in der Datum, Zeit und 
%     Name des Jobs ausgegeben werden. Eine zur Reproduktion des
%     Ausdrucks besonders n\"utzliche Option wurde der Stilvariante
%     \texttt{svma.sty} des Springer Verlages \cite{sznyter:87}
%     entnommen. Die Option \texttt{crosshair} f\"ugt auf leeren
%     Seiten eine Markierung am oberen Seitenrand ein. Im Befehl 
%     |\caption{|\emph{text}|}| wird ein Kurzname wie "`Fig."'
%     bzw. "`Tab."' benutzt, und dieser Name wird bei der
%     Absatzformatierung langer Beschriftungen ausgespart. Einige
%     weitere Kommandos, die die Erstellung von Tabellen und
%     speziellen Randnotizen unterst\"utzen, werden im Zusammenhang mit
%     der Stiloption \texttt{etc} beschrieben\ifsolodoc\else\ 
%              (Vgl. \ref{tabcmd}, \ref{see})\fi.
%    Schlie\ss{}lich sind die Voreinstellungen dahingehend 
%    ge\"andert, da\ss{} der |\pagestyle{|\emph{headings}|}| die 
%    Regel ist.
%
% \ifsolodoc 
%     \subsection{Einstellung der Schriftarten} 
% \else
%    \subsubsection{Einstellung der Schriftarten} 
% \fi
%
%    Die in den Stiloptionen \texttt{slanted}, \texttt{bold} und
%    \texttt{sfbold} voreingestellten Schriften k\"onnen bei Bedarf
%    modifiziert werden. Zu diesem Zweck stehen vier Schnittstellen
%    zur Verf\"ugung, welche die gew\"unschten Schriftarten als
%    Argumente \"ubernehmen. Die  Befehle sind in Tabelle
%    \ref{schriftwahl} zusammengestellt. 
%
%\begin{table}\centering \small
%\caption{\label{schriftwahl}Einstellung der Schriftarten}
%\begin{tabular}{@{}ll@{}}
%\hline
%|\partfont{|\emph{decl}|}|            &|\theorembodyfont{|\emph{decl}|}|  \\
%|\chapterfont{|\emph{decl}|}|         &|\itemfont{|\emph{decl}|}|         \\  
%|\chapterauthorfont{|\emph{decl}|}|   &|\examplefont{|\emph{decl}|}|      \\  
%|\sectionfont{|\emph{decl}|}|         &|\headingstextfont{|\emph{decl}|}|\\   
%|\subsectionfont{|\emph{decl}|}|      &|\pagenumberfont{|\emph{decl}|}|\\     
%|\subsubsectionfont{|\emph{decl}|}|   &|\captionheaderfont{|\emph{decl}|}|\\  
%|\paragraphfont{|\emph{decl}|}|       &|\captionbodyfont{|\emph{decl}|}|\\    
%|\subparagraphfont{|\emph{decl}|}|    &|\figurefont{|\emph{decl}|}|\\         
%|\titlefont{|\emph{decl}|}|           &|\tablefont{|\emph{decl}|}|\\          
%|\subtitlefont{|\emph{decl}|}|        &|\indexsize{|\emph{decl}|}|\\       
%|\authorfont{|\emph{decl}|}|          &|\bibsize{|\emph{decl}|}|\\          
%|\translatorfont{|\emph{decl}|}|      &|\theoremheaderfont{|\emph{decl}|}|\\
%|\institutionfont{|\emph{decl}|}|     &\\
%\hline
%\end{tabular}
%\end{table}
%
%
% Der Befehl |\chapterfont| ist f\"ur die Variante \texttt{slanted}
% beispielsweise folgenderma\ss{}en definiert als |\chapterfont{large}|.
%
% \ifsolodoc \section{Sammelwerke mit \texttt{thema}} \else
%    \subsection{Die Stilart \texttt{thema}} \fi
%
% \DescribeMacro{\chapterauthor}
% \DescribeMacro{\shortauthor}
% \DescribeEnv{chapterabstract}
%     Die Stilart \texttt{thema} stellt zus\"atzlich einige Kommandos
%     zur Verf\"ugung,  die bei der Erstellung von Sammelwerken
%     n\"utzlich sind. Dies sind zum einen  der Befehl
%     |\chapterauthor{|\emph{text}|}|,  der den Satz der Autoren eines
%     Kapitels \"ubernimmt. Sollen die Autoren ins Inhaltsverzeichnis
%     und in die Kopfzeilen der geraden Seiten \"ubernommen werden,
%     ist  zus\"atzlich das Kommando |\shortauthor{|\emph{text}|}| zu
%     spezifizieren. Wichtig ist, da\ss{} das Kommando \emph{vor} der
%     mit |\chapter| spezifizierten \"Uberschrift angegeben werden
%     mu\ss{}; sonst  wird es erst im folgenden Kapitel wirksam. Die
%     neue Umgebung |\chapterabstract| erlaubt eine Zusammenfassung im
%     jeweiligen Kapitel.
%
%    In diesem Zusammenhang ist auch auf die zus\"atzliche Option
%    \texttt{chapterbib} hinzuweisen, die dem Substil
%    \texttt{bibperinclude.sty} entspricht
%    \label{chapterbib}\cite{schrod:91}. Diese Option erm\"oglicht die
%     Erstellung mehrerer Literaturverzeichnisse mittels \BibTeX{} in
%     einem Dokument, was insbesondere im Fall von Sammelwerken
%     vorteilhaft sein kann. Zu diesem Zweck sind die Kapitel in
%     getrennten Dateien zu halten und mittels
%     |\include{|\emph{datei}|}| Befehlen in einer Formatierdatei zu
%     b\"undeln\ifsolodoc\else\ (Vgl. auch die Formatierdatei im
%     Anwendungsbeispiel \ref{jourtest})\fi.  Am Ende jedes Kapitels
%     sind die Kommandos |\bibliographystyle{|\emph{stil}|}| und
%     |\bibliography{|\emph{datenbanken}|}|  anzugeben. \BibTeX{}
%     kann nun  jedes Kapitel getrennt bearbeiten. Im Anschlu\ss{}
%     erzeugt \LaTeX{} ein Literaturverzeichnis f\"ur das jeweilige
%     Kapitel.
%
% 
% \StopEventually{\ifsolodoc
% \begin{thebibliography}{1}
%
% \bibitem{neukam:92}  Neukam, Frank (1992).
% \newblock Die Document-Style-Familie "`Script"', Vers. 1.0, 6. Juni 1992.
%
% \bibitem{schrod:91}  Schrod, Joachim (1991).
%    \newblock  Document Style Option `bibperinclude.sty', Vers. 25 Juni 1991.
%
% \bibitem{sznyter:87}  Sznyter, Ed (1987).
%    \newblock Springer-Verlag Document Styles for \LaTeX, November 1, 1987.
%
% \end{thebibliography}
%  \fi}
%
%    \ifsolodoc 
%        \section{Implementation} 
%    \else
%       \subsection{Implementation} 
%    \fi
%
%
% \changes{thema~0.9a}{1991/08/01}{%
%          \texttt{thema.sty}.}
% \changes{thesis~0.9b}{1992/01/15}{%
%          Dokumentation in \texttt{docstyle} Format.}
% \changes{thesis~0.9c}{1993/07/01}{%
%          Draft Headings, Schriftwahl als Optionen., 
%          neue Kommandos \texttt{thickhline} und \texttt{doublehline}}
% \changes{thesis~0.9d}{1993/08/01}{%
%          Redefinition des \texttt{and} Kommandos.}
% \changes{thesis~0.9e}{1993/10/15}{%
%          Bug Fix in \texttt{makechapter}.}
% \changes{thesis~1.0a}{1995/05/20}{%
%          \texttt{thesis.cls} Update auf \LaTeX3e.}
% \changes{thesis~1.0b}{1995/10/25}{%
%         Vorschub in \texttt{caption}.}
% \changes{thesis~1.0c}{1996/02/21}{%
%          Fehlende Schriftart in \texttt{sfbold}-Variante erg\"anzt.}
% \changes{thesis~1.0d}{1996/04/10}{%
%          Abstand in Kopfzeilen korrigiert}
% \changes{thesis~1.0e}{1996/04/15}{%
%          Schreibfehler \texttt{translator} verbessert.}
% \changes{thesis~1.0f}{1996/05/16}{%
%          Kapiteleintr\"age ins Inhaltsverzeichnis auch ohne
%          \texttt{shortauthor} (\texttt{thema}).}
% \changes{thesis~1.0g}{1996/07/01}{%
%          Fehler in \texttt{appendix} korrigiert. (Uwe Schellhorn)}
% \changes{thesis~1.0g}{1996/07/25}{%
%          Fehler in \texttt{chapter} korrigiert. (Uwe Schellhorn)}
% \changes{thesis~1.0g}{1996/07/25}{%
%          Dokumentation (no)itemize, (no)enumerate korrigiert.%%
%         (Klaus Bergner)}
%
%    Die Implementation enth\"alt den Code f\"ur die Klassen \texttt{thesis}
%    und \texttt{thema} sowie f\"ur die kompatiblen Style-Files.
%
%    \begin{macrocode}
%<*thesis|thesis.sty|thema|thema.sty>
\NeedsTeXFormat{LaTeX2e}
%</thesis|thesis.sty|thema|thema.sty>
%    \end{macrocode}
%
%    Die kompatiblen Styles laden die zugeh\"orige Klasse.
%
%    \begin{macrocode}
%<*thesis.sty>
\@obsoletefile{thesis.cls}{thesis.sty}
\LoadClass{thesis}
%</thesis.sty>
%<*thema.sty>
\@obsoletefile{thema.cls}{thema.sty}
\LoadClass{thema}
%</thema.sty>
%    \end{macrocode}
%
%    Im Fall der Klasse \texttt{thema} wird eine
%    Startmeldung ausgegeben, die Optionen werden der zugeh\"origen 
%    Klasse |thesis| \"ubergeben und diese wird eingelesen.
%
%   \begin{macrocode}
%<*thema>
\ProvidesClass{thema}[\filedate\space\fileversion\space%
       LaTeX document class (wm).]
\DeclareOption*{\PassOptionsToClass{\CurrentOption}{thesis}}
\ProcessOptions
\LoadClass[thema]{thesis}
%</thema>
%    \end{macrocode}
%
%    Die Implementation des Hauptstils \texttt{thesis.cls} beginnt 
%    mit der Startmeldung f\"ur die Klasse \texttt{thesis.cls}.
%
%    \begin{macrocode}
%<*thesis>
\ProvidesClass{thesis}[\filedate\space\fileversion\space%
       LaTeX document class (wm).]
%    \end{macrocode}
%
%
%    Es folgen weitere Definitionen und Initialisierungen, die den 
%    \LaTeXe{} Standard Klassen entnommen sind.
%
%    Das Kommando kontrolliert die Schriftgr\"o\ss{}e.
%    \begin{macrocode}
\newcommand\@ptsize{}
%    \end{macrocode}
%
%
%    Schalter, um zwischen zwei- und einspaltigem Satz zu wechseln.
%    \begin{macrocode}
\newif\if@restonecol
%    \end{macrocode}
%
%    Schalter, um die Erzeugung einer Titelseite anzuzeigen.
%    \begin{macrocode}
\newif\if@titlepage 
\@titlepagetrue
%    \end{macrocode}
%
%    Schalter, der anzeigt, ob Kapitel auf der rechten Seite beginnen.
%    \begin{macrocode}
\newif\if@openright
%    \end{macrocode}
%
%    Schalter f\"ur "`offenes"' oder "`geschlossenes"' Format der
%    Bibliographie. 
%
%    \begin{macrocode}
\newif\if@openbib 
\@openbibfalse
%    \end{macrocode}
%
%    Schalter, der anzeigt, ob der Hauptteil des Buches produziert wird.
%
%    \begin{macrocode}
\newif\if@mainmatter \@mainmattertrue
%    \end{macrocode}
%
%    Definitionen der Papierformate
%    \begin{macrocode}
\DeclareOption{a4paper}
   {\setlength\paperheight {297mm}%
    \setlength\paperwidth  {210mm}}
\DeclareOption{a5paper}
   {\setlength\paperheight {210mm}%
    \setlength\paperwidth  {148mm}}
\DeclareOption{b5paper}
   {\setlength\paperheight {250mm}%
    \setlength\paperwidth  {176mm}}
\DeclareOption{letterpaper}
   {\setlength\paperheight {11in}%
    \setlength\paperwidth  {8.5in}}
\DeclareOption{legalpaper}
   {\setlength\paperheight {14in}%
    \setlength\paperwidth  {8.5in}}
\DeclareOption{executivepaper}
   {\setlength\paperheight {10.5in}%
    \setlength\paperwidth  {7.25in}}
%    \end{macrocode}
%
%    Die Option \texttt{landscape} tauscht die Werte f\"ur Seitenh\"ohe 
%    und Seitenbreite.
%    \begin{macrocode}
\DeclareOption{landscape}
   {\setlength\@tempdima   {\paperheight}%
    \setlength\paperheight {\paperwidth}%
    \setlength\paperwidth  {\@tempdima}}
%    \end{macrocode}
%
%    Optionen f\"ur Schriftgr\"o\ss{}en.
%
%    \begin{macrocode}
\DeclareOption{10pt}{\renewcommand\@ptsize{0}}
\DeclareOption{11pt}{\renewcommand\@ptsize{1}}
\DeclareOption{12pt}{\renewcommand\@ptsize{2}}
%    \end{macrocode}
%
%    Zwei oder einseitiger Druck.
%    \begin{macrocode}
\DeclareOption{oneside}{\@twosidefalse \@mparswitchfalse}
\DeclareOption{twoside}{\@twosidetrue  \@mparswitchtrue}
%    \end{macrocode}
%
%    Definitionen f\"ur die \texttt{draft} Option. Die Makros 
%    |\SetTime| und |\now| sind aus \texttt{tugboat.com}
%    \"ubernommen. Diese werden in der \texttt{draft} Option zur 
%    Gestaltung der Kopfzeile benutzt. Ferner wird ein unmaskierter 
%    Schalter |\iffinal| definiert, der standardm\"a\ss{}ig wahr, im 
%    Fall der Option \texttt{draft} dagegen falsch ist. 
%
%    \begin{macrocode}
\newcount\hours \newcount\minutes
\def\SetTime{\hours=\time
        \global\divide\hours by 60
        \minutes=\hours
        \multiply\minutes by 60
        \advance\minutes by-\time
        \global\multiply\minutes by-1 }
\def\now{\number\hours:\ifnum\minutes<10 0\fi\number\minutes}
\newif\iffinal \finaltrue
\DeclareOption{draft}{\setlength\overfullrule{5pt}\finalfalse \SetTime}
\DeclareOption{final}{\setlength\overfullrule{0pt}\finaltrue}
%    \end{macrocode}
%
%    Definition des Schalters |\if@thema|. Der Zustand dieses
%    Schalters ist von der Option der |thema| 
%    abh\"angig und steuert im folgenden die Auswahl der Makros.
%
%    \begin{macrocode}
\newif\if@thema       \@themafalse
\DeclareOption{thema}{\@thematrue}
%    \end{macrocode}
%
%    Die Optionen initialisiert die Variable |option@crosshair|,
%    die abgefragt wird, um auf leeren Seiten eine Markierung des 
%    Seitenkopfs einzuf\"ugen.
%
%    \begin{macrocode}
\newif\if@crosshair       \@crosshairfalse
\DeclareOption{crosshair}{\@crosshairtrue}
\DeclareOption{nocrosshair}{\@crosshairfalse}
%    \end{macrocode}
%
%    Die Optionen setzen den Schalter |\@itemization|, der
%    abgefragt wird, um die Staffelung der \texttt{itemize}
%    Umgebung zu kontrollieren. In \texttt{thesis.cls}
%    ist dieser Schalter standardm\"a\ss{}ig 
%    wahr.
%
%    \begin{macrocode}
\newif\if@itemization     \@itemizationtrue
\DeclareOption{itemize}  {\@itemationtrue}
\DeclareOption{noitemize}{\@itemizationfalse}
%    \end{macrocode}
%
%    Die Optionen setzen den Schalter |\@enumeration|, der
%    abgefragt wird, um die Numerierung der \texttt{enumerate} 
%    Umgebung zu kontrollieren. In \texttt{thesis.cls} standardm\"a\ss{}ig 
%    alphanumerische Z\"ahlung.
%
%    \begin{macrocode}
\newif\if@enumeration       \@enumerationtrue           
\DeclareOption{enumerate}  {\@enumerationtrue}
\DeclareOption{noenumerate}{\@enumerationfalse}
%    \end{macrocode}
%
%    Die Optionen setzen den Schalter |\@noind|, der im folgenden
%    abgefragt wird, um Absatzabst\"ande und Fu\ss{}notenstil zu modifizieren.
%    Voreingestellt ist der Satz von Abschnitten und Fu\ss{}noten, bei denen
%    die erste Zeile einger\"uckt gesetzt wird.
%
%    \begin{macrocode}
\newif\if@noind          \@noindfalse           
\DeclareOption{indent}  {\@noindfalse}
\DeclareOption{noindent}{\@noindtrue}
%    \end{macrocode}
%
%    Die Optionen setzen den Schalter |\@center|, der im folgenden
%    abgefragt wird, um \"Uberschriften, Kopfzeilen und bestimmte Eintr\"age 
%    ins Inhaltsverzeichnis zu zentrieren. Standardm\"a\ss{}ig
%    werden diese in \texttt{thesis.cls} rechtsb\"undig gesetzt.
%
%    \begin{macrocode}
\newif\if@center         \@centerfalse
\DeclareOption{center}  {\@centertrue}
\DeclareOption{nocenter}{\@centerfalse}
%    \end{macrocode}
%
%    Die Optionen setzen den Schalter |\@upper|, der im folgenden
%    abgefragt wird, um Teil\"uberschriften in Gro\ss{}buchstaben zu setzen.
%    Standardm\"a\ss{}ig benutzt \texttt{thesis.cls} keine Gro\ss{}buchstaben.
%
%    \begin{macrocode}
\newif\if@upper        \@upperfalse
\DeclareOption{upper}  {\@uppertrue}
\DeclareOption{noupper}{\@upperfalse}
%    \end{macrocode}
%
%    Die Optionen setzen den Schalter |\@headline|, der 
%    abgefragt wird, um Kopfzeilen zu unterstreichen. Voreinstellung 
%    in \texttt{thesis.cls} sind unterstrichene Kopfzeilen.
%
%    \begin{macrocode}
\newif\if@headline        \@headlinetrue
\DeclareOption{headline}  {\@headlinetrue}
\DeclareOption{noheadline}{\@headlinefalse}
%    \end{macrocode}
%
%    Die Optionen setzen den Schalter |\@headcount|, der 
%    abgefragt wird, um ggf. Abschnittsz\"ahler in Kopfzeilen
%    auszugeben. Standardm\"a\ss{}ig wird in \texttt{thesis.cls} der
%    Z\"ahler in der Kopfzeile ausgegeben.
%
%    \begin{macrocode}
\newif\if@headcount        \@headcounttrue
\DeclareOption{headcount}  {\@headcounttrue}
\DeclareOption{noheadcount}{\@headcountfalse}
%    \end{macrocode}
%
%
%    Die Benutzerschnittstellen zur Definition der Schriftarten in 
%    \"Uberschriften, Titelei, Kopfzeilen, Abbildungen etc. Die 
%    Voreinstellungen erfolgen in Optionen.
%
%    \begin{macrocode}
\def\partfont#1{\def\p@font{#1}}             \def\p@font{}          
\def\chapterfont#1{\def\c@font{#1}}          \def\c@font{}          
\def\chapterauthorfont#1{\def\ca@font{#1}}   \def\ca@font{}
\def\sectionfont#1{\def\s@font{#1}}          \def\s@font{}         
\def\subsectionfont#1{\def\ss@font{#1}}      \def\ss@font{}        
\def\subsubsectionfont#1{\def\sss@font{#1}}  \def\sss@font{}       
\def\paragraphfont#1{\def\pg@font{#1}}       \def\pg@font{}        
\def\subparagraphfont#1{\def\spg@font{#1}}   \def\spg@font{}       
\def\titlefont#1{\def\t@font{#1}}            \def\t@font{}          
\def\subtitlefont#1{\def\st@font{#1}}        \def\st@font{}          
\def\authorfont#1{\def\a@font{#1}}           \def\a@font{}          
\def\translatorfont#1{\def\a@font{#1}}       \def\tr@font{}          
\def\institutionfont#1{\def\in@font{#1}}     \def\in@font{}          
\def\theoremheaderfont#1{\def\thh@font{#1}}  \def\thh@font{}
\def\theorembodyfont#1{\def\thb@font{#1}}    \def\thb@font{}       
\def\itemfont#1{\def\item@font{#1}}          \def\item@font{}       
\def\examplefont#1{\def\ex@font{#1}}         \def\ex@font{}        
\def\headingstextfont#1{\def\h@font{#1}}     \def\h@font{}
\def\pagenumberfont#1{\def\pn@font{#1}}      \def\pn@font{}
\def\captionheaderfont#1{\def\cph@font{#1}}  \def\cph@font{}
\def\captionbodyfont#1{\def\cpb@font{#1}}    \def\cpb@font{}        
\def\figurefont#1{\def\fig@font{#1}}         \def\fig@font{}        
\def\tablefont#1{\def\tab@font{#1}}          \def\tab@font{}        
\def\indexsize#1{\def\index@size{#1}}        \def\index@size{}        
\def\bibsize#1{\def\bib@size{#1}}            \def\bib@size{}        
%    \end{macrocode}
%
%    Die Option definiert die Schriften der \texttt{slanted} Variante.
%    Gleichzeitig werden Gro\ss{}buchstaben im Titel voreingestellt.
%    Diese Option wird an sp\"aterer Stelle voreingestellt. 
%
%    \begin{macrocode}
\DeclareOption{slanted}{
    \partfont{\Large}
    \chapterfont{\large}
    \chapterauthorfont{\large}
    \sectionfont{\large\slshape}
    \subsectionfont{\slshape}
    \subsubsectionfont{\slshape}
    \paragraphfont{\slshape}
    \subparagraphfont{\slshape}
    \titlefont{\LARGE}
    \subtitlefont{\large}
    \authorfont{}
    \institutionfont{\slshape}
    \translatorfont{}
    \theoremheaderfont{\upshape}
    \theorembodyfont{}
    \itemfont{\slshape}
    \examplefont{}
    \headingstextfont{\small\slshape}
    \pagenumberfont{\small}
    \captionheaderfont{\slshape\small}
    \captionbodyfont{\small}
    \figurefont{}
    \tablefont{}
}
%    \end{macrocode}
% 
%    Die Option f\"ur die Schriften der \texttt{bold} Variante.
%
%    \begin{macrocode}
\DeclareOption{bold}{%
    \partfont{\LARGE\bfseries}
    \chapterfont{\LARGE\bfseries}
    \chapterauthorfont{\large}
    \sectionfont{\Large\bfseries}
    \subsectionfont{\large\bfseries}
    \subsubsectionfont{\bfseries}
    \paragraphfont{\bfseries}
    \subparagraphfont{\bfseries}
    \titlefont{\LARGE\bfseries}
    \subtitlefont{\large}
    \authorfont{}
    \translatorfont{}
    \institutionfont{\slshape}
    \theoremheaderfont{\bfseries}
    \theorembodyfont{\itshape}
    \itemfont{\bfseries}
    \examplefont{}
    \headingstextfont{\small\bfseries}
    \pagenumberfont{\small}
    \captionheaderfont{\bfseries}
    \captionbodyfont{}
    \figurefont{}
    \tablefont{}
}
%    \end{macrocode}
%
% Die Option f\"ur die Schriften der \texttt{sfbold} Variante.
%
%    \begin{macrocode}
\DeclareOption{sfbold}{%
    \partfont{\LARGE\sffamily\bfseries}
    \chapterfont{\LARGE\sffamily\bfseries}
    \chapterauthorfont{\Large\sffamily\bfseries}
    \sectionfont{\large\sffamily\bfseries}
    \subsectionfont{\large\sffamily\bfseries}
    \subsubsectionfont{\sffamily\bfseries}
    \paragraphfont{\sffamily\bfseries}
    \subparagraphfont{\sffamily}
    \titlefont{\LARGE\sffamily\bfseries}
    \subtitlefont{\large\sffamily}
    \authorfont{\Large\sffamily\slshape}
    \translatorfont{}
    \institutionfont{\sffamily}
    \theoremheaderfont{\sffamily}
    \theorembodyfont{}
    \itemfont{\sffamily}
    \examplefont{}
    \headingstextfont{\small\sffamily}
    \pagenumberfont{\small\rmfamily}
    \captionheaderfont{\sffamily}
    \captionbodyfont{}
    \figurefont{}
    \tablefont{}
}
%    \end{macrocode}
%
%   Option zur Erzeugung einer Titelseite.
%   
%    \begin{macrocode}
\DeclareOption{titlepage}{\@titlepagetrue}
\DeclareOption{notitlepage}{\@titlepagefalse}
%    \end{macrocode}
%
%    Optionen, ob Kapitel generell auf der rechten Seite beginnen.
%
%    \begin{macrocode}
\DeclareOption{openright}{\@openrighttrue}
\DeclareOption{openany}{\@openrightfalse}
%    \end{macrocode}
%
%     Option f\"ur zweispaltigen Satz.
%    \begin{macrocode}
\DeclareOption{onecolumn}{\@twocolumnfalse}
\DeclareOption{twocolumn}{\@twocolumntrue}
%    \end{macrocode}
%
%    Numerierung der Formeln auf der linken Seite.
%    \begin{macrocode}
\DeclareOption{leqno}{\input{leqno.clo}}
%    \end{macrocode}
%
%    Links ausgerichtete mathematische Umgebungen.
%    \begin{macrocode}
\DeclareOption{fleqn}{\input{fleqn.clo}}
%    \end{macrocode}
%
%    Offenes Bibliographie Format.
%    \begin{macrocode}
\DeclareOption{openbib}{\@openbibtrue}
%    \end{macrocode}
%
%  Die Option gestattet mehrere Literaturverzeichnisse im Text.
%
%    \begin{macrocode}
\newif\if@chapterbib \@chapterbibfalse
\DeclareOption{chapterbib}{\@chapterbibtrue}
%    \end{macrocode}
%    Ausf\"uhren der voreingestellten Optionen.
%    \begin{macrocode}
\ExecuteOptions{letterpaper,10pt,oneside,onecolumn,final,openright,%
                slanted,center,upper}
%    \end{macrocode}
%
%    Ausf\"uhren der benutzerspezifischen Optionen.
%
%    \begin{macrocode}
\ProcessOptions*
%    \end{macrocode}
%
%    Einlesen der Schriftgr\"o\ss{}en.
%    \begin{macrocode}
\input{bk1\@ptsize.clo}
%    \end{macrocode}
%
%
%    Standardwerte beim Satz von Paragraphen (Zeilenabst\"ande,
%    Zeileneinzug Trennungen, etc.). Standardvorgaben der
%    Document-Classes.
%
%    \begin{macrocode}
\setlength\lineskip{1\p@}
\setlength\normallineskip{1\p@}
\renewcommand\baselinestretch{}
\if@noind
  \setlength\parskip{0.5\baselineskip 
         \@plus.1\baselineskip \@minus.1\baselineskip}
  \setlength\parindent{\z@} 
  \def\noparskip{\par\vspace{-\parskip}}
\else
  \setlength\parskip{0\p@ \@plus 1\p@}
  \let\noparskip\relax
\fi
\@lowpenalty   51
\@medpenalty  151
\@highpenalty 301
%    \end{macrocode}
%
%    Unver\"anderte Standardvorgaben zur Behandlung von Floats.
%    \begin{macrocode}
\setcounter{topnumber}{2}
\renewcommand\topfraction{.7}
\setcounter{bottomnumber}{1}
\renewcommand\bottomfraction{.3}
\setcounter{totalnumber}{3}
\renewcommand\textfraction{.2}
\renewcommand\floatpagefraction{.5}
\setcounter{dbltopnumber}{2}
\renewcommand\dbltopfraction{.7}
\renewcommand\dblfloatpagefraction{.5}
%    \end{macrocode}
%
%
% Variable Kopfzeilen, ggf. zentriert und unterstrichen. 
%
%    \begin{macrocode}
\def\e@skip{\h@font{\phantom{y}}}
\if@twoside                             
  \def\ps@headings{
      \let\@oddfoot\@empty\let\@evenfoot\@empty
      \def\@evenhead{\vbox{\hsize=\textwidth
        \hbox to \textwidth{%
        {\pn@font\thepage}\hfill{\h@font\leftmark}\e@skip\if@center\hfill\fi}
        \if@headline \vskip 1.5pt \hrule \fi}}%
      \def\@oddhead{\vbox{\hsize=\textwidth
         \hbox to \textwidth{%
         \if@center\hfill\fi{\h@font\rightmark}\e@skip\hfill{\pn@font\thepage}}
         \if@headline \vskip 1.5pt \hrule \fi}}%
      \let\@mkboth\markboth
    \def\chaptermark##1{%
      \markboth {%
        \ifnum \c@secnumdepth >\m@ne
          \if@mainmatter
            \@chapapp\ 
              \if@headcount 
                \thechapter. \ %
              \fi 
          \fi
        \fi
        ##1}{}}%
    \def\sectionmark##1{%
      \markright {%
        \ifnum \c@secnumdepth >\z@
          \if@headcount 
            \thesection. \ %
          \fi
        \fi
        ##1}}}
\else
  \def\ps@headings{%
    \let\@oddfoot\@empty
    \def\@oddhead{\vbox{\hsize=\textwidth
      \hbox to \textwidth{%
      \if@center\hfill\fi{\h@font\rightmark}\e@skip\hfill{\pn@font\thepage}}
      \if@headline \vskip 1.5pt \hrule \fi}}%
    \let\@mkboth\markboth
    \def\chaptermark##1{%
      \markright {%
        \ifnum \c@secnumdepth >\m@ne
          \if@mainmatter
            \@chapapp\ 
            \if@headcount
              \thechapter. \ %
            \fi
          \fi
        \fi
        ##1}}}
\fi
%    \end{macrocode}
%
% Variable Kopfzeilen, ggf. zentriert und unterstrichen.
%
%    \begin{macrocode}
\def\ps@myheadings{%
    \let\@oddfoot\@empty\let\@evenfoot\@empty
    \def\@oddhead{\vbox{\hsize=\textwidth
      \hbox to \textwidth{%
      \if@center\hfill\fi{\h@font\rightmark}\e@skip\hfill{\pn@font\thepage}}
      \if@headline \vskip 1.5pt \hrule \fi}}%
    \def\@evenhead{\vbox{\hsize=\textwidth
      \hbox to \textwidth{%
      {\pn@font\thepage}\hfill{\h@font\leftmark}\e@skip\if@center\hfill\fi}
      \if@headline \vskip 1.5pt \hrule \fi}}%
    \let\@mkboth\@gobbletwo
    \let\chaptermark\@gobble
    \let\sectionmark\@gobble
    }
%    \end{macrocode}
%
%    Kopfzeile f\"ur vorl\"aufige Formatierungen im Zusammenhang mit
%    der Option \texttt{draft}. Die Kopfzeile enth\"alt Datum, Uhrzeit
%    und Seitenzahl. 
%
%    \begin{macrocode}
\def\ps@draft{%
    \let\@oddfoot\@empty\let\@evenfoot\@empty
    \def\@oddhead{\vbox{\hsize=\textwidth
      \hbox to \textwidth{%
      {\pn@font\today\ \now\ --- {\h@font \draftname: ``\jobname''}
       \hfil\e@skip \thepage}}
       \if@headline \vskip 1.5pt \hrule \fi}}%
    \def\@evenhead{\vbox{\hsize=\textwidth
       \hbox to \textwidth{%
       \pn@font\thepage\e@skip\hfil {\h@font \draftname: ``\jobname''} ---
       \today\ \now\ }
       \if@headline \vskip 1.5pt \hrule \fi}}%
    \let\@mkboth\@gobbletwo
    \let\chaptermark\@gobble
    \let\sectionmark\@gobble
    }
%    \end{macrocode}
%
%    Leere Seiten erhalten eine Markierung am oberen Seitenrand. Dem
%    Springer Makro `svma' entliehen. 
%
%    \begin{macrocode}
\@ifundefined{option@crosshair}{}{%
  \def\clap#1{\hbox to 0pt{\hss#1\hss}} \newdimen\@crosshairrule
  \@crosshairrule=.24pt \def\@crosshairs{\vbox to
    0pt{\hsize=0pt\baselineskip=0pt\lineskip=0pt \vss \clap{\vrule
        height .125in width \@crosshairrule depth 0pt} \clap{\vrule
        width .25in height \@crosshairrule depth 0pt} \clap{\vrule
        height .125in width \@crosshairrule depth 0pt} \vss}}
 \def\ps@empty{%
    \let\@oddfoot\@empty\let\@evenfoot\@empty
    \def\@oddhead{\hfill\raise\headheight\@crosshairs}
    \let\@evenhead\@oddhead}}
%    \end{macrocode}
%
%    Es werden sechs zus\"atzliche Titelangaben definiert. Die Titelei
%    durch das Makro |\maketitle| ist entscheidend davon abh\"angig,
%    ob einseitiger oder zweiseitiger Druck gew\"unscht ist. Bei zweiseitigem 
%    Druck wird eine verk\"urzte Titelseite (Umschlag),
%    ggf. ein Schmutztitel, Standardtitelseite, ggf. Copyrightseite
%    und ggf. Widmung ausgegeben. Bei einseitigem Druck nur der
%    Standardtitel und die Widmung. 
%
%    \begin{macrocode}
\def\@subtitle{}         \def\subtitle#1{\gdef\@subtitle{#1}}
\def\@translator{}       \def\translator#1{\gdef\@translator{#1}}
\def\@institution{}      \def\institution#1{\gdef\@institution{#1}}
\def\@dedication{}       \def\dedication#1{\def\@dedication{#1}}
\def\@uppertitleback{}   \long\def\uppertitleback#1{\def\@uppertitleback{#1}}
\def\@middletitleback{}  \long\def\middletitleback#1{\def\@middletitleback{#1}}
\def\@lowertitleback{}   \long\def\lowertitleback#1{\def\@lowertitleback{#1}}
\def\new@tpage{\newpage\thispagestyle{empty}\null}
\def\and{\end{tabular}\hskip 1em plus.17fil
   \if@center
      \begin{tabular}[t]{c}
   \else
     \begin{tabular}[t]{@{}l@{}}
   \fi}
\def\maketitle{%
  \begin{titlepage}
    \let\footnotesize\small
    \let\footnoterule\relax
    \def\thefootnote{\fnsymbol{footnote}}
     \if@twoside  
       \new@tpage 
       \begin{raggedright}
       {\t@font \@title \par}\vskip 1em
       {\st@font \@subtitle \par}\vspace{50pt}
       {\a@font \lineskip 1.25em
       \begin{tabular}[t]{@{}l@{}}
       \@author
       \end{tabular}\par}
       \vfill
       {\in@font\@institution\par}
       \end{raggedright}
       \new@tpage \new@tpage
       {\ss@font\@title}\par
       {\sss@font\@subtitle}
       \new@tpage
    \fi
    \new@tpage
     \begin{center} 
     {\t@font\@title \par}\vskip 1em
     {\st@font\@subtitle \par}
     \vspace{50pt}
     {\a@font \lineskip 1.25em
     \begin{tabular}[t]{c}
     \@author
     \end{tabular} \par }\vfil
     {\tr@font \@translator \par}
     \vfil
     {\st@font \@date \par}
     \vskip 2em
     {\in@font\@institution \par}
     \end{center} 
     \par \@thanks
   \if@twoside \new@tpage 
      \noindent\begin{minipage}[t]{\textwidth}
      \@uppertitleback
      \end{minipage}\par
      \vfill
      \noindent\begin{minipage}[t]{\textwidth}
      \@middletitleback
      \end{minipage}\par
      \vfill
      \noindent\begin{minipage}[b]{\textwidth}
      \@lowertitleback
      \end{minipage}
   \fi
   \ifx\@dedication\@empty\else \new@tpage
        {\centering \Large \@dedication \par}
   \fi
   \if@twoside \new@tpage \fi
   \end{titlepage}
   \def\thefootnote{\arabic{footnote}}
   \setcounter{footnote}{0}
   \if@thema\else\let\thanks\relax\fi
   \gdef\@thanks{}\gdef\@author{}\gdef\@translator{}\gdef\@institution{}
   \gdef\@uppertitleback{}\gdef\@lowertitleback{}\gdef\@dedication{}%
   \gdef\@title{}\gdef\@subtitle{}\let\maketitle\relax}
%    \end{macrocode}
%
%
% Im Fall des \texttt{thesis.cls} wird die Zusammenfassung als Teil
% der Titelseite wie ein Kapitel gesetzt, die nicht zur Titelseite
% z\"ahlt. Handelt es sich um ein
% \texttt{thema.cls}-Dokument, wird zus\"atzlich die Umgebung
% |\chapterabstract| 
% definiert, die eine Zusammenfassung nach jeder Kapitel\"uberschrift erlaubt.
% Diese wird in Abh\"angigkeit von der Stiloption \texttt{center} zentriert
% oder als |subsection*| gesetzt.
%
%    \begin{macrocode}
\def\abstract{%\titlepage
  \chapter*{\abstractname\@mkboth{\abstractname}{\abstractname}}}
\def\endabstract{\par\vfil\null\endtitlepage}
\if@thema
   \if@center
     \def\chapterabstract{\if@twocolumn
        \subsection*{\abstractname}%
     \else \small
       \begin{center}%
        {\pg@font \abstractname\vspace{-.5em}\vspace{\z@}}%
        \end{center}%
        \quotation
     \fi}
     \def\endchapterabstract{\if@twocolumn\else\endquotation\fi}
   \else
     \def\chapterabstract{\if@twocolumn
        \subsection*{\abstractname}%
     \else \small
        \subsection*{\abstractname}%
     \fi}
     \def\endchapterabstract{\par\bigskip}
  \fi
\fi
%    \end{macrocode}
%
%    Definition und Initialisierung der Gliederungsz\"ahler.
%
%    \begin{macrocode}
\newcommand*{\chaptermark}[1]{}
\setcounter{secnumdepth}{2}
\newcounter {part}
\newcounter {chapter}
\newcounter {section}[chapter]
\newcounter {subsection}[section]
\newcounter {subsubsection}[subsection]
\newcounter {paragraph}[subsubsection]
\newcounter {subparagraph}[paragraph]
\renewcommand{\thepart}         {\Roman{part}}
\renewcommand{\thechapter}      {\arabic{chapter}}
\renewcommand{\thesection}      {\thechapter.\arabic{section}}
\renewcommand{\thesubsection}   {\thesection.\arabic{subsection}}
\renewcommand{\thesubsubsection}{\thesubsection .\arabic{subsubsection}}
\renewcommand{\theparagraph}    {\thesubsubsection.\arabic{paragraph}}
\renewcommand{\thesubparagraph} {\theparagraph.\arabic{subparagraph}}
%    \end{macrocode}
%
%    Die Bezeichnung des Kapitels bleibt standardm\"a\ss{}ig
%    leer. Soll eine Kapitelbezeichnung (z.Z. nur in Kopfzeilen)
%    ausgegeben werden, ist dies mittels der Zuweisung
%    |\chapapp{\chaptername}| im Dokument m\"oglich. 
%
%    \begin{macrocode}
\def\chapapp#1{\def\@chapapp{#1}}      \def\@chapapp{}
%    \end{macrocode}
%
%    Die folgenden Kommandos trennen die logischen Teile eines
%    Buches. |\frontmatter| beginnt eine neue Seite mit r\"omischer
%    Numerierung und schaltet die Ausgabe des Kapitelz\"ahlers
%    aus. |\mainmatter| er\"offnet eine neue Seite, schaltet die Ausgabe
%    des Kapitelz\"ahlers ein und wechselt auf arabische
%    Zahlen. |\backmatter| schaltet die Kapitelz\"ahlung wieder ab und
%    l\"a\ss{}t die Seitenz\"ahlung unver\"andert.
%
%    \begin{macrocode}
\newcommand{\frontmatter}{\cleardoublepage
            \@mainmatterfalse\pagenumbering{roman}}
\newcommand{\mainmatter}{\cleardoublepage
       \@mainmattertrue\pagenumbering{arabic}}
\newcommand{\backmatter}{\if@openright\cleardoublepage\else\clearpage\fi
      \@mainmatterfalse}
%    \end{macrocode}
%
%    Der Satz einer Teil\"uberschrift erfolgt ggf. in
%    Gro\ss{}buchstaben und der voreingestellten Schriftart.
%
%    \begin{macrocode}
\newcommand{\part}{\cleardoublepage
    \thispagestyle{empty}%
    \if@twocolumn 
        \onecolumn\@tempswatrue 
     \else 
       \@tempswafalse 
     \fi 
     \vspace*{\beforechaptervspace}%
     \secdef\@part\@spart}
\def\@part[#1]#2{%
    \ifnum \c@secnumdepth >-2\relax \refstepcounter{part}
      \addcontentsline{toc}{part}{\partname\ \thepart \hspace{1em}#1}%
    \else
     \addcontentsline{toc}{part}{#1}\fi \markboth{}{}%
     {\if@center
       \centering
      \else
       \raggedright
      \fi 
     \reset@font
      \ifnum \c@secnumdepth >-2\relax 
        \p@font \partname{} \thepart \par
         \vskip 20pt 
      \fi 
      \p@font
      \if@upper
        \uppercase{#2}
      \else #2
      \fi
  \par}
\@endpart}
\def\@spart#1{%
      {\if@center
          \centering
       \else
         \raggedright
       \fi
       \reset@font\p@font
       \if@upper
          \uppercase\expandafter{#1}
       \else #1%
       \fi\par}
  \@endpart}
\def\@endpart{\vfil\newpage
              \if@twoside
                \hbox{}%
                \thispagestyle{empty}%
                \newpage
              \fi
              \if@tempswa
                \twocolumn
              \fi}
%    \end{macrocode}
%
%     Definition des |\chapterauthor|, der vom Makro
%     |\@makechapterauthor| gesetzt wird. |\@shortauthor| 
%     \"ubernimmt die Namen f\"ur die Kopfzeile und den Eintrag im 
%     Inhaltsverzeichnis.
%
%    \begin{macrocode}
\if@thema
  \def\chapterauthor#1{\gdef\@chapterauthor{#1}}  \def\@chapterauthor{}
  \def\shortauthor#1{\gdef\@shortauthor{#1}}      \def\@shortauthor{}
  \def\@makechapterauthor{\par    
     \def\thefootnote{\fnsymbol{footnote}}%
     \def\@makefnmark{\hbox
         to\z@{$\m@th^{\@thefnmark}$\hss}}%
     \if@center
         \centering
     \else\parindent\z@
         \raggedright
     \fi
     {\ca@font  \lineskip .5em
     \if@center 
         \begin{tabular}[t]{c} 
     \else 
         \begin{tabular}[t]{@{}l@{}} 
     \fi
     \@chapterauthor
     \end{tabular} \par}\@thanks
     \vskip30\p@
     \gdef\@chapterauthor{}\gdef\@shortauthor{}
     \gdef\@thanks{}\setcounter{footnote}{0}}
\fi
%    \end{macrocode}
%
%    Die Definition von |\@chapter| wird so ge\"andert, da\ss{} im
%    Fall eines Sammelwerkes ggf. die Autoren des Kapitels ins
%    Inhaltsverzeichnis aufgenommen werden. 
%
%    \begin{macrocode}
\newcommand{\chapter}{\if@openright\cleardoublepage\else\clearpage\fi
                    \thispagestyle{empty}%
                    \global\@topnum\z@
                    \@afterindentfalse
                    \secdef\@chapter\@schapter}
\def\@chapter[#1]#2{%
    \ifnum \c@secnumdepth >\m@ne
       \if@mainmatter
          \refstepcounter{chapter}%
          \typeout{\chaptername\space\thechapter.} 
          \if@thema
             \ifx\@shortauthor\@empty
                \addcontentsline{toc}{chapter}{%
                \protect\numberline{\thechapter.}#1}%
             \else
                \addcontentsline{toc}{chapter}{%
                \protect\numberline{\thechapter.}%
                \@shortauthor\hfill\mbox{}\vskip\normallineskip #1}%
             \fi
          \else
             \addcontentsline{toc}{chapter}{%
             \protect\numberline{\thechapter.}#1}%
          \fi
      \else
        \addcontentsline{toc}{chapter}{#1}
      \fi
   \else
      \addcontentsline{toc}{chapter}{#1}
   \fi
   \chaptermark{#1}
   \addtocontents{lof}{\protect\addvspace{10pt}}
   \addtocontents{lot}{\protect\addvspace{10pt}}
   \if@twocolumn
      \@topnewpage[\@makechapterhead{#2}]
    \else
      \@makechapterhead{#2}
      \@afterheading 
    \fi}
%    \end{macrocode}
%
%
%    Der Anfangs- und der Endeabstand eines Kapitels sind frei w\"ahlbar.
%
%    \begin{macrocode}
\newlength{\beforechaptervspace}
\setlength{\beforechaptervspace}{50pt}
\newlength{\afterchaptervspace}
\setlength{\afterchaptervspace}{30pt}
%    \end{macrocode}
%
%    Ist |\chapapp| nicht definiert, erfolgt nur der Satz von
%    Kapitelz\"ahler und \"Uberschrift. Das Makro wird derart
%    variiert, da\ss{} entweder zentrierte oder rechtsb\"undige
%    \"Uberschriften erzeugt werden. Im Fall rechtsb\"undiger
%    \"Uberschriften wird die Numerierung ausgespart. Anderfalls
%    erfolgt die Ausgabe des Kapitelnames und des Z\"ahlers in einer
%    eigenen Zeile und anschlie\ss{}end die \"Uberschrift. Wiederum
%    ggf. zentriert. Abschlie\ss{}end werden ggf. die Namen der
%    Kapitelautoren in die Kopfzeile aufgenommen und das Makro
%    |\@makechapterauthors| wird aufgerufen.
%
%    \begin{macrocode}
\def\@makechapterhead#1{%
  \vspace*{\beforechaptervspace}{%
  \ifx\@chapapp\@empty
     \if@center\centering
       \ifnum \c@secnumdepth >\m@ne 
         {\c@font\thechapter.\ }
       \fi
       {\c@font 
       \if@upper
          \uppercase{#1} 
       \else 
          #1 
       \fi
       \par \nobreak}
     \else\raggedright
         \ifnum \c@secnumdepth >\m@ne
           \setbox\@tempboxa\hbox{\c@font\thechapter.\ %
           \c@font
           \if@upper
             \uppercase{#1}
           \else 
              #1% 
           \fi}
         \ifdim \wd\@tempboxa >\hsize
           \@hangfrom{%
             \c@font\thechapter.\ }{\c@font
           \if@upper
              \uppercase{#1}
           \else 
              #1%
           \fi
           \par\nobreak}
         \else 
           \hbox to\hsize{\box\@tempboxa\hfil} 
         \fi
       \fi
     \fi
  \else
    \if@center 
       \centering 
     \else 
       \raggedright 
     \fi
     \ifnum \c@secnumdepth >\m@ne 
       \if@mainmatter
          {\s@font \@chapapp{} \thechapter}
          \par \vskip 15\p@ 
       \fi 
     \fi 
     {\c@font  
     \if@upper
        \uppercase{#1} 
     \else 
       #1% 
     \fi
     \par \nobreak}
  \fi
  \vspace{\afterchaptervspace}
  \if@thema
    \ifx\@shortauthor\@empty
    \else
      \@mkboth{\@shortauthor}{#1}
    \fi
    \ifx\@chapterauthor\@empty
    \else
       \@makechapterauthor
    \fi
  \fi}
}
%    \end{macrocode}
%
%     Definitionen der Sternform.
%
%    \begin{macrocode}
\def\@schapter#1{%
    \if@twocolumn
       \@topnewpage[\@makeschapterhead{#1}]
    \else 
       \@makeschapterhead{#1}\@afterheading
    \fi}
\def\@makeschapterhead#1{%
   \vspace*{\beforechaptervspace}{%
     \if@center
        \centering
     \else
        \parindent\z@\raggedright 
     \fi
     {\c@font
     \if@upper
       \uppercase\expandafter{#1} 
     \else 
        #1% 
     \fi
     \par\nobreak
     \vskip 30\p@}
     \if@thema
       \ifx\@shortauthor\@empty
       \else
          \@mkboth{\@shortauthor}{#1}
       \fi
       \ifx\@chapterauthor\@empty
       \else
          \@makechapterauthor
       \fi\fi
     }}
%    \end{macrocode}
%
%    Die folgenden Kommandos definieren \"Uberschriften auf tieferer
%    Gliederungsebene. Gegen\"uber den Standardklassen wurde jeweils die
%    M\"oglichkeit zentrierter \"Uberschriften eingef\"uhrt, und es wurden
%    die Schriftarten ver\"andert.
%
%    \begin{macrocode}
\newcommand\section{\@startsection {section}{1}{\z@}%
                                   {-3.5ex \@plus -1ex \@minus -.2ex}%
                                   {2.3ex \@plus.2ex}%
                                   {\if@center\centering\else\raggedright\fi 
                                    \reset@font\s@font}}
\newcommand\subsection{\@startsection{subsection}{2}{\z@}%
                                     {-3.25ex\@plus -1ex \@minus -.2ex}%
                                     {1.5ex \@plus .2ex}%
                                     {\if@center\centering\else\raggedright\fi 
                                      \reset@font\ss@font}}
\newcommand\subsubsection{\@startsection{subsubsection}{3}{\z@}%
                                     {-3.25ex\@plus -1ex \@minus -.2ex}%
                                     {1.5ex \@plus .2ex}%
                                     {\if@center\centering\else\raggedright\fi 
                                      \reset@font\sss@font}}
\newcommand\paragraph{\@startsection{paragraph}{4}{\z@}%
                                    {3.25ex \@plus1ex \@minus.2ex}%
                                    {-1em}%
                                    {\reset@font\pg@font}}
\newcommand\subparagraph{\@startsection{subparagraph}{5}{\parindent}%
                                       {3.25ex \@plus1ex \@minus .2ex}%
                                       {-1em}%
                                      {\reset@font\spg@font}}
%    \end{macrocode}
%
%    Es folgt die Definition verschiedener Listenumgebungen. Zun\"achst
%    werden eine Reihe globaler Definitionen und Einstellungen
%    \"ubernommen. 
% 
%    \begin{macrocode}
\if@twocolumn
  \setlength\leftmargini  {2em}
\else
  \setlength\leftmargini  {2.5em}
\fi
\setlength\leftmarginii  {2.2em}
\setlength\leftmarginiii {1.87em}
\setlength\leftmarginiv  {1.7em}
\if@twocolumn
  \setlength\leftmarginv  {.5em}
  \setlength\leftmarginvi {.5em}
\else
  \setlength\leftmarginv  {1em}
  \setlength\leftmarginvi {1em}
\fi
\setlength\leftmargin    {\leftmargini}
\setlength  \labelsep  {.5em}
\setlength  \labelwidth{\leftmargini}
\addtolength\labelwidth{-\labelsep}
\@beginparpenalty -\@lowpenalty
\@endparpenalty   -\@lowpenalty
\@itempenalty     -\@lowpenalty
%    \end{macrocode}
%
%    Definitionen der Listenumgebungen. Zun\"achst die Voreinstellungen
%    f\"ur die Umgebung |enumerate| f\"ur Standard bzw. dekadische
%    Numerierung. 
%
%    \begin{macrocode}
\if@enumeration
  \renewcommand\theenumi{\arabic{enumi}}
  \renewcommand\theenumii{\alph{enumii}}
  \renewcommand\theenumiii{\roman{enumiii}}
  \renewcommand\theenumiv{\Alph{enumiv}}
  \newcommand\labelenumi{\theenumi.}
  \newcommand\labelenumii{(\theenumii)}
  \newcommand\labelenumiii{\theenumiii.}
  \newcommand\labelenumiv{\theenumiv.}
  \renewcommand\p@enumii{\theenumi}
  \renewcommand\p@enumiii{\theenumi(\theenumii)}
  \renewcommand\p@enumiv{\p@enumiii\theenumiii}
\else
  \renewcommand\theenumi{\arabic{enumi}}
  \renewcommand\theenumii{\arabic{enumii}}
  \renewcommand\theenumiii{\arabic{enumiii}}
  \renewcommand\theenumiv{\arabic{enumiv}}
  \newcommand\labelenumi{\theenumi.}
  \newcommand\labelenumii{\theenumi.\theenumii.}
  \newcommand\labelenumiii{\theenumi.\theenumii.\theenumiii.}
  \newcommand\labelenumiv{\theenumi.\theenumii.\theenumiii.\theenumiv.}
  \renewcommand\p@enumii{\theenumi}
  \renewcommand\p@enumiii{\theenumi(\theenumii)}
  \renewcommand\p@enumiv{\p@enumiii\theenumiii}
\fi
%    \end{macrocode}
%
%    Definition \texttt{itemize} der Item-Markierungen. Entweder
%    \"ubliche Staffelung der Markierungen oder keine Hervorhebung der
%    "`items"'. 
%
%    \begin{macrocode}
\if@itemization
  \newcommand\labelitemi{$\m@th\bullet$}
  \newcommand\labelitemii{\normalfont\bfseries --}
  \newcommand\labelitemiii{$\m@th\ast$}
  \newcommand\labelitemiv{$\m@th\cdot$}
\else
  \newcommand\labelitemi{\bfseries --}
  \newcommand\labelitemii{\bfseries --}
  \newcommand\labelitemiii{\bfseries --}
  \newcommand\labelitemiv{\bfseries --}
\fi
%    \end{macrocode}
%
%    Die Labels der \texttt{description} Umgebung werden in der
%    Schriftart |\item@font| gesetzt. 
%
%    \begin{macrocode}
\newenvironment{description}
               {\list{}{\labelwidth\z@ \itemindent-\leftmargin
                        \let\makelabel\descriptionlabel}}
               {\endlist}
\newcommand\descriptionlabel[1]{\hspace\labelsep
                                \item@font #1}
%    \end{macrocode}
%
%    Unver\"anderte Definition der \texttt{verse} Umgebung.
%
%    \begin{macrocode}
\newenvironment{verse}
               {\let\\=\@centercr
                \list{}{\itemsep      \z@
                        \itemindent   -1.5em%
                        \listparindent\itemindent
                        \rightmargin  \leftmargin
                        \advance\leftmargin 1.5em}%
                \item[]}
               {\endlist}
%    \end{macrocode}
%
%     Unver\"anderte Definition der \texttt{quotation} Umgebung.
%
%    \begin{macrocode}
\newenvironment{quotation}
               {\list{}{\listparindent 1.5em%
                        \itemindent    \listparindent
                        \rightmargin   \leftmargin
                        \parsep        \z@ \@plus\p@}%
                \item[]}
               {\endlist}
%    \end{macrocode}
%
%     Unver\"anderte Definitionen der \texttt{quote} Umgebung.
%
%    \begin{macrocode}
\newenvironment{quote}
               {\list{}{\rightmargin\leftmargin}%
                \item[]}
               {\endlist}
%    \end{macrocode}
%
%    Die Bezeichnung einer \texttt{theorem} Umgebung wird in der
%    Schriftart |\thh@font| gesetzt, der Text selbst in
%    |\thb@font|. 
%
%    \begin{macrocode}
\def\@begintheorem#1#2{\reset@font\thb@font\trivlist
      \item[\hskip \labelsep{\reset@font\thh@font #1\ #2:}]}
\def\@opargbegintheorem#1#2#3{\reset@font\thb@font\trivlist
      \item[\hskip \labelsep{\reset@font\thh@font #1\ #2\ (#3):}]}
\def\@endtheorem{\endtrivlist}
%    \end{macrocode}
%
%    Die Umgebung zur Beschreibung von Beispielen.
%
%    \begin{macrocode}
\newlength{\exampleindent}    \setlength{\exampleindent}{\parindent}
\newenvironment{example}%
   {\begin{list}{}{%
    \setlength{\leftmargin}{\exampleindent}}
    \ex@font \item[]}
   {\end{list}}
%    \end{macrocode}
%
%    Die \texttt{describe}-Umgebung. Das \"ubergebene Argument
%    dient zur Berechnung des breitesten Labels. 
%
%    \begin{macrocode}
\newenvironment{describe}[1][\quad]%
  {\begin{list}{}{%
    \renewcommand{\makelabel}[1]{{\item@font ##1}\hfil}%
    \settowidth{\labelwidth}{{\item@font #1}}%
    \setlength{\leftmargin}{\labelwidth}%
    \addtolength{\leftmargin}{\labelsep}}}%
  {\end{list}}
%    \end{macrocode}
%
%    Umgebung zur Erzeugung der Titelseite.
%
%    \begin{macrocode}
\newenvironment{titlepage}
    {%
      \cleardoublepage
      \if@twocolumn
        \@restonecoltrue\onecolumn
      \else
        \@restonecolfalse\newpage
      \fi
      \thispagestyle{empty}%
      \if@compatibility
        \setcounter{page}{0}
      \fi}%
    {\if@restonecol\twocolumn \else \newpage \fi
    }
%    \end{macrocode}
%
%    Ist |\@chapapp| nicht initialisiert (Voreinstellung), wird
%    der Anhang durch einen Teil abgesetzt.  
% 
%    Das Kommando \verb|\addtocontentsline| wird durch die Definition 
%    \verb|\@addappendixtocontents| ersetzt, soda\ss{} ein Anhang auch als
%    \verb|\include|-Datei eingelesen werden kann (Dank an Uwe Schellhorn).
%        
%    \begin{macrocode}
\newcommand\appendix{\par
  \setcounter{chapter}{0}%
  \setcounter{section}{0}%
  \ifx\@chapapp\@empty
    \def\@addappendixtocontents{\addcontentsline{toc}{part}{\appendixname}}
    \part*{\appendixname\@mkboth{\appendixname}{\appendixname}%%
           \@addappendixtocontents}
  \else
    \renewcommand{\@chapapp}{\appendixname}%
  \fi
  \renewcommand{\thechapter}{\Alph{chapter}}}
%    \end{macrocode}
%
%    Unver\"anderte Voreinstellungen der \texttt{array} Umgebung.
%
%    \begin{macrocode}
\setlength\arraycolsep{5\p@}
\setlength\tabcolsep{6\p@}
\setlength\arrayrulewidth{.4\p@}
\setlength\doublerulesep{2\p@}
\setlength\tabbingsep{\labelsep}
%    \end{macrocode}
%
%    Unver\"anderte Voreinstellungen der \texttt{minipage} Umgebung.
%
%    \begin{macrocode}
\skip\@mpfootins = \skip\footins
%    \end{macrocode}
%
%    Unver\"anderte Voreinstellungen der \texttt{fbox}.
%
%    \begin{macrocode}
\setlength\fboxsep{3\p@}
\setlength\fboxrule{.4\p@}
%    \end{macrocode}
%
%    Z\"ahler zur Erzeugung von Gleichungsnummern wird nach jedem
%    Kapitel zur\"uckgesetzt.
%
%    \begin{macrocode}
\@addtoreset{equation}{chapter}
\renewcommand{\theequation}{\thechapter.\arabic{equation}}
%    \end{macrocode}
%
%    Die Definition einiger Kommandos zur Erzeugung von Randnotizen in
%    Anlehnung an H. Partls |\refman.sty|. Im Unterschied zur
%    dortigen Definition erscheinen die Notizen nicht nur am linken
%    Seitenrand. 
%
%    \begin{macrocode}
\def\marginlabel#1{\marginpar%
   {\if@twoside 
       \ifodd\c@page 
          \raggedright 
       \else 
          \raggedleft 
       \fi
     \else 
        \raggedright 
     \fi #1}}
\def\attention{\mbox{}%
    \marginpar[\raggedleft\large\bf! $\rightarrow$]%
        {\raggedright\large\bf $\leftarrow$ !}}
\def\seealso#1{\mbox{}%
    \marginpar[\raggedleft$\rightarrow$ \small #1]%
        {\raggedright\small  #1 $\leftarrow$}\ignorespaces}
%    \end{macrocode}
%
%     Definition des Abbildungsz\"ahlers.
%
%    \begin{macrocode}
\newcounter{figure}[chapter]
\renewcommand{\thefigure}{\thechapter.\@arabic\c@figure}
%    \end{macrocode}
%
%     Voreinstellungen der \texttt{figure}  Umgebung. Im Makro
%     |\fnum@figure| wird der Kurzname benutzt. Die "`Floats"'
%     werden in der voreingestellten Schrift |\fig@font| gesetzt.
%
%    \begin{macrocode}
\def\fps@figure{tbp}
\def\ftype@figure{1}
\def\ext@figure{lof}
\def\fnum@figure{\figureshortname~\thefigure}
\newenvironment{figure}
               {\fig@font\@float{figure}}
               {\end@float}
\newenvironment{figure*}
               {\fig@font\@dblfloat{figure}}
               {\end@dblfloat}
%    \end{macrocode}
%
% Definition des Tabellenz\"ahlers.
%
%    \begin{macrocode}
\newcounter{table}[chapter]
\renewcommand{\thetable}{\thechapter.\@arabic\c@table}
%    \end{macrocode}
%
%     Voreinstellungen der \texttt{table}  Umgebung. Im Makro
%     |\fnum@table| wird der Kurzname benutzt. Die "`Floats"'
%     werden in der voreingestellten Schrift |\tab@font| gesetzt.
%
%    \begin{macrocode}
\def\fps@table{tbp}
\def\ftype@table{2}
\def\ext@table{lot}
\def\fnum@table{\tableshortname~\thetable}
\newenvironment{table}
               {\tab@font\@float{table}}
               {\end@float}
\newenvironment{table*}
               {\tab@font\@dblfloat{table}}
               {\end@dblfloat}
%    \end{macrocode}
%
%    Unver\"anderte Abst\"ande.
%
%    \begin{macrocode}
\newlength\abovecaptionskip
\newlength\belowcaptionskip
\setlength\abovecaptionskip{10\p@}
\setlength\belowcaptionskip{0\p@}
%    \end{macrocode}
%
%    Lange Figuren oder Tabellenbeschriftungen werden um Bezeichnung
%    und Nummer einger\"uckt. Beschriftungen werden in den Fonts
%    |cph@font| und |\cpb@font| gesetzt.
%
%    \begin{macrocode}
\long\def\@makecaption#1#2{%
 \vskip\abovecaptionskip
 \setbox\@tempboxa\hbox{{\cph@font #1:} {\cpb@font #2}}%
 \ifdim \wd\@tempboxa >\hsize 
    \@hangfrom{\cph@font #1: }{\cpb@font #2\par}%
 \else 
    \hbox to\hsize{\hfil\box\@tempboxa\hfil}%
 \fi
 \vskip\belowcaptionskip}
%    \end{macrocode}
%
%    Einfache St\"utze zur Tabellenkonstruktion.
%
%    \begin{macrocode}
\def\rb#1{\raisebox{1.5ex}[-1.5ex]{#1}}
%    \end{macrocode}
%
%    Einfache horizontale Linien in Tabellen. Die Definition der
%    Makros entspricht der Konstruktion von |\hline|.
%    Das von den Befehlen aufgerufene Makro
%    |\@xhline| f\"ugt einen zus\"atzlichen Vorschub ein, falls der
%    Befehl wiederholt gegeben wird.  
%
%    \begin{macrocode}
\def\tablerule{\noalign{\ifnum0=`}\fi
   \hrule \@height \arrayrulewidth \vskip\doublerulesep
   \futurelet \@tempa\@xhline}
\def\thicktablerule{\noalign{\ifnum0=`}\fi
   \hrule \@height 2\arrayrulewidth \vskip\doublerulesep
   \futurelet \@tempa\@xhline}
\def\doubletablerule{\noalign{\ifnum0=`}\fi
   \hrule \@height \arrayrulewidth \vskip2\arrayrulewidth
   \hrule \@height \arrayrulewidth \vskip\doublerulesep
   \futurelet \@tempa\@xhline}
%    \end{macrocode}
%
%    Definition st\"arkerer und doppelter |\hline| Varianten.
%
%    \begin{macrocode}
\def\thickhline{\noalign{\ifnum0=`}\fi
   \hrule \@height 2\arrayrulewidth\futurelet \@tempa\@xhline}
\def\doublehline{\noalign{\ifnum0=`}\fi
   \hrule \@height \arrayrulewidth\vskip2\arrayrulewidth 
   \hrule \@height \arrayrulewidth \futurelet \@tempa\@xhline}
%    \end{macrocode}
%
%
%    Unver\"anderte Definition der \LaTeX{} 2.09 Schriftartenkommandos
%    und der Kommandos f\"ur mathematische Zeichens\"atze.
%
%    \begin{macrocode}
\DeclareOldFontCommand{\rm}{\normalfont\rmfamily}{\mathrm}
\DeclareOldFontCommand{\sf}{\normalfont\sffamily}{\mathsf}
\DeclareOldFontCommand{\tt}{\normalfont\ttfamily}{\mathtt}
\DeclareOldFontCommand{\bf}{\normalfont\bfseries}{\mathbf}
\DeclareOldFontCommand{\it}{\normalfont\itshape}{\mathit}
\DeclareOldFontCommand{\sl}{\normalfont\slshape}{\@nomath\sl}
\DeclareOldFontCommand{\sc}{\normalfont\scshape}{\@nomath\sc}
\DeclareRobustCommand*{\cal}{\@fontswitch{\relax}{\mathcal}}
\DeclareRobustCommand*{\mit}{\@fontswitch{\relax}{\mathnormal}}
%    \end{macrocode}
%
%    Definition von Abst\"anden und Gliederungstiefe f\"ur das
%    Inhaltsverzeichnis. 
%
%    \begin{macrocode}
\newcommand{\@pnumwidth}{1.55em}
\newcommand{\@tocrmarg} {2.55em}
\newcommand{\@dotsep}{4.5}
\setcounter{tocdepth}{2}
%    \end{macrocode}
%

%    Makro zur Erzeugung des Inhaltsverzeichnisses. 
%
%    \begin{macrocode}
\newcommand{\tableofcontents}{%
    \if@twocolumn
      \@restonecoltrue\onecolumn
    \else
      \@restonecolfalse
    \fi
    \chapter*{\contentsname
        \@mkboth{\contentsname}%
                {\contentsname}}%
    \@starttoc{toc}%
    \if@restonecol\twocolumn\fi
    }
%    \end{macrocode}
%
%    Nur die Teil\"uberschrift, nicht aber die Seitenzahl wird
%    im Inhaltsverzeichnis hervorgehoben. 
%
%    \begin{macrocode}
\newcommand*{\l@part}[2]{%
  \ifnum \c@tocdepth >-2\relax
    \addpenalty{-\@highpenalty}%
    \addvspace{2.25em \@plus\p@}%
    \begingroup
      \setlength\@tempdima{3em}%
      \parindent \z@ \rightskip \@pnumwidth
      \parfillskip -\@pnumwidth
      {\leavevmode
       {\sss@font #1}\hfil \hbox to\@pnumwidth{\hss #2}}\par
       \nobreak
         \global\@nobreaktrue
         \everypar{\global\@nobreakfalse\everypar{}}
    \endgroup
  \fi}
%    \end{macrocode}
%
%    Auch der Kapiteleintrag wird als |\dottedtocline| gesetzt. Die
%    Seitenzahl wird nicht hervorgehoben. 
%
%    \begin{macrocode}
\newcommand*{\l@chapter}[2]{%
  \ifnum \c@tocdepth >\m@ne
    \addpenalty{-\@highpenalty}%
    \vskip 1.0em \@plus\p@
    \setlength\@tempdima{1.5em}%
    \begingroup
      \parindent \z@ \rightskip \@pnumwidth
      \parfillskip -\@pnumwidth
      \leavevmode
      \advance\leftskip\@tempdima
      \hskip -\leftskip
      {\sss@font #1}%
      \nobreak\leaders\hbox{%
         $\m@th \mkern \@dotsep mu.\mkern \@dotsep mu$}
      \hfill\nobreak\hbox to\@pnumwidth{\hfil\textrm{#2}}\par 
      \penalty\@highpenalty
    \endgroup
  \fi}
%    \end{macrocode}
%
%    Unver\"anderte Definition der Untereintr\"age ins Inhaltsverzeichnis.
%
%    \begin{macrocode}
\newcommand*{\l@section}      {\@dottedtocline{1}{1.5em}{2.3em}}
\newcommand*{\l@subsection}   {\@dottedtocline{2}{3.8em}{3.2em}}
\newcommand*{\l@subsubsection}{\@dottedtocline{3}{7.0em}{4.1em}}
\newcommand*{\l@paragraph}    {\@dottedtocline{4}{10em}{5em}}
\newcommand*{\l@subparagraph} {\@dottedtocline{5}{12em}{6em}}
%    \end{macrocode}
%
%    Die Definitionen f\"ur das Abbildungs- und das Tabellenverzeichnis. 
%    Kopfzeilenmarkierungen werden nicht in Gro\ss{}buchstaben
%    umgewandelt.  
%
%    \begin{macrocode}
\newcommand{\listoffigures}{%
    \if@twocolumn
      \@restonecoltrue\onecolumn
    \else
      \@restonecolfalse
    \fi
    \chapter*{\listfigurename
      \@mkboth{\listfigurename}%
              {\listfigurename}}%
    \@starttoc{lof}%
    \if@restonecol\twocolumn\fi
    }
\newcommand*{\l@figure}{\@dottedtocline{1}{1.5em}{2.3em}}
\newcommand{\listoftables}{%
    \if@twocolumn
      \@restonecoltrue\onecolumn
    \else
      \@restonecolfalse
    \fi
    \chapter*{\listtablename
      \@mkboth{\listtablename}%
              {\listtablename}}%
    \@starttoc{lot}%
    \if@restonecol\twocolumn\fi
    }
\let\l@table\l@figure
%    \end{macrocode}
%
%    Die Definition des Literaturverzeichnisses. Die
%    Kopfzeilenmarkierung wird nicht in Gro\ss{}buchstaben
%    umgewandelt. Ist die Option \texttt{chapterbib}  gew\"ahlt, wird
%    das Literaturverzeichnis als |\section| gesetzt.
%
%    \begin{macrocode}
\newdimen\bibindent
\bibindent=1.5em
\newenvironment{thebibliography}[1]
     {\if@chapterbib
        \section*{\refname\@mkboth{\refname}{\refname}}% 
      \else
        \chapter*{\bibname\@mkboth{\bibname}{\bibname}}% 
      \fi
      \list{\@biblabel{\arabic{enumiv}}}%
           {\settowidth\labelwidth{\@biblabel{#1}}%
            \leftmargin\labelwidth
            \advance\leftmargin\labelsep
            \if@openbib
              \advance\leftmargin\bibindent
              \itemindent -\bibindent
              \listparindent \itemindent
              \parsep \z@
            \fi
            \usecounter{enumiv}%
            \let\p@enumiv\@empty
            \renewcommand\theenumiv{\arabic{enumiv}}}%
      \if@openbib
        \renewcommand\newblock{\par}
      \else
        \renewcommand\newblock{\hskip .11em \@plus.33em \@minus.07em}%
      \fi
      \sloppy\clubpenalty4000\widowpenalty4000%
      \sfcode`\.=\@m}
     {\def\@noitemerr
       {\@latex@warning{Empty `thebibliography' environment}}%
      \endlist}
\newcommand\newblock{}
%    \end{macrocode}
%
%    Die folgenden Makros werden nur bei Verwendung der Option
%    \texttt{chapterbib} aktiviert. Das Kommando |\thebibliograpy|
%    liest nicht mehr |\jobname.bbl|, sondern
%    |@\bblfile|. Das Makro |\@include| wird dahingehend
%    ge\"andert, da\ss{} |\@bblfile| auf den Namen der entsprechenden
%    Include Datei initialisiert. Die Makros stammen von Joachim
%    Schrodt \texttt{`bibperinclude.sty'} .
%   
%    \begin{macrocode}
\if@chapterbib
  \def\@mainbblfile{\jobname.bbl}
  \let\@bblfile=\@mainbblfile
  \def\bibliography#1{%
    \if@filesw\immediate\write\@auxout{\string\bibdata{#1}}\fi
    \@input{\@bblfile}}
  \def\@include#1 {\clearpage
    \if@filesw \immediate\write\@mainaux{\string\@input{#1.aux}}\fi
    \@tempswatrue
    \if@partsw \@tempswafalse\edef\@tempb{#1}
       \@for\@tempa:=\@partlist\do{\ifx\@tempa\@tempb\@tempswatrue\fi}
    \fi
    \if@tempswa 
       \if@filesw \let\@auxout=\@partaux 
           \immediate\openout\@partaux #1.aux
           \immediate\write\@partaux{\relax}
       \fi
       \def\@bblfile{#1.bbl}\@input{#1.tex}
       \let\@bblfile\@mainbblfile\clearpage
       \@writeckpt{#1}
       \if@filesw 
          \immediate\closeout\@partaux 
       \fi
       \let\@auxout=\@mainaux\else\@nameuse{cp@#1}
   \fi}
\fi
%    \end{macrocode}
%
%    Definition des Index. Die Kopfzeilenmarkierung wird nicht in
%    Gro\ss{}buchstaben umgewandelt. Der Seitenstil ist standardm\"a\ss{}ig
%    \texttt{empty}, da Kopfzeilen voreingestellt sind. Der Satz
%    erfolgt in der Schriftgr\"o\ss{}e |\indexsize|.
% 
%    \begin{macrocode}
\newenvironment{theindex}
               {\if@twocolumn
                  \@restonecolfalse
                \else
                  \@restonecoltrue
                \fi
                \columnseprule \z@
                \columnsep 35\p@
                \twocolumn[\@makeschapterhead{\indexname}]%
                \@mkboth{\indexname}%
                        {\indexname}%
                \thispagestyle{plain}\parindent\z@
                \parskip\z@ \@plus .3\p@\relax
                \let\item\@idxitem\index@size}
               {\if@restonecol\onecolumn\else\clearpage\fi}
\newcommand{\@idxitem}  {\par\hangindent 40\p@}
\newcommand{\subitem}   {\par\hangindent 40\p@ \hspace*{20\p@}}
\newcommand{\subsubitem}{\par\hangindent 40\p@ \hspace*{30\p@}}
\newcommand{\indexspace}{\par \vskip 10\p@ \@plus5\p@ \@minus3\p@\relax}
%    \end{macrocode}
%
%   Unver\"anderte Definition der |\footnoterule|.
%
%    \begin{macrocode}
\renewcommand\footnoterule{%
  \kern-3\p@
  \hrule width .4\columnwidth
  \kern 2.6\p@}
%    \end{macrocode}
%
%    Der Satz von Fu\ss{}notenabs\"atzen variiert in Abh\"angigkeit von der
%    Option \texttt{par}; ggf. werden Fu\ss{}noten ohne Einzug
%    geblockt. In jedem Fall wird nach dem Fu\ss{}notenzeichen ein
%    Zwischenraum von \texttt{.25em} eingef\"ugt.
%
%    \begin{macrocode}
\@addtoreset{footnote}{chapter}
\if@noind
  \long\def\@makefntext#1{%
        \leftskip 2.0em%
        \noindent
        \hbox to 0em{\hss\@makefnmark\kern 0.25em}#1}
\else
  \long\def\@makefntext#1{%
      \parindent 1em%
      \noindent
      \hbox to 1.8em{\hss\@makefnmark\kern 0.25em}#1}
\fi           
%    \end{macrocode}
%
%    Definition der "`Captions"'. Standardm\"a\ss{}ig werden Kurznamen f\"ur
%    Abbildungen und Tabellen benutzt, die im entsprechenden
%    |captions|\emph{language} Makro des \texttt{german.sty}
%    eingef\"ugt werden sollten. Die Definitionen der deutschen und
%    franz\"osischen Namen folgen im Anschlu\ss{} an die Liste als
%    Metakommentar.  
%
%    \begin{macrocode}
\def\contentsname{Contents}
\def\listfigurename{List of Figures}
\def\listtablename{List of Tables}
\def\bibname{Bibliography}
\def\refname{References}
\def\indexname{Index}
\def\figurename{Figure}
\def\tablename{Table}
\def\chaptername{Chapter}
\def\appendixname{Appendix}
\def\partname{Part}
\def\abstractname{Abstract}
\def\draftname{preliminary draft}
\def\figureshortname{Fig.}                      % <-- thesis
\def\tableshortname{Tab.}                       % <-- thesis
\def\prefacename{Preface}                       % <-- thesis
%    \end{macrocode}
%\iffalse
% in: \def\captionsenglish{% 
% \def\figureshortname{Fig.}%
% \def\tableshortname{Tab.}%
% \def\prefacename{Preface}}%
% \def\keywordname{Keywords}
% in: \def\captionsfrench{% 
% \def\prefacename{Pr\'eface}%
% \def\figureshortname{Fig.}%
% \def\tableshortname{Tab.}}%
% \def\keywordname{Motcl\'e}
% in: \def\captionsgerman{%  
% \def\figureshortname{Abb.}
% \def\tableshortname{Tab.}
% \def\keywordname{Deskriptoren}           
% \fi
%
%    Definition des Datums und verschiedene Initialisierungen. 
%    Im Unterschied zum Standardstil \texttt{article} ist die
%    Benutzung von Kopfzeilen voreingestellt.
%
%    \begin{macrocode}
\newcommand{\today}{\ifcase\month\or
  January\or February\or March\or April\or May\or June\or
  July\or August\or September\or October\or November\or December\fi
  \space\number\day, \number\year}
\setlength\columnsep{10\p@}
\setlength\columnseprule{0\p@}
\iffinal
  \pagestyle{headings}
\else
  \pagestyle{draft}
\fi
\pagenumbering{arabic}
\if@twoside
\else
  \raggedbottom
\fi
\if@twocolumn
  \twocolumn
  \sloppy
  \flushbottom
\else
  \onecolumn
\fi
%    \end{macrocode}
%    \begin{macrocode}
%</thesis>
%    \end{macrocode}
%    \section{Treiber-Datei}
%
%    Der letzte Abschnitt enth\"alt die Treiberdatei zur Erstellung der
%    Dokumentation.
%    \begin{macrocode}
%<*driver>
\typeout{*******************************************************}
\typeout{* Documentation for LaTeX styles `thesis' & `thema'   *}
\typeout{*******************************************************}

\documentclass[11pt]{ltxdoc}
\usepackage{german}

\makeatletter
\newif\ifsolodoc
 \@ifundefined{solo}{\solodoctrue}{\solodocfalse}
\IndexPrologue{\section*{Index}%
               \markboth{Index}{Index}%
               Die kursiv gesetzten Seitenzahlen
               verweisen auf Beschreibungen der Makros,
               unterstrichene Programmzeilennummern
               auf deren Definitionen.}
\GlossaryPrologue{\section*{Neuerungen}%
                 \markboth{Neuerungen}{Neuerungen}}
\def\saved@macroname{Neuerung}
\renewenvironment{theglossary}{%
    \glossary@prologue%
    \GlossaryParms \let\item\@idxitem \ignorespaces}%
   {}
\makeatother
\setcounter{StandardModuleDepth}{1}
%   \OnlyDescription
%   \CodelineIndex
\CodelineNumbered 
\RecordChanges
\setlength{\parindent}{0pt}
\begin{document}
\DocInput{thesis.dtx} \newpage \PrintChanges % \newpage \PrintIndex
\end{document}
\endinput
%</driver>
%    \end{macrocode}
%% \CharacterTable
%%  {Upper-case    \A\B\C\D\E\F\G\H\I\J\K\L\M\N\O\P\Q\R\S\T\U\V\W\X\Y\Z
%%   Lower-case    \a\b\c\d\e\f\g\h\i\j\k\l\m\n\o\p\q\r\s\t\u\v\w\x\y\z
%%   Digits        \0\1\2\3\4\5\6\7\8\9
%%   Exclamation   \!     Double quote  \"     Hash (number) \#
%%   Dollar        \$     Percent       \%     Ampersand     \&
%%   Acute accent  \'     Left paren    \(     Right paren   \)
%%   Asterisk      \*     Plus          \+     Comma         \,
%%   Minus         \-     Point         \.     Solidus       \/
%%   Colon         \:     Semicolon     \;     Less than     \<
%%   Equals        \=     Greater than  \>     Question mark \?
%%   Commercial at \@     Left bracket  \[     Backslash     \\
%%   Right bracket \]     Circumflex    \^     Underscore    \_
%%   Grave accent  \`     Left brace    \{     Vertical bar  \|
%%   Right brace   \}     Tilde         \~}
%%
% \Finale
% \endinput
# Local Variables:
# mode: latex
# TeX-master: "thesis.drv" 
# End:



