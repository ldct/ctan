% local01.tex -- updated for thesis1e 15 Apr 1996
%                updated for LaTeX2e 11 Dec 1994
%                first released 1 Sept 1993
%
% Copyright (C) 1993, 1996 by Wenzel Matiaske, mati1831@perform.ww.tu-berlin.de
%
% As input to the local LaTeX-guide "`local.tex"'.
%
% For distribution of this document see the copyright notice in the
% original sources mentioned below.
%

\makeatletter
\newif\iflocalin
\newif\ifappendixin
\@ifundefined{localin}{\localintrue}{\localinfalse}
\@ifundefined{appendixin}{\appendixinfalse}{\appendixintrue}
\@ifundefined{docdir}{\def\docdir{\dots /emtex/doc/}}{}
\makeatother

\iflocalin
   \subsubsection{The style files \texttt{thesis} and \texttt{thema}}
\fi
\ifappendixin
  \subsection{The style files \texttt{thesis} and \texttt{thema}}
\fi

The style files \verb|thesis| and \verb|thema| base on the standard
layout \verb|report|. In contrast to this standard document style the
outlook of these styles is more European and more flexible. The
layout may easily be changed via options like (defaults are underlined):

\begin{description}
\item[\underline{\texttt{indent}} or \texttt{noindent}] (no) indent of
  paragraphs and footnotes,

\item[\texttt{itemization} or \underline{\texttt{noitemization}}] (no)
  bullets, stars etc. in the \texttt{itemize} environ\-ment,

\item[\underline{\texttt{enumeration}} or \texttt{noenumeration}]
  alphanumeric or numeric \texttt{enumerate} environ\-ment,

\item[\underline{\texttt{headline}} or \texttt{noheadline},
  \texttt{headcount} or \underline{\texttt{noheadcount}}] change the
  layout of the headings,

\item[\underline{\texttt{center}} or \texttt{nocenter},
  \underline{\texttt{upper}} or \texttt{noupper}] change the style of the
  sectioning,

\item[\underline{\texttt{slanted}} or \texttt{bold} or
  \texttt{sfbold}] change the fonts used in chapters, titles or
  headings,

\item[\texttt{crosshair}] marks empty pages with a crosshair, which is
  useful for camera reproducing of the output, and

\item[\texttt{chapterbib}] for chapter bibliographies with \BibTeX.

\end{description}


For default, the text ``Chapter'' is not printed in front of a chapter. 
If you want a more standard \LaTeX-layout define \verb|\chapapp{|{\em
Chapter}\verb|}| in the preamble.

The style defines an advanced section of commands for the title page. The
commands \verb|\subtitle{|\emph{text}\verb|}| and
\verb|\translator{|\emph{text}\verb|}| take some text for the
title page; \verb|\dedication{|\emph{text}\verb|}| produce a separate
dedication page, and notes on the backside of the title page
\verb|\uppertitleback{|\emph{text}\verb|}|, \verb|\middletitleback|,
and \verb|\lowertitleback| may be used.

The style file \texttt{thema} defines some more commands, which may be
useful in collections. The command \verb|\chapterauthors{|{\em
authors}\verb|}| takes the authors as an argument, which are printed
in the chapter title and in the table of contents. If you want to put the
author into the headline of the odd side pages the command
\verb|\shortauthor{|\emph{text}\verb|}| may be used. This command must
be specified before the corresponding \verb|\chapter|. The environment
\texttt{chapterabstract} prints the abstract of the chapter. Especially
with \texttt{thema} the style option \texttt{chapterbib} may be useful. This
option allows \BibTeX{} to produce a bibliography for every file which
is read by the command \verb|\include{|\emph{file}\verb|}| in a control
file.

The fonts used in \texttt{thesis} and \texttt{thema} may be changed
with the following commands which take a font declaration as
argument:

\begin{center}\small
\begin{tabular}{@{}ll@{}}
\verb|\partfont{|\emph{decl}\verb|}|            &\verb|\theorembodyfont{|\emph{decl}\verb|}|  \\
\verb|\chapterfont{|\emph{decl}\verb|}|         &\verb|\itemfont{|\emph{decl}\verb|}|         \\  
\verb|\chapterauthorfont{|\emph{decl}\verb|}|   &\verb|\examplefont{|\emph{decl}\verb|}|      \\  
\verb|\sectionfont{|\emph{decl}\verb|}|         &\verb|\headingstextfont{|\emph{decl}\verb|}|\\   
\verb|\subsectionfont{|\emph{decl}\verb|}|      &\verb|\pagenumberfont{|\emph{decl}\verb|}|\\     
\verb|\subsubsectionfont{|\emph{decl}\verb|}|   &\verb|\captionheaderfont{|\emph{decl}\verb|}|\\  
\verb|\paragraphfont{|\emph{decl}\verb|}|       &\verb|\captionbodyfont{|\emph{decl}\verb|}|\\    
\verb|\subparagraphfont{|\emph{decl}\verb|}|    &\verb|\figurefont{|\emph{decl}\verb|}|\\         
\verb|\titlefont{|\emph{decl}\verb|}|           &\verb|\tablefont{|\emph{decl}\verb|}|\\          
\verb|\subtitlefont{|\emph{decl}\verb|}|        &\verb|\indexsize{|\emph{decl}\verb|}|\\       
\verb|\authorfont{|\emph{decl}\verb|}|          &\verb|\bibsize{|\emph{decl}\verb|}|\\          
\verb|\translatorfont{|\emph{decl}\verb|}|      &\verb|\theoremheaderfont{|\emph{decl}\verb|}|\\
\verb|\institutionfont{|\emph{decl}\verb|}|     &\\
\end{tabular}
\end{center}

\iflocalin
For more details see the German documentation
\file{\docdir thesis}.
\fi


