% master: main
% format: latex

%%--------------------------------------------------------------------------
\part{\LaTeXe\ Document Class for Pitt Theses}
%%--------------------------------------------------------------------------

This document will illustrate some of the \LaTeX\ powers in typesetting
a thesis at University of Pittsburgh. The source file of this document
will be used as a sample \LaTeXe\ file for thesis writing.
It is to be compiled with \LaTeXe, and the following class files should be
accessible for writing thesis at University of Pittsburgh during the
compilation: \\

\begin{tabular}{ll}
\texttt{pitthesis.cls} & the main Pitt thesis document class file \\
\texttt{pitteng.clo}   & class option file for Pitt Engineering thesis \\
\texttt{pittstd.clo}   & class option file for Pitt standard thesis \\
\end{tabular} \\

The document is divided into three parts. The first part describes the usage of
the document class \texttt{pitthesis}.	The second part is the level one sample
input.	Except some sectioning headings added in by the current author for
illustration and clarity, it is the verbatim copy of the Leslie~Lamport's
\texttt{sample2e.tex} (which is a sample file from the \LaTeXe\ package).
Effort is made for the user to see most of the likely needed \LaTeXe\ commands
in writing a thesis.  The User should compare this source file with the
document result as well as the \LaTeXe\ Manual.\cite{lp:latex}
The third part is for the mathematical equations (instead of simple formula)
and figures/tables and the third part is about the special symbols.
I recommend the user compare every command used in this file with the document
result and the \LaTeXe\ Manual, but some of the commands may not be found there,
in AMS\LaTeX\ by the American Mathematical Society, which can be obtained from
CTAN (Comprehensive TeX Archive Network) sites, e.g.,
\verb|ftp://ctan.tug.org/tex-archive/macros/latex/|.
The bibliography is also sneaked in. The user should also watch out for it.

%%--------------------------------------------------------------------------
\chapter{PITT THESIS DOCUMENT CLASS OPTIONS}
%%--------------------------------------------------------------------------

Pitt thesis document class package (including this document) borrowed some
Jinbai Wang's work on \LaTeX 2.09 style files for theses at the University of
Pittsburgh\cite{jbw:tex} and is completely rewritten by the current author
for \LaTeXe 's report class, so that all options of report class can be used.
The reader should also notice that the document class \textbf{pittthesis} that
I wrote has two important class options, \texttt{pitteng} and \texttt{pittstd},
namely, \textbf{engineering thesis} (by default) and \textbf{Pitt standard
thesis}, respectively.	One has to give the option ``\texttt{pittstd}'' for
Pitt standard thesis style, otherwise ``\texttt{pitteng}'' is assumed.
\begin{verbatim}
\documentclass{pitthesis}	    % document class for Engineering thesis.
\documentclass[pittstd]{pitthesis}  % for Pitt standard thesis.
\end{verbatim}
These classes are designed so close as possible to the thesis requirements set
by various schools at the University of Pittsburgh (including the School of
Engineering)\cite{pitteng:thesis,pittstd:thesis}. However, minor modifications
may have to be done by the individual users to fit individual needs.
A \LaTeXe\ document for a Pitt thesis starts with
\begin{verbatim}
\documentclass[options]{pitthesis}
\end{verbatim}
Note that any report class options may be used and additional class options are
provided by the \texttt{pitthesis} class for theses at the University of
Pittsburgh, which are listed in Table \ref{tab:pitthesis}.
%
\begin{table}[b!]
  \centering
  \caption{\{\texttt{pitthesis}\} document class options}
  \label{tab:pitthesis}
  \begin{tabular}{ll}
    \hline
    \verb|pitteng|*   & Pitt Engineering thesis \\
    \verb|pittstd|  & Pitt (Standard) thesis
			(selects \texttt{nofloatchap} by default)\\
    \verb|phd|*       & Ph.D. thesis \\
    \verb|ms|	      & M.S. thesis \\
    \verb|proposal|   & for proposal presentation \\
    \verb|headings|   & default running heads with page numbers \\
    \verb|noheadings|* & do not put default running heads \\
    \verb|floatchap|* & floats (figures and tables) are numbered by chapters \\
    \verb|nofloatchap|& floats (figures and tables) are numbered sequentially\\
    \verb|single|	& single line spacing \\
    \verb|onehalf|*	& 1.5 line spacing \\
    \verb|double|	& double line spacing \\
    \hline
  \end{tabular} \\
  (* are the default options.)
\end{table}
%
The default options are \texttt{11pt}, \texttt{pitteng}, \texttt{phd},
\texttt{noheadings}, \texttt{onehalf}.	And, \texttt{floatchap} is selected
by default for \texttt{pitteng} and \texttt{nofloatchap} for \texttt{pittstd},
respectively.
Hence,
\begin{verbatim}
 \documentclass{pitthesis}
\end{verbatim}
is equivalent to
\begin{verbatim}
 \documentclass[11pt,pitteng,phd,noheadings,onehalf,floatchap]{pitthesis}
\end{verbatim}
and
\begin{verbatim}
 \documentclass[pittstd]{pitthesis}
\end{verbatim}
is equivalent to
\begin{verbatim}
 \documentclass[11pt,pittstd,phd,noheadings,onehalf,nofloatchap]{pitthesis}
\end{verbatim}
For Engineering thesis, floats (figures,tables) will be numbered by
chapters and appendices if the class option \texttt{nofloatchap} is not given.
If the option \texttt{nofloatchap} is given, all figures and tables will be
numbered sequentially throughtout the whole document including appendices.
If you want to number figures and tables separately for appendices, then the
command \verb|\floatsbyappendix| can be used anywhere before appendix
environment in the document. Then figures and tables in appendices are numbered
like A-1, A-2, B-1, etc.
A typical 12pt Pitt Engineering Ph.D. thesis document may start with
\begin{verbatim}
 \documentclass[12pt]{pitthesis}
\end{verbatim}
For double-sided printing, which would be handy for personal copies,
\begin{verbatim}
 \documentclass[12pt,twoside]{pitthesis}
\end{verbatim}
will typeset even and odd pages differently with proper margins for binding
and the page numbers are printed away from the bound sides of papers.
The \texttt{twoside} option may not change the page numbers unless
a \verb|\part{}| sectioning command is used in a thesis.

\section{Line Spacing}
The line spacing of texts can be selected by the class options of
\texttt{single}, \texttt{onehalf}, and \texttt{double}.  The default line
spacing is \texttt{onehalf} (1.5 line spacing), which I prefer for the final
printout, but double line spaced texts will give more space for proofreading.
A double line spaced thesis can be created by
\begin{verbatim}
 \documentclass[12pt,double]{pitthesis}
\end{verbatim}
The line spacing of texts can be further controlled by using
\verb|\baselinestretch| in the preamble of your document.  For example,
\begin{verbatim}
 \renewcommand{\baselinestretch}{1.3}
\end{verbatim}
will stretch the line spacing of texts.  As \verb|\baselineskip| is set
right after \verb|\begin{document}| by \texttt{pitthesis} class,
\verb|\baselinestretch| does not change the line spacing of the main texts
but the line spacing of other texts like tables.

%%--------------------------------------------------------------------------
\chapter{USING \texttt{pittthesis} DOCUMENT CLASS}
%%--------------------------------------------------------------------------

\section{Thesis Title Page}
Thesis title page carries much more information than the original \LaTeXe 's
\verb|\maketitle| can handle.  Therefore, more fields are introduced to
\verb|\maketitle| for the thesis classes (Please look at the source file
(\texttt{part1.tex}) of this document).  Beside the standard fields, i.e.
\verb|\title{}|, \verb|\author{}|, and \verb|\date{}|, the following additional
fields are introduced (these can not be found in the \LaTeXe\ manual).
\begin{enumerate}
    \item \verb|\degrees{}| -- for the degrees earned,
    \item \verb|\degree{}| -- for the candidate degree (the degree the thesis is
	    for),
    \item \verb|\school{}| -- such as School of Engineering or
	    School of Arts and Science
    \item \verb|\university{}| -- such as University of Pittsburgh
    \item \verb|\year{}| -- year to be printed in the title page
			  and the abstract page
\end{enumerate}

\section{Preparing a Thesis Proposal}

The class option \texttt{proposal} will prepare the title page for a proposal.
For example,
\begin{verbatim}
 \documentclass[12pt,proposal,phd]{pitthesis}
\end{verbatim}
and, in preamble (i.e., before \verb|\begin{document}|),
\begin{verbatim}
 \proposal{Ph.D. Dissertation Proposal}
\end{verbatim}
will typeset the title page for a thesis proposal.
A thesis proposal does not create a Committee Signature Page and does not
make the signature field for advisor's signature in the abstract page.
(It ignores \verb|\begin{committeesignature}| environment in the document, if
exists.)

\section{Committee Signature Page}

The heading of the committee signature page is generated by
\begin{verbatim}
\begin{committeesignature}
\end{verbatim}
and an optional argument of the number of committee members may be added by
\begin{verbatim}
\begin{committeesignature}[#]
\end{verbatim}
where \verb|#| controls the spacing of signature fields of committee members.
If this number is not given, the committee signature page will be typeset best
for 5 committee members (including the advisor).  If there are more or less
committee members than five, it would be better to give the number to the
optional argument for a proper spacing of committee members on the page. (e.g.,
\verb|\begin{committeesignature}[4]|)  The committee members in
\verb|{committeesignature}| environment must be placed in a proper order, as
those members will be typeset immediately at each command of the following:
\begin{enumerate}
    \item \verb|\advisor{}| (or \verb|\chairperson{}|)
    \item \verb|\coadvisor{}| (or \verb|\cochairperson{}|)
    \item \verb|\committeemember{}|
\end{enumerate}
A \verb|\committeemember{}| is added for each additional committee member.
The committee signature environment is ended by the following:
\begin{verbatim}
\end{committeesignature}
\end{verbatim}

\section{Abstract Page}

The advisor's name to be printed in the abstract page of Pitt thesis is defined
by either of the following commands:
\begin{verbatim}
 \advisorname{Firstname1 Lastname1, Ph.D.}
 \chairpersonname{Firstname1 Lastname1, Ph.D.}
\end{verbatim}
Otherwise, the name defined by \verb|\advisor{}| (or \verb|\chairperson{}|) will be
used instead. The following command
\begin{verbatim}
\abstract
\end{verbatim}
will generate the heading of the abstract page with a signature field,
advisor's name, thesis title, author's name and title, and the name of
a university.	For Pitt standard thesis, the year (set by \verb|\year{}|)
will be printed after the university name with a comma.  The texts of thesis
abstract will immediately follow the \verb|\abstract| command.
The author's title may be explicitly set by \verb|\authortitle{}| command
to change the default value set by the \texttt{pitthesis} class.  For example,
\begin{verbatim}
 \authortitle{M.A.}
\end{verbatim}
before \verb|\abstract| command will set the author's title to ``M.A.'' in the
Abstract page.

A list of key-words is listed under the heading ``DESCRIPTOR'' after the
abstract texts.  The descriptor list of this document was created by the
following commands:
\begin{verbatim}
\vspace{1em}
\section*{DESCRIPTORS}
\vspace{1em}
\begin{center}
\renewcommand{\arraystretch}{1.5}
\begin{tabular*}{\textwidth}{p{0.47\textwidth}p{0.47\textwidth}}
  \LaTeXe\ document class &
  Pitt engineering thesis \\
  Pitt standard thesis &
  Thesis document sample file \\
  University of Pittsburgh
\end{tabular*}
\end{center}
\end{verbatim}

\section{Appendices}

Appendix chapters are enclosed by the environment \verb|{appendices}|:
\begin{verbatim}
\begin{appendices}
\chapter{...}	% the first appendix
    ...
\chapter{...}	% the second appendix
    ...
\end{appendices}
\end{verbatim}
for multiple appendix chapters.  Each appendix chapters are numbered
alphbetically: APPENDIX A, APPENDIX B, APPENDIX C, and so on.
If there is only one appendix chapter in a thesis, the single appendix chapter
may not be numbered, which should be enclosed by a \verb|{singleappendix}|
environment:
\begin{verbatim}
\begin{singleappendix}	% single appendix chapter
\chapter{...}
    ...
\end{singleappendix}
\end{verbatim}
This single appendix chapter is referred by just APPENDIX without a chapter
number.

