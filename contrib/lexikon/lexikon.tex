%% 
%% This is file `lexikon.sty'
%% 
%% Copyright (C) 1998-1999 by Axel Kielhorn.  All rights reserved.
%% For additional copyright information see further down in this file.
%% 
%% This file is to be used with the LaTeX2e system.
%% ------------------------------------------------
%% 
%% This program can be redistributed and/or modified under the terms
%% of the LaTeX Project Public License Distributed from CTAN
%% archives in directory macros/latex/base/lppl.txt; either
%% version 1 of the License, or any later version.
%% 
%% IMPORTANT NOTICE:
%% 
%% Error reports in case of UNCHANGED versions to
%% Axel Kielhorn
%% A.Kielhorn@tu-bs.de
%% 

%\documentclass[a5paper,10pt,twoside,titlepage]{article}
\documentclass[a5paper,10pt,titlepage]{article}
\usepackage{german}
\usepackage{lexikon}

\def\dictfile{test}

\title{Englisch Kurzreferenz}
\author{Axel Kielhorn}

\begin{document}
\maketitle

\section*{Einleitung}

Die Eintr\"{a}ge in dieser Kurzreferenz sind folgenderma\ss en aufgebaut:
\begin{itemize}
\item Schl\"{u}sselwort in deutsch (\textbf{fett})
\item Informatioen zur Aussprache gem\"{a}\ss\ "`Wahrig Deutsches W\"{o}rterbuch"''
\item Grammatikalische Information (\textit{kursiv}) hierbei bedeutet
  \begin{description}
  \item [\textit{f}] weiblich
  \item [\textit{m}] m\"{a}nnlich
  \item [\textit{n}] s\"{a}chlich
  \end{description}
\item Erkl\"{a}rung zum Schl\"{u}sselwort
\item Schl\"{u}sslewort in Englisch (\textbf{fett})
\item Information zur Aussprache gem\"{a}\ss\ "`Merriam Webster"'
\item Grammatikalische Information (\textit{kursiv}) hierbei bedeutet
  \begin{description}
  \item [\textit{n}] noun (Hauptwort)
  \item [\textit{v}] verb (Tuwort)
  \item [\textit{[C]}] countable (z\"{a}hlbar)
  \item [\textit{[U]}] uncountable (nicht z\"{a}hlbar, diese W\"{o}rter haben keine 
  Mehrzahl und k\"{o}nnen nicht mit W\"{o}rtern wie "`ein"' oder "`viele"'' 
  benutzt werden. Klassisches Beispiel: \textbf{Information}.)
  \end{description}
\item Erkl\"{a}rung zum englischen Schl\"{u}sselwort. Schr\"{a}nkt u.\,U. die 
Verwendung im Englischen ein.
\end{itemize}

\newpage 

\pagestyle{dictheadings} %% use scrpage or fancyhdr for a differnt layout

\dictchar{D}
\dictentry{Drehbank}{}{f}{}
  {lathe}{}{n}{machine for holding and turning pieces of metal while they are being shaped}
\dictentry{Drehmoment}{}{n}{}
  {torque}{}{n [U]}{twisting force causing rotation}

\dictchar{G}
\dictentry{Gabelstabler}{}{m}{}
  {fork-lift truck}{}{}{powered truck or trolley with mechanical means of lifting and lowering goods}
\dictentry{Geschwindigkeit}{}{f}{}
  {velocity}{}{n [U]}{speed, rate of motion}
\dictentry{Getriebe}{}{n}{}
  {gear drive}{}{n}{}

\dictchar{L}
\dictentry{Lager}{}{n}{}
  {bearing}{}{n (usu pl)}
  {device that supports moving parts an reduces friction}
\dictentry{Leim}{}{m}{}
  {glue}{}{n [U]}{thick, sticky liquid used for joining things}


\end{document}
\endinput
%% 
%% End of file `dictionary.tex'.
