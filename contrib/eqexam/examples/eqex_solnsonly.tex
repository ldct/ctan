\documentclass{article}
\usepackage[fleqn]{amsmath}
%\usepackage[pointsonleft,nototals,forpaper,useforms,vspacewithsolns]{eqexam}
\usepackage[pointsonleft,nototals,forpaper,useforms,solutionsonly]{eqexam}

\solAtEndFormatting{\eqequesitemsep{3pt}}

\subject[MC]{My Course}
\title[T1]{Test 1}
\author{Dr.\ D. P. Story}
\date{\thisterm, \the\year}
\duedate{2012/04/28}
\keywords{My Course, Exam \nExam, {\thisterm} semester, \theduedate, at AcroTeX.Net}

%
% Compile the file with the vspacewithsolns option to create the .sol auxiliary file
% that contains a listing of all the solutions, like so
%   \usepackage[pointsonleft,nototals,forpaper,useforms,vspacewithsolns]{eqexam}
% then compile with the solutionsonly option, like so
%   \usepackage[pointsonleft,nototals,forpaper,useforms,solutionsonly]{eqexam}
%
\encloseProblemsWith{theseproblems}

\begin{document}

\maketitle

\begin{exam}{myProblems}

\ifsolutionsonly
\begin{instructions}[Solutions.]
The solutions to Test~1.
\end{instructions}
\belowsqskip{\par} % removes the skip after the exam env
\else
\begin{instructions}
Solve each problem and box in your final $\boxed{\text{answer}}$.
\end{instructions}
\fi

\begin{theseproblems}

\begin{problem*}[3ea]
\leadinitem
    It is well known that \fillin{1in}{Newton} and
    \fillin{1in}{Leibniz} are jointly credited as the founders
    of modern calculus.
\ifkeyalt
\begin{solution}
    It is well known that \fillin{1in}{Newton} and
    \fillin{1in}{Leibniz} are jointly credited as the founders
    of modern calculus.
\end{solution}
\fi

\begin{parts}
\item \TF{T} (True `T' or False `F') The area of a circle is $\pi r^2$.
\ifkeyalt
\begin{solution}
\item\TF{T} The area of a circle is $\pi r^2$.
\end{solution}
\fi

\item Suppose the \emph{discriminant} of a quadratic equation is
\emph{negative}, which statement describes the roots to
the equation?
\begin{answers}{2}
\bChoices
    \Ans0 There is only one real root\eAns
    \Ans0 There are two distinct real roots\eAns
    \Ans1 There are two complex roots\eAns
    \Ans0 None of these\eAns
\eChoices
\end{answers}
\ifkeyalt
\begin{solution}[]
\parbox[t]{\linewidth}{\sqTabPos{t}%
\begin{answers}{2}
\bChoices
    \Ans0 There is only one real root\eAns
    \Ans0 There are two distinct real roots\eAns
    \Ans1 There are two complex roots\eAns
    \Ans0 None of these\eAns
\eChoices
\end{answers}}%
\adjDisplayBelow
\end{solution}
\fi
\end{parts}
\end{problem*}

\begin{problem}[8]
In the space below, solve the quadratic equation
$ 2x^2 - 3x + 2 = 0 $ using any valid method.
\begin{solution}[1.5in]
We use the quadratic formula:
\begin{equation*}
    x  = \frac{3 \pm \sqrt{9-4\cdot2\cdot2}}{2\cdot2}
       = \frac{3 \pm \sqrt{-7}}{4}
       = \boxed{ \frac{3 \pm \sqrt{7}\,\imath}{4} }
\end{equation*}
\end{solution}
\end{problem}

\begin{problem}[8]
Find the equation of the line perpendicular to $ 3x - 5y = 2 $ and passing
through the point $P(1,7)$. Leave your answer in the general form.
\[
    \text{Ans:\quad}
    \fillin[boxed,boxsize=Large]{2in}{5x+3y=26}
\]
\begin{solution}[.5in]\ifkeyalt$ \text{Ans:\quad}
    \fillin[boxed,boxsize=Large]{2in}{5x+3y=26}$\\[3pt]\fi
Apparently the given line has slope $m=3/5$, so $ m_{\perp}=-5/3$. The rest
is left to the reader.
\end{solution}
\end{problem}

\end{theseproblems}
\end{exam}
\end{document}
