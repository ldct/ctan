%% uhrzeit-doc.tex
%%
%% Deutsche Dokumentation zu uhrzeit.sty
%% Olaf Meltzer <olaf.meltzer@gmx.de>
%% 
\documentclass{scrartcl}
\usepackage[utf8]{inputenc}
\usepackage[T1]{fontenc}
\usepackage{microtype,url,uhrzeit}
\usepackage[german]{babel}

\usepackage{hyphsubst}% Trennregeln austauschen
\HyphSubstIfExists{ngerman-x-latest}{%
   \HyphSubstLet{ngerman}{ngerman-x-latest}}{}
\HyphSubstIfExists{german-x-latest}{%
   \HyphSubstLet{german}{german-x-latest}}{}
\usepackage{babel}

\usepackage{graphicx}
\usepackage{xcolor,loremipsum}

%% Schusterjungen und Hurenkinder vermeiden:
\clubpenalty=10000
\widowpenalty=10000


\begin{document}

\title{uhrzeit.sty\\\protect\Large v0.2c}
\author{Olaf Meltzer\thanks{\protect\url{olaf.meltzer@gmx.de}}}
\date{1.\,Februar 2016}

\maketitle

\begin{abstract}
Formatierung von Uhrzeit und Uhrzeitbereichen in der ehemals in
Deutschland gebräuchlichen Form \uhr{8}{45} bzw. \vonbis{8}{45}{10}{30}.
\end{abstract}

\tableofcontents

\newpage

\section{Zweck}
Das \LaTeX2e-Paket \verb|uhrzeit.sty| soll hauptsächlich das
Setzen von Uhrzeiten und Uhrzeitbereichen in den ehemals in
Deutschland gebräuchlichen Formen

\begin{center}\uhr{8}{45}\end{center}

beziehungsweise

\begin{center}\vonbis{8}{45}{10}{30}\end{center}

erleichtern. Außerdem stellt es Makros zur Ausgabe der
aktuellen Uhrzeit in einigen derzeit noch üblichen Formaten bereit.

\section{Einbindung}

Man bindet das Paket \verb|uhrzeit.sty| durch folgende Zeile in der
Präambel eines \LaTeX2e-Dokumentes ein.

\begin{center}
\verb|\usepackage{uhrzeit}|
\end{center}

Das Paket \verb|soul.sty|, das \verb|uhrzeit.sty| zur Darstellung
hochgestellter, unterstrichener Minutenzahlen benutzt, wird
automatisch einbezogen. Notfalls kann man \verb|soul.sty| aus dem
CTAN installieren, aber das Paket sollte eigentlich in jeder
\TeX-Distribution enthalten sein.

\section{Anwendung}

\subsection{Manuelle Eingabe}

Das ehemals in Deutschland gebräuchliche Format der Uhrzeit
bzw. eines Uhrzeitbereiches in den Formen \uhr{8}{45}
bzw. \vonbis{8}{45}{10}{30} erreicht man duch die Eingabe als \verb|\uhr{8}{45}|
bzw. \verb|\vonbis{8}{45}{10}{30}|.

\subsection{Automatische Ausgabe der aktuellen Uhrzeit}

Für die automatische Ausgabe der zum Zeitpunkt des Kompilierens
(des \LaTeX2e-Laufs) aktuellen Uhrzeit stellt \verb|uhrzeit.sty|
die in Tabelle \ref{aktuhr} auf Seite \pageref{aktuhr} genannten
Makros zur Verfügung.

\begin{table}
\caption{Makros für die aktuelle Uhrzeit}
\label{aktuhr}
\begin{center}
\begin{tabular}{lrl}
Makro&Format&Gebrauch\\
\hline
\verb|\dtd|&5.07&Eher ungebräuchlich\\
\verb|\dtc|&05:07&International gebräuchlich\\
\verb|\uhri|&5.07\,Uhr&Duden, 22. Auf\/lage, S. 110\\
\verb|\uhrii|&05:07\,Uhr&DIN 5008\\
\verb|\uhriii|&\uhr{5}{07}&Ehemals in Deutschland üblich\\
\verb|\uhriv|&\uhr{05}{07}&Auch früher eher selten gebraucht\\
\hline
\end{tabular}
\end{center}
\end{table}

\section{Historie}

Leif Albers \verb|<leif@mathematik.uni-bielefeld.de>| entwarf das
Makro \verb|\mytime|, und veröffentlichte es am 04.10.1999 in
\verb|de.comp.text.tex|.

Andreas Matthias \verb|<amat@kabsi.at>| zeigte seine Definition von
\verb|\uhr{}{}| am 24.09.2000 in \verb|de.comp.text.tex|.

\subsection{v0.2b}
Eine erste zur Veröffentlichung geeignete Version von
\verb|uhrzeit.sty| stellte ich als v0.2b vom 09.08.2001 zur
allgemeinen Verfügung ins CTAN.

\subsection{v0.2c}
Die jetzt veröffentlichte Version v0.2c ist von den Makros her
mit der Version v0.2b identisch. Ich habe lediglich die Historie
des Paketes angepasst, ein deutsches Manual im PDF-Format samt
dazu gehörigem Quelltext bereit gestellt und alles unter LaTeX
Project Public License veröffentlicht.

\section{Lizenz}

Das Paket \verb|uhrzeit.sty| soll unter der LaTeX Project Public
License (LPPL)\footnote{\url{http://www.latex-project.org/lppl/}}
verfügbar sein.

\end{document}

%%% Local Variables: 
%%% mode: latex
%%% TeX-PDF-mode: t
%%% TeX-master: t
%%% End: 
