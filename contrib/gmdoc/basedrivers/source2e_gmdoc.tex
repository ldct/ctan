% \iffalse meta-comment
%
% Copyright 1993 1994 1995 1996 1997 1998 1999 2000 2001 2006
% The LaTeX3 Project and any individual authors listed elsewhere
% in this file. 
% 
% This I~hope would be part of the LaTeX base system.
% -------------------------------------------
% 
% It may be distributed and/or modified under the
% conditions of the LaTeX Project Public License, either version 1.2
% of this license or (at your option) any later version.
% The latest version of this license is in
%    http://www.latex-project.org/lppl.txt
% and version 1.2 or later is part of all distributions of LaTeX 
% version 1999/12/01 or later.
% 
% The list of all files belonging to the LaTeX base distribution is
% given in the file `manifest.txt'. See also `legal.txt' for additional
% information.
% 
% \fi
%
% \division{Some Typesetting Remarks}
%
% This driver typesets The Source $2_\varepsilon$ included in the 
% \TeX Live 2005 distribution. Some tricks here are done just for
% fixing typos in the Source Files. The Source Files themselves are
% intact.
%
% Most probably you should redefine the |\BasePath| macro so that it
% was the path of the \file{\dots/source/latex/base} directory on your
% system. The path levels should be separated with slashes (even on
% Windows) and should also end with a~slash (to concatenate well with
% the file name).
%
% \stanza
% While \TeX ing The Source again after a~fatally erroneous pass there
% happened the `\TeX\ capacity exceded error' sometimes. \TeX ing once
% again was the right thing to do.
%
% The \pk{hyperref} package usually issues some warnings about non
% existence of some hypertargets. I~consider it rather a~feature of
% \pk{hyperref} (a~bug?) than a~bug in the typeset file(s).
%
% \stanza
% One more thing you shouldn't bother of is the differences of the
% checksums, I~mean the usual \pk{gmdoc} message that the checksum
% stated in the file differs from \pk{gmdoc}'s own count. That is
% O.K.\ since the checksum stated in a~traditional \file{.dtx}
% is the number of backslashes in the macrocodes while the checksum
% handled and expected by \pk{gmdoc} is the number of \emph{the escape ^^B
% chars}. Don't get the difference? Assume the declared code escape
% char is |\| (as usual) and consider |\\| in the code. Due to
% the traditional counting this CS increases the checksum by 2 while
% due to mine by 1: the second bslash is \emph{not} escape char: it's
% the CS name.
%
% Moreover, when you declare |\CodeEscapeChar\!| e.g., the code 
%\[\hbox{|!Alice !\!has !an !aligator|}\]
% increases the `new way' checksum by 5 not by 1 as it would do the
% traditional one.
%
% \stanza
% This driver uses an unofficial little package \pk{gmeometric} to
% allow the |\geometry| command also inside \env{document}. This
% package is included in the drivers' directory.
%
%
% \division{The Body}
%
% ``This document will typeset the \LaTeX\ sources as a single document.
% This will produce quite a large file (roughly 555 pages) and may
% take a long time.
%
% Some notes on processing this document are contained at the end
% of this document's source file, after |\end{document}| (not typeset).''
% 
% GM 2006/12/1
%
% First a special index style for makeindex.
%

\begin{filecontents}{gmsource2e.ist}
preamble
"\n \\begin{theindex} \n"
postamble
"\n\n \\end{theindex}\n"
% ^^A~The next lines will produce some warnings when
% ^^A~running Makeindex as they try to cover two different
% ^^A~versions of the program:
% ^^A
% ^^A~The three lines below caused unclosed groups in the .ind
% file. May they be cursed!
% ^^A~lethead_prefix   "{\\bfseries\\hfill "
% ^^A~lethead_suffix   "\\hfill}\\nopagebreak\n"
% ^^A~lethead_flag       1
heading_prefix   "{\\bfseries\\hfill "
heading_suffix   "\\hfill}\\nopagebreak\n"
headings_flag       1

% and just for source2e:\\
% Remove R so I is treated in sequence I J K not I II III
page_precedence "rnaA"
\end{filecontents}


\PassOptionsToPackage{codespacesgrey, indexallmacros}{gmdoc}

\if11
  \documentclass[debug, minion, cronos, cursor, fontspec=quiet]{gmdocc}% 

  % ^^A[FakeBold=2, FakeStretch=0.87]{Nimbus Mono L}
  \mcdiagOn
  
\else
  \documentclass[fontspec=quiet]{gmdocc}%
\fi

\foone{\catcode`_=12 }
{\if1 1\includeonly{source2e_by_gmdoc}\fi}

%^^A\usepackage[margin=2cm,b5paper]{geometry}% and this
\usepackage{gmoldcomm}% Definitions of \env{oldcomments} and |\task|.


\listfiles

\ltxLookSetup
\gmdoccMargins
\olddocIncludes% This is the crucial declaration to drive \pk{gmdoc} into
% the traditional settings.
\twocoltoc% For towocolumn table of contents.

\DeleteShortVerb\|
\OldMakeShortVerb*\|% To define shortverb \verb+|+ such that it
% remains shortverb in math mode (by default I~define it to be $\vert$ in
% math mode.

% Do not index some TeX primitives, and some common plain TeX commands.

\DoNotIndex{\def,\long,\edef,\xdef,\gdef,\let,\global}
\DoNotIndex{\if,\ifnum,\ifdim,\ifcat,\ifmmode,\ifvmode,\ifhmode,%
            \iftrue,\iffalse,\ifvoid,\ifx,\ifeof,\ifcase,\else,\or,\fi}
\DoNotIndex{\box,\copy,\setbox,\unvbox,\unhbox,\hbox,%
            \vbox,\vtop,\vcenter}
\DoNotIndex{\@empty,\immediate,\write}
\DoNotIndex{\egroup,\bgroup,\expandafter,\begingroup,\endgroup}
\DoNotIndex{\divide,\advance,\multiply,\count,\dimen}
\DoNotIndex{\relax,\space,\string}
\DoNotIndex{\csname,\endcsname,\@spaces,\openin,\openout,%
            \closein,\closeout}
\DoNotIndex{\catcode,\endinput}
\DoNotIndex{\jobname,\message,\read,\the,\m@ne,\noexpand}
\DoNotIndex{\hsize,\vsize,\hskip,\vskip,\kern,\hfil,\hfill,\hss}
\DoNotIndex{\m@ne,\z@,\z@skip,\@ne,\tw@,\p@}
\DoNotIndex{\dp,\wd,\ht,\vss,\unskip}

% Set up the Index and Change History to use |\part|.
\makeatletter
\def\indexdiv{\part*}
\AtDIPrologue{\@ifnotmw{%
    \markboth{Index}{Index}%
    \addcontentsline{toc}{part}{Index}}{}%
}

\GlossaryPrologue{\part*{Change History}%
  % Allow control names to be hyphenated here\dots
  {\GlossaryParms\ttfamily\hyphenchar\font=`\-}%
  \@ifnotmw{%
    \markboth{Change History}{Change History}%
    \addcontentsline{toc}{part}{Change History}}{}%
}

% ``The standard |\changes| command modified slightly to better cope with
% this multiple file document.''--- Not quite:
\makeatletter
% \begin{verbatim}
% % \def\changes@#1#2#3{%
% %   \let\protect\@unexpandable@protect
% %   \edef\@tempa{\noexpand\glossary{#2\space\currentfile\space#1\levelchar
% %                                  \ifx\saved@macroname\@empty
% %                                    \space
% %                                    \actualchar
% %                                    \generalname
% %                                  \else
% %                                    \expandafter\@gobble
% %                                    \saved@macroname
% %                                    \actualchar
% %                                    \string\verb\quotechar*%
% %                                    \verbatimchar\saved@macroname
% %                                    \verbatimchar
% %                                  \fi
% %                                  :\levelchar #3}}%
% %   \@tempa\endgroup\@esphack}
%\end{verbatim}
\makeatother

% Produce a Change Log and (2 column) Index.
\RecordChanges
\CodelineIndex
\EnableCrossrefs
\setcounter{IndexColumns}{2}

% Needed for documentation in \file{ltoutenc.dtx}.
\usepackage{textcomp}

\olddocIncludes
\HideAllDefining

\fooatletter{%
  \@ifXeTeX{%
    \def\"#1{%
      \if o#1\oumlaut\fi
      \if u#1\uumlaut\fi
%^^A  \else\typeout{****** \cs{"} with argument }\show#1\fi
    }}{}}

\foone{\makeatletter\catcode`\#=12 }{%
  \def\gmd@wykrzykniki{# # # # # # # # #}}

\begin{document}


 \title{The \LaTeXe\ Sources\thanks{Typeset with \pk{gmdoc} by Natror
     on \today.}}
 \author{%
  Johannes Braams\\
  David Carlisle\\
  Alan Jeffrey\\
  Leslie Lamport\\
  Frank Mittelbach\\
  Chris Rowley\\
  Rainer Sch\"opf}


\def\BasePath{/home/natror/texmf/source/latex/base/}


% This command will be used to input the patch file
% if that file exists.
\newcommand{\includeltpatch}{%
  \def\currentfile{ltpatch.ltx}
  \part{ltpatch}
  {\let\ttfamily\relax
    \xdef\filekey{\filekey, \thepart={\ttfamily\currentfile}}}%
  Things we did wrong\ldots
  \IndexInput{ltpatch.ltx}}



% Get the date from \file{ltvers.dtx}
\makeatletter
\let\patchdate=\@empty
\begingroup
   \def\ProvidesFile#1\fmtversion#2{\date{#2}\endinput}
   \input{\BasePath ltvers.dtx}
\global\let\X@date=\@date

% Add the patch version if available.
   \long\def\Xdef#1#2#3\def#4#5{%
    \xdef\X@date{#2}%
    \xdef\patchdate{#5}%
    \endinput}%
   \InputIfFileExists{ltpatch.ltx}
    {\let\def\Xdef}{\global\let\includeltpatch\relax}
\endgroup

\ifx\@date\X@date
   \def\Xpatch{0}
   \ifx\patchdate\Xpatch\else
     \edef\@date{\@date\space Patch level \patchdate}
   \fi
\else
   \@warning{ltpatch.ltx does not match ltvers.dtx!}
   \let\includeltpatch\relax
\fi
\makeatother

\pagenumbering{roman}
\thispagestyle{empty}

%^^A\traceon
\maketitle
\relax
%^^A \traceoff
\emptify\maketitle

\tableofcontents

\clearpage

\pagenumbering{arabic}


%%%%%%%%%%%%%%%%%%%%%%%%%%%%%%%%%%%%%%%%%%%%%%%%%%%%%%%%%%%%

%``Each of the following |\DocInclude| lines includes a file with extension
% \file{.dtx}. Each of these files may be typeset separately. For instance
%\[\hbox{|latex ltboxes.dtx|}\]
% will typeset the source of the \LaTeX\ box commands.'' 
%
% (Well, I~(Natror) prepared only this common driver.)
%
% If this file is processed, each of these separate \file{.dtx} files will be
% contained as a part of a single document. Using \file{ltxdoc.cfg} you can
% then optionally produce a combined index and/or change history for
% the entire source of the format file. Note that such a document will
% be quite large (about 555 pages).
%
 
\DocInclude[\BasePath]{ltdirchk} % System dependent initialisation

\AfterMacrocode{53}{\def\do{\cs{do}}}% A~bare |\do| in narration on
% line 161.
 \DocInclude[\BasePath]{ltplain}  % LaTeX version of Knuth's \file{plain.tex}.

 \DocInclude[\BasePath]{ltvers}   % Current version date.

 \DocInclude[\BasePath]{ltdefns}  % Initial definitions.

 \DocInclude[\BasePath]{ltalloc}  % Allocation of counters and others.

 \DocInclude[\BasePath]{ltcntrl}  % Program control macros.

 \DocInclude[\BasePath]{lterror}  % Error handling.

 \DocInclude[\BasePath]{ltpar}    % Paragraphs.

 \DocInclude[\BasePath]{ltspace}  % Spacing, line and page breaking.

 \DocInclude[\BasePath]{ltlogos}  % Logos.

 \DocInclude[\BasePath]{ltfiles}  % |\input| files and related commands.

\AtBegInputOnce{\let\task\gobble}% In general |\task| gobbles two, but
% in this file it's used with one argument and next to it is
% |\changes| (which in \pk{gmdoc} is |\outer| so gobbling it raises an
% error). 
 \DocInclude[\BasePath]{ltoutenc} % Output encoding interface.

 \DocInclude[\BasePath]{ltcounts} % Counters.

 \DocInclude[\BasePath]{ltlength} % Lengths.

 \DocInclude[\BasePath]{ltfssbas} % NFSS Base macros.

%^^A\AtBegInputOnce{\addtomacro\maketitle{\traceoff}\traceon}%{\nameshow{@author}\show\maketitle\nameshow{@maketitle}}
 \DocInclude[\BasePath]{ltfsstrc} % NFSS Tracing (and \file{tracefnt.sty}).

 \DocInclude[\BasePath]{ltfsscmp} % NFSS1 Compatibility.

 \DocInclude[\BasePath]{ltfssdcl} % NFSS Declarative interface.

 \DocInclude[\BasePath]{ltfssini} % NFSS Initialisation.

 \DocInclude[\BasePath]{fontdef}  % \file{fonttext.ltx/fontmath.ltx}

 \DocInclude[\BasePath]{preload}  % \file{preload.ltx}

 \DocInclude[\BasePath]{ltfntcmd} % |\textrm| etc.
 
 \DocInclude[\BasePath]{ltpageno} % Page numbering.

 \DocInclude[\BasePath]{ltxref}   % Cross referencing.

 \AfterMacrocode{1137}{\let\GMDebugCS\cs 
   \def\cs##1{\expandafter\GMDebugCS\expandafter{\string##1}}}
 % |\cs{\@defaultsubs}| on line 257, |\cs{\@refundefined}| on line
 % 263. It's the first step. The next is done before |\PrintChanges|.
 \AfterMacrocode{1139}{\let\cs\GMDebugCS}
 \AfterMacrocode{1183}{% The last |\changes| have second argument
   % |{1994/05/26/16}|. 
   \csname changes\endcsname{v0.9i}{1993/12/16}{\cs{literal} added}%
   \csname changes\endcsname{v1.0r}{1994/05/26}{\cs{literal} removed}%
   \gdef\GMdebugChanges{\expandafter\def\csname
     changes\endcsname####1####2####3{}}%
   \aftergroup\GMdebugChanges}% A~trick with |\aftergroup| 'cause 
 % that \env{macrocode} is inside \env{macro}.
 \DocInclude[\BasePath]{ltmiscen} % Miscellaneous environment definitions.

 \DocInclude[\BasePath]{ltmath}   % Mathematics set up.

 \DocInclude[\BasePath]{ltlists}  % List and related environments.

 \DocInclude[\BasePath]{ltboxes}  % Parbox and friends.

 \DocInclude[\BasePath]{lttab}    % \env{tabbing}, \env{ tabular} and \env{array}.

 \DocInclude[\BasePath]{ltpictur} % Picture mode.

 \DocInclude[\BasePath]{ltthm}    % Theorem environments.

 \DocInclude[\BasePath]{ltsect}   % Sectioning.

 \DocInclude[\BasePath]{ltfloat}  % Floats.

 \DocInclude[\BasePath]{ltidxglo} % Index and Glossary.

 \DocInclude[\BasePath]{ltbibl}   % Bibliography.

 \DocInclude[\BasePath]{ltpage}   % |\pagestyle|, |\raggedbottom|, |\sloppy|.

 \DocInclude[\BasePath]{ltoutput} % Output routine.

 \DocInclude[\BasePath]{ltclass}  % Package \& Class interface.

 \DocInclude[\BasePath]{lthyphen} % Hyphenation (\file{hyphen.ltx}).

 \DocInclude[\BasePath]{ltfinal}  % Last minute initialisations and dump.

 \includeltpatch       % Corrections distributed after the full release.

% Stop here if \file{ltxdoc.cfg} says |\AtEndOfClass{\OnlyDescription}|
\StopEventually{\end{document}}

\clearpage
\pagestyle{headings}

% Make \TeX\ shut up.
\hbadness=10000
\newcount\hbadness
\hfuzz=\maxdimen

\typeout{%
  ^^JProduce change log with^^J%
  makeindex -r -s gmglo.ist -o \jobname.gls \jobname.glo^^J}

{%^^A\def\.{\cs{.}}% 'cause there's |\cs {\.}| in change log on l.4314
  % The next step of debug of \file{ltmiscen.dtx}'s
  % |\changes...{...\cs{\@defaultsubs...}}| etc.\\ How does it work?
  % Remember |\cs| is robust. The typo lies in giving it a~CS argument
  % instead of expected CS name without backslash. So, in the step 1
  % we only |\string| the argument CS to let it be written outto the
  % \file{.glo} file. Then, in step 2, we redefine |\cs| to first
  % |\string| its argument inside an |\if|. Remember that |\if|
  % expands two tokens next to it until it finds sth.\ unexpandable,
  % so it'll execute |\string|. Then, if the first char of the
  % |\stringed| argument is |\|\catother, the condition is satisfied
  % and |\if...\fi| expands to what follows that backslash and
  % precedes |\else|. So if the argument was a~CS, its backslash will
  % be gobbled by |\if|. Otherwise |\if...\fi| expands to what is
  % between |\else| and |\fi|, to the unchanged argument that is. Then
  % to that list of tokens the original |\cs| is applied.
  \let\GMDebugCS\cs
  \def\cs#1{\GMDebugCS{\if\bslash\string#1\else#1\fi}}%
  \PrintChanges}

\typeout{%
  ^^JProduce index with^^J%
  makeindex -r -s gmsource2e.ist \jobname.idx^^J}

% ``Makeindex needs a symbol between the parts of composite page numbers
% but we dont want one, so:''---I~skip that.
%
%\begin{verbatim}
%% \begingroup
%% \def\endash{--}
%% \catcode`\-\active
%% \def-{\futurelet\temp\indexdash}
%% \def\indexdash{\ifx\temp-\endash\fi}
%\end{verbatim}
\geometry{bottom=3.6cm}
\clearpage
%^^A\onecolumn
%^^A~\clearpage is contained in |\geometry|.


\PrintIndex
%^^A\endgroup

% Make sure that the index is not printed twice
% (ltxdoc.cfg might have a second \PrintIndex command)
\let\PrintChanges\relax
\let\PrintIndex\relax

\gmdoccMargins
\clearpage
\csname @ifnotmw\endcsname{\pagestyle{headings}}{\pagestyle{outer}}

\gmdocIncludes
\SelfInclude{%^^A\def\CommonEntryCmd{UsgEntry}%\label{UsgHack}
  % ^^A\label{SelfIncludeUsg}
  \csname gag@index\endcsname% we turn writing outto the \file{.idx}
  % out for the driver since it's \emph{not} a~part of The Source.
%^^A  \filediv[The New Driver?]{How I~did It or \gmhypertarget{The Driver} File}% 
}

\end{document}
%\NoEOF

%%%%%%%%%%%%%%%%%%%%%%%%%%%%%%%%%%%%%%%%%%%%%%%%%%%%%%%%%%%%%%

To use this file to produce a fully indexed source code
you need to execute the following (or equivalent) commands:

   latex source2e_by_gmdoc.tex

   makeindex -s gmsource2e.ist source2e_by_gmdoc.idx
   makeindex -s gmglo.ist -o source2e_by_gmdoc.gls source2e_by_gmdoc.glo

   latex source2e_by_gmdoc.tex
   latex source2e_by_gmdoc.tex


The makeindex style gmsource2e.ist is used in place of the usual
doc gind.ist to ensure that I is used in the sequence I J K
not I II II, which would be the default makeindex behaviour.

The third run with latex is only required to get the table of
contents entries for the change log and index. You may speed things up
by using the \includeonly mechanism so as not to typeset the source
files on the second run. This involves changing the file
ltxdoc.cfg
between the latex runs.

The following unix script automates this.
  (It could easily be ported to scripts for DOS or VMS,
   rm is ReMove a file, and echo "..." > file writes ... to "file".)


After this script (after the second ==============) is a similar script
that will produce the documentation for all the files in the base
distribution that are *not* included in source2e.dvi. This second script
was requested, but before using it, beware it will take a long time!
It may however be modified as required, eg to not typeset the fdd files
or whatever...


!!!!!!!!!!!!!!!!!!!!!!!!!!!!!!!!!!!!!!!!!!!!
Natror (GM): I~didn't touch the following so it's probably not quite suitable
for gmdoc-ing.
!!!!!!!!!!!!!!!!!!!!!!!!!!!!!!!!!!!!!!!!!!!!


==============
#!/bin/sh

rm  -f source2e_by_gmdoc.gls source2e_by_gmdoc.ind source2e_by_gmdoc.toc

# First run: 
# Create new standard ltxdoc.cfg file
# Pass the (possibly empty) list of arguments supplied on the
# command line to article class.
#
# If you use A4 paper, running this script with argument
#    a4paper
# may save about 30 pages.
#
echo "\PassOptionsToClass{$*}{article}" > ltxdoc.cfg


# Now LaTeX the file with this cfg file.
#
latex source2e.tex


# Make the Change log and Glossary.
#
makeindex -s source2e.ist source2e.idx
makeindex -s gglo.ist -o source2e.gls source2e.glo


# Second run: append \includeonly{} to ltxdoc.cfg to speed up things
# (this run needed only to get changes and index listed in .toc file)
#
# Note that the index will not be made incorrect by the insertion
# of the table of contents as the front matter uses a diferent page
# numbering scheme.
#
echo "\includeonly{}" >> ltxdoc.cfg

latex source2e.tex


# Third and final run, to put everything together.
# First restore the cfg file:
#
echo "\PassOptionsToClass{$*}{article}" > ltxdoc.cfg
latex source2e.tex


==============
#!/bin/sh

# Running this script will process all the dtx fdd and *guide.tex
# and ltnews*.tex files in the LaTeX distribution, except the dtx
# files included in source2e.tex. 
# (The shell first script in the comments of source2e.tex will
#  process those.)

# Any command line arguments (eg a4paper) are taken as options to the
# article class.

# This script is likely to take ages!

echo "\PassOptionsToClass{$*}{article}"                  > ltxdoc.cfg
echo "\batchmode"                                       >> ltxdoc.cfg

# The next four lines produce full indexes and change logs
# you may not want those.
echo "\AtBeginDocument{\RecordChanges}"                 >> ltxdoc.cfg
echo "\AtEndDocument{\PrintChanges}"                    >> ltxdoc.cfg
echo "\AtBeginDocument{\CodelineIndex\EnableCrossrefs}" >> ltxdoc.cfg
echo "\AtEndDocument{\PrintIndex}"                      >> ltxdoc.cfg

# If you do not want any code listings, just documentation, then instead
# of the above four lines, uncomment the following:
# echo "\AtBeginDocument{\OnlyDescription}"                >> ltxdoc.cfg

echo "\PassOptionsToClass{$*}{article}"                  > ltxguide.cfg
echo "\batchmode"                                       >> ltxguide.cfg

cp ltxguide.cfg ltnews.cfg


for i in *dtx *fdd *guide.tex ltnews*.tex
do
B=`basename $i .dtx`

if (grep "Include{$B}" source2e.tex >/dev/null ; )
then
echo In source2e: $i
else
echo latex $i
  if (latex $i > /dev/null) 
  then
    echo latex $i
    latex $i > /dev/null
    echo makeindex -s gind.ist $B.idx
    makeindex -s gind.ist $B.idx > /dev/null 2> /dev/null
    echo makeindex -s gglo.ist -o $B.gls $B.glo
    makeindex -s gglo.ist -o $B.gls $B.glo > /dev/null 2> /dev/null
    echo latex $i
    latex $i > /dev/null
  else
    echo "!!! LaTeX ERROR: $i. (See $B.log.)"
  fi
fi

done
