% \iffalse
%%
%% ----------------------------------------------------------------------------
%%
%% Jochen�s LaTeX document class for typesetting theses
%%
%% Copyright (C) 2000,2001,2002 Jochen K�pper
%% $Id: jkthesis.dtx,v 1.107 2002/07/18 06:11:05 jochen Exp $
%%
%% This file may be distributed and/or modified under the conditions of the 
%% LaTeX Project Public License, either version 1.2 of this license or (at your
%% option) any later version. The latest version of this license is in
%%    http://www.latex-project.org/lppl.txt
%% and version 1.2 or later is part of all distributions of LaTeX version 
%% 1999/12/01 or later.
%%
%% It would be nice if you recognize using this class in your copyright page.
%% See the default \copyrightpage entry for details.
%%
%% ----------------------------------------------------------------------------
%%
% \fi
%%
%% \CharacterTable
%%  {Upper-case    \A\B\C\D\E\F\G\H\I\J\K\L\M\N\O\P\Q\R\S\T\U\V\W\X\Y\Z
%%   Lower-case    \a\b\c\d\e\f\g\h\i\j\k\l\m\n\o\p\q\r\s\t\u\v\w\x\y\z
%%   Digits        \0\1\2\3\4\5\6\7\8\9
%%   Exclamation   \!     Double quote  \"     Hash (number) \#
%%   Dollar        \$     Percent       \%     Ampersand     \&
%%   Acute accent  \'     Left paren    \(     Right paren   \)
%%   Asterisk      \*     Plus          \+     Comma         \,
%%   Minus         \-     Point         \.     Solidus       \/
%%   Colon         \:     Semicolon     \;     Less than     \<
%%   Equals        \=     Greater than  \>     Question mark \?
%%   Commercial at \@     Left bracket  \[     Backslash     \\
%%   Right bracket \]     Circumflex    \^     Underscore    \_
%%   Grave accent  \`     Left brace    \{     Vertical bar  \|
%%   Right brace   \}     Tilde         \~}
%%
% \iffalse
%
\newcommand{\docdate}{}  % See the renewcommands below,
\newcommand{\filedate}{} % these contain the content;-))
\newcommand{\fileversion}{}
%
%<*driver>
\documentclass[a4paper,ngerman]{ltxdoc}
\usepackage{babel}
\usepackage{fixltx2e}
\usepackage{ae}
\usepackage{mathpazo}
  \renewcommand{\sfdefault}{phvv} % Helvetica scaled by 0.95
  \renewcommand{\ttdefault}{aett} % Courier
\usepackage[colorlinks,hyperindex=false]{hyperref}
\AtBeginDocument{
%</driver>
\renewcommand{\docdate}{2002/07/18}
\renewcommand{\filedate}{2002/07/18}
\renewcommand{\fileversion}{1.1}
%<*driver>
}
\CodelineIndex
\setcounter{tocdepth}{2}
\setlength{\abovecaptionskip}{2mm}
\setlength{\belowcaptionskip}{2mm}
\newcommand{\origttfamily}{}
\let\origttfamily=\ttfamily
\renewcommand{\ttfamily}{\origttfamily \hyphenchar\font=`\-}
\newcommand{\origtexttt}[1]{}
\let\origtexttt=\texttt
\renewcommand{\texttt}[1]{\origtexttt{\hyphenchar\font=`\- #1}}
\hypersetup{
   pdfauthor = {Jochen K\"upper},
   pdfkeywords = {Dissertation LaTeX Heinrich-Heine-Universit\"at},
   pdftitle = {Die jkthesis Dokumentklasse},
   pdfsubject = {LaTeX document class for typesetting a thesis}
}
\renewcommand{\BibTeX}{\texorpdfstring{%
      \textsc{Bib}\kern-.08em\TeX}{BibTeX}}
\newcommand{\AMSLaTeX}{\texorpdfstring{%
      $\mathcal{A}$\kern-.1667em\lower.5ex\hbox{$\mathcal{M}$}%
      \kern-.125em$\mathcal{S}$-\LaTeX}{AMS-LaTeX}}
\begin{document}
\title{Die jkthesis Dokumentklasse\thanks{Diese Datei hat die Version
      \fileversion, letzte \"Anderungen der Makros am \filedate, der 
      Dokumentation am \docdate.}}
\author{Jochen K\"upper \\ \small jochen@jochen-kuepper.de}
\date{\filedate}
\maketitle
\begin{abstract}
   \noindent 
   Eine \LaTeXe-Dokumentklasse f\"ur eine Dissertation an der
   Heinrich-Heine-Universit\"at D\"usseldorf, unter Ber\"ucksichtigung der
   M\"oglichkeit zur elektronischen Ver\"offentlichung der Dissertation.
\end{abstract}
\tableofcontents
\DocInput{jkthesis.dtx}
\PrintIndex
\end{document}
%</driver>
% \fi
%
%
% \section{Warnung}
%
% Dies ist eine Vorabversion dieser Klasse um R\"uckmeldungen und W\"unsche von
% anderen Benutzern zu erhalten. Die Schnittstelle kann jederzeit (und ohne
% Vorwarnung) ge\"andert werden.
%
% Nichtsdestotrotz ist die Klasse voll funktionsf\"ahig. Da sie zur Zeit noch
% "`h\"ochstnichtstandard"' ist, sollte sie ohnehin mit dem Hauptdokument
% zusammen gspeichert werden, so dass in dieser Hinsicht keinerlei Probleme
% zu erwarten sind.
% 
% Ich werde mich bem\"uhen, nach au\ss{}en sichtbare \"Anderungen klein zu
% halten, werde aber zur Zeit noch keinen "`Kompatibilit\"atsmodus"'
% einf\"uhren.
%
%
% \section{\"Anderungen}
%
% \"Anderungen werden in der Datei \texttt{ChangeLog.jkthesis}, die Bestandteil
% der Distribution ist, protokolliert.
%
%
% \section{Einleitung}
%
% Ich habe die \textsf{jkthesis} Dokumentklasse urspr\"unglich zur einfachen
% Anfertigung einer Dissertation mit \LaTeXe{} oder pdf\LaTeX{} an der
% Heinrich-Heine-Universit\"at --- insbesonders in der Math.-Nat.\ Fakult\"at ---
% entwickelt. Sie soll einem einheitlichen und (hoffentlich) guten Layout von
% elektronischen Dissertationen dienen. Da dies aber auch an anderen
% Universit\"aten n\"utzlich sein kann, unterst\"utzt die Klasse beliebige
% Institutionen. Falls es Fakult\"aten mit spezifischen, vom normalen Verhalten
% abweichenden Richtlinien gibt, kann die Klasse entsprechend erweitert werden
% (email an \url{jochen@jochen-kuepper.de}).
%
% \subsection{Weiterf\"uhrende Dokumentation}
%
% Alle verbreiteten \TeX-Distributionen schlie\ss{}en eine grosse Menge
% Dokumentation mit ein, nach der TDS\footnote{\TeX-Directory-Structure;
% ein Vorschlag des \LaTeX3-Teams f\"ur einen einheitlichen Aufbau von
% \TeX-Installationen.} ist diese unterhalb \$TEXMF/doc abgelegt.
% Insbesonders sollten dort die Dokumentation aller in dieser Anleitung
% erw\"ahnten Standardpakete enthalten sein.
%
% Manche Systeme haben zus\"atzliche Erleichterungen zum Auffinden von
% \LaTeX-Dokumentation, z.~B.\ kann man unter tetex die Dokumentation
% zu einem Paket mittels \texttt{texdoc <paketname>} anzeigen lassen.
%
%
% \section{Voraussetzungen}
%
% Die Verwendung von pdflatex-1.0 (oder neuer) wird empfohlen, da seit dieser
% Version \emph{protuding characters} gut unterst�tzt werden, d.\ h.\ Zeichen
% einem niedrigeren Grauwert ragen in den Rand hinein. Dadurch erh�lt man
% keinen geometrisch geraden Rand mehr, sondern eine optisch/visuell gerade
% erscheinenden --- und das nimmt man als Mensch ja nunmal wahr.
%
% Dazu muss man das Paket \texttt{pdfcprot} aus dem CTAN 
% (\url{http://www.ctan.org}) installieren und mittels 
% \begin{verbatim}
%\usepackage[activate]{pdfcprot}
% \end{verbatim}
% die Verwendung aktivieren.
%
% Ebenso kann aber auch jede andere (neuere) \LaTeX-Version mit dieser Klasse
% verwendet werden.
%
% Das verwendete \TeX-System sollte die in Tabelle~\ref{tab:voraussetzungen}
% angegebnenen Voraussetzungen erf\"ullen, was bei vielen aktuellen
% \TeX-Distributionen praktisch erf\"ullt ist. Ansonsten m\"ussen zun\"achst die
% fehlenden Pakete installiert werden. Neuere Pakete als die angegebenen
% Versionen sollten immer funktionieren, ansonsten bitte Fehlerbericht an mich.
% \begin{table}\centering
% \caption{Von \textsf{jkthesis} ben\"otigte Pakete. Es werden weitere
%    Standardpakete ben\"otigt, die in jeder \"ublichen \TeX-Distribution
%    enthalten sind. Neuere als die hier angegebenen Versionen sollten in
%    jedem Fall funktionieren, \"altere unter Umst\"anden nicht.}
% \label{tab:voraussetzungen}
% \begin{tabular}{lll}
%    \hline
%    \LaTeXe        &        & 1999/12/01 \\
%    extsizes       & v1.4a  & 1999/01/07 \\
%    fancyref       & v0.9c  & 1999/02/03 \\
%    fixltx2e       & v1.0b  & 1999/12/01 \\
%    footmisc       & v4.0a  & 2001/06/12 \\
%    hyperref       & v6.71  & \\
%    mathpazo       &        & 2000/06/26 \\
%    nomencl        & v2.9   & 1999/11/23 \\
%    natbib         & v7.0   & 1999/05/28 \\
%    titlesec       & v2.5   & 2002/04/07 \\
%    \hline
% \end{tabular}
% \end{table}
% Es werden noch ein Reihe weiterer Pakete verwendet, bei denen die verwendeten
% Versionen keine besondere Rolle im Zusammenhang mit \textsf{jkthesis} haben.
% Man sollte aber bei Problemen immer die aktuellen Versionen dieser Pakete
% installieren. Eine List der verwendeten Pakete erh\"alt man mit einem
% \texttt{\bslash{}listfiles}-Kommando in der Pr\"aamble.
%
%
% \section{Installation}
%
% Kopieren Sie die Dateien \texttt{jkthesis.dtx} und \texttt{jkthesis.ins}
% in ein tempor\"ares Verzeichnis. F\"uhren Sie nacheinander die folgenden
% Befehle aus:
% \begin{verbatim}
%latex jkthesis.ins
%pdflatex jkthesis.dtx
%makeindex -g -s gind jkthesis
%pdflatex jkthesis.dtx
% \end{verbatim}
% Es sollten jetzt die Dateien \texttt{jkthesis.cls} und
% \texttt{jkthesis.pdf} erzeugt worden sein. Falls Sie kein \textsf{pdflatex}
% installiert haben, k\"onnen Sie auch das "`normale"' \texttt{latex} verwenden;
% sie erhaltne dann anstelle der \texttt{jkthesis.pdf}-Datei das Handbuch
% als \texttt{jkthesis.dvi}.
%
% Die einfachst M\"oglichkeit, diese Klasse zu verwenden ist, sie in das gleiche
% Verzeichnis wie Ihr Dokument zu kopieren.
%
% Um die Klasse global auf Ihrem System zu installieren, sollten Sie die Datei
% \texttt{jkthesis.cls} nach \texttt{\$TEXMF/tex/latex/jkthesis/jkthesis.cls}
% kopieren, sowie das Handbuch nach
% \texttt{\$TEXMF/""doc/""latex/""jkthesis/""jkthesis.pdf} kopieren.
% Im Fall der globalen Installation m\"ussen Sie Ihrem \TeX-Systems die Dateien
% evtl. noch bekanntmachen, indem Sie die \texttt{ls-R}-Datenbank auffrischen;
% bei te\TeX z.~B.\ mittels \texttt{mktexlsr}.
% Falls Sie die \BibTeX-Stildatei \texttt{jkthesis.bst} verwenden, kopieren Sie
% diese entweder auch in Ihr lokales Verzeichnis, oder nach
% \texttt{\$TEXMF/""bibtex/""bibtex/""bst/""jkthesis/""jkthesis.bst}.
%
% Bei Problemen sollte zun\"achst \"uberpr\"uft werden, ob alle in
% Tabelle~\ref{tab:voraussetzungen} angegebenen Voraussetzungen erf\"ullt sind!
%
% \section{Klassenoptionen}
% 
% Die \textsf{jkthesis}-Klasse erwartet die Angabe eines
% Ausgabetreibers f\"ur das \textsf{hyperref}-Paket, z.B.\ \texttt{dvips} oder
% \texttt{pdftex}; bzgl.\ Informationen \"uber weitere Treiber sei auf die
% \textsf{hyperref} Dokumentation verwiesen. Falls keine solche Option
% angegeben wird, wird als Defaultwert \texttt{pdftex} verwendet.
% Zus\"atzlich definiert das Paket die Option \texttt{online}, die farbige
% Querverweise im Dokument erzeugt, w\"ahrend die Standardeinstellung zwar
% Hyperlinks erstellt, diese aber alle schwarz l\"a\ss{}t, da die farbigen Links
% im Ausdruck h\"aufig mit schlechtem Kontrast erscheinen.
%
% Weiterhin wird eine Sprachoption f\"ur das \textsf{babel}"=System erwartet,
% \"ublicher"-weise wird das wohl \texttt{ngerman} oder evtl.\ \texttt{german}
% --- f\"ur die alten Rechtschreiberegeln --- sein. Voreinstellung von
% \textsf{jkthesis} ist \texttt{ngerman}.
%
% Als Papierformate wird zur Zeit nur \texttt{a5paper} unterst\"utzt, sowie
% \texttt{scaled2a4} als hochskalierte \texttt{a5paper} Version. Diese kann als
% Kopiervorlage verwendet werden, um auf DIN-A4 auszudrucken und beim Kopieren
% wieder auf DIN-A5 zu verkleinern.
%
% Die Seitenr�nder werden mittels \textsf{typearea} aus dem
% \textsf{koma-script}-Paket bestimmt, siehe \texttt{scrguide.dvi} f�r weitere
% Einzelheiten hierzu.
%
%
% \section{Schriften}
%
% Mit der \textsf{jkthesis}-Klasse werden als Brotschrift Palatino, eine
% passend skalierte Helvetica als serifenlose Schrift und Courier als
% Schreibmaschinenschrift. Diese Schriften sind bei jeder mir bekannten
% \LaTeXe-Distribution als PostScript Type1-Schriften vorhanden.
% 
% Bitmapschriften sollten nicht zur Erzeugung von PDF-Dokumenten angewandt
% werden, da der Acrobat Reader diese extrem schlecht anzeigt --- obwohl
% Ausdrucke in gewohnter guter \LaTeX-Qualit\"at sind.
% Dies ist besonders bei Verwendung der \texttt{ec}-Schriften zu beachten,
% da diese nur kommerziell als Typ1-Schriften erh\"altlich
% sind.\footnote{\href{http://www.micropress-inc.com}{MicroPress Inc.}}
% 
%
%
% \section{Die Pr\"aamble}
%
% Die \textsf{jkthesis}-Klasse definiert eine Reihe von Befehlen, die die
% Pr\"aamble definieren. Diese sollten alle direkt nach
% \texttt{$\backslash$begin\{document\}} definiert werden und dann mittels
% \DescribeMacro{\maketitle} \texttt{$\backslash$maketitle} gesetzt werden.
%
% 
% \subsection{Die Titelei}
%
% \DescribeMacro{\title}
% \DescribeMacro{\subtitle}
% \DescribeMacro{\author}
% \DescribeMacro{\place}
% \DescribeMacro{\birthplace}
% \DescribeMacro{\date}
% F\"ur die Titelei werden die Angabe des Titels der Arbeit, des Autors, des
% Geburtsorts, des Abgabeorts und des Datums ben\"otigt. Aufgrund dieser Angaben
% werden dann ein Schmutztitel und eine vollst\"andige Titelseite nach den
% Richtlinien der Promotionsordnung der Math.-Nat.\ Fakult\"at erstellt.
%
% Weiterhin werden die PDF-Dokumentinformationen f�r Autor und Titel
% definiert.
%
% \DescribeMacro{\faculty}
% \DescribeMacro{\school}
% \DescribeMacro{\schoolplace}
% Mit den Befehlene \texttt{\bslash{}faculty}, \texttt{\bslash{}school} und
% \texttt{\bslash{}schoolplace} k\"onnen die entsprechenden Angaben f\"ur die
% Titelseite ge\"andert werden, voreingestellt sind die
% Math.-Nat.\ Fakult\"at der Heinrich-Heine-Universit\"at.
%
% 
% \subsection{Der weitere Vorspann}
%
% \DescribeMacro{\copyrightpage}
% Witerhin kann der Autor einen Copyrightvermerk machen, der zus\"atzlich zur
% Druckgenehmigung der Fakult\"at und der Honorierung des \LaTeXe-Teams auf
% Seite vier der Arbeit gesetzt wird. Der untere Standardtext kann mittels
% eines optionale zweiten Arguments ge\"andert werden.
% \DescribeMacro{\dedication}
% \DescribeMacro{\acknowledgement}
% \"Ublicher"-weise wird die Dissertation auch eine Danksagungsseite enthalten,
% evtl.\ auch eine Widmung.
% Nun werden das Inhaltsverzeichnis, die Liste der Abbildungen und die Liste
% der Tabellen gestzt.
%
% \DescribeMacro{\symbols}
% \DescribeMacro{\nomenclature}
% Der Befehl \texttt{\bslash{}symbols} setzt ein Symbol- und
% Abk\"urzungsverzeichnis. 
% Dazu muss mittels
% \begin{verbatim}
%makeindex -g -s symbole.ist -o <filename>.gls \
%          -t <filename>.gll <filename>.glo
% \end{verbatim}
% das Glossary bearbeitet werden.
% Eine Kapitel\"uberschrift wird automatisch gesetzt.
% Symbole k\"onnen mittels 
% \texttt{\bslash{}nomenclature[Sortier\-schl\"ussel]""\{Symbol\}""\{Er\-kl\"arung\}}
% definiert werden.
%
% \DescribeMacro{\preface}
% Manche Arbeiten werden ein Vorwort enthalten, obwohl das f\"ur eine
% Dissertation schon ungew\"ohnlich ist.
%
% \DescribeMacro{\morepreamble}
% Um zus\"atzlich Vorspannseiten zu erzeugen dient der Befehl
% \texttt{\bslash{}morepreamble}, der eine neue Doppelseite beginnt und dann
% ohne weitere explizite Formatierung den Inhalt ausgibt.
%
% Letztlich wird unter Umst\"anden nochmals die Doppelseite beendet und die
% Seitennummerierung f\"ur den Haupttext initialisiert.
%
%
% \section{Tabellen und Abbildungen}
%
% Die \textsf{jkthesis}-Klasse l\"adt bereits die Standard-\LaTeXe{}-Klassen
% \textsf{array} und \textsf{longtable}. Somit k\"onnen die erweiterten
% Spaltenformatierungen des \textsf{array}-Pakets verwendet werden. Ausserdem
% k\"onnen mehrseitige Tabellen gesetzt werden, dazu sei auf die
% \textsf{longtable}-Dokumentation verwiesen.
%
%
% \subsection{\"Uber- bzw.\ Unterschriften}
%
% Bei Tabellen und Abbildungen sollte man beachten das Tabellen eine
% \emph{\"Uber\-schrift}, Abbildungen jedoch eine \emph{Unterschrift} haben;
% d.h.\ dass der \texttt{$\backslash$caption}-Befehl innerhalb einer
% \texttt{table}-Umgebung vor der \texttt{tabular} stehen muss, innerhalb einer
% \texttt{figure}-Umgebung jedoch hinter der
% \texttt{$\backslash$includegraphics}-Anweisung. \texttt{$\backslash$label}
% m\"ussen in jedem Fall hinter der \texttt{$\backslash$caption}-Anweisung
% angeordnet werden --- nicht in der Anweisung.
%
% Alle Tabellen- und Abbildungs\"uberschriften werden mit ein wenig Abstand oben
% und unten gesetzt, sowie auf beiden Seiten 1~cm eingezogen. Ausserdem wird
% eine kleinere Schrift verwendet.
%
% \subsection{Linien in Tabellen}
%
% \DescribeMacro{\toprule}
% \DescribeMacro{\midrule}
% \DescribeMacro{\bottomrule}
% Die Klasse \textsf{jkthesis} stellt f\"ur die horizontalen Linien in Tabellen
% die Befehle \texttt{\bslash{}toprule}, \texttt{\bslash{}midrule} und
% \texttt{\bslash{}bottomrule} zur Verf\"ugung. Diese sollten dann auch
% ausschliesslich zur Erzeugung von Linien in Tabellen verwendet werden.
% \DescribeMacro{\cmidrule}
% Mit dem \texttt{\bslash{}cmidrule}-Befehl k\"onnen horizontale Linien \"uber
% einen Bereich von Spalten gezogen werden, die Syntax ist
% \texttt{\bslash{}cmidrule(<trim>){a-b}}.
% Dieses Kommando zieht eine Linie von Spalte a bis Spalte b, wobei das optionale
% Argument \texttt{<trim>} (in runden Klammern!) die Werte "`l"' "`r"', "`lr"'
% oder "`rl"' annehmen kann und angibt, an welchen Seiten die Linie eingezogen
% werden soll.
% Vertikale Linien und die Verwendung von \texttt{\bslash{}cmidrule} sollten
% wenn m\"oglich vermieden werden, aber letzlich ist eine Spaltengruppierung
% mittels \texttt{\bslash{}cmidrule} vertikalen Linien vorzuziehen.
%
% Auf keinen Fall sollten die Befehle \texttt{\bslash{}hline} bzw.\
% \texttt{\bslash{}cline} verwendet werden, welche zu unvorhersehbaren
% Probleme f\"uhren k\"onnen, wenn Sie mit den hier vorgestellten Befehlen
% gemischt werden; bzgl.\ weitere Informationen siehe die Dokumentation zum
% \textsf{booktabs} Paket.
%
%
% \section{Querverweise}
%
% Wir verwenden hier das Paket \textsf{fancyref} um Querverweise leichter
% handhaben zu k\"onnen.
% \DescribeMacro{\fref}
% \DescribeMacro{\Fref}
% Insbesoders steht das Makro \texttt{\bslash{}fref} zur Verf\"ugung, das
% neben dem Querverweis selber auch die Art des Objektes mit ausgibt;
% au\ss{}erdem wird die zus\"atzlich Zielseite des Verweises ausgegeben.
%
% Weitere Hinweise zur Anwendung sind der \textsf{fancyref}-Anweisung zu
% entnehmen.
%
%
% \section{Mathematik}
%
% \subsection{\protect\AMSLaTeX}
%
% Die \AMSLaTeX{} Pakete \textsf{amsmath} und \textsf{amssymb} werden automatisch
% geladen.
%
% \subsection{Komma}
%
% Das Komma wird im Mathemodus so umdefiniert, dass es als Dezimaltrennzeichen
% korrekt gesetzt wird, falls ihm kein Leerzeichen folgt.
% Folgt hingegen ein Leerzeichen, wird es als Listenseparator erkannt und
% korrekt etwas Abstand eingef\"ugt, wie z.\ B.\ in
% $f(x, y)$.\footnote{Zur Erkl\"arung siehe auch die de-\TeX-FAQ, Frage 8.2.4:
% Warum f\"ugt TeX in einer mathematischen Formel hinter einem Dezimalkomma
% einen kleinen Zwischenraum ein?}
%
% \subsection{Schriften}
%
% Es werden die Pakete \textsf{mathpazo} und f\"ur die kaligrafischen Zeichen
% \textsf{eucal} geladen. Somit stehen zu der Brotschrift Palatino passende
% Mathesymbole zur Verf\"ugung.
%
%
% \section{\TeX-Namen}
%
% \DescribeMacro{\AMSLaTeX}
% \DescribeMacro{\BibTeX}
% Diese Makros werden so definiert, dass sie in jedem Kontext verwendet werden
% k\"onnen.
%
%
% \section{Abk\"urzungen}
%
% \DescribeMacro{\bzgl}
% \DescribeMacro{\ca}
% \DescribeMacro{\evtl}
% \DescribeMacro{\ua}
% \DescribeMacro{\usw}
% \DescribeMacro{\vgl}
% \DescribeMacro{\zB}
% \DescribeMacro{\ZB}
% Es werden die in Tabelle~\ref{tab:abk} aufgelisteten Bequemheitsmakros
% definiert, die vor allem f\"ur den korrekten Freiraum hinter Punkten sorgen.
% Au\ss{}erdem werden bei mehrgliedrigen Abk\"urzungen kleinere Zwischenr\"aume
% verwendet.
% (Duden: Richtlinien f\"ur den Schriftsatz)
% \begin{table}
% \centering
% \caption{Abk\"urzungsmakros, etc.}
% \label{tab:abk}
% \begin{tabular}{ll}
%     \hline
%     \verb+\bzgl+       & bzgl. \\
%     \verb+\ca+         & ca. \\
%     \verb+\evtl+       & evtl. \\
%     \verb+\ua+         & u.\thinspace a. \\
%     \verb+\usw+        & usw. \\
%     \verb+\vgl+        & vgl. \\
%     \verb+\zB+         & z.\thinspace B. \\
%     \verb+\ZB+         & Zum Beispiel \\
%     \hline
% \end{tabular}
% \end{table}
%
%
% \section{Thumbnails}
%
% Thumbnails k�nnen mit dem perl-Skript \texttt{thumbpdf} von Heiko Oberdiek
% erzeugt werden. Dazu muss das Paket thumbpdf eingebunden werden und
% das Dokument mit thumbpdf bearbeitet werden. Bei einem anschlie�enden
% pdf\LaTeX-Lauf werden die erzeugtne Thumbnails eingebunden.
%
%
% \section{Dank}
%
% Eine Reihe von Leuten haben zu dieser Klasse beigetragen.
% Insbesonders m\"ochte ich allen \TeX{}nicians im \textsf{usenet}
% (\url{news:de.comp.text.tex} und \url{news:comp.text.tex}) und dem
% \LaTeXe-Team danken.
% Besonders danken m�chte ich auch Axel K�hl f�r ausgiebige Tests der Klasse.
%
% Ich hoffe ich habe alle individuellen Beitr\"age in der Datei 
% \texttt{ChangeLog.jkthesis} vermerkt; falls ich jemanden zu erw\"ahnen
% vergessen habe, bitte ich mir das mitzuteilen.
%
% \section{Der Programmcode}
%
% Nachdem wir die ben\"otigte \LaTeXe-Version angegeben haben und die
% Dokumentklasse und seine aktuelle Version identifiziert haben,
%    \begin{macrocode}
\NeedsTeXFormat{LaTeX2e}[1999/12/01]
\ProvidesClass{jkthesis}%
[\filedate\space v\fileversion\space HHU thesis class]
%    \end{macrocode}
% werden einige Pakete geladen, die f\"ur die Auswertung der Optionen n\"otig
% sind,
%    \begin{macrocode}
\RequirePackage{ifthen}
%    \end{macrocode}
%
% sowie einige interne Variable, die evtl.\ durch Optionen beeinflu\ss{}t werden,
% deklariert und mit Defaultwerten definiert.
%
% Standardm\"a\ss{}ig wird eine Druckversion erstellt:
%    \begin{macrocode}
\newboolean{onlineversion}
\setboolean{onlineversion}{false}
%    \end{macrocode}
%
% Der pdftex-Treiber kann etwas mehr als die anderen.
%    \begin{macrocode}
\newboolean{pdftexdriver}
\setboolean{pdftexdriver}{false}
%    \end{macrocode}
%
%
% \subsection{Deklaration der Optionen}
%
% Jetzt werden die akzeptierten Optionen definiert:
%
% Die unterst\"utzten \textsf{hyperref}-Treiber.
%    \begin{macrocode}
\DeclareOption{pdftex}{
   \setboolean{pdftexdriver}{true}
   \PassOptionsToPackage{\CurrentOption}{hyperref}}
\DeclareOption{dvips,dvipsone,dviwindo,hypertex,ps2pdf,vtex}{
   \setboolean{pdftexdriver}{false}
   \PassOptionsToPackage{\CurrentOption}{hyperref}}
%    \end{macrocode}
%
% Die Sprachoptionen. \textsf{fancyref} versteht z.\ Z.\ kein \texttt{ngerman},
% also Fragen wir es in diesem Fall nach \texttt{german}.
%    \begin{macrocode}
\DeclareOption{ngerman}{
   \PassOptionsToPackage{\CurrentOption}{babel}
   \PassOptionsToPackage{german}{fancyref}}
\DeclareOption{american,english,french,german}{
   \PassOptionsToPackage{\CurrentOption}{babel,fancyref}}
%    \end{macrocode}
%
% Papierformat
%    \begin{macrocode}
\DeclareOption{a5paper}{
   \PassOptionsToClass{\CurrentOption}{extbook}
   \PassOptionsToPackage{\CurrentOption}{typearea}}
\DeclareOption{scaled2a4}{
   \mag1414
   \setlength{\hoffset}{1in}
   \addtolength{\hoffset}{-1truein}
   \setlength{\voffset}{1in}
   \addtolength{\voffset}{-1truein}
   \PassOptionsToClass{a5paper}{extbook}
   \PassOptionsToPackage{a5paper}{typearea}}
%    \end{macrocode}
%
%
% Ein \texttt{online}-Dokument sollte Hyperlinks farbig darstellen, w\"ahrend
% standardm\"a\ss{}ig schwarze Links verwendet werden, um im Ausdruck einen h\"oheren
% Kontrast zu erhalten:
%    \begin{macrocode}
\DeclareOption{online}{\setboolean{onlineversion}{true}}
%    \end{macrocode}
%
%
% Setze Defaultwerte f\"ur Otionen und werte die \"ubergebenen Optionen aus
%    \begin{macrocode}
\ExecuteOptions{a5paper,ngerman,vario}
\ProcessOptions\relax
%    \end{macrocode}
%
%
% \subsection{Einbindung der Basisklassen}
%
% Lade die \LaTeXe-Klasse \textsf{extbook}, auf der diese Dokumentklasse
% \texttt{jkthesis} basiert.
%    \begin{macrocode}
\LoadClass[9pt,titlepage,twoside]{extbook}
%    \end{macrocode}
%
%
% Den Durchschuss f�r normalen Text erh�hen wir ein wenig, um die Langen Zeilen
% besser lesbar zu machen (11pt -> 11.5pt).
% Die Definition ist ansonsten aus size9.clo (extsizes) kopiert.
%    \begin{macrocode}
\renewcommand\normalsize{%
   \@setfontsize\normalsize\@ixpt{11.5}%
   \abovedisplayskip 8\p@ \@plus4\p@ \@minus4\p@
   \abovedisplayshortskip \z@ \@plus3\p@
   \belowdisplayshortskip 5\p@ \@plus3\p@ \@minus3\p@
   \belowdisplayskip \abovedisplayskip
   \let\@listi\@listI}
\normalsize
%    \end{macrocode}
%
%
% Einige Fixes an der Basis von \LaTeX.
%    \begin{macrocode}
\RequirePackage{fixltx2e}[1999/12/01]
%    \end{macrocode}
%
% \"Unterst\"utzung f\"ur die ausgew\"ahlte Sprache.
%    \begin{macrocode}
\RequirePackage{babel}
%    \end{macrocode}
%
%
% \subsubsection{Schriften}
%
% Hier werden die Schriftkodierung definiert, die zus\"atzlichen Textsymbole
% aus \textsf{textcomp} und die kaligrafischen Eulerschriften geladen.
%
% Das Dokument verwendet Palatino als Brotschrift, eine passend skalierte
% Helvetica als serifenlose Schrift und Courier als Schreibmaschinenschrift.
% F\"ur den Mathemodus werden die virtuellen \textsf{mathpazo}-Schriften
% verwendet, die aus eigenen Zeichen und Computer Modern-Zeichen
% zusammengestzt und in Hinblick auf die verwendung mit Palatino optimiert
% sind. Die kaligrafischen Zeichen werden aus der Euler-Schrift genommen.
%
%    \begin{macrocode}
\RequirePackage{eucal}
\RequirePackage{ae,aecompl}
\RequirePackage{mathpazo}[2000/06/26]
\renewcommand{\sfdefault}{aess}
\renewcommand{\ttdefault}{aett}
\RequirePackage{textcomp}
%    \end{macrocode}
%
% Der Zeichensatz Palatino skaliert anders als die \TeX\ eigenen
% cm-Zeichens\"atze, daher werden hier die Gr\"o\ss{}en der verschiedenen
% Vergr\"o\ss{}erungsstufen ge\"andert. Im folgenden werden die Indexgr\"o\"sen
% weniger verkleinert als es Standard ist.
%    \begin{macrocode}
\DeclareMathSizes{9}{9}{7}{5}
\DeclareMathSizes{10}{10}{7.6}{6}
%    \end{macrocode}
%
%
% \section{Layout-Grundlagen}
%
% Strafpunkte f�r Zeilenumbruch et al.
%    \begin{macrocode}
\frenchspacing
\clubpenalty=2000
\emergencystretch=20pt
\pretolerance=500
\tolerance=2500
\widowpenalty=2000
%    \end{macrocode}
%
% Zeilenabst�nde
%    \begin{macrocode}
%%\lineskip=1pt
\lineskiplimit=-0.3pt
%    \end{macrocode}
%
% Textbereich
% (See scrguide documentation for details.)
%    \begin{macrocode}
\RequirePackage[headinclude,DIV13,BCOR10mm]{typearea}
%    \end{macrocode}
%
% Nummerierungen
%    \begin{macrocode}
\setcounter{secnumdepth}{3}
\setcounter{tocdepth}{1}
%    \end{macrocode}
%
% sonstiges
%    \begin{macrocode}
\emergencystretch=1.5em
\parindent=2em
\parfillskip=2em plus 1fil
%    \end{macrocode}
%
%
% \subsubsection{Zitate}
%
%    \begin{macrocode}
\RequirePackage[comma,numbers,sort&compress,square]{natbib}
\DeclareRobustCommand\cite{%
   \unskip\@ \nolinebreak[3]
   \begingroup\def\NAT@ctype{0}\NAT@partrue\NAT@swatrue
   \@ifstar{\NAT@fulltrue\NAT@cites}{\NAT@fullfalse\NAT@cites}}
\setlength\bibsep{1mm}
%    \end{macrocode}
%
% Jetzt �berreden wir \textsf{natbib} noch, trotz \textsf{hyperref} Referenzen
% zusammenzufassen. Zwischen zwei Referenzen wird immer ein \verb+\thinspace+
% gesetzt.
%
% Weiter unten (hinter \textsf{hyperref}) werden dann noch ein paar weitere
% Definitionen im Zusammenhang mit Zitaten vorgenommen.
%    \begin{macrocode}
\AtBeginDocument{
   \def\NAT@sort{2}
   \def\NAT@space{\thinspace}
   }
%    \end{macrocode}
%
%
% \subsubsection{Querverweise}
%
% \iffalse
% We have to care about conciding \textsf{units.sty} options. We want
% \texttt{loose} (the default) here, but \texttt{tight} \textsf{units}.
% otherwise.
% \fi
%    \begin{macrocode}
\RequirePackage{fancyref}
%    \end{macrocode}
% FMi hat in \textsf{varioref} wohl "`vergessen"�, das es auch im
% deutschen Ausdr\"ucke f\"ur die "`gegen\"uberliegende"� Seite gibt ;-)
% Da dieses Paket, und somit auch das davon abgeleitete
% \textsf{fancyref}, aus Kompatibilt\"atsgr\"unden nicht mehr ge\"andert
% werden sollen, definieren wir hier unsere eigenen Phrasen.
%    \begin{macrocode}
\renewcommand{\reftextfaceafter}{%
   auf der gegen\"uberliegenden Seite}
\renewcommand{\reftextfacebefore}{%
   auf der gegen\"uberliegenden Seite}
\renewcommand{\reftextpagerange}[2]{%
   auf den Seiten \pageref{#1}--\pageref{#2}}
%    \end{macrocode}
%
%
% \subsubsection{Mathe}
%
% Lade \AMSLaTeX. Das weiter unten geladene Paket \textsf{mathpazo} erzeugt
% richtig skalierte Mathezeichen, daher wird hier die Option
% \texttt{psamsfonts} nicht angegeben.
%    \begin{macrocode}
\RequirePackage[centertags,intlimits]{amsmath}
\RequirePackage{amssymb}
%    \end{macrocode}
%
% Gleichungen werden Kapitelweise numeriert.
%    \begin{macrocode}
\renewcommand{\theequation}{\thechapter-\arabic{equation}}
\@addtoreset{equation}{chapter}
%    \end{macrocode}
%
% Hier wird ',' zu einem normalen Zeichen in Dezimalzahlen definiert, um
% die \"ublichen Zeichenabst\"ande in Zahlen zu erhalten; siehe Erl\"auterung
% im Anleitungsteil.
%    \begin{macrocode}
\mathchardef\CommaOrdinary="013B
\mathchardef\CommaPunct   ="613B
\mathcode`,="8000   % , im Mathemodus aktiv ("8000) machen
{\catcode`\,=\active
   \gdef ,{\obeyspaces\futurelet\next\CommaCheck}}
\def\CommaCheck{\if\space\next\CommaPunct\else\CommaOrdinary\fi}
%    \end{macrocode}
%
%
% \subsection{\"Uberschriften}
% 
% Alle \"Uberschriften werden in \texttt{\bslash{}sffamily} gesetzt, die
% Kapitel\"uberschriften noch mit einem ge\"anderten Layout, in dem
% Kapitelnummer und -\"uberschrift durch eine Linie getreennt werden.
% Part-\"Uberschriften werden mit der Nummer in der gleichen Zeile im
% oberen Drittel der Seite gesetzt.
%
% \begin{itemize}
% \item titlesec muss vor fancyhdr stehen.
% \item titlesec-<irgendwas> hat am Ende {\bslash{}makeatother} stehen 
%       -- sollte hier ber�cksichtigt werden.
% \item titlesec's keyval interface funktioniert nur wenn die Definitionen
%       in ein {\bslash{}AtBeginDocument} geschachtelt werden.
% \end{itemize}
%    \begin{macrocode}
\RequirePackage[clearempty]{titlesec}
\AtBeginDocument{
   \titleformat{name=\part}
      [block]
      {\thispagestyle{empty}\centering\scshape\Huge}
      {\Roman{part}}
      {1ex}
      {}
      [\vspace*{0.5\textheight}]
   \titleformat{name=\chapter}
      [display]
      {\vspace*{-15mm}\sffamily\bfseries\huge}
      {\filleft\LARGE\ifthenelse{\equal{\@chapapp}{\appendixname}}
         {\Alph{chapter}}
         {\Roman{chapter}}}
      {0pt}
      {\titlerule\filright}
      [\vspace*{-5mm}]
   \titleformat{name=\chapter,numberless}
      [display]
      {\vspace*{-15mm}\sffamily\bfseries\huge}
      {}
      {0pt}
      {\raggedright}
      [\vspace*{-5mm}]
   \titleformat{name=\section}
      {\itshape\large}
      {\thesection}
      {1em}
      {}
   \titleformat{name=\subsection}
      {\scshape}
      {\thesubsection}
      {0.666em}
      {}
   \titleformat{name=\subsubsection}
      {\itshape}
      {\thesubsubsection}
      {0.4em}
      {}
   \titleformat{name=\paragraph}
      [runin]
      {\scshape}
      {}
      {opt}
      {}
      [ --- ]
   \titleformat{name=\subparagraph}
      [runin]
      {\itshape}
      {}
      {0pt}
      {}
      [ --- ]
}
%   \end{macrocode}
%
%
% \subsection{Seitenlayout}
%
% Kolumnentitel:
%    \begin{macrocode}
\RequirePackage[bottom,flushmargin,multiple,norule,stable]{footmisc}
\RequirePackage{fancyhdr}
\pagestyle{fancy}
%    \end{macrocode}
%
% Umdefinition von plain, wird auf der ersten Seite jedes Kapitels verwendet.
%    \begin{macrocode}
\fancypagestyle{plain}{\fancyhf{}
   \renewcommand{\headrulewidth}{0pt}
   \renewcommand{\footrulewidth}{0pt}
   }
\renewcommand{\chaptermark}[1]{\markboth{\thechapter.\ #1}{}}
\renewcommand{\sectionmark}[1]{\markright{\thesection\ #1}}
\fancyhf{}
\fancyhead[LE,RO]{\thepage}
\fancyhead[RE]{\slshape\leftmark}
\fancyhead[LO]{\slshape\rightmark}
\renewcommand{\headrulewidth}{1pt}
%    \end{macrocode}
%
%
%
% \subsubsection{Verzeichnisse}
%
% Layout von Inhalts-, Abbildungs-, Tabellenverzeichnis. Diese "`L�ngen"' sind
% als Kommandos in latex/base/classes.dtx definiert und beschrieben.
%    \begin{macrocode}
\renewcommand{\@pnumwidth}{2.5em}
\renewcommand{\@tocrmarg}{2em}
\renewcommand{\@dotsep}{5.0}
%    \end{macrocode}
%
%    \begin{macrocode}
\def\@dottedtocline#1#2#3#4#5{%
  \ifnum #1>\c@tocdepth \else
    \vskip \z@ \@plus.2\p@
    {\leftskip #2\relax \rightskip \@tocrmarg \parfillskip -\rightskip
     \parindent #2\relax\@afterindenttrue
     \interlinepenalty\@M
     \leavevmode
     \@tempdima #3\relax
     \advance\leftskip \@tempdima \null\nobreak\hskip -\leftskip
     \normalfont
     {#4}\nobreak
     \leaders\hbox{$\m@th
        \mkern \@dotsep mu\hbox{.}\mkern \@dotsep
        mu$}\hfill
     \nobreak
     \hb@xt@\@pnumwidth{\hfil\normalcolor #5}%
     \normalfont%
     \par}%
  \fi}
%    \end{macrocode}
%
%
% Neudefinition, modifizierte Kopie aus
% latex/base/classes.dtx[1999/09/10 v1.4a].
%    \begin{macrocode}
\renewcommand*\l@chapter[2]{%
  \ifnum \c@tocdepth >\m@ne
    \addpenalty{-\@highpenalty}%
    \vskip 1.0em \@plus\p@
    \setlength\@tempdima{1.5em}%
    \begingroup
      \parindent \z@ \rightskip \@pnumwidth
      \parfillskip -\@pnumwidth
      \leavevmode
      \advance\leftskip\@tempdima
      \hskip -\leftskip
      \normalfont\bfseries
      #1\nobreak
      \leaders\hbox{$\m@th
         \mkern \@dotsep mu\hbox{.}\mkern \@dotsep
         mu$}\hfill
      \nobreak
      \nobreak\hb@xt@\@pnumwidth{\hss #2}\par
      \penalty\@highpenalty
    \endgroup
  \fi}
%    \end{macrocode}
%
% Keine Einr�ckunk der Eintr�ge im lof/lot.
%    \begin{macrocode}
\renewcommand*{\l@figure}{\@dottedtocline{1}{0em}{3em}}
\renewcommand*{\l@table}{\l@figure}
%    \end{macrocode}
%
% Symbol- und Abk\"urzungsverzeichnis
%    \begin{macrocode}
\RequirePackage[noprefix]{nomencl}
%    \end{macrocode}
%
% \subsubsection{Tabellen und Abbildungen}
%
%    \begin{macrocode}
% \subsection{Abbildungen} 
\RequirePackage{array}
\RequirePackage{graphicx}
\RequirePackage[figuresright]{rotating}
\setlength{\extrarowheight}{1pt}
%    \end{macrocode}
%
% Define the thickness of different rules and some little extra separation
% around these rules.
%    \begin{macrocode}
\RequirePackage{booktabs}
\setlength{\heavyrulewidth}{0.3mm}
\setlength{\lightrulewidth}{0.15mm}
\setlength{\cmidrulewidth}{0.1mm}
\setlength{\aboverulesep}{1pt}
\setlength{\belowrulesep}{1pt}
%    \end{macrocode}
%
% Formatierung der Tabellen\"uber und Abbildungsunterschriften
%    \begin{macrocode}
\RequirePackage{subfigure}
\RequirePackage[small]{caption2}
\setlength{\captionmargin}{5mm}
\setlength{\abovecaptionskip}{1mm}
\setlength{\belowcaptionskip}{1mm}
%    \end{macrocode}
%
% Plazierungen von floats (Tabellen und Abbildungen).
%    \begin{macrocode}
\renewcommand{\floatpagefraction}{.6}  % vorher: .5
\renewcommand{\textfraction}{.05}      % vorher: .2
\renewcommand{\topfraction}{1.0}       % vorher: .7
\renewcommand{\bottomfraction}{1.0}    % vorher: .3
\setcounter{topnumber}{1}              % vorher: 2
\setcounter{bottomnumber}{1}           % vorher: 1
\setcounter{totalnumber}{2}            % vorher: 3
%    \end{macrocode}
%
%    \begin{macrocode}
\RequirePackage{dcolumn}
   \newcolumntype{d}{D{.}{,}{-1}}
\RequirePackage[nice,tight]{units}
\RequirePackage{xspace}
%    \end{macrocode}
%
%
%
% \subsection{Hyperlinks}
%
% Letzlich laden wir noch das \textsf{hyperref}-Paket. Es wird erst jetzt
% geladen, da es zahlreiche Befehle \"andert und \"uberschreibt, und so verhindert
% wird das diese \"Anderungen durch andere Pakete wieder aufgehoben werden.
% Falls eine online-Version erstellt wird, werden farbige Links verwendet,
% ansonsten schwarze.
%    \begin{macrocode}
\ifthenelse{\boolean{onlineversion}}{
   \RequirePackage[hyperindex=true,
      naturalnames=true,
      hypertexnames=true,
      plainpages=false,
      breaklinks=true,colorlinks=true,
      citecolor=blue,filecolor=blue,linkcolor=red,
      pagecolor=blue,urlcolor=red]
      {hyperref}}{
   \RequirePackage[hyperindex=true,
      naturalnames=true,
      hypertexnames=true,
      plainpages=false,
      breaklinks=true,colorlinks=true,
      citecolor=black,filecolor=black,linkcolor=black,
      pagecolor=black,urlcolor=black]
      {hyperref}}
%    \end{macrocode}
%
% Und jetzt definienieren wir ein bisserl das natbib/hyperref-Interface um,
% um auch mit hyperref komprimierte Referenzlisten zu erhalten.
%    \begin{macrocode}
\ifthenelse{\boolean{pdftexdriver}}
{\renewcommand\NAT@citexnum{}
\def\NAT@citexnum[#1][#2]#3{%
 \NAT@sort@cites{#3}%
 \let\@citea\@empty
  \@cite{\def\NAT@num{-1}\let\NAT@last@yr\relax\let\NAT@nm\@empty
    \@for\@citeb:=\NAT@cite@list\do
    {\edef\@citeb{\expandafter\@firstofone\@citeb}%
     \if@filesw\immediate\write\@auxout{\string\citation{\@citeb}}\fi
     \@ifundefined{b@\@citeb\@extra@b@citeb}{%
       {\reset@font\bfseries?}
        \NAT@citeundefined\PackageWarning{natbib}%
       {Citation `\@citeb' on page \thepage \space undefined}}%
     {\let\NAT@last@num\NAT@num\let\NAT@last@nm\NAT@nm
      \NAT@parse{\@citeb}%
      \ifNAT@longnames\@ifundefined{bv@\@citeb\@extra@b@citeb}{%
        \let\NAT@name=\NAT@all@names
        \global\@namedef{bv@\@citeb\@extra@b@citeb}{}}{}%
      \fi
      \ifNAT@full\let\NAT@nm\NAT@all@names\else
        \let\NAT@nm\NAT@name\fi
      \ifNAT@swa
       \ifnum\NAT@ctype>1\relax\@citea
        \hyper@natlinkstart{\@citeb\@extra@b@citeb}%
            \ifnum\NAT@ctype=2\relax\NAT@test{\NAT@ctype}%
            \else\NAT@alias
            \fi\hyper@natlinkend\else
       \ifnum\NAT@sort>1
         \begingroup\catcode`\_=8
            \ifcat _\ifnum\z@<0\NAT@num _\else A\fi
              \global\let\NAT@nm=\NAT@num \else \gdef\NAT@nm{-2}\fi
            \ifcat _\ifnum\z@<0\NAT@last@num _\else A\fi
              \global\@tempcnta=\NAT@last@num \global\advance\@tempcnta by\@ne
              \else \global\@tempcnta\m@ne\fi
         \endgroup
         \ifnum\NAT@nm=\@tempcnta
           \ifx\NAT@last@yr\relax
             %\edef\NAT@last@yr{\@citea \mbox{\noexpand\citenumfont\NAT@num}}%
             \edef\NAT@last@yr{\@citea \mbox{%
                   \noexpand\hyper@natlinkstart{\@citeb\@extra@b@citeb}%
                   \noexpand\citenumfont\NAT@num%
                   \hyper@natlinkend%
                   }%
                }%
           \else
             %\edef\NAT@last@yr{--\penalty\@m\mbox{\noexpand\citenumfont\NAT@num}}%
             \edef\NAT@last@yr{--\penalty\@m\mbox{%
                   \noexpand\hyper@natlinkstart{\@citeb\@extra@b@citeb}%
                   \citenumfont\NAT@num\hyper@natlinkend}}%
           \fi
         \else
           \NAT@last@yr \@citea
           %\mbox{\citenumfont\NAT@num}%
           %%%%%%%%%%%%%%%%%%%%%%%%%%%%%%%%%%%%%%%%%%%%%%%%%%%%%%%%%%%%%%%%
           \mbox{\hyper@natlinkstart{\@citeb\@extra@b@citeb}%
            {\citenumfont\NAT@num}\hyper@natlinkend}%
           %%%%%%%%%%%%%%%%%%%%%%%%%%%%%%%%%%%%%%%%%%%%%%%%%%%%%%%%%%%%%%%%
           \let\NAT@last@yr\relax
         \fi
       \else
         \@citea \mbox{\hyper@natlinkstart{\@citeb\@extra@b@citeb}%
                       {\citenumfont\NAT@num}\hyper@natlinkend}%
       \fi
       \fi
       \def\@citea{\NAT@sep\penalty\@m\NAT@space}%
      \else
        \ifcase\NAT@ctype\relax
          \ifx\NAT@last@nm\NAT@nm \NAT@yrsep\penalty\@m\NAT@space\else
          \@citea \NAT@test{1}\ \NAT@@open
          \if*#1*\else#1\ \fi\fi \NAT@mbox{%
          \hyper@natlinkstart{\@citeb\@extra@b@citeb}%
          {\citenumfont\NAT@num}\hyper@natlinkend}%
          \def\@citea{\NAT@@close\NAT@sep\penalty\@m\ }%
        \or\@citea
          \hyper@natlinkstart{\@citeb\@extra@b@citeb}%
           \NAT@test{\NAT@ctype}\hyper@natlinkend
          \def\@citea{\NAT@sep\penalty\@m\ }%
        \or\@citea
          \hyper@natlinkstart{\@citeb\@extra@b@citeb}%
           \NAT@test{\NAT@ctype}\hyper@natlinkend
          \def\@citea{\NAT@sep\penalty\@m\ }%
        \or\@citea
          \hyper@natlinkstart{\@citeb\@extra@b@citeb}%
           \NAT@alias\hyper@natlinkend
          \def\@citea{\NAT@sep\penalty\@m\ }%
        \fi
      \fi
      }}%
      \ifnum\NAT@sort>1\relax\NAT@last@yr\fi
      \ifNAT@swa\else\ifnum\NAT@ctype=0\if*#2*\else
      \NAT@cmt#2\fi \NAT@@close\fi\fi}{#1}{#2}}
}{}
%    \end{macrocode}
%
%
% \subsection{Titelei}
%
% Definition neuer Befehle f\"ur die Titelei
%    \begin{macrocode}
\newcommand{\@subtitle}{}
\newcommand{\subtitle}[1]{
   \renewcommand{\@subtitle}{#1}}

\newcommand{\@birthplace}{}
\newcommand{\birthplace}[1]{
   \renewcommand{\@birthplace}{aus #1}}

\newcommand{\@place}{D\"usseldorf}
\newcommand{\place}[1]{\renewcommand{\@place}{#1}}

\newcommand{\@faculty}{%
   Mathematisch-Natur\-wissen\-schaftlichen Fakult\"at}
\newcommand{\faculty}[1]{\renewcommand{\@faculty}{#1}}

\newcommand{\@school}{Heinrich-Heine-Universit\"at}
\newcommand{\school}[1]{\renewcommand{\@school}{#1}}

\newcommand{\@schoolplace}{\unskip}
\newcommand{\schoolplace}[1]{
   \renewcommand{\@schoolplace}{#1}}

\newcommand{\@hhucopyrightpage}{\noindent 
   Gedruckt mit Genehmigung der
   \@faculty\ der \@school\ \@schoolplace.}
\newcommand{\@jkcopyrightpage}{%
   \noindent Diese Arbeit wurde mit \LaTeXe{} unter Verwendung
   der \textsf{jkthesis} Dokumentklasse gesetzt.}
\newcommand{\@copyrightpage}{
   \@hhucopyrightpage
   \vfill\@jkcopyrightpage
   \cleardoublepage}
\newcommand{\copyrightpage}[3][\@jkcopyrightpage]{%
   \renewcommand{\@copyrightpage}{%
      \noindent #3 \par\vspace*{1em}
      \@hhucopyrightpage \par\vspace*{2em}
      \noindent #2 \vfill \noindent #1
      \cleardoublepage}}
\newcommand{\@dedication}[1]{}
\newcommand{\dedication}[1]{\renewcommand{\@dedication}[1]{%
      \begin{flushright}
         \hspace*{0mm}\vfill
         \textit{\Large #1 \\}
         \vfill\vfill\vfill
      \end{flushright}
      \cleardoublepage}}
\newcommand{\@acknowledgement}[1]{}
\newcommand{\acknowledgement}[1]{\renewcommand{\@acknowledgement}[1]{%
      \hspace*{0mm}\vfill \noindent #1 \cleardoublepage}}
%    \end{macrocode}
%
%    \begin{macrocode}
\newcommand{\@morelists}[1]{}
\newcommand{\morelists}[1]{\renewcommand{\@morelists}[1]{#1}}
%    \end{macrocode}
%
% \subsubsection{Symbolverzeichnis}
%
% Symbole werden mittels \textsf{nomencl} ausgegeben. Die linke Spalte des
% Glossars wird hier auf 12\% der Textbreite gesetzt. Dies kann \"uber eine
% \"Anderung der L\"ange \texttt{\bslash{}nomlabelwidth} angepasst werden.
%
%    \begin{macrocode}
\makeglossary
\renewcommand{\nomname}{Verwendete Symbole und Abk\"urzungen}
\setlength{\nomlabelwidth}{0.12\textwidth}
\newcommand{\@symbols}{}
\newcommand{\symbols}{\renewcommand{\@symbols}{%
      \phantomsection\addcontentsline{toc}{chapter}{\nomname}
      \setlength{\nomitemsep}{0pt}
      \printglossary}}
%    \end{macrocode}
%
%    \begin{macrocode}
\newcommand{\@preface}[1]{}
\newcommand{\preface}[1]{\renewcommand{\@preface}[1]{
      \thispagestyle{empty} \phantomsection
      \chapter*{Vorwort}
      \addcontentsline{toc}{chapter}{Vorwort}
      \noindent #1 }}
\newcommand{\@morepreamble}[1]{}
\newcommand{\morepreamble}[1]{\renewcommand{\@morepreamble}[1]{#1}}
%    \end{macrocode}
% Auslieferung der Titelei. Zun\"achst die Titelseiten:
%    \begin{macrocode}
\renewcommand{\maketitle}{
   \hypersetup{
      pdfauthor  = {\@author},
      pdftitle   = {\@title: \@subtitle},
      }
   \pdfbookmark[-1]{Pr\"aamble}{preamble}
   \pagestyle{empty}\pagenumbering{Roman}
   \begin{center}
      {\large \@title\relax \\[1ex]}
      \@subtitle\relax
      \cleardoublepage
      \pdfbookmark[0]{Titelseite}{title}
      {\fontsize{16}{24}\selectfont \@title\relax \\[1ex]}
      {\fontsize{14}{18}\selectfont \@subtitle\relax \\}
   \end{center}
   \vfill
   {\raggedright
      \textsc{Inaugural-Dissertation} \\[1ex]
      Von \@author\space\@birthplace\space
      an der \@faculty\space der 
      \@school\space\@schoolplace\space vorgelegt.\\[0.5ex]
      \@place, \@date}
   \clearpage
%    \end{macrocode}
%
% Die \textit{Copyright}-Seite
%    \begin{macrocode}
   \@copyrightpage\relax
%    \end{macrocode}
%
% Eine evtl.\ Widmung:
%    \begin{macrocode}
   \@dedication\relax
%    \end{macrocode}
%
% Die Danksagung
%    \begin{macrocode}
   \@acknowledgement\relax
%    \end{macrocode}
%
% Inhaltsverzeichnis, Tabellenverzeichnis und Abbildungsverzeichnis
% \changes{0.4}{2000/08/08}{Add toc-entries for lot and lof by our own.}
%    \begin{macrocode}
   \pagestyle{fancy}
   \phantomsection \thispagestyle{empty}
   \pdfbookmark[0]{\contentsname}{toc}
   \tableofcontents
   \cleardoublepage
   \phantomsection \thispagestyle{empty}
   \addcontentsline{toc}{chapter}{\listtablename}
   \listoftables
   \cleardoublepage
   \phantomsection \thispagestyle{empty}
   \addcontentsline{toc}{chapter}{\listfigurename}
   \listoffigures
   \cleardoublepage
   \@morelists\relax
   \cleardoublepage
%    \end{macrocode}
% Symbole und Abk\"urzungen
%    \begin{macrocode}
   \@symbols\relax
   \cleardoublepage
%    \end{macrocode}
% Vorwort
%    \begin{macrocode}
   \@preface\relax
   \cleardoublepage
%    \end{macrocode}
% Evtl.\ weitere vom Autor definierte Pr\"aamble-Seiten
%    \begin{macrocode}
   \@morepreamble\relax
   \cleardoublepage
%    \end{macrocode}
% Ende der Preamble. Umschalten auf ``normale'' Seitenzahlen.
%    \begin{macrocode}
   \pagenumbering{arabic}
   }
%    \end{macrocode}
%
% Set tableofcontents, listoffigures and listoftables headings sf as well.
% We don�t care on twocolumn mode, as we don�t use it anyway. We also redefine
% the index and bibliography commands to change the headings they generate.
%
% \changes{0.3c}{2000/08/06}{Umdefinition der toc-, lof-, lot-Umgebungen;
%     analog \texttt{\bslash{}thebibliography} und \texttt{\bslash{}theindex}.
%     Insbesonders werden die Header nicht mehr UpperCase gesetzt!
%     Nur einspaltiger Text wird ber\"ucksichtigt, da wir eh kein twocolumn
%     unterst\"utzen.}
% \changes{0.4}{2000/08/08}{bibliography and index get add themselfs to toc.}
% \changes{0.5}{2000/08/10}{Index is set in 8pt font.
%     Add \bslash{}indexpreamble und \bslash{}bibpreamble.}
%    \begin{macrocode}
\AtBeginDocument{
   \renewcommand{\tableofcontents}{%
      \sffamily%
      \chapter*{\contentsname\@mkboth{\contentsname}{\contentsname}}%
      \@starttoc{toc}%
      \normalfont}
   \renewcommand\listoffigures{%
      \chapter*{\listfigurename\@mkboth{\listfigurename}{\listfigurename}}%
      \@starttoc{lof}}
   \renewcommand\listoftables{%
      \chapter*{\listtablename\@mkboth{\listtablename}{\listtablename}}%
      \@starttoc{lot}%
      }
   \newcommand{\bibpreamble}{}
   \renewenvironment{thebibliography}[1]{%
      \chapter*{\bibname}\@mkboth{\bibname}{\bibname}%
      \addcontentsline{toc}{chapter}{\bibname}%
      \bibpreamble%
      \list{\@biblabel{\@arabic\c@enumiv}}{%
         \settowidth\labelwidth{\@biblabel{#1}}%
         \leftmargin\labelwidth \advance\leftmargin\labelsep%
         \@openbib@code \usecounter{enumiv} \let\p@enumiv\@empty%
         \renewcommand\theenumiv{\@arabic\c@enumiv}}%
      \sloppy%
      \clubpenalty4000 \@clubpenalty \clubpenalty \widowpenalty4000%
      \sfcode`\.\@m%
      }{%
      \def\@noitemerr{%
         \@latex@warning{Empty `thebibliography' environment}%
         }%
      \endlist}
   \newcommand{\indexpreamble}{}
   \renewenvironment{theindex}{%
      \columnsep 5\p@%
      \@mkboth{\indexname}{\indexname}
      \twocolumn[\@makeschapterhead{\indexname}%
                 \addcontentsline{toc}{chapter}{\indexname}%
                 \indexpreamble]%
      \thispagestyle{plain}%
      \parindent\z@ \parskip\z@ \@plus .3\p@\relax%
      \let\item\@idxitem%
      \footnotesize%
      }{
      \clearpage
      }
   }
%    \end{macrocode}
%
% Some index-stuff. First we copy the makeidx style here, as it is so small.
% We slightly modify it, though. Esp. the preamble attribute is needed to
% work with hyperrefs hypertexnames=false (needed for subfigures).
% \begin{macro}{\printindex}
%    \begin{macrocode}
\newcommand\printindex{\@input@{\jobname.ind}}
\newcommand*\see[2]{\protect\seename #1}
   \newcommand*{\seename}{$\to$}
\newcommand*{\main}[1]{\textbf{\hyperpage{#1}}}
%    \end{macrocode}
% \end{macro}
%
% Then we change the index formatting.
%    \begin{macrocode}
\renewcommand\@idxitem{\par\vspace*{0.5\p@ \@plus0.3\p@ \@minus0.2\p@}%
   \hangindent 20\p@}
\renewcommand\subitem{\par\nopagebreak%
   \hangindent 30\p@ \hspace*{10\p@}}
\renewcommand\subsubitem{\par\nopagebreak%
   \hangindent 40\p@ \hspace*{20\p@}}
\renewcommand\indexspace{\par \vskip 5\p@ \@plus5\p@ \@minus2\p@\relax}
%    \end{macrocode}
%
% Weil ifthen, bzw. die Redefinition von babel komplett kaputt sind 
% (bugs babel/3224, babel/3239), funktioniert \verb+\pageref+ nicht mehr
% richtig. Bugfix von Heiko Oberdiek:
%    \begin{macrocode}
\AtBeginDocument{%
   \@ifpackageloaded{ifthen}{%
      \long\def\ifthenelse#1#2#3{%
         \let\fix@bbl@tempa\pageref
         \let\pageref\org@pageref
         \@safe@activestrue
         \org@ifthenelse{#1}{%
            \let\pageref\fix@bbl@tempa
            \@safe@activesfalse
            #2%
         }{%
            \let\pageref\fix@bbl@tempa
            \@safe@activesfalse
            #3%
         }%
      }%
   }{}%
}
%    \end{macrocode}
%
%
% \subsection{Chemische Formeln, etc.}
%
% \begin{macro}{\chemsum}
% Korrekter Satz von chemischen Summenformeln, nach Ideen von Kopka und
% \url{news:de.comp.text.tex}.
%    \begin{macrocode}
\DeclareRobustCommand{\chemsum}[1]{%
  {\(\m@th
   \edef\resetfontdimens{\noexpand\)%
       \fontdimen16\textfont2=\the\fontdimen16\textfont2
       \fontdimen17\textfont2=\the\fontdimen17\textfont2\relax}%
   \fontdimen16\textfont2=2.7pt \fontdimen17\textfont2=2.7pt
   \mathsf{#1}%
   \resetfontdimens}}
%    \end{macrocode}
% \end{macro}
%
%
% \subsection{Epigraphen}
%
% \begin{macro}{\epigraph}
% Setzt einen Epigraphen an der angegebenen Stelle. Argumente sind
% der Text und ein evtl.\ Autor.
%    \begin{macrocode}
\newcommand{\epigraph}[2]{\begin{flushright} \textit{#1}\\
      \textsc{#2} \end{flushright} \par\noindent}
%    \end{macrocode}
% \end{macro}
%
%
% \subsection{\TeX-Names}
%
% These are taken from texnames.sty, simplified for the newer LaTeX versions,
% and extended to work in all contexts with pdf\LaTeX\ and \textsf{hyperref}.
%
% \begin{macro}{\BibTeX}
% \BibTeX
%    \begin{macrocode}
\newcommand{\BibTeX}{\texorpdfstring{%
      \textsc{Bib}\kern-.08em\TeX}{BibTeX}}
%    \end{macrocode}
% \end{macro}
%
% \begin{macro}{\AMSLaTeX}
% \AMSLaTeX
%    \begin{macrocode}
\newcommand{\AMSLaTeX}{\texorpdfstring{%
      $\mathcal{A}$\kern-.1667em\lower.5ex\hbox{$\mathcal{M}$}%
      \kern-.125em$\mathcal{S}$-\LaTeX}{AMS-LaTeX}}
%    \end{macrocode}
% \end{macro}
%
%
% \subsection{Abk\"urzungen}
%
% \begin{macro}{\bzw}
% "`beziehungsweise"�
%    \begin{macrocode}
\newcommand{\bzw}{bzw.\@\xspace}
%    \end{macrocode}
% \end{macro}
% 
% \begin{macro}{\bzgl}
% "`bez\"uglich"�
%    \begin{macrocode}
\newcommand{\bzgl}{bez\"uglich\xspace}
%    \end{macrocode}
% \end{macro}
% 
% \begin{macro}{\ca}
% "`circa"�
%    \begin{macrocode}
\newcommand{\ca}{circa\xspace}
%    \end{macrocode}
% \end{macro}
% 
% \begin{macro}{\dah}
% "`das hei\ss{}t"�
%    \begin{macrocode}
\newcommand{\dah}{das heisst\xspace}
%    \end{macrocode}
% \end{macro}
% 
% \begin{macro}{\Dah}
% "`Das hei\ss{}t"�
%    \begin{macrocode}
\newcommand{\Dah}{Das heisst\xspace}
%    \end{macrocode}
% \end{macro}
% 
% \begin{macro}{\evtl}
% "`eventuell"�
%    \begin{macrocode}
\newcommand{\evtl}{eventuell\xspace}
%    \end{macrocode}
% \end{macro}
% 
% \begin{macro}{\ua}
% "`unter anderem"�
%    \begin{macrocode}
\newcommand{\ua}{unter anderem\xspace}
%    \end{macrocode}
% \end{macro}
% 
% \begin{macro}{\usw}
% "`und so weiter"�
%    \begin{macrocode}
\newcommand{\usw}{usw.\@\xspace}
%    \end{macrocode}
% \end{macro}
% 
% \begin{macro}{\vgl}
% "`vergleiche"�
%    \begin{macrocode}
\newcommand{\vgl}{vergleiche\@\xspace}
%    \end{macrocode}
% \end{macro}
% 
% \begin{macro}{\zB}
% "`zum Beispiel"�
%    \begin{macrocode}
\newcommand{\zB}{z.\thinspace{}B.\@\xspace}
%    \end{macrocode}
% \end{macro}
% 
% \begin{macro}{\ZB}
% Da am Satzanfang laut den Duden-Richtlinien zum Schriftsatz "`Abk\"urzungen"'
% ausgesetzt werden, hier also "`Zum Beispiel"�.
%    \begin{macrocode}
\newcommand{\ZB}{Zum Beispiel\xspace}
%    \end{macrocode}
% \end{macro}


% \iffalse
%% Local Variables:
%% mode: LaTeX
%% End:
% \fi
