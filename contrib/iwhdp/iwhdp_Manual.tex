%=====================================================% 
%            IWH DISKUSSIONSPAPIER README             %
%                   Wilfried Ehrenfeld                %
%=====================================================%
\documentclass[german]{iwhdp}
\bibliography{iwhdp_Manual}
\begin{document}
%-----------------------------------------------------%
\title[IWH-Diskussionspapiere - Benutzeranweisung\\für die \LaTeX-Dokumentenklasse \texttt{iwhdp}]
  {IWH-Diskussionspapiere -\\Benutzeranweisung für die \LaTeX-Dokumentenklasse \texttt{iwhdp}}
\titlealt{IWH Discussion Papers - The \texttt{iwhdp} User Guide}
%
\author{Wilfried Ehrenfeld}
\department{Abteilung Strukturwandel und Produktivität}
\tel{+49~345~7753\,832}
\email{wilfried.ehrenfeld@iwh-halle.de}
%
\dpnumber{\TeX{} \iwhdpversion}
%
\dpabstract{Dieses Manual beschreibt die \LaTeX-Vorlage für IWH-Diskussionspapiere.}
\dpabstractalt{This manual outlines the \LaTeX{} template for IWH discussion papers.}
%
\keywords{IWH, \LaTeX, Vorlage, iwhdp}
\keywordsalt{IWH, \LaTeX, template, iwhdp}
%
\jel{Z0}
%-----------------------------------------------------%
\makeiwhtitle
%-----------------------------------------------------%
\section{Benutzung der Vorlage}
%
\subsection{Grundsätzliches}
Für Diskussionspapiere des IWH habe ich die Dokumentenklasse \texttt{iwhdp} geschaffen, 
die bisher jedem Mitarbeiter am IWH frei zur Verfügung steht.
Die aktuelle Version kann von 
\url{http://www.ctan.org/pkg/iwhdp}
% \url{http://mirror.ctan.org/tex-archive/macros/latex/contrib/iwhdp}
% \url{http://www.iwh-halle.de/e/Abteil/STWA/wed/pers.htm} 
bezogen werden. %
%\footnote{Die jeweils letzte "`größere"' Release steht auf dem CTAN zur Verfügung:\\
% \url{http://mirror.ctan.org/tex-archive/macros/latex/contrib/iwhdp}.}
Von der Verwendung alter Versionen wird ausdrücklich abgeraten.

Die Klasse wird mit
\verb+\documentclass[+\emph{optionale Parameter}\verb+]{iwhdp}+
aufgerufen.
Dazu muss die Datei \texttt{iwhdp.cls} im Arbeitsverzeichnis liegen.
Um das Titelblatt im IWH-Stil darzustellen (IWH-Hintergrund), muss dort ebenfalls die Datei \texttt{DP\textunderscore Deckblatt.pdf} verfügbar sein.
Die Dokumentenklasse besitzt optionale Parameter für 
die Sprache des Papiers sowie für den Umgang mit dem Literaturverzeichnis.


\subsection{Optionale Parameter}
%
Optionale Parameter für die \textit{Sprache} des Papiers sind:
%
\begin{itemize}
\item \emph{german} für deutsche Diskussionspapiere
\item \emph{english} für englische Diskussionspapiere
\end{itemize}
Die Standardeinstellung (keine optionalen Parameter) ist \emph{english}. 


Optionale Parameter für die \textit{Literaturverwaltung} sind:
%
\begin{itemize}
\item \emph{harvard} stellt Zitierbefehle zur Verfügung, welche zum \texttt{harvard}-Paket kompatibel sind (\emph{Beta-Version}).
\item \emph{nobib} schaltet die automatische Literaturverwaltung ganz aus.
Statt dessen wird die Hilfs-Umgebung \texttt{iwh\_ref} zur Verfügung gestellt.\\ 
Diese kann wie folgt benutzt werden:

{\small \setstretch{1.05}
\begin{verbatim}
%-----------------------------------------------------%
\narrowspacing % engerer Zeilenabstand
\clearpage % Neue Seite
\makebib % Überschrift Literaturverzeichnis/References
\vspace{-18pt} % Korrektur: Abstand zu Quellen (falls nötig)
%-----------------------------------------------------%
\begin{flushleft} % Linksbündig
\begin{iwh_ref} % Umgebung für Literatur initialisieren
%
\item Demmig, T. (2004): Jetzt lerne ich \LaTeX{}2$_\epsilon$. 
  Pearson Education. Review online at: 
  \url{http://mirror.ctan.org/tex-archive/digests/dtk/book_reviews/dtk_ 
  jetztlerne.pdf}.
%
\end{iwh_ref}
\end{flushleft}
%-----------------------------------------------------%
\end{verbatim}
} % Ende Gruppe
%
\end{itemize}

Als Backend für die Literaturverwaltung verwendet \texttt{iwhdp} das Programm \texttt{biber}%
\footnote{\url{http://www.ctan.org/pkg/biber}}%
%\footnote{\url{http://mirror.ctan.org/pkg/biber}}%
. Dieses muss installiert sein und befindet sich bei z.B. bei Miktex-Installationen im Pfad der Miktex-Binärdateien.

Für die Vorbereitung von Diskussionspapieren zu Konferenz-Einreichungen wurde die \emph{conference} Option implementiert.
% requested by Phillip - 2011-05-19 + 2011-12-06
Wird diese gesetzt, so werden die IWH-typischen Bestandteile des Diskussionspapiers nicht ausgegeben.
Diese sind: Datum und Nummer auf der Titelseite, Disclaimer auf der zweiten Seite, die dritte Seite in Alternativsprache sowie die IWH-spezifischen Kopf- und Fußzeile im Text.

% Neu: 2012-01-27
Die \emph{draft} Option ist für die Konvertierung des Dokumentes zu einer Textdatei oder letztendlich zu Word gedacht, um beispielsweise das Papier zum Korrekturlesen im Word-Format weiterzugeben.
Die Option bewirkt folgendes: 
Keine Grafiken, kein Deckblatt, keine Kopf- und Fußzeilen, keine Seitannummern, Silbentrennung aus.
Die Konvertierung könnte dann beispielsweise mit \texttt{pdftotext} aus dem \texttt{xpdf}-Paket erfolgen.

% Für Ingrid Dede
%Schließlich gibt es die Option \textit{issn} die dafür sorgt, dass auf der zweiten Seite die ISSN-Nummer für IWH-Diskussionspapiere ausgibt. 
%Diese wird für die Druckfassung des Papiers benötigt, nicht aber für die im Internet veröffentlichte pdf-Version.
%Um der Abteilung Öffentlichkeitsarbeit einen Gefallen zu tun wäre es wünschenswert zwei Versionen des Papiers abzugeben: 
%Eine Version mit ISSN, eine ohne.

% Neu: 2014-06-19
Oftmals ist es auch hilfreich, sich bei Diskussionen über den Inhalt eines Papiers zwischen mehreren Autoren 
auf die Zeilennummern der Seiten beziehen zu können. 
Dafür wurde die Option \emph{lineno} implementiert.



\section{Typischer Aufbau eines Dokumentes}
%
Die folgenden Erläuterungen erfolgen anhand der Vorlagedatei \texttt{iwhdp\_paper.tex}.
Die Benutzung dieser Vorlage als Grundlage für eigene Diskussionspapier wird empfohlen.\\ 
Das Dokument beginnt mit der Deklarartion der Klasse:

\verb+\documentclass[german]{iwhdp}+

Dies wird also ein deutsches Diskussionspapier. 
Danach folgt die Angabe der Literaturdatenbank:

\verb+\bibliography{iwhdp_paper.bib}+

Die Literaturdatenbank hat also den Namen \texttt{iwhdp\_paper.bib} und steht im Projektverzeichnis, also im selben Verzeichnis wie \texttt{iwhdp\_paper.tex}.


Das Dokument wird nun mit
\verb+\begin{document}+
initialisiert.


Als Erstes sollte der Titel des Papiers in der Sprache des Papiers (\texttt{title}) und in Alternativsprache (\texttt{titlealt}) angegeben werden:

\verb|\title{Die \LaTeX-Vorlage für IWH-Diskussionspapiere}|\\
\verb|\titlealt{The \LaTeX{} template for IWH discussion papers}|


Für \texttt{title} ist es auch möglich, mittels optionalen Parameter den Titel für die dritte Seite des Papiers gesondert anzugeben um dort etwa individuelle Zeilenumbrüche zu verwenden.

\verb|\title[Die \LaTeX-Vorlage\\für\\IWH-Diskussionspapiere]%|\\
\verb|      {Die \LaTeX-Vorlage für IWH-Diskussionspapiere}|


Nun folgen die bibliographischen Angaben für den/die Autoren:

\verb|\author{Wilfried Ehrenfeld}|\\
\verb|\department{Abteilung Strukturökonomik}|\\
% Bisher
%\verb|\tel{+49\,(0)\,345 7753-832}|\\
% Neu: 2012-01-17 Auf Forderung von Fr. S. Müller:
\verb|\tel{+49~345~7753\,832}|\\
\verb|\email{wilfried.ehrenfeld@iwh-halle.de}|

Das Format der Telefonnummern sollte diesem Schema folgen.


Die Parameter für den (ersten) Koautor heißen analog:\\%
\texttt{cauthor, cdepartment, ctel und cemail}.\\
Für den zweiten Koautor:\\%
\texttt{ccauthor, ccdepartment, cctel und ccemail}.\\
Und schließlich für den dritten Koautor:\\%
\texttt{cccauthor, cccdepartment, ccctel und cccemail}.\\
Insgesamt können also vier Autoren erfasst werden.


Sämtliche Angaben für die Autoren sind bis auf den Namen des ersten Autoren \emph{optional}.
Falls sie nicht benötigt werden, können Sie aus dem Papier-Quelltext entfernt werden.
Alternativ dazu besteht die Möglichkeit, die Angaben für einen Koautor blockweise komplett stillzulegen.
Dazu trägt man für den Namen des Koautors (\texttt{cauthor, ccauthor, cccauthor}) einfach \texttt{NN} ein.


Als nächstes besteht die Möglichkeit den Publikationsmonat des Papiers anzugeben.
Dies ist aber nur dann nötig, falls dieser vom aktuellen Monat abweicht.
Dazu trägt man das gewünschte Datum im Zahlenformat 
\verb|\dpdate{|\emph{Monat}\verb|}{|\emph{Jahr}\verb|}| ein.
\verb|\dpdate{06}{2014}| bezeichnet also den Juni 2014.


Die Nummer des Diskussionspapier wird mit \verb|\dpnumber{|\emph{Nummer}\verb|}| angegeben.
Diese Nummer wird nach erfolgter Zuteilung durch Fr. Dede hier eingetragen.


IWH-Diskussionspapiere werden prinzipiell entweder in Deutsch oder in Englisch geschrieben.
Ein Merkmal der Diskussionspapiere ist die Existenz einer doppelten dritten Seite.
Dies führt dazu, dass Angaben wie Zusammenfassungen, Stichwörter und Titelfußnoten jeweils 
in der Sprache des Papiers (hier: deutsch) angegeben werden müssen und zusätzlich in der
\emph{Alternativsprache} (hier also englisch).


Deshalb folgen nun die Zusammenfassung und der Abstract des Papiers
zuerst in der Sprache des Papiers:
\verb|\dpabstract{|\emph{Text der Zusammenfassung}\verb|}|,
danach in der Alternativsprache:
\verb|\dpabstractalt{|\emph{Text des Abstracts}\verb|}|.
Bei einem deutschen Text ist also \texttt{dpabstract} in Deutsch zu verfassen 
und \texttt{dpabstractalt} in Englisch - und umgekehrt.


Analog hierzu werden die Schlagwörter mit \verb|\keywords{|\emph{Schlagwörter}\verb|}| in der Sprache des Papers angegeben.
Die einzelnen Schlagwörter sind mit je einem Komma zu trennen.
In der Alternativsprache werden diese mit \verb|\keywordsalt{|\emph{Keywords}\verb|}| angegeben.


Die JEL-Klassifikation wird mit \verb|\jel{|\emph{Klasse}\verb|}| eingetragen. 
Mehrere Klassen sollten mit je einem Komma getrennt werden.


Als letzte bibliographische Daten können nun noch die Titelfußnoten angelegt werden.
Anwendungsbeispiele sind Hinweise (Drittmittelförderung) oder Danksagungen.
In der Sprache des Papiers werden diese mit \verb|\thanks{|\emph{Dankeschön!}\verb|}| verfasst,
in der Alternativsprache mit \verb|\thanksalt{|\emph{Thanks!}\verb|}|.


Der IWH-spezifische Vorspann wird nun mit
\verb+\makeiwhtitle+
ausgegeben.
Zum Entwickeln der Papiere kann man diesen Befehl gut mit einem "`\texttt{\%}"' vor diesem Befehl auskommentieren.


Nun folgt der Text des Diskussionspapiers, beispielsweise beginnend mit\\
\verb+\section{Einführung}+.


Das Papier endet mit dem Literaturverzeichnis. 
Dazu kann mit
\verb+\clearpage+
eine neue Seite begonnen werden.
Das Literaturverzeichnis wird mit
\verb+\makebib+
ausgegeben.
Das Dokument endet mit
\verb+\end{document}+.


Der Aufbau eines typischen (hier englischsprachigen) Diskussionspapiers sieht dann ungefähr so aus:

%
%%
%%%
\newpage
{\small \setstretch{1.05}
\begin{verbatim}
%=====================================================% 
%               IWH DISCUSSION PAPER                  %
%       Template: Wilfried Ehrenfeld (2014-06)        %
%=====================================================%
\documentclass[english]{iwhdp}
\addbibresource{iwhdp_paper.bib}
\begin{document}
%-----------------------------------------------------%
\title{The \LaTeX{} template for IWH discussion papers}
\titlealt{Die \LaTeX-Vorlage für IWH-Diskussionspapiere}
%
\author{Wilfried Ehrenfeld}
\department{Department of Structural Change}
\tel{+49~345~7753\,832}
\email{wilfried.ehrenfeld@iwh-halle.de}
%
%\cauthor{}
%\cdepartment{}
%\ctel{}
%\cemail{}
%
%\dpdate{06}{2014}% {Monat}{Jahr} - Nur nötig, wenn abeichend.
%\dpnumber{1}% Nach Zuteilung aktivieren und eintragen.
%
\dpabstract{This is the \LaTeX{} template for IWH discussion papers.}
\dpabstractalt{Dies ist die \LaTeX-Vorlage für IWH-Diskussionspapiere.}
%
\keywords{IWH, \LaTeX, template, iwhdp}
\keywordsalt{IWH, \LaTeX, Vorlage, iwhdp}
%
\jel{Z0}
%
%\thanks{Thanks.}
%\thanksalt{Danke.}
%
%-----------------------------------------------------%
	\makeiwhtitle
%-----------------------------------------------------%
\section{Introduction}
In the beginning God created the heaven and the earth. ...
%-----------------------------------------------------%
\clearpage \makebib
\end{document}
%
% EOF iwhdp_paper.tex
\end{verbatim}
}%Gruppe
\newpage
%%%
%%
%

\section{Zitierstil}
%
Die Vorlage benutzt für die Erstellung von Zitaten das \texttt{biblatex}-Paket%
\footnote{\url{http://www.ctan.org/pkg/biblatex}}%
%\footnote{\url{http://mirror.ctan.org/tex-archive/macros/latex/exptl/biblatex}}%
. Die Zitierbefehle richten sich folglich nach diesem.

Für die beiden häufigsten Zitierweisen wurden Kurzformen geschaffen:\\
\verb|\zit{demmig2004}{13ff.}| $\rightarrow$ \zit{demmig2004}{13ff.}\\
und\\
\verb|\tzit{demmig2004}{13ff.}| $\rightarrow$ \tzit{demmig2004}{13ff.}

Die danach gebräuchlichsten Zitierbefehle sind \texttt{parencite} und \texttt{textcite}, 
bzw. deren Kurzformen \texttt{pcite} und \texttt{tcite}:\\
\verb+\pcite{demmig2004}+ $\rightarrow$ \parencite{demmig2004}\\
\verb+\pcite[13ff.]{demmig2004}+ $\rightarrow$ \parencite[13ff.]{demmig2004}\\
\verb+\pcite[siehe][]{demmig2004}+ $\rightarrow$ \parencite[siehe][]{demmig2004}\\
\verb+\pcite[siehe][13ff.]{demmig2004}+ $\rightarrow$ \parencite[siehe][13ff.]{demmig2004}\\
und\\
\verb+\tcite{demmig2004}+ $\rightarrow$ \textcite{demmig2004}.\\
\verb+\tcite[13ff.]{demmig2004}+ $\rightarrow$ \textcite[13ff.]{demmig2004}.\\
\verb+\tcite[siehe][]{demmig2004}+ $\rightarrow$ \textcite[siehe][]{demmig2004}.\\
\verb+\tcite[siehe][13ff.]{demmig2004}+ $\rightarrow$ \textcite[siehe][13ff.]{demmig2004}.

Daneben existieren eine Reihe weiterer Zitierbefehle:\\
\verb+\cite{demmig2004}+ $\rightarrow$ \cite{demmig2004}\\
\verb+\cite[13ff.]{demmig2004}+ $\rightarrow$ \cite[13ff.]{demmig2004}\\
\verb+\cite[siehe][13ff.]{demmig2004}+ $\rightarrow$ \cite[siehe][13ff.]{demmig2004}

\verb+\cite*{demmig2004}+ $\rightarrow$ \cite*{demmig2004} (gibt nur die Jahreszahl aus)\\
\verb+\cite*[13ff.]{demmig2004}+ $\rightarrow$ \cite*[13ff.]{demmig2004}

\verb+\parencite*[13ff.]{demmig2004}+ $\rightarrow$ \parencite*[13ff.]{demmig2004}

\verb+\parencites[siehe][13ff.]{demmig2004}[sowie][123]{schlager2005}+ $\rightarrow$\\ \parencites[siehe][13ff.]{demmig2004}[sowie][123]{schlager2005}.

Einfacher dürfte es hier jedoch sein zu schreiben:\\
\verb+(siehe \cite[13ff.]{demmig2004} sowie \cite[123]{schlager2005})+ \\
% $\rightarrow$\\ (siehe \cite[13ff.]{demmig2004} sowie \cite[123]{schlager2005})
was das selbe Ergebnis liefert.

\verb+\footcite[siehe][13ff.]{demmig2004}+ $\rightarrow$ \footcite[siehe][13ff.]{demmig2004}

Auch einzelne Felder können angesprochen werden:\\
\verb+\citeauthor{demmig2004}+ $\rightarrow$ \citeauthor{demmig2004}\\
\verb+\citetitle{demmig2004}+ $\rightarrow$ \citetitle{demmig2004}\\
\verb+\citeyear{demmig2004}+ $\rightarrow$ \citeyear{demmig2004}\\
\verb+\citeurl{demmig2004}+ $\rightarrow$ \citeurl{demmig2004}\\
\verb+\citetype{demmig2004}+ $\rightarrow$ Das \texttt{type}-Feld des Eintrages.
 

\section{Tips für die Literaturdatenbank}
%
Die Datenbank sollte im Encoding \emph{ISO8859-15} vorliegen.
Diese Kodierung entspricht in etwa \texttt{latin9} bzw. \texttt{ansinew} unter \LaTeX.
Unter \texttt{JabRef} stellt man dies unter \texttt{Optionen / Einstellungen / Standard-Zeichenkodierung} ein.
Mit der Kodierung \texttt{CP1252} gab es bereits Probleme.
Bei dieser Gelegenheit sollte man auch bei "`Neue Einträge mit Datum versehen. Datumsformat:"'
\texttt{yyyy-MM-dd} eintragen. 
So kann man später diesen "`timestamp"' in andere Datumsfelder (z.\,B. \texttt{urldate}) übernehmen.
Datumsangaben müssen in der sog. ISO-8602-Notierung (JJJJ-MM-TT) vorliegen. 
Ein Beispiel: Das Datum, an dem eine Internetquelle abgerufen wurde (\texttt{urldate}) ist: 
2010-02-16 für den 16. Februar 2010.

Das Eingabeformat für Autoren ist: "`\emph{Vorname Nachname}"'.
Verschiedene Autoren werden mit \texttt{and} getrennt.
Für weitere Autoren z.\,B. "`u.a."' schreibt man \texttt{and others}.
den Namen von Institutionen schreibt man am besten in geschweifte Klammern, 
so wird die automatische Erkennung von Vornamen und Nachnamen ausgeschaltet und der Name nicht getrennt.
Nicht nötig ist es, den ersten Buchstaben von Autoren oder Titeln in geschweifte Klammern zu schreiben.

Artikel, Bücher, und Beiträge aus Sammelbänden werden (wie gewohnt) als 
\texttt{article}, \texttt{book} und \texttt{inbook} eingetragen.
Die Felder \texttt{url} und \texttt{urldate} können verwendet werden.
Der Ort des Verlages heisst "`\texttt{location}"'.

Diskussionspapiere und ähnliche Dokumente sollten als \texttt{report} eingetragen werden.
Die Felder \texttt{author}, \texttt{title} und \texttt{year} sind obligatorisch.
Vermerke wie "`IWH Diskussionspapier 04/2010"' werden als \texttt{type} eingetragen.
Die Felder \texttt{url} und \texttt{urldate} können verwendet werden.

Quellen aus dem Internet, welche keiner anderen Dokumentenklasse zugeordnet werden können, 
werden als \texttt{electronic} eingetragen.
Die Felder \texttt{url} und \texttt{urldate} gelten hierbei als Mindestanforderungen.

\texttt{JabRef} bietet die Möglichkeit, zu jeder Quelle in der Datenbank einen \emph{Review} anzulegen.
Die können Textstellen und Bemerkungen eingetragen werden. 
Später kann die Datenbank mit der Volltextsuche nach Begriffen durchsucht werden.
Den \texttt{abstract} einer Quelle sollte man nicht hier, sondern in das Feld \texttt{Abstract} eingeben bzw. kopieren.


%\section{Hinweis zur "`doppelten zweiten Seite"'}
%%
%Die zweite Seite des IWH-Vorspanns wird zwei Mal ausgegeben - einmal mit ISSN und einmal ohne ISSN.
%Beide Male trägt sie auch die Seitennummer \emph{2}.
%Dies ist eine Vorgabe von Frau Dede, die anschließend hieraus zum einen 
%die Online-Version und zum anderen eine Kopiervorlage erstellt.
%
%Hiermit hängt auch die Warnung\\
%\verb+pdfTeX warning (ext4):+\newline \verb+destination with the same identifier (name{page.2})+\\
%zusammen. Diese kann ignoriert werden.


\section{Einfügen von Grafiken}

Um Abbildungen oder auch Tabellen im pdf- oder jpg-Format in sein Dokument einzubinden, 
eignet sich folgende Herangehensweise:

{\small \setstretch{1.05}
\begin{verbatim}
\begin{figure}[ht]
  \centering
  \fbox{%
    \includegraphics[width = .975\textwidth,%
                     trim = 0 0 0 0,% luro.
                     clip = true%
    ]{actors}
  }% Ende fbox
  \caption[Total number of identified actors in the observed states]%
          {Total number of identified actors in the observed states.\\
           \fns Source: Own illustration.}
  \label{fig:actors}
\end{figure}
\end{verbatim}
} % Ende Gruppe

Die Umgebungsgrenzen \verb+\begin{figure}[ht]+ und \verb+\end{figure}+ klassifizieren die eingefügte Bilddatei als Abbildung.
Es ist auch möglich, hier \verb+\begin{table}+ und \verb+\end{table}+ zu verwenden, falls die eingefügte Datei eine Tabelle darstellt.
Dies ist vor allem dann gut machbar, wenn die Tabelle als pdf vorliegt.
Eine solche Datei kann beispielsweise mit der Speichern-unter Option von Excel erzeugt werden.
Eine in \LaTeX{} gesetzte Tabelle wird sich allerdings harmonischer in das Gesamtdokument einfügen.

Der \textit{width}-Parameter des \texttt{includegraphics}-Befehls gibt die Breite der Grafikdatei an.
Hier wurde dieser relativ verwendet und beträgt 97,5\% der Textbreite.
Da der Höhenparameter (\textit{height}) nicht angegeben wurde, wird das Seitenverhältnis der Grafik beibehalten.

Mittels des \textit{trim}-Parameters kann die Grafik zugeschnitten werden.
Die Werte haben die Einheit Punkt (pt) und sind in der Reihenfolge links unten rechts oben (Merkwort luro) angegeben.

Schließlich steuert die \textit{clip=true}-Option dafür, dass die Grafik auch tatsächlich zugeschnitten wird.
Kommentiert man diese Option aus (oder setzt clip=false), dann bleiben die Teile der Grafik erhalten, 
die mit \textit{trim} entfernt werden sollten.
Dieses Verhalten eignet sich gut, um bei aktivierter \texttt{fbox} die Parameter für \textit{trim} zu ermitteln, da diese dann außerhalb der \texttt{fbox} liegen.

Die Bildunterschrift sowie einen Eintrag in das Abbildungsverzeichnis erzeugt der \texttt{caption}-Befehl.
Der Parameter in geschweiften Klammern liefert die Bildunterschrift.
Der optionale Parameter in eckigen Klammern erzeugt den Eintrag ins Abbildungsverzeichnis.

Der \texttt{label}-Befehl setzt eine Marke, die später für Verweise genutzt werden kann.
In diesem Fall kann mit \verb+\ref{fig:actors}+ die Nummer der Abbildung ausgegeben werden.
Es empfiehlt sich, die labels nach Gruppen zu ordnen - beispielsweise \texttt{sec:} für Abschnitte, \texttt{fig:} für Abbildungen, \texttt{tab:} für Tabellen und \texttt{eq:} für Gleichungen.



\section{Literaturhinweise für \LaTeX -Anfänger}

Wer kein Geld für Bücher ausgeben will, dem seien die einführenden Skripte "`\citetitle{Hagen1}"' und "`\citetitle{Hagen2}"' der FernUni Hagen (\url{http://www.fernuni-hagen.de/zmi/download/}) ans Herz gelegt \parencites{Hagen1}{Hagen2}.

Gute einführende Bücher zum Thema \LaTeX{} sind z.\,B. "`\citetitle{schlager2005}"' \parencite{schlager2005} und "`\citetitle{demmig2004}"' \parencite{demmig2004}.

Weitere (\TeX nische) Informationen über die hier verwendete vorgestellte Dokumentenklasse \texttt{iwhdp} finden sich in
\tcite{Ehrenfeld2010LaTeX} und \tcite{Ehrenfeld2011iwhdp}.


\section{\TeX nische Hinweise - einige Nachfragen}
%
\emph{Wie kann die (einzeilige) Tabellen- bzw. Abbildungsüberschrift zentriert werden?}\\
Dazu dient der Befehl:\\
\verb+\KOMAoption{captions}{nooneline}% oneline: zentriert. nooneline: links+\\
\verb+oneline+ zentriert die einzeilige Überschrift.
\verb+nooneline+ setzt sie wie mehrzeilige - also linksbündig.
Der Parameter sollte vor \verb+\begin{document}+ gesetzt werden.

\emph{Wie können die Absatzzwischenräume geändert werden?}\\
Dazu dient der Befehl:
\verb+\setlength{\parskip}+, beispielsweise\\
\verb+\setlength{\parskip}{8pt plus0pt minus2pt}+.
\verb+parkip+ ist ein sog. elastisches Maß: 
Das \verb+plus+ gibt an, um wieviel der Parameter notfalls gedehnt werden kann -
\verb+minus+ gibt an, um wieviel der Parameter gestaucht werden darf.
Der Parameter sollte nach \verb+\makeiwhtitle+ gesetzt werden.\\
\textbf{!} Es wird ausdrücklich davor gewarnt, diesen Parameter zu verändern \textbf{!}

\clearpage
\makebib

\end{document}

%
% EOF iwhdp_Manual.tex
