\chapter{The introduction}
Modern research on lexical access began in the 1950's (though
\cite{ColeRudnicky1983} note very similar research performed in the 1890's by
William Chandler Bagley).  Several statistical properties of the mental
lexicon have consistently been found to influence how humans process speech.
One of the earliest and most robust findings was that lexical frequency has a
strong influence on lexical access. Repeated research has shown that high
frequency words elicit quicker and more accurate responses than low
frequency words in a large variety of experimental conditions
(e.g.\ \cite{Broadbent1967, Taft1979, BenkiJASA}). Another factor which has
been reliably shown to affect lexical access is neighborhood density.
Neighborhood density is a metric of similarity, roughly defined as the degree
to which a word is similar to others (both phonological and orthographical
measures have been used). Words which have many similar words are said to be
in dense neighborhoods, whereas words which have few similar words are said to
be in sparse neighborhoods. In contrast to lexical frequency, which
facilitates the activation of a word in the brain, neighborhood density has
been found to inhibit activation (e.g.\ \cite{Luce1986, Luce1998, BenkiJASA,
Imai2005}). Of course these are not the only factors which affect language
processing, but they are the most frequently cited, and will be referred to
again in the following sections.
\begin{table*}[!htb]
  \centering
  \caption[Basic Predictions]{Basic Predictions: Predicted results are marked
  with a checkmark, and a relative effect size is also given.}
  \label{T:predictions}
  \begin{tabularx}{\textwidth}{%
    >{\setlength{\hsize}{1.5\hsize}\raggedright\arraybackslash}X%
    *{2}{>{\setlength{\hsize}{.7\hsize}\raggedright\arraybackslash}X}%
    *{2}{>{\setlength{\hsize}{1.05\hsize}\raggedright\arraybackslash}X}}
  \hline\hline
  \rule{0em}{1.1em}& English native listeners& German native listeners & 
       English non-native listeners & German non-native listeners\\[.3em]
  \cline{2-5}
  \rule{0em}{1.1em}lexical status & \checkmark robust& \checkmark
  robust& \checkmark less than native listeners& \checkmark less than native
  listeners\\
  morphology & marginal & more than English & less than L1& less than L1\\
  lexical frequency & \checkmark robust& \checkmark
  robust& \checkmark less than native listeners& \checkmark less than native
  listeners\\
  neighborhood density & \checkmark robust & \checkmark robust & \checkmark less
  than L1& \checkmark less than L1\\[.3em]
  \hline\hline
  \end{tabularx}
\end{table*}
