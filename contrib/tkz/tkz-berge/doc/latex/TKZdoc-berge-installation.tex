%!TEX root = /Users/ego/Boulot/TKZ/tkz-berge/doc-us/TKZdoc-berge-main.tex

\section{Installation}
\subsection{How to install the package \texttt{\textcolor{red}{berge.sty}}}


It is possible that when you read this document, \tkzname{tkz-berge} is present on the \tkzname{CTAN}\footnote{\tkzname{tkz-berge} is not still a part of \tkzname{TeXLive} but it will  be soon possible to install it with \tkzname{tlmgr}} servers.  If \tkzname{tkz-berge} is not still a part of your distribution, this chapter  shows you how to install it. 

\subsection{With TeXLive under OS X and Linux}\NameDist{TeXLive}

You could simply  create a folder (directory) \tikz[remember picture,baseline=(n1.base)]\node [fill=green!20,draw] (n1) {tkz};  which path is : \colorbox{blue!20}{ texmf/tex/latex/tkz}. 

\colorbox{blue!20}{texmf} is generally the personnal folder. For example the paths of this folder on my two computers are

\medskip
\begin{itemize}\setlength{\itemsep}{10pt}
\item   with OS X\NameSys{OS X} \colorbox{blue!20}{\textbf{/Users/ego/Library/texmf}}; 
\item   with Ubuntu\NameSys{Linux Ubuntu} \colorbox{blue!20}{\textbf{/home/ego/texmf}}.
\end{itemize}

If you choose a custom location for  your files, I suppose that you know why!
The installation that I propose, is valid only for one user.
 

\medskip
\begin{enumerate}
\item Store the files \tikz[remember picture,baseline=(n2.base)]\node [fill=green!20,draw] (n2) {tkz-arith.sty, tkz-graph.sty et tkz-berge.sty}; in the folder  \colorbox{green!50}{prof}. Be careful to have the file \tkzname{tkz-tool-arith.tex}. This file is provided by \tkzname{tkz-base}.
\item Open a terminal, then type \colorbox{red!20}{|sudo texhash|}
\item Check that \tkzname{xkeyval}\index{xkeyval} version 2.5 or more, and \tkzname{Ti\emph{k}Z 2.1}\index{TikZ@Ti\emph{k}Z} are installed because they are obligatory.\\

\end{enumerate}

\medskip
My folder texmf is structured as in the diagram below because I use the \tkzname{CVS}\footnote{You can find the cvs version here : \url{http://www.texample.net/tikz/builds/} without CVS\\ or here with CVS \url{http://sourceforge.net/projects/pgf/}} version of \TIKZ. You don't need all the \tkzname{pgf} folders.

\medskip

\vfill
\begin{tikzpicture} [remember picture,rotate=90] 
% nodes
\node (texmf)   at (4,2)   [draw,fill=blue!20 ] {texmf};

\node (tex)     at (6,0)   [draw ] {tex}; 
\node (doc)     at (2,0)   [draw ] {doc};

\node (texgen)  at (7,-2)  [draw ] {generic};
\node (docgen)  at (0,-2)  [draw ] {generic};

\node (latex)   at (4,-2)  [draw ] {latex}; 

\node (genpgf)  at (7,-4)  [draw] {pgf};
\node (latpgf)  at (5,-4)  [draw] {pgf};
\node (tkz)     at (4,-4)  [draw,fill=green!20 ] {tkz};

\node (docpgf)  at (0,-4)  [draw] {pgf};

\node (tkb)     at (6,-6)  [draw,fill=orange!20]  {tkzbase};
\node (tke)     at (2,-6)  [draw,fill=orange!20]  {tkzeuclide};

\node (tari)    at (7,-11)  [draw,fill=orange!20] {tkz-tools-arith.tex};   
\node (tary)    at (5,-11)  [draw,fill=green!20]  {tkz-arith.sty};
\node (tgra)    at (4,-11)  [draw,fill=green!20]  {tkz-berge.sty}; 
\node (tber)    at (3,-11)  [draw,fill=green!20]  {tkz-graph.sty};

% edges
\draw[-open triangle 90](texmf.north east) -- (tex.south west)    ;
\draw[-open triangle 90](texmf.south east) -- (doc.north west)    ;
                                                                  
\draw[-open triangle 90](tex.north east)   -- (texgen.south west) ;
\draw[-open triangle 90](tex.south east)   -- (latex.north west)  ; 
\draw[-open triangle 90](texgen.east)      -- (genpgf.west)       ;  
                                                                  
\draw[-open triangle 90](doc.south east)   -- (docgen.north west) ; 
\draw[-open triangle 90](docgen.east)      -- (docpgf.west)       ; 

\draw[-open triangle 90](latex.north east) -- (latpgf.south west) ; 
\draw[-open triangle 90](latex.east)       -- (tkz.west)          ;    
 
\draw[-open triangle 90,orange!80](tkz.east) to [out=-90,in=90](tkb.west)  ; 
\draw[-open triangle 90,orange!80](tkz.east) to [out=-90,in=90](tke.west)  ; 
\draw[-open triangle 90,orange!80](tkb.east) to [out=-90,in=90](tari.west) ; 
\draw[-open triangle 90,green!80](tkz.east) to [out=-90,in=90](tary.west) ; 
\draw[-open triangle 90,green!80](tkz.east) to [out=-90,in=90](tgra.west) ; 
\draw[-open triangle 90,green!80](tkz.east) to [out=-90,in=90](tber.west) ;  

\end{tikzpicture}

\begin{tikzpicture}[remember picture,overlay]
        \path[->,thin,green!80,>=latex] (n1) edge [bend left] (tkz);
        \path[->,thin,green!80,>=latex] (n2) edge [bend left] (tgra);
\end{tikzpicture}     

\vfill
\newpage

\subsection{How to work with the tkz-\LaTeX-package under Windows?}
\NameDist{MikTeX}\NameSys{Windows XP}
Download and install the following files (if not yet done):
\begin{enumerate}

  \item the \LaTeX-system MiKTeX from

      \url{http://www.miktex.org/}.

      What file you need (e.g.
      \texttt{basic-miktex-2.7.2904.exe}) and how to install
      this program is explained there in the "Download"
      section of the respective version (current version is
      2.7). In general and as usual in windows, you run the
      setup process by starting the setup file :\newline (e.g.\texttt{basic-miktex-2.7.2904.exe}).

  \item Till Tantau's \LaTeX-package \texttt{pgf-tikZ} from

      \url{http://sourceforge.net/projects/pgf/}

      "For MiKTeX, use the update wizard [of MiKTeX] to
      install the (latest versions of the) packages called
      \texttt{pgf}, \texttt{xcolor}, and \texttt{xkeyval}."
      (cited from the pgf manual, contained in the files
      downloaded).
       \item the sty-files and the doc-files of Alain's tkz-package
            from the CTAN servers or

            \url{http://www.altermundus.fr/pages/download.html}.
                
      or
      
      \url{http://altermundus.com/pages/downloads/index.html}.
      
            To add the files to MiKTeX:

            \begin{itemize}
              \item add a directory \texttt{prof} in the
                  directory \colorbox{blue!30}{\texttt{[MiKTeX-dir]/tex/latex}},
                  e.g. in windows explorer,
              \item copy the sty-files in this directory
                  \texttt{tkz},
              \item update the MiKTeX system, ether by running
                  in a DOS shell the command\newline "\colorbox{red!30}{|mktexlsr -u|}" \newline or by clicking\newline
                  "\colorbox{red!30}{|Start/Programs/Miktex/Settings/General|}", then
                  push the button \colorbox{red!30}{|Refresh FNDB|}.
            \end{itemize}
      \end{enumerate}

\subsection{The next version} 

Actually, the package uses \tkzname{xkeyval}, in the next version I will use \tkzname{pgfkeys}. It's possible that the syntax should be modified. My first idea is to keep \tkzname{tkz-graph} and to create a new name for the next version like  \tkzname{tkz-graph-x}.

Some of the main macros used in the file \tkzname{tkz-tool-arith.tex} are now in the CVS version of PGF. With the next version of PGF, it would be possible to remove the   file \tkzname{tkz-tool-arith.tex}.
\endinput
