\newpage\section{Levi Graph}\label{levi}
%<––––––––––––––––––––––––––––––––––––––––––––––––––––––––––––––––––––––––––>
%<––––––––––––––––––––      Levy             –––––––––––––––––––––––––––––––>
%<––––––––––––––––––––––––––––––––––––––––––––––––––––––––––––––––––––––––––>
\begin{NewMacroBox}{grLevi}{\oarg{options}}

\medskip

From Wikipedia \url{http://en.wikipedia.org/wiki/Levi_graph}

\emph{In combinatorics a Levi graph or incidence graph is a bipartite graph associated with an incidence structure. From a collection of points and lines in an incidence geometry or a projective configuration, we form a graph with one vertex per point, one vertex per line, and an edge for every incidence between a point and a line.\hfil\break 
In the mathematical field of graph theory, the Tutte–Coxeter graph or Tutte eight-cage is a 3-regular graph with 30 vertices and 45 edges. As the unique smallest cubic graph of girth 8 it is a cage and a Moore graph. It is bipartite, and can be constructed as the Levi graph of the generalized quadrangle. }

From MathWord : \url{http://mathworld.wolfram.com/LeviGraph.html} 

\emph{It has 30 nodes and 45 edges. It has girth 8, diameter 4, chromatic number 2. The Levi graph is a generalized polygon which is the point/line incidence graph of the generalized quadrangle . The graph  was first discovered by Tutte (1947), and is also called the Tutte-Coxeter graph , Tutte's cage  or "Tutte's (3,8)-cage".The Levi graph is the unique (3,8)-cage graph.\hfil\break
The incidence graph of a generic configuration is sometimes known as a Levi graph (Coxeter 1950).} 

\href{http://mathworld.wolfram.com/topics/GraphTheory.html}%
           {\textcolor{blue}{MathWorld}} by \href{http://en.wikipedia.org/wiki/Eric_W._Weisstein}%
           {\textcolor{blue}{E.Weisstein}}   
           
Some examples of Levi Graphs with this definition are~: 
\begin{itemize}
\item Desargues graph
\item Heawood graph
\item Heawood graph
\item Pappus graph
\item Gray graph 
\item Tutte eight-cage 
\end{itemize}

\end{NewMacroBox}

The two forms can be draw with :

 \begin{tkzexample}[code only]
   \grLevi[RA=7]\end{tkzexample}

and

 \begin{tkzexample}[code only]
    \grLevi[form=2,RA=7,RB=5,RC=3]\end{tkzexample}

You can see on the next pages, the two forms.
\vfill\newpage
Now I show you how to code this graph.

\subsection{\tkzname{Levy graph :form 1}}

\bigskip
\begin{center}
\begin{tkzexample}[vbox]
  \begin{tikzpicture}
    \GraphInit[vstyle=Art]
    \grLCF[prefix=a,RA=6]{-13,-9,7,-7,9,13}{5}
 \end{tikzpicture}
\end{tkzexample} 
\end{center}


\vfill\newpage

\subsection{\tkzname{Levy graph :form 2}}

\bigskip
\begin{center}
\begin{tkzexample}[vbox]
\begin{tikzpicture}
  \GraphInit[vstyle=Art]
  \grCycle[prefix=a,RA=7]{10}
  \EdgeInGraphMod{a}{10}{5}
  \grEmptyCycle[prefix=b,RA=5]{10}
  \grEmptyCycle[prefix=c,RA=3]{10}
  \EdgeInGraphMod{c}{10}{4}
 \end{tikzpicture}
\end{tkzexample} 
\end{center}
\endinput