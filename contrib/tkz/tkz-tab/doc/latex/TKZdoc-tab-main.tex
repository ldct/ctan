
% 25 / 02 /2009 v1.0c TKZdoc-tab-main        11h     
%  $Id: TKZdoc-tab-main.tex   alain matthes $  
%  encoding : utf8 
%  linknodesdoc.tex
%  Created by Alain Matthes  on 2008-01-19.
%  Copyright (C) 2009 Alain Matthes  
%
% This file may be distributed and/or modified
%
% 1. under the LaTeX Project Public License , either version 1.3
% of this license or (at your option) any later version and/or
% 2. under the GNU Public License.
%
% See the file doc/generic/pgf/licenses/LICENSE for more details.%
% See http://www.latex-project.org/lppl.txt for details.
%
%
% ``TKZdoc-tab-main '' is the french doc of tkz-tab
%
%
\documentclass[DIV=14,fontsize=10,headinclude=false,index=totoc,footinclude=false,headings=small]{tkz-doc} 

\usepackage{tkz-tab} 
\usetikzlibrary[petri]
\usepackage[frenchb]{babel}
\usepackage[np,autolanguage]{numprint}

\usepackage[pdftex,unicode,
                colorlinks=true,
                pdfpagelabels, 
                urlcolor=blue,
                filecolor=pdffilecolor,
                linkcolor=blue,
                breaklinks =false,
                hyperfootnotes=false,
                bookmarks=false,
                bookmarksopen=false, 
                linktocpage=true,
                pdfsubject={qcm},
                pdfauthor={Alain Matthes},
                pdftitle={alterqcm},
                pdfkeywords={qcm, mathematics, table},
                pdfcreator={LaTeX}
                ]{hyperref}  
\usepackage{url}
\def\UrlFont{\small\ttfamily}
\usepackage[protrusion = true,
            expansion,
            final,
            verbose = false,
            babel   = true]{microtype} 


\DisableLigatures{encoding = T1, family = tt*}   
\usepackage[parfill]{parskip}    



\gdef\urlauthorcom{http://altermundus.com}  

\gdef\nameofpack{Tkz-Tab}
\gdef\versionofpack{1.1c}
\gdef\dateofpack{2010/02/24}
\gdef\nameofdoc{doc-tkz-tab}
\gdef\dateofdoc{2010/02/24}
\gdef\authorofpack{Alain Matthes}
\gdef\adressofauthor{}
\gdef\namecollection{AlterMundus} 
\gdef\urlauthor{http://altermundus.fr}
\gdef\urlauthor{http://altermundus.com}   

\usepackage{shortvrb}  
\makeatletter
\renewcommand*\l@subsubsection{\bprot@dottedtocline{3}{3.8em}{4em}}  
\makeatother
\AtBeginDocument{\MakeShortVerb{\|}}

\pdfcompresslevel=9
\pdfinfo{
    /Title (doc-tkz-tab.pdf)
    /Creator (TeX)
    /Producer (pdfeTeX)
    /Author (Alain Matthes)
    /CreationDate (13 mars 2009)
    /Subject (Documentation du package tkz-tab v 1.1c)
    /Keywords (pdfeTeX, tab, variations, function, sign, maths, pdflatex) }
    
\colorlet{graphicbackground}{white}
\colorlet{codebackground}{Peach!20}
\newcommand*{\E}{\ensuremath{\mathrm{e}}}
\usepackage{tkzexample}
%<---------------------------------------------------------------------------> 
\begin{document} 


\title{\nameofpack}
\date{\today}
\clearpage
\thispagestyle{empty}
\maketitle
\clearpage

\definecolor{fondpaille}{cmyk}{0,0,0.1,0}
\colorlet{graphicbackground}{fondpaille}
\colorlet{codebackground}{brown!15}
\colorlet{codeonlybackground}{brown!15} 
\colorlet{textcodecolor}{Maroon} 
\pagecolor{fondpaille}
\color{Maroon}    
\tkzTabColors[backgroundcolor=fondpaille,color=Maroon]  

\nameoffile{\nameofpack} 
\defoffile{\textbf{tkz-tab.sty} est un package pour créer à l'aide de \TIKZ des tableaux de signes et de variations le plus simplement possible. Il est dépendant de \TIKZ et fera partie d'une série de packages ayant comme point commun, la création de dessins utiles dans l'enseignement des mathématiques. La lecture de cette documentation va, je l'espère vous permettre d'apprécier la simplicité d'utilisation de \TIKZ et vous permettre de commencer à le pratiquer.}

\presentation  

\vfill
\lefthand\ Je remercie  \tkzimp{Till Tantau} pour nous permettre d'utiliser \tkzname{\TIKZ}.

\lefthand\ Je remercie  \tkzimp{Michel Bovani} pour nous permettre d'utiliser \tkzname{fourier} et \tkzname{utopia} avec \LaTeX.

\lefthand\ Je remercie  \tkzimp{Henri-Claude Dufresne } pour sa lecture approfondie de la documentation et ses propositions de correction.  

henri-claude Dufresne
\lefthand\ Je remercie également  \tkzimp{Jean-Côme Charpentier},  \tkzimp{Manuel Pégourié-Gonnard},  \tkzimp{Franck Pastor}, \tkzimp{Ulrike Fischer} et \tkzimp{Josselin Noirel}  pour les différentes idées et conseils qui m'ont permis de faire ce package, ainsi que \tkzimp{Herbert Vo\ss }  pour son document \tkzname{mathmode.pdf}.



\clearpage
\tableofcontents
%<---------------------------------------------------------------------------> 
\newcommand*{\va}{\colorbox{red!50}    {$\scriptscriptstyle V_a$}}
\newcommand*{\vb}{\colorbox{blue!50}   {$\scriptscriptstyle V_b$}}
\newcommand*{\vbo}{\colorbox{blue!50}  {$\scriptscriptstyle V_{b1}$}}
\newcommand*{\vbt}{\colorbox{yellow!50}{$\scriptscriptstyle V_{b2}$}}
\newcommand*{\vc}{\colorbox{gray!50}   {$\scriptscriptstyle V_c$}} 
\newcommand*{\vd}{\colorbox{magenta!50}{$\scriptscriptstyle V_d$}} 
\newcommand*{\ve}{\colorbox{orange!50} {$\scriptscriptstyle V_e$}} 
%<--------------------------------------------------------------------------->
\tkzTabColors[backgroundcolor=fondpaille,color=Maroon] 

%!TEX root = /Users/ego/Boulot/TKZ/tkz-tab/doc/TKZdoc-tab-main.tex 
% 20 / 02 /2009 v1.00c TKZdoc-tab-install
\section{Installation}
Il est possible que lorsque vous lirez ce document, \tkzname{tkz-tab} soit présent sur le serveur du \tkzname{CTAN}\footnote{\tkzname{tkz-tab} ne fait pas encore partie de \tkzname{TeXLive} mais il sera bientôt possible de l'installer avec \emph{tlmgr}}.  Si   \tkzname{tkz-tab} ne fait pas encore partie de votre distribution, ce chapitre vous montre comment l'installer. 

\subsection{Avec TeXLive sous OS X, Linux et Windows}\NameDist{TeXLive}
Créer un dossier \tikz[remember picture,baseline=(n1.base)]\node [fill=green!50,draw] (n1) {prof};  avec comme chemin : \colorbox{blue!50}{ texmf/tex/latex/prof}.

 \colorbox{blue!50}{texmf} est un dossier personnel, voici les chemins de ce dossier sur mes deux ordinateurs:

\medskip
\begin{itemize}\setlength{\itemsep}{10pt}

\item   sous OS X\NameSys{OS X} \colorbox{blue!30}{\textbf{/Users/ego/Library/texmf}}; 

\item   sous Ubuntu\NameSys{Linux Ubuntu} \colorbox{blue!30}{\textbf{/home/ego/texmf}};

\item sous Windows je ne connais pas cette distribution sous ce système mais je suppose que l'installation doit ressembler à ce qui se passe sous Linux et OS X.
\end{itemize}

\medskip
\begin{enumerate}
\item Placez \tikz[remember picture,baseline=(n2.base)]\node [fill=orange,draw] (n2) {tkz-tab.sty}; dans le dossier \colorbox{green!50}{prof}.
\item Ouvrir un terminal, puis faire \colorbox{red!50}{|sudo texhash|}
\item Vérifier que \tkzname{xkeyval}\index{xkeyval} version 2.5 minimum et \tkzname{Ti\emph{k}Z 2.00}\index{TikZ@Ti\emph{k}Z} sont installés car ils sont obligatoires, pour le bon fonctionnement de tkz-tab.
\end{enumerate}
Mon dossier texmf est structuré ainsi : \emph{Attention, la présence dans mon dossier texmf, des fichiers de \PGF, s'explique par l'utilisation de la version CVS de \PGF}.

\medskip
\begin{tikzpicture} [remember picture,rotate=90] 

\node (texmf)   at (4,2)  [draw,fill=blue!30 ] {texmf};

\node (tex)     at (6,0)   [draw ] {tex}; 
\node (doc)     at (0,0)   [draw ] {doc};

\node (generic) at (7,-4)  [draw ] {generic};
\node (docgen)  at (0,-4)  [draw ] {generic};

\node (latex)   at (4,-4)  [draw ] {latex}; 

\node (pgf1)     at (7,-7)  [draw,fill=orange] {pgf};

\node (pgf2)    at (5,-7)  [draw,fill=orange] {pgf};
\node (prof)    at (4,-7)  [draw,fill=green ] {{prof}};
\node (etc)     at (3,-7)  [draw ] {etc...}; 
\node (dpgf)     at (0,-7)  [draw,fill=orange] {pgf};

\node (qcm)     at (7,-11)  [draw,fill=green ] {alterqcm.sty};
\node (fonc)    at (6,-11)  [draw,fill=orange] {tkz-graph.sty};
\node (esp)     at (5,-11)  [draw,fill=orange] {tkz-berge.sty};
\node (tab)     at (4,-11)  [draw,fill=orange] {tkz-tab.sty};
\node (tuk)     at (3,-11)  [draw,fill=orange] {tkz-tukey.sty};
\node (base)    at (2,-11)  [draw,fill=orange] {tkz-base.sty};
\node (gra)     at (1,-11)  [draw,fill=orange] {tkz-fct.sty};

\draw (doc.west)        |- (4, 1);
\draw (tex.west)        |- (4, 1);

\draw (latex.west)      |- (6,-2);
\draw (generic.west)    |- (6,-2);

\draw (pgf2.west)       |- (4,-6);
\draw (prof.west)       |- (4,-6);
\draw (etc.west)        |- (4,-6);

\draw (qcm.west)        |- (4,-9);
\draw (fonc.west)       |- (6,-9);
\draw (esp.west)        |- (5,-9);
\draw (tuk.west)        |- (4,-9);
\draw (tab.west)        |- (3,-9);
\draw (base.west)       |- (2,-9);
\draw (gra.west)        |- (4,-9);


\draw[-open triangle 90] (pgf1.west)     --  (generic.east);
\draw[-open triangle 90] (4,1)          --  (texmf.east);
\draw[-open triangle 90] (6,-2)         --  (tex.east);
\draw[-open triangle 90] (4,-6)         --  (latex.east);
\draw[-open triangle 90] (4,-9)         --  (prof.east);
\draw[-open triangle 90] (dpgf.west)    --  (docgen.east);
\draw[-open triangle 90] (docgen.west)  --  (doc.east);
\end{tikzpicture}

\begin{tikzpicture}[remember picture,overlay]
        \path[->,thin,red,>=latex] (n1) edge [bend left] (latex);
        \path[->,thin,red,>=latex] (n2) edge [bend left] (prof);
\end{tikzpicture}

\subsection{Avec MikTeX sous Windows XP}\NameDist{MikTeX}\NameSys{Windows XP}

Je ne connais pas grand-chose à ce système mais un utilisateur de mes packages \textbf{Wolfgang Buechel} a eu la gentillesse de me faire parvenir ce qui suit~:

Pour ajouter \tkzname{tkz-tab.sty} à MiKTeX\footnote{Essai réalisé avec la version \tkzname{2.7}}:

\begin{itemize}\setlength{\itemsep}{10pt}
  \item ajouter un dossier \tkzname{prof} dans le dossier
       \colorbox{blue!30}{\texttt{[MiKTeX-dir]/latex/tex}}
  \item copier \tkzname{tkz-tab.sty} dans ce dossier,
  \item mettre à jour  MiKTeX, pour cela dans shell DOS lancer la commande   \colorbox{red!50}{|mktexlsr -u|} 
  
   ou bien encore, choisir \colorbox{red!50}{|Start/Programs/Miktex/Settings/General|}
   
    puis appuyer sur le bouton  \colorbox{red!50}{|Refresh FNDB|}.
\end{itemize}

\vfill
%!TEX root = /Users/ego/Boulot/TKZ/tkz-tab/doc/TKZdoc-tab-main.tex 
% 21 / 02 /2009 v1.00c TKZdoc-tab-init  18 h
\section{Initialisation d'un tableau : \addbs{tkzTabInit} }
\subsection{Définition}

\begin{NewMacroBox}{tkzTabinit}%
{\oarg{local options}\var{e(1)/h(1),...,e(p)/h(p)}\var{a(1),...,a(n)}}

\begin{tabular}{lllc}
\toprule
arguments &  défaut  & définition                 \\ 
\midrule
\TAline{liste1} {no default }  {\var{e(1)/h(1),...,e(p)/h(p)} }
\TAline{liste2} {no default }  {\var{a(1),...,a(n)}}
\bottomrule
\end{tabular}

\medskip
\noindent\emph{Les arguments obligatoires de cette macro sont deux listes dont les éléments sont séparés par des virgules. La première contient $p$ éléments qui définissent $p$ lignes dans le tableau. La seconde liste contient $n$ éléments qui définissent $n$ antécédents. À un antécédent  correspond  une colonne.}

 \begin{itemize}

\item Liste 1 : \emph{les éléments de la première liste sont des paires \tkzname{e(i)/h(i)} où \tkzname{/} est un séparateur entre d'une part, une expression \tkzname{e(1)} et d'autre part, un nombre exprimé en \tkzname{centimètres}. \tkzname{h(i)} est pour tout $i$ un nombre décimal qui fait référence à la hauteur en \tkzname{cm} de la ligne qui contient l'expression \tkzname{e(i)}. Les nombres décimaux utilisent le point comme séparateur.}

\item Liste 2 : \emph{On ne peut pas utiliser les symboles \og \tkzname{/} \fg\ et \og \tkzname{,} \fg\   dans \tkzname{e(i)} sauf si on les protège dans un groupe\protect\footnotemark. La protection de la virgule par une paire d'accolades |\{4,5\}| peut avantageusement être remplacée par une commande comme \tkzcname{numprint\{4,5\}} ou encore \tkzcname{np\{4,5\}}\protect\footnotemark.}
 \end{itemize}
\NamePack{numprint}   

\medskip
\begin{tabular}{lllc}
\toprule
options             & défaut & définition                         \\ 
\midrule
\TOline{espcl}  {2 cm}{espacement entre deux valeurs                  } 
\TOline{lgt}     { 2 cm}{largeur de la première colonne                }
\TOline{deltacl}{0.5 cm}{marge avant le premier et le dernier antécédent}
\TOline{lw}    {0.4 pt}{épaisseur des lignes  du tableau                }
\TOline{nocadre}{false}{par défaut, on encadre le tableau               }
\TOline{color} {false}{booléen autorise la couleur\protect\footnotemark}
\TOline{colorC} {white}{couleur de la première colonne              }
\TOline{colorL} {white}{couleur de la première ligne               }
\TOline{colorT} {white}{couleur de la partie centrale              }
\TOline{colorV} {white}{couleur de la case de la variable          }
\TOline{help}   {false}{affiche les noms des points de construction}  \bottomrule
\end{tabular}

\medskip
\noindent\emph{Le tableau ci-dessus décrit les options actuelles de la macro. Les trois premières sont essentielles pour l'esthétisme de votre tableau, ainsi que pour ses dimensions finales. Il reste cependant une possibilité  car on peut encore jouer avec les options de l'environnement  \tkzname{tikzpicture} qui sont \tkzname{scale}, \tkzname{xscale} et \tkzname{yscale}.}
\end{NewMacroBox}

\footnotetext[3]{expression entre accolades.}
\footnotetext[4]{Voir la documentation du package \tkzname{numprint}.}
\footnotetext[5]{Il est préférable de charger le package \tkzname{xcolor} avec des options comme \tkzname{usenames}, \tkzname{dvipsnames} ou encore \tkzname{pdftex}.}
\NamePack{xcolor}
\subsection{Utilisation des arguments}
\subsubsection{Tableau simple}
\medskip
Exemple : \begin{tkzexample}[code only]\tikz \tkzTabInit{$x$ /.8 , $f(x)$ /.8}{$0$ , $+\infty$}; \end{tkzexample}crée un tableau de \tkzname{deux} lignes. La première ligne fait $\np{0.8}$ cm de hauteur, ainsi que la seconde. La colonne de droite a pour bornes $0$ et $+\infty$.

\medskip
\begin{center}
  \tikz \tkzTabInit{$x$ /.8 , $f(x)$ /.8}{$0$ , $+\infty$};
\end{center}
\subsubsection{Ajout de lignes et de colonnes}
                                                                   
La première liste permet d'obtenir trois lignes qui ont pour hauteur  $1$ cm. La seconde liste comporte trois antécédents qui déterminent deux intervalles (zones). Il sera possible de placer des filets verticaux sous ces antécédents.

\begin{tkzexample}[width=10cm,small]
\begin{tikzpicture}
\tkzTabInit
 {$x$      /1,
  $f(x)$   /1,
  $g(x)$   /1}
 {$0$,$\E$,$+\infty$}
\end{tikzpicture}
\end{tkzexample}
Il est à noter l'utilisation de la macro \tkzcname{E} \footnote{\tkzcname{E} est définie ainsi \BS newcommand*\{\BS E\}\{\BS ensuremath\{\BS mathrm\{e\}\}\}.}
\subsubsection{Tableau minimum}
\index{Tableau minimum}
Le premier  argument est \tkzname{ /1}, c'est l'argument minimum. L'argument est une liste avec comme séparateur  le symbole \tkzname{/}. Celui-ci est  précédé d'un blanc ou d'un vide. La première case de la ligne sera vide. Le \tkzname{$1$} signifie \tkzname{$1$ cm} car une  dimension en cm est obligatoire pour donner la hauteur de la ligne. Le deuxième argument est constitué de deux éléments vides ou bien de deux blancs séparés par une virgule. Cet argument doit contenir  au minimum deux éléments. Ces deux éléments sont les bornes d'un intervalle.

\begin{tkzexample}[width=8cm,small]
\begin{tikzpicture}
 \tkzTabInit{  / 1}
            {  ,  }
\end{tikzpicture}
\end{tkzexample}  

\subsection{Utilisation des options}

Tout d'abord on peut modifier certaines dimensions concernant les colonnes. Voyons les valeurs par défaut.

\begin{center}
  \begin{tikzpicture}
     \tkzTabInit
        {$x$ /  1}
        {$a_1$ ,  $a_2$ , $a_3$}
    \begin{scope}[arstyle/.style={>=latex,#1,<->}] 
      \draw[arstyle=blue] (N10) to node[above,color=blue]%
           {\scriptsize $ espcl = 2$ cm} (N20);
      \draw[arstyle=blue] (N20) to node[above,color=blue]%
           {\scriptsize $ espcl = 2$ cm} (N30);
      \draw[arstyle=red] (T10) to node[above=12pt,color=red]%
           {\scriptsize $ deltacl = 0,5$ cm} (N10);
      \draw[arstyle=red] (N30) to node[above=12pt,color=red]%
           {\scriptsize $ deltacl = 0,5$ cm} (T20);
      \draw[arstyle=blue] (T00) to node[above,color=blue]%
           {\scriptsize $ lgt = 2$ cm} (T10);
    \end{scope}
  \end{tikzpicture}
\end{center}

 
\subsubsection{\texttt{\textcolor{red}{lgt}} : modification de la largeur de la première colonne}\Iopt{tkzTabInit}{lgt}

Par défaut la largeur de cette première colonne est de $2$ cm. L'unité est toujours le \tkzname{cm}.

\begin{tkzexample}[width=8cm,small]
\begin{tikzpicture}
  \tkzTabInit[lgt=3]{ $x$ / 1}
                    { $1$  , $3$ }
\end{tikzpicture}
\end{tkzexample}  

\subsubsection{\texttt{\textcolor{red}{espcl}} : modification de l'espacement entre deux valeurs}\Iopt{tkzTabInit}{espcl}

\begin{tkzexample}[width=9cm,small]
\begin{tikzpicture}
   \tkzTabInit[lgt=3,espcl=4]% 
    { $x$ / 1}
    { $1$ , $4$  }
\end{tikzpicture}
\end{tkzexample}  

\subsubsection{\texttt{\textcolor{red}{deltacl}} : modification des espacements aux extrémités}\Iopt{tkzTabInit}{deltacl}

\begin{tkzexample}[width=9cm,small]
\begin{tikzpicture}
  \tkzTabInit[lgt=3,deltacl=1]% 
  { $x$ / 1}
  { $1$ , $4$ }
\end{tikzpicture}
\end{tkzexample}  

\subsubsection{\texttt{\textcolor{red}{lw}} : épaisseur des lignes du tableau}
\Iopt{tkzTabInit}{lw}
Ce n'est pas recommandé. Il est préférable que tous les traits d'un document aient la même épaisseur qui par défaut est de $\np{0,4}$ pt.

\begin{tkzexample}[width=8cm,small]
\begin{tikzpicture}
  \tkzTabInit[lw=2pt]{ / 1}
                    { , }
\end{tikzpicture}
\end{tkzexample}  

\subsubsection{\texttt{\textcolor{red}{nocadre}} : suppression du cadre externe}
\Iopt{tkzTabInit}{nocadre}

\begin{tkzexample}[width=8cm,small]
\begin{tikzpicture}
  \tkzTabInit[nocadre]{ / 1, /1, /1}
                      { , }
\end{tikzpicture}
\end{tkzexample}  

\subsubsection{\texttt{\textcolor{red}{color}} : utilisation de la couleur dans un tableau}
\Iopt{tkzTabInit}{color}\NamePack{amsmath}
\tkzname{color} est un booléen et indique que l'on veut utiliser la couleur. Pour cela, il faut donner les couleurs attribuées à la première ligne \tkzname{colorL}, la première colonne \tkzname{colorC}, à la case de la variable \tkzname{colorV} et aux lignes \tkzname{colorT}. Il est possible d'attribuer une couleur  pour une ligne particulière.

\tkzname{tkzTabInit\{[color]\}} signifie que le booléen color est à vrai.

\begin{tkzexample}[width=8cm,small]
\begin{tikzpicture}
  \tkzTabInit[color,
              colorT = yellow!20,
              colorC = orange!20,
              colorL = green!20,
              colorV = lightgray!20]
             { /1 , /1}{ , }
\end{tikzpicture}
\end{tkzexample}  

\begin{tkzexample}[width=8cm]
\begin{tikzpicture}
 \tkzTabInit[color,
             colorT = yellow!20,
             colorC = red!20,
             colorL = green!20,
             colorV = lightgray!20,
             lgt    = 1, 
             espcl  = 2.5]%
   {$t$/1,$a$/1,$b$/1,$c$/1,$d$/1}%
   {$\alpha$,$\beta$,$\gamma$}%
\end{tikzpicture}
\end{tkzexample}   

\subsubsection{\texttt{\textcolor{red}{help}} : Affiche la structure du tableau}
\Iopt{tkzTabInit}{help}    
Voir la section \og personnalisation \fg\ (\ref{pers}).
\endinput



%!TEX root = /Users/ego/Boulot/TKZ/tkz-tab/doc/TKZdoc-tab-main.tex 
% 22 / 02 /2009 v1.0c TKZdoc-tab-sign   10h
\section{Création d'une ligne de signes : \tkzcname{tkzTabLine}}
\subsection{Définition}

\begin{NewMacroBox}{tkzTabLine}{\oarg{local options}\var{s(1),...,s(2n-1)}}
  $n$ est le nombre d'éléments du second argument de \tkzname{tkzTabInit}. 

\medskip
\begin{tabular}{lll}
\toprule
symbole de rang impair &  & définition \\ 
\midrule
\TAline{z} {}  {place un trait en pointillés et un zéro centré}
\TAline{t} {}  {place un trait en pointillés centré}
\TAline{d} {}  {place une double barre centrée}
\TAline{\BS textvisiblespace}{} {aucune action }
\bottomrule
\end{tabular}

\medskip
\begin{tabular}{lll}
\toprule
symbole de rang pair & & définition                   \\ 
\midrule
\TAline{h}  {}  {zone interdite}
\TAline{+}  {}  {le signe $+$} 
\TAline{-}  {}  {le signe $-$}
\TAline{\BS textvisiblespace}{} {aucune action }
\bottomrule

\end{tabular}


\medskip
\noindent\emph{\tkzcname{tkzTabLine} accepte  comme argument une liste constituée de symboles. Dans une utilisation \tkzname{normale}, les symboles font partie de   deux catégories; les symboles de rang impair et les symboles de rang pair. Cette distinction est due au fait que les symboles de rang impair sont en général des traits (filets) et ceux pour les places de rang pair sont en général des signes \og $+$ ou $-$ \fg. Les symboles de rang impair agissent graphiquement, et permettent de tracer des filets verticaux. L'argument de \tkzcname{tkzTabLine} en contient \tkzname{$n$} si on suppose que le deuxième argument de  \tkzcname{tkzTabInit} possède \textcolor{red}{$n$} éléments (antécédents). Les symboles de rang pair  permettent d'obtenir un signe \og $+$ ou $-$ \fg\ ou bien une zone interdite (hachurée ou colorée). Chaque ligne de signes en contient \tkzname{$n-1$} et contiendra donc un total de \tkzname{$2n-1$} éléments, c'est à dire \tkzname{$2n-2$} virgules !\\
Les différents symboles "reconnus" sont donnés dans le tableau ci-dessus,  mais vous devez savoir que l'on peut mettre pratiquement n'importe quoi. Cependant attention! la virgule \textcolor{red}{(,)} est le séparateur de liste aussi vous devez prendre des précautions pour introduire un nombre à virgule. Vous avez plusieurs possibilités~:
\begin{itemize}
\item[--] \{4,5\} on place le nombre entre des accolades.
\item[--] \tkzcname{numprint\{4,5\}} ou encore  \tkzcname{np\{4,5\}}, ce qui nécessite de charger l'excellent package \tkzname{numprint}\NamePack{numprint} avec l'option \tkzname{np} pour le raccourci.
\end{itemize}}
 
\medskip
\begin{tabular}{lllc}
\toprule
options   & défaut       & définition         \\ 
\midrule
\TOline{style}  {dotted    } {style des traits verticaux                }
\TOline{help}   {no default} {affiche la structure d'une ligne de signes}
\bottomrule
\end{tabular}

\medskip 
\noindent\emph{Il est possible de changer localement le style des filets verticaux et il est possible d'avoir des renseignements sur la structure de la ligne.}
\end{NewMacroBox}

\subsection{Nombre d'arguments utilisés.}
La syntaxe générale est :

\begin{tkzltxexample}[]
 \tkzTabInit{ e(1),...,e(i),...,e(p)} % tableau à p lignes.
            { a(1),...,a(i),...,a(n)} % n antécédents
 \tkzTabLine{ s(1),...,s(i),...,s(2n-1)}
\end{tkzltxexample}


\medskip
Si on utilise \tkzname{$n$} antécédents pour la première ligne alors il y aura \tkzname{$n$} symboles de rang impair et \tkzname{$n-1$} symboles de rang pair, soit \tkzname{$2n-1$} symboles. 

 Les principaux symboles  utilisés sont : \tkzname{z} pour un zéro placé sur un trait, \tkzname{t} pour un trait correspondant  à un zéro d'une autre ligne, \tkzname{d} pour une valeur pour laquelle l'expression n'est pas définie.
 
Voyons une illustration  simple : trois antécédents $a_1$, $a_2$, et $a_3$ permettront de mettre $2\times3 -1 =5$ symboles. Les $3$ valeurs de la première ligne impliquent pour l'argument de   \tkzcname{tkzTabLine} de posséder {$2\times 3-1=5$} éléments  c'est-à-dire être une liste comportant   $3$ symboles de rang impair et $2$ symboles de rang pair, soit un total de $5$ symboles qui seront séparés par $4$ virgules.

\begin{center}
  \begin{tikzpicture}
    \tkzTabInit[espcl=3,lgt=2]%
    {\colorbox{red!40}{\textcolor{white}{$x$}}    / 1,%
     \colorbox{red!40}{\textcolor{white}{$f(x)$}} / 1}%
    {\colorbox{blue!40}{\textcolor{white}{ $a_1$}},%
     \colorbox{blue!40}{\textcolor{white}{ $a_2$}},%
     \colorbox{blue!40}{\textcolor{white}{ $a_3$}}}%
    \path (N11)--(N12) node[circle,draw, fill= lightgray,midway] {1};
    \path (N12)--(N21) node[midway] {$,$};
    \path (N21)--(N22) node[circle, draw, fill= lightgray,midway] {3};
    \path (N31)--(N32) node[circle,draw,  fill= lightgray,midway] {5};
    \path (M11)--(M12) node[circle, draw, fill= orange!30,midway] {2};
    \path (M21)--(M22) node[circle,draw,  fill= orange!30,midway] {4};
  \end{tikzpicture} 
\end{center}

Pour obtenir cette ligne, il faut entrer
\begin{tkzexample}[code only]\tkzTabLine{ $1$ , $2$ , $3$ , $4$ , $5$}
\end{tkzexample}     
  




\subsection{Emploi minimum}
La deuxième ligne est vide mais l'argument \tkzcname{tkzTabLine} doit comporter \tkzname{$4$} virgules. C'est en effet une liste comportant \tkzname{$5 = 2\times3-1$} valeurs.

\begin{tkzexample}[code only]
 \tkzTabLine{,,,,} ou \tkzTabLine{ , , , , }\end{tkzexample}
 
\begin{tkzexample}[width=8cm,small]
\begin{tikzpicture}
  \tkzTabInit[espcl=1.5]
     {$x$  / 1 ,$f(x)$ /1 }%
     {$v_1$ , $v_2$ , $v_3$ }%
  \tkzTabLine{ , , , , }
\end{tikzpicture}
\end{tkzexample}  

\subsubsection{\texttt{\textcolor{red}{t}} : ajout d'un trait} 
Cette option place un simple trait verticalement.
\begin{tkzexample}[width=8cm,small]
\begin{tikzpicture}
  \tkzTabInit[espcl=1.5]
     {$x$  / 1 ,$f(x)$ /1 }%
     {$v_1$ , $v_2$ , $v_3$ }%
  \tkzTabLine{ t, , t , ,t }
\end{tikzpicture}
\end{tkzexample}  

\subsubsection{\texttt{\textcolor{red}{z}} : ajout d'un zéro sur un trait vertical}
\begin{tkzexample}[width=8cm,small]
\begin{tikzpicture}
  \tkzTabInit[espcl=1.5]
     {$x$  / 1 ,$f(x)$ /1 }%
     {$v_1$ , $v_2$ , $v_3$ }%
  \tkzTabLine{ z, , z , ,z }
\end{tikzpicture}
\end{tkzexample}  

\subsubsection{\texttt{\textcolor{red}{d}} : double barre}

On peut aussi avoir le cas d'une fonction  non définie en $0$ et en $2$ mais s'annulant en $1$. On place à chaque extrémité le symbole |d|.

\begin{tkzexample}[width=7cm,small]
\begin{tikzpicture}
  \tkzTabInit[espcl=1.5]%
    {$x$ / 1,$g(x)$ / 1}%
    {$0$,$1$,$2$}%
\tkzTabLine{d,+,0,-,d}
\end{tikzpicture}
\end{tkzexample}  
  
On peut aussi avoir le cas d'une fonction admettant une dérivée à droite différente de la dérivée à gauche

\begin{tkzexample}[width=7cm,small]
\begin{tikzpicture}
   \tkzTabInit[lgt=1.5,espcl=1.75]%
       {$x$ / 1,$f'(x)$ / 1}%
       {$-\infty$,$0$,$+\infty$}%
   \tkzTabLine{,+,d,-,}
\end{tikzpicture}
\end{tkzexample}  

\subsection{Utilisation des symboles de rang pair}  

Pour un tableau de signe, en principe les symboles de rang pair mais il est possible de détourner l'emploi de base de cette macro. L'exemple suivant montre un cas classique d'une zone du tableau qui correspond à des valeurs interdites. par défaut avec le symbole \tkzname{h}, la zone est grisée mais on peut hachurer cette zone si on préfère.
Le dernier exemple montre comment détourner l'usage principal.

\subsubsection{\texttt{\textcolor{red}{h}} : zone interdite}

Une fonction peut ne pas être définie sur un intervalle, ici $[1~;~2]$. La partie du tableau qui correspond à cet intervalle sera hachurée ou bien colorée (par défaut, la zone est grisée). Des options permettant de personnaliser seront offertes. Pour l'exemple suivant, il suffit de placer |h| entre les deux |d| qui correspondent aux valeurs interdites\index{valeurs interdites} $1$ et  $2$.

\begin{tkzexample}[width=8cm, small]
\begin{tikzpicture}
  \tkzTabInit[color,espcl=1.5]
  {$x$ / 1,$g(x)$ / 1}
  {$0$,$1$,$2$,$3$}%
  \tkzTabLine{z, + , d , h , d , - , t}
\end{tikzpicture}
\end{tkzexample}  

\subsection{Utilisation des options} 

\subsubsection{\tkzname{t style} : modification du style des traits verticaux}
\Istyle{tkzTabLine}{t style}

\begin{tkzexample}[width=8cm, small]
\begin{tikzpicture} 
\tikzset{t style/.style = {style = dashed}} 
  \tkzTabInit[espcl=1.5]
     {$x$  / 1 ,$f(x)$ /1 }%
     {$v_1$ , $v_2$ , $v_3$ }%
  \tkzTabLine{ t, , t , ,t }
\end{tikzpicture}
\end{tkzexample}  


\begin{tkzexample}[width=8cm, small]
    \tikzset{t style/.style = {style = densely dashed}}   
\begin{tikzpicture}
  \tkzTabInit[espcl=1.5]
     {$x$  / 1 ,$f(x)$ /1 }%
     {$v_1$ , $v_2$ , $v_3$ }%
  \tkzTabLine{ z, , z , ,z }
\end{tikzpicture}
\end{tkzexample}  

\subsubsection{\tkzname{help} : Affiche la structure du tableau} 
\Iopt{tkzTabLine}{help}   
Voir la section \og personnalisation \fg\ (\ref{pers}).


\subsection{Utilisation des styles}

\subsubsection{\tkzname{h style} : modification de la couleur d'une zone interdite}
\Istyle{tkzTabLine}{h style} \index{zone interdite}
Si vous préférez hachurer une zone du tableau, alors  il faut modifier un style. 
\begin{tkzexample}[small]
\begin{tikzpicture}
      \tikzset{h style/.style = {fill=red!50}}
      \tkzTabInit[color,espcl=1.5]%
        {$x$ / 1,$g(x)$ / 1}%
        {$0$,$1$,$2$,$3$}%
      \tkzTabLine{z,+,d,h,d,-,t}
\end{tikzpicture}
\end{tkzexample}


Cette fois la zone est hachurée.
 \index{hachures}
\begin{tkzexample}[small]
\begin{tikzpicture}
  \tikzset{h style/.style =
          {pattern=north west lines}}
  \tkzTabInit[color,espcl=1.5]%
    {$x$ / 1,$g(x)$ / 1}%
    {$0$,$1$,$2$,$3$}%
  \tkzTabLine{z,+,,h,d,-,t}
\end{tikzpicture}
\end{tkzexample}


 
\subsection{Exemples}

\subsubsection{Simplification d'une expression comportant une valeur absolue }

\begin{tkzexample}[width=7cm,small]
 \begin{tikzpicture}
 \tkzTabInit[lgt=2,espcl=1.75]%
  {$x$/1,$2-x$/1, $\vert 2-x \vert $/1}%
  {$-\infty$,$2$,$+\infty$}%
 \tkzTabLine{ , + , z , - , }
 \tkzTabLine{ , 2-x ,z, x-2, }
\end{tikzpicture}
\end{tkzexample}    

\subsubsection{Tableau de signes}

\begin{tkzexample}[ small]
\begin{tikzpicture}
  \tkzTabInit[lgt=3,espcl=1.5]%
     {$x$                                   /1,
      $x^2-3x+2$                            /1,
      $(x-\E)\ln x$                   /1,
      $\dfrac{x^2-3x+2}{(x-\E)\ln x}$ /2}
              {$0$  , $1$  , $2$  , $\E$  ,$+\infty$}
              \tkzTabLine{ t,+,z,-,z,+,t,+,}
              \tkzTabLine{ d,+,z,-,t,-,z,+,}
              \tkzTabLine{ d,+,d,+,z,-,d,+,}
\end{tikzpicture}
\end{tkzexample}

  
\subsubsection{Signe d'une expression du second degré}

Si $\Delta  \geq 0$ on peut écrire  $\displaystyle ax^2+bx+c=a\left(x-\dfrac{-b-\sqrt{b^2-4ac}}{2a}\right)\left(x-\dfrac{-b+\sqrt{b^2-4ac}}{2a}\right)$


\begin{tkzexample}[vbox,small]
\begin{tikzpicture}
 \tkzTabInit[color,lgt=5,espcl=3]%
   {$x$ / .8,$\Delta>0$\\ Le signe de\\ $ax^2+bx+c$ /1.5}%
   {$-\infty$,$x_1$,$x_2$,$+\infty$}%
 \tkzTabLine{ , \genfrac{}{}{0pt}{0}{\text{signe de}}{a}, z
              , \genfrac{}{}{0pt}{0}{\text{signe}}{\text{opposé de}\ a}, z
              , \genfrac{}{}{0pt}{0}{\text{signe de}}{a},  }
 \end{tikzpicture}
\end{tkzexample}  
 Il faut noter l'emploi de la macro \tkzcname{genfrac}\footnote{\tkzcname{genfrac} est une macro du package \tkzname{amsmath}}.

\medskip
Si $\Delta  = 0$ alors on peut écrire   $\displaystyle ax^2+bx+c=a\left(x+\dfrac{b}{2a}\right)^2$

\begin{tkzexample}[vbox,width=7cm,small]
\begin{tikzpicture}
  \tkzTabInit[color,lgt=5,espcl=3]%
   {$x$ / 1 , $\Delta=0$\\ Le signe de\\ $ax^2+bx+c$  / 2}%
   {$-\infty$,$\dfrac{-b}{2a}$,$+\infty$}%
  \tkzTabLine{ , \genfrac{}{}{0pt}{0}{\text{signe de}}{ a} , z
               , \genfrac{}{}{0pt}{0}{\text{signe de}}{a}, }
\end{tikzpicture}
\end{tkzexample}  

Si $\Delta  < 0$  alors  $\displaystyle ax^2+bx+c=a\left[\left(x+\dfrac{b}{2a}\right)^2 -\dfrac{b^2-4ac}{4a^2}\right]$

\begin{tkzexample}[vbox,width=7cm,small]
\begin{tikzpicture}
  \tkzTabInit[color,lgt=5,espcl=5]%
   {$x$/.8,$\Delta<0$\\ Le signe de\\ $ax^2+bx+c$/2}%
   {$-\infty$,$+\infty$}%
 \tkzTabLine{ , \genfrac{}{}{0pt}{0}{\text{signe de}}{ a}, }
\end{tikzpicture}
\end{tkzexample}


\endinput
%!TEX root = /Users/ego/Boulot/TKZ/tkz-tab/doc/TKZdoc-tab-main.tex 
% 22 / 02 /2009 v1.0c TKZdoc-tab-variation      10h
\section{Création d'une ligne de variations : \addbs{tkzTabVar}}
\subsection{Définition}

\begin{NewMacroBox}{tkzTabVar}{\oarg{local options}\var{el($1$),\dots,el($n$)}}
  avec \tkzname{el($i$) = s($i$) / e($i$)} ou bien     \tkzname{el($i$) = s($i$) / eg($i$) / ed($i$)}.
  
\noindent\tkzname{s($i$)} est une série de symboles à choisir dans le tableau ci-dessous. \tkzname{eg($i$)} et \tkzname{ed($i$)} sont des expressions mathématiques qui se placent à gauche et à droite des filets verticaux. \tkzname{e($i$)} est une expression centrée sur un filet.

\medskip  
 \begin{tabular}{llrl}
\toprule 
 Groupe 1& \emph{avec un seul signe} &&\\
\midrule 
\tkzname{s($i$)}  & Position des expressions & \tkzname{el($i$)}  &      \\
\midrule                                                                                   
\IargName{tkzTabVar}{$-~~~$}& expression  unique et centrée en bas  eg=ed   &  $-   $&$/e    $    \\
\IargName{tkzTabVar}{$+~~~$}& expression  unique et centrée en haut eg=ed   &  $+   $&$/e    $    \\
\IargName{tkzTabVar}{$~R~~$}& rien, on passe à l'expression  suivante           &  $~R  $&$(/)   $    \\
\IargName{tkzTabVar}{$-C$}  & prolongement par continuité en bas, centrée &  $-C  $&$/e    $    \\
\IargName{tkzTabVar}{$+C$}  & prolongement par continuité en haut, centrée&  $+C  $&$/e    $    \\
\IargName{tkzTabVar}{$-H$}  & expression en bas et centrée puis zone interdite  &  $-H  $&$/e    $    \\
\IargName{tkzTabVar}{$+H$}  & expression en haut et centrée puis zone interdite &  $+H  $&$/e    $    \\
\IargName{tkzTabVar}{$+D~~$}& discontinuité, expression en haut à gauche        &  $+D  $&$/e    $    \\
\IargName{tkzTabVar}{$-D~~$}& discontinuité, expression en bas à gauche         &  $-D  $&$/e    $    \\
\IargName{tkzTabVar}{$~D+~$}& discontinuité, expression en haut et à droite     &  $D+  $&$/e    $    \\
\IargName{tkzTabVar}{$~D-~$}& discontinuité, expression en bas et droite        &  $D-  $&$/e    $    \\
\IargName{tkzTabVar}{$+DH$} & discontinuité à gauche et en haut puis zone interdite&  $+DH $&$/e    $  \\
\IargName{tkzTabVar}{$-DH$} & discontinuité à gauche et en bas puis zone interdite &  $-DH $&$/e    $    \\
\IargName{tkzTabVar}{$+CH$} & prolongement par continuité puis zone interdite    &  $+CH $&$/e    $    \\
\IargName{tkzTabVar}{$-CH$} & idem mais  expression en bas et à gauche           &  $-CH $&$/e    $    \\
\midrule
 Groupe 2& \emph{avec deux signes}& &\\
\midrule 
\IargName{tkzTabVar}{$+D-$} & discontinuité,\hfill deux expressions    &  $+D- $&$/eg/ed$    \\
\IargName{tkzTabVar}{$-D+$} & discontinuité, \dots \hfill      qui sont    &  $-D+ $&$/eg/ed$    \\
\IargName{tkzTabVar}{$+D+$} & discontinuité, \dots \hfill soit  à gauche ,soit à droite &$+D+ $&$/eg/ed$    \\
\IargName{tkzTabVar}{$-D-$} & discontinuité, \dots \hfill soit en haut, soit en bas&$-D- $&$/eg/ed$    \\
\IargName{tkzTabVar}{$+CD+$}& prolongement par continuité à gauche et          &  $+CD+$&$/eg/ed$    \\
\IargName{tkzTabVar}{$-CD-$}& \dots \hfill  deux expressions qui sont        &  $-CD-$&$/eg/ed$    \\
\IargName{tkzTabVar}{$+CD-$}& \dots \hfill  soit  à gauche ,soit à droite  &  $+CD-$&$/eg/ed$    \\
\IargName{tkzTabVar}{$-CD+$}& \dots \hfill   soit en haut, soit en bas   &  $-CD+$&$/eg/ed$    \\
\IargName{tkzTabVar}{$+DC+$}& prolongement par continuité à droite et           &  $+DC+$&$/eg/ed$    \\
\IargName{tkzTabVar}{$-DC-$}& \dots \hfill  deux expressions qui sont          &  $-DC-$&$/eg/ed$    \\
\IargName{tkzTabVar}{$+DC-$}& \dots \hfill  soit  à gauche ,soit à droite      &  $+DC-$&$/eg/ed$    \\
\IargName{tkzTabVar}{$-DC+$}& \dots \hfill   soit en haut, soit en bas         &  $-DC+$&$/eg/ed$    \\
\IargName{tkzTabVar}{$+V+$} & comme une discontinuité mais sans double barre  et&  $+V+ $&$/eg/ed$    \\
\IargName{tkzTabVar}{$-V-$} & \dots \hfill  deux expressions qui sont     &  $-V- $&$/eg/ed$    \\
\IargName{tkzTabVar}{$+V-$} & \dots \hfill  soit  à gauche ,soit à droite &  $+V- $&$/eg/ed$    \\
\IargName{tkzTabVar}{$-V+$} & \dots \hfill   soit en haut, soit en bas    &  $-V+ $&$/eg/ed$    \\
\midrule
\tkzname{\textvisiblespace} & laisse la place vide dans certains cas& &   \\ 
\bottomrule
\end{tabular}

\medskip
\noindent\emph{La macro \tkzcname{tkzTabVar} nécessite un argument qui est une liste. Cette liste contient \tkzname{$n$} éléments correspondant aux \tkzname{$n$} antécédents de la première ligne. Chaque élément donne la position d'une ou de deux expressions par rapport à la ligne avec un signe $+$ (en haut) ou bien un signe $-$ (en  bas). Ces expressions sont, soit des images, soit des limites.}       

\noindent\emph{Les éléments \tkzname{el($i$)} ont pour forme~:\\
 soit \mbox{\tkzname{\{ s($i$)/ e($i$)\}}} ou bien \mbox{\tkzname{\{ s($i$)/ e($i$) / \}}}, soit \mbox{\tkzname{\{ s($i$)/ eg($i$) / ed($i$)\}}}.}
 
\noindent\emph{La première forme correspond aux symboles qui ne possèdent qu'un signe \tkzname{$+$} ou \tkzname{$-$} et qui  placent une seule expression; la seconde correspond aux symboles qui  possèdent deux signes et qui placent deux expressions. Les expressions sont des valeurs prises à gauche \tkzname{ eg($i$)\}} ou bien à droite \tkzname{ed($i$)} par la fonction ou encore des limites mais les expressions peuvent être vides. Un signe \tkzname{$+$} ou  \tkzname{$-$} à gauche (resp. à droite) des symboles correspond à \tkzname{eg($i$)} (resp. à \tkzname{ed($i$)}).}

\medskip
\begin{tabular}{lllc}
\toprule
\texttt{options}               & \texttt{défaut} & \texttt{définition}              \\
\midrule
\IoptName{tkzTabVar}{color}  & |black|         & couleur des flèches                \\
\IoptName{tkzTabVar}{help}   &  affiche la structure d'une ligne de variations      \\ 
 \bottomrule
\end{tabular} 

\end{NewMacroBox}  
  
 \medskip
Un schéma étant parfois plus simple qu'un long discours \dots

 \begin{center}
\begin{tikzpicture}
\tkzTabInit[lgt=2,espcl=3]{$x$/1,$f'(x)$/1,$f(x)$/3}%
{$0$,$1$,$2$,$+\infty$}%
\tkzTabLine{t,-,d,-,z,+,}%
\tkzTabVar{}% 
\node[below =3pt]  (FN12) at (N12){};  
\node[below =3pt]  (FN22) at (N22){}; 
\node[below =3pt]  (FN32) at (N32){};
\node[below =3pt]  (FN42) at (N42){};

\node[above =3pt]  (FN13) at (N13){};
\node[above =3pt]  (FN23) at (N23){};
\node[above =3pt]  (FN33) at (N33){};
\node[above =3pt]  (FN43) at (N43){};

\node[below right=3pt]  (FRN12) at (N12){};
\node[below right=3pt]  (FRN22) at (N22){}; 
\node[below right=3pt]  (FRN32) at (N32){};
\node[below right=3pt]  (FRN42) at (N42){};

\node[above right=3pt]  (FRN13) at (N13){};
\node[above right=3pt]  (FRN23) at (N23){};
\node[above right=3pt]  (FRN33) at (N33){};

\node[below left=3pt]  (FLN22) at (N22){}; 
\node[below left=3pt]  (FLN32) at (N32){};
\node[below left=3pt]  (FLN42) at (N42){};

\node[above left=3pt]  (FLN13) at (N13){};
\node[above left=3pt]  (FLN23) at (N23){};
\node[above left=3pt]  (FLN33) at (N33){};
\node[above left=3pt]  (FLN43) at (N43){};

\draw[fill=red!50] (FN12) circle(2pt)  node[below=2pt,text=red!50] {\small e} node[left=1cm,red] {$+$};
\draw[fill=red!50] (FN22) circle(2pt);
\draw[fill=red!50] (FN32) circle(2pt);
\draw[fill=red!50] (FN42) circle(2pt);

\draw[fill=red!50] (FN13) circle(2pt)  node[left=1cm,red] {$-$};
\draw[fill=red!50] (FN23) circle(2pt);
\draw[fill=red!50] (FN33) circle(2pt) node[above=2pt,text=red!50] {\small e};
\draw[fill=red!50] (FN43) circle(2pt);

\draw[fill=blue!30] (FRN12) circle(2pt);
\draw[fill=blue!30] (FRN22) circle(2pt) node[below right=2pt,text=blue!30] {\small ed};
\draw[fill=blue!30] (FRN32) circle(2pt);
\draw[fill=blue!30] (FRN13) circle(2pt);
\draw[fill=blue!30] (FRN23) circle(2pt);
\draw[fill=blue!30] (FRN33) circle(2pt);

\draw[fill=green!50] (FLN22) circle(2pt);
\draw[fill=green!50] (FLN32) circle(2pt);
\draw[fill=green!50] (FLN42) circle(2pt) node[below left =2pt,text=green!50] {\small eg};
\draw[fill=green!50] (FLN23) circle(2pt) node[above left =2pt,text=green!50] {\small eg};
\draw[fill=green!50] (FLN33) circle(2pt);
\draw[fill=green!50] (FLN43) circle(2pt);
\end{tikzpicture}
\end{center}
                                                       
\medskip
Pour les besoins de certains tableaux , j'ai employé les macros suivantes~:

\begin{tkzexample}[code only]
	\newcommand*{\va}{\colorbox{red!50}    {$\scriptscriptstyle V_a$}}
	\newcommand*{\vb}{\colorbox{blue!50}   {$\scriptscriptstyle V_b$}}
	\newcommand*{\vbo}{\colorbox{blue!50}  {$\scriptscriptstyle V_{b1}$}}
	\newcommand*{\vbt}{\colorbox{yellow!50}{$\scriptscriptstyle V_{b2}$}}
	\newcommand*{\vc}{\colorbox{gray!50}   {$\scriptscriptstyle V_c$}} 
	\newcommand*{\vd}{\colorbox{magenta!50}{$\scriptscriptstyle V_d$}} 
	\newcommand*{\ve}{\colorbox{orange!50} {$\scriptscriptstyle V_e$}}
\end{tkzexample}

\medskip
 \begin{center}
\begin{tikzpicture}
\tkzTabInit[lgt=2,espcl=3]{$x$/1,$f'(x)$/1,$f(x)$/3}%
{$0$,$1$,$2$,$+\infty$}%
  \tkzTabLine{t,-,d,-,z,+,}%
	\tkzTabVar{+/\va , -D+/\vb/\vc,-/\vd, +D/\ve}%
\end{tikzpicture}
\end{center}

\begin{tkzexample}[code only]
	\begin{tikzpicture}
	\tkzTabInit[lgt=2,espcl=3]{$x$/1,$f'(x)$/1,$f(x)$/3}%
	{$0$,$1$,$2$,$+\infty$}%
\tkzTabLine{t,-,d,-,z,+,}%
	\tkzTabVar{+/\va , -D+/\vb/\vc,-/\vd, +D/\ve}%
\end{tikzpicture}
\end{tkzexample}  

Commentaires : Les signes $+$ et $-$ permettent de positionner une extrémité de la flèche en haut ou en bas de la ligne. Ensuite, en présence d'un seul signe, une seule expression est nécessaire. La position par rapport à la colonne est donnée par la position du signe par rapport aux autres symboles (voir \tkzname{$+D$}).  \tkzname{$-D+$} nécessite deux expressions. 

\subsection{Utilisation des symboles}

\medskip
\bgroup\parindent=0pt
\begin{minipage}{7cm} 
\begin{tkzexample}[code only]
 {+ /\va , -/\vb }
\end{tkzexample}
  \begin{tikzpicture}
  \tkzTabInit[lgt=1]{ /0.5,/2 }{ a , b }
  \tkzTabVar% 
  {+ /\va , -/\vb }  
  \end{tikzpicture}
\end{minipage}
\hfill 
\begin{minipage}{7cm}
\begin{tkzexample}[code only]
 {-/\va   ,   +/\vb}
\end{tkzexample}
  \begin{tikzpicture}
  \tkzTabInit[lgt=1]{ /0.5,/2 }{ a , b }
  \tkzTabVar% 
{-/\va   ,   +/\vb} 
  \end{tikzpicture}
\end{minipage}

\begin{minipage}{7cm} 
\begin{tkzexample}[code only]
 {+/\va   ,   +/\vb}
\end{tkzexample}
  \begin{tikzpicture}
  \tkzTabInit[lgt=1]{ /0.5,/2 }{ a , b }
  \tkzTabVar% 
{+/\va   ,   +/\vb}   
  \end{tikzpicture}
\end{minipage}
\hfill 
\begin{minipage}{7cm}
\begin{tkzexample}[code only]
 {-/\va   ,   -/\vb}
\end{tkzexample}
  \begin{tikzpicture}
  \tkzTabInit[lgt=1]{ /0.5,/2 }{ a , b }
  \tkzTabVar% 
{-/\va   ,   -/\vb}   
  \end{tikzpicture}
\end{minipage}

\begin{minipage}{7cm} 
\begin{tkzexample}[code only]
 {+/\va  , -C / \vb}
\end{tkzexample}
  \begin{tikzpicture}
  \tkzTabInit[lgt=1]{ /0.5,/2 }{ a , b }
  \tkzTabVar% 
{+/\va  , -C / \vb}  
  \end{tikzpicture}
\end{minipage}
\hfill 
\begin{minipage}{7cm}
\begin{tkzexample}[code only]
{-/\va  , +C / \vb }\end{tkzexample}
  \begin{tikzpicture}
  \tkzTabInit[lgt=1]{ /0.5,/2 }{ a , b }
  \tkzTabVar% 
{-/\va  , +C / \vb }   
  \end{tikzpicture}
\end{minipage}

\begin{minipage}{7cm}
\begin{tkzexample}[code only]
 {+C  / \va ,  -C / \vb}
\end{tkzexample}
\begin{tikzpicture}
  \tkzTabInit[lgt=1]{ /0.5,/2 }{ a , b }
   \tkzTabVar% 
 {+C  / \va ,  -C / \vb } 
\end{tikzpicture}
\end{minipage}  
\hfill 
\begin{minipage}{7cm}
\begin{tkzexample}[code only]
 {-C /\va  , +C /\vb}
\end{tkzexample}
\begin{tikzpicture}
  \tkzTabInit[lgt=1]{ /0.5,/2 }{ a , b }
     \tkzTabVar% 
 {-C /\va  , +C /\vb}  
\end{tikzpicture}
\end{minipage}  

\begin{minipage}{7cm} 
\begin{tkzexample}[code only]
 { D+  /\va  ,  -/\vb}
\end{tkzexample}
  \begin{tikzpicture}
  \tkzTabInit[lgt=1]{ /0.5,/2 }{ a , b }
  \tkzTabVar% 
 { D+  /\va  ,  -/\vb}  
  \end{tikzpicture}
\end{minipage}
\hfill 
\begin{minipage}{7cm}
\begin{tkzexample}[code only]
 { D-  /\va  ,  +/\vb} 
\end{tkzexample}
  \begin{tikzpicture}
  \tkzTabInit[lgt=1]{ /0.5,/2 }{ a , b }
  \tkzTabVar
  { D-  /\va  ,  +/\vb}  
  \end{tikzpicture}
\end{minipage}

\begin{minipage}{7cm} 
\begin{tkzexample}[code only]
 {+/\va  , -D / \vb} 
\end{tkzexample}
  \begin{tikzpicture}
  \tkzTabInit[lgt=1]{ /0.5,/2 }{ a , b }
  \tkzTabVar% 
{+/\va  , -D / \vb}  
  \end{tikzpicture}
\end{minipage}
\hfill 
\begin{minipage}{7cm}
\begin{tkzexample}[code only]
 {-/\va  , +D / \vb }
\end{tkzexample}
  \begin{tikzpicture}
  \tkzTabInit[lgt=1]{ /0.5,/2 }{ a , b }
  \tkzTabVar% 
{-/\va  , +D / \vb }   
  \end{tikzpicture}
\end{minipage}

\begin{minipage}{7cm}
\begin{tkzexample}[code only]
 {D+ / \va , -D / \vb }
\end{tkzexample}
\begin{tikzpicture}
  \tkzTabInit[lgt=1]{ /0.5,/2 }{ a , b }
   \tkzTabVar% 
 {D+  / \va ,  -D / \vb } 
\end{tikzpicture}
\end{minipage}  
\hfill 
\begin{minipage}{7cm}
\begin{tkzexample}[code only]
 {D- /\va , +D /\vb}
\end{tkzexample}
\begin{tikzpicture}
  \tkzTabInit[lgt=1]{ /0.5,/2 }{ a , b }
     \tkzTabVar% 
 {D- /\va  , +D /\vb}  
\end{tikzpicture}
\end{minipage}  

\begin{minipage}{7cm}
\begin{tkzexample}[code only]
 {+/ \va , -/ \vb , +/ \vc}
\end{tkzexample}
\begin{tikzpicture}
\tkzTabInit[lgt=1,espcl=2.5]{ /0.5,/2 }{ a , b , c }
\tkzTabVar  {+/ \va , -/  \vb ,+/  \vc} 
\end{tikzpicture}
\end{minipage}   
\hfill 
\begin{minipage}{7cm}
\begin{tkzexample}[code only]
 {+/ \va ,-C/ \vb , +/  \vc/ }
\end{tkzexample}
\begin{tikzpicture}
\tkzTabInit[lgt=1,espcl=2.5]{ /0.5,/2 }{ a , b , c }
\tkzTabVar {+/ \va  ,-C/ \vb , +/  \vc/ } 
\end{tikzpicture}
\end{minipage} 

\begin{minipage}{7cm}
\begin{tkzexample}[code only]
 {-  /\va , R , +/\vc}
\end{tkzexample}
\begin{tikzpicture}
\tkzTabInit[lgt=1,espcl=2.5]{ /0.5,/2 }{ a , b , c }
\tkzTabVar% 
{-  /\va  ,  R,  +/\vc}     
\end{tikzpicture}
\end{minipage}   
\hfill 
\begin{minipage}{7cm}
\begin{tkzexample}[code only]
 {-  /\va , R , +/\vc}
\end{tkzexample}
\begin{tikzpicture}
\tkzTabInit[lgt=1,espcl=2.5]{ /0.5,/2}{ a , b , c }
\tkzTabVar% 
{-  /\va  ,  R  ,  +/\vc}  
\end{tikzpicture}
\end{minipage} 

\begin{minipage}{7cm}
\begin{tkzexample}[code only]
 {D-/\va , +DH/\vbo/ , }
\end{tkzexample}
\begin{tikzpicture}
\tkzTabInit[lgt=1,espcl=2.5]{ /0.5,/2 }{ a , b , c }
\tkzTabVar% 
{D-/\va  ,  +DH/\vbo/ ,  }     
\end{tikzpicture}
\end{minipage}   
\hfill 
\begin{minipage}{7cm}
\begin{tkzexample}[code only]
 {D-/\va  , -DH/\va/\vb ,  D+/}
\end{tkzexample}
\begin{tikzpicture}
\tkzTabInit[lgt=1,espcl=2.5]{ /0.5,/2 }{ a , b , c }
\tkzTabVar% 
{D-/\va  ,  -DH/\vbo ,  D+/}  
\end{tikzpicture}
\end{minipage} 

\begin{minipage}{7cm}
\begin{tkzexample}[code only]
 {D-/\va , +D-/\vbo/\vbt , +D/\vc}
\end{tkzexample}
\begin{tikzpicture}
\tkzTabInit[lgt=1,espcl=2.5]{ /0.5,/2 }{ a , b , c }
\tkzTabVar% 
{D-/\va  ,  +D-/\vbo/\vbt , +D/\vc}     
\end{tikzpicture}
\end{minipage}   
\hfill 
\begin{minipage}{7cm}
\begin{tkzexample}[code only]
 {D-/\va , +D-/\vbo/\vbt , +D/\vc}
\end{tkzexample}
\begin{tikzpicture}
\tkzTabInit[lgt=1,espcl=2.5]{ /0.5,/2 }{ a , b , c }
\tkzTabVar% 
{D-/\va  ,  -D-/\vbo/\vbt , +D/\vc}  
\end{tikzpicture}
\end{minipage} 

\begin{minipage}{7cm}
\begin{tkzexample}[code only]
 {+/\va , -D- / \vbo/\vbt , +/\vc}
\end{tkzexample}
\begin{tikzpicture}
\tkzTabInit[lgt=1,espcl=2.5]{ /0.5,/2 }{ a , b , c }
\tkzTabVar  {+/ \va , -D- /\vbo/\vbt,+/\vc  } 
\end{tikzpicture}
\end{minipage} 
\hfill 
\begin{minipage}{7cm}
\begin{tkzexample}[code only]
 {+ /\va,-DC- /\vbo/\vbt,+ /\vc}
\end{tkzexample}
\begin{tikzpicture}\tikzset{low/.style   = {above  =  15pt}}
\tkzTabInit[lgt=1,espcl=2.5]{ /0.5,/2 }{ a , b , c }
\tkzTabVar {+   /\va ,-DC- /\vbo/\vbt  ,+ /\vc} 
\end{tikzpicture}
\end{minipage}  

\begin{minipage}{7cm}
\begin{tkzexample}[code only]
 {D-/\va, +DC-/\vbo/\vbt, +D/\vc}
\end{tkzexample} 
\begin{tikzpicture}
\tkzTabInit[lgt=1,espcl=2.5]{ /0.5,/2 }{ a , b , c }
\tkzTabVar% 
{D-/\va  ,  +DC-/\vbo/\vbt ,+D/\vc} 
\end{tikzpicture}
\end{minipage}
\hfill 
\begin{minipage}{7cm}
\begin{tkzexample}[code only]
 {D+/\va , +DC-/\vbo/\vbt , +D/\vc}
\end{tkzexample}
\begin{tikzpicture}
\tkzTabInit[lgt=1,espcl=2.5]{ /0.5,/2 }{ a , b , c }
\tkzTabVar% 
{D+/\va  ,  +DC-/\vbo/\vbt ,+D/\vc}  
\end{tikzpicture}
\end{minipage}

\begin{minipage}{7cm}
\begin{tkzexample}[code only]
 {D-/\va , +CD-/\vbo/\vbt , +D/\vc}
\end{tkzexample}
\begin{tikzpicture}
\tkzTabInit[lgt=1,espcl=2.5]{ /0.5,/2 }{ a , b , c }
\tkzTabVar% 
{D-/\va  ,  +CD-/\vbo/\vbt , +D/\vc} 
\end{tikzpicture}
\end{minipage}
\hfill 
\begin{minipage}{7cm}
\begin{tkzexample}[code only]
 {D-/\va , +CD-/\vbo/\vbt ,+D/\vc}
\end{tkzexample}
\begin{tikzpicture}
\tkzTabInit[lgt=1,espcl=2.5]{ /0.5,/2 }{ a , b , c }
\tkzTabVar% 
{D+/\va , +CD-/\vbo/\vbt ,  +D/\vc}  
\end{tikzpicture}
\end{minipage}

\begin{minipage}{7cm}
\begin{tkzexample}[code only]
 {+/\va, -DC+ /\vbo/\vbt, - /\vc}
\end{tkzexample}
\begin{tikzpicture}
\tkzTabInit[lgt=1,espcl=2.5]{ /0.5,/2 }{ a , b , c }
\tkzTabVar  {+ /\va ,-DC+ /\vbo/\vbt , -/\vc} 
\end{tikzpicture}
\end{minipage} 
\hfill 
\begin{minipage}{7cm}
\begin{tkzexample}[code only]
 {D- /\va, -DC- /\vbo/\vbt,+D/\vc}
\end{tkzexample}
\begin{tikzpicture}\tikzset{low/.style           = {above       =  15pt}}
\tkzTabInit[lgt=1,espcl=2.5]{ /0.5,/2 }{ a , b , c }
\tkzTabVar% 
{D- /\va  , -DC- /\vbo/\vbt , +D/\vc}   
\end{tikzpicture}
\end{minipage}

\begin{minipage}{7cm}
\begin{tkzexample}[code only]
 {+/\va  , -CH /\vbo/\vbt , D+/}
\end{tkzexample}
\begin{tikzpicture}
\tkzTabInit[lgt=1,espcl=2.5]{ /0.5,/2 }{ a , b , c }
\tkzTabVar% 
{+/\va  , -CH /\vbo/\vbt , D+/}  
\end{tikzpicture}
\end{minipage}
\hfill
\begin{minipage}{7cm}
\begin{tkzexample}[code only]
 {+  /\va  , -CH/\vb,  //}
\end{tkzexample}
\begin{tikzpicture}
\tkzTabInit[lgt=1,espcl=2.5]{ /0.5,/2 }{ a , b , c }
\tkzTabVar% 
{+  /\va  , -CH/\vb,  //}  
\end{tikzpicture}
\end{minipage}

\begin{minipage}{7cm}
\begin{tkzexample}[code only]
 {+/\va , -V- /\vbo /\vbt, +/\vc}
\end{tkzexample}
\begin{tikzpicture}
\tkzTabInit[lgt=1,espcl=2.5]{ /0.5,/2 }{ a , b , c }
\tkzTabVar  {+/\va,-V- /\vbo /\vbt, +/\vc} 
\end{tikzpicture}
\end{minipage} 
\hfill 
\begin{minipage}{7cm}
\begin{tkzexample}[code only]
 {+/ \va ,-V+ / \vbo/ \vbt ,-/ \vc}
\end{tkzexample}
\begin{tikzpicture}
\tkzTabInit[lgt=1,espcl=2.5]{ /0.5,/2 }{ a , b , c }
\tkzTabVar  {+/ \va ,-V+ / \vbo/ \vbt ,-/ \vc}   
\end{tikzpicture}
\end{minipage} 

\begin{minipage}{7cm}
\begin{tkzexample}[code only]
 {+/ \va ,+V- /\vbo/ \vbt , -/\vc}
\end{tkzexample}
\begin{tikzpicture}
\tkzTabInit[lgt=1,espcl=2.5]{ /0.5,/2 }{ a , b , c }
\tkzTabVar  {+/ \va ,+V- / \vbo/ \vbt , -/\vc} 
\end{tikzpicture}
\end{minipage}   
\hfill 
\begin{minipage}{7cm}
\begin{tkzexample}[code only]
 {-/ \va, +V+ / \vbo/\vbt, -/\vc}
\end{tkzexample}
\begin{tikzpicture}
\tkzTabInit[lgt=1,espcl=2.5]{ /0.5,/2 }{ a , b , c }
\tkzTabVar  {-/ \va ,+V+ / \vbo / \vbt, -/\vc}   
\end{tikzpicture}
\end{minipage} 

\begin{minipage}{16cm}
\begin{tkzexample}[code only]
 {-/ \va ,+H/\vb,-/\vc, +/ \vd}
\end{tkzexample}
\begin{tikzpicture}
\tkzTabInit[lgt=1,espcl=3]{ /0.5,/2 }{ a , b , c , d }
\tkzTabVar  {-/  \va  ,+H/\vb,-/\vc, +/  \vd}   
\end{tikzpicture}
\end{minipage}

\begin{minipage}{16cm}
\begin{tkzexample}[code only]
 {+/  \va  ,-H/\vb,-/\vc, +/  \vd}
\end{tkzexample}
\begin{tikzpicture}
\tkzTabInit[lgt=1,espcl=3]{ /0.5,/2 }{ a , b , c , d }
\tkzTabVar  {+/  \va  ,-H/\vb,-/\vc, +/  \vd}   
\end{tikzpicture}
\end{minipage}

\begin{minipage}{16cm}
\begin{tkzexample}[code only]
 {-/  \va  , R , R , R , +/  \ve}
\end{tkzexample}
\begin{tikzpicture}
\tkzTabInit[lgt=1,espcl=3]{ /0.5,/2 }{ a , b , c , d , e}
\tkzTabVar  {-/  \va  ,R,R,R, +/  \ve}   
\end{tikzpicture}
\end{minipage}

\begin{minipage}{16cm}
\begin{tkzexample}[code only]
  {-/ \va , +/\vb , -DH/\vc , -/\vd , +/ \ve}
 \end{tkzexample}
\begin{tikzpicture}
\tkzTabInit[lgt=1,espcl=3]{ /0.5,/2 }{ a , b , c , d , e}
\tkzTabVar  {-/ \va  ,+/\vb ,-DH/\vc,-/\vd, +/ \ve}   
\end{tikzpicture}
\end{minipage} 

\begin{minipage}{16cm}
\begin{tkzexample}[code only]
 {D-/ \va , +DH/\vb/ , D-/\vc , +/\vd , +D/\ve}
\end{tkzexample}
\begin{tikzpicture}
\tkzTabInit[lgt=1,espcl=3]{ /0.5,/2 }{ a , b , c , d , e}
\tkzTabVar  {D-/  \va ,+DH/\vb/,D-/\vc,+/\vd, -D/\ve}   
\end{tikzpicture}
\end{minipage}
\egroup

\medskip
Commentaires
\begin{itemize}
  \item on peut employer la syntaxe suivante dans pratiquement tous les cas $s(i)/\ldots/\ldots$ mais alors il faut bien positionner les expressions;
  
  \item l'argument vide est employé parfois à la fin d'une ligne mais dans ce cas aucune flèche n'est tracée;
  
  \item $C+$ et $C-$ n'existent pas. $+C$ et $-C$ suffisent car les expressions sont centrées;
  \item  $D+$ et $D-$ existent .

\end{itemize}


\subsection{Utilisation des options}

\subsubsection{\texttt{\textcolor{red}{color}} : modification de la couleur des flèches}
Il est possible de personnaliser le tableau à l'aide de styles.
\begin{tkzexample}[vbox, small]
\begin{tikzpicture}
  \tkzTabInit[color,espcl=8]%
    {$x$   /1,%
     Signe\\ de $\dfrac{1}{x}$ /1.5,
     Variation\\ de $\ln$      /1.5}%
    {$0$,$+\infty$}%
  \tkzTabLine{d,+,}%
  \tkzTabVar[color=red]%
    {D-/    /  $-\infty$,+/  $+\infty$  /}
\end{tikzpicture}
\end{tkzexample} 
 
\subsubsection{\texttt{\textcolor{red}{help}} : affiche la structure du tableau} 
\Iopt{tkzTabVar}{help}   
Voir le chapitre personnalisation\ref{pers}
\subsection{Utilisation des styles}
 
\subsubsection{Modification de la couleur d'une zone interdite}
\Istyle{tkzTabvar}{h style}
Si vous préférez hachurer une zone du tableau, alors  il faut modifier un style. 

Par défaut, \tkzname{h style}  est défini ainsi:
\begin{tkzexample}[code only] 
 \tikzset{h style/.style = {fill=gray,opacity=0.4}}
\end{tkzexample}

Une autre définition peut être :

\begin{tkzexample}[code only]
 \tikzset{h style/.style = {fill=red!50}}
\end{tkzexample}

\begin{tkzexample}[vbox,width=9cm]
\begin{tikzpicture}
 \tikzset{h style/.style = {fill=red!50}}
 \tkzTabInit[lgt=1,espcl=2]{$x$ /1,  $f$ /2}{$0$,$1$,$2$,$3$}%
 \tkzTabVar{+/ $1$  / , -CH/ $-2$ / , +C/  $5$, -/ $0$  /  }
\end{tikzpicture}
\end{tkzexample}  

\subsubsection{\texttt{\textcolor{red}{h style}} Zone interdite hachurée}
\Istyle{tkzTabVar}{h style}

\begin{tkzexample}[code only]
  \tikzset{h style/.style = {pattern=north west lines}} 
\end{tkzexample}
 Ce code permet d'hachurer la zone
 
\begin{tkzexample}[vbox,width=9cm,small]
\begin{tikzpicture}
 \tikzset{h style/.style = {pattern=north west lines}}
 \tkzTabInit[lgt=1,espcl=2]{$x$ /1,  $f$ /2}{$0$,$1$,$2$,$3$}%
 \tkzTabVar{+/ $1$  / , -CH/ $-2$ / , +C/  $5$, -/ $0$  /  }
\end{tikzpicture}
\end{tkzexample}


\subsubsection{\texttt{\textcolor{red}{arrow style}} style des flèches.}
\Istyle{tkzTabVar}{arrow} 
Le style des flèches est \tkzname{arrow style} et il est défini ainsi :

\begin{tkzexample}[code only]
   \tikzset{arrow style/.style   = {\cmdTAB@VA@color,
                                    ->,
                                    >           = latex',
                                    shorten >   =  2pt,
                                    shorten <   =  2pt}}
\end{tkzexample}

 On limite l'approche des nodes par les arrows. Voici une modification possible du style

\begin{tkzexample}[code only]
  \tikzset{arrow style/.style   = {blue,
                                   ->,
                                   >           = latex',
                                   shorten >   =  6pt,
                                   shorten <   =  6pt}}
\end{tkzexample}
 
 La couleur et l'approche des flèches sont modifiées.
 
\begin{tkzexample}[vbox,small]
\begin{tikzpicture}
  \tikzset{arrow style/.style   = {blue,
                                   ->,
                                   >           = latex',
                                   shorten >   =  6pt,
                                   shorten <   =  6pt}}  
  \tkzTabInit[espcl=5]{$x$ /1, $\ln x +1$ /1.5, $x \ln x$ /2}%
     {$0$ ,$1/\E$ , $+\infty$}%
  \tkzTabLine{d,-,z,+,}
  \tkzTabVar%
  { D+/   / $0$ ,%
     -/ \colorbox{black}{\textcolor{white}{$\dfrac{-1}{e}$}}/ ,%
     +/ $+\infty$  /  }%
\end{tikzpicture}
\end{tkzexample}

\subsubsection{\texttt{\textcolor{red}{node style}} Style des nodes}
\Istyle{tkzTabVar}{node style} 
Par défaut, Le style des nodes est \tkzname{node style} et il est défini ainsi : 
\begin{tkzexample}[code only]  
\tikzset{node style/.style    = {inner sep   =  2pt,
                                 outer sep   =  2pt,
                                 fill        =  \cmdTAB@tbs@colorT}}\end{tkzexample}
Si on veut apporter des modifications mais conserver une partie de ce style, on peut agir ainsi :

\begin{tkzexample}[code only]  
  \tikzset{node style/.append style = {draw,circle,fill=red!40,opacity=.4}}
\end{tkzexample}
 
 Par défaut les nodes sont des rectangles non tracés, ils deviennent des disques

\begin{tikzpicture}
  	\tikzset{node style/.append style    = {draw,circle,fill=red!40,opacity=.4}}
  \tkzTabInit[espcl=5]{$x$ /1, $\ln x +1$ /1.5, $x \ln x$ /2}%
     {$0$ ,$1/\E$ , $+\infty$}%
  \tkzTabLine{d,-,z,+,}
  \tkzTabVar%
  { D+/   / $0$ ,%
     -/ \colorbox{black}{\textcolor{white}{$\dfrac{-1}{e}$}}/ ,%
     +/ $+\infty$  /  }%
\end{tikzpicture}   
                               
\begin{tkzexample}[code only,small]
\begin{tikzpicture}
  	\tikzset{node style/.append style    = {draw,circle,fill=red!40,opacity=.4}}
  \tkzTabInit[espcl=5]{$x$ /1, $\ln x +1$ /1.5, $x \ln x$ /2}%
     {$0$ ,$1/\E$ , $+\infty$}%
  \tkzTabLine{d,-,z,+,}
  \tkzTabVar%
  { D+/   / $0$ ,%
     -/ \colorbox{black}{\textcolor{white}{$\dfrac{-1}{e}$}}/ ,%
     +/ $+\infty$  /  }%
\end{tikzpicture}
\end{tkzexample}  

\subsection{Quelques exemples}
\subsubsection{Fonction inverse}
Étude de la fonction inverse $i~:~ x \longmapsto \frac{1}{x}$ sur $]-\infty~;~0[ \cup ]0~;~+\infty[$ 


\begin{tkzexample}[vbox,small]
\begin{tikzpicture}
  \tkzTabInit[lgt=1.5,espcl=6.5]{$x$  /1,$i'(x)$  /1,$i$ /3}
                            {$-\infty$,$0$,$+\infty$}% 
  \tkzTabLine{,-,d,-,}
  \tkzTabVar{+/   $0$ /  ,-D+/ $-\infty$ / $+\infty$ , -/ $0$ /}
\end{tikzpicture}
\end{tkzexample}

\subsubsection{Fonction avec des paliers, emploi du symbole \texttt{\textcolor{red}{R}}}
Il est possible avec R de passer plusieurs valeurs.

\begin{tkzexample}[vbox,small]
\begin{tikzpicture}
 \tkzTabInit[espcl=4]{$x$  /1,$f'(x)$ /1,$f(x)$ /2}
                     {$0$ , $1$ ,$2$, $+\infty$}%
 \tkzTabLine         {d,+ , z,+ , z,+ ,        }
 \tkzTabVar{D-/ / $-\infty$,R/  /,R/   /,+/ $+\infty$   /}%
\end{tikzpicture}
\end{tkzexample}

\subsubsection{Zone interdite}

\begin{tkzexample}[vbox,small]
\begin{tikzpicture}
 \tkzTabInit[lgt=1,espcl=2]{$x$ /1,  $f$ /2}{$0$,$1$,$2$,$3$}%
 \tkzTabVar{+/ $1$ /   ,-DH/ $-\infty$ /   ,D+/   / $+\infty$, -/ $2$ / }
\end{tikzpicture}
\end{tkzexample}

\subsubsection{Zone interdite + prolongement par continuité}
\index{zone interdite}
\begin{tkzexample}[vbox,small]
\begin{tikzpicture}
 \tkzTabInit[lgt=1,espcl=2]{$x$ /1,  $f$ /2}{$0$,$1$,$2$,$3$}%
 \tkzTabVar{+/ $1$  / ,-CH/ $-2$ /, D+/ / $+\infty$,-/ $2$  / }
\end{tikzpicture}
\end{tkzexample}

\subsubsection{Zone interdite + double prolongement par continuité}
\index{prolongement par continuité} 
\begin{tkzexample}[vbox]
\begin{tikzpicture}
 \tkzTabInit[lgt=1,espcl=2]{$x$ /1,  $f$ /2}{$0$,$1$,$2$,$3$}%
 \tkzTabVar{+/ $1$  / , -CH/ $-2$ / , +C/  $5$, -/ $0$  /  }
\end{tikzpicture}
\end{tkzexample}

\subsubsection{Exemple d'une fonction partiellement constante}

Utilisation de l'option nocadre qui supprime le cadre extérieur, sinon on peut constater que l'on peut mettre pratiquement ce que l'on veut avec la macro \tkzcname{signe}.
\begin{tkzexample}[vbox,small]
\begin{tikzpicture}
 \tkzTab[nocadre,lgt=3,espcl=4]
  {$x$                  /1,
  Signe\\ de $f'(x)$    /1.5,
  Variations\\ de\\ $f$ /2}
  {$-\infty$, $-2$,$\dfrac{1}{\E}$,$\E$}%
  {z, <--- 0 --->,d, -, d, \genfrac{}{}{0pt}{0}{\text{signe de}}{ a}, d}
  {+/ $\dfrac{2}{3}$, +/ $\dfrac{2}{3}$,
   -D-/ $-\infty$ / $-\infty$,+D/ $+\infty$ }
\end{tikzpicture}
\end{tkzexample}

\subsubsection{Double variations}

\begin{tkzexample}[vbox,small]
\begin{tikzpicture}
 \tkzTabInit[espcl=6]
     {$x$   /1, $f''{x}$ /1,$f'(x)$ /2,  $f(x)$ /2}%
     {$0$ , $1$ , $+\infty$    }%
 \tkzTabLine{d,+,z,-, }%
 \tkzTabVar {D-/   /$1$,+/ $\E$ /,-/  $0$ /}%
 \tkzTabVar {D-/    /$-\infty$ ,R/    $0$   /, +/   $+8$ /}
\end{tikzpicture}
\end{tkzexample} 



\endinput
% 20 / 02 /2009 v1.00c TKZdoc-tab-tv
\section{Création d'un tableau de variations : \addbs{tkzTab}}
\subsection{Définition} 
\begin{NewMacroBox}{tkzTab}%
{\oarg{local options}\var{liste1}\var{liste2}\var{liste3}\var{liste4}}

\medskip
\begin{tabular}{lllc}
\toprule
\texttt{arguments}   & \texttt{défaut}    & \texttt{définition}       \\
\midrule
\IargName{tkzTab}{liste1}  & |no default|  & \var{e(1)/h(1),\dots,e(p)/h(p)} première colonne     \\
\IargName{tkzTab}{liste2}  & |no default|  & \var{a(1),\dots,a(n)}    antécédents de la première ligne  \\
\IargName{tkzTab}{liste3}  & |no default|  & \var{s(1),\dots,s(2n-1)} symboles de la ligne de signes   \\
\IargName{tkzTab}{liste4}  & |no default|  & \var{s(1)/eg(1)/ed(1),\dots,s($q$)/eg($q$)/ed($q$)} variations     \\
\bottomrule
\end{tabular}

\medskip
\noindent\emph{La macro \emph{\tkzcname{tkzTab}} est un raccourci pour enchaîner \tkzcname{tkzTabInit}, \tkzcname{tkzTabLine}  et \tkzcname{tkzTabVar}. Les \tkzname{options} sont identiques à celles de \tkzcname{tkzTabInit}. Ces tableaux ne concernent que les tableaux à trois lignes pour la variable, le signe de la dérivée et les variations de la fonction.}
\end{NewMacroBox}

\medskip

\begin{tkzexample}[code only]
 \tkzTab{ e(1) / h(1) ,
             ... ,
          e(p) / h(p)}
        { v(1), ... ,v(n) }
        { a(1),...,a(2n-1)}
        { s(1) / eg(1) / ed(1), ... ,s(n) / eg(n) / ed(n)}
\end{tkzexample}

\subsection{Exemple 1} 

Étude de la fonction $f~:~ x \longmapsto x^2$ sur $[-5~;~7]$

\begin{tkzexample}[vbox,small]
\begin{tikzpicture}
\tkzTab[lgt=3,espcl=5]{ $x$                    / 1,
                        $f'(x)$                / 1,
                        Variations de \\$f$ / 2}
                      { $-5$ , $0$ ,$7$}
                      { ,-,z,+,}
                      { +/$25$  , -/$0$  , +/ $49$}%
\end{tikzpicture}
\end{tkzexample}  

\subsection{Exemple 2} 
Étude de la fonction $f~:~ x \longmapsto x \ln x $ sur $]0~;~+\infty]$

\begin{tkzexample}[vbox,small]
 \begin{tikzpicture}
 \tkzTab[espcl=5,lgt=3]{$x$ / 1, Signe de \\$\ln x +1$ / 1.5,%
   Variations de \\$f$ / 3}%
   {$0$ ,$1/\E$ , $+\infty$}{d,-,z,+,}
   {D+/ $0$,%
    -/ \colorbox{black}{\textcolor{white}{$\dfrac{-1}{e}$}}  ,%
    +/ $+\infty$ }%
\end{tikzpicture}
\end{tkzexample}  

\subsection{Exemple 3} 
 Étude de  la fonction $f~:~ $x$ \longmapsto \sqrt{x^2-1}$ sur $]-\infty~;~-1] \cup [1~;~+\infty[$
 
\begin{tkzexample}[vbox,small]
\begin{tikzpicture}
  \tkzTab{ $x$ / 1, $f'(x)$ / 1, $f$ / 2}%
         { $-\infty$, $-1$ ,$1$, $+\infty$}
         { ,-,d,h,d,+, }
         { +/$+\infty$  , -H/$0$, -/$0$  , +/ $+\infty$  }%
\end{tikzpicture}
\end{tkzexample}  

\subsection{Exemple 4} 
 Étude de  la fonction $f~:~ $t$ \longmapsto \frac{t^2}{t^2-1}$ sur $[0~;~+\infty[$
 
\begin{tkzexample}[vbox,small]
\begin{tikzpicture}
  \tkzTab{ $t$ / 1, Signe de\\ $f'(t)$ / 2, Variation de \\$f$ / 2}%
         { $0$, $1$, $+\infty$}
         { z , - , d , - , }
         { +/$0$  , -D+/$-\infty$/$+\infty$, -/ $1$  }%
\end{tikzpicture}
\end{tkzexample}  
\endinput


%!TEX root = /Users/ego/Boulot/TKZ/tkz-tab/doc/TKZdoc-tab-main.tex 
% 20 / 02 /2009 v1.00c TKZdoc-tab-valeurs
\section{Valeurs intermédiaires \addbs{tkzTabVal}}
Cette macro permet de placer une valeur sur une flèche de la ligne des variations. Elle doit être employée juste après la commande \tkzcname{tkzTabVar} définissant la ligne de variations sur laquelle on souhaite placer les valeurs intermédiaires. On ne peut placer une valeur que dans un intervalle où la fonction est \tkzname{monotone}. Cette macro permet d'afficher une nouvelle valeur (intermédiaire) dans la première ligne. 

\subsection{Définition de   \addbs{tkzTabVal}}

\begin{NewMacroBox}{tkzTabVal}{\oarg{local options}\{Début\}\{Fin\}\{Position\}\{Antécédent\}\{Image\}}

\begin{tabular}{lllc}
\toprule
\texttt{arguments}   & \texttt{défaut}    & \texttt{définition}                        \\
\midrule
\IargName{tkzTabVal}{Début}     & |no default|  & rang de l'origine de la flèche       \\
\IargName{tkzTabVal}{Fin}       & |no default|  & rang de l'extrémité de la flèche     \\
\IargName{tkzTabVal}{Position}  & |no default|  & nombre décimal entre $0$ et $1$      \\
\IargName{tkzTabVal}{Antécédent}& |no default|  & valeur de l'antécédent si nécessaire \\
\IargName{tkzTabVal}{Image}     & |no default|  & valeur de l'image  si nécessaire     \\
\bottomrule
\end{tabular}

\medskip
\noindent\emph{Ceci mérite quelques commentaires : Il s'agit de savoir sur quelle flèche, on va positionner l'image. \tkzname{Début} et \tkzname{Fin} sont les rangs des valeurs qui déterminent les extrémités de la flèche. \tkzname{Antécédent}  \tkzname{Image} sont les valeurs que l'on veut placer. \tkzname{Position} est    un nombre qui est  obligatoirement compris entre $0$ et $1$. C'est une abscisse en prenant comme origine  \tkzname{Début} et comme extrémité  \tkzname{Fin}.}

\medskip
\begin{tabular}{lllc}
\toprule
\texttt{options}   & \texttt{défaut}    & \texttt{définition}                                      \\
\midrule
\IoptName{tkzTabVal}{draw}    & |true|   & dessin d'une flèche entre l'antécédent et son image     \\
\IoptName{tkzTabVal}{remember}& |lastval|& définit un node personnalisé                            \\
\bottomrule
\end{tabular}

\medskip
\noindent\emph{Si vous  voulez  une flèche entre l'antécédent et l'image, il vous suffit de passer en option  \tkzname{draw}. Si vous voulez référencer le point où se situe l'image alors il faut utiliser l'option \tkzname{remember}.}
\end{NewMacroBox}

\subsubsection{Ajout de valeurs intermédiaires} 

Le premier exemple montre des valeurs remarquables pour la fonction $\ln$. Il s'agit de mettre en évidence des valeurs importantes pour la fonction. La fonction est monotone  entre les valeurs de rang $1$ ($0$) et $2$ ($+\infty$), ainsi les deux premiers arguments sont $1$  et $2$. Les coefficients utilisés pour  \tkzname{Position} sont des nombres \tkzname{décimaux} ici $0.33$ et $0.66$. Les antécédents n'étaient pas présents dans la première ligne aussi leurs valeurs sont passées dans les arguments.

\begin{tkzexample}[code only]
    \tkzTabVal{1}{2}{0.33}{1}{0}
    \tkzTabVal{1}{2}{0.66}{\E}{1}
\end{tkzexample}

\begin{tkzexample}[vbox, small]
\begin{tikzpicture}
\tkzTabInit[lgt=3,espcl=10] {$x$    /1, Signe\\ de $\dfrac{1}{x}$  /1.5,%
                       Variation\\ de $\ln$   /2} {$0$ , $+\infty$}%
    \tkzTabLine{d,+,}%
    \tkzTabVar[color=red]{ D- /  $-\infty$, + /  $+\infty$ }
    \tkzTabVal{1}{2}{0.33}{1}{0}
    \tkzTabVal{1}{2}{0.66}{\E}{1}
\end{tikzpicture}
\end{tkzexample}

\subsubsection{Ajout de valeurs intermédiaires avec une fonction non monotone } 

On ne peut utiliser la macro que sur un intervalle où la fonction est monotone, ici il y a trois valeurs 
\mbox{$0$, $\E$ et $+\infty$}. La fonction est monotone entre les deux premières c'est à dire entre  les valeurs de rang $1$ et $2$ ainsi qu'entre les deux dernières de rang $2$ et $3$.

\begin{tkzexample}[vbox,small]
\begin{tikzpicture}
  \tkzTabInit[espcl=6]{$x$ / 1 , $f'(x)$ / 1, $f(x)$ / 2}
                      {$0$, $\E$ , $+\infty$}%
  \tkzTabLine{d,+,0,-,}%
  \tkzTabVar{D- / $-\infty$, + / $\E$, - / $0$  }%
  \tkzTabVal[draw]{1}{2}{0.6}{$1$}{$\dfrac{1}{\E}$}%
  \tkzTabVal[draw]{2}{3}{0.4}{$\E^2$}{$1$}%
\end{tikzpicture}
\end{tkzexample}

\subsubsection{Ajout de valeurs intermédiaires  avec un palier} 

Il ne faut pas s'arrêter au deuxième antécédent. La fonction est  monotone mais admet un palier. L'option \tkzname{R} permet d'éviter qu'une flèche s'arrête pour $\sqrt\E$. La flèche va donc de la valeur de rang $1$ à la valeur de rang $3$. Le code est donc :
\begin{tkzexample}[code only]
  \tkzTabVal[draw]{1}{3}{0.6}{\E}{$\dfrac{-1}{\E}$} 
\end{tkzexample}

\begin{tkzexample}[vbox,small]
\begin{tikzpicture}
 \tkzTabInit[espcl=6]{$x$/1,$f'(x)$/1, $f(x)$/2}
                     {$0$,$\sqrt\E$,$+\infty$}%
 \tkzTabLine{d,+,0,+,}%
 \tkzTabVar{D-  / $-\infty$,R  /     ,+  / $0$  }
 \tkzTabVal[draw]{1}{3}{0.4}{$1$}{$-\E$}
\end{tikzpicture}
\end{tkzexample}


\subsubsection{Valeurs intermédiaires et plusieurs lignes de variations }
\Iopt{tkzTabVal}{draw}

Les variations de  $f$ et $f'$ sont représentées. Pour $f$ la valeur $1$ n'est pas utilisée, on passe donc du rang $1$ au rang $3$.

\begin{tkzexample}[vbox,small]
\begin{tikzpicture} 
  \tkzTabInit[espcl=6]{$x$/1,$f''(x)$/1,$f'(x)$/3,$f(x)$/3}
  {$0$,$1$,$+\infty$}% 
  \tkzTabLine{d,+,0,-, }%
  \tkzTabVar{-/ $-\infty$ ,+/ ,-/ $-\infty$ }
  \tkzTabVal[draw]{1}{2}{0.3}{$0,3$}{$-2$}
  \tkzTabVal[draw]{2}{3}{0.6}{$4$}{$-1$} 
  \tkzTabVar{+/ $+\infty$,R ,-/ $-1$} 
  \tkzTabVal[draw]{1}{3}{0.6}{$2$}{$0$}
\end{tikzpicture}  
\end{tkzexample}
 
\subsection{Utilisation des options}

\subsubsection{\texttt{\textcolor{red}{draw}} : ajout d'une flèche vers la valeur ajoutée}\Iopt{tkzTabVal}{draw}
L'option a déjà été utilisée dans les exemples précédents, en voici un autre. 

\begin{tkzexample}[vbox,small]
\begin{tikzpicture}
 \tkzTabInit[lgt=3,espcl=10]{$x$                 /1,
                       Signe\\ de $\dfrac{1}{x}$ /2,
                       Variation\\ de $\ln$      /3}
                       {$0$           , $+\infty$  }%
 \tkzTabLine           {d,+           ,            }%
 \tkzTabVar[color=red]{D-/ $-\infty$ , +/$+\infty$}%
 \tkzTabVal[draw]{1}{2}{0.24}{\scriptsize $1-h$}{$<0$}%
 \tkzTabVal[draw]{1}{2}{0.3}{$1$}{$0$}%
 \tkzTabVal[draw]{1}{2}{0.36}{\scriptsize $1+h$}{$>0$}%
 \tkzTabVal[draw]{1}{2}{0.64}{$2,7$}{$<$}%
 \tkzTabVal[draw]{1}{2}{0.7}{$\E$}{$1$}%
 \tkzTabVal[draw]{1}{2}{0.76}{$2,8$}{$>$}%
\end{tikzpicture}
\end{tkzexample}


\subsubsection{\texttt{\textcolor{red}{remember}} : attribuer un nom à  un point ou un node.}
\Iopt{tkzTabVal}{remember}

Cette option permet d'utiliser \tkzcname{tkzTabImaFrom} mais il est possible de récupérer les noms des nodes et de les traiter avec par exemple du code de \TIKZ.

\begin{tkzexample}[code only]
  \draw[opacity=0.4,fill=red!20]  (vb) circle(3ex);
  \draw[opacity=0.4,fill=blue!20]  (vc) circle(3ex);
\end{tkzexample}


\begin{tkzexample}[,small]
  \begin{tikzpicture}
  \tkzTabInit[lgt=3,espcl=6]{ $x$/1,/1,/3,/3 }%
             {  $a$    , $d$    ,$e$}
  \tkzTabLine{  z,+    ,z,-     ,z  } 
  \tkzTabVar  {-/\va  ,+/\vd   , -/  \ve}
  
  \tkzTabVal[draw,remember=vb]{1}{2}{0.333}{$b$}{$0$}
  \tkzTabVal[draw,remember=vc]{1}{2}{0.666}{$c$}{$1$}
  
  \tkzTabVar{-/\va  ,R/   , +/  \ve}
  
  \tkzTabVal[draw]{1}{3}{0.5}{}{$0$}
  
  \draw[opacity=0.4,fill=red!20]  (vb) circle(3ex);
  \draw[opacity=0.4,fill=blue!20]  (vc) circle(3ex);
  \end{tikzpicture}
\end{tkzexample}

\medskip
Il faut remarquer que $b$ et $c$ sont des valeurs intermédiaires car le tableau a été défini avec $a$, $d$ et $e$.
\endinput
% 20 / 02 /2009 v1.00c TKZdoc-tab-image
\section{Ajout d'images \addbs{tkzTabIma} et \addbs{tkzTabImaFrom}}
Ces macros permettent de placer une valeur sur une flèche de la ligne des variations. On ne peut placer une valeur que dans un intervalle où la fonction est \tkzname{monotone}, de plus l'image est celle d'un antécédent déjà défini dans la première ligne.  La première macro est \tkzcname{tkzTabIma}.

\subsection{Définition de   \addbs{tkzTabIma}}

\begin{NewMacroBox}{tkzTabIma}{\oarg{local options}\{Début\}\{Fin\}\{Position\}\{Antécédent\}\{Image\}}

\begin{tabular}{lllc}
\toprule
\texttt{arguments}   & \texttt{défaut}    & \texttt{définition}                        \\
\toprule
\IargName{tkzTabIma}{Début} & |no default|& rang de l'origine de la flèche      \\
\IargName{tkzTabIma}{Fin}  & |no default|& rang de l'extrémité de la flèche   \\
\IargName{tkzTabIma}{Position} & |no default| & rang de l'antécédent correspondant à l'image      \\
\IargName{tkzTabIma}{Image}     & |no default|  & valeur de l'image  si nécessaire     \\
\bottomrule
\end{tabular}

\medskip
\noindent\emph{Ceci mérite quelques commentaires : Il s'agit de savoir sur quelle flèche, on va positionner l'image. \tkzname{Début} et \tkzname{Fin} sont les rangs des valeurs qui déterminent les extrémités de la flèche. \tkzname{Image} est la valeur que l'on veut placer. \tkzname{Position} est un nombre entier qui est le rang de l'antécédent.}

\medskip
\begin{tabular}{lllc}
\toprule
\texttt{options}   & \texttt{défaut}    & \texttt{définition}                                      \\
\midrule
\IoptName{tkzTabIma}{draw}    & |true|   & dessin d'une flèche entre l'antécédent et son image     \\
\IoptName{tkzTabIma}{remember}& |lastval|& définit un node personnalisé                            \\
\bottomrule
\end{tabular}

\medskip
\noindent\emph{Si vous  voulez  une flèche entre l'antécédent et l'image, il vous suffit de passer en option  \tkzname{draw}. Si vous voulez référencer le point où se situe l'image alors il faut utiliser l'option \tkzname{remember}.}
\end{NewMacroBox}

\subsubsection{Ajout de valeurs intermédiaires à partir d'un antécédent donné}
Il y a plusieurs  possibilités mais la suivante est préférable. L'antécédent est de rang $2$.
 La fonction est monotone entre les valeurs de rang $1$ et $3$. Voici comment faire apparaître l'image par $f$ de $\sqrt\E$.

\begin{tkzexample}[vbox,small]
\begin{tikzpicture}
\tkzTabInit[espcl=6]%
  {$x$/1,$f'(x)$/1, $f(x)$/2}{$0$,$\sqrt\E$,$+\infty$}%
 \tkzTabLine{d,+,0,+,}%
 \tkzTabVar{D- /$-\infty$ , R /    ,+ / $0$ }%
 \tkzTabIma{1}{3}{2}{-5}
\end{tikzpicture}
\end{tkzexample}  
\Iopt{tkzTabVal}{draw}

Une autre possibilité est d'utiliser la macro \tkzcname{tkzTabImaFrom} ainsi que les nodes créés pour construire le tableau ;  voir la section \og personnalisation \fg\ (\ref{pers}) et la fin de ce chapitre. 

\subsubsection{Exemple avec plusieurs lignes de variations}
\begin{tkzexample}[vbox,small]
	\begin{tikzpicture}

	  \tkzTabInit[espcl=4]
	     { $x$                     /1,
	       $f''(x)$                /1,
	       $f'$                     /2,
	        Signe de\\ $f'(x)$      /2,
	       $f$                     /3}%
	     { $0$      , $1$ ,     $\alpha$,$+\infty$ }%
	  \tkzTabLine {d    , + ,   z  , -  ,      , -  }%
	  \tkzTabVar 
	      {- / $1$       ,
	       + /           ,
	       R /           ,
	       - / $-\infty$ }
	  \tkzTabIma[draw]{2}{4}{3}{$0$}
    % ou bien  \tkzTabVal[draw]{2}{4}{0.5}{}{0} obsolète
    	  \tkzTabLine {     , + ,      , +  ,  z   , -  }% 
	  \tkzTabVar   
	     {- / $-\infty$ ,
	      R /     ,
	      + / $1$ ,
	      - / $0$       }
	  \tkzTabIma[draw]{1}{3}{2}{$0$}
	\end{tikzpicture}
\end{tkzexample}


\subsubsection{Fonctions paramétrées}
\NameFonct{Fonctions paramétrées}

\begin{tkzexample}[vbox,small]
  \begin{tikzpicture}
  \tkzTabInit[ lgt=4,  deltacl=1, espcl=2]%
    {$t$                             /1,
    Signe de\\      $x'(t)$          /1.5,
    Variations de\\ $x$              /3,
     Variations de\\  $y$            /3,
    Signe de\\ $y'(t)$              /1.5}
    { $-\infty$ , $-4$ , $-1$ , $0$, $2$ , $+\infty$}%
    
  \tkzTabLine { , - , z , + , d , + , z , - , d , - , }
  
  \tkzTabVar  {+/$1$ , -/$ \frac{8}{9}$ ,+D-/$+\infty$/$-\infty$ ,
               +/$0$/ ,-D+ /$-\infty$/ $+\infty$ , -/$1$ /   }
               
  \tkzTabVar  {+/$+\infty$ , R/ ,-D+/$-\infty$/$+\infty$ ,
               -/$0$ ,R / , +/$+\infty$ }
               
  \tkzTabIma{1}{3}{2}{$\frac{32}{3}$} 
  \tkzTabIma{4}{6}{5}{$\frac{16}{3}$} 
  
  \tkzTabLine{ , - , \frac{-64}{9} , - , d , - , z , + , \frac{44}{9} , + , }
 \end{tikzpicture}
\end{tkzexample}

\subsection{Définition de \addbs{tkzTabImaFrom}}
Cette macro ressemble à la précédente mais elle permet de placer une image relativement à une autre image ou relativement à un point quelconque du tableau auquel on a attribué un nom.

\begin{NewMacroBox}{tkzTabImaFrom}{\oarg{local options}\{Début\}\{Fin\}\{From\}\{Image\}}

\begin{tabular}{lllc}
\toprule
\texttt{arguments}   & \texttt{défaut}    & \texttt{définition}                 \\
\midrule
\IargName{tkzTabImaFrom}{Début}  & |no default|  & rang de l'origine de la flèche       \\
\IargName{tkzTabImaFrom}{Fin}  & |no default|  & rang de l'extrémité de la flèche     \\
\IargName{tkzTabImaFrom}{From}  & |no default|  & nom d'un point      \\
\IargName{tkzTabImaFrom}{Image}  & |no default|  & valeur de l'image     \\
\bottomrule
\end{tabular}

\medskip
\noindent\emph{Comme pour \tkzcname{tkzTabVal}, \tkzname{Début} et \tkzname{Fin} sont les rangs des valeurs qui déterminent les extrémités de la flèche.  \tkzname{Image} est la valeur que l'on veut placer. \tkzname{From} est  le nom du node qui correspond à l'antécédent.}

\medskip
\begin{tabular}{lllc}
\toprule
\texttt{options}   & \texttt{défaut}    & \texttt{définition}                                     \\
\midrule
\IoptName{tkzTabImaFrom}{draw}    & |true|   & dessin d'une flèche entre l'antécédent et son image     \\
\IoptName{tkzTabImaFrom}{remember}& |lastval|& définit un node personnalisé     \\
\bottomrule
\end{tabular}

\medskip
\noindent\emph{Si vous  voulez  une flèche entre l'antécédent et l'image, il vous suffit de passer en option  \tkzname{draw}. Si vous voulez référencer le point où se situe l'image alors il faut utiliser l'option \tkzname{remember}.}

\end{NewMacroBox}

\subsubsection{Utilisation d'un node défini par la macro \addbs{tkzTabInit}}
Il s'agit ici de \tkzname{N21}. C'est un node, plus exactement un point situé sous la seconde valeur $\sqrt\E$  et sur le premier filet horizontal   sous cette valeur. Voir le chapitre \tkzname{personnalisation} et en particulier l'option \tkzname{help} qui permet d'afficher différents points de construction.

\begin{tkzexample}[vbox,small]
\begin{tikzpicture}
\tkzTabInit[espcl=6]%
  {$x$/1,$f'(x)$/1, $f(x)$/3}{$0$,$\sqrt\E$,$+\infty$}%
 \tkzTabLine{d,+,0,+,}%
 \tkzTabVar{D-/      $-\infty$, R/      , +/$0$   }
 \tkzTabImaFrom[draw]{1}{3}{N21}{-5}
 \draw[opacity=0.4,fill=red!30]  (N21) circle(3ex);
 \draw[fill=red]  (N21) circle(2pt);
 \node[above right= 12pt,red](txt) at (N21) {$N21$};
\end{tikzpicture}
\end{tkzexample}  
\Iopt{tkzTabImaFrom}{draw}

\subsubsection{Utilisation d'un point défini  par l'utilisateur avec \texttt{\textcolor{red}{remember}}}

\begin{tkzexample}[vbox, num,small]
	\begin{tikzpicture}
	\tkzTabInit[lgt=3,espcl=6]{ $x$/1, $f'(x)$/1, $f(x)$/3,/3 }%
	           {  $a$    , $d$    ,$e$}
	\tkzTabLine{  z,+    ,z,-     ,z  } 
	\tkzTabVar  {-/\va  ,+/\vd   , -/  \ve}
	\tkzTabVal[draw,remember=vb]{1}{2}{0.333}{b}{$0$}
	\tkzTabVal[draw,remember=vc]{1}{2}{0.666}{c}{$1$}
	\tkzTabVar{-/\va  ,R/   , +/  \ve}
	\tkzTabVal[draw]{1}{3}{0.5}{}{$0$}
	\tkzTabImaFrom[draw]{1}{3}{vc}{$-1$}
	\tkzTabImaFrom[draw]{1}{3}{vb}{$-2$}
	\end{tikzpicture}
\end{tkzexample}
\Iopt{tkzTabImaFrom}{remember}

\endinput
%!TEX root = /Users/ego/Boulot/TKZ/tkz-tab/doc/TKZdoc-tab-main.tex   
% $Id$  
% 20 / 02 /2009 v1.00c TKZdoc-tab-tangente    
%  Created by Alain Matthes on 2010-02-23.
%  Copyright (c) 2010 __Collège Sévigné__. All rights reserved.
% 

\section{Tangente horizontale : \addbs{tkzTabTan} et  \addbs{tkzTabTanFrom}}
\subsection{Définition de \tkzcname{tkzTabTan}}

\begin{NewMacroBox}{tkzTabTan}{\oarg{local options}\{Début\}\{Fin\}\{Position\}\{Image\}}

\begin{tabular}{lllc}
\toprule
\texttt{arguments}   & \texttt{défaut}    & \texttt{définition}         \\
\midrule
\IargName{tkzTabTan}{Début}&|no default|&rang de l'origine de la flèche   \\
\IargName{tkzTabTan}{Fin}&|no default|& rang de l'extrémité de la flèche   \\
\IargName{tkzTabTan}{Position}  & |no default|  & rang de l'antécédent     \\
\IargName{tkzTabTan}{Image}  & |no default|  & valeur de l'image     \\
\bottomrule
\end{tabular}

\medskip
\noindent\emph{Il s'agit de savoir sur quelle flèche, on va positionner la tangente. \tkzname{Début} et \tkzname{Fin} sont les rangs des valeurs qui déterminent les extrémités de la flèche. \tkzname{Position} est le rang de la valeur qui correspond à la tangente. \tkzname{Image} est la valeur que l'on peut joindre à la tangente (ordonnée du point de contact).}

\medskip
\begin{tabular}{lllc}
\toprule
\texttt{options}   & \texttt{défaut}    & \texttt{définition}                    \\
\midrule
\IoptName{tkzTabTan}{pos}     & |below| & position de la valeur                  \\
\bottomrule
\end{tabular}

\medskip
\noindent\emph{Il existe une option \tkzname{pos} qui permet de positionner cette valeur sous la tangente.\\}

\end{NewMacroBox}


\subsection{Utilisation des arguments}

\subsubsection{Palier}
La flèche débute pour la valeur initiale $0$ donc de rang $1$ et se termine pour $+\infty$, valeur de rang $3$.  La tangente est ici en $x=1$ soit la valeur de rang $2$.

\begin{tkzexample}[vbox]
\begin{tikzpicture}
 \tkzTab[espcl=6]{$x$/1,$f'(x)$ /1, $f$/3}%
 {$0$ , $1$ , $+\infty$}%
 {d , + , 0 , + , }
 {D- / $-\infty$ , R /  , +/ $+\infty$}%
 \tkzTabTan{1}{3}{2}{\scriptsize $2$}
\end{tikzpicture}
\end{tkzexample}  

\subsubsection{Tangente à l'extrémité d'un intervalle}
Dans l'exemple ci-dessous, la flèche débute pour la valeur initiale $0$ donc de rang $1$ et se termine pour $1$, valeur de rang $2$. La tangente est ici en $x=1$ soit la valeur de rang $2$. Il faut remarquer que la macro \tkzcname{tkzTabTan} s'applique à la ligne de variations qui la précède.

La valeur $0$ de l'image de $1$ par $f$ n'est pas indiquée dans \tkzcname{tkzTabVar}. Elle serait sous les flèches représentant la tangente, aussi elle est passée comme argument de \tkzcname{tkzTabTan} avec l'option \tkzname{pos=below}.

\begin{tkzexample}[vbox]
\begin{tikzpicture}
 \tkzTabInit[espcl=6]{$x$ /1,$f'(x)$/1,$f$/2}{$0$,$1$,$+\infty$}%
 \tkzTabLine{t , + , z , - , }%
 \tkzTabVar{-/ $-1$ , +/  , -/$-\infty$ }
 \tkzTabTan[pos=below]{1}{2}{2}{$0$}
\end{tikzpicture}
\end{tkzexample}  

\subsection{Utilisation des options}

\subsubsection{\texttt{\textcolor{red}{pos}} : position de la valeur} 

\begin{tkzexample}[vbox]
\begin{tikzpicture}
 \tkzTabInit[espcl=5]{$x$/1,$f''{x}$/1,$f'(x)$/2,$f(x)$/2}{$0$,$1$,$+\infty$}%
 \tkzTabLine{d,+,0,-,}%
 \tkzTabVar{-/ $-\infty$  ,+/ ,-/$-\infty$}
 \tkzTabTan[pos=below]{1}{2}{2}{$0$}
 \tkzTabVar{+/ $+\infty$ , R/ , -/ $0$}
 \tkzTabTan{1}{3}{2}{$1$}
\end{tikzpicture}
\end{tkzexample}  

\subsubsection{Variations imbriquées} 
\begin{tkzexample}[vbox]
\begin{tikzpicture}
  \tkzTabInit[espcl=3]
  {$x$      /1,
   $f''(x)$ /1,
    $f'$    /3,
     $f$    /3}%
  {$0$ , $\alpha$ , $1$ , $\beta$, $+\infty$ }%
  \tkzTabLine {d , +,  , + , z , - , , - }%
  \tkzTabVar {-/ $-1$  / , R/ ,+/ , R/ , -/ $-\infty$ }
  \tkzTabIma[draw]{1}{3}{2}{0}
  \tkzTabIma[draw]{3}{5}{4}{0}
  \tkzTabTan[pos]{1}{3}{3}{$2$}
  \tkzTabVar{+/ $+\infty$ , - / , R/,+/  , -/ $0$ }
  \tkzTabTan[]{1}{2}{2}{$1$}
  \tkzTabTan[pos=below]{2}{4}{4}{$2$}
\end{tikzpicture}
\end{tkzexample}  


\subsection{Définition de \textcolor{red}{tkzTabTanFrom}}

\begin{NewMacroBox}{tkzTabTanFrom}{\oarg{local options}\{Début\}\{Fin\}\{Position\}\{Image\}}

\begin{tabular}{lllc}
\toprule
\texttt{arguments}   & \texttt{défaut}    & \texttt{définition}         \\
\midrule
\IargName{tkzTabTanFrom}{Début} & |no default|  & rang de l'origine de la flèche       \\
\IargName{tkzTabTanFrom}{Fin} & |no default|  & rang de l'extrémité de la flèche     \\
\IargName{tkzTabTanFrom}{Position} & |no default|  & nom d'un point        \\
\IargName{tkzTabTanFrom}{Image} & |no default|  & valeur de l'image        \\
\bottomrule
\end{tabular}

\medskip
\noindent\emph{La position est donnée  par le nom d'un point ou d'un node.}

\medskip
\begin{tabular}{lllc}
\toprule
\texttt{options}   & \texttt{défaut}    & \texttt{définition}       \\
\midrule
\IoptName{tkzTabTan}{pos}     & |below| & position de la valeur      \\
\bottomrule
\end{tabular}


\end{NewMacroBox}
\subsection{Le nom est défini par le tableau}
Le nom du node qui correspond à $\alpha$ est ici \tkzname{N21} (antécédent de rang 2, premier filet sous la valeur.)
\begin{tkzexample}[vbox,small]
\begin{tikzpicture}
\tkzTabInit[ espcl=6]
{ $x$              /1,
  $f'(x)$          /1,
  $f$              /3}
{ $0$ ,  $\alpha$ , $+\infty$ }%
\tkzTabLine { , ,+, , }%
\tkzTabVar{-/ $-1$ , R , +/ $+1$ /}%
\tkzTabTanFrom[pos=below]{1}{3}{N21}{$0$}
\end{tikzpicture}
\end{tkzexample}  

\subsection{Le nom est donné par l'utilisateur avec l'option \texttt{\textcolor{red}{remember}}}

\begin{tkzexample}[vbox,small]
\begin{tikzpicture}
\tkzTabInit[ espcl=4]
{ $x$              /1,
  $f''(x)$         /1,
  $f'$          /2,
  Signe de $f'(x)$ /2,
  $f$           /3}
{ $0$ , $1$ , $\alpha$ , $+\infty$ }%
\tkzTabLine {d,+,0,-, ,- }%
\tkzTabVar
{-/ $1$       ,
 +/           ,
 R/           ,
 -/ $-\infty$ }%
\tkzTabTan[pos,remember=v1]{1}{2}{2}{$2$}%
\tkzTabVal[remember=v2]{2}{4}{0.5}{}{0}%
\tkzTabLine { ,, +,, z,- }%
\tkzTabVar
{-/ $-\infty$ ,
 R/           ,
 +/           ,
 -/ $0$       }
\tkzTabImaFrom[]{1}{3}{v1}{0}%
\tkzTabImaFrom[]{3}{4}{v2}{}%
\tkzTabTanFrom[pos=below]{3}{4}{v2}{$1$}
\end{tikzpicture}
\end{tkzexample}  


\endinput
%!TEX root = /Users/ego/Boulot/TKZ/tkz-tab/doc/TKZdoc-tab-main.tex 
% 20 / 02 /2009 v1.00c TKZdoc-tab-slope
\section{Nombres dérivés : \addbs{tkzTabSlope}}

\begin{NewMacroBox}{tkzTabSlope}{\{Liste\}}

\begin{tabular}{lllc}
\toprule
\texttt{arguments}   & \texttt{défaut}    & \texttt{définition}                            \\
\midrule
\IargName{tkzTabSlope}{Liste}     & |no default|  & $i$/eg($i$)/ed($i$)       \\
\bottomrule
\end{tabular}

\medskip
\noindent\emph{$i$ est compris entre $1$ et $n$, $n$ étant le nombre de valeurs de la première ligne. 
Cette macro permet de personnaliser les signes d'une fonction dérivée en indiquant par exemples des limites, les valeurs  d'une dérivée à droite, à gauche. $i$ est le rang de l'antécédent qui correspond à la valeur de la dérivée, \tkzname{eg} et \tkzname{ed} sont les expressions que l'on veut placer soit à gauche et soit à droite.}
\end{NewMacroBox}

\subsection{Ajout de nombres dérivés}
\Iaccent{nombres deriv}{nombres dérivés}
Étude de la fonction $f~:~ x \longmapsto \sqrt {x(x-1)^2}$ sur $[0~;~4]$

\begin{tkzexample}[vbox,small]
\begin{tikzpicture}
\tkzTabInit[lgt=3]%
    {$x$/1,%
     Signe\\ de $f'(x)$ /1,%
     Variations\\ de\\ $\sqrt {x(x-1)^2}$ /4}%
    {$0$ , $\dfrac{1}{3}$ , $1$ , $4$}%
\tkzTabLine{d ,+, 0 ,-, d ,+, }
\tkzTabSlope{1//+\infty,3/-1 /+1}
\tkzTabVar %
 {-   /     $0$                  ,
  +   /    $\dfrac{2\sqrt3}{9}$  ,
  -   /     $0$                  ,
  +   /    $6$    }
\end{tikzpicture}
\end{tkzexample}  

\endinput
%!TEX root = /Users/ego/Boulot/TKZ/tkz-tab/doc/TKZdoc-tab-main.tex

% $Id$  
%  TKZdoc-tab-style
%  v 1.0c
%  Created by Alain Matthes on 2010-02-23.
%  Copyright (c) 2010 __Collège Sévigné__. All rights reserved.
%
\section{Utilisation des styles}
\subsection{Définition de \tkzcname{tkzTabSetup}} 

Le plus simple est d'utiliser la macro  \tkzcname{tkzTabSetup}. Celle-ci permet de modifier les styles principaux.

\begin{NewMacroBox}{tkzTabSetup}{\oarg{local options}}

\begin{tabular}{llc}
\toprule
\texttt{arguments}   & \texttt{défaut}    & \texttt{définition}                 \\
\midrule
\IargName{tkzTabSetup}{doubledistance} & |1pt| & écart double barre         \\
\IargName{tkzTabSetup}{doublecolor}  & |white|  & couleur centrale dans la double barre         \\
\IargName{tkzTabSetup}{lw}  & |0.4pt| & épaisseur d'un trait                \\
\IargName{tkzTabSetup}{color}  & |black| & couleur d'un trait               \\
\midrule
\IargName{tkzTabSetup}{tstyle}  & |dotted|  & style des traits verticaux        \\
\IargName{tkzTabSetup}{tcolor  } & |black| & couleur des  traits verticaux       \\
\IargName{tkzTabSetup}{tanarrowstyle}&|latex'|&style d'une flèche pour une tangente  \\
\IargName{tkzTabSetup}{tanstyle}& |->| & style  d'une tangente              \\
\IargName{tkzTabSetup}{tancolor}& |black| & couleur d'une tangente          \\
\IargName{tkzTabSetup}{tanwidth}& |0.4pt|& épaisseur d'une tangente         \\
\IargName{tkzTabSetup}{fromarrowstyle}&|latex'|&style d'une flèche antécédent -> image   \\
\IargName{tkzTabSetup}{fromstyle }& |->|  & style antécédent -> image              \\
\IargName{tkzTabSetup}{fromcolor }& |black|  & couleur antécédent -> image         \\
\IargName{tkzTabSetup}{fromwidth }& |0.4pt|  & épaisseur  antécédent -> image      \\
\IargName{tkzTabSetup}{hcolor   } & |gray|  & couleur d'une zone interdite        \\ 
\IargName{tkzTabSetup}{hopacity } & |0.4|  & transparence de la couleur d'une zone interdite         \\  
\IargName{tkzTabSetup}{crosslines}& |false|  & booléen true hachure la zone interdite  \\  
\IargName{tkzTabSetup}{arrowcolor}& |black|  & couleur d'une flèche de variation     \\  
\IargName{tkzTabSetup}{arrowstyle}& |latex'| & style d'une flèche de variation     \\  
\IargName{tkzTabSetup}{arrowlinewidth}&|0.4pt|&épaisseur d'une flèche de variation  \\  
\bottomrule           
\end{tabular}

\medskip               
\noindent\emph{Cette macro s'utilise dès le début. Les épaisseurs sont en générale donnée en \tkzname{pt}, la valeur par défaut est la plus fréquente. }

\end{NewMacroBox}

\subsubsection{Utilisation de \tkzname{doubledistance} et  \tkzname{hcolor}}

\begin{center}
\begin{tikzpicture}
   \tkzTabColors[backgroundcolor=fondpaille,%
             color=Maroon]
  \tkzTabSetup[doubledistance = 2pt] 
  \tkzTabInit[lgt=2,espcl=1] 
  {$x$         /1, $x^2-3x+2$   /1, $\ln (x^2-1)$  /1,  $E(x)$       /1}% 
  {$-\infty$ ,$-\sqrt{2}$, $-1$ , $1$ ,$\sqrt{2}$ , $2$ , $+\infty$}% 
  \tkzTabLine{ , + , t , + , t , + , z , - , t , - , z , + , }
  \tkzTabLine{ , + , z , - , d , h , d , - , z , + , t , + , }
  \tkzTabLine{ , + , z , - , d , h , d , +  ,z , - , z , + , }
\end{tikzpicture}    
\end{center}

  
\begin{tkzexample}[code only,small]
\begin{tikzpicture}
 \tkzTabColors[backgroundcolor=fondpaille,%
           color=Maroon]   
\tkzTabSetup[doubledistance = 2pt] 
\tkzTabInit[lgt=2,espcl=1] 
{$x$         /1, $x^2-3x+2$   /1, $\ln (x^2-1)$  /1,  $E(x)$       /1}% 
{$-\infty$ ,$-\sqrt{2}$, $-1$ , $1$ ,$\sqrt{2}$ , $2$ , $+\infty$}% 
\tkzTabLine{ , + , t , + , t , + , z , - , t , - , z , + , }
\tkzTabLine{ , + , z , - , d , h , d , - , z , + , t , + , }
\tkzTabLine{ , + , z , - , d , h , d , +  ,z , - , z , + , }
\end{tikzpicture} 
\end{tkzexample}     

\subsubsection{Utilisation de \tkzname{fromcolor} et  \tkzname{tancolor}}

\begin{center}
	\begin{tikzpicture}
	 \tkzTabSetup[fromcolor        = red, tancolor         = blue,,backgroundcolor=fondpaille,%
	             color=Maroon] 
	 \tkzTabInit[espcl=4]
	    { $x$          /1,  $f''(x)$     /1,  $f'$      /3, $f$       /4}%
	    { $0$      , $1$ ,     $\alpha$,$+\infty$ }%
	 \tkzTabLine {d    , + ,   z  , -  ,      , -  }%
	 \tkzTabVar 
	   {- / $1$       /,  + /           /,  R /           /, - / $-\infty$ /}

	 \tkzTabVal[draw]{2}{4}{0.5}{}{0}
	 \tkzTabIma[draw]{2}{4}{3}{$0$}
	 \tkzTabTan[pos]{1}{2}{2}{$2$}
	 \tkzTabVar   
	    {- / $-\infty$ ,  R /  , + / $1$ , - / $0$       }
	  \tkzTabIma[draw]{1}{3}{2}{$0$}
	\end{tikzpicture} 
\end{center}


\begin{tkzexample}[code only,small]  
\begin{tikzpicture}
 \tkzTabSetup[fromcolor        = red, tancolor         = blue,,backgroundcolor=fondpaille,%
             color=Maroon] 
 \tkzTabInit[espcl=4]
    { $x$          /1,  $f''(x)$     /1,  $f'$      /3, $f$       /4}%
    { $0$      , $1$ ,     $\alpha$,$+\infty$ }%
 \tkzTabLine {d    , + ,   z  , -  ,      , -  }%
 \tkzTabVar 
   {- / $1$       /,  + /           /,  R /           /, - / $-\infty$ /}

 \tkzTabVal[draw]{2}{4}{0.5}{}{0}
 \tkzTabIma[draw]{2}{4}{3}{$0$}
 \tkzTabTan[pos]{1}{2}{2}{$2$}
 \tkzTabVar   
    {- / $-\infty$ ,  R /  , + / $1$ , - / $0$       }
  \tkzTabIma[draw]{1}{3}{2}{$0$}
\end{tikzpicture} 
\end{tkzexample}
  
\subsection{Utilisation de \tkzcname{tikzset} pour modifier les styles}

Voici la liste des styles qui sont utilisés et leurs définitions.

\begin{tabular}{ll}
\toprule
\tkzname{node style} & style des nodes utilisé pour les valeurs placées dans le tableau  \\
\tkzname{low left} &  valeur située en bas et à gauche d'un trait vertical  \\
\tkzname{low right} & valeur située en bas et à droite d'un trait vertical\\ 
\tkzname{hight left} & valeur située en haut et à gauche d'un trait vertical\\ 
\tkzname{hight right}& valeur située en haut et à droite d'un trait vertical\\ 
\tkzname{low} & valeur située en bas d'un trait vertical \\ 
\tkzname{hight } & valeur située en haut d'un trait vertical \\ 
\tkzname{on double} & couleur du fond sous une double barre \\ 
\tkzname{tan style} & style pour une tangente\\
\tkzname{arrow style} & style pour les flèches des variations\\
\tkzname{from style} & style pour la ligne allant d'un antécédent à une image\\
\tkzname{h style} & style pour une zone interdite\\
\tkzname{double style} & style pour une double barre\\
\tkzname{t style} & style pour un trait vertical\\  
\bottomrule           
\end{tabular}          


Les valeurs par défaut utilisées sont les suivantes :
\begin{tkzexample}[code only, small]
\def\tkzTabDefaultWritingColor{black}
\def\tkzTabDefaultBackgroundColor{white}
\def\tkzTabDefaultLineWidth{0.4pt}
\def\tkzTabDefaultArrowStyle{latex'}
\def\tkzTabDefaultSep{2pt} 
\end{tkzexample}

les principaux styles par défaut sont :

\begin{tkzexample}[code only, small]   
\tikzset{node style/.style  = {inner sep   =  \tkzTabDefaultSep,
                               outer sep   =  \tkzTabDefaultSep,
                               fill        =  \tkzTabDefaultBackgroundColor}}
\tikzset{tan style/.style   = {>           =  \tkzTabDefaultArrowStyle,
                               ->,
                               color       =  \tkzTabDefaultBackgroundColor}}
\tikzset{arrow style/.style = {\tkzTabDefaultWritingColor,
                               ->,
                               >            = \tkzTabDefaultArrowStyle,
                               shorten >    = \tkzTabDefaultSep,
                               shorten <    = \tkzTabDefaultSep}}
\tikzset{from style/.style   = {shorten >   = \tkzTabDefaultSep,
                                shorten <   = \tkzTabDefaultSep,
                                line width  = \tkzTabDefaultLineWidth,
                                >           = \tkzTabDefaultArrowStyle,
                                ->,
                                draw        = \tkzTabDefaultWritingColor,
                                dotted}}
\tikzset{t style/.style = {style  = dotted,
                           draw   = \tkzTabDefaultWritingColor}}
\tikzset{h style/.style = {pattern       = north west lines,
                           pattern color = \tkzTabDefaultWritingColor}}
\tikzset{on double/.style   = {fill      =  \tkzTabDefaultBackgroundColor}}
\tikzset{double style/.append style = {%
         draw            = \tkzTabDefaultWritingColor,
         double          =  \tkzTabDefaultBackgroundColor}}
\end{tkzexample}
 
Les couleurs de fond pour les différentes sont définies par les styles :
\begin{tkzexample}[code only, small]   
\tikzset{fondC/.style={fill = \tkzTabDefaultBackgroundColor}}
\tikzset{fondL/.style={fill = \tkzTabDefaultBackgroundColor}}
\tikzset{fondT/.style={fill = \tkzTabDefaultBackgroundColor}}
\tikzset{fondV/.style={fill = \tkzTabDefaultBackgroundColor}}
\end{tkzexample}
Enfin les approches des valeurs par les flèches sont :
\begin{tkzexample}[code only, small]    
\tikzset{low left/.style    = {above left  =  \tkzTabDefaultSep}}
\tikzset{low right/.style   = {above right =  \tkzTabDefaultSep}}
\tikzset{high right/.style  = {below right =  \tkzTabDefaultSep}}
\tikzset{high left/.style   = {below left  =  \tkzTabDefaultSep}}
\tikzset{low/.style         = {above       =  \tkzTabDefaultSep}}
\tikzset{high/.style        = {below       =  \tkzTabDefaultSep}}
\end{tkzexample}   

\subsubsection{Utilisation de \tkzcname{tikzset} et \tkzname{h style}}


\begin{tkzexample}[latex=7cm,small]  
\begin{tikzpicture}
  \tkzTabColors[backgroundcolor=fondpaille,%
                color=Maroon]
  \tkzTabSetup[doubledistance = 2pt]  
  \tikzset{h style/.style = {fill=red!50}}
  \tkzTabInit[color,espcl=1.5]%
    {$x$ / 1,$g(x)$ / 1}%
    {$0$,$1$,$2$,$3$}%
  \tkzTabLine{z,+,d,h,d,-,t}
\end{tikzpicture} 
\end{tkzexample}

\subsubsection{Utilisation de \tkzcname{tikzset} et \tkzname{h style}}  

\begin{tkzexample}[latex=7cm,small]  
\begin{tikzpicture}
\tikzset{h style/.append style = {%
         pattern=north east lines}}
\tkzTabInit[color,espcl=1.5]%
    {$x$ / 1,$g(x)$ / 1}%
    {$0$,$1$,$2$,$3$}%
\tkzTabLine{z,+,,h,d,-,t}
\end{tikzpicture}  
\end{tkzexample}

\subsubsection{Utilisation de \tkzcname{tikzset} et \tkzname{arrow style}} 

\begin{center}
	\begin{tkzexample}[vbox,small]  
	  %\newcommand*{\E}{\ensuremath{\mathrm{e}}}     
	\begin{tikzpicture}
	   
	\tikzset{arrow style/.append style = {red,shorten >=6pt,shorten <=6pt}} 
	\tkzTabInit[espcl=5]{$x$ /1, $\ln x +1$ /1.5, $x \ln x$ /2}% 
	{$0$ ,$1/\E$ , $+\infty$}% 
	\tkzTabLine{d,-,z,+,} 
	\tkzTabVar% 
	{ D+/ / $0$ ,% 
	-/ \colorbox{black}{\textcolor{white}{$\dfrac{-1}{e}$}}/ ,% 
	+/ $+\infty$ / }% 
	\end{tikzpicture}  
	\end{tkzexample}
\end{center}


\subsubsection{Utilisation de \tkzcname{tikzset} et \tkzcname{tkzTabSetup}} 

 On remarquera la dernière utilisation de   \tkzcname{tkzTabSetup} qui remet les valeurs par défaut.
 
\begin{center}
	\begin{tkzexample}[vbox,small]     
	% \newcommand*{\va}{\colorbox{red!50}    {$\scriptscriptstyle V_a$}}
	% \newcommand*{\vb}{\colorbox{blue!50}   {$\scriptscriptstyle V_b$}}
	% \newcommand*{\vc}{\colorbox{gray!50}   {$\scriptscriptstyle V_c$}} 
	% \newcommand*{\vd}{\colorbox{magenta!50}{$\scriptscriptstyle V_d$}} 
	% \newcommand*{\ve}{\colorbox{orange!50} {$\scriptscriptstyle V_e$}}  
	 
	\begin{tikzpicture} 
	\tkzTabSetup[fromcolor      =  red,
	            fromstyle       = dashed,
	            fromwidth       = 1pt,
	            fromarrowstyle  = stealth',
	            arrowcolor      = green ]    
	  \tkzTabInit[lgt=1.5,espcl=5]{ $x$/.7,$f''(x)$/.7,$f'$/3,$f$/3 }%
	             {  $a$    , $d$    ,$e$}
	  \tkzTabLine{  z,+    ,z,-     ,z  } 
	  \tkzTabVar  {-/\va  ,+/\vd   , -/  \ve}
	  \tkzTabVal[draw,remember=vb]{1}{2}{0.333}{b}{$1$} 
	  \tikzset{from style/.append style = {draw      =  blue}}
	  \tkzTabVal[draw,remember=vc]{1}{2}{0.666}{c}{$2$}
	  \tkzTabVar{-/$-\infty$  ,R/   , +/  $+\infty$}
	  \tkzTabSetup
	  \tkzTabVal[draw]{1}{3}{0.5}{}{$0$}
	  \draw[opacity=0.5,fill=red!40]  (vb) circle(2ex);
	  \draw[opacity=0.5,fill=blue!40] (vc) circle(2ex);
	\end{tikzpicture}  
	\end{tkzexample}
\end{center}
    

 
\endinput




%!TEX root = /Users/ego/Boulot/TKZ/tkz-tab/doc/TKZdoc-tab-main.tex    
% $Id$  
% v1.0c TKZdoc-tab-adapt
%
%  Created by Alain Matthes on 2010-02-23.
%  Copyright (c) 2010 __Collège Sévigné__. All rights reserved.
\section{Personnalisation des tableaux}\label{pers}

\subsection{\texttt{\textcolor{red}{help}} :  option commune aux principales macros}

\subsubsection{\texttt{\textcolor{red}{help}} :  option de \addbs{tkzTabInit}}
Cette option permet de connaître la structure d'un tableau. \tkzname{deltacl=1} permet d'espacer un peu les points et les labels

\begin{tkzexample}[small]
	\begin{tikzpicture}
	 \tkzTabInit[deltacl=1,espcl=8,help]%
	 {$x$/1,Signe\\ de $\dfrac{1}{x}$/1.5/1.5,Variation\\ de $\ln$/2}%
	 {$0$,$+\infty$}%
	 \end{tikzpicture}
\end{tkzexample}

 
\subsubsection{\texttt{\textcolor{red}{help}} :  option de \addbs{tkzTabLine}}
Afin de mieux voir les labels il est préférable de pas employer l'option \tkzname{help} en même temps sur toutes les macros. 
\begin{tkzexample}[small]
	\begin{tikzpicture}
	 \tkzTabInit[deltacl=1,espcl=8]%
	 {$x$/1,Signe\\ de $\dfrac{1}{x}$/1.5/1.5,Variation\\ de $\ln$/2}%
	 {$0$,$+\infty$}%
	 \tkzTabLine[help]{,,}%
%  \tkzTabVar {D-/ $-\infty$,  +/$+\infty$  }
	 \end{tikzpicture}
\end{tkzexample}

\subsubsection{\texttt{\textcolor{red}{help}} :  option de \addbs{tkzTabVar}} 
Cette option montre les nodes qui sont utilisés pour le tracé des flèches de variations. Afin de ne pas multiplier les labels de nodes, seuls les nodes utilisés ont été nommés. Une flèche débute par un node  nommé \tkzname{FR} (right = droite du node) et se termine par un node nommé \tkzname{FL} (left gauche du node)
\begin{tkzexample}[small]
	\begin{tikzpicture}
	 \tkzTabInit[deltacl=1,espcl=8]%
	 {$x$/1,Signe\\ de $\dfrac{1}{x}$/1.5/1.5,Variation\\ de $\ln$/1.5}%
	 {$0$,$+\infty$}%
	 \tkzTabLine{d,+,}%
   \tkzTabVar [help]{D-/ $-\infty$,  +/$+\infty$  }
	 \end{tikzpicture}
\end{tkzexample}

Voici un exemple plus complexe

\begin{tkzexample}[small]
	\begin{tikzpicture} 
	\tkzTabInit 
	{$x$ /1, 
	$\dfrac{-1}{x^2}\ {\E}^{\left(\dfrac{1}{x}\right)}$ /1.5, 
	${\E}^{\left(\dfrac{1}{x}\right)}$ /2}% 
  {$-\infty$ ,$0$ , $+\infty$}% 
		\tkzTabLine{t,-,d,-,t} 
		\tkzTabVar[help]{ + / $1$ ,-CD+ / $0$ / $+\infty$ , - / $1$ }% 
	\end{tikzpicture}
\end{tkzexample}

ce qui donne
\begin{tkzexample}[small]
	\begin{tikzpicture} 
	\tkzTabInit 
	{$x$ /1, 
	$\dfrac{-1}{x^2}\ {\E}^{\left(\dfrac{1}{x}\right)}$ /1.5, 
	${\E}^{\left(\dfrac{1}{x}\right)}$ /2}% 
  {$-\infty$ ,$0$ , $+\infty$}% 
		\tkzTabLine{t,-,d,-,t} 
		\tkzTabVar{ + / $1$ ,-CD+ / $0$ / $+\infty$ , - / $1$ }% 
	\end{tikzpicture}
\end{tkzexample} 

La connaissance de tous ces points et nodes permet de personnaliser les tableaux. Quelques explications supplémentaires sont données dans le paragraphe suivant.

\subsection{Les structures}
\subsubsection{La structure principale}

La macro \tkzname{tkzTabInit} définit  les principaux \tkzname{nodes}. Ce sont les arguments de cette macro qui déterminent le nombre de nodes.

Par exemple, si le tableau comporte 3 lignes alors les nodes $T00$, $T01$, $T02$, $T10$, $T11$, $T12$, $T03$, $T13$ et $T23$ sont créés, ainsi que $F0$, $F1$ et $F2$.  $Tij$ représente un point de la colonne $i$ et de la ligne $j$. Pourquoi cet ordre ? je n'en sais rien

\begin{tkzexample}[vbox,small]
	\begin{tikzpicture}
	\tkzTabInit[color=false,espcl=4,lgt=3]{%
	\colorbox{red}{\textcolor{white}{$\scriptscriptstyle F0$}} / 1,%
	\colorbox{red}{\textcolor{white}{$\scriptscriptstyle F1$}} / 1,%
	\colorbox{red}{\textcolor{white}{$\scriptscriptstyle F2$}} / 1}%
	{ , }%
	\foreach \ligne in {0,...,3}{%
	   \foreach \colonne in {0,1,2}{%
	      \draw[fill=blue] (T\colonne\ligne) circle(2pt) ;}}
	\draw (T00) node[above right=4pt] {\scriptsize T00};
	\draw (T01) node[above right=4pt] {\scriptsize T01};
	\draw (T02) node[above right=4pt] {\scriptsize T02};
	\draw (T03) node[above right=4pt] {\scriptsize T03};
	\draw (T20) node[above right=4pt] {\scriptsize T20};
	\draw (T21) node[above right=4pt] {\scriptsize T21};
	\draw (T22) node[above right=4pt] {\scriptsize T22};
	\draw (T23) node[above right=4pt] {\scriptsize T23};
	\draw (T10) node[above right=4pt] {\scriptsize T10};
	\draw (T13) node[above right=4pt] {\scriptsize T13};
	\draw (T11) node[above right=3pt] {\scriptsize T11};
	\draw (T12) node[above right=3pt] {\scriptsize T12};
	\tikzset{bluesty/.style={fill=blue,<-,>=latex,shorten <=2pt}}
	\draw[bluesty] (T20) -- +(2,0) node[right,blue]{ligne $0$};
	\draw[bluesty] (T21) -- +(2,0) node[right,blue]{ligne $1$};
	\draw[bluesty] (T22) -- +(2,0) node[right,blue]{ligne $2$};
	\draw[bluesty] (T23) -- +(2,0) node[right,blue]{ligne $3$};
	\draw[bluesty] (T03) -- +(0,-2) node[midway,above,sloped,blue]{colonne $0$};
	\draw[bluesty] (T13) -- +(0,-2) node[midway,above,sloped,blue]{colonne $1$};
	\draw[bluesty] (T23) -- +(0,-2) node[midway,above,sloped,blue]{colonne $2$};
	\end{tikzpicture}
\end{tkzexample}


Ainsi la structure principale de ce tableau possède exactement trois filets verticaux  et quatre horizontaux. Soient \tkzname{12} points principaux définis par les intersections et trois nodes  $F0$, $F1$ et $F2$.

\subsubsection{La structure interne}
J'appelle structure interne, l'ensemble des points et nodes qui vont être définis par les antécédents dans la partie droite du tableau.  Le second argument de la macro \tkzname{tkzTabInit} définit cette structure. Cet argument donne le nombre de labels (antécédents) qui vont être placés sur la première ligne et qui vont être les repères pour les lignes de signes et de variations.


\begin{tkzexample}[vbox,small]
	\begin{tikzpicture}
	\tkzTabInit[color=false,espcl=4,lgt=3]{%
	\colorbox{red} {\textcolor{white}{$\scriptscriptstyle F0$}} / 1,
	\colorbox{red} {\textcolor{white}{$\scriptscriptstyle F1$}} / 1,
	\colorbox{red} {\textcolor{white}{$\scriptscriptstyle F2$}} / 1}{%
	\colorbox{blue}{\textcolor{white}{$\scriptscriptstyle L1$}},
	\colorbox{blue}{\textcolor{white}{$\scriptscriptstyle L2$}},
	\colorbox{blue}{\textcolor{white}{$\scriptscriptstyle L3$}}}%
	\foreach \ligne in {0,...,3}{%
	  \foreach \colonne in {0,1,2}{%
	    \draw[fill=blue] (T\colonne\ligne) circle(2pt) ;}}
	\foreach \colonne in {1,2,3}{%
	  \draw[fill=red] (N\colonne 0) circle(2pt)%
	       node[above,red] {\scriptsize N{\colonne 0}};}
	\foreach \ligne in {1,2,3}{%
	  \foreach \colonne in {1,2,3}{%
	    \draw[fill=red] (N\colonne\ligne) circle(2pt)%
	       node[above,red] {\scriptsize N\colonne\ligne};}}
	\foreach \ligne in {0,1,2,3}{%
	  \foreach \colonne in {1,2}{%
	    \draw[fill=green] (M\colonne\ligne) circle(2pt)
	       node[below right,green] {\scriptsize M\colonne\ligne};}}
	\tikzset{redsty/.style={fill=red,<-,>=latex,shorten <=2pt}}
	\draw[redsty] (T20) -- +(2,0) node[right,red]{ligne $0$};
	\draw[redsty] (T21) -- +(2,0) node[right,red]{ligne $1$};
	\draw[redsty] (T22) -- +(2,0) node[right,red]{ligne $2$};
	\draw[redsty] (T23) -- +(2,0) node[right,red]{ligne $3$};
	\draw[redsty] (N13) -- +(0,-2) node[midway,above,sloped,red]{colonne $1$};
	\draw[redsty] (N23) -- +(0,-2) node[midway,above,sloped,red]{colonne $2$};
	\draw[redsty] (N33) -- +(0,-2) node[midway,above,sloped,red]{colonne $3$};
	\end{tikzpicture}
\end{tkzexample}


\subsubsection{La structure secondaire}
Les points $Zij$, $Sij$ sont définis à partir de la structure interne (voir le tableau précédent) mais seulement avec l'usage de la macro \tkzcname{tkzTabLine}. Les points $FRij$ et $FLij$ eux sont définis avec l'usage de la macro \tkzcname{tkzTabVar}

\begin{tkzexample}[small]
	\begin{tikzpicture} 
	\tkzTabInit 
	{$x$ /1, 
	$\dfrac{-1}{x^2}\ {\E}^{\left(\dfrac{1}{x}\right)}$ /1.5, 
	${\E}^{\left(\dfrac{1}{x}\right)}$ /2}% 
  {$-\infty$ ,$0$ , $+\infty$}% 
		\tkzTabLine[help]{t , - , d , - , t} 
		\tkzTabVar[help]{ + / $1$ ,-CD+ / $0$ / $+\infty$ , - / $1$ }% 
	\end{tikzpicture}
\end{tkzexample}
\subsubsection{Conclusion}
\begin{tikzpicture}

\tkzTabInit[color=false,espcl=4,lgt=3,deltacl=1]{%
\colorbox{red}{\textcolor{white} {$\scriptscriptstyle F0$}} / 1,
\colorbox{red}{\textcolor{white} {$\scriptscriptstyle F1$}} / 2,
\colorbox{red}{\textcolor{white} {$\scriptscriptstyle F2$}} / 2}{%
\colorbox{blue}{\textcolor{white}{$\scriptscriptstyle L1$}} ,
\colorbox{blue}{\textcolor{white}{$\scriptscriptstyle L2$}} ,
\colorbox{blue}{\textcolor{white}{$\scriptscriptstyle L3$}} }
\foreach \ligne in {0,...,3}{%
   \foreach \colonne in {0,1,2}{%
   \draw[fill=blue] (T\colonne\ligne) circle(2pt) ;}}
\draw (T00) node[left      = 4pt] {\scriptsize T00};
\draw (T01) node[left      = 4pt] {\scriptsize T01};
\draw (T02) node[left      = 4pt] {\scriptsize T02};
\draw (T03) node[left      = 4pt] {\scriptsize T03};
\draw (T20) node[right     = 4pt] {\scriptsize T20};
\draw (T21) node[right     = 4pt] {\scriptsize T21};
\draw (T22) node[right     = 4pt] {\scriptsize T22};
\draw (T23) node[right     = 4pt] {\scriptsize T23};
\draw (T10) node[above     = 4pt] {\scriptsize T10};
\draw (T13) node[below     = 4pt] {\scriptsize T13};
\draw (T11) node[above left= 3pt] {\scriptsize T11};
\draw (T12) node[above left= 3pt] {\scriptsize T12};
 \foreach \colonne in {1,2,3}
  {\draw[fill=red] (N\colonne 0) circle(2pt)%
       node[above right] {\scriptsize N{\colonne 0}};}
\foreach \ligne in {1,2,3}
{ \foreach \colonne in {1,2,3}
  {\draw[fill=red] (N\colonne\ligne) circle(2pt)%
       node[below right] {\scriptsize N\colonne\ligne};}}
\foreach \ligne in {0,1,2,3}
{ \foreach \colonne in {1,2}
    {\draw[fill=green] (M\colonne\ligne) circle(2pt)%
       node[below right] {\scriptsize M\colonne\ligne};}}
\foreach \colonne in {1,2}
   {\path (M\colonne 1) to (M\colonne 2)%
       node[midway](S\colonne 1) {};
    \draw[fill=yellow] (S\colonne 1) circle(2pt)%
       node[below right] {\scriptsize S\colonne 1};}
\foreach \colonne in {1,2,3}
  {\path (N\colonne 1) to (N\colonne 2)%
      node[midway](Z\colonne 1) {};
   \draw[fill=yellow] (Z\colonne 1) circle(2pt)%
     node[below right] {\scriptsize Z\colonne 1};}
\end{tikzpicture}

\bigskip
\begin{tabular}{ccllc}
\toprule
type &  notation &   repère & conditions & utilisation\\
\midrule
\tikz \draw[fill=red] (0,0) rectangle (0.3,0.3) node(X) {}; &   Fj &  ligne & $0\leq j\leq p$ & expressions,formules  \\
\midrule
\tikz \draw[fill=blue] (0,0) rectangle (0.3,0.3)node(X) {}; &   Li & colonne& $1\leq i\leq n$  & valeurs significatives pour les variations \\
\midrule
\tikz \draw[fill=blue] circle (2pt)node(X) {}; &%
      Tij &  colonne& $0\leq i\leq 2$ ;& structure principale du tableau\\
&               & ligne  &$0\leq j\leq p$ & il existe une ligne $0$ et une colonne $0$\\
\midrule
\tikz \draw[fill=green] circle (2pt)node(X) {}; & Nij &colonne &$1\leq i\leq n$  & structure interne du tableau \\
& & ligne &$0\leq j\leq p$ &  \\
\midrule
\tikz \draw[fill=green] circle (2pt)node(X) {}; &   Mij &colonne &$1\leq i\leq n$ &  structure interne du tableau \\
& & ligne &$0\leq j\leq p$ & \\
\midrule
\tikz \draw[fill=yellow] circle (2pt)node(X) {}; &   Sij &colonne &$1\leq i\leq n$ & structure secondaire du tableau  \\
& & ligne &$1\leq j\leq q$ & \\
\midrule
\tikz \draw[fill=yellow] circle (2pt)node(X) {};& Zij &colonne &$1\leq i\leq n$ & structure secondaire du tableau \\
& & ligne &$1\leq j\leq q$ & \\
\midrule
\end{tabular}


\subsection{Ajustement des dimensions}
Nous avons vu précédemment que l'on pouvait modifier certaines dimensions à l'aide de l'emploi d'options. Le code du tableau suivant utilise les structures du tableau

\begin{tkzexample}[small]
	\begin{tikzpicture}
		 \tkzTabInit
		    {$x$ /  1}
		    {$a_1$ ,  $a_2$ , $a_3$}
		\begin{scope}[arstyle/.style={>=latex,#1,<->}] 
		  \draw[arstyle=blue] (N10) to node[above,color=blue]%
			     {\scriptsize $ espcl = 2$ cm} (N20);
			\draw[arstyle=blue] (N20) to node[above,color=blue]%
				   {\scriptsize $ espcl = 2$ cm} (N30);
			\draw[arstyle=red] (T10) to node[above=12pt,color=red]%
			     {\scriptsize $ deltacl = 0,5$ cm} (N10);
			\draw[arstyle=red] (N30) to node[above=12pt,color=red]%
			     {\scriptsize $ deltacl = 0,5$ cm} (T20);
			\draw[arstyle=blue] (T00) to node[above,color=blue]%
			     {\scriptsize $ lgt = 2$ cm} (T10);
		\end{scope}
	\end{tikzpicture}
\end{tkzexample}

\subsubsection{\texttt{\textcolor{red}{scale}} permet d'ajuster la taille d'un tableau}
\index{scale}


\begin{tkzexample}[width=7cm,small]
\begin{tikzpicture}[scale=.8]
  \tkzTabInit[lgt=3]{ $x$ / 1 , $\ln(x)$ /2}
                    { $0$  , $+\infty$ }
\end{tikzpicture}
\end{tkzexample}  

Il est aussi possible d'utiliser  \tkzname{xscale} et \tkzname{yscale}.

\begin{tkzexample}[width=7cm,small]
\begin{tikzpicture}[xscale=.8,yscale=1.5]
  \tkzTabInit[lgt=3]{ $x$ / 1 , $\ln(x)$ /2}
                    { $0$  , $+\infty$ }
\end{tikzpicture}
\end{tkzexample}  


\subsection{Exemples d'utilisation}
\subsubsection{Une croix sur un tableau}

\begin{tkzexample}[vbox,small]
\begin{tikzpicture}
 \tkzTab{$x$ / 1, $f'(x)$ / 1.5, $f(x)$ / 3}%
     {$-5$ , $-2$ , $1$ , $+\infty$}%
 {d,+,0,-,0,+,}
 { D-/                / $-\infty$ ,%
    +/ $\dfrac{2}{3}$ / ,%
    -/ $0$            / ,%
    +/ $+\infty$      / }%
 \draw[line width=2pt,red] (T00) to (T23);
 \draw[line width=2pt,red] (T03) to (T20);
\end{tikzpicture}
\end{tkzexample}  

\subsubsection{Une croix sur une case}

L'intérêt est de faciliter la personnalisation d'un tableau. Par exemple, si nous souhaitons ajouter un tracé comme une croix dans une case, on peut procéder ainsi :

\begin{tkzexample}[small]
\begin{tikzpicture}
\tkzTabInit%
   {$x$                   /1,
    $x^2-3x+2$            /1,
    $(x-\E)\ln x$   /1}
   {$0$  , $\E$  , $+\infty$}
   \draw[red] (T12) -- (T23);
   \draw[red] (T13) -- (T22);
\end{tikzpicture}
\end{tkzexample}  

\subsubsection{Mise en évidence de signes}
\medskip
On peut ainsi placer des signes sur la seconde ligne qui n'a pas été mise en forme par \tkzname{tkzTabLine} mais en connaissant un peu la programmation à l'aide de \TIKZ.

    \begin{tkzexample}[code only]
      \path (M11)--(M12) node[midway,draw,fill=red!10] {-};
      \path (M31)--(M32) node[midway,draw,fill=blue!10] {+};
    \end{tkzexample}
    
  \begin{tkzexample}[small]
	  \begin{tikzpicture}
	    \tkzTabInit{$x$ / 1, $\dfrac{2x}{x^2-1}$ /1}
	               {$-\infty$ , $-1$ , $1$ , $+\infty$}
	    \path (M11)--(M12) node[midway,draw,fill=red!10] {-};
	    \path (M31)--(M32) node[midway,draw,fill=blue!10] {+};
	  \end{tikzpicture}
  \end{tkzexample}

mais on peut aussi utiliser un node de la structure secondaire pour cela on utilise 
\begin{tkzexample}[]
	\tkzname{tkzTabLine}[help] \end{tkzexample}

\begin{tikzpicture}
  \tkzTabInit{$x$ / 1, $\dfrac{2x}{x^2-1}$ /1}
             {$-\infty$ , $-1$ , $1$ , $+\infty$}
  \tkzTabLine[help]{,,,,,,}
\end{tikzpicture}

Ensuite il reste à créer des nodes
\begin{tkzexample}[code only]
	  \node[draw,fill=red!10] at (S11) {-};
	  \node[draw,fill=red!10] at (S31) {+}; \end{tkzexample}

  
  
\begin{tkzexample}[small]
\begin{tikzpicture}
  \tkzTabInit{$x$ / 1, $\dfrac{2x}{x^2-1}$ /1}
             {$-\infty$ , $-1$ , $1$ , $+\infty$}
  \tkzTabLine{,,,,,,}
  \node[draw,fill=red!10] at (S11) {-};
  \node[draw,fill=red!10] at (S31) {+};
\end{tikzpicture}
\end{tkzexample}
  
\subsubsection{Structure principale : hachurer une zone}
  
On veut par exemple hachurer une zone mais vous ne connaissez pas la notation des nodes. Il suffit de passer \tkzname{help} en option. On obtient ainsi l'emplacement et les noms des nodes.
  
\begin{tkzexample}[code only]
	\begin{tikzpicture}
	\tkzTabInit[help,deltacl=1]{$x$ / 1, $\dfrac{2x}{x^2-1}$ /1,$\ln{(x^2-1)}$/1}
	                 {$-\infty$ , $-1$ , $1$ , $+\infty$}
	\end{tikzpicture}
\end{tkzexample}


\medskip
On peut  hachurer un rectangle par 

    \begin{tkzexample}[code only]
    \pattern[pattern=north west lines] (N21) rectangle (N33);
    \end{tkzexample}
    
    
\begin{tkzexample}[small]
\begin{tikzpicture}
  \tkzTabInit{$x$ / 1 , $\ln{(x^2-1)}$/1}
             {$-\infty$ , $-1$ , $1$ , $+\infty$}
  \pattern[pattern=north west lines] (N21) rectangle (N32);
\end{tikzpicture}
\end{tkzexample}

  

\subsubsection{Mise en évidence de certaines zones}

Afin de mettre en évidence le signe d'une expression du second degré, il est possible de mettre en couleur les parties extérieures. \tkzcname{draw[fill=Red!80,opacity=0.4](N11) rectangle (N22);}. La syntaxe est celle de \TIKZ. Un rectangle est défini par deux sommets opposés. 

\begin{tkzexample}[vbox,small]
\begin{tikzpicture}
 \tkzTabInit[deltacl=1,lgt=3,espcl=2]%
   {$x$ /1,$x^2-3x+2$ /1}%
   {$-\infty$ , $1$ , $2$, $+\infty$}%
 \tkzTabLine {t,+,0,-,0,+,t}
 \draw[fill=Red!80,opacity=0.4](N11) rectangle (N22);
 \draw[fill=Red!80,opacity=0.4](N31) rectangle (N42);
\end{tikzpicture}
\end{tkzexample}  

\subsubsection{Mise en évidence de valeurs}
\begin{tkzexample}[vbox,small]
\begin{tikzpicture}
\tkzTab
{$x$                                                       /1,
$\dfrac{-1}{x^2}\ {\E}^{\left(\dfrac{1}{x}\right)}$ /1.5,
${\E}^{\left(\dfrac{1}{x}\right)}$                  /2}%
{$-\infty$ ,$0$ , $+\infty$}%
{t,-, ,-,t}
{ + /  $1$ , -CD+ / \colorbox{red}{\textcolor{white}{$0$}} / $+\infty$ , - / $1$ }%
 \node[draw,inner sep=2pt,circle,fill=yellow] at (Z21) {$0$} ;
\end{tikzpicture}
\end{tkzexample}  

\subsubsection{Mise en évidence de limites}

\begin{tkzexample}[vbox,small]
\begin{tikzpicture}
 \tkzTabInit[espcl=8]%
 {$x$/1 , Variation\\ de $\ln$/2}%
 {$0$,$+\infty$}%
 \tkzTabVar {D-/ $-\infty$, +/$+\infty$ }
 \draw[opacity=.3,fill=red] (FR11) circle (10pt);
 \draw[opacity=.3,fill=red] (FL21) circle (10pt);
 \end{tikzpicture}
\end{tkzexample}  
 
 
\subsubsection{Décoration}
Il est nécessaire de charger une librairie de \TIKZ\footnote{pgf/tikz version 2.00} qui permet des actions de décoration. \NameLib{decorations.pathreplacing}

\begin{tkzexample}[code only]
	\usetikzlibrary{decorations.pathreplacing}
	\ldots
	\draw[decoration={brace,amplitude=12pt},
	      decorate,line width=2pt,red] (T10) -- (T20);\end{tkzexample}


\begin{tkzexample}[small]
	\begin{tikzpicture}
	\tkzTabInit[lgt=3,espcl=1.5]
	    {$x$                 /1,
	     $x^2-3x+2$          /1,
	     $(x-\E)\ln x$ /1}%
	    {$0$,$1$,$2$,$\E$,$+\infty$}%
	\draw[fill=Orange,opacity=.3] (N10) rectangle (N53.west);
	\draw[decoration={brace,amplitude=12pt},
	      decorate,line width=2pt,red] (T10) -- (T20);
	\end{tikzpicture}
\end{tkzexample}


\subsubsection{Avec de la couleur}
\begin{tkzexample}[vbox,small]
\begin{tikzpicture}
    \tkzTabInit[color,colorC = blue!30,colorL = orange!50,
                colorT = green!30,colorV = red!50,espcl=8]
    {$x$/1,Signe\\ de $\dfrac{1}{x}$ /1.5,Variation\\ de $\ln$ /3}
    {$0$,$+\infty$}%
    \tkzTabLine{d,+,}%
    \tkzTabVar[color=red]{D-/$-\infty$ , +/$+\infty$}
    \tkzTabVal[draw]{1}{2}{0.3}{\textcolor{red}{$\text{1}$}}{\textcolor{blue}{$0$}}
    \tkzTabVal[draw]{1}{2}{0.6}{\textcolor{red}{$\text{\large e}$}}{\textcolor{blue}{$1$}}%
    \draw[fill=gray,opacity=0.6] (T11) rectangle (N13);
\end{tikzpicture}
\end{tkzexample}  


\subsubsection{Écrire dans un tableau}

Aucune restriction au niveau de l'écriture, l'exemple suivant :

\medskip
\begin{tikzpicture}
\tkzTabInit[lgt=5,espcl=3]%
      { $x$               /1,%
        Il est parfois possible d'obtenir les variations d'une fonction sans déterminer sa dérivée    /2,%
        $\ln (x) +x$     /1%
      }%
      { $0$ , $1$ , $+\infty$ }%
\end{tikzpicture}

\begin{tkzexample}[code only]
\begin{tikzpicture}
\tkzTabInit[lgt=3,espcl=4]%
      { $x$                    /1,%
        Il est parfois ...  /2,%
        $\ln (x) +x$           /1%
      }%
      { $0$ , $1$ , $+\infty$ }%
\end{tikzpicture}
\end{tkzexample}  

\subsubsection{Tableau de proportionnalité}

On utilise ici un compteur interne \tkzname{tkz@cnt@pred} du package. l' arrobase \tkzname{@} devient une lettre ordinaire à l'aide des macros \tkzname{makeatletter} et \tkzname{makeatother}. Ce compteur va servir à tracer des filets verticaux afin de séparer les antécédents et les images.

\begin{tkzexample}[vbox,small]
\begin{tikzpicture}
  \tkzTabInit[espcl=0.5]{ $x$/1,$f(x)$ /1}%
  {1,,2,,3,,4,,5,,6}%
  \tkzTabLine{5,,,,10,,,,15,,,,20,,,,25,,,,30}%
  \makeatletter
  \foreach \x in {1,...,5}{%
     \setcounter{tkz@cnt@pred}{\x}\addtocounter{tkz@cnt@pred}{\x}
     \draw (N\thetkz@cnt@pred 0.center) to (N\thetkz@cnt@pred 2.center);}
  \makeatother
  \begin{scope}[->,red,line width=1pt,>=latex']
	  \draw (M20) to [bend left] node[above]{$\times 3$} (7.5,0);
	  \draw (M22) to [bend right] node[below]{$\times 3$} (7.5,-2);
	  \draw (8,-0.25) to [post,bend left=60] node[midway,above,sloped] {$\times 5$}  (8,-1.75);
  \end{scope}
\end{tikzpicture}
\end{tkzexample}  
\endinput




% 24 / 02 /2009 v1.00c TKZdoc-tab-examples
\section{Galerie}
\subsection{Tableaux de signes}
L'exemple suivant provient de la documentation de l'excellent \tkzname{tablor.sty}.
Voici le code complet 

\medskip
\begin{center}
\begin{tkzexample}[code only]
\documentclass{article}
\usepackage[utf8]{inputenc}
\usepackage[T1]{fontenc}
\usepackage{ifthen,fp}
\usepackage{tikz,tkz-tab}
\usepackage{amsmath,amssymb}
\usepackage[frenchb]{babel}

\begin{document}
\begin{tikzpicture}
  \tkzTabInit[lgt=3]
     {$x$                                  /1,
      Signe de\\ $-2+3$                    /1.5,
      Signe de\\ $-x+5$                    /1.5,
      Signe de\\ $(-2x+3)(-x+5)$           /1.5 }%
     {$-\infty$,$\dfrac{3}{2}$,$5$,$+\infty$}%
  \tkzTabLine { ,+,z,-,t,-, }
  \tkzTabLine { ,+,t,+,z,-, }
  \tkzTabLine { ,+,z,-,z,+, }
\end{tikzpicture}
\end{document}
\end{tkzexample}
\end{center}

\begin{tikzpicture}
  \tkzTabInit[lgt=3]
     {$x$                                  /1,
      Signe de\\ $-2+3$                    /1.5,
      Signe de\\ $-x+5$                    /1.5,
      Signe de\\ $(-2x+3)(-x+5)$           /1.5 }%
     {$-\infty$,$\dfrac{3}{2}$,$5$,$+\infty$}%
  \tkzTabLine { ,+,z,-,t,-, }
  \tkzTabLine { ,+,t,+,z,-, }
  \tkzTabLine { ,+,z,-,z,+, }
\end{tikzpicture}
 

Quelques remarques sur ce code. Le codage utilisé n'a pas d'importance, si vous préférez \tkzname{latin1}, alors remplacez  \tkzname{utf8} par \tkzname{latin1}, bien évidemment  \tkzname{tkz-tab} est essentielle. Si vous utilisez \tkzname{fourier} alors vous pouvez supprimer   \tkzcname{usepackage[T1]\{fontenc\}} et \tkzcname{usepackage\{ammsymb\}}.

\subsection{Variations de fonctions}
 \subsubsection{Variation d'une fonction rationnelle}
 
 Cet exemple a été cité dans la documentation du package \tkzname{tabvar}

 Étude de la fonction $f~:~ x \longmapsto \frac{x^3+2}{2x}$ sur $]-\infty~;~+\infty[$
 
\begin{tkzexample}[vbox,small]
\begin{tikzpicture}
  \tkzTabInit[]
  {$x$  /1, $f'(x)$ /1,$f$  /3}
  {$-\infty$ , $0$ , $1$ , $+\infty$}
\tkzTabLine{,-,d,-,z,+,}
\tkzTabVar{+/$+\infty$ ,-D+/$-\infty$ / $+\infty$ ,-/$\frac{3}{2}$, +/$+\infty$}
\tkzTabVal{1}{2}{0.4}{$ -\sqrt[3]{2}$}{$0$}
\end{tikzpicture}
\end{tkzexample}  


\subsubsection{Variation d'une fonction irrationnelle}
 
Autre exemple  cité dans la documentation du package \tkzname{tabvar}

Étude de la fonction $f~:~ x \longmapsto \sqrt{\frac{x-1}{x+1}}$ sur $]-\infty~;~-1[\cup ]1~;~+\infty[$

\begin{tkzexample}[vbox,small] 
\begin{tikzpicture}
  \tkzTabInit[]
  {$x$  /1, $f'(x)$ /1,$f$  /3}
  {$-\infty$ , $-1$ , $1$ , $+\infty$}
\tkzTabLine{,+,d,h,d,+, }
\tkzTabSlope{ 3/ /+\infty}
\tkzTabVar{-/$1$ ,+DH/$+\infty$  ,-/$0$, +/$1$}
\end{tikzpicture}
\end{tkzexample}  

Un prolongement par continuité pourrait être : $f(x)=0$ sur $[-1~;~1]$ alors le tableau deviendrait

\begin{tkzexample}[vbox,small] 
\begin{tikzpicture}
  \tkzTabInit[]
  {$x$  /1, $f'(x)$ /1,$f$  /3}
  {$-\infty$ , $-1$ , $1$ , $+\infty$}
\tkzTabLine{,+,d,0,d,+, }
\tkzTabSlope{ 3/ /+\infty}
\tkzTabVar{-/$1$ ,+D-/$+\infty$/$0$  ,-/$0$, +/$1$}
\end{tikzpicture}
\end{tkzexample}  


\subsection{Fonctions trigonométriques}
\NameFonct{Fonctions trigonométriques}
 \subsubsection{Variation de la fonction tangente}
  \NameFonct{Fonction  tangente}
 Étude de la fonction $f~:~ x \longmapsto \tan{x}$ sur $[0~;\pi]$
 
 \begin{tkzexample}[vbox]
\begin{tikzpicture}
\tkzTabInit[espcl=6]{$x$ / 1,Signe de\\f'(x)/1, Variations de\\ $f$ / 3}%
     {$0$ ,$\frac{\pi}{2}$ , $\pi$}%
\tkzTabLine{ ,+,d,+, }
\tkzTabVar{-/$0$ ,  +D-/$+\infty$/$-\infty$  , +/$0$ }
\tkzTabVal{1}{2}{0.5}{$\frac{\pi}{4}$}{$1$}
\tkzTabVal{2}{3}{0.5}{$\frac{3\pi}{4}$}{$1$}
\end{tikzpicture}
\end{tkzexample}  

\subsubsection{Variation de la fonction cosinus}
 \NameFonct{Fonction  cosinus}
  Étude de la fonction $f~:~ x \longmapsto \cos{x}$ sur $[-\pi~;~+\pi]$
  
 \begin{tkzexample}[vbox]
\begin{tikzpicture}
\tkzTabInit[espcl=6]{$x$ / 1,Signe de\\f'(x)/1, Variations de\\ $f$ / 3}%
     {$0$ , $\pi$}%
\tkzTabLine{ , + , }
\tkzTabVar{+/$1$ ,   -/$-1$ }
\tkzTabVal{1}{2}{0.5}{$\frac{\pi}{2}$}{$0$}
\end{tikzpicture}
\end{tkzexample}  

\subsection{Fonctions paramétrées et trigonométriques}
\NameFonct{Fonctions paramétrées} \NameFonct{Fonctions trigonométriques}

Étude sur $\left[0~;~\frac{\pi}{2}\right]$
 \begin{equation*} 
\left\{%
\begin{array}{l} 
 x(t) = \cos(3t)\\ 
 y(t) = \sin(4t)
\end{array}\right. 
\end{equation*} 

\begin{tkzexample}[vbox,small]
  \begin{tikzpicture}
  \tkzTabInit[ lgt=3 , espcl=3]%
    {$t$                               /1,
    Signe de\\        $x'(t)$          /1.5,
    Variations de\\   $x$              /3,
     Variations de\\  $y$              /3,
    Signe de\\        $y'(t)$          /1.5}
              {$0$   , $\frac{\pi}{8}$             , $\frac{\pi}{3}$ ,
               $\frac{3\pi}{8}$ , $\frac{\pi}{2}$ }%
  \tkzTabLine {z , - ,-3\sin\left(\frac{3\pi}{8}\right) , - , z , +  ,%
   3\sin\left(\frac{\pi}{8}\right),+,3}
   \tkzTabVar  { +/$1$ , R/    , -/$-1$/ , R/     , +/$0$ }
    \tkzTabIma{1}{3}{2}{$\cos\left(\frac{3\pi}{8}\right)$} 
   \tkzTabIma{3}{5}{4}{$-\cos\left(\frac{\pi}{8}\right)$} 
     \tkzTabVar  { -/$0$ , +/$1$ , R/      , -/$-1$ , +/$0$ }
   \tkzTabIma{2}{4}{3}{$\frac{-\sqrt{3}}{2}$} 
   \tkzTabLine {4 , + , z , - , -2 , - ,  z ,+,4}
 \end{tikzpicture}
\end{tkzexample}


\subsection{Baccalauréat Asie ES 1998}
\index{Baccalauréat}
Une petite astuce, en principe \tkzname{z}  est le symbole à mettre dans la liste pour obtenir un zéro centré sur un trait en pointillés. Si on veut que le zéro soit sans le trait , il suffit de remplacer \tkzname{z} par \tkzname{0}. Celui-ci n'est pas un symbole reconnu, il est donc traiter comme une chaîne normale.

  Soit $f$ la fonction de variable réelle $x$, définie sur $\mathbf{R}$ par :
  \[
      f(x)=\E^x(\E^x+a)+b
  \]
  où $a$ et $b$ sont deux constantes réelles.

  Les renseignements connus sur $f$ sont donnés dans le tableau de variation ci-dessous.
   
  \medskip
  \begin{center}
    \begin{tikzpicture}
    \tkzTab[lgt=3,espcl=4]{$x$/1,Signe de $f'(x)$ /1,Variations de $f$ /2}%
    {$-\infty$,$0$,$+\infty$}%
    {,, z ,,}%
     {+/    $-3$  ,
      -/          ,
      +/          } 
    \end{tikzpicture}
  \end{center}

  \medskip
  \begin{enumerate}
      \item Calculer $f'(x)$ en fonction de $a$ ($f'$ désigne la fonction dérivée de $f$).
      \item \begin{enumerate}
             \item déterminer $a$ et $b$ en vous aidant des informations contenues dans le
              tableau ci-dessus.
             \item Calculer $f(0)$ et calculer la limite de $f$ en $+\infty$.
             \item Compléter, après l'avoir reproduit, le tableau de variations de $f$.
             \end{enumerate}
      \item Résoudre dans $\mathbf{R}$ l'équation 
      \[
          \E^x(\E^x-2)-3=0
      \]
      (on pourra pose $X=\E^x$).
      \item Résoudre dans $\mathbf{R}$ les inéquations : 
      \[
          \E^x(\E^x-2)-3\geq -4
      \]
      \[
          \E^x(\E^x-2)-3 \leq 0
      \]
       (On utilisera le tableau de variations donné ci-dessus et en particulier les
        informations obtenues en 2.b)
  \end{enumerate}

\begin{tkzexample}[code only,small]
  Soit $f$ la fonction de variable réelle $x$, définie sur $\mathbf{R}$ par :
  \[
      f(x)=\E^x(\E^x+a)+b
  \]
  où $a$ et $b$ sont deux constantes réelles.

  Les renseignements connus sur $f$ sont donnés dans le tableau de variation ci-dessous.
   
  \medskip
  \begin{center}
    \begin{tikzpicture}
    \tkzTab[lgt=3,espcl=4]{$x$/1,Signe de $f'(x)$ /1,Variations de $f$ /2}%
    {$-\infty$,$0$,$+\infty$}%
    {,, z ,,}%
     {+/    $-3$  ,
      -/          ,
      +/          } 
    \end{tikzpicture}
  \end{center}

  \medskip
  \begin{enumerate}
      \item Calculer $f'(x)$ en fonction de $a$ ($f'$ désigne la fonction dérivée de $f$).
      \item \begin{enumerate}
             \item déterminer $a$ et $b$ en vous aidant des informations contenues dans le
              tableau ci-dessus.
             \item Calculer $f(0)$ et calculer la limite de $f$ en $+\infty$.
             \item Compléter, après l'avoir reproduit, le tableau de variations de $f$.
             \end{enumerate}
      \item Résoudre dans $\mathbf{R}$ l'équation 
      \[
          \E^x(\E^x-2)-3=0
      \]
      (on pourra pose $X=\E^x$).
      \item Résoudre dans $\mathbf{R}$ les inéquations : 
      \[
          \E^x(\E^x-2)-3\geq -4
      \]
      \[
          \E^x(\E^x-2)-3 \leq 0
      \]
       (On utilisera le tableau de variations donné ci-dessus et en particulier les
        informations obtenues en 2.b)
  \end{enumerate}
\end{tkzexample}


\subsection{Baccalauréat}
\index{Baccalauréat}
	On considère la fonction $f$ définie sur $]-\infty~;~0[$ :

	\[
	  f(x)=ax+b+\ln(-2x)
	\]
	où $a$ et $b$ sont deux réels donnés.

	\begin{enumerate}
	\item Calculer $f'(x)$ en fonction de $a$ et $b$.
	\item Le tableau ci-dessous représente les variations d'une fonction particulière $f$.

	\medskip
	\begin{center}
	  \begin{tikzpicture}
	  \tkzTab[]%
	  {$x$/1.25,Signe de\\ $f'(x)$/1.5, Variations\\ de $f$/1.5}%
	  {$-\infty$,$\dfrac{-1}{2}$,$0$}%
	  {,+,$0$,-,}%
	  {-//,
	  +/$2$/,
	  -//}
	  \end{tikzpicture}
	\end{center}

	\medskip
	\begin{enumerate}
	\item En utilisant les données du tableau déterminer les valeurs $a$ et $b$ qui caractérisent
	 cette fonction.
	\item Pour cette fonction particulière $f$, déterminer 
	        $\displaystyle \lim_{x \xrightarrow[x<0]{} 0} f(x)$.
	\item Montrer que, dans l'intervalle $\Big[\dfrac{-1}{2}~;~0,01\Big]$, l'équation $f(x)=0$
	 admet une solution unique. En donner une valeur approchée à $10^{-3}$ près.
	\end{enumerate}
	\end{enumerate}
	
\begin{tkzexample}[small,code only]
	On considère la fonction $f$ définie sur $]-\infty~;~0[$ :

	\[
	  f(x)=ax+b+\ln(-2x)
	\]
	où $a$ et $b$ sont deux réels donnés.

	\begin{enumerate}
	\item Calculer $f'(x)$ en fonction de $a$ et $b$.
	\item Le tableau ci-dessous représente les variations d'une fonction particulière $f$.

	\medskip
	\begin{center}
	  \begin{tikzpicture}
	  \tkzTab[]%
	  {$x$/1.25,Signe de\\ $f'(x)$/1.5, Variations\\ de $f$/1.5}%
	  {$-\infty$,$\dfrac{-1}{2}$,$0$}%
	  {,+,$0$,-,}%
	  {-//,
	  +/$2$/,
	  -//}
	  \end{tikzpicture}
	\end{center}

	\medskip
	\begin{enumerate}
	\item En utilisant les données du tableau déterminer les valeurs $a$ et $b$ qui caractérisent
	 cette fonction.
	\item Pour cette fonction particulière $f$, déterminer 
	        $\displaystyle \lim_{x \xrightarrow[x<0]{} 0} f(x)$.
	\item Montrer que, dans l'intervalle $\Big[\dfrac{-1}{2}~;~0,01\Big]$, l'équation $f(x)=0$
	 admet une solution unique. En donner une valeur approchée à $10^{-3}$ près.
	\end{enumerate}
	\end{enumerate}
\end{tkzexample}



\subsection{Baccalauréat Guyane ES 1998 }
\index{Baccalauréat}
C'est cet exemple qui m'a obligé à penser aux   commandes du style $+V+$. Sans doute, voulait-on ne pas influencer les élèves avec la vision d'une double barre (trop souvent associée à la présence d'une asymptote).

\textbf{Le sujet :}

{\parindent=0pt
On considère une fonction $f$ de la variable $x$, dont on donne le tableau de variations :

\begin{center}
\begin{tikzpicture}
\tkzTab[lgt=3]%
{$x$/1.25,Signe de\\ $f'(x)$/1.5, Variations\\ de $f$/2.5}
{$-\infty$,$\dfrac{-1}{2}$,$1$,$+\infty$}
{,-,$0$,+, ,-,}
{+/ $1$ , -/$\dfrac{-1}{3}$ , +V+/ $+\infty$ /$+\infty$ , -/$1$}
\end{tikzpicture}
\end{center}

On appelle (C) la courbe représentative de $f$ dans un repère Le plan est muni d'un repère orthonormé $(O;\vec{\imath};\vec{\jmath})$ (unités graphiques 2 cm sur chaque axe)
 
\vspace{6pt}
\textbf{Première partie}

En interprétant le tableau donné ci-dessus :%
 
 \begin{enumerate}
 \item  Préciser l'ensemble de définition de $f$.
 \item  Placer dans le repère $(O;\vec{\imath};\vec{\jmath})$  :
 \begin{enumerate}
 \item  l'asymptote horizontale (D);
 \item  l'asymptote verticale (D');
 \item  le point $A$ où la tangente à (C) est horizontale.
 \end{enumerate}
 \end{enumerate}

\textbf{Seconde partie}

On donne maintenant l'expression de $f$ :
\[
f(x)=1 + \dfrac{4}{(x-1)}    + \dfrac{3}{(x-1)^2}
\]
\begin{enumerate}
 \item Résoudre les équations $f(x)=0$ et $f(x)=1$.
 \item  Au moyen de votre calculatrice, remplir le tableau suivant
  ( recopier ce tableau sur votre copie).
\end{enumerate}
\begin{tikzpicture}
   \tkzTabInit[deltacl=1,espcl=1]{ $x$/1 , $f(x)$ /1}%
              {-1,,{-0,75},,{0,5},,2,,3,,4}%
   \tkzTabLine{,,,,,,,,,,,,,,,,,,,,}%
   \makeatletter
   \foreach \x in {1,...,5}
    \setcounter{tkz@cnt@pred}{\x}\addtocounter{tkz@cnt@pred}{\x}
    \draw (N\thetkz@cnt@pred 0.center) to (N\thetkz@cnt@pred 2.center);
\end{tikzpicture}
}


\begin{tkzexample}[code only,small]
On considère une fonction $f$ de la variable $x$, dont on donne le tableau de variations :

\begin{center}
\begin{tikzpicture}
\tkzTab[]%
{$x$/1.25,Signe de\\ $f'(x)$/1.5, Variations\\ de $f$/2.5}
{$-\infty$,$\dfrac{-1}{2}$,$1$,$+\infty$}
{,-,$0$,+, ,-,}
{+/ $1$ , -/$\dfrac{-1}{3}$ , +V+/ $+\infty$ /$+\infty$ , -/$1$}
\end{tikzpicture}
\end{center}

 \vspace{6pt}
On appelle (C) la courbe représentative de $f$ dans un repère. Le plan est muni d'un repère%
 orthonormal  $(O;\vec{\imath};\vec{\jmath})$ (unités graphiques 2 cm sur chaque axe)%

\textbf{Première partie}

En interprétant le tableau donné ci-dessus :%
 
 \begin{enumerate}
 \item  Préciser l'ensemble de définition de $f$.
 \item  Placer dans le repère $(O;\vec{\imath};\vec{\jmath})$  :
 \begin{enumerate}
 \item  l'asymptote horizontale (D);
 \item  l'asymptote verticale (D');
 \item  le point $A$ où la tangente à (C) est horizontale.
 \end{enumerate}
 \end{enumerate}

\textbf{Seconde partie}

On donne maintenant l'expression de $f$ :
\[
f(x)=1 + \dfrac{4}{(x-1)}    + \dfrac{3}{(x-1)^2}
\]
\begin{enumerate}
 \item Résoudre les équations $f(x)=0$ et $f(x)=1$.
 \item  Au moyen de votre calculatrice, remplir le tableau suivant
  ( recopier ce tableau sur votre copie).
   \begin{tikzpicture}
   \tkzTabInit[deltacl=1,espcl=1]{ $x$/1,$f(x)$ /1}%
   {-1,,{-0,75},,{0,5},,2,,3,,4}%
   \tkzTabLine{,,,,,,,,,,,,,,,,,,,,}%
   \makeatletter
   \foreach \x in {1,...,5}
    \setcounter{tkz@cnt@pred}{\x}\addtocounter{tkz@cnt@pred}{\x}
    \draw (N\thetkz@cnt@pred 0.center) to (N\thetkz@cnt@pred 2.center);
    \end{tikzpicture}
\end{enumerate}

\vfill
\end{tkzexample}

\endinput 
%<--------------------------------------------------------------------------> 
\printindex
\end{document}
%<------------------------------------------------------------------------->



