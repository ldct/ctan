\documentclass[12pt]{memoir}
\usepackage[dutch]{babel}
\usepackage[
  %corrected,
  %uncorrected,
  onehalfspacing,
  %doublespacing,
  frame,
]{proofread}
\usepackage{ctable}
\marginparwidth=35mm
\begin{document}
Een canon van het Nederlands verleden.

Iedere tijd\footnote{Dit is een voetnoot waarin ik een verbetering
\rep{moet}{wil} aanbrengen} ontwerpt een eigen \hilite{versie van het verleden}.
Vroeger gebeurde dat bij het kampvuur. Later werd dat op scholen meeslepend
verteld. Nu zou televisie het voor de hand liggende medium zijn. Alleen, dat
gebeurt niet. Daar \rep{schijnt}{lijkt} de gedachte te heersen dat geschiedenis
eigenlijk te saai is om een aardig kijkerspubliek te trekken. Daarom krijgt het
de vorm van een gezelschaps\del{ }spelletje, zoals de verkiezing van `de
grootste Nederlander aller tijden'.

Het is de vraag op grond van welke kennis de kijkers straks deze historische
`idol' kiezen. \yel[commentaar op een zin; ik maak het lekker lang om nieuwe
regels te forceren.]{Gezien de verbrokkeling van het geschiedenisonderwijs kan
het hier moeilijk om veel meer gaan dan bliksemschichten in het donker.}

\ctable[caption=test]{c}{}{\FL
  Test met een margin yel \yel[een lange opmerking hierbij]{tast}\LL
}

Dat er weinig belangstelling zou zijn voor serieuze geschiedenis is een
merkwaardig vooroordeel in `Hilversum', dat ook heerst in `Den Haag'. Alleen al
de opmerkelijk hoge oplagen van een auteur als Geert Mak zouden op het tegendeel
kunnen wijzen.

\begin{minipage}{100mm}
Weinig behulpzaam zijn pleidooien van al \rep{diegenen die}{wie}, bezield van de
meest nobele gedachten, beweren dat er geen historisch besef meer bestaat.
Daarbij blijken zij meestal te bedoelen dat niemand meer een proefwerk kan maken
dat sommigen van ons nog wel kennen van de middelbare school (althans: van
v\'o\'or de Mammoetwet): `wie deed wat, wanneer en waarom?' Met hoon duikt
telkens weer het beruchte proefwerk van het Historisch Nieuwsblad op, waarin
Kamerleden op dit soort vragen een dikke onvoldoende scoorden. Een herhaling
onlangs onder `gewone mensen' leverde een nog droeviger beeld op.

\end{minipage}

Naast onwil heerst ook onzekerheid: welk verhaal valt hier te vertellen? De
schuchterheid over het historisch verhaal is vooral een gevolg van veranderingen
in de samenleving. Nederland is `ontzuild'. Kerken en politieke bewegingen
hebben aan overtuigingskracht ingeboet. Het (emancipatie-)verleden van de eigen
bevolkingsgroep is minder \rep{bruikbaar geworden}{relevant gemaakt}.
Bijpassende rituelen zijn vervaagd. De geschiedenis is minder vanzelfsprekend.
Niet langer zijn hieraan wijze lessen en aansporingen te ontlenen. Historische
kennis is verschraald tot eruditie.

Als we weer historisch besef willen kweken, dan gaat het niet zozeer om de
weetjes: de jaartallen van rampen, vorsten en ontdekkingen\,---\,al is daar op
zich\add{zelf} helemaal niets mis mee. Het gaat om de gedachte dat het heden
niet goed te begrijpen valt zonder enig inzicht in de ontwikkeling tot dat
heden. Anders gezegd, het gaat om het vermijden van het misverstand dat de
wereld een schouwtoneel is waarvoor iedere dag het doek opnieuw wordt opgehaald.

Tegen deze ambitie klinken doorgaans twee bezwaren. Het eerste is dat dit
allemaal voortvloeit uit nostalgie. Oude schoolmeesters zouden een achterhaald
verlangen koesteren naar de overzichtelijkheid van de `nationalistische en
etnocentrische geschiedschrijving van onze blanke voorvaderen'. Dit klinkt al
erg genoeg, maar gewoonlijk volgt dan nog dat we nu eenmaal in een `postmoderne
wereld' leven. De fragmentatie van mens en samenleving is niet langer in een
consistent verhaal te vatten. `Laat duizend bloemen bloeien' en het komt
vanzelf wel goed.

Het tweede bezwaar is dat historisch besef op zichzelf, als een manier om naar
de werkelijkheid te kijken, een tijdelijk verschijnsel is. Het is aan het eind
van de achttiende eeuw opgekomen, beleefde zijn bloeitijd in de negentiende eeuw
en zal nu ten onder gaan. We leven in een `posthistorische wereld'. Dit bezwaar
versterkt de gedachte\,---\,die ook bij enkele belangrijke historici
leeft\,---\,dat de strijd al is verloren en dat het geschiedenisonderwijs maar
beter helemaal kan worden afgeschaft. Belangstelling voor geschiedenis is op z'n
best een oude-mannenkwaal.

\begin{itemize}
\item test
\item test\add{ nog een test}
\item test
\item test
\end{itemize}

En zo mogen we kiezen uit tuchteloosheid of vruchteloosheid. Het gevolg van deze
verwarring is de opmerkelijke dictatuur van het nu en hier, het zogenaamde
presentisme. Bijna tien jaar geleden merkte Rudy Kousbroek al eens op dat in
Nederland, in tegenstelling tot allerlei andere landen, vooral de gedachte leeft
dat weinig uit het verleden nog interessant of waardevol kan zijn: ``Dat is wat
in dit land dat eigenaardige gevoel geeft dat er een dimensie ontbreekt.'' In
het openbare debat is de continu\"iteit met het verleden nagenoeg afwezig.

\del{Een dergelijke }\skp\rep{c}{C}ontinu\"iteit wordt \skp[2]\rep{doorgaans}{veelal}
gevonden in een canon: een geheel aan kennis en inzichten, aan ordening en
interpretatie van het verleden. Daaraan dienen we meteen toe te voegen dat een
dergelijk geheel niet onveranderlijk is. Integendeel, een canon mag en kan niet
worden gecanoniseerd. Essentieel is juist dat deze voortdurend onderwerp is van
reflectie. Wie en wat verdienen een plek in de canon en waarom? Micha\"el Zeeman
zei hierover eens: ``De canon leert geen vaststaande grootheden, de canon
onderwijst lezen, kijken en luisteren, dat is kritisch oordelen.'' De canon
nodigt uit tot kritiek, tot aanvulling en in ieder geval tot gebruik.

We hebben geprobeerd\com{a comment without highlighting} een beknopte canon te
formuleren voor de `Nederlandse' geschiedenis. Bij de samenstelling hebben drie
criteria een rol gespeeld. Hoe heeft het huidige Nederland zich gevormd? Welk
politiek-bestuurlijk systeem was in dit \yel{gebied} overheersend? Welke
ontwikkelingen hebben de Nederlandse samenleving sterk hebben be\"invloed?

Aan deze proeve van een canon zouden we een motto willen meegeven dat aan
\hilite[fill=red, draw=blue,opacity=.5,line width=3pt]{Willem van Oranje} is
toegeschreven: `Hoop is niet vereist om ergens aan te beginnen, succes niet
nodig om te volharden.'

\end{document}
vim: syntax=proofread
