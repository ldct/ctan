\documentclass[a4paper]{article}
\usepackage[obeyFinal]{easy-todo}

\begin{document}
	\title{The \texttt{easy-todo} package}
	\author{by Juan Rada-Vilela (\texttt{jcrada@fuzzylite.com})}
	\date{January, 2014}
	
	\maketitle
	
	\begin{abstract}
		The \texttt{easy-todo} package allows you to create, track, show and hide notes in a document. In addition, the package allows you to create a summary of notes with their respective points of reference. 
	\end{abstract}
	
	\section{Options}
	The following options are available:
	
	\begin{description}
        \item [enable] Show the notes and the index.
		\item [disable] Hide the notes and the index.
        \item [obeyFinal] Hide the notes except when the documentclass is \textbf{final}. This option overrides the options \textbf{enable} and \textbf{disable}.
		\item [chapter] Prints the list of TODOs as a chapter.
		\item [section] Prints the list of TODOs as a section.
	\end{description}
	

	\section{Commands}
	\begin{description}
		\item [\textbackslash todo\{note\}] Creates a note that shows the text ``\texttt{note}''.
        \item [\textbackslash todoi\{note\}] Creates a note that only shows an automatically generated number. The text ``\texttt{note}'' appears in the index.
        \item [\textbackslash todoii\{note\}\{information\}] Creates a note that shows the text ``\texttt{note}''. The text ``\texttt{information}'' appears at the index.
		\item [\textbackslash listoftodos] Creates the list of notes.
	\end{description}
    
    \section{Example}	
    	\texttt{\textbackslash usepackage[obeyFinal]\{easy-todo\}}
    
	This is a todo \todo{note} that appears in full everywhere. This is a todoi \todoi{note i}  that appears as a number and the note at the index. This is a todoii \todoii{hint}{information} note that shows a hint in place and the information at the index. All these notes disappear when using \texttt{\textbackslash documentclass[final]...}, or when using instead \texttt{\textbackslash usepackage[disable]\{easy-todo\}}.
	
    \section{Changelog}
    
    \subsection{Version 3.0}
        \begin{itemize}
        \item Added option \textbf{obeyFinal}
        \item Removed spaces caused by notes when hidden.
        \item Option \textbf{section} is default.
        \item Removed options \textbf{enabled}, \textbf{disabled} and \textbf{final}.
        \end{itemize}
    
    \subsection{Version 2.0}
        \begin{itemize}
            \item Added options \textbf{enable} and \textbf{disable}
            \item Added commands \textbf{todoi}, \textbf{todoii}
        \end{itemize}
    
    \listoftodos
\end{document}
