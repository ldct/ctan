%
% Modified version of the sample_ndthesis.tex
% by Sameer Vijay
% Last Change: Wed Jul 27 2005 14:00 CEST
%
%%%%%%%%%%%%%%%%%%%%%%%%%%%%%%%%%%%%%%%%%%%%%%%%%%%%%%%%%%%%%%%%%%%%%%%%
%
% Sample Notre Dame Thesis/Dissertation
% Using Donald Peterson's ndthesis classfile
%
% Written by Jeff Squyres and Don Peterson
%
% Provided by the Information Technology Committee of
%   the Graduate Student Union
%   http://www.gsu.nd.edu/
%
% Nothing in this document is serious except the format.  :-)
%
%%%%%%%%%%%%%%%%%%%%%%%%%%%%%%%%%%%%%%%%%%%%%%%%%%%%%%%%%%%%%%%%%%%%%%%%
% This is *not* a substitute for Donald's orginial documentation.  See
% /afs/nd.edu/usr/local/src/tex/texmf/doc/latex/ndthesis/ndthesis.dvi
% for documentation on the particular commands and whatnot.
%%%%%%%%%%%%%%%%%%%%%%%%%%%%%%%%%%%%%%%%%%%%%%%%%%%%%%%%%%%%%%%%%%%%%%%%
%
% You should *also* have a ND formatting guide to ensure that you have
% all the relevant parts, put the captions in the right place, etc.
% Just because you have this wonderful style classfile doesn't mean
% that it removes *all* the formatting onus from you.  :-)
%
% Normally, you should break all of this stuff up into separate files
% (at the very least, one chapter per file) and use the \input
% command.  This is all one file for brevity's (and clarity's) sake.
%
% Note that you should also have a good Makefile; one that invokes
% LaTeX as many times as necessary (up to 4) and bibtex if necessary.
% One should be included in this distribution.  You may want to modify
% the Makefile to make separate chapters, if necessary.
%
% If you have any suggestions, comments, questions, please send e-mail
% to: ndthesis@gsu.nd.edu
%
%%%%%%%%%%%%%%%%%%%%%%%%%%%%%%%%%%%%%%%%%%%%%%%%%%%%%%%%%%%%%%%%%%%%%%%%

\documentclass[textrefs,review]{nddiss2e}

% % uncomment the following line 
% if using chapter-wise bibliography
% \usepackage{chapterbib}
% \renewcommand{\bibname}{Cited Works}
% \renewcommand{\bibsection}{\section{\bibname}}

\begin{document}

\frontmatter

\title{GNUS AND YOU \\ {\small\scshape A BRIEF ON ALL AND EVERYTHING ABOUT GNUS IN OUR
SOCIETY} }
\author{Gerald G. Gnastich}
\work{Dissertation}
\degprior{B.S., M.S.}
\degaward{Doctor of Philosophy\\in\\Gnulogical Science}
\advisor{Gary Greenfield}
% \secondadvisor{Gordon Gray}
\department{Gnulogy}

\maketitle
%%%%%%%%%%%%%%%%%%%%%%%%%%%%%%%%%%%%%%%%%%%%%%%%%%%%%%%%%%%%%%%%%%%%%%%%
%
% Front stuff
%
%%%%%%%%%%%%%%%%%%%%%%%%%%%%%%%%%%%%%%%%%%%%%%%%%%%%%%%%%%%%%%%%%%%%%%%%

\copyrightholder{Garry Greene}
\copyrightyear{2005}
\makecopyright

\begin{abstract}
  Please note that the full \LaTeX\ source code (and an associated
  {\tt Makefile}) is available from the University of Notre Dame
  Graduate Student Union web site.  The Information Technology
  Committee page\footnote{{\tt http://www.gsu.nd.edu/}}
  has all the necessary files in download-able form.  This particular
  dissertation was developed under Unix, but should also be usable
  under Windows with the appropriate \LaTeX\ setup.  
  
  While the source code for this document provides an excellent
  example for how to use the \nddiss\ \LaTeX\ class to write a
  Notre Dame thesis, it is {\em not} a substitution for the
  documentation of the \nddiss\ \LaTeX\ class (also available on
  the ND GSU web site).

  In this thesis, I will tell all that I know about Gnus.  Gnus are
  wonderful little creatures that inhabit the center of the earth and
  give us wonderful and plentiful trees, dirt, and other
  earthly-things.
  
  In short, we should love and cherish the Gnus.  They can be very
  friendly, and are often mistaken for squirrels on the University of
  Notre Dame campus.  Feed them whenever possible.  If they get caught
  in trash cans, tip them over so that they can get out.

  This abstract is going to continue on, including a few formulas,
  just for the sake of spilling over on to two pages so that we can
  see the author's name in the top right corner:

  \begin{eqnarray*}
    a^2 + b^2 & = & c^2 \\
    E & = & mc^2 \\
    \frac{e}{m} & = & c^2 \\
    a^2 + b^2 & = & \frac{e}{m}
  \end{eqnarray*}

  These equations, by themselves mean nothing.  But to the common Gnu,
  they define a whole way of living.  While intricate mathematical
  implications certainly do not infiltrate the majority of humans'
  lives, every Gnu, from birth, is imbued with a sense of mathematical
  certainty and guidance.  All Gnus, great and small, feel at one with
  mathematics.  The cute furry bit is just a scam for their
  calculating minds.
\end{abstract}

\renewcommand{\dedicationname}{New Dedication Name}

\begin{dedication}
  To George, my favorite Gnu
\end{dedication}

\tableofcontents
\listoffigures
\listoftables

\begin{preface}
  I would like to preface this work with all the wonderful things that
  Gnus have brought to our society: trees, dirt, flowers, grass,
  lakes, and other earthly-things.  We should not forget them in our
  daily lives.

  Additionally, we should offer them food for all their hard work.  In
  fact, Gnus work so hard that they sleep for the colder half of
  the year.  As such, they tend to grow a little rotund.  Humans
  should not fault them for this, as it is necessary for their
  survival.  Indeed, many humans grow rotund on their on accord!
\end{preface}

\begin{acknowledge}
  I would like to acknowledge all the loving Gnus at Notre Dame.
  Particularly the one that comes to the window in the Hayes Healy
  building.  He (she?) has given me much inspiration, love, and dirt.
  I would also like to thank my advisor, Dr.\ Gary Greenfield, with
  whom this work would not have been possible.

  Finally, I would like to thank the U.S.\ Government, Department of
  Gnus, for their generous grant, number GNU3042920920.3, which
  allowed me to pursue my work.
\end{acknowledge}

\begin{symbols}
  \sym{\mathcal{F}}{sighting frequency of Gnus about campus}
  \sym{p}{student population}
  \sym{f}{type of food available}
  \sym{d}{day of week}
  \sym{c}{speed of light}
  \sym{m}{mass}
  \sym{e}{elementary charge}
  \sym{a,b}{miscellaneous constants}  
  \sym{E}{energy}  
\end{symbols}

\mainmatter

%
% Chapter 0 (features)
%
\setcounter{chapter}{-1}
\chapter{FEATURES OF FORMATTING IN THIS EXAMPLE FILE}

This \verb+chapter+ has been added to the original sample file to highlight the
various features with the formatting that conforms to the Graduate school
guidelines --- whether obtained due to the use of \nddiss\/ class file or just
plain good practice.
\begin{itemize}
\item An important thing you might notice is that the title of this chapter 
is not in all CAPS. This is a feature since using \verb+\MakeTextUpperCase{}+ 
is not helpful
when you have a math formula or something that just doesnt go CAPS (for eg.
elemental symbols).
\item If you're looking at a pdf document, the pdf bookmarks (left column) link
to all major textual sections including abstract, toc, lof, lot and
bibliography.
\item In the {\em dedication}, the title name has been modified. So, you know
how to and that it can be done.
\item The entries in the {\em List of figures} and {\em List of Tables} are
single-spaced themselves but are double-spaced from the other.
\item The table captions are not in all CAPS as well for the reason mentioned
above.
\item Appropriate space is left between the \verb+Table xx+ and its
corresponding caption (which is double-spaced itself) as in table \ref{tbl:bogus1}.
\item Tables look much better without the vertical lines (good practice).
\item There is double-spacing between the table entries but single-spacing
within the entry.
\item The chapter (see Chapter \ref{chap:golfing}) or section titles are
double-spaced as mentioned in the guidelines.
\item There is a \verb+subsubsection+ present (eg. section \ref{sec:data}) and
is properly formatted in the TOC.
\item Table \ref{tbl:defs} is an example of the use of \textsf{landscape}
environment in which a normal table is formatted in a {\em landscape} mode.
\item The \textsf{longtable} environment is used in Tables \ref{tbl:votes} and
\ref{tbl:rotated-rankings}, in normal and \verb+landscape+ mode, respectively. The
table captions are formatted properly in both cases.
\item In the table \ref{tbl:votes}, the \verb+footnote+ in the table header 
does not appear at all. This is not an error of the \nddiss\/ class but of the
\textsf{longtable} package.
\item An example of citing a website is shown in the bibliography (see
\citep{gairley2000}) which is formatted using the 
\verb+nddiss2e.bst+
citation style file.
\item A bit of information on the \nddiss\/ class file and the typesetting program
used is included in a box on the last page of the thesis.
\end{itemize}

%
% Chapter 1
%

\include{chapter1}


%
% Chapter 2
%

\include{chapter2}


%
% Appendix
%

\appendix

% master: main
% format: latex

%%--------------------------------------------------------------------------
\begin{appendices}
%%--------------------------------------------------------------------------

%%--------------------------------------------------------------------------
\chapter{FIRST APPENDIX}
%%--------------------------------------------------------------------------

This is the 1st appendix.  Now alphabetic numbering starts for Appedices.
\ifAMS
\begin{equation}
    A=
    \begin{pmatrix}
	a_{11}&a_{12}&\ldots&a_{1n}\\
	a_{21}&a_{22}&\ldots&a_{2n}\\
	\vdots&\vdots&\ddots&\vdots\\
	a_{m1}&a_{m2}&\ldots&a_{mn}
    \end{pmatrix}
\end{equation}
\else
\begin{equation}
    A=
    \left(
    \begin{array}{cccc}
	a_{11}&a_{12}&\ldots&a_{1n}\\
	a_{21}&a_{22}&\ldots&a_{2n}\\
	\vdots&\vdots&\ddots&\vdots\\
	a_{m1}&a_{m2}&\ldots&a_{mn}
    \end{array}
    \right)
\end{equation}
\fi

%%--------------------------------------------------------------------------
\chapter{MATHEMATICAL SYMBOLS}
%%--------------------------------------------------------------------------
\label{sec:mathsym}

Here is the second appendix.  See how equations, figures, and tables in
appendices are numbered.
\ifAMS
\begin{equation}
    A=
    \begin{pmatrix}
	a_{11}&a_{12}&\ldots&a_{1n}\\
	a_{21}&a_{22}&\ldots&a_{2n}\\
	\vdots&\vdots&\ddots&\vdots\\
	a_{m1}&a_{m2}&\ldots&a_{mn}
    \end{pmatrix}
\end{equation}
\else
\begin{equation}
    A=
    \left(
    \begin{array}{cccc}
	a_{11}&a_{12}&\ldots&a_{1n}\\
	a_{21}&a_{22}&\ldots&a_{2n}\\
	\vdots&\vdots&\ddots&\vdots\\
	a_{m1}&a_{m2}&\ldots&a_{mn}
    \end{array}
    \right)
\end{equation}
\fi
%
\begin{eqnarray}
 \left(\int_{-\infty}^\infty e^{-x^2}\,dx\right)^2
 & =& \int_{-\infty}^\infty\int_{-\infty}^\infty
   e^{-(x^2+y^2)}\,dx\,dy \nonumber \\
 & =& \int_0^{2\pi}\int_0^\infty e^{-r^2}r\,dr\,d\theta \nonumber \\
 & =& \int_0^{2\pi}\left(\left. -\frac{e^{-r^2}}{2}
   \right|_{r=0}^{\infty}\,\right)\,d\theta \nonumber \\
 & =& \pi
\end{eqnarray}

otherwise

\begin{eqnarray}
\textstyle\sin18^\circ={\frac{1}{4}}(\sqrt5-1)\\
\ifAMS
x \in \mathbb{R} \\
\fi
k=1.38\times10^{-23}\rm\,J/^\circ K.
\end{eqnarray}

% Math-mode symbol & verbatim
\def\W#1#2{$#1{#2}$ &\ttfamily\string#1\string{#2\string}}
\def\X#1{$#1$ &\ttfamily\string#1}
\def\Y#1{$\big#1$ &\ttfamily\string#1}
\def\Z#1{\ttfamily\string#1}

%
\begin{table}
\caption{Greek Letters}\label{tab:greek}
\vspace{1ex}
\begin{tabular}{*8l}
\X\alpha	&\X\theta	&\X o		&\X\tau 	\\
\X\beta 	&\X\vartheta	&\X\pi		&\X\upsilon	\\
\X\gamma	&\X\iota	&\X\varpi	&\X\phi 	\\
\X\delta	&\X\kappa	&\X\rho 	&\X\varphi	\\
\X\epsilon	&\X\lambda	&\X\varrho	&\X\chi 	\\
\X\varepsilon	&\X\mu		&\X\sigma	&\X\psi 	\\
\X\zeta 	&\X\nu		&\X\varsigma	&\X\omega	\\
\X\eta		&\X\xi						\\
								\\
\X\Gamma	&\X\Lambda	&\X\Sigma	&\X\Psi 	\\
\X\Delta	&\X\Xi		&\X\Upsilon	&\X\Omega	\\
\X\Theta	&\X\Pi		&\X\Phi
\end{tabular}
\end{table}

\begin{table}
\caption{Binary Operation Symbols}\label{tab:bin}
\vspace{1ex}
\begin{tabular}{*8l}
\X\pm		&\X\cap 	&\X\diamond		&\X\oplus     \\
\X\mp		&\X\cup 	&\X\bigtriangleup	&\X\ominus    \\
\X\times	&\X\uplus	&\X\bigtriangledown	&\X\otimes    \\
\X\div		&\X\sqcap	&\X\triangleleft	&\X\oslash    \\
\X\ast		&\X\sqcup	&\X\triangleright	&\X\odot      \\
\X\star 	&\X\vee 	&\X\lhd$^*$		&\X\bigcirc   \\
\X\circ 	&\X\wedge	&\X\rhd$^*$		&\X\dagger    \\
\X\bullet	&\X\setminus	&\X\unlhd$^*$		&\X\ddagger   \\
\X\cdot 	&\X\wr		&\X\unrhd$^*$		&\X\amalg     \\
\X+		&\X-
\end{tabular}

$^*$ Not predefined in \LaTeXe.
     Use one of the packages  \textsf{latexsym}, \textsf{amsfonts} or
     \textsf{amssymb}.

\end{table}


\begin{table}
\caption{Relation Symbols}\label{tab:rel}
\vspace{1ex}
\begin{tabular}{*8l}
\X\leq		&\X\geq 	&\X\equiv	&\X\models	\\
\X\prec 	&\X\succ	&\X\sim 	&\X\perp	\\
\X\preceq	&\X\succeq	&\X\simeq	&\X\mid 	\\
\X\ll		&\X\gg		&\X\asymp	&\X\parallel	\\
\X\subset	&\X\supset	&\X\approx	&\X\bowtie	\\
\X\subseteq	&\X\supseteq	&\X\cong	&\X\Join$^*$	\\
\X\sqsubset$^*$ &\X\sqsupset$^*$&\X\neq 	&\X\smile	\\
\X\sqsubseteq	&\X\sqsupseteq	&\X\doteq	&\X\frown	\\
\X\in		&\X\ni		&\X\propto	&\X=		\\
\X\vdash	&\X\dashv	&\X<		&\X>		\\
\X:
\end{tabular}

$^*$ Not predefined in \LaTeXe.
     Use one of the packages  \textsf{latexsym}, \textsf{amsfonts} or
     \textsf{amssymb}.

\end{table}


\begin{table}
\caption{Punctuation Symbols}\label{tab:punct}
\vspace{1ex}
\begin{tabular}{*{5}{lp{3.2em}}}
\X,	&\X;	&\X\colon	&\X\ldotp	&\X\cdotp
\end{tabular}
\end{table}

\begin{table}
\caption{Arrow Symbols}\label{tab:arrow}
\vspace{1ex}
\begin{tabular}{*6l}
\X\leftarrow		&\X\longleftarrow	&\X\uparrow	\\
\X\Leftarrow		&\X\Longleftarrow	&\X\Uparrow	\\
\X\rightarrow		&\X\longrightarrow	&\X\downarrow	\\
\X\Rightarrow		&\X\Longrightarrow	&\X\Downarrow	\\
\X\leftrightarrow	&\X\longleftrightarrow	&\X\updownarrow \\
\X\Leftrightarrow	&\X\Longleftrightarrow	&\X\Updownarrow \\
\X\mapsto		&\X\longmapsto		&\X\nearrow	\\
\X\hookleftarrow	&\X\hookrightarrow	&\X\searrow	\\
\X\leftharpoonup	&\X\rightharpoonup	&\X\swarrow	\\
\X\leftharpoondown	&\X\rightharpoondown	&\X\nwarrow	\\
\X\rightleftharpoons	&\X\leadsto$^*$
\end{tabular}

$^*$ Not predefined in \LaTeXe.
     Use one of the packages  \textsf{latexsym}, \textsf{amsfonts} or
     \textsf{amssymb}.

\end{table}

\begin{table}
\caption{Miscellaneous Symbols}\label{tab:ord}
\vspace{1ex}
\begin{tabular}{*8l}
\X\ldots	&\X\cdots	&\X\vdots	&\X\ddots	\\
\X\aleph	&\X\prime	&\X\forall	&\X\infty	\\
\X\hbar 	&\X\emptyset	&\X\exists	&\X\Box$^*$	\\
\X\imath	&\X\nabla	&\X\neg 	&\X\Diamond$^*$ \\
\X\jmath	&\X\surd	&\X\flat	&\X\triangle	\\
\X\ell		&\X\top 	&\X\natural	&\X\clubsuit	\\
\X\wp		&\X\bot 	&\X\sharp	&\X\diamondsuit \\
\X\Re		&\X\|		&\X\backslash	&\X\heartsuit	\\
\X\Im		&\X\angle	&\X\partial	&\X\spadesuit	\\
\X\mho$^*$	&\X.		&\X|
\end{tabular}

$^*$ Not predefined in \LaTeXe.
     Use one of the packages  \textsf{latexsym}, \textsf{amsfonts} or
     \textsf{amssymb}.

\end{table}

\begin{table}
\caption{Variable-sized  Symbols}\label{tab:op}
\vspace{1ex}
\begin{tabular}{*6l}
\X\sum		&\X\bigcap	&\X\bigodot	\\
\X\prod 	&\X\bigcup	&\X\bigotimes	\\
\X\coprod	&\X\bigsqcup	&\X\bigoplus	\\
\X\int		&\X\bigvee	&\X\biguplus	\\
\X\oint 	&\X\bigwedge
\end{tabular}
\end{table}


\begin{table}
\caption{Log-like Symbols}\label{tab:log}
\vspace{1ex}
\begin{tabular}{*8l}
\Z\arccos &\Z\cos  &\Z\csc &\Z\exp &
	   \Z\ker    &\Z\limsup &\Z\min &\Z\sinh \\
\Z\arcsin &\Z\cosh &\Z\deg &\Z\gcd &
	   \Z\lg     &\Z\ln	&\Z\Pr	&\Z\sup  \\
\Z\arctan &\Z\cot  &\Z\det &\Z\hom &
	   \Z\lim    &\Z\log	&\Z\sec &\Z\tan  \\
\Z\arg	  &\Z\coth &\Z\dim &\Z\inf &
	   \Z\liminf &\Z\max	&\Z\sin &\Z\tanh
\end{tabular}
\end{table}


\begin{table}
\caption{Delimiters\label{tab:dels}}
\vspace{1ex}
\begin{tabular}{*8l}
\X(		&\X)		&\X\uparrow	&\X\Uparrow	\\
\X[		&\X]		&\X\downarrow	&\X\Downarrow	\\
\X\{		&\X\}		&\X\updownarrow &\X\Updownarrow \\
\X\lfloor	&\X\rfloor	&\X\lceil	&\X\rceil	\\
\X\langle	&\X\rangle	&\X/		&\X\backslash	\\
\X|		&\X\|
\end{tabular}
\end{table}

\begin{table}
\caption{Large Delimiters\label{tab:ldels}}
\vspace{1ex}
\begin{tabular}{*8l}
\Y\rmoustache&	\Y\lmoustache&	\Y\rgroup&	\Y\lgroup\\[5pt]
\Y\arrowvert&	\Y\Arrowvert&	\Y\bracevert
\end{tabular}
\end{table}

\begin{table}
\caption{Math mode accents}\label{tab:accent}
\vspace{1ex}
\begin{tabular}{*{10}l}
\W\hat{a}     &\W\acute{a}  &\W\bar{a}	  &\W\dot{a}	&\W\breve{a}\\
\W\check{a}   &\W\grave{a}  &\W\vec{a}	  &\W\ddot{a}	&\W\tilde{a}\\
\end{tabular}
\end{table}

\begin{table}
\caption{Some other constructions}\label{tab:other}
\vspace{1ex}
\begin{tabular}{*4l}
\W\widetilde{abc}	&\W\widehat{abc}			\\
\W\overleftarrow{abc}	&\W\overrightarrow{abc} 		\\
\W\overline{abc}	&\W\underline{abc}			\\
\W\overbrace{abc}	&\W\underbrace{abc}			\\[5pt]
\W\sqrt{abc}		&$\sqrt[n]{abc}$&\verb|\sqrt[n]{abc}|	\\
$f'$&\verb|f'|          &$\frac{abc}{xyz}$&\verb|\frac{abc}{xyz}|
\end{tabular}
\end{table}


%%--------------------------------------------------------------------------
\chapter{THIRD APPENDIX}
%%--------------------------------------------------------------------------

Here is the third appendix.
Watch the number of Figure \ref{fig:3} in this appendix.

\begin{figure}[h]
  \centering
  \unitlength 1in	    % make unit length to be 1 inch
  \begin{picture}(6,4)(0,0) % picture coordinates 6 in width, 4 in height,
			    % origin 0,0
    \put(1.4,2.6){\line(3,-1){3.0}} % draw a straight line at slope -1/3
				% starting at (1.4,2.6) of length 3.0
    \put(0,0){\vector(1,0){5.5}}
    \put(0,0){\vector(0,1){3}}
  \end{picture}
  \caption{A Picture Drawn with \LaTeX\ Commands}\label{fig:3}
\end{figure}

%%--------------------------------------------------------------------------
\end{appendices}
%%--------------------------------------------------------------------------




%
% Back stuff
%

% % comment out the following three lines
% if using chapter-wise bibliography

 \backmatter
 \bibliographystyle{nddiss2e}
 \bibliography{example}

\end{document}

% End of ``example.tex''
