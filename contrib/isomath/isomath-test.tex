% isomath-test.tex: test and template for isomath.sty
% ===================================================
%
% Copyright © 2008 Günter Milde
% Released under the terms of the GNU General Public License (v. 2 or later)
%
% General settings
% ----------------
% ::

\documentclass[a4paper]{article}
\usepackage[T1]{fontenc}
% \usepackage[utf8]{inputenc}
\usepackage{amsmath}
\pagestyle{empty}


% Customisation
% -------------
%
% Font packages
% ~~~~~~~~~~~~~
%
% (Auxiliary commands for package and option selection, so that the
% selections can be shown in the test sheet.)
%
% Uncomment the package you want to test ::

\newcommand*{\fontsetup}{\usepackage%
% {ae}%             \sfdefault is cmss
% {anttor}%         \sfdefault is cmss (use iwona)
% {arev}%           normal and sans identic (also in math)
% {beton}%          bitmap fonts
% {concmath}%       
% {cmbright}%       
% {fourier}%        incompatible (uses private font encoding)
% [default]{gfsneohellenic}% no bold
% {hfoldsty}%       bitmap fonts
% {iwona}%          sets cmss as sans font (use iwona)
% [nomathscript]{kpfonts}% "Too many math alphabets" with OMLmathsfit option
% {kerkis}%         no bold math,
% {lmodern}%        \sfdefault is cmss
% {lucidabr}%       commercial, subset (no math) free as bitmap
% {lxfonts}%        normal and sans identic,
[utopia]{mathdesign}% \sfdefault is cmss, provides roman and roman-bold in OML
% [charter]{mathdesign}%
% [garamond]{mathdesign}%
% {mathptmx}%       in mathnormal, Greek is larger than Latin!
% {mbtimes}%        \sfdefault is cmss
% {mathpazo}%       \sfdefault is cmss
% {newcent}%        \sfdefault is cmss
% {pxfonts}%        needs reuseMathAlphabets, no sans-serif
% {tgcursor}%       avoid bitmap tt
% {tgheros}%        avoid bitmap sf
% {tgpagella}%      \sfdefault is cmss
% {tgtermes}%       \sfdefault is cmss
% {txfonts}%        no sans-serif
}

% Isomath setup
% ~~~~~~~~~~~~~
%
% Uncomment the option(s) you want to test ::

\newcommand*{\isomathsetup}{\usepackage[%
% reuseMathAlphabets,%
OMLmathrm,%
OMLmathbf,%
% OMLmathsf,%
% OMLmathsfit,%
% OMLmathtt,%
% OMLmathsans,% backwards compatibility option alias
% rmdefault=zpple,% Mathpazo alternative
% rmdefault=qtxmia,% TeX Gyre Termes math with alternative glyphs
% sfdefault=cmbr,%   default
% sfdefault=iwona,%  Iwona sans (Greek glyphs too close to roman)
sfdefault=fav,%    Arev sans (scale down (ca. 0.87))
scaled=0.875%      scaling for Arev (small letters)
% sfdefault=llcmss,% LX sans (glyphs too close to italic, scale down)
% sfdefault=jkpss,%   Kepler Sans
% scaled=0.95%      scaling for Kepler Sans (small letters)
]{isomath}%
}

% Load customisable packages
% --------------------------
% ::

\fontsetup
\isomathsetup

% Auxiliary definitions
% ---------------------
%
% Re-define \vec to comply with ISO 31::

\renewcommand*{\vec}{\vectorsym}

% Fall-back definition for \mathsfit::

\providecommand*{\mathsfit}[1]%
{\emph{mathsfit not defined (requires \texttt{OMLmathsfit} option)}}


% use Arev as "heavy" sans serif font::

\DeclareFontShape{OML}{fav}{bx}{it}{<-> s * [0.875] zavmbi7m}{}
\SetMathAlphabet{\mathsfbfit}{bold}{OML}{fav}{bx}{it}


% Test sheet
% ----------
% ::

\begin{document}

\section*{Test the isomath Package}


% Print font package, isomath options, and resulting font families::

Font Setup: \detokenize\expandafter{\fontsetup}\\
Isomath:    \detokenize\expandafter{\isomathsetup}\\
Default font families:
\makeatletter
\begin{tabular}[t]{lrlrl}
 Text & serif      & \textsf{\rmdefault} &
        sans-serif & \textsf{\sfdefault}\\
 Math & serif      & \textsf{\isomath@rmdefault} &
        sans-serif & \textsf{\isomath@sfdefault}\\
\end{tabular}
\makeatother

% A teststring with Latin and Greek letters::

\newcommand{\teststring}{%
% capital Latin letters
A,B,C,
% capital Greek letters
\Gamma,\Delta,\Theta,\Lambda,\Xi,\Pi,\Sigma,\Upsilon,\Phi,\Psi,\Omega,
% small Greek letters
\alpha,\beta,\pi,\nu,\omega,
% small Latin letters:
% compare \nu, \omega, v, and w
v,w,
% is there a true italic form of a and g?
a,g,
% digits
0,1,9
}

\subsection*{Math alphabets}

If there are other symbols in place of Greek letters in a math
alphabet, it uses T1 or OT1 font encoding instead of OML.

\begin{eqnarray*}
\mbox{mathnormal} &  & \teststring \\
\mbox{mathit} &  & \mathit{\teststring}\\
\mbox{mathrm} &  & \mathrm{\teststring}\\
\mbox{mathbf} &  & \mathbf{\teststring}\\
\mbox{mathsf} &  & \mathsf{\teststring}\\
\mbox{mathtt} &  & \mathtt{\teststring}
\end{eqnarray*}
 New alphabets bold-italic, sans-serif-italic, and sans-serif-bold-italic.
\begin{eqnarray*}
\mbox{mathbfit}     &  & \mathbfit{\teststring}\\
\mbox{mathsfit}     &  & \mathsfit{\teststring}\\
\mbox{mathsfbfit} &  & \mathsfbfit{\teststring}
\end{eqnarray*}
%
Do the math alphabets match?

$
\mathnormal  {a x \alpha \omega}
\mathbfit    {a x \alpha \omega}
\mathsfbfit{a x \alpha \omega}
\quad
\mathsfbfit{T C \Theta \Gamma}
\mathbfit    {T C \Theta \Gamma}
\mathnormal  {T C \Theta \Gamma}
$

\subsection*{Vector symbols}

Alphabetic symbols for vectors are boldface italic,
$\vec{\lambda}=\vec{e}_{1}\cdot\vec{a}$,
while numeric ones (e.g. the zero vector) are bold upright,
$\vec{a} + \vec{0} = \vec{a}$.

\subsection*{Matrix symbols}

Symbols for matrices are boldface italic, too:%
\footnote{However, matrix symbols are usually capital letters whereas vectors
are small ones. Exceptions are physical Quantities like the force
vector $\vec{F}$ or the electrical field $\vec{E}$.%
}
$\matrixsym{\Lambda}=\matrixsym{E}\cdot\matrixsym{A}.$


\subsection*{Tensor symbols}

Symbols for tensors are sans-serif bold italic,

\[
   \tensorsym{\alpha}  =  \tensorsym{e}\cdot\tensorsym{a}
   \quad \Longleftrightarrow \quad
   \alpha_{ijl}  =  e_{ijk}\cdot a_{kl}.
\]


The permittivity tensor describes the coupling of electric field and
displacement: \[
\vec{D}=\epsilon_{0}\tensorsym{\epsilon}_{\mathrm{r}}\vec{E}\]


\subsection*{Bold math version}

The ``bold'' math version is selected with the commands
\verb+\boldmath+ or \verb+\mathversion{bold}+
\boldmath


\begin{eqnarray*}
\mbox{mathnormal} &  & \teststring \\
\mbox{mathit} &  & \mathit{\teststring}\\
\mbox{mathrm} &  & \mathrm{\teststring}\\
\mbox{mathbf} &  & \mathbf{\teststring}\\
\mbox{mathsf} &  & \mathsf{\teststring}\\
\mbox{mathtt} &  & \mathtt{\teststring}
\end{eqnarray*}
 New alphabets bold-italic, sans-serif-italic, and sans-serif-bold-italic.
\begin{eqnarray*}
\mbox{mathbfit}     &  & \mathbfit{\teststring}\\
\mbox{mathsfit}     &  & \mathsfit{\teststring}\\
\mbox{mathsfbfit} &  & \mathsfbfit{\teststring}
\end{eqnarray*}
%
Do the math alphabets match?

$
\mathnormal  {a x \alpha \omega}
\mathbfit    {a x \alpha \omega}
\mathsfbfit{a x \alpha \omega}
\quad
\mathsfbfit{T C \Theta \Gamma}
\mathbfit    {T C \Theta \Gamma}
\mathnormal  {T C \Theta \Gamma}
$

\subsection*{Vector symbols}

Alphabetic symbols for vectors are boldface italic,
$\vec{\lambda}=\vec{e}_{1}\cdot\vec{a}$,
while numeric ones (e.g. the zero vector) are bold upright,
$\vec{a} + \vec{0} = \vec{a}$.


\subsection*{Matrix symbols}

Symbols for matrices are boldface italic, too:%
\footnote{However, matrix symbols are usually capital letters whereas vectors
are small ones. Exceptions are physical Quantities like the force
vector $\vec{F}$ or the electrical field $\vec{E}$.%
}
$\matrixsym{\Lambda}=\matrixsym{E}\cdot\matrixsym{A}.$


\subsection*{Tensor symbols}

Symbols for tensors are sans-serif bold italic,

\[
   \tensorsym{\alpha}  =  \tensorsym{e}\cdot\tensorsym{a}
   \quad \Longleftrightarrow \quad
   \alpha_{ijl}  =  e_{ijk}\cdot a_{kl}.
\]

The permittivity tensor describes the coupling of electric field and
displacement: \[
\vec{D}=\epsilon_{0}\tensorsym{\epsilon}_{\mathrm{r}}\vec{E}\]

\end{document}
