\NeedsTeXFormat{LaTeX2e}[1995/12/01]

\documentclass[draft]{ltxguide}[1995/11/28]
%\usepackage{draftcopy}

\newcommand{\SJour}{\textsc{SVJour}}

\title{The \SJour\ document class user's guide\\supplement
for\\Archive for Rational Mechanics and Analysis}

\author{\copyright~1998, Springer Verlag Heidelberg\\
   All rights reserved.}

\date{20 June 1998}

\newcommand{\command}[1]{{\ttfamily\upshape\char92#1}}

\begin{document}

\maketitle

\section{Introduction}
\label{sec:intro}
This document describes the \textit{arma} option for the \SJour\
\LaTeXe\ document class. For details on manuscript handling and the
reviewing process we refer to the \emph{Instructions for authors} which
can be found at the Internet address
\texttt{http://link.springer.de/link/service/journals/00205/index.htm}
via the link ``About this Journal" and in the printed journal. For style
matters please consult previous issues of the journal.

\section{Initializing the class} \label{sec:opt}

As explained in the main \emph{User's guide} you can begin a document
for \emph{Archive for Rational Mechanics and Analysis} by including
\begin{verbatim}
   \documentclass[arma]{svjour}
\end{verbatim}
as the first line in your text. All other options are also described
in the main \emph{User's guide}.

\section{Changes to the \SJour\ class}
The header information (usually typeset by using the command
\verb|\maketitle|) will be split so that the basic information (title
and author(s)) relevant for the actual
article appear at the top and address(es) of the author(s) at the end
of your article.

For this purpose the affiliations made with \verb|\inst| in the
\verb|\author| field as well as the \verb|\institute| command have been
withdrawn and a new command \verb|\address| has been added instead.
Place it after the bibliographic section according to the following
scheme:

\begin{verbatim}
   \address{<<address of \\
             first author>>
            \and
            <<address of \\
             second author>>}
\end{verbatim}
If more than one name/address combination is necessary, please use
|\and| to separate them.

Please avoid adding footnote elements to the header. If, for example,
you wish to express your thanks to someone who has supported your work,
place this in the |acknowledgement| environment at the end of
your article, directly before the bibliographic section.

\end{document}
