%%%%%%%%%%%%%%%%%%%%%%%%%%%%%%%%%%%%%%%%%%%%%%%%%%%%%%%%%%%%%%%%%%%%%%%%%%
% An example input file demonstrating the tcfd option of the SVJour      %
% document class for the journal: Theoret. Comput. Fluid Dynamics        %
%%%%%%%%%%%%%%%%%%%%%%%%%%%%%%%%%%%%%%%%%%%%%%%%%%%%%%%%%%%%%%%%%%%%%%%%%%
%
\documentclass[tcfd]{svjour}

\usepackage{graphicx}
%%% \usepackage{times}
%%% \usepackage{mathtime}

\sloppy

%%%%%%%%%%%%%%%%%%%%%%%%%%%%%%%%%%%%%%%%%%%%%%%%%%%%%%%%%%%%%%%%%%%%%%%%%%
% Several Macro's for this article programmed by the Author              %
%%%%%%%%%%%%%%%%%%%%%%%%%%%%%%%%%%%%%%%%%%%%%%%%%%%%%%%%%%%%%%%%%%%%%%%%%%
\newcommand {\ExampleTitle} [4][e]{%
  \pagestyle{empty}
  \begingroup
  \fontfamily{cmr}\selectfont
  \Facies \tituli {##1}
  \SpatiumSuper \titulum {4ex}
  \SpatiumInfra {8ex}
  \RelSize{4}
  \setbox\z@=\hbox{\MakeUppercase{#2}}
  \ifdim \wd\z@ > \hsize
    \RelSize{-5}
  \else
    \RelSize{-4}
  \fi
  \titulus {\RelSize{4}\MakeUppercase{#2}}
  \titulus {\RelSize{4}#3}
  \vfill
  \titulus {\RelSize{2}\scshape
    \ifx #1e\relax
	  \ifx \@undefined \languagename
	  \else
		\gdef \languagename {english}
	  \fi
      technica editions\\[.5ex]ventimiglia $\cdot$ milano%
    \else \ifx #1f\relax
      \'editions technica\\[.5ex]ventimille $\cdot$ milan%
    \else \ifx #1i\relax
      edizioni technica\\[.5ex]ventimiglia $\cdot$ milano%
    \else \ifx #1d\relax
      technica verlag\\[.5ex]ventimiglia $\cdot$ milano%
    \else \ifx #1s\relax
      ediciones technica\\[.5ex]ventimiglia $\cdot$ milano%
    \else \ifx #1l\relax
      albintimili $\cdot$ mediolani\\[.5ex]%
      {\RelSize{-1}e typographeo technico}%
    \fi \fi \fi \fi \fi \fi
  }
  \newpage
  \ifx #4\empty
	\vfill
  \else
    \ \par\vfill
    \RelSize{2}
    \Facies \tituli {\itshape##1}
    \SpatiumSuper {0ex}
    \SpatiumInfra {.5ex} 
	\def \TXN@temp {#4}%
	\ifx \empty	\TXN@temp
	\else
      \titulus {Typeset in the style of}
      \Facies \tituli {\upshape##1}
      \titulus{#4}
    \fi
  \fi
  \newpage
  \endgroup
}



%%%%%%%%%%%%%%%%%%%%%%%%%%%%%%%%%%%%%%%%%%%%%%%%%%%%%%%%%%%%%%%%%%%%%%%%%%

\begin{document}

\title{Subgrid-Scale Models for Compressible Large-Eddy
Simulations\thanks{The authors gratefully acknowledge the support from
the Air Force Off\/ice of Scientific Research, under Grant
Nos. AF/F49620-98-1-0035 (MPM and GVC) and AF/F49620-97-1-0244 (UP),
monitored by D.L.~Sakell. This work was also sponsored by the Army High
Performance Computing Research Center under the auspices of the
Department of the Army, Army Research Laboratory cooperative agreement
number DAAH04-95-2-0003/contract number DAAH04-95-C-0008, the content of
which does not necessarily ref\/lect the position or the policy of the
government, and no off\/icial endorsement should be inferred. A portion
of the computer time was provided by the University of Minnesota
Supercomputing Institute.}}

%%%%%%%%%%%%%%%%%%%%%%%%%%%%%%%%%%%%%%%%%%%%%%%%%%%%%%%%%%%%%%%%%%%%%%%%%%
% You have to protect commands within the "Author"-macro by using the
% \protect-command
%%%%%%%%%%%%%%%%%%%%%%%%%%%%%%%%%%%%%%%%%%%%%%%%%%%%%%%%%%%%%%%%%%%%%%%%%%
\Author{M. Pino Mart\protect{\'\i}n}
{Department of Aerospace Engineering and Mechanics, University of Minnesota,\\
110 Union St. SE, Minneapolis, MN 55455, USA\\
pino@aem.umn.edu}

\Author{Ugo Piomelli}
{Department of Mechanical Engineering, University of Maryland,\\
College Park, MD 20742, USA\\
ugo@eng.umd.edu}

\Author{Graham V. Candler}
{Department of Aerospace Engineering and Mechanics, University of Minnesota,\\
110 Union St. SE, Minneapolis, MN 55455, USA\\
candler@aem.umn.edu}

\commun{Communicated by M.Y. Hussaini}

\date{Received 12 March 1999 and accepted 11 August 1999}

\abstract{An {\it a priori} study of subgrid-scale (SGS) models for the
unclosed terms in the energy equation is carried out using the f\/low
f\/ield obtained from the direct simulation of homogeneous isotropic
turbulence. Scale-similar models involve multiple f\/iltering operations
to identify the smallest resolved scales that have been shown to be the
most active in the interaction with the unresolved SGSs. In the present
study these models are found to give more accurate prediction of the SGS
stresses and heat f\/luxes than eddy-viscosity and eddy-diffusivity
models, as well as improved predictions of the SGS turbulent diffusion,
SGS viscous dissipation, and SGS viscous diffusion.}


\authorrunning{M.P. Mart\protect{\'\i}n, U. Piomelli, and G.V. Candler}
\maketitle

%%%%%%%%%%%%%%%
%% Section 1 %%
%%%%%%%%%%%%%%%

\section{Introduction}
\label{sec:1}

Large-eddy simulation (LES) is a technique intermediate between the
direct numerical simulation (DNS) of turbulent f\/lows and the solution of the
Reynolds-averaged equations. In LES the contribution of the large,
energy-carrying structures to momentum and energy transfer is computed
exactly, and only the effect of the smallest scales of turbulence is
modeled. Since the small scales tend to be more homogeneous and
universal, and less affected by the boundary conditions, than the
large ones, there is hope that their models can be simpler and require
fewer adjustments when applied to different f\/lows than similar models
for the Reynolds-averaged Navier--Stokes equations.

While a substantial amount of research has been carried out into
modeling for the LES of incompressible f\/lows, applications to
compressible f\/lows have been signif\/icantly fewer, due to the increased
complexity introduced by the need to solve an energy equation, which
introduces extra unclosed terms in addition to the subgrid-scale (SGS)
stresses that must be modeled in incompressible f\/lows. Furthermore,
the form of the unclosed terms depends on the energy equation chosen
(internal or total energy, total energy of the resolved f\/ield, or
enthalpy).

Early applications of LES to compressible f\/lows used a transport
equation for the internal energy per unit mass $\varepsilon$
\citep{moi91,elh94} or for the enthalpy per unit mass $h$
\citep{spe88,erl92}. In these equations the SGS heat f\/lux was modeled
in a manner similar to that used for the SGS stresses, while two
additional terms, the SGS pressure-dilatation $\Pi_{\rm dil}$ and the
SGS contribution to the viscous dissipation $\varepsilon_{\rm v}$, were
neglected.

\citet[b]{vre95a} performed {\it a priori} tests using DNS data obtained
from the calculation of a mixing layer at Mach numbers from 0.2 to 0.6.
They found that the SGS pressure-dilatation $\pi_{\rm dil}$ and SGS
viscous dissipation $\varepsilon_{\rm v}$ are of the same order as the
divergence of the SGS heat f\/lux $Q_j$, and that modeling
$\varepsilon_{\rm v}$ improves the results, especially at moderate or
high Mach numbers. They also proposed the use of a transport equation
for the total energy of the f\/iltered f\/ield, rather than either the
enthalpy or the internal energy equations; the same unclosed terms that
appear in the internal energy and enthalpy equations are also present in
this equation.

Very few calculations have been carried out using the transport equation
for the total energy, despite the desirable feature that it is a
conserved quantity, and that all the SGS terms in this equation can be
cast in conservative form. This equation has a different set of unclosed
terms, whose modeling is not very advanced yet. \citet{nor92} performed
calculations of a transitional boundary layer, and modeled only the SGS
heat f\/lux, neglecting all the other terms. \citet{kni98} performed the
LES of isotropic homogeneous turbulence on unstructured grids and
compared the results obtained with the \citet{sma63} model with those
obtained when the energy dissipation was provided only by the
dissipation inherent in the numerical algorithm. They modeled the SGS
heat f\/lux and an SGS turbulent diffusion term, and neglected the SGS
viscous diffusion. \citet{com98} proposed the use of an eddy-diffusivity
model for the sum of the SGS heat f\/lux and SGS turbulent diffusion,
neglecting the SGS viscous diffusion.

In this paper the f\/low f\/ield obtained from a DNS of homogeneous
isotropic turbulence is used to compute the terms in the energy
equations, and evaluate eddy-viscosity and scale-similar models for
their parametrization. We place emphasis on the total energy equation,
both because of the lack of previous studies in the terms that appear in
it, and because of the desirability of solving a transport equation for
a conserved quantity. In the remainder of the paper the governing
equations are presented and the unclosed terms are def\/ined. The DNS
database used for the {\it a priori} tests is described. Finally,
several models for the unclosed terms are presented and tested.

%%%%%%%%%%%%%%%
%% Section 2 %%
%%%%%%%%%%%%%%%

\section{Governing Equations}
\label{sec:2}

To obtain the equations governing the motion of the resolved eddies,
we must separate the large from the small scales. LES is based on the
def\/inition of a f\/iltering operation: a resolved variable, denoted by
an overbar, is def\/ined as \citep{leo74}
\begin{equation}
 \overline{f}({\bf x}) = \int_D f({\bf x'})
 G({\bf x},{\bf x'};\delbar)
 \D{\bf x'},
 \label{eq:filtering}
\end{equation}
where $D$ is the entire domain, $G$ is the f\/ilter function, and
$\delbar$ is the f\/ilter-width associated with the wavelength of the
smallest scale retained by the f\/iltering operation. Thus, the f\/ilter
function determines the size and structure of the small scales.

In compressible f\/lows it is convenient to use Favre-f\/iltering
\citep[b]{fav65a} to avoid the introduction of SGS
terms in the equation of conservation of mass. A Favre-f\/iltered
variable is def\/ined as $\widetilde{f}=\overline{\rho
f}/\overline{\rho}$. In addition to the mass and momentum
equations, one can choose solving an equation for the internal energy,
enthalpy, or total energy. Applying the Favre-f\/iltering operation, we
obtain the resolved transport equations
\begin{equation}
 \label{eq:mass-ff}
 \frac{\partial\rhob}{\partial t}
 + \frac{\partial}{\partial x_{j}} \left(\rhob\util_j\right)
 = 0 ,
\end{equation}
%%%%%%%%%%%%%%%%%%%%%%%%%%%%%%%%%%%%%%%%%%%%%%%%%%%%%%%%%%%%%%%%%%%%%%
\begin{equation}
 \label{eq:mom-ff}
 \frac{\partial\rhob\,\util_i}{\partial t}
 + \frac{\partial}{\partial x_{j}}\left(\rhob\util_i\util_j
 + \pb \delta_{ij} - \widetilde{\sigma}_{ji} \right)
 = - \frac{\partial\tau_{ji}}{\partial x_{j}} ,
\end{equation}
%%%%%%%%%%%%%%%%%%%%%%%%%%%%%%%%%%%%%%%%%%%%%%%%%%%%%%%%%%%%%%%%%%%%%%
\begin{equation}
 \label{eq:int-en-ff}
 \frac{\partial (\rhob\,\widetilde{\varepsilon}\,)}{\partial t}
 + \frac{\partial}{\partial x_{j}}
 \left(\rhob\util_j\widetilde\varepsilon\right)
 + \frac{\partial\widetilde{q}_j}{\partial x_j}
 + \pb\Stkk - \widetilde{\sigma}_{ji}\Stij
 = - C_{\rm v}\frac{\partial Q_j}{\partial x_{j}} - \Pi_{\rm dil} +
 \varepsilon_{\rm v} ,
\end{equation}
%%%%%%%%%%%%%%%%%%%%%%%%%%%%%%%%%%%%%%%%%%%%%%%%%%%%%%%%%%%%%%%%%%%%%%
\begin{equation}
 \label{eq:enth-ff}
 \frac{\partial (\rhob\,\widetilde{h})}{\partial t}
 + \frac{\partial}{\partial x_j}\left(\rhob\util_j\htil\right)
 + \frac{\partial\widetilde{q}_j}{\partial x_j}
 - \frac{\partial\pb}{\partial t}
 - \util_j\frac{\partial\pb}{\partial x_j}
 - \widetilde{\sigma}_{ji}\Stij =
 - C_{\rm v}\frac{\partial Q_j}{\partial x_j}
 - \Pi_{\rm dil}
 + \varepsilon_{\rm v} ,
\end{equation}
%%%%%%%%%%%%%%%%%%%%%%%%%%%%%%%%%%%%%%%%%%%%%%%%%%%%%%%%%%%%%%%%%%%%%%
\begin{equation}
 \label{eq:tot-en-ff}
 \frac{\partial}{\partial t} (\rhob\,\Etil)
 + \frac{\partial}{\partial x_j}
 \left[(\rhob\,\Etil + \pb)\util_j + \widetilde{q}_j
 - \widetilde{\sigma}_{ij}\util_i \right] =
 - \frac{\partial}{\partial x_j}\left(\gamma C_{\rm v}Q_j
 + {\textstyle\half}{\cal J}_j
 - {\cal D}_j \right) .
\end{equation}
%%%%%%%%%%%%%%%%%%%%%%%%%%%%%%%%%%%%%%%%%%%%%%%%%%%%%%%%%%%%%%%%%%%%%
Here $\rho$ is the density, $u_j$ is the velocity in the $x_j$ direction,
$p$ is the pressure, $\varepsilon=c_{\rm v} T$ is the internal energy
per unit mass, $T$ is the temperature; $h=\varepsilon + p/\rho$ is the
enthalpy per unit mass; $E=\varepsilon +u_iu_i/2$ is the total energy per
unit mass, and the diffusive f\/luxes are given by
\begin{equation}
 \label{eq:sig-hat}
 \widetilde{\sigma}_{ij} = 2\widetilde{\mu} \Stij
 - {\textstyle\frac{2}{3}}\widetilde{\mu} \delta_{ij}\Stkk, \qquad
 \widetilde{q}_j = - \widetilde{k}\frac{\partial\Ttil}{\partial x_j} ,
\end{equation}
where $S_{ij}=\frac{1}{2}
(\partial u_i/\partial x_i + \partial u_j/\partial x_i)$ is
the strain rate tensor, and $\widetilde{\mu}$ and $\widetilde{k}$ are the
viscosity and thermal conductivity corresponding to the f\/iltered temperature
$\Ttil$.

The effect of the SGSs appears on the right-hand side of the governing
equations through the SGS stresses $\tij$; SGS heat f\/lux $Q_j$; SGS
pressure-dilatation $\Pi_{\rm dil}$; SGS viscous dissipation $\varepsilon_{\rm v}$;
SGS turbulent diffusion $\partial{\cal J}_j/\partial x_j$; and SGS viscous
diffusion $\partial{\cal D}_j/\partial x_j$. These quantities are
def\/ined as
\begin{eqnarray}
 \label{eq:tauij}
 \tij & = & \rhob(\widetilde{u_iu_j}-\util_i\util_j), \\
 \label{eq:qj}
 Q_j & = & \rhob \left(\widetilde{u_j T}-\util_j\widetilde{T}\right), \\
 \label{eq:pdil}
 \Pi_{\rm dil} & = & \overline{p\Skk}-\pb\Stkk, \\
 \label{eq:vdiss}
 \varepsilon_{\rm v} & = & \overline{\sigma_{ji}\Sij}
 - \widetilde{\sigma}_{ji}\Stij,\\
 \label{eq:tdiff}
 {\cal J}_j & = & \rhob\left(\widetilde{u_ju_ku_k} -
 \util_j\widetilde{u_ku_k} \right), \\
 \label{eq:vdiff}
 {\cal D}_j & = & \overline{\sigma_{ij}u_i}
 -\widetilde{\sigma}_{ij}\util_i .
\end{eqnarray}
The equation of state has been used to express pressure-gradient and
pressure-diffusion correlations in terms of $Q_j$ and $\Pi_{\rm dil}$. It
is also assumed that $\overline{\mu(T)\Sij} \simeq \mu(\Ttil)\Stij$,
and that an equivalent equality involving the thermal conductivity
applies. \citet{vre95b} performed {\it a priori}
tests using DNS data obtained from the calculation of a mixing layer
at Mach numbers in the range 0.2--0.6, and concluded that neglecting
the nonlinearities of the diffusion terms in the momentum and energy
equations is acceptable.

%%%%%%%%%%%%%%%
%% Section 3 %%
%%%%%%%%%%%%%%%

\section{{\em A priori} Test}
\label{sec:3}

One method to evaluate the performance of models for LES or RANS
calculations is the {\it a priori} test, in which the velocity
f\/ields obtained from a direct simulation are f\/iltered to yield the
exact SGS terms, and the f\/iltered quantities are used to assess the
accuracy of the parametrization. The database used in this study was
obtained from the calculation of homogeneous isotropic turbulence
decay.

The Navier--Stokes equations were integrated in time using a
fourth-order Runge--Kutta method.\break The spatial derivatives were
computed using an eighth-order accurate central f\/inite-difference
scheme. The\pagebreak\ results have been validated by comparison with the
DNS data of \citet[1999]{mar98}. The simulations were performed
on grids with 256$^3$ points. The computational domain is a periodic
box with length 2$\pi$ in each dimension. The f\/luctuating f\/ields were
initialized as in \citet{ris97}.

The calculation was performed at a Reynolds number
$Re_{\lambda}=u'\lambda/\nu=50$, where
$\lambda=\langle u^2\rangle^{1/2}/$\break
$\langle{(\partial u/\partial x)^2}\rangle^{1/2}$ is the Taylor
microscale and $u'=(u_iu_i)^{1/2}$ is the turbulence intensity, and
at a turbulent Mach number $M_t=u'/a=0.52$, where $a$ is the speed
of sound. The initial f\/low f\/ield is allowed to evolve for four
dimensionless time units $\tau_t=\lambda/u'$, so that the energy
spectrum may develop an inertial range that decays as $k^{-5/3}$,
where $k$ is the nondimensional wave number.

The f\/iltered f\/ields were obtained using a top-hat f\/ilter, which is
def\/ined in one dimension as
\begin{equation}
 \label{eq:tophat}
 \overline{f}_i = \frac{1}{2n}
 \left( f_{i-{n}/{2}}+2\sum_{i-{n}/{2}+1}^{i+{n}/{2}-1}f_{i}+f_{i+{n}/{2}}\right) .
\end{equation}
Various f\/ilter-widths $\delbar=n\Delta$ (where $\Delta$ is the grid
size and $n=4$, 8, 16, and 32) were used. Note that the grid
resolution is high enough that $n=2$ would correspond to a DNS. The
location of the various f\/ilter cutoffs along the energy spectrum at
$t/\tau_t=6.5$ are shown in Figure~\ref{fig:fig01}; they cover the decaying range of
the spectrum ($n=4$), the inertial range ($n=8$ and 16), and the
energy-containing range ($n=32$). With the f\/ilters used,
respectively, 5\%, 15\%, 40\%, and 70\% of the total turbulent kinetic
energy resides in the SGSs. The two intermediate values are
representative of actual LES calculations, in which the SGS kinetic energy\break
is typically between 15\% and 30\% of the total energy. A higher
percentage of SGS energy in general indicates an under-resolved
calculation. In the following, results will be shown for
$\delbar=8\Delta$, except when evaluating the effect of f\/ilter-width.

The accuracy of a model is evaluated by computing the exact term $R$
and its model representation $M$ and comparing the two using the
correlation coeff\/icient $C(R)$ and the root-mean-square (rms) amplitudes
$\langle (R-\langle R\rangle)^2\rangle^{1/2}$ and $\langle(M-\langle
M\rangle)^2\rangle^{1/2}$. The correlation coeff\/icient is given by
\begin{equation}
 \label{eq:corr-coeff}
 C(R) = \frac{ \langle(R-\langle R \rangle) (M-\langle M \rangle) \rangle}
 {\left(\langle (R-\langle R \rangle)^2 \rangle
 \langle(M-\langle M \rangle)^2 \rangle\right)^{1/2}} ,
\end{equation}
where the brackets $\langle\cdot\rangle$ denote averaging over the
computational volume. A ``perfect'' model would give a correlation
coeff\/icient of 1. In the following, the quantities plotted are made
nondimensional using the initial values of $\rho$, $u'$, and
$\lambda$.

\begin{figure}[t]
%\centering
%\includegraphics[height=51mm]{01fig.eps}
\vspace{51mm}% to simulate the figure
\caption{Energy spectrum; \diam, location of the f\/ilter-widths used
in the {\it a priori} test; \solid $k^{-5/3}$ slope; \dotted, DNS. $q^2 = u_iu_i$, and $\eta$ is
the Kolmogorov length scale.}
\label{fig:fig01}
\end{figure}

%%%%%%%%%%%%%%%
%% Section 4 %%
%%%%%%%%%%%%%%%

\section{Models for the Momentum Equation}
\label{sec:4}

The SGS stresses (\ref{eq:tauij}) are the only unclosed term that
appears in the momentum equation. Various types of models have been
devised to represent the SGS stresses. Eddy-viscosity models
try to reproduce the global exchange of energy between the resolved
and unresolved stresses by mimicking the drain of energy associated
with the turbulence energy cascade. Yoshizawa (1986)
proposed an eddy-viscosity model for weakly compressible turbulent
f\/lows using a multiscale direct-interaction approximation method.
The anisotropic part of the SGS stresses is parametrized using the
\citet{sma63} model, while the SGS energy $\tkk$ is
modeled separately:
\begin{equation}
 \label{eq:tauij-yoshi}
 \tij - \frac{\delta_{ij}}{3}\qsgs =
 - C_s^22\delbar^2\rhob|\St|\left(\Stij
 -\frac{\delta_{ij}}{3}\Stkk\right) = C_s^2\aij
 , \qquad
 \qsgs = C_I 2\rhob\delbar^2|\St|^2 = C_I \alpha,
\end{equation}
with $C_s=0.16$, $C_I=0.09$, and $|\St|=(2 \St_{ij}\St_{ij})^{1/2}$.

\citet{moi91} proposed a modif\/ication of the
eddy-viscosity model (\ref{eq:tauij-yoshi}) in which the two model
coeff\/icients were determined dynamically, rather than input {\it a
priori}, using the Germano identity $\Lij = T_{ij} -
\widehat{\tau_{ij}}$ \citep{ger92}, which relates the SGS
stresses $\tij$ to the ``resolved turbulent stresses''
$\Lij=\left(\widehat{\overline{\rho u_i}\,\overline{\rho
 u_j}/\rhob}\right) - \widehat{\overline{ \rho u_i }}\,
\widehat{\overline{ \rho u_j}}/\rhobh$, and the subtest stresses
$\Tij=\rhobh\breve{\widetilde{u_iu_j}} - \rhobh\utbi\utbj$
(where $\breve{\widetilde{f}}=\widehat{\overline{\rho f}}/\rhobh$, and the
hat represents the application of the test f\/ilter $\widehat{{G}}$ of
characteristic width $\delhat=2\delbar$)
that appear if the f\/ilter $\widehat{{G}}$ is applied to (\ref{eq:mom-ff}).
\citet{moi91} determined the model coeff\/icients by
substituting (\ref{eq:tauij-yoshi}) into the Germano identity and
contracting with $\Stij$. In the present paper the contraction
proposed by \citet{lil92} to minimize the error in a
least-squares sense are used instead. Accordingly, the two model
coeff\/icients for the dynamic eddy-viscosity (DEV) model will be given
by
\begin{equation}
 \label{eq:coeff_dsm}
 C = C_s^2 = \frac{ \langle {\cal L}_{ij}M_{ij}\rangle }
 { \langle M_{kl}M_{kl}\rangle } , \qquad
 C_I = \frac{ \langle {\cal L}_{kk}\rangle }
 { \langle\beta-\widehat{\alpha}\rangle } ,
\end{equation}
where $\beta_{ij} = -2\delhat^2\rhobh|\Stb|
(\Stbij-\delta_{ij}\Stbkk/3)$, $\Mij=\bij-\widehat{\aij}$, and
$\beta=2\delhat^2 \rhobh|\Stb\,|^2$.

Scale-similar models are based on the assumption that the most active
SGSs are those closer to the cutoff, and that the scales
with which they interact are those immediately above the cutoff wave number
\citep{bar80}. Thus, scale-similar models employ
multiple operations to identify the smallest resolved scales and use
the smallest ``resolved'' stresses to represent the SGS stresses.
Although these models account for the local energy events, they
underestimate the dissipation.

\citet{spe88} proposed the addition of a scale-similar
part to the eddy-viscosity model of \citet{yos86}
introducing the mixed model. In this way, the eddy-viscosity
contribution provides the dissipation that is underestimated by purely
scale-similar models. This mixed model was also used by \citet{erl92}
and \citet{zan92}, and is given by
\begin{equation}
 \label{eq:tauij-sezhu}
 \tij - \frac{\delta_{ij}}{3}\qsgs = C_s\aij + \Aij
 - \frac{\delta_{ij}}{3} \Akk , \qquad
 \qsgs = C_I \alpha + \Akk,
\end{equation}
where $\Aij=\rhob(\widetilde{\uti\util}_j-\utti\uttj)$.

\citet{erl92} tested the constant
coeff\/icient model {\it a priori} by comparing DNS and LES
results of compressible isotropic turbulence and found good agreement
in the dilatational statistics of the f\/low, as well as high
correlation between the exact and the modeled stresses.
\citet{zan92} compared the DNS and LES results of isotropic
turbulence with various initial ratios of compressible to total
kinetic energy. They obtained good agreement for the evolution of
quantities such as compressible kinetic energy and f\/luctuations of the
thermodynamic variables.

Dynamic model adjustment can be also applied to the mixed model
(\ref{eq:tauij-sezhu}), to yield the dynamic mixed model (DMM)
\begin{equation}
 \label{eq:c-dmm}
 C = \frac{\langle\Lij\Mij\rangle - \langle\Nij\Mij\rangle}
 {\langle\Mlk\Mlk\rangle} , \qquad
 C_I = \frac{\langle\Lkk-\Nkk\rangle}
 {\langle \beta-\widehat{\alpha}\rangle},
\end{equation}
with $\Bij=\rhobh(\breve{\widetilde{\utbi\utbj}} - \utbtbi\utbtbj)$,
and $\Nij=\Bij-\widehat{\Aij}$.

An issue that requires some attention is the necessity to model
separately the trace of the SGS stresses $\qsgs$. \citet{yos86},
\citet{moi91}, and \citet{spe88} proposed a separate model
for this term. \citet{erl92} conjectured that, for
turbulent Mach numbers $M_t<0.4$ this term is negligible; their DNS of
isotropic turbulence conf\/irm this conjecture. \citet{zan92}
conf\/irmed these results {\it a posteriori}:
they ran calculations with $0\leq C_I\leq0.066$ (the latter value is ten
times higher than that predicted by the theory) and observed little
difference in the results.

\citet{com98} proposed incorporating this term into a
modif\/ied pressure ${\cal P}$. This leads to the presence of an
additional term in the equation of state, which takes the form
\begin{equation}
 \label{eq:cl-state}
 {\cal P} = \rhob R\Ttil + \frac{3\gamma-5}{6}\qsgs ;
\end{equation}
for $\gamma=\frac{5}{3}$ the additional term is zero, and for $\gamma=\frac{7}{5}$ it
might be negligible, unless $M_t$ is very large. This observation can
be used to explain {\it a posteriori} the insensitivity of the LES
results to the value of $C_I$ discussed by \citet{zan92}:
the SGS stress trace can be approximately incorporated in the pressure
with no modif\/ication to the equation of state. Another factor may be
that both the calculations by \citet{erl92} and those by \citet{zan92}
used mixed models, in which the
scale-similar part gave a contribution to the normal SGS stresses.
Thus, $\qsgs$ is taken into account, at least partially, by the
scale-similar contribution.

If the mixed model is used, the trace of the SGS stresses can be
parameterized without requiring a separate term. A one-coeff\/icient
dynamic mixed model (DMM-1) would be of the form
\begin{equation}
 \label{eq:tauij-dmm-oc}
 \tij = C \aij + \Aij ,
\end{equation}
with
\begin{equation}
 \label{eq:c-dmm-oc}
 C = \frac{\langle\Lij\Mij\rangle - \langle\Nij\Mij\rangle}
 {\langle\Mlk\Mlk\rangle}.
\end{equation}

\begin{figure}[b]
%\centering
%\includegraphics[height=84mm]{02fig.eps}
\vspace{84mm}% to simulate the figure
\caption{{\it A priori} comparison of the normal SGS stresses
$\tau_{11}$. (a) Correlation coeff\/icient and (b) nondimensional rms magnitude. \solid,
Eddy-viscosity model DEV (\protect\ref{eq:tauij-yoshi})--(\protect\ref{eq:coeff_dsm});
\dashed, two-coeff\/icient mixed model DMM
(\protect\ref{eq:tauij-sezhu})--(\protect\ref{eq:c-dmm}); \chndot, one-coeff\/icient mixed model
DMM-1 (\protect\ref{eq:tauij-dmm-oc})--(\protect\ref{eq:c-dmm-oc}); \trian, DNS.}
\label{fig:fig02}
\end{figure}

The models DEV (\ref{eq:tauij-yoshi})--(\ref{eq:coeff_dsm}), DMM
(\ref{eq:tauij-sezhu})--(\ref{eq:c-dmm}), and DMM-1
(\ref{eq:tauij-dmm-oc})--(\ref{eq:c-dmm-oc}) are evaluated in
Figures~\ref{fig:fig02}--\ref{fig:fig04}. Figure~\ref{fig:fig02}(a) shows that the DMM-1 model gives the highest
correlation for the diagonal components of the SGS stress tensor;
Figure~\ref{fig:fig02}(b) shows that neither the eddy-viscosity model nor the
two-coeff\/icient mixed model DMM predict the rms of the SGS stresses
accurately. The DMM-1 model gives the most accurate prediction among
those tested.

\begin{figure}[t]
%\centering
%\includegraphics[height=86mm]{03fig.eps}
\vspace{86mm}% to simulate the figure
\caption{{\it A priori} comparison of the off-diagonal SGS stresses
$\tau_{12}$. (a) Correlation coeff\/icient and (b) nondimensional rms magnitude. \solid,
Eddy-viscosity model DEV (\protect\ref{eq:tauij-yoshi})--(\protect\ref{eq:coeff_dsm}); \dashed,
two-coeff\/icient mixed model DMM (\protect\ref{eq:tauij-sezhu})--(\protect\ref{eq:c-dmm}) and
one-coeff\/icient mixed model DMM-1 (\protect\ref{eq:tauij-dmm-oc})--(\protect\ref{eq:c-dmm-oc});
\trian, DNS.}
\label{fig:fig03}
\end{figure}

Figure~\ref{fig:fig03}(a) shows the correlation coeff\/icient for the off-diagonal
components of the SGS stress. As in incompressible f\/lows, the
eddy-viscosity model gives very poor correlation (near 0.2), while much
improved results are obtained with the mixed models. Note that the
correlation coeff\/icient for DMM and DMM-1 overlap in the f\/igure.
Figure~\ref{fig:fig03}(b) shows the rms of $\tau_{12}$. DEV underpredicts the rms
magnitude of the exact term, while DMM and DMM-1 slightly overpredict it.

The coeff\/icient $C_s$ remained nearly constant at a value of 0.15
throughout the calculation, consistent with the theoretical arguments
\citep{yos86}. The coeff\/icient of the SGS energy, $C_I$, on the
other hand, has a value three times higher than predicted by the
theory, consistent with the results of \citet{moi91}.

\begin{figure}[t]
%\centering
%\includegraphics[height=46mm]{04fig.eps}
\vspace{46mm}% to simulate the figure
\caption{Nondimensional rms magnitude of $\tau_{11}$ versus
f\/ilter-width at $t/\tau_t=6.5$. \solid, Eddy-viscosity model DEV
(\protect\ref{eq:tauij-yoshi})--(\protect\ref{eq:coeff_dsm}); \dashed, two-coeff\/icient mixed
model DMM (\protect\ref{eq:tauij-sezhu})--(\protect\ref{eq:c-dmm}); \chndot, one-coeff\/icient
mixed model DMM-1 (\protect\ref{eq:tauij-dmm-oc})--(\protect\ref{eq:c-dmm-oc}); \trian,
DNS.}
\label{fig:fig04}
\end{figure}

Figure~\ref{fig:fig04} shows the rms magnitude of $\tau_{11}$ versus the
f\/ilter-width, at time $t/\tau_t=6.5$. For very small f\/ilter-widths
($\delbar/\Delta=4$), all the models are accurate, ref\/lecting the
capability of dynamic models to turn off the model contribution when
the grid becomes suff\/iciently f\/ine to resolve all the turbulent
structures (models with constants assigned {\it a priori}, such as
the \citet{sma63} model, do not have this characteristic). For
$\delbar/\Delta=8$, consistent with the results shown above, the
one-coeff\/icient mixed model DMM gives the most accurate predictions.
For intermediate f\/ilter-widths, up to $\delbar/\Delta=16$, the best
prediction is given by the DMM-1 model; when this f\/ilter-width is used
the unresolved scales contain a considerable amount of energy, 40\%.
For $\delbar/\Delta=32$, it appears that the DMM model predicts the
rms magnitude accurately. However, since the DMM model overpredicts
the rms signif\/icantly for $\delbar/\Delta=8$ and 16, the accurate
prediction given by DMM for $\delbar/\Delta=32$ is a coincidence.
When $\delbar/\Delta=32$ the SGSs contain a large contribution
from the energy-containing eddies (70\% of the energy is in the SGS);
since $\delbar/\Delta=32$ is not in the inertial range the
assumptions on which LES modeling is based fail. The same results are found
for $\tau_{12}$ (not shown).

\begin{figure}[t]
%\centering
%\includegraphics[height=90mm]{05fig.eps}
\vspace{90mm}% to simulate the figure
\caption{Comparison of unclosed terms in the energy equations. (a)
Nondimensional terms in the internal energy or enthalpy equations and (b) nondimensional terms
in the total energy equation. \solid, Divergence of the SGS heat f\/lux, $C_{\rm v}\ \partial
Q_j/\partial x_j$; \chndot, SGS viscous dissipation $\varepsilon_{\rm v}$; \dashed, pressure
dilatation $\Pi_{\rm dil}$; \chndotdot, divergence of the SGS heat f\/lux, $\gamma C_{\rm
v}\ \partial Q_j/\partial x_j$; \dotted, SGS turbulent diffusion $\partial{\cal J}_j/\partial
x_j$; \ldash, SGS viscous diffusion $\partial{\cal D}_j/\partial x_j$.}
\label{fig:fig05}
\end{figure}

%%%%%%%%%%%%%%%
%% Section 5 %%
%%%%%%%%%%%%%%%

\section{Models for the Energy Equations}
\label{sec:5}

Figure~\ref{fig:fig05} compares the magnitude of the unclosed terms
appearing in the internal-energy and enthalpy equations
(\ref{eq:int-en-ff}) and (\ref{eq:enth-ff}), respectively (Figure~\ref{fig:fig05}(a)) and
in the total energy equation (Figure~\ref{fig:fig05}(b)). Unlike in the mixing layer
studied by \citet{vre95b}, in this f\/low the pressure
dilatation $\Pi_{\rm dil}$ is negligible, and the viscous dissipation
$\varepsilon_{\rm v}$ is one order of magnitude smaller than the divergence of
the SGS heat f\/lux. In the total energy equation (\ref{eq:tot-en-ff}),
the SGS turbulent diffusion $\partial{\cal J}_j/\partial x_j$ is
comparable with the divergence of the SGS heat f\/lux and the SGS viscous
diffusion is one order of magnitude smaller than the other terms. In
this section several models for the more signif\/icant terms are examined.

%%%%%%%%%%%%%%%%%%%%
%% Subsection 5.1 %%
%%%%%%%%%%%%%%%%%%%%

\subsection{SGS Heat Flux}
\label{sec:5.1}

The simplest approach to modeling the SGS heat f\/lux $Q_j$ is to use an
eddy-diffusivity model of the form
\begin{equation}
 \label{eq:eddy-diff}
 Q_j = -\frac{\rhob\nu_{\rm T}}{\prt}\frac{\partial \Tt}{\partial x_j}
 = -C \frac{\delbar^2\rhob|\St|}{\prt}\frac{\partial \Tt}{\partial x_j},
\end{equation}
where $C$ is the eddy-viscosity coeff\/icient that can be either
assigned if a model of the form (\ref{eq:tauij-yoshi}) is used, or
computed dynamically as in (\ref{eq:coeff_dsm}). The turbulent
Prandtl number $\prt$ can be also f\/ixed or calculated dynamically
according to\pagebreak
\begin{equation}
 \label{eq:prt-dev}
 \prt = \frac{C\langle T_kT_k\rangle}{\langle\Kj T_j\rangle},
\end{equation}
where
\begin{equation}
 T_j= -\delhat^2\rhobh|\Stb|\frac{\partial\Ttb}{\partial x_j}
 +\delbar^2\widehat{\rhob|\St|\frac{\partial\Tt}{\partial x_j}}
 ,\qquad
 \Kj = \left(\frac{\widehat{\overline{\rho u_j}
 \,\overline{\rho T}}}{\rhob} \right)
 - \frac{\widehat{\overline{ \rho u_j }}\,
 \widehat{\overline{ \rho T }}}{\rhobh}.
\end{equation}

A mixed model of the form
\begin{equation}
 \label{eq:qj-dmm}
 Q_j = -C\frac{\delbar^2\rhob|\St|}{\prt}\frac{\partial\Tt}{\partial x_j}
 +\rhob \left(\widetilde{\utj\Tt} - \uttj\Ttt\right)
\end{equation}
was proposed by \citet{spe88}. The model coeff\/icients
$C$ and $\prt$ can again be assigned or adjusted dynamically
according to (\ref{eq:c-dmm}) and
\begin{equation}
 \label{eq:prt-dmm}
 \prt = C \frac{\langle T_kT_k\rangle}
 {\langle\Kj T_j\rangle-\langle V_jT_j\rangle } ,
\end{equation}
with
\begin{equation}
 V_j = \rhobh\left(\breve{\widetilde{\utbj\Ttb}}
 - \utbtbj\Ttbtb \right) -
 \widehat{\rhob \left(\widetilde{\utj\Tt} - \uttj\Ttt\right)}.
\end{equation}

\begin{figure}[t]
%\centering
%\includegraphics[height=85mm]{06fig.eps}
\vspace{85mm}% to simulate the figure
\caption{{\it A priori} comparison of the SGS heat f\/lux $Q_j$. (a)
Correlation coeff\/icient and (b) nondimensional rms magnitude. \solid, Eddy-diffusivity model
(\protect\ref{eq:eddy-diff}), $Pr_{\rm T}=0.7$; \dashed, eddy-diffusivity model
(\protect\ref{eq:eddy-diff}), Prandtl number adjusted according to (\protect\ref{eq:prt-dev});
\chndot, mixed model (\protect\ref{eq:qj-dmm})--(\protect\ref{eq:prt-dmm}); \trian$\!$,\ $\,$
DNS.}
\label{fig:fig06}
\end{figure}

Figure~\ref{fig:fig06}(a) shows the correlation coeff\/icient for the three models
described above. Both eddy-viscosity models overlap on the plot
giving a poor correlation factor, roughly 0.2, whereas the mixed model
gives a correlation above 0.6. Both eddy viscosity models
under-predict the rms of the exact $Q_j$, shown in Figure~\ref{fig:fig06}(b), while the
mixed model is more accurate.
The mixed model maintains accuracy for all f\/ilter-widths
$\delbar/\Delta\leq16$ (Figure~\ref{fig:fig07}).

\begin{figure}[t]
%\centering
%\includegraphics[height=45mm]{07fig.eps}
\vspace{45mm}% to simulate the figure
\caption{Nondimensional rms magnitude of $Q_j$ versus f\/ilter-width
at $t/\tau_t=6.5$. \solid, eddy-diffusivity model (\protect\ref{eq:eddy-diff}), $Pr_{\rm
T}=0.7$; \dashed, eddy-diffusivity model (\protect\ref{eq:eddy-diff}), Prandtl number adjusted
according to (\protect\ref{eq:prt-dev}); \chndot, mixed model
(\protect\ref{eq:qj-dmm})--(\protect\ref{eq:prt-dmm}); \trian, DNS.}
\label{fig:fig07}
\end{figure}

%%%%%%%%%%%%%%%%%%%%
%% Subsection 5.2 %%
%%%%%%%%%%%%%%%%%%%%

\subsection{SGS Viscous Dissipation}
\label{sec:5.2}

The other term in the enthalpy or internal energy equations that was
found to be signif\/icant in the present f\/low is the viscous dissipation
$\varepsilon_{\rm v}$. In this section the three models proposed by
\citet{vre95b} are tested:\pagebreak
\begin{eqnarray}
 \label{eq:ev-dss}
 \varepsilon_{\rm v}^{(1)} & = &
 C_{\varepsilon1} \left(\widetilde{\widetilde{\sigma}_{ji}\Stij}
 -\widetilde{\widetilde{\sigma}}_{ij}\Sttij
 \right) ;\\
 \label{eq:ev-dtone}
 \varepsilon_{\rm v}^{(2)} & = & C_{\varepsilon2}\rhob \widetilde{q}^3/\delbar ,
 \qquad \widetilde{q}^2 \sim\delbar^2|\St|^2 ; \\
 \label{eq:ev-dttwo}
 \varepsilon_{\rm v}^{(3)} & = & C_{\varepsilon3}\rhob \widetilde{q}^3\delbar ,
 \qquad \widetilde{q}^2 \sim \widetilde{\utk\util}_k-\uttk\uttk .
\end{eqnarray}
The f\/irst is a scale-similar model; the second and third represent the
SGS dissipation as the ratio between the cube of the SGS velocity
scale, $\widetilde{q}$, and the length scale. The velocity scale can
be obtained using either the \citet{yos86} model, as in
(\ref{eq:ev-dtone}), or the scale-similar model as in
(\ref{eq:ev-dttwo}). \citet{vre95b} f\/ixed the values
of the coeff\/icients by matching the rms magnitude of the modeled and
exact terms obtained from the {\it a priori} test, and obtained
$C_{\varepsilon1}=8$, $C_{\varepsilon2}=1.6$, and $C_{\varepsilon3}=0.6$. In
the present study the dynamic procedure will be used instead to
determine the coeff\/icients. The analog of the Germano identity for
this term reads
\begin{equation}
 \left\langle\widehat{\widetilde\sigma_{ji}\Stij}
 - \widehat{\overline{\rho\sigma_{ij}}}\,
 \widehat{\overline{\rho\Sij}}/\rhobh^2\right\rangle =
 \left\langle E_{\rm v}^{(n)}-\widehat{\varepsilon_{\rm v}^{(n)}}\right\rangle,
\end{equation}
and the modeled terms $\varepsilon_{\rm v}^{(n)}$ can be given
respectively by (\ref{eq:ev-dss})--(\ref{eq:ev-dttwo}), while the
$E_{\rm v}^{(n)}$ are
\begin{eqnarray}
 E_{\rm v}^{(1)} & = &
 C_{\varepsilon1}
 \left(\breve{\widetilde{\breve{\widetilde{\sigma}}_{ji}\Stbij}}
 -\breve{\widetilde{\breve{\widetilde{\sigma}}}}_{ij}\Stbtbij
 \right) ;\\
 E_{\rm v}^{(2)} & = & C_{\varepsilon2} \rhobh \breve{\widetilde{q}}^3/\delhat ,
 \qquad \breve{\widetilde{q}}^2 \sim\delhat^2|\Stb|^2 ; \\
 E_{\rm v}^{(3)} & = & C_{\varepsilon3} \rhobh \breve{\widetilde{q}}^3/\delhat ,
 \qquad \breve{\widetilde{q}}^2 \sim \breve{\widetilde{\utbi\utbj}} -
 \utbtbi\utbtbj.
\end{eqnarray}

\begin{figure}[t]
%\centering
%\includegraphics[height=85mm]{08fig.eps}
\vspace{85mm}% to simulate the figure
\caption{{\it A priori} comparison of the SGS viscous dissipation
$\varepsilon_{\rm v}$. (a) Correlation coeff\/icient and (b) nondimensional rms magnitude.
\solid, scale-similar model (33); \dashed, dynamic model (34); \chndot, dynamic model (35);
\trian, DNS.}
\label{fig:fig08}
\end{figure}

Figure~\ref{fig:fig08}(a) shows the correlation coeff\/icient for the three models.
The scale-similar model gives the highest correlation. The use of a
velocity scale obtained from the scale-similar assumption, however,
results in improved prediction of the rms magnitude; using
$q\sim\delbar|\St|$ yields a signif\/icant overprediction of the rms.
The values of the coeff\/icients obtained from the dynamic adjustment in
this f\/low are signif\/icantly lower than those obtained in the mixing
layer by \citet{vre95b}. For the particular
f\/ilter-width shown, we obtained $C_{\varepsilon 1}=2.4$, and $C_{\varepsilon
 2}=0.03$, while $C_{\varepsilon3}$ increased monotonically in time from
0.25 to 0.4. The fact that with these values the f\/irst and third
models match the rms magnitude of the exact term indicates a lack of
universality of these constants. Dynamic adjustment of the model
coeff\/icient appears to be benef\/icial for this term.

\begin{figure}[t]
%\centering
%\includegraphics[height=45mm]{09fig.eps}
\vspace{45mm}% to simulate the figure
\caption{Nondimensional rms magnitude of $\varepsilon_{\rm v}$
versus f\/ilter-width at $t/\tau_t=6.5$. \solid, scale-similar model (33); \dashed, dynamic
model (34); \chndot, dynamic model (35); \trian, DNS.}
\label{fig:fig09}
\end{figure}

The modeling of the viscous dissipation is more sensitive than the
other terms to the f\/ilter-width. The prediction accuracy deteriorates
with increasing f\/ilter-width, and in this case even for
$\delbar/\Delta=16$ none of the models is particularly accurate
(Figure~\ref{fig:fig09}).
\pagebreak

%%%%%%%%%%%%%%%%%%%%
%% Subsection 5.3 %%
%%%%%%%%%%%%%%%%%%%%

\subsection{SGS Turbulent Diffusion}
\label{sec:5.3}

The SGS turbulent diffusion $\partial{\cal J}_j/\partial x_j$ appears
in the total energy equation (\ref{eq:tot-en-ff}). Comte and Lesieur (1998)
did not model this term explicitly, but added it to
the SGS heat f\/lux by using an eddy-diffusivity model to parametrize
\begin{equation}
 \label{eq:comtel-qj}
 \left( \widetilde{\rho Eu_j} +\widetilde{pu_j} \right) -
 \left(\rhob \widetilde{E}\util_j+\pb\util_j\right)
 = \gamma \rhob \left(\widetilde{u_j T}-\util_j\widetilde{T}\right)
 + {\cal J}_j
 \simeq -\frac{\nu_{\rm T}}{\prt}
 \frac{\partial\widetilde{T}}{\partial x_j};
\end{equation}
with this model, however, the SGS turbulent diffusion ${\cal J}_j$,
which depends mostly on the unresolved velocity f\/luctuations, is
modeled in terms of the temperature gradient. In an isothermal f\/low,
${\cal J}_j$ may be nonzero, and, even if the temperature is
not constant, there is no reason to couple a term due to mechanical
energy gradients to the temperature. A model of the form
(\ref{eq:comtel-qj}) effectively neglects ${\cal J}_j$.

The only attempt to model the SGS turbulent diffusion was that by
\citet{kni98}. They argue that $\uti\simeq\utti$ and
propose a model of the form
\begin{equation}
 \label{eq:tdif-knight}
 {\cal J}_j \simeq \utk\tau_{jk}.
\end{equation}

A dynamic scale-similar model can be obtained using the generalized
central moments \citep{ger92}
\begin{eqnarray}
 \label{eq:gcm1}
 \tau(u_i,u_j) & = & \rhob\left[\widetilde{u_iu_j} - \uti\utj\right], \\
 \label{eq:gcm2}
 \tau(u_i,u_j,u_k) & = & \rhob\widetilde{u_iu_ju_k}
 - \uti\tau(u_j,u_k) - \utj\tau(u_i,u_k)
 - \utk\tau(u_i,u_j) - \rhob\uti\utj\utk.
\end{eqnarray}
Using this notation the turbulent diffusion term can be written as
\begin{equation}
 \label{eq:tdiff2}
 2 {\cal J}_j = \tau(u_j,u_k,u_k) + 2\utk\tau(u_j,u_k) ,
\end{equation}
since $\tau(u_j,u_k)=\tau_{jk}$. Using this formalism, scale-similar
models can be derived by approximating the quadratic terms using the
f\/iltered velocities $\util_j$ to replace the velocities $u_j$; for
instance, one can write
\begin{equation}
 \label{eq:gcm_ss1}
 \tau(u_i,u_j) \sim \tau(\uti,\utj) \quad \Rightarrow \quad
 \rhob\left(\widetilde{u_iu_j}-\uti\utj\right) \sim
 \rhob\left(\widetilde{\uti\utj}-\utti\uttj\right) .
\end{equation}
If the proportionality constant in (\ref{eq:gcm_ss1}) is set to one,
the scale-similar part of the mixed model (\ref{eq:tauij-sezhu})
is obtained. Analogously, the triple product can be written as
\begin{eqnarray}
 \label{eq:tdiff_mod}
 2{\cal J}_j & = & \tau(u_j,u_k,u_k) + 2\utk\tau(u_j,u_k) \nonumber \\
 & \simeq & C_{J}\tau(\utj,\utk,\utk) + 2\utk\tau(u_j,u_k) \nonumber \\
 & = & C_{J} \left[ \rhob\widetilde{\utj\utk\utk}
 - \rhob\uttj\uttk\uttk
 - \uttj\Akk - 2\uttk\Ajk \right]
 + 2\utk\tau_{jk} ,
\end{eqnarray}
the last term is parametrized by the same model used in the momentum
equation. The coeff\/icient $C_J$ can be set using the identity
\begin{equation}
\widehat{\rhob\utj\utk\utk} - \rhobh\utbj\utbk\utbk
 = 2 J_j - 2 \widehat{ {\cal J}_j} ,
\end{equation}
where
\begin{eqnarray}
 2 J_j & = & C_{J} \left[ \rhobh\breve{\widetilde{\utbj\utbk\utbk}}
 - \rhobh\utbtbj\utbtbk\utbtbk
 - \utbtbj\Bkk - 2\utbtbk\Bjk \right]
 + 2\utbk T_{jk}, \nonumber \\
\Bjk & = & \rhobh\left(\breve{\widetilde{\utbj\utbk}}
 - \utbtbj\utbtbk\right),
\end{eqnarray}
to yield
\begin{equation}
 C_J = \frac{\left\langle \left(\widehat{\rhob\utj\utk\utk} -
 \rhobh\utbj\utbk\utbk\right) {\cal P}_j
 -{\cal Q}_j{\cal P}_j\right\rangle}
 {\langle{\cal P}_k{\cal P}_k\rangle},
\end{equation}
where
\begin{eqnarray}
 {\cal P}_j & = & \left[ \rhobh\breve{\widetilde{\utbj\utbk\utbk}}
 - \rhobh\utbtbj\utbtbk\utbtbk
 - \utbtbj\Bkk - 2\utbtbk\Bjk \right]
 \nonumber \\
 & - & \left[ \widehat{\rhob\widetilde{\utj\utk\utk}}
 -\widehat{\rhob\uttj\uttk\uttk}
 - \widehat{\uttj\Akk} - 2\widehat{\uttk\Ajk}
 \right], \\
 {\cal Q}_j & = & 2 \left( \utbk T_{jk} - \widehat{\utk\tau_{jk}}
 \right) .
\end{eqnarray}

\begin{figure}[t]
%\centering
%\includegraphics[height=86mm]{10fig.eps}
\vspace{86mm}% to simulate the figure
\caption{{\it A priori} comparison of the SGS turbulent diffusion
${\cal J}_j$. (a) Correlation coeff\/icient and (b) nondimensional rms magnitude. \solid, knight
\etal (1998); \dashed, scale-similar, one-coeff\/icient model; \trian, DNS.}
\label{fig:fig10}
\end{figure}

Figure~\ref{fig:fig10}(a) shows the correlation coeff\/icient for the two models
(\ref{eq:tdif-knight}) and (\ref{eq:tdiff_mod}) and using (21)--(22) to
model $\tau_{jk}$. The correlation factor is greater than 0.7 for
both models, and both models overpredict slightly the rms magnitude
of ${\cal J}_j$ (Figure~\ref{fig:fig10}(b)). When the one-coeff\/icient, scale-similar
model is used this overprediction is signif\/icantly reduced. Both
models perform equally well for $\delbar/\Delta\leq16$, while neither
is accurate for $\delbar/\Delta=32$.
\pagebreak

\begin{figure}[t]
%\centering
%\includegraphics[height=86mm]{11fig.eps}
\vspace{86mm}% to simulate the figure
\caption{{\it A priori} comparison of the SGS viscous diffusion
${\cal D}_j$. (a) Correlation coeff\/icient and (b) nondimensional rms magnitude. \solid,
scale-similar model; \trian, DNS.}
\label{fig:fig11}
\end{figure}

%%%%%%%%%%%%%%%%%%%%
%% Subsection 5.4 %%
%%%%%%%%%%%%%%%%%%%%

\subsection{SGS Viscous Diffusion}
\label{sec:5.4}

The SGS viscous diffusion $\partial{\cal D}_j/\partial x_j$ is the
smallest of the terms in the total energy equation, and is about 5\%
of the divergence of $Q_j$. No model for this term has been proposed
in the literature to date. One possibility is to parametrize it
using a scale-similar model of the form\pagebreak
\begin{equation}
 \label{eq:vdif}
 {\cal D}_j = C_D ( \widetilde{\widetilde{\sigma}_{ij}\uti}
 - \widetilde{\widetilde{\sigma}}_{ij}\utti ) ,
\end{equation}
in which the coeff\/icient can be obtained from
\begin{equation}
 \label{eq:cd}
 C_D = \frac{\left\langle\left[
 \widehat{{\overline{\rho\sigma_{ij}}\,\overline{\rho u_i}}/
 {\rhob^2}}
 - {\widehat{\overline{\rho\sigma_{ij}}}\,
 \widehat{\overline{\rho u_i}}}/
 {\rhobh^2}
 \right] {\cal R}_j \right\rangle}
 { \left\langle{\cal R}_k{\cal R}_k\right\rangle } ,
\end{equation}
where
\begin{equation}
 {\cal R}_l =
 \left(\breve{\widetilde{ \breve{\widetilde{\sigma}}_{lk}\utbk}}
 -\breve{\widetilde{\breve{\widetilde{\sigma}}}}_{lk}\utbtbk \right)
 - \left(\widehat{\widetilde{\widetilde{\sigma}_{lk}\utk}}
 -\widehat{\widetilde{\widetilde{\sigma}}_{lk}\uttk} \right) .
\end{equation}
As can be seen from Figure~\ref{fig:fig11}, this model gives a poor correlation and poor
agreement for the prediction of the rms magnitude. However, since
the viscous diffusion is relatively small, its contribution to
the total energy spectrum does not go to the inertial range, but rather
to the decaying range. In this situation the accuracy of the model
is degraded, as shown by \citet{men97}. Thus, the scale-similar
approach may still give good predictions when this term is signif\/icant.
In this particular f\/low, the error given by the model (or by not using
a model) may be tolerable given the small contribution that the term
gives to the energy budget.

%%%%%%%%%%%%%%%%%%%%
%% Subsection 5.5 %%
%%%%%%%%%%%%%%%%%%%%

\subsection{General Considerations}
\label{sec:5.5}

In addition to the term-by-term comparisons shown before, it is
possible to evaluate the global accuracy of the models by comparing
the sum of the exact SGS terms and the modeled quantity, namely,
\begin{equation}
 \label{eq:global}
 E_{SGS}=\gamma C_{\rm v}Q_j + {\textstyle\half}{\cal J}_j - {\cal D}_j.
\end{equation}

\begin{figure}[t]
%\centering
%\includegraphics[height=85mm]{12fig.eps}
\vspace{85mm}% to simulate the figure
\caption{{\it A priori} comparison of the sum of the SGS terms in
the total energy equation (\ref{eq:tot-en-ff}). (a) Correlation coeff\/icient and (b)
nondimensional rms magnitude. \solid, Model; \trian, DNS.}
\label{fig:fig12}
\end{figure}

The mixed model (26)--(27) was used for the SGS heat f\/lux, the
scale-similar model (44)--(45) for the SGS turbulent diffusion, and the SGS
viscous diffusion has been neglected. Figure~\ref{fig:fig12}(a) shows the correlation
coeff\/icient for the exact and modeled quantities. While the
individual correlations were roughly 0.6 and 0.7 for the SGS heat f\/lux
model and the SGS turbulent diffusion, respectively, the global
correlation drops just below 0.6 when considering the sum of the
terms. Figure~\ref{fig:fig12}(b) shows the rms for both quantities.
The agreement between the exact and modeled quantities is slightly
less\pagebreak\ accurate than for the SGS heat f\/lux alone, Figure~\ref{fig:fig06}(b), but more
accurate than for the SGS turbulent diffusion alone, Figure~\ref{fig:fig10}(b).
Figure~\ref{fig:fig12} shows that the overall performance is very good.

%%%%%%%%%%%%%%%
%% Section 6 %%
%%%%%%%%%%%%%%%

\section{Conclusions}
\label{sec:6}

Several mixed and eddy-viscosity models for the momentum and energy
equations have been tested. The velocity, pressure, density, and
temperature f\/ields obtained from the DNS of homogeneous isotropic
turbulence at $Re_\lambda=50$, $M_t=0.52$ were f\/iltered and the
unclosed terms in the momentum, internal energy, and total energy
equations were computed.

In the momentum equation, mixed models were found to give better
prediction, in terms of both correlation and {\rm rms} amplitude,
than the pure eddy-viscosity models. The dynamic adjustment of the
model coeff\/icient was benef\/icial, as already observed by
\citet{moi91}.

In the internal energy and enthalpy equations only the divergence of
the SGS heat f\/lux was signif\/icant in this f\/low; the SGS pressure
dilatation $\Pi_{\rm dil}$ and viscous dissipation $\varepsilon_{\rm v}$,
which were signif\/icant in the mixing layer studied by \citet{vre95b},
were found to be negligible here. Once again, mixed dynamic models
gave the most accurate results.

In the total energy equation two additional terms are present, one of
which, the turbulent diffusion $\partial{\cal J}_j/\partial x_j$, is
signif\/icant. The model proposed by \citet{kni98} and a new
scale-similar model proposed here correlate well with the actual SGS
turbulent diffusion, and predict the correct {\rm rms} amplitude.
However, the new scale-similar model was found to be more accurate. A
mixed model for the SGS viscous diffusion was also proposed and
tested, although this term is much smaller than the others. The
accuracy of the models for the sum of the terms was also evaluated,
and it was found that the models proposed still predict nearly the
correct {\rm rms} amplitude, and an acceptable value of the
correlation coeff\/icient.

The results obtained in this investigation are promising and indicate
that it is possible to model accurately the terms in the energy
equations. Further work may extend these results to cases in which the
pressure-dilatation is signif\/icant, as well as to inhomogeneous
f\/lows, and evaluate these models {\it a posteriori}.

%%%%%%%%%%%%%%%%
%% References %%
%%%%%%%%%%%%%%%%

\begin{thebibliography}{}

\bibitem[Bardina \etal(1980)]{bar80} Bardina, J., Ferziger, J.H., and Reynolds, W.C. (1980). Improved subgrid-scale models
for large eddy simulation. AIAA Paper 80-1357.

\bibitem[Comte and Lesieur(1998)]{com98} Comte, P., and Lesieur, M. (1998). Large-eddy simulation of compressible turbulent
f\/lows. In: {\it Advances in Turbulence Modeling}, edited by D.~Olivari. Von Karman Institute
for Fluid Dynamics, Rhode-Ste-Gen\`ese, 4:1--4:133.

\bibitem[El-Hady \etal(1994)]{elh94} El-Hady, N., Zang, T.A. , and Piomelli, U. (1994). Application of the dynamic
subgrid-scale model to axisymmetric transitional boundary layer at high speed. \pofa{6},
1299--1309.

\bibitem[Erlebacher \etal(1992)]{erl92} Erlebacher, G., Hussaini, M.Y., Speziale, C.G., and Zang, T.A. (1992). Toward the
large-eddy simulation of compressible turbulent f\/lows. \jfm{238}, 155--185.

\bibitem[Favre(1965a)]{fav65a} Favre, A. (1965a). \'{E}quations des gaz turbulents compressible. I. Formes
g\'{e}n\'{e}rales. {\it J. M\'ec.}, {\bf 4}, 361--390.

\bibitem[Favre(1965b)]{fav65b} Favre, A. (1965b). \'{E}quations des gaz turbulents compressible. II. M\'{e}thode des
vitesses moyennes; m\'{e}thode des vitesses macroscopiques pond\'{e}r\'{e}es par la masse
volumique. {\it J. M\'ec.}, {\bf 4}, 391--421.

\bibitem[Germano(1992)]{ger92} Germano, M. (1992). Turbulence: the f\/iltering approach. \jfm{238}, 325--336.

\bibitem[Knight \etal(1998)]{kni98} Knight, D., Zhou, G., Okong'o, N., and Shukla, V. (1998). Compressible large eddy
simulation using unstructured grids. AIAA Paper 98-0535.

\bibitem[Leonard(1974)]{leo74} Leonard, A. (1974). Energy cascade in large-eddy simulations of turbulent f\/luid f\/lows.
{\it Adv. Geophys.}, {\bf 18A}, 237--248.

\bibitem[Lilly(1992)]{lil92} Lilly, D.K. (1992). A proposed modif\/ication of the Germano subgrid-scale closure
method. \pofa{4}, 633--635.

\bibitem[Mart\protect{\'\i}n and Candler(1998)]{mar98} Mart\'{\i}n, M.P., and Candler, G.V. (1998). Effect of chemical reactions on decaying
isotropic turbulence. {\it Phys.\ Fluids}, {\bf 10}, 1715--1724.

\bibitem[Mart\protect{\'\i}n and Candler(1999)]{mar99} Mart\'{\i}n, M.P., and Candler, G.V. (1999). Subgrid-scale model for the temperature
f\/luctuations in reacting hypersonic turbulent f\/lows. {\it Phys.\ Fluids}, {\bf 11},
2765--2771.

\bibitem[Meneveau and Lund(1997)]{men97} Meneveau, C., and Lund, T.S. (1997). The dynamic Smagorinsky model and
scale-dependent coeff\/icients in the viscous range of turbulence. {\it Phys.\ Fluids}, {\bf 9},
3932--3934.

\bibitem[Moin \etal(1991)]{moi91} Moin, P., Squires, K.D., Cabot, W.H., and Lee, S. (1991). A dynamic subgrid-scale
model for compressible turbulence and scalar transport. \pofa{3}, 2746--2757.

\bibitem[Normand and Lesieur(1992)]{nor92} Normand, X., and Lesieur, M. (1992). Direct and large-eddy simulation of laminar
breakdown in high-speed axisymmetric boundary layers. {\it Theoret.\ Comput.\ Fluid Dynamics},
{\bf 3}, 231--252.

\bibitem[Ristorcelli and Blaisdell(1997)]{ris97} Ristorcelli, J.R., and Blaisdell, G.A. (1997). Consistent initial conditions for the
DNS of compressible turbulence. {\it Phys.~Fluids}, {\bf 9}, 4--6.

\bibitem[Smagorinsky(1963)]{sma63} Smagorinsky, J. (1963). General circulation experiments with the primitive equations.
I.~The basic experiment. {\it Mon. Weather Rev.} {\bf 91}, 99--164.

\bibitem[Speciale \etal(1988)]{spe88} Speziale, C.G., Erlebacher, G., Zang, T.A., and Hussaini, M.Y. (1988) The
subgrid-scale modeling of compressible turbulence. \pofa{31}, 940--942.

\bibitem[Vreman \etal(1995a)]{vre95a} Vreman, B., Geurts, B., and Kuerten, H. (1995a). A priori tests of large eddy
simulation of the compressible mixing layer. {\it J. Engrg.\ Math.}, {\bf 29}, 299--327.

\bibitem[Vreman \etal(1995b)]{vre95b} Vreman, B., Geurts, B., and Kuerten, H. (1995b). Subgrid-modeling in LES of
compressible f\/low. {\it Appl. Sci.\ Res.}, {\bf 54}, 191--203.

\bibitem[Yoshizawa(1986)]{yos86} Yoshizawa, A. (1986). Statistical theory for compressible turbulent shear f\/lows, with
the application to subgrid modeling. \pofa{29}, 2152--2164.

\bibitem[Zang \etal(1992)]{zan92} Zang, T.A., Dahlburg, R.B., and Dahlburg, J.P. (1992). Direct and large-eddy
simulations of three-dimensional compressible Navier--Stokes turbulence. \pofa{4}, 127--140.

\end{thebibliography}

\end{document}
