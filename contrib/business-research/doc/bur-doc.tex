\documentclass{article}
\usepackage{bur}
\usepackage{shortvrb}
\MakeShortVerb{|}


\Volume{1}
\Issue{1}
\Month{May}
\Year{2008}

\begin{document}

\title{bur.sty: A TeX style for typesetting contributions for Business Research}


\author{Thomas Schmidt
  \inst{le-tex publishing services, Leipzig, Germany}}

\begin{abstract}
  This document describes the usage of bur.sty. It was developed to support
  the preparation of contributions to the open access journal ``Business
  Research'' (www.business-research.org). The main target group are users who
  work with {pdf\LaTeX} on MS Windows. The style will work on other platforms,
  too, but more additional preparation is needed as the fonts ``Georgia'' and
  ``Tahoma'' are part of the MS Windows operating system, but are not
  immediately available on other platforms.

\end{abstract}

\keywords{business research, \LaTeX, style file}

\maketitle


\section*{Prerequisites}

  |bur.sty| was developed for {pdf\LaTeX}. You need a~running {\LaTeX} system
  to apply this style. Basically it will run on every {\LaTeX} distribution,
  but the installation of the fonts ``Georgia'' and ``Tahoma'' is
  required. The package delivered with the style file contains all files
  required to support the use of these fonts with {pdf\LaTeX} on MS Windows
  systems. The fonts themselves are included in MS Windows operating systems
  by default. Together with a~current MiKTeX contribution (we tested version
  2.7) the access to these fonts will work out-of-the-box.

  If you use an older version of MiKTeX or a different {\LaTeX} distribution,
  it might be necessary to configure access to the MS Windows font
  directory. Please consult the documentation of your {\LaTeX}
  distribution. To use the package on platforms other than MS Windows, you
  need to acquire a~copy of Georgia and Tahoma and you will have to
  install these fonts for use with {\LaTeX}. A simple way for the installation
  of these fonts you will find below.

   To work with the package at least a basic knowledge of {\LaTeX} is
   required, see, e.g., \cite{mittelbach:2005}.

\section*{Font installation}

  If you want to use the style on a system where ``Georgia'' and ``Tahoma''
  are not available, you have to install them:
  \begin{enumerate}
    \item Get a copy of both font families\footnotemark
    \item Extract the packages (e.g. by using cabextract)
    \item Copy all extracted .ttf files (Georgia.ttf, Georgiab.ttf,
    Georgiai.ttf, Georgiaz.ttf, Tahoma.ttf, Tahomabd.ttf) into your working
    directory.
  \end{enumerate}
  To put the fonts in the working directory is the most simple though not a
  smart way to access them. You can also install them as well to your TeX
  distribution. Please refer to ist documentation for details.\footnotetext{E.g.
    http://sourceforge.net/projects/corefonts/files/the\%20\break
fonts/final/georgi32.exe/download,
http://download.microsoft.com/download/office97pro/fonts/\break 1/w95/en-us/tahoma32.exe}

\section*{Getting started}

  To get started ensure that the prerequisites are met. Then simply extract
  the zip package into a~working directory, change to this directory and
  execute on a~command line ``pdflatex sample''. {\LaTeX} will compile the
  {\LaTeX} source and create a~pdf file called ``sample.pdf''.

  To get the references included from the bib\TeX{} database call ``bibtex
  sample'' and again ``pdflatex sample''.

  It is strongly recommended, that you use the file ``sample.tex'' contained
  in the zip package as a~template for your contribution. It includes useful
  comments on table formatting.

\section*{Document structure}

  The sample file contains all important structures to markup a~BuR
  contribution. To meet the layout specifications of the journal, you should
  restrict the use of commands for structuring your document to the elements
  contained in the sample file. If you need further elements to structure your
  document please consult the BuR staff.

  |bur.sty| is based on the article.cls contained in every {\LaTeX}
  distribution. Mainly BuR-specific commands and features are described here.

\paragraph{Document head}
  In the head of the sample file you will find the style-specific commands
  |\Volume|, |\Issue|, |\Month|, |\Year|. These commands control the
  corresponding text in the page headers. The correct information will be
  given to you later by the editorial staff or will be inserted
  elsewhere. |\title| and |\author| is mandatory information. Your
  contribution should also include an abstract marked by a~|abtract|
  environment and keywords (|\keywords|).

  The two-column layout uses the |twocolumn| option of \LaTeX. Please be aware
  of the restrictions concerning float placement.

\paragraph{Headings}
  Within the text three levels of headings (|\section|, |\subsection| and
  |\paragraph|) are provided. Further elements you may add are tables and
  figures.


\paragraph{Tables}
  in a BuR contribution have a~dedicated layout. They should extend over
  a~whole column or over the whole text width. To support an automatic scaling
  of the table columns to |columnwidth| or |textwidth|, respectively,
  |bur.sty| uses the |tabularx| package. To support gray lines and gray cell
  background the |colortbl| package is used. Please refer to the sample
  document for information on how to markup your tables. Of course other
  packages may also be used for this purpose, but it is recommended to do so
  only if there is a~good reason. The proper layout of the tables must be
  assured.

  The alternating row colour of the tables should support the readability. It
  is not appropriate for all kinds of tables. In this case you may also use
  tables without gray background. You will find and example in the sample
  document, too.

  Tables may have captions, which appear above the table. Please use the
  |\caption| command to markup your captions. |\caption| automatically adds
  and increases a~table counter. You can use |\caption*| if you don't want
  a~table number added to your caption. Additional descriptive text below
  a~table should be marked up with |\legend|.

  The sample document uses the |graphicx| package to include figures. You may
  use other packages to include your figures, but it is necessary that the
  |\includegraphics| command is supported as the inclusion of the BuR logo is
  based on that command.

\paragraph{Figures}
  should be placed in a~|figure| environment. They may have a~heading
  (|\heading|) above and a~caption below. The |\caption| command automatically
  produces a~label ``Figure'' and a~figure count. |\caption*| omits the label
  and the counter.

  The figure environment may also be used to place other objects that need
  a~heading and/or caption. The sample document contains as an example an
  object ``Video''. Please note that there is no automatic counter/label
  support for such objects.

\paragraph{References and Biographies}
  At the end of your contribution you should place a~reference list and short
  biographies of the contributing authors. The use of bib\TeX{} and the natbib
  package is recommended. An appropriate bib\TeX{} style (|bur.bst|) and some
  sample references (|bur.bib|) are contained in this package.

  Please refer to the sample document how to include BuR bibliographies.

  Please be aware, that the BuR style does not support numbered references.



\section*{Math and tables}

  The number glyphs in the font ``Georgia'' are old style figures. The
  characters for numbers vary in width, height and position. Whereas this
  feature is nice for text passages, it makes math and tables more difficult
  to read. Therefore math mode in |bur.sty| switches to lining figures
  {\endash} all numbers will have the same width, height and position.

\begin{table}[t!]
\small
\caption{Package contents}\label{tab:1}
\noindent\begin{tabularx}{\columnwidth}{lX}
% use \normalsize and \textbf to highlight the table head
  \normalsize\textbf{File}&
  \normalsize\textbf{Description}\\[0.5mm]\bhline
  \rowcolor[gray]{0.9}
    |bur.bib|
      &A sample bibliography database\\\hline
    |bur.bst|
      &The bib\TeX{} style\\\hline
  \rowcolor[gray]{0.9}
    |bur.sty|
      &The style file\\\hline
    |bur-doc.*|
      &Files for this documentation\\\hline
  \rowcolor[gray]{0.9}
    |bur-logo.pdf|
      &The BuR logo used in the page headers\\\hline
    |grgmath.sty|
      &Math support for ``Georgia''\\\hline
  \rowcolor[gray]{0.9}
    |fig*.*|
      &Figure files used by the sample document\\\hline
    |*.map|
      &Font mapping files for Georgia and Tahoma\\\hline
  \rowcolor[gray]{0.9}
    |*.tfm|, |*.afm|
      &Font metric files for Georgia and Tahoma\\\hline
    |*.fd|
      &Font definition files for Georgia and Tahoma\\\hline
  \rowcolor[gray]{0.9}
    |T1-WGL4.enc|
      &Encoding file for TrueType fonts\\\hline
    |bur-doc.*|
      &Tex source and pdf of this documentation\\\hline
  \rowcolor[gray]{0.9}
    |sample.*|
      &Template document for the preparation of new contributions
\end{tabularx}
\end{table}

  It ist strongly recommended to use math mode for all numbers in tables,
  especially for those which should line up vertically. Please note the\vadjust{\vfil\break}
  different numbers in Tables 1~and~2 in the sample document.

  Also in text passages it's worth paying attention to question where to use
  math mode for numbers and where not.


\section*{Typography}

  The endash (\endash) and emdash (\emdash) characters are not available by
  typing |--| or |---| as usual. Please use |\endash| and |\emdash| instead.

  The |twocolumn| option of \LaTeX\ does not support automatic balancing of
  columns on the last page. Please use |\newpage| to adjust the break between
  the columns, manually.


\section*{Package contents}

  |bur.sty| comes within a zip file which contains additional supporting
  files. Table~\ref{tab:1} describes the function of the files included.

\vspace*{2mm}

\bibliographystyle{bur}
\bibliography{bur}

\end{document}
