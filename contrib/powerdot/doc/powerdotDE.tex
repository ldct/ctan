\batchmode
\documentclass[10pt,a4paper,ngerman]{article}
\usepackage[T1]{fontenc}
\usepackage[utf8]{inputenc}
\usepackage{babel}

\setlength{\parindent}{0pt}
\setlength{\parskip}{1ex plus 0.2ex minus 0.1ex}

\usepackage{dcolumn,dtklogos,multicol,fancyvrb,graphicx,url,varioref,latexsym}
\usepackage[table]{xcolor} 
\usepackage{pstricks}
\usepackage{makeidx}

\begin{document}

\def\slash#1{\textbackslash{\tt #1}}
\def\pf#1{\texttt#1}

 \title{{\bf The} \texttt{powerdot} {\bf class} vl.3 (2005/12/06)}

 \author{Hendri Adriaens and Christopher
 Ellison\footnote{"Ubersetzung: Lennart Petersen, Christiane Matthieu, Christoph
 K"ohler, Julia Babigian, Christine R"omer (Institut für Germanistische
 Sprachwissenschaft der Friedrich-Schiller-Universität Jena), 1.\,"Ubersetzungsversion,
 letzte "Anderung 21.\,4. 2009}}



\date{\today}

\maketitle

\begin{footnotesize}
\begin{abstract}
\pf{Powerdot} ist eine Präsentationsklasse für \LaTeX\, welche die 
schnelle und einfache Entwicklung von professionellen Präsentationen 
erlaubt. Sie enthält viele Möglichkeiten, die Präsentation zu verbessern 
und den Präsentierenden zu unterstützten, beispielsweise automatische 
Overlays, das Einfügen persönlicher Anmerkungen und einen Handout-Modus. 
DVI, PS und PDF sind mögliche Ausgabeformate, um die Präsentation 
anzuzeigen. Ein leistungsstarkes template-System erlaubt das einfache 
Entwickeln neuer Stile. Eine \LyX -Layoutdatei ist enthalten.
\end{abstract}

\tableofcontents

\end{footnotesize}

\section{Einführung}\label{sec:intro}
Mithilfe dieser Klasse ist es möglich, professionell aussehende 
Folien herzustellen. Die Klasse wurde so entworfen, dass die Entwicklung 
von Präsentationen möglichst einfach gehalten ist, sodass man sich auf 
den eigentlichen Inhalt der Präsentation konzentrieren kann, anstatt sich 
mit technischen Details befassen zu müssen. Nichtsdestoweniger sind 
grundlegende \LaTeX\ -Kenntnisse für die Nutzung von Powerdot notwendig.

Diese Klasse baut auf die \pf{prosper}-Klasse \cite{prosper} und das 
\pf{HA-prosper}-Paket \cite{HA-prosper} auf und erweitert diese. Das 
\pf{HA-prosper}-Paket war anfangs dafür gedacht, \pf{prosper} zu erweitern 
und einige Fehler und Probleme dieser Klasse zu korrigieren. Beim Voranschreiten 
der Entwicklung des \pf{HA-prosper}-Paketes bemerkte man, dass nicht alle 
Probleme auf diese Weise gelöst werden konnten. Diese Entdeckung war der 
Beginn eines neuen Projektes, aus dem eine neue Klasse hervorgehen sollte, 
welche die sonst nötige Kombination von \pf{prosper} und \pf{HA-prosper} 
ersetzen konnte. Die \pf{Powerdot}-Klasse ist das Ergebnis dieses Projektes.

Der verbleibende Abschnitt dieses Kapitels sei nun einem Blick auf die 
Struktur einer \pf{Powerdot}präsentation gleichsam wie einem Überblick über die
gesamte Dokumentation gewidmet.

Der Aufbau einer Präsentation ist stets derselbe. Er ist im folgenden Beispiel 
nachvollziehbar.

\begin{Verbatim}[frame=single,fontsize=\small]
 \documentclass[<class options>]{powerdot}
 \pdsetup{<presentation options>}
 \begin{document}
   \begin{slide}{a slide}
     Contents of the slide.
   \end{slide}
   \section{first section}
   \begin{slide}[<slide options>]{another slide}
     Contents of the slide.
   \end{slide}
   \begin{note}{personal note}
     The note.
   \end{note}
 \end{document}
\end{Verbatim}

Einige dieser Elemente definieren die Struktur des Dokuments. Zuvorderst 
sind einige Optionen zum \slash{documentclass}-Befehl möglich, welche die Art 
der Ausgabe des Dokuments betreffen, beispielsweise das Papierformat. 
Diese Optionen werden in Abschnitt~\ref{sec:classopts} besprochen. Ferner 
gibt es Spezifikationsoptionen, die einige globale Elemente der Präsentation 
wie zum Beispiel Fußnoten kontrollieren. Sie werden in Abschnitt~\ref{sec:pdsetup} 
behandelt.

Sind diese Entscheidungen getroffen, können die slide-Umgebung zum Erstellen 
von Folien (siehe Abschnitt~\ref{sec:slides}) und die note-Umgebung zum 
Erstellen von Anmerkungen, die mit den Folien angezeigt werden (siehe 
Abschnitt~\ref{sec:notes}), genutzt werden. Overlays ermöglichen, Inhalte 
nacheinander anzuzeigen, was in Abschnitt~\ref{sec:overlays} beschrieben wird. 
Der \slash{section} Befehl dient der Strukturierung der Präsentation. Dieses wird 
in Abschnitt~\ref{sec:structure} dargestellt. Abschnitt~\ref{sec:styles} zeigt 
eine Übersicht über die Stile, die in dieser Klasse enthalten sind, und deren 
Charakteristiken. Abschnitt~\ref{sec:compiling} behandelt die Möglichkeiten 
der Ausgabe und beinhaltet damit wichtige Informationen bezüglich der für 
\pf{Powerdot} benötigten Pakete.

Abschnitt~\ref{sec:writestyle} ist am interessantesten für jene, die ihren 
eigenen Stil für \pf{Powerdot} entwickeln oder einen bestehenden modifizieren 
wollen. Abschnitt~\ref{sec:lyx} erklärt, wie \LyX\ \cite{LyXWeb} genutzt werden 
kann, um \pf{Powerdot}präsentationen zu erstellen. Das Ende der Dokumentation 
schließlich (Abschnitt~\ref{sec:questions}) wurde einem Kapitel über Fragen wie 
"`Wo kann ich Beispiele finden?"' gewidmet und enthält ferner die Informationen, 
wohin man sich wenden kann, wenn die eigenen Fragen zu \pf{Powerdot} noch nicht 
gelöst sein sollten.

\section{Das Erstellen einer Präsentation}\label{sec:setup}

Dieser Abschnitt beschreibt die möglichen Optionen zur Kontrolle der Ausgabe 
der Präsentationen und dem Erscheinungsbild eben jener.

\subsection{documentclass-Optionen}\label{sec:classopts} 

Begonnen wird mit 
den Optionen, die in den \slash{documentclass} Befehl mittels einer durch Kommata 
separierten Liste eingefügt werden. Für jede Möglichkeit wird der 
Standardwert\footnote{Das ist der Wert, der als Standard genutzt wird, wenn 
nichts anderes explizit gewählt wird.} in der Beschreibung erwähnt. Das ist 
der Wert, der genutzt wird, wenn entschieden wurde, einer Option keinen Wert 
zu geben oder die Option gar nicht zu verwenden.

\marginpar{\textsl{option}\\\texttt{mode}}
Die Optionen \texttt{mode} bestimmen die Art der Ausgabe, die produziert werden soll. 
Der Standardwert ist \texttt{present}.
\begin{description}
\item[\fcolorbox{black}{gray!30}{mode=present}]

Dieser Modus wird dazu genutzt, die eigentliche Präsentation zu erstellen. 
Er aktiviert Overlays und Übergangseffekte. Auf Overlays wird in 
Abschnitt~\ref{sec:overlays} näher eingegangen.

\item[\fcolorbox{black}{gray!30}{mode=print}]
Dieser Modus kann verwendet werden, um die Folien inklusive ihrer visuellen 
Aufmachung, aber ohne Overlays oder Übergangseffekte auszudrucken.
\item[\fcolorbox{black}{gray!30}{mode=handout}]
Dieser Modus erstellt einen schwarzweißen Überblick über die Folien, der 
genutzt werden kann, um persönliche Notizen darauf zu verzeichnen, die 
Präsentation an Studenten auszugeben, ihn als eigenen Leitfaden einzusetzen, etc.
\begin{description}
\item[\fcolorbox{black}{gray!30}{nopagebreaks}]
Standardmäßig gibt der handout-Modus Dokumente mit zwei Folien pro Seite aus. 
Wenn mehr Folien pro Seite platziert werden sollen, füge man diese Option in den 
\slash{documentclass} Befehl ein und \pf{Powerdot} wird es \LaTeX\ überlassen, zu 
entscheiden, wann der Seitenumbruch stattfinden soll, meistens also, wenn eine 
Seite gefüllt ist.
\end{description}
\end{description}

\marginpar{\textsl{option}\\\texttt{paper}}
Die Option \texttt{paper} hat drei mögliche Werte. 
Der Standardwert ist \texttt{screen}.
\begin{description}
\item[\fcolorbox{black}{gray!30}{paper=screen}]
Dies ist ein Spezialformat mit optimalem Bildschirmverhältnis (4/3). Die entsprechenden 
Seitenlängen sind 8.25 Inch zu 11 Inch (ungefähr 21cm zu 28cm). Dieses Papierformat ist für den 
print- oder 
handout-Modus nicht vorhanden. In diesen Modi wird \pf{Powerdot} automatisch zur Größe 
DIN A4 wechseln und eine Warnung diesbezüglich in die Logdatei der Präsentation schreiben.
\item[\fcolorbox{black}{gray!30}{paper=a4paper}]
Das Format DIN A4 wird für die Präsentation oder das Handout genutzt.
\item[\fcolorbox{black}{gray!30}{paper=letterpaper}]
Ein Briefpapierformat wird genutzt.
\end{description}

Einige wichtige Informationen zum Papierformat, zum Kompilieren und dem Betrachten von 
Präsentationen finden sich im Abschnitt~\ref{sec:compiling}.

\marginpar{\textsl{option}\\\texttt{orient}}Die Option \texttt{orient} kontrolliert die 
Ausrichtung der Präsentation. Der 
Standardwert ist \texttt{landscape}.

\begin{description}
\item[\fcolorbox{black}{gray!30}{orient=landscape}]
Die Präsentation erhält das Querformat. Dieser Wert ist im handout-Modus nicht verfügbar. 
Sollte er dennoch gewählt werden, wird ihn \pf{Powerdot} zum Hochformat wechseln und einen 
entsprechend warnenden Vermerk in die Logdatei schreiben.
\item[\fcolorbox{black}{gray!30}{orient=portrait}]
Dies erstellt Folien im Hochformat. Es ist zu beachten, 
dass nicht alle Stile diese Ausrichtung unterstützen. In Abschnitt~\ref{sec:styles} sind 
diesbezüglich entsprechende Informationen für jeden Stil zu finden.
\end{description}

\marginpar{\textsl{option}\\\texttt{display}} Die Option \texttt{display} beeinflusst die 
Produktion von Folien und Anmerkungen. Der 
Standardwert ist \texttt{slides}.

\begin{description}
\item[\fcolorbox{black}{gray!30}{display=slides}]
Dies wird nur die Folien in der Präsentation anzeigen.
\item[\fcolorbox{black}{gray!30}{display=slidesnotes}]
Dies wird sowohl die Folien als auch die Anmerkungen in der Präsentation anzeigen. Mehr 
Informationen bezüglich der Anmerkungen enthält Abschnitt~\ref{sec:notes}.
\item[\fcolorbox{black}{gray!30}{display=notes}]
Dies wird nur die Anmerkungen anzeigen.
\end{description}

Hier nun noch einige weitere Optionen, welche die Ausgabe beeinflussen.
\begin{description}
\marginpar{\textsl{option}\\\texttt{size}} 
\item[\fcolorbox{black}{gray!30}{size}]
ist die Schriftgröße von normalem Text in Punkten (points). 
Mögliche Werte sind 8pt, 
9pt, 10pt, 11pt, 12pt, 14pt, 17pt und 20pt. Der Standardwert ist 11pt.\footnote{Es sollte 
beachtet werden, dass von 10pt, 11pt und 12pt abweichende Schriftgrößen keine Standardvarianten
sind und man für diese das \pf{extsizes}-Paket \cite{extsizes}, welches diese Größen beinhaltet, 
installieren muss.}

\marginpar{\textsl{option}\\\texttt{style}} 
\item[\fcolorbox{black}{gray!30}{style}]
definiert den Stil, der für 
die Präsentation geladen werden soll. Standardmäßig wird der 
\pf{default}-Stil geladen. Mögliche Varianten sind im Abschnitt~\ref{sec:styles} zu finden.

\marginpar{\textsl{option}\\\texttt{fleqn}}
\item[\fcolorbox{black}{gray!30}{fleqn}]
setzt die Gleichungen linksbündig. 
Eben so, wie es die gleichnamige Option in der 
article-Klasse ebenfalls setzt.

\marginpar{\textsl{option}\\\texttt{display}}
\item[\fcolorbox{black}{gray!30}{display}]
setzt die Gleichungsnummern linksseitig. 
Auch hier wieder entsprechend der gleichnamigen 
Option in der article-Klasse.

\marginpar{\textsl{option}\\\texttt{nopsheader}}
\item[\fcolorbox{black}{gray!30}{nopsheader}]
Standardmäßig schreibt \pf{Powerdot} einen 
Postscriptbefehl in die PS-Datei, um sicherzugehen, 
dass die nachfolgenden Bearbeitungsprogramme wie ps2pdf wissen, welches Papierformat sie nutzen 
sollen, auch wenn keine explizite Spezifikation dazu in der Kommandozeile steht. Dazu ist 
Abschnitt~\ref{sec:compiling} zu beachten. Wenn Probleme mit den nachfolgenden 
Bearbeitungsprogrammen oder dem Ausdrucken auftreten, oder wenn man selbst das 
Papierformat in den nachfolgenden Bearbeitungsstufen definieren möchte, nutzt man 
diese Option.

\marginpar{\textsl{option}\\\texttt{hlentries}}
\item[\fcolorbox{black}{gray!30}{hlentries}]
Dies hebt Einträge des Inhaltsverzeichnisses 
hervor, wenn der Eintrag mit 
der aktuellen Folie übereinstimmt. Der Standardwert ist \texttt{true}. Man beachte 
dazu auch Abschnitt~\ref{sec:structure}. Wenn keine solchen Hervorhebungen 
gewünscht werden (beispielsweise im print-Modus), nutzt man
\texttt{hlentries=false}.

\marginpar{\textsl{option}\\\texttt{hlsections}}
\item[\fcolorbox{black}{gray!30}{hlsections}]
Dies hebt die Abschnitte (sections) des 
Inhaltsverzeichnisses hervor, wenn 
diese mit den Abschnitten der laufenden Präsentation übereinstimmen. Der 
Standardwert ist \texttt{false}. Dazu ist ebenfalls Abschnitt~\ref{sec:structure} 
zu beachten. Eine Spezifikation dieser Option wendet die Hervorhebung der 
Abschnitte an. Dies kann hilfreich sein, wenn man einen Stil nutzt, der ein 
geteiltes Inhaltsverzeichnis enthält.

\marginpar{\textsl{option}\\\texttt{balckslide}}
\item[\fcolorbox{black}{gray!30}{blackslide}]
Diese Option fügt eine schwarze Folie auf der 
ersten Seite der Präsentation 
ein und wird ferner automatisch zur zweiten Seite wechseln, wenn die 
Präsentation in einem PDF-Viewer wie dem Acrobat Reader geöffnet wird. 
Ferner fügt diese Option einen Verweis zu jeder Folien- oder Abschnittsüberschrift 
ein, der zurück auf die schwarze Folie führt. Klickt man dann dagegen irgendwo 
auf die schwarze Folie, kommt man zurück zu jener Folie, auf der man sich 
zuvor befand. Diese Option kann dazu genutzt werden, eine Präsentation 
temporär zu pausieren, wenn man beispielsweise einen Beweis an der Tafel 
anführen möchte.
\marginpar{\textsl{option}\\\texttt{clock}}
\item[\fcolorbox{black}{gray!30}{clock}]
Dies zeigt eine kleine Digitaluhr auf den Folien, 
die dafür genutzt werden 
kann, die Zeit während des Vortrages im Auge zu behalten.
\end{description}

Hier ist ein Beispiel für einen \slash{documentclass} Befehl.
\begin{Verbatim}[frame=single,fontsize=\small]
 \documentclass[
   size=12pt,
   paper=screen,
   mode=present,
   display=slidesnotes,
   style=tycja,
   nopagebreaks,
   blackslide,
   fleqn
 ]{powerdot}
\end{Verbatim}
Dieses Beispiel definiert eine Präsentation im \pf{tycja}-Stil, mit schwarzer 
Folie, einer Schriftgröße von 12 Punkten und linksbündigen Gleichungen.
\begin{Verbatim}[frame=single,fontsize=\small]
 \documentclass[
   size=12pt,
   paper=letterpaper,
   mode=handout,
   display=slidesnotes,
   style=tycja,
   nopagebreaks,
   blackslide,
   fleqn
 ]{powerdot}
\end{Verbatim}
Mit dem Wechsel der \texttt{paper}- und \texttt{mode}-Optionen wird nun ein Handout mit mehr 
als zwei Folien pro Seite, ganz so, wie es die \texttt{nopagebreaks}-Option definiert, 
ausgegeben.

\subsection{Setup Optionen}\label{sec:pdsetup}
\marginpar{\textsl{option}\\\texttt{$\backslash$pdsetup}}
Es gibt einige Zusatzoptionen, die dabei helfen, eine Präsentation den eigenen 
Wünschen anzupassen. Diese Optionen sind allerdings nicht im
\slash{documentclass} Befehl 
enthalten, was technisch begründet ist.\footnote{Der interessierte Leser wende 
sich dazu an den Abschnitt bezüglich des \pf{xkvltxp}-Paketes in der 
\pf{xkeyval}-Dokumentation \cite{xkeyval}.}
Wir unterscheiden zwei Arten von Optionen. Jene Optionen, die nur global, also 
die gesamte Präsentation betreffend, mittels des \slash{pdsetup} Befehls definiert 
werden können, und jene Optionen, die sowohl global (mittels
\slash{pdsetup}) als 
auch lokal (mittels Folienumgebungen, mehr dazu in Abschnitt~\ref{sec:slides}) 
nutzbar sind.

\subsubsection{Globale Optionen}\label{sec:gopts}
Dieser Abschnitt beschreibt Optionen, die einzig global in der Präambel der 
Präsentation mittels des \slash{pdsetup} Befehls genutzt werden können.
\begin{description}
\marginpar{\textsl{option}\\\texttt{palette}}
\item[\fcolorbox{black}{gray!30}{palette}]
Dies definiert die nutzbare Farbpalette. Eine Farbpalette
ist eine Sammlung 
von Farben, die von einem Stil bereitgestellt wird. Abschnitt~\ref{sec:styles} 
erläutert genau, welche Stile welche Farbpaletten enthalten.

\marginpar{\textsl{option}\\\texttt{lf}\\\texttt{rf}}
\item[\fcolorbox{black}{gray!30}{lf} \fcolorbox{black}{gray!30}{rf}]
Dies definiert den Inhalt der linken und der rechten Fußzeile. Sie sind 
standardmäßig leer.

\marginpar{\textsl{option}\\\texttt{theslide}}
\item[\fcolorbox{black}{gray!30}{theslide}]

Diese Option legt fest, wie die Foliennummern auf der 
Folie gesetzt werden. Der Standardwert ist
\slash{arabic}\{slide\}$\sim$/$\sim$\slash{pageref*\{lastslide\}}, 
was in der Form 
\texttt{5/22} dargestellt wird. Man beachte, dass der
Befehlszeilenabschnitt \slash{arabic\{slide\}} 
die Nummer der gerade angezeigten Folie und der Befehlszeilenabschnitt 
\slash{pageref*\{lastslide\}} die Nummer der letzten Folie anzeigt.\footnote{Es wird 
\slash{pageref} mit Stern verwendet, was mittels \slash{hyperref} 
definiert wird und so keinen Verweis auf die Seite, auf die sich bezogen wird, erstellt.}

\marginpar{\textsl{option}\\\texttt{thenote}}
\item[\fcolorbox{black}{gray!30}{thenote}]
Dies ist gleich dem \texttt{theslide} Befehl, bezieht sich 
allerdings auf die 
Foliennummern der Anmerkungen. Der Standardwert ist\\
\texttt{note$\sim$$\backslash$arabic\{note\}}\texttt{$\sim$of$\sim$slide$\sim$$\backslash$arabic\{slide\}}, 
wobei hier\\ \slash{arabic\{note\}} die Nummer der aktuellen Anmerkung auf der aktuellen 
Folie zeigt. Dies könnte beispielsweise so aussehen: \texttt{note 2 of
slide 7}. (Für eine deutsche Ausgabe wäre folgender Wert möglich:\\
\texttt{Anmerkung$\sim$}\slash{arabic\{note\}}\texttt{$\sim$auf$\sim$Folie$\sim$}\slash{arabic\{slide\}},
was folgendermaßen angezeigt werden würde: \texttt{Anmerkung 2 auf Folie 7}.)

\marginpar{\textsl{option}\\\texttt{counters}}
\item[\fcolorbox{black}{gray!30}{counters}]
Die \texttt{counters} Option listet Zähler auf, die 
möglicherweise bei Overlays geschützt 
werden sollen. Da Inhalte sowie auch \LaTeX\ -Zähler bei Overlays (Man beachte 
Abschnitt~\ref{sec:overlays}) mehrfach bearbeitet werden, wie es beispielsweise 
beim \texttt{equation} Zähler der Fall sein kann, ist es möglich, dass zu hohe Zählungen 
entstehen. Um zu vermeiden, dass die Zähler bei verschiedenen Overlays unterschiedliche 
Nummern anzeigen, nutzt man diese Option. Die \texttt{equation},
\texttt{table}, \texttt{figure}, 
\texttt{footnote} und \texttt{mpfootnote} Zähler sind bereits geschützt. Wenn man weitere 
Zähler benutzt, beispielsweise für Theoreme, sollte man diese mittels dieser 
Option auflisten. Zum Beispiel:
\begin{Verbatim}[frame=single,fontsize=\small]
 counters={theorem,lemma}
\end{Verbatim}

\marginpar{\textsl{option}\\\texttt{list}}
\item[\fcolorbox{black}{gray!30}{list}]
Diese Option enthält eine ganze Liste von Optionen, die dem \pf{enumitem}-Paket 
angepasst sind, welches das Layout jener Listen, die mithilfe der
\texttt{enumerate} 
und \texttt{itemize} Umgebungen erzeugt werden, definiert. Ein Beispiel:

\begin{Verbatim}[frame=single,fontsize=\small]
 list={labelsep=1em,leftmargin=*,itemsep=0pt,topsep=5pt,parsep=0pt}
\end{Verbatim}

Man beachte für mehr Informationen bezüglich des Layouts von Listen das 
\pf{enumitem}-Paket \cite{enumitem}.

\marginpar{\textsl{option}\\\texttt{enumerate}\\\texttt{itemize}}
\item[\fcolorbox{black}{gray!30}{enumerate} \quad
\fcolorbox{black}{gray!30}{itemize}]

Dies ist gleich der \texttt{list} Optionen, kontrolliert allerdings nur die
\texttt{enumerate} bzw. \texttt{itemize} Umgebungen.
\end{description}

\subsubsection{Globale und lokale Optionen}\label{sec:glopts}
Dieser Abschnitt beschreibt Optionen, die sowohl global mittels des 
\texttt{pdsetup} Befehls als auch lokal mittels Folienumgebungen (mehr dazu in 
Abschnitt~\ref{sec:slides}) definiert werden können.

\begin{description}
\marginpar{\textsl{option}\\\texttt{trans}}
\item[\fcolorbox{black}{gray!30}{trans}] Diese Option definiert, dass der Standardübergangseffekt für
die Präsentation 
genutzt wird. Diese Übergangseffekte werden erst nach der Kompilierung der 
Präsentation ins PDF-Format angezeigt. Man beachte dazu auch 
Abschnitt~\ref{sec:compiling}. Die folgenden Übergangseffekte werden unterstützt: 
\texttt{Split}, \texttt{Blinds}, \texttt{Box}, \texttt{Wipe},
\texttt{Dissolve}, \texttt{Glitter} und \texttt{Replace}. Wenn mit 
einem PDF-Viewer gearbeitet wird, der PDF 1.5 anzeigen kann, sind ebenfalls
\texttt{Fly}, 
\texttt{Push}, \texttt{Cover}, \texttt{Uncover} oder \texttt{Fade} nutzbar. Es ist wichtig, 
zu beachten, dass 
die meisten PDF-Viewer technisch empfindlich sind, beispielsweise wird
\texttt{box} nicht funktionieren.

Der Standardeffekt ist \texttt{Replace}, welcher einzig eine Folie zur nächsten wechseln 
wird, wenn sie durchgesehen werden. Man beachte, dass manche PDF-Viewer (wie der 
Acrobat Reader 5 oder höher) die Übergangseffekte nur im Vollbildmodus anzeigen. 
Wenn ein bisher nicht aufgeführter Übergangseffekt (beispielsweise ein Wischeffekt 
mit einer speziellen Bewegungsrichtung) genutzt werden soll, ist das durchaus möglich. 
\pf{Powerdot} wird eine Warnung in die Logdatei schreiben, dass der ausgewählte 
Effekt möglicherweise nicht vom PDF-Viewer angezeigt werden wird. Hier ein 
Beispiel, das funktioniert:
\begin{Verbatim}[frame=single,fontsize=\small]
 trans=Wipe /Di 0
\end{Verbatim}
Der Acrobat Reader zeigt diesen Wischeffekt von rechts nach links, statt, wie 
es der Standard ist, von oben nach unten. Für weitere Informationen sei das 
PDF-Referenzhandbuch empfohlen.
\marginpar{\textsl{option}\\
  \texttt{method}}

\item[\fcolorbox{black}{gray!30}{method}]
Diese Option kann dann genutzt werden, wenn die Präsentation spezielles Material 
enthält, das nicht in der "`üblichen Art"' von \LaTeX\ behandelt werden soll. 
Verbatim-Material kann beispielhaft angeführt werden. Mögliche Werte sind 
\texttt{normal} (der Standardwert), \texttt{direct} und \texttt{file}. Diese Optionen werden 
genauer im Abschnitt~\ref{sec:verbatim} erklärt.
\marginpar{\textsl{option}\\\texttt{logohook}\\\texttt{logopos}\\\texttt{logocmd}}
\item[\fcolorbox{black}{gray!30}{logohook} \quad
\fcolorbox{black}{gray!30}{logops} \quad
\fcolorbox{black}{gray!30}{logocmd}]

Wenn \texttt{logopos} spezifiziert wurde, wird ein Logo 
vom \texttt{logocmd}-Wert definiert 
und auf jede Folie gesetzt. Die Position des Logos kann in relativen Werten 
abhängig von der Breite und Höhe der Folien gesetzt werden. \{{0,0\} ist die 
untere linke Ecke des Papiers und
\{\slash{slidewidth},\slash{slideheight}\} ist die obere 
rechte Ecke. Für die Positionierung des Logos wird der \slash{rput} Befehl von 
\pf{pstricks} \cite{PSTricksWeb,PSTricks} genutzt. Dieser Befehl erlaubt es 
ebenfalls, einen genauen Punkt als Logoposition zu definieren. Dieser Punkt 
kann mithilfe der \texttt{logohook}-Option eingetragen werden und die
Werte \texttt{tl}, 
\texttt{t}, \texttt{tr}, \texttt{r}, \texttt{Br}, \texttt{br},
\texttt{b}, \texttt{bl}, \texttt{Bl}, \texttt{l}, \texttt{B} und
\texttt{c} sind möglich. 
Weitere Informationen bezüglich des \slash{rput} Befehls enthält die 
\pf{pstricks}-Dokumentation. Hier ist ein Beispiel, in dem die Blume des 
\pf{default}-Stiles in den \pf{husky}-Stil integriert wird.

\begin{Verbatim}[frame=single,fontsize=\small]
 \documentclass[style=husky]{powerdot}
 \pdsetup{
   logohook=t,
   logopos={.088\slidewidth,.99\slideheight},
   logocmd={\includegraphics[height=.08\slideheight]
                            {powerdot-default.ps}}
 }
 \begin{document}
 ...
 \end{document}
\end{Verbatim}
Der Standardwert von \texttt{logohook} ist \texttt{tl}.
\end{description}

Eine besondere Eigenschaft von \pf{Powerdot}, mit deren Hilfe
Präsentationen 
ein wenig Leben eingehaucht werden kann, ist die Nutzung von Zufallspunkten. 
Diese Punkte werden irgendwo auf den Folien gesetzt und nutzen die Farben 
der ausgewählten Farbpalette. Auch Overlays werden dieselben Punkte aufweisen. 
Diese Eigenschaft basiert auf \texttt{random.tex}~\cite{random}. Verschiedenste 
Optionen sind möglich, um das Erscheinen der Zufallspunkte zu kontrollieren.

\begin{description}
\marginpar{\textsl{option}\\\texttt{randomdots}}
\item[\fcolorbox{black}{gray!30}{randomdots}]
Standardmäßig sind die Zufallspunkte ausgeschaltet. Sie
werden dann generiert, 
wenn diese Option mit dem Wert \texttt{true} versehen wird, während der
Wert \texttt{false}
sie wiederum ausschaltet. Wenn kein expliziter Wert zu dieser Option gesetzt 
ist, wird \texttt{true} angenommen.

\marginpar{\textsl{option}\\\texttt{dmindots}\\\texttt{dmaxdots}}
\item[\fcolorbox{black}{gray!30}{dmindots} \quad
\fcolorbox{black}{gray!30}{dmaxdots}]
Die Anzahl der Punkte pro Folie ist ebenfalls zufällig. Diese Optionen legen 
die minimale und maximale Anzahl Punkte pro Folie fest. Die Standardwerte sind 
\texttt{5} bzw. \texttt{40}.

\marginpar{\textsl{option}\\\texttt{dminsize}\\\texttt{dmaxsize}}
\item[\fcolorbox{black}{gray!30}{dminsize} \quad
\fcolorbox{black}{gray!30}{dmaxsize}] 
Dies ist der minimale und maximale Radius der Punkte. Standardwerte sind
\texttt{5pt} bzw. \texttt{40pt}.

\marginpar{\textsl{option}\\\texttt{dminwidth}\\\texttt{dmaxwidth}\\\texttt{dminheight}\\
\texttt{dmaxheight}} 
\item[\fcolorbox{black}{gray!30}{dminwidth} \quad
\fcolorbox{black}{gray!30}{dmaxwidth} \quad
\fcolorbox{black}{gray!30}{dminheight} \quad
\fcolorbox{black}{gray!30}{dmaxheight}]
Diese Option bestimmt das Areal auf der Folie, in dem 
die Zufallspunkte erscheinen sollen. Der Standardwert dieser Option definiert, dass die 
Punkte überall auf der Folie erscheinen können, aber das ist variabel, 
beispielsweise so, dass die Punkte nur noch im Textfeld vorkommen. Die 
Standardwerte sind \texttt{0pt}, \slash{slidewidth}, \texttt{0pt},
\slash{slideheight}.

Hier ein Beispiel, das Punkte in einem kleineren Rechteck auf der Folie erlaubt.
\begin{Verbatim}[frame=single,fontsize=\small]
 \pdsetup{
   dminwidth=.1\slidewidth,dmaxwidth=.9\slidewidth,
   dminheight=.2\slideheight,dmaxheight=.8\slideheight
 }
\end{Verbatim}
\marginpar{\textsl{option}\\\texttt{dbright}}
\item[\fcolorbox{black}{gray!30}{dbright}]
Diese Option dient der Helligkeitseinstellung der Punkte. 
Der Wert sollte 
eine Zahl zwischen -100 und 100 sein. Wenn die Zahl negativ ist, wird die 
Farbe zu Schwarz abgedunkelt, wobei -100 Schwarz ergibt. Ist die Zahl dagegen 
positiv, wird die Farbe zu Weiß aufgehellt, wobei wiederum 100 Weiß ergibt. 
Ist der Folienhintergrund hell, nutzt man eher einen positiven Wert der Option 
\texttt{bright}, ist er dagegen dunkel, sollte man sich wohl für einen negativen Wert 
entscheiden. Der Standardwert ist \texttt{60}, was eine Mischung aus 40\ der Originalfarbe 
und 60\ Weiß ergibt.

\marginpar{\textsl{option}\\\texttt{dprop}}
\item[\fcolorbox{black}{gray!30}{dprop}]
Mithilfe dieser Option können zusätzliche Parameter zum 
\slash{psdot} Befehl hinzugefügt 
werden, welcher die Zufallspunkte erstellt. Beispielsweise der Stil oder die 
Linienstärke der Punkte sind so veränderbar. Für weitere Informationen bezüglich 
des \slash{psdot} Befehls sei auf die \pf{pstricks}-Dokumentation \cite{PSTricksWeb,PSTricks} 
verwiesen. \pf{Powerdot} enthält zwei zusätzliche Stile, die für Zufallspunkte 
eingesetzt werden können. Diese sind \texttt{ocircle} (offener Kreis) und
\texttt{osquare} (offenes Quadrat).
\end{description}

Hier sind zwei Beispiele für den Gebrauch von Zufallspunkten.
\begin{Verbatim}[frame=single,fontsize=\small]
 \pdsetup{
   randomdots,dminwidth=.2\slidewidth
 }
\end{Verbatim}
Dieses Beispiel beinhaltet Zufallspunkte, verhindert aber, dass die linken 
20\ der Folie von ihnen genutzt wird.
\begin{Verbatim}[frame=single,fontsize=\small]
 \pdsetup{
   randomdots,dprop={dotstyle=ocircle,linewidth=.5pt},
   dminsize=500pt,dmaxsize=600pt,dmindots=2,dmaxdots=5
 }
\end{Verbatim}
Dieses Beispiel setzt höchstens fünf große Kreise auf die Folie. Diese 
Kreise sind zu groß, um vollständig auf der Folie angezeigt werden zu können, 
weswegen man nur Teile von ihnen als Kurven sehen wird.

\subsubsection{Ein \slash{pdsetup} \textbf{Beispiel}}

Hier ein Beispiel für einen \slash{pdsetup} Befehl, mit dem man eine Präsentation 
beginnen könnte.
\begin{Verbatim}[frame=single,fontsize=\small]
 \pdsetup{
   lf=My first presentation,
   rf=For some conference,
   trans=Wipe,
   theslide=\arabic{slide},
   randomdots,dmaxdots=80
 }
\end{Verbatim}
Dies setzt die linke und rechte Fußzeile und initialisiert den
\texttt{Wipe}-Übergangseffekt. 
Ferner beinhaltet die Foliennummerierung nicht die Nummer der letzten Folie, sondern 
einzig jene der aktuellen. Und schließlich werden die Folien mit bis zu 80 
Zufallspunkten bedeckt sein.

Hier ist eine kurze Anmerkung bezüglich des Erscheinens von Fußzeilen nötig. 
Die Foliennummer (definiert von der \texttt{theslide}-Option) wird in einer Fußzeile 
angezeigt werden. Die meisten Stile setzen sie in die rechte Fußzeile. Wenn 
Fußzeile und Foliennummer nicht leer sind, wird \texttt{$\sim$--$\sim$} zwischen sie eingefügt 
werden, um sie voneinander abzutrennen. Einige Stile modifizieren womöglich 
diese Standardvorgehensweise.

\section{Das Erstellen von Folien}\label{sec:slides}
\subsection{Die Titelfolie}\label{sec:titleslide}

\marginpar{\slash{title}\\\slash{author}\\\slash{and}\\\slash{date}\\\slash{maketitle}}
Die Titelfolie wird mittels des \slash{maketitle} Befehls erstellt.
\begin{Verbatim}[frame=single,fontsize=\small,fillcolor=\color{yellow}]
 \{maketitle}<options>
\end{Verbatim}
Dieser Befehl nutzt dieselben Werte wie jener des \LaTeX\ -Standarddokuments. 
Das optionale Argument <\texttt{options}> kann jede Option des Abschnittes~\ref{sec:glopts} 
enthalten. Das Einfügen einer solchen Option in den \slash{maketitle} Befehl betrifft 
nur die Titelfolie und keine andere. Man beachte das untere Beispiel.
\begin{Verbatim}[frame=single,fontsize=\small]
 \documentclass{powerdot}
   \title{Title}
   \author{You \and me}
   \date{August 21, 2005}
 \begin{document}
    \maketitle
    ...
 \end{document}
\end{Verbatim}
Die \texttt{author}, \texttt{title} und \texttt{date}-Angaben definieren den beim Erstellen 
der Titelfolie zu nutzenden Text. Die Gestaltung der Titelseite wird durch den 
ausgesuchten Stil bestimmt. Man beachte den Gebrauch des \slash{and} Befehls, um 
mehrere Autoren aufzuführen. Weitere Informationen bezüglich der Befehle wie 
\slash{title} und \slash{author} sind im \LaTeX\ -Handbuch \cite{companion} enthalten.

\subsection{Andere Folien}\label{sec:otherslides}

\marginpar{\texttt{slide}}
Das Herzstück jeder Präsentation sind die Folien. Bei 
\pf{Powerdot} wird der Inhalt jeder Folie innerhalb einer \texttt{slide}-Umgebung 
definiert.
\begin{Verbatim}[frame=single,fontsize=\small,fillcolor=\color{yellow}]
 \begin{slide}[<options>]{<slide title>}
 <body>
 \end{slide}
\end{Verbatim}

Im Abschnitt~\ref{sec:overlays} werden wir sehen, wie man die Folien etwas 
lebendiger gestalten kann. Bleiben wir jetzt aber erst einmal bei einem simplen 
Beispiel.
\begin{Verbatim}[frame=single,fontsize=\small]
 \begin{slide}{First slide}
   Hello World.
 \end{slide}
\end{Verbatim}
Die Folienumgebung hat ein notwendiges Argument, nämlich den Folientitel. 
Sobald eine Folie erstellt wird, wird der Folientitel dafür genutzt, einen 
Eintrag ins Inhaltsverzeichnis und in die Lesezeichenliste einzufügen. 
Das Inhaltsverzeichnis ist eine Liste der Folien- und Abschnitttitel der 
Präsentation, die auf jeder Folie erscheint.

Die aufgeführten Titel des Inhaltsverzeichnisses sind mit deren Folien und 
Abschnitten verlinkt (sofern die Präsentation ins PDF-Format kompiliert wurde) 
und bieten somit eine praktische Hilfe, um innerhalb der Präsentation rasch 
zwischen benötigten Folien zu wechseln. Die Lesezeichenliste erscheint erst, 
wenn die Kompilierung ins PDF-Format abgeschlossen ist. Sie dient ebenfalls 
als eine Art Inhaltsverzeichnis, erscheint allerdings auf \textit{keiner} der Folien, 
sondern in einem zusätzlichen Fenster im PDF-Viewer. Im obigen Beispiel werden 
die Einträge im Inhaltsverzeichnis wie in der Lesezeichenliste beide als 
\texttt{First slide} geführt.

Die <\texttt{options}> für die \texttt{slide}-Umgebung können jene Optionen enthalten, die 
im Abschnitt~\ref{sec:glopts} enthalten sind. Zusätzlich können folgende Optionen 
benutzt werden.
\begin{description}
\marginpar{\textsl{option}\\\texttt{toc}}
\item[\fcolorbox{black}{gray!30}{toc}]
Wenn definiert, wird dieser Wert für den Eintrag ins Inhaltsverzeichnis genutzt; 
ansonsten wird dazu der Folientitel herangezogen. Wenn \texttt{toc=} definiert ist, 
wird kein Eintrag erstellt.
\marginpar{\textsl{option}\\\texttt{bm}}
\item[\fcolorbox{black}{gray!30}{bm}]
Falls definiert, wird dieser Wert für den Eintrag in die Lesezeichenliste 
genutzt; 
ansonsten wird dazu der Folientitel herangezogen. Wenn \texttt{bm=} definiert ist, wird 
kein Eintrag erstellt.
\end{description}

Diese optionalen Argumente sind besonders nützlich, wenn der Folientitel sehr 
lang ist oder wenn er \LaTeX\ -Befehle enthält, die in den Lesezeichen nicht 
korrekt angezeigt werden würden.\footnote{Der Prozess, der die Lesezeichen 
setzt, nutzt \slash{pdfstringdef} vom \pf{hyperref}-Paket und kann mit akzentuierten 
Zeichen  wie \slash{"i} umgehen.}
Beim Erstellen der Einträge sollte darauf geachtet werden, spezielle Zeichen 
wie `,' und `=' zwischen den geschweiften Klammern `\{' und `\}' zu 
verstecken. Werfen wir einen Blick auf das Beispiel, das diese optionalen 
Argumente nutzt.
\begin{Verbatim}[frame=single,fontsize=\small]
 \begin{slide}[toc=,bm={LaTeX, i*i=-1}]{\color{red}\LaTeX, $i^2=-1$}
   My slide contents.
 \end{slide}
\end{Verbatim}

In diesem Beispiel wird der Folientitel als \textcolor{red}{\LaTeX, $i^2=-1$} 
erscheinen. 
Dieser Text wird nicht korrekt als Lesezeichen 
angezeigt werden. Es wurde also ein Versuch unternommen, dieses zu 
korrigieren, aber oftmals ergibt dies nicht denselben Text. Der genannte 
Titel würde als Lesezeichen folgendermaßen angezeigt werden:
\texttt{redLaTeX, i2=-1}.
Auf der anderen Seite wird der manuell erstellte Lesezeicheneintrag so angezeigt:
\texttt{LaTeX, i*i=-1}. Man beachte, dass kein Eintrag im Inhaltsverzeichnis vorgenommen 
werden wird, da der Wert \texttt{toc=} genutzt wurde.

Zusätzlich zur \texttt{slide}-Umgebung kann jeder individuelle Stil seine eigenen 
Umgebungen definieren. Viele Stile haben eine \texttt{wideslide}-Umgebung. Dahinter 
steht die Idee, dass gewisser Inhalt aus Platzgründen schlecht oder überhaupt 
nicht mit dem Inhaltsverzeichnis zusammen auf einer Folie realisiert werden kann. 
In diesem Fall ist es von Vorteil, eine Folie zu nutzen, die das Inhaltsverzeichnis 
nicht aufführt. Die \texttt{wideslide}-Umgebung enthält diese Funktion und bietet so 
mehr Platz für den eigentlichen Folieninhalt. Abschnitt~\ref{sec:styles} enthält 
mehr Informationen bezüglich der unterschiedlichen Umgebungen der einzelnen Stile.

\section{Overlays}\label{sec:overlays}

Oft möchte man nicht, dass alle Informationen auf einer Folie gleichzeitig erscheinen, sondern vielmehr, 
dass eine nach der anderen auftaucht. Bei \pf{powerdot} wird das mittels Overlays realisiert. 
Jede einzelne Folie kann viele Overlays enthalten, wobei die Overlays eines nach dem anderen ausgegeben
werden.

\subsection{Der \slash{pause} -Befehl}\label{sec:pause}

\marginpar{\slash{pause}}
Der Befehl \slash{pause} ist die einfachste Möglichkeit, Informationen aufeinanderfolgend 
auszugeben.
\begin{Verbatim}[frame=single,fontsize=\small,fillcolor=\color{yellow}]
  \{pause}[<number>]
\end{Verbatim}
Hier ein simples Beispiel:
\begin{Verbatim}[frame=single,fontsize=\small]
 \begin{slide}{Simple overlay}
   power\pause dot
 \end{slide}
\end{Verbatim}
Der Informationstext auf einer Folie wird nur bis zum Befehl \slash{pause}  ausgegeben, es erscheint also 
nichts anderes auf der Folie als dieses bestimmte Stück Text, solange nicht ein Klick mit der Maus erfolgt 
oder eine Taste gedrückt wird. Erst dann wird der weitere Inhalt der Folie ausgegeben, entweder bis diese 
keine weiteren Informationen mehr enthält oder bis zum nächsten
\slash{pause} Befehl innerhalb derselben Folie. In diesem Beispiel
erscheint \texttt{power} mit dem ersten und \texttt{powerdot} mit dem
zweiten Overlay. Der \slash{pause} Befehl wird oft innerhalb von
\texttt{itemize}- und \texttt{enumerate}-Umgebungen gebraucht, zum Beispiel:\\
\begin{Verbatim}[frame=single,fontsize=\small]
 \begin{slide}{Multiple pauses}
   power\pause dot \pause
   \begin{itemize}
     \item Let me pause\ldots \pause
     \item \ldots while I talk \pause and chew bubble gum. \pause
     \item Perhaps you'll be persuaded.
     \item Perhaps not.
   \end{itemize}
 \end{slide}
\end{Verbatim}
Indem \slash{pause} vor der \texttt{itemize}-Umgebung verwendet wurde, erscheint kein Stichpunkt vor dem 
dritten Overlay. Danach wird ein Stichpunkt nach dem anderen ausgegeben, wobei jeder sein eigenes Overlay 
hat. Mehr Informationen bezüglich der Verwendung von Listen folgen im nächsten Abschnitt.

Ein optionales Argument (in eckigen Klammern) des \slash{pause} Befehls spezifiziert die Nummer der 
Overlays, die als Pause fungieren. Ein Verwendungsbeispiel ist:\\
\begin{Verbatim}[frame=single,fontsize=\small]
 \begin{slide}{Pause longer}
   \begin{itemize}
     \item A \pause
     \item B \pause[2]
     \item C
   \end{itemize}
 \end{slide}
\end{Verbatim}
Bei diesem Beispiel erscheint Stichpunkt \texttt{C} mit dem vierten Overlay. Die Nützlichkeit dieser 
Möglichkeit wird im nächsten Abschnitt deutlicher, entsprechend werden wir dann noch mal ein ähnliches 
Beispiel betrachten.

\subsection{Listenumgebungen}\label{sec:lists}

Die Listenumgebungen \texttt{itemize} und \texttt{enumerate} werden bei \pf{powerdot} in besonderer Weise 
behandelt. Sie haben ein optionales Argument, das im \pf{enumitem}-Paket enthalten ist (siehe [4]). 
\pf{powerdot} liefert eine extra Verschlüsselung für dieses optionale Argument. In den folgenden 
Beispielen wird die \texttt{itemize}-Umgebung zur Auflistung der einzelnen
Punkte verwendet, mit der \texttt{enumerate}-Umgebung funktioniert es aber genauso.

Hier ein Beispiel für den üblichen Gebrauch der \texttt{itemize}-Umgebung:

\begin{Verbatim}[frame=single,fontsize=\small]
 \begin{slide}{Basic itemize}
   \begin{itemize}
     \item A \pause
     \item B \pause
     \item C
   \end{itemize}
 \end{slide}
\end{Verbatim}
Die Ausgabe erfolgt, indem einfach mit jedem Overlay ein Stichpunkt nach dem anderen erscheint.

\marginpar{\textsl{type}}

Angenommen, wir wollten, dass alle Stichpunkte einer Folie zeitgleich auftauchen, dabei aber nur einer 
davon zum entsprechenden Zeitpunkt `aktiv' sein soll. Das wird realisiert
mittels der \texttt{type}-Option für die \texttt{itemize}-Umgebung. Der
vorgegebene Wert ist \texttt{0}.
\begin{Verbatim}[frame=single,fontsize=\small]
 \begin{slide}{Type 1 itemize}
   \begin{itemize}[type=1]
     \item A \pause
     \item B \pause
     \item C
   \end{itemize}
 \end{slide}
\end{Verbatim}
Jetzt wird jeder Stichpunkt in der \emph{inaktiven Farbe}\index{inaktive Farbe|usage} (die durch den 
verwendeten \pf{powerdot}-Stil festgelegt ist) ausgegeben. Mit dem Overlay, mit dem ein Stichpunkt 
normalerweise erst erscheinen würde, bekommt dieser seine eigentliche Farbe und wird darüber aktiv. 
Das Standardverhalten ist mit \texttt{type=0} gegeben.

Listen können auch ineinander gestapelt sein, um so kompliziertere Strukturen zu schaffen. Wenn eine Liste 
eingebettet ist in eine andere, enthält sie die \texttt{type}-Option-Einstellungen der `Mutter'-Liste. Das 
kann 
aber aufgehoben werden, indem die \texttt{type}-Option beim optionalen Argument der eingebetteten Liste 
genauer definiert wird. Unser Beispiel zeigt nur ein mögliches, mittels gestapelter Listen produziertes 
Konstrukt, es können jedoch auch Konstrukte anderer Art und auf andere Weise kreiert werden.
\begin{Verbatim}[frame=single,fontsize=\small]
 \begin{slide}{Nested lists}
   \begin{itemize}
     \item A\pause
     \begin{itemize}[type=1]
       \item B\pause
     \end{itemize}
     \item C
   \end{itemize}
 \end{slide}
\end{Verbatim}
Hier werden \texttt{A} und \texttt{B} mit dem ersten Overlay ausgegeben,
aber \texttt{B} ist inaktiv. Erst mit dem zweiten Overlay wird \texttt{B}
aktiv, mit Overlay 3 wird \texttt{C} sichtbar. 

\subsection{Der \slash{item} -Befehl}

\marginpar{\slash{item}}

Dieser Befehl hat bei \pf{powerdot} noch ein extra Argument (\emph{optional}), das eine flexiblere Produktion 
von Overlays erlaubt als der \slash{pause} Befehl. 
\begin{Verbatim}[frame=single,fontsize=\small,fillcolor=\color{yellow}]
  \item[<label>]<<overlays>>
\end{Verbatim}
Mit diesem optionalen Argument kann man spezifizieren, mit welchem Overlay ein bestimmter Stichpunkt 
ausgegeben wird. Diese Spezifikation ist eine durch ein Komma separierte Liste, wo jeder Stichpunkt die in 
Tabelle \ref{tab:item} angegebenen Notationen nutzen kann.
\begin{table}[htb]\centering
\begin{tabular}{c|l}
Syntax&Meaning\\\hline
\texttt{x}&Nur Overlay \texttt{x}\\
\texttt{-x}&Alle Overlays bis zu \texttt{x}, \texttt{x} eingeschlossen\\
\texttt{x-}&Alle Overlays ab \texttt{x}, \texttt{x} eingeschlossen\\
\texttt{x-y}&Alle Overlays von \texttt{x} bis \texttt{y}, \texttt{x} und \texttt{y} eingeschlossen\\
\end{tabular}
\caption{\slash{item} und \slash{onslide}-Notation}\label{tab:item}
\end{table}
Das <label>-Argument ist für das optionale Argument des \slash{item}
Befehls bei \LaTeX\ Standard. Das
\LaTeX\ -Handbuch \cite{companion} enthält noch mehr Informationen bezüglich dieses Arguments. 

Hier ein Beispiel:\\
\begin{Verbatim}[frame=single,fontsize=\small]
 \begin{slide}{Active itemize}
   \begin{itemize}[type=1]
    \item<1> A
    \item<2> B
    \item<3> C
   \end{itemize}
 \end{slide}
\end{Verbatim}
Wie oben besprochen, sollte \texttt{A} nur bei Overlay 1, \texttt{B} nur bei
Overlay 2 und \texttt{C} nur bei Overlay 3 aktiv sein, im inaktiven Status sollten die jeweiligen 
Stichpunkte wegen \texttt{type=1} in der inaktiven Farbe erscheinen. 

Wenn die \texttt{type}-Option aber als \texttt{type=0} definiert und jedem Stichpunkt eine Overlay-Option 
gegeben wurde, erscheint jeder Stichpunkt nur, wenn er aktiv ist. Ist er inaktiv, wird er nicht auf der Folie 
angezeigt. Weitere Beispiele, die die Syntax für <overlays> demonstrieren, werden im nächsten Abschnitt 
diskutiert.

\subsection{Der \slash{onslide} -Befehl}\label{sec:onslide}

\marginpar{\slash{onslide}}
Overlays können auch unter Verwendung des \slash{onslide} Befehls geschaffen werden.

\begin{Verbatim}[frame=single,fontsize=\small,fillcolor=\color{yellow}]
 \onslide{<overlays>}{<text>}
\end{Verbatim}

Der Befehl benötigt eine <overlays>-Spezifizierung als erstes Argument und
den <text>, der auf der Folie erscheinen soll, als zweites Argument. Die
<overlays>, auf denen der Text erscheinen wird, werden genauer definiert als eine durch ein Komma 
separierte Aufzählung mit der in Tabelle \ref{tab:item} dargestellten Syntax. Wir beginnen mit einem 
einfachen Beispiel:\\
\begin{Verbatim}[frame=single,fontsize=\small]
 \begin{slide}{Simple onslide}
   \onslide{1,2}{power}\onslide{2}{dot}
 \end{slide}
\end{Verbatim}
Wir haben eingerichtet, dass \texttt{power} mit den Overlays 1 und 2
erscheint, \texttt{dot} nur mit Overlay 2. Wie bereits vermutet wird bei diesem Beispiel das Gleiche 
ausgegeben wie bei dem ersten \slash{pause}-Beispiel, einziger Unterschied ist die etwas kompliziertere 
Syntax des \slash{onslide} Befehls. Allerdings erlaubt genau das ein bisschen mehr Flexibilität. 

\marginpar{\slash{onslide+}}
Betrachten wir dasselbe Beispiel mit den folgenden Modifikationen:

\begin{Verbatim}[frame=single,fontsize=\small]
 \begin{slide}{Simple onslide+}
  \texttt{onslide }: \onslide{1}{power}\onslide{2}{dot}\\
  \texttt{onslide+}: \onslide+{1}{power}\onslide+{2}{dot}
 \end{slide}
\end{Verbatim}
Der \slash{onslide+} Befehl gibt seinen Inhalt in völlig anderer Art und
Weise aus. Jetzt erscheint \texttt{dot} mit jedem Overlay, allerdings wird
es außer bei Overlay 2 \textit{nur} in seiner inaktiven
Farbe\index{inaktiven Farbe|usage} ausgegeben. Das ist vergleichbar mit dem
\texttt{type=1}-Verhalten für die Listen (siehe Abschnitt~\ref{sec:lists}).

Wenn wir dieses Beispiel ausführen, werden wir zudem feststellen, dass der
\slash{onslide} Befehl zunächst Material verbirgt, und doch die richtige Menge an Platz dafür reserviert. Bei 
Overlay 2 erscheinen die \texttt{dot}s alle übereinander. Der nächste Befehl reserviert keinen Platz. 

\marginpar{\slash{onslide*}}
Statt bestimmtes Material zu verbergen und Platz dafür zu reservieren
(\slash{onslide}) oder <text> außer beim entsprechenden Overlay (<overlays>) in der inaktiven Farbe
auszugeben (\slash{onslide+}), gibt dieser Befehl das Material einfach ohne jede weitere Formatierung aus. 
Betrachten wir folgendes Beispiel, um den Unterschied zu verstehen:

\begin{Verbatim}[frame=single,fontsize=\small]
 \begin{slide}{Simple onslide*}
  \texttt{onslide }: \onslide{1}{power}\onslide{2}{dot}\\
  \texttt{onslide+}: \onslide+{1}{power}\onslide+{2}{dot}\\
  \texttt{onslide*}: \onslide*{1}{power}\onslide*{2}{dot}
 \end{slide}
\end{Verbatim}
Wir sind bereits vertraut mit der Ausgabe der ersten zwei Zeilen. Die
dritte Zeile gibt \texttt{power} mit Overlay 1 und \texttt{dot} mit Overlay 2 aus, allerdings ist bei 
Overlay 2 kein Platz für \texttt{power} reserviert. Stattdessen wird
\texttt{dot} an der gleichen Position des Cursors beginnen, bei der
\texttt{power} mit dem ersten Overlay ausgegeben wurde, und es ist auch nicht in einer Linie unter den 
anderen \texttt{dot}s angeordnet.

Abschließend betrachten wir ein Syntax-Beispiel, das sowohl mit
\slash{item} als auch mit \slash{onslide} möglich ist.
Noch mal zur Erinnerung: Diese Befehle benötigen eine durch ein Komma separierte Aufzählung für die genauere 
Definierung des <overlays>-Arguments. Dabei kann jedes Element die in Tabelle \ref{tab:item} beschriebene 
Syntax nutzen. Die verschiedenen Varianten sind in folgendem Beispiel demonstriert:

\begin{Verbatim}[frame=single,fontsize=\small]
 \begin{slide}{Lists}
   \onslide{10}{on overlay 10 only}\par
   \onslide{-5}{on every overlay before and including overlay 5}\par
   \onslide{5-}{on every overlay after  and including overlay 5}\par
   \onslide{2-5}{on overlays 2 through 5, inclusive}\par
   \onslide{-3,5-7,9-}{on every overlay except overlays 4 and 8}
 \end{slide}
\end{Verbatim}

\subsection{Relative Overlays}

Manchmal ist es sehr lästig, im Auge zu behalten, wann ein Stichpunkt auftauchen beziehungsweise aktiv 
werden soll, zum Beispiel, wenn man möchte, dass ein bestimmter Text auf dem entsprechenden Overlay 
\textit{nach} einem speziellen Stichpunkt erscheinen soll. Abhilfe dafür leisten relative Overlays, die 
allerdings nicht außerhalb von \slash{item}-Listenumgebungen verwendet werden sollten. Betrachten wir ein 
einfaches, einleuchtendes Beispiel:

 \begin{Verbatim}[frame=single,fontsize=\footnotesize]
 \begin{slide}{Relative overlays}
   \begin{itemize}
    \item A \pause
    \item B \onslide{+1}{(visible 1 overlay after B)}\pause
    \item C \onslide{+2-}{(appears 2 overlays after C, visible until the end)}
    \pause
    \item D \onslide{+1-6}{(appears 1 overlay after D, visible until overlay 6)}
    \pause
    \item E \pause
    \item F \pause
    \item G \onslide{+1-+3}{(appears 1 overlay after G for 3 overlays)}
    \pause
    \item H \pause
    \item I \pause
    \item J \pause
    \item K
   \end{itemize}
 \end{slide}
\end{Verbatim}
Wie zu sehen ist, wird auch hier der \slash{onslide} Befehl genutzt, die einzige Veränderung der Syntax ist 
die Auflistung der Overlays. Dadurch kann ein `\texttt{+}'-Zeichen in der Liste genauer definiert werden. In 
der simpelsten Verwendung wird durch den \slash{onslide}\{+1\} Befehl der entsprechende Text ein Overlay nach 
demjenigen ausgegeben, auf dem er \textit{eigentlich} erschienen wäre. Nach wie vor kann die in Tabelle 
\ref{tab:item} dargestellte Syntax verwendet werden, demonstriert im oben
stehenden Beispiel. \slash{onslide}\{+1-6\} bewirkt ebenfalls, dass der entsprechende Text ein Overlay nach 
demjenigen ausgegeben wird, auf dem er eigentlich erschienen wäre, und dass schon ausgegebene Textpassagen
bis Overlay 7 gezeigt bleiben. In der letzten Demonstration des oben stehenden Beispiels wird gezeigt, wie 
man den Text einer ganzen Reihe von relativen Overlays erscheinen lassen kann.

\section{Präsentationsstrukturen}\label{sec:structure}
\subsection{Abschnitte herstellen}\label{sec:section}

\marginpar{\slash{section}}
Dieser Abschnitt beschreibt den \slash{section} Befehl, der die Möglichkeit eröffnet, eine Präsentation zu 
strukturieren.
\begin{Verbatim}[frame=single,fontsize=\small,fillcolor=\color{yellow}]
  \section[<options>]{<section title>}
\end{Verbatim}
Dieser Befehl produziert eine Folie mit dem <section title> und nutzt außerdem den eingesetzten Text für die 
Repräsentation des entsprechenden Abschnitts in einem Inhaltsverzeichnis und der Lesezeichenliste. Es gibt 
diverse <options> um die Ausgabe zu kontrollieren.

\marginpar{\textsl{option}\\\texttt{tocsection}}
Diese Option kontrolliert die Repräsentation eines Abschnittes im Inhaltsverzeichnis. Der vorgegebene Wert 
ist \texttt{true}.
\begin{description}
\item[\fcolorbox{black}{gray!30}{tocsection=true}]
So wird ein Abschnitt im Inhaltsverzeichnis geschaffen. Das bedeutet, dass alle nun folgenden Folien unter 
diesem Gliederungspunkt erscheinen, bis ein neuer eingefügt wird. 
\item[\fcolorbox{black}{gray!30}{tocsection=false}]
Auf diese Art und Weise wird kein Abschnitt im Inhaltsverzeichnis geschaffen, somit wird dieser als normale 
Folie eingeordnet und aufgeführt.
\item[\fcolorbox{black}{gray!30}{tocsection=hidden}]

So wird ein Abschnitt im Inhaltsverzeichnis hergestellt, aber er ist nur sichtbar, wenn man eine Folie 
ansieht, die zu diesem Abschnitt gehört. Diese Funktion könnte verwendet werden, um einen zu diskutierenden 
Abschnitt an die Präsentation anzuhängen, der aber nur dann gebraucht und entsprechend gezeigt wird, wenn
noch genügend Zeit für diese Diskussion ist. 
\end{description}

\marginpar{\textsl{option}\\\texttt{slide}}

Diese Option bestimmt, ob der \slash{section} Befehl eine Folie schafft.
Der vorgegebene Wert ist \slash{true}.
\begin{description}
\item[\fcolorbox{black}{gray!30}{slide=true}]

So wird eine Folie hergestellt.
\item[\fcolorbox{black}{gray!30}{slide=false}]

So wird keine Folie hergestellt. Wenn auch \texttt{tocsection=false} ist,
bewirkt der \slash{section} Befehl gar nichts. Wenn ein Inhaltsverzeichnisabschnitt hergestellt wird 
(\texttt{tocsection= true  oder hidden}), der Abschnitt aber selbst keine eigene Folie hat, verweist seine 
Verknüpfung auf die erste Folie unter diesem Abschnitt.
\end{description}

\marginpar{\textsl{option}\\\texttt{template}} 
Diese Option kann verwendet werden, um eine Abschnittsfolie mit einer anderen Schablone zu erstellen. Bei 
default wird eine normale \slash{slide}-Umgebung genutzt, um eine Abschnittsfolie herzustellen, aber wenn 
ein Stil andere Schablonen bietet, die für einen bestimmten Zweck genutzt werden könnten (zum Beispiel die 
\texttt{wideslide}-Umgebung), dann ermöglicht diese Option die Nutzung der entsprechenden Schablone. 
Abschnitt ~\ref{sec:styles} gibt einen Überblick über die verschiedenen Stile und deren verfügbaren 
Schablonen. 

Letztendlich können alle für normale Folien verfügbare Optionen auch für Folien verwendet werden, die mit 
dem \slash{section} Befehl erstellt worden sind (siehe Abschnitt ~\ref{sec:slides}). Wenn ein Abschnitt mit 
einer \texttt{tocsection}-Option erstellt wird, entfernen \texttt{toc=} oder
\texttt{bm=} das einleitende Inhaltsverzeichnis oder das entsprechende Lesezeichen nicht.

\subsection{Das Erstellen einer Übersicht}\label{sec:tableofcontents}

\marginpar{\slash{tableofcontents}}
Dieser Befehl erstellt eine Übersicht (Gliederung) für Präsentationen und kann nur für Folien genutzt werden.
\begin{Verbatim}[frame=single,fontsize=\small,fillcolor=\color{yellow}]
  \tableofcontents[<options>]
\end{Verbatim}
Es gibt diverse <options> um die Ausgabe dieses Befehls zu kontrollieren.

\marginpar{\textsl{option}\\\texttt{type}}
Diese Option bestimmt, ob gewisses Material (abhängig von der Eingabe für
die \texttt{content}-Option weiter unten) versteckt oder in der inaktiven Farbe\index{inaktiven Farbe|usage} 
ausgegeben wird. Der vorgegebene Wert ist \texttt{0}. Sie ist vergleichbar
mit der \texttt{type}-Option für Listenumgebungen (Abschnitt~\ref{sec:lists}).

\begin{description}
\item[\fcolorbox{black}{gray!30}{type=0}]

Wenn Material nicht dem gefragten Typ (bei der \texttt{content}-Option spezifiziert) entspricht, wird es 
versteckt.
\item[\fcolorbox{black}{gray!30}{type=1}]

Genau wie \texttt{type=0}, nur dass das Material nicht versteckt, sondern in der inaktiven Farbe ausgegeben 
wird.
\end{description}

\marginpar{\textsl{option}\\\texttt{content}}
Die \texttt{content}-Option kontrolliert, welche Elemente in den Überblick aufgenommen werden. Der 
vorgegebene Wert ist \texttt{all}. Die unten stehende Beschreibung setzt
für die \texttt{type}-Option \texttt{type=0} voraus, es ist aber kein Problem, den alternativen Text für 
\texttt{type=1} daraus zu folgern.

\begin{description}
\item[\fcolorbox{black}{gray!30}{content=all}]

So wird der vollständige Überblick einer Präsentation ausgeben, einschließlich aller Abschnitte und der 
Folien, die nicht in Abschnitten versteckt sind (siehe Abschnitt~\ref{sec:section}).
\item[\fcolorbox{black}{gray!30}{content=sections}]

So werden nur die Abschnitte in der Präsentation ausgegeben.
\item[\fcolorbox{black}{gray!30}{content=currentsection}]

So wird nur der aktuelle Abschnitt ausgegeben.
\item[\fcolorbox{black}{gray!30}{content=future}]

So wird der gesamte Inhalt beginnend bei der aktuellen Folie ausgegeben.
\item[\fcolorbox{black}{gray!30}{content=futuresections}]

So werden alle Abschnitte beginnend beim aktuellen Abschnitt ausgegeben.
\end{description}

Diesen Abschnitt beendet ein kleines Beispiel, das demonstriert, wie man eine Präsentation entwickelt, die 
einen allgemeinen Überblick der Abschnitte in der Präsentation enthält, eine grundsätzliche Idee des Inhalts 
und für jeden Abschnitt eine detaillierte Übersicht über seine einzelnen Folien liefert.
\begin{Verbatim}[frame=single,fontsize=\small]
 \begin{slide}[toc=,bm=]{Overview}
   \tableofcontents[content=sections]
 \end{slide}
 \section{First section}
 \begin{slide}[toc=,bm=]{Overview of the first section}
   \tableofcontents[content=currentsection,type=1]
 \end{slide}
 \begin{slide}{Some slide}
 \end{slide}
 \section{Second section}
 ...
\end{Verbatim}

\section{Sonstiges}
\subsection{Anmerkungen}\label{sec:notes}

\marginpar{\texttt{note}}
Die \texttt{note}-Umgebung kann verwendet werden, um persönliche Anmerkungen in den eigentlichen Folientext 
einzugliedern. Die Ausgabe der Anmerkungen kann mittels der
\texttt{display}-Option kontrolliert werden (siehe Abschnitt~\ref{sec:classopts}). Hier ist ein Beispiel:

\begin{Verbatim}[frame=single,fontsize=\small]
 \begin{slide}{Chewing gum}
 ...
 \end{slide}
 \begin{note}{Reminder for chewing gum}
   Don't forget to mention that chewing gum is sticky.
 \end{note}
\end{Verbatim}

\subsection{Leere Folien}\label{sec:emptyslides}

\marginpar{\texttt{emptyslide}}
Die \texttt{emptyslide}-Umgebung stellt eine völlig leere Folie her. Die Textbox auf der Folie könnte für 
spezielle Dinge verwendet werden, wie zum Beispiel die Ausgabe von Fotos. Diese Funktion erlaubt also auch 
das Erstellen und Wiedergeben einer Diashow. Zum Beispiel:

\begin{Verbatim}[frame=single,fontsize=\small]
 \begin{emptyslide}{}
   \centering
   \vspace{\stretch{1}}
   \includegraphics[height=0.8\slideheight]{me_chewing_gum.eps}
   \vspace{\stretch{1}}
 \end{emptyslide}
\end{Verbatim}
Der \slash{includegraphics} Befehl wird durch das \pf{graphicx}-Paket \cite{graphics} definiert. Der 
\slash{stretch} Befehl wird verwendet, um das Bild vertikal zu zentrieren. Beide Befehle sind in Ihrem 
Lieblings-\LaTeX\ -Handbuch beschrieben, zum Beispiel \cite{companion}. Man kann zudem die Längen 
\slash{slideheight} und \slash{slidewidth} nutzen, um das Bild maßstabsgetreu der Folie anzupassen.

\subsection{Die Folie mit der Bibliographie}\label{sec:bib}

\marginpar{\texttt{thebibliography}}
\pf{powerdot} benennt die standardisierte \pf{article}
\texttt{thebibliography}-Umgebung um, damit die Erstellung einer Abschnittsüberschrift und fortlaufender 
Kopfzeilen unterdrückt wird. Alle anderen Bestandteile wurden beibehalten. Jeweils eines der folgenden 
Beispiele können Sie verwenden (abhängig davon, ob Sie BiB\TeX\ nutzen oder nicht):\\
\begin{minipage}[t]{.49\linewidth}
\begin{Verbatim}[frame=single,fontsize=\small]
 \begin{slide}{Slide}
  \cite{someone}
\end{slide}
\begin{slide}{References}
 \begin{thebibliography}{1}
 \bibitem{someone} Article of 
                   someone.
 \end{thebibliography}
\end{slide}
\end{Verbatim}
\end{minipage}\hfill
\begin{minipage}[t]{.49\linewidth}
\begin{Verbatim}[frame=single,fontsize=\small]
 \begin{slide}{Slide}
   \cite{someone}
 \end{slide}
 \begin{slide}{References}
   \bibliographystyle{plain}
   \bibliography{YourBib}
 \end{slide}
\end{Verbatim}
\end{minipage}

Für den Fall längerer Bibliographien, die auf mehrere Folien verteilt werden sollen, empfiehlt sich die 
Verwendung der Pakete \pf{natbib} und \pf{bibentry} \cite{natbib}. Das erlaubt Folgendes:
\begin{Verbatim}[frame=single,fontsize=\small]
 \begin{slide}{References (1)}
   \bibliographystyle{plain}
   \nobibliography{YourBib}
   \bibentry{someone1}
   \bibentry{someone2}
 \end{slide}
 \begin{slide}{References (2)}
   \bibentry{someone3}
 \end{slide}
\end{Verbatim}
Werfen Sie einen Blick in Ihr Lieblings-\LaTeX\ -Handbuch für weitere Informationen zum Zitieren und zu 
Bibliographien.

\subsection{Wortwörtliche Wiedergabe auf Folien}\label{sec:verbatim}

\marginpar{\textsl{option}\\\texttt{verbatim}}
\pf{powerdot} hat drei verschiedene Methoden, Folien aufzubereiten, von denen zwei hauptsächlich entwickelt 
wurden, um das Einbeziehen wortwörtlichen Inhalts\footnote{Und anderer Inhalt, der bei der Verarbeitung 
catcode-Veränderungen benötigt wird.} auf Folien einfacher zu gestalten. Diese
Methoden können beim \texttt{method} key abgerufen werden, der in
Folienumgebungen und dem \slash{pdsetup} Befehl verfügbar ist (siehe Abschnitt~\ref{sec:glopts}).
\begin{description}
\item[\fcolorbox{black}{gray!30}{method=normal}]

Dies ist die vorgegebene Methode der Aufbereitung von Folien. Sie ist schnell und erlaubt die Verwendung 
von Overlays, nicht aber die wortwörtliche Wiedergabe.\footnote{Außer wenn es in einer Box außerhalb der 
Folie gesichert wurde.} 
\item[\fcolorbox{black}{gray!30}{method=direct}]

Diese Methode ist auch schnell, aber sie erlaubt nicht die Verwendung von Overlays. Overlays werden 
unauffällig außer Gefecht gesetzt. Allerdings ermöglicht diese Methode wortwörtlichen Inhalt auf Folien.
\item[\fcolorbox{black}{gray!30}{method=file}]

Diese Methode nutzt einen provisorischen Ordner, um den Folienkörper zu exportieren und wieder einzulesen. 
Damit erlaubt sie die Verwendung von Overlays und wortwörtlichen Inhalt, aber sie ist mitunter langsamer, 
wenn viele Folien mit dieser Methode aufbereitet werden, da das Ordnersystem dann intensiv genutzt ist. 
\end{description}

Hier ist ein Beispiel, dass die Nutzung aller drei Folienaufbereitungsmethoden demonstriert:

\begin{Verbatim}[frame=single,fontsize=\small]
 \documentclass{powerdot}
 \usepackage{listings}
 \lstnewenvironment{code}{
   \lstset{frame=single,escapeinside=`',
   backgroundcolor=\color{yellow!20},
   basicstyle=\footnotesize\ttfamily}
 }{}
 \begin{document}
 \begin{slide}{Slide 1}
 Normal \pause content.
 \end{slide}
 \begin{slide}[method=direct]{Slide 2}
 Steps 1 and 2:
 \begin{code}
 compute a;`\pause'
 compute b;
 \end{code}
 \end{slide}
 \begin{slide}[method=file]{Slide 3}
 Steps 1 and 2:
 \begin{code}
 compute a;`\pause'
 compute b;
 \end{code}
 \end{slide}
 \end{document}
\end{Verbatim}
Die erste Folie zeigt das Standardverhalten für normalen Inhalt, sie produziert zwei Overlays. Trotz der 
Verwendung des \slash{pause} Befehls erstellt die zweite Folie keine Overlays. Dieser Befehl wurde 
untauglich gemacht mit der Wahl für die \texttt{direct}-Methode, um wortwörtlichen Inhalt aufzubereiten. Die 
dritte Folie hat das gleiche Aussehen, wie die zweite, aber jetzt werden zwei Overlays produziert, weil die 
Methode, die einen provisorischen Ordner nutzt, verwendet wurde.
\slash{pause} wurde hier innerhalb der Auflistung gebraucht, der Befehl kann aber genauso außerhalb von 
Listenumgebungen benutzt werden.

\subsection{Der \slash{twocolumn} Befehl}\label{sec:twocolumn}

\marginpar{\slash{twocolumn}}
Das \slash{twocolumn}-Makro erlaubt, den Inhalt auf zwei Spalten aufzuteilen.
\begin{Verbatim}[frame=single,fontsize=\small,fillcolor=\color{yellow}]
  \twocolumn[<options>]{<left>}{<right>}
\end{Verbatim}
So wird <left> und <right> in zwei Spalten gesetzt. Die Abmessungen dieser
Spalten können mit <options> kontrolliert werden. 
Hier die verfügbaren Optionen:

\begin{description}
\marginpar{\textsl{option}\\\texttt{lineheight}}
\item[\fcolorbox{black}{gray!30}{lineheight}]

Wenn \texttt{lineheight} spezifiziert wird, erscheint durch \slash{psline} eine Linie von bestimmter Höhe 
zwischen den Spalten. Beispiel: \texttt{lineheight=6cm}.

\marginpar{\textsl{option}\\\texttt{lineprop}}
\item[\fcolorbox{black}{gray!30}{lineprop}]

Mit jeder \pf{pstricks}-Angabe können die Linienproportionen näher bestimmt werden. Beispiel:

\begin{Verbatim}[frame=single,fontsize=\small]
 lineprop={linestyle=dotted,linewidth=3pt}
\end{Verbatim}

\marginpar{\textsl{option}\\\texttt{lfrheight\\lfrprop}}
\item[\fcolorbox{black}{gray!30}{lfrheight} \quad \fcolorbox{black}{gray!30}{lfrprop}]

Ersteres schafft einen Rahmen von bestimmter Höhe um die linke Spalte,
zweiteres ist wie \texttt{lineprop}, aber für den linken Rahmen.

\marginpar{\textsl{option}\\\texttt{rfrheight\\rfrprop}}
\item[\fcolorbox{black}{gray!30}{rfrheight} \quad 
\fcolorbox{black}{gray!30}{rfrprop}]

Genau wie \texttt{lfrheight} und \texttt{lfrprop}, allerdings für den rechten Rahmen.

\marginpar{\textsl{option}\\\texttt{lcolwidth\\rcolwidth}}
\item[\fcolorbox{black}{gray!30}{lcolwidth}
\quad \fcolorbox{black}{gray!30}{rcolwidth}]

Die Weite von linker und rechter Spalte. Beide sind vorgegeben mit
\texttt{0.47}\slash{linewidth}.

\marginpar{\textsl{option}\\\texttt{frsep}}
\item[\fcolorbox{black}{gray!30}{frsep}]

Platz zwischen Text und Rahmen, vorgegeben mit \texttt{1,5mm}.

\marginpar{\textsl{option}\\\texttt{colsep}}
\item[\fcolorbox{black}{gray!30}{colsep}]

Platz zwischen den beiden Spalten, vorgegeben mit \texttt{0.06}\slash{linewidth}.

\marginpar{\textsl{option}\\\texttt{topsep}}
\item[\fcolorbox{black}{gray!30}{topsep}]

Der extra Platz (zusätzlich zu \slash{baselineskip}) zwischen dem Text über den Spalten und dem Text in den 
Spalten, vorgegebener Wert: \texttt{0cm}.

\marginpar{\textsl{option}\\\texttt{bottomsep}}
\item[\fcolorbox{black}{gray!30}{bottomsep}]

Der extra Platz zwischen dem Text in den Spalten und dem Text darunter,
vorgegebener Wert: \texttt{0cm}.

\marginpar{\textsl{option}\\\texttt{indent}}
\item[\fcolorbox{black}{gray!30}{indent}]

Horizontaler Zeileneinzug links zur linken Spalte, vorgegebener Wert:
\texttt{0cm}. 
\end{description}
Die oben beschriebenen Abmessungen sind in Figur \ref{fig:twocolumndim} graphisch dargestellt.
\begin{figure}[htb]
\centering
\begin{pspicture}(0,.5)(13,10.5)
\psline(0,0.5)(0,10)
\rput[tl](.05,9.95){Top}
\psframe[dimen=middle](1,9)(7,2)
\psline{C-C}(8.5,9)(11,9)
\psline{C-C}(8.5,2)(8.5,9)
\psline{C-C}(8.5,2)(11,2)
\qdisk(1.7,8.3){.1cm}
\psset{linestyle=dashed}
\psline{C-C}(1.7,8.3)(6.3,8.3)
\psline{C-C}(1.7,8.3)(1.7,3)
\psline{C-C}(6.3,5)(6.3,8.3)
\psline{C-C}(11,9)(12,9)
\psline{C-C}(11,2)(12,2)
\psline{C-C}(11,7)(12,7)
\psline{C-C}(9.2,8.3)(12,8.3)
\psline{C-C}(9.2,8.3)(9.2,3)
\rput[tl](1.75,8.25){Left column text}
\rput[tl](9.25,8.25){Right column text}
\rput[tl](.05,1){Bottom}
\psset{linestyle=dotted,dotsep=2pt}
\psline(0,8.3)(1.7,8.3)
\psline(0,9.6)(1,9.6)
\psline(0,2)(1,2)
\psline(0,1.1)(1,1.1)
\psset{linestyle=solid}
\psline{<->}(.2,8.33)(.2,9.57)
\psline{<->}(4,8.33)(4,8.97)
\psline{<->}(1.73,7)(6.27,7)
\psline{<->}(1.03,6.5)(1.67,6.5)
\psline{<->}(0.03,5.5)(1.67,5.5)
\psline{<->}(6.33,7.4)(9.17,7.4)
\psline{<->}(8.53,6.5)(9.17,6.5)
\psline{<->}(6.33,6.5)(6.97,6.5)
\psline{<->}(10.7,8.33)(10.7,8.97)
\psline{<->}(7.3,8.97)(7.3,2.03)
\psline{<->}(.2,1.13)(.2,1.97)
\psline{->}(1.7,9.3)(1.7,8.45)
\psline{<-}(9.23,7)(11,7)
\cput(4,6.6){\small 1}
\cput(11.1,6.6){\small 2}
\cput(8,7){\small 3}
\cput(7.7,3){\small 4}
\cput(4.4,8.65){\small 5}
\cput(1.35,6.1){\small 5}
\cput(8.85,6.1){\small 5}
\cput(11.1,8.65){\small 5}
\cput(6.65,6.1){\small 5}
\cput(0.6,8.95){\small 6}
\cput(0.6,5.1){\small 7}
\cput(0.6,1.55){\small 8}
\cput(1.7,9.6){\small 9}
\end{pspicture}
\begin{tabular}{c p{4cm}cl}
\multicolumn{4}{c}{Bedeutung der labels}\\\hline
1&\texttt{lcolwidth}&5&\texttt{frsep}\\
2&\texttt{rcolwidth}&6&\texttt{topsep}\\
3&\texttt{colsep}&7&\texttt{indent}\\
4&\texttt{lfrheight}, \texttt{rfrheight},&8&\texttt{bottomsep}\\
&\texttt{lineheight}&9&Reference point
\end{tabular}
\caption{Two-column dimensions.}\label{fig:twocolumndim}
\end{figure}
Wichtig zu bemerken ist, dass das \slash{twocolumn}-Makro die aktuelle Cursor-Position als Bezugspunkt für 
die erste Zeile des Textes der linken Spalte nutzt (siehe auch Figur \ref{fig:twocolumndim}). Das heißt, 
dass der optionale Rahmen bis zum Text der vorausgehenden Zeile ausgedehnt werden kann. Man kann in diesem 
Fall beispielsweise \texttt{topsep=0,3cm} einfügen, um extra Platz zwischen diesen beiden Textzeilen zu 
schaffen. Der vorgegebene \texttt{topsep}-Wert basiert darauf, dass sich kein Text unmittelbar über den 
Spalten befindet. In diesem Fall lokalisiert man am besten den Ort der ersten Textzeile der linken Spalte am 
gleichen Punkt, bei dem ein gewisser Text nicht durch den \slash{twocolumn} Befehl auf anderen Folien 
erstellt wird. Die \texttt{topsep=0}-Einstellung bewirkt genau das. Durch
die Kombination von \texttt{topsep} und \texttt{indent} lässt sich dieses Verhalten und die Position der 
ersten Textzeile der linken Spalte beliebig ändern.

Das \slash{twocolumn}-Makro errechnet die Größe der Konstruktion, um den Text darunter korrekt zu 
positionieren. Die Berechnung ist fertig, wenn für \texttt{lfrheight},
\texttt{rfrheight}, \texttt{lineheight} (falls genauer bestimmt) das Maximum festgesetzt wird. So werden 
rechte und linke Spalte stimmig ausgegeben. Wenn weder Rahmen noch Linien
eingefügt werden, setzt \texttt{bottomsep} den horizontalen Platz zwischen der untersten Textzeile der 
Spalten und dem Text unter den Spalten (zusätzlich zu \slash{baselineskip}). Hier ein Beispiel:

\begin{Verbatim}[frame=single,fontsize=\small]
 \begin{slide}{Two columns}
   Here are two columns.
   \twocolumn[
     lfrprop={linestyle=dotted,linewidth=3pt},
     lfrheight=4cm,rfrheight=5cm,lineheight=3cm,topsep=0.3cm
   ]{left}{right}
   Those were two columns.
 \end{slide}
\end{Verbatim}
Hier könnte die Verwendung der \pf{xkeyval}-Befehle \slash{savevalue} und
\slash{usevalue} \index{savevalue=\verb!*+\savevalue+|usage}\index{usevalue=\verb!*+\usevalue+|usage} 
nützlich sein, zum Beispiel wenn man die Eigenschaften des linken Rahmens für den rechten kopieren will. So 
vermeidet man nicht nur eine überflüssige zweite Typisierung, sondern auch, Fehler zu machen, die in 
unterschiedlich großen Rahmen resultieren würden. Betrachten wir das unten stehende Beispiel:
\begin{Verbatim}[frame=single,fontsize=\small]
\twocolumn[
   \savevalue{lfrheight}=3cm,
   \savevalue{lfrprop}={
     linestyle=dotted,framearc=.2,linewidth=3pt},
   rfrheight=\usevalue{lfrheight},
   rfrprop=\usevalue{lfrprop}
 ]{left}{right}
\end{Verbatim}
Ziehen Sie die \pf{xkeyval}-Dokumentation \cite{xkeyval} zu Rate, um mehr
über die \slash{savevalue} und \slash{usevalue} Befehle zu erfahren.

\section{Zur Verfügung stehende Stile}\label{sec:styles}

\pf{Powerdot} enthält eine Zahl von Stilen, welche im nachfolgenden 
Überblick aufgeführt sind. Die Charakteristik jedes Stils ist kurz 
beschrieben und von einem Beispiel einer Titelfolie und einer normalen 
Folie begleitet. Styles die auf der \texttt{wideslide} Umgebung beruhen, haben 
ein Inhaltsverzeichnis auf dem linken Rand im Querformat (landscape) 
und am unteren Rand im Hochformat (portrait) der Folien. Das Hochformat 
wird unterstützt, sofern es nicht anders angelegt wird.
\begin{description}
\item\pf{default}\\
Dieser Standard-Stil bietet sechs verschiedene Layouts. Jedes wird von
einer Blume in der linken, oberen Ecke dekoriert. Das blaue Layout des
default-Stils hat als Hauptfarben hellblau, blau und weiß wie im Beispiel
unterhalb. Andere verfügbare Layouts sind \texttt{red}, \texttt{green}, 
\texttt{yellow}, \texttt{brown} and \texttt{purple}.\\

\includegraphics[angle={-90},width=5cm]{powerdot-styleexample-default.001}
\quad
\includegraphics[angle={-90},width=5cm]{powerdot-styleexample-default.002}

\end{description}


\begin{description}
\item\pf{simple}\\
Dies ist ein einfacher Stil in schwarz und weiß. Er könnte für Folien
nützlich sein, die ausgedruckt werden sollen.\\

\includegraphics[angle={-90},width=5cm]{powerdot-styleexample-simple.001}
\quad
\includegraphics[angle={-90},width=5cm]{powerdot-styleexample-simple.002}

\item\pf{tycja}\\
Dieser Stil zeigt sich in gelben und dunkelblauen Schattierungen. Im 
Querformat ist das Inhaltsverzeichnis auf der rechten Seite und im Hochformat
am unteren Rand der Folien.\\

\includegraphics[angle={-90},width=5cm]{powerdot-styleexample-tycja.001}
\quad
\includegraphics[angle={-90},width=5cm]{powerdot-styleexample-tycja.002}

\item\pf{ikeda}\\
In diesem Stil zeigen sich die Folien in dunklen Schattierungen von Rot und Blau
sowie dazu heller Textfarbe. Zudem wird ein schmuckvolles Muster auf der Folie
verwendet.\\

\includegraphics[angle={-90},width=5cm]{powerdot-styleexample-ikeda.001}
\quad
\includegraphics[angle={-90},width=5cm]{powerdot-styleexample-ikeda.002}

\item\pf{fyma}\\
Dieser Stil wurde von Laurent Jaques für \pf{prosper} entworfen und darauf 
basierend, entwickelte er eine Version für \pf{HA-prosper} mit zusätzlichen
Eigenschaften. Mit freundlicher Genehmigung wurde der Stil von Shun'ichi J. Amano
konvertiert, um ihn auch für \pf{powerdot} verfügbar zu machen. Das Layout 
\texttt{blue} hat ein elegantes Design mit einem Verlauf aus hellblau und weiß als
Hintergrund (siehe Beispiel). Weitere Layouts sind \texttt{green}, \texttt{gray},
\texttt{brown} und \texttt{orange}. Außerdem gibt es spezielle Schablonen für
Bereiche auf Folien und breite Folien.\\

\includegraphics[angle={-90},width=5cm]{powerdot-styleexample-fyma.001}
\quad
\includegraphics[angle={-90},width=5cm]{powerdot-styleexample-fyma.002}

\item\pf{ciment}\\
Entworfen von Mathieu Goutelle für \pf{prosper} und \pf{HA-prosper} ist auch
dieser Stil dank freundlicher Genehmigung umgewandelt worden und somit für
\pf{powerdot} verfügbar. Der Stil hat einen mit hellgrauen Linien, horizontal
schraffierten Hintergrund. Inhaltsverzeichnis und Überschriften sind mit einem
dunklen Rot hervor gehoben.\\

\includegraphics[angle={-90},width=5cm]{powerdot-styleexample-ciment.001}
\quad
\includegraphics[angle={-90},width=5cm]{powerdot-styleexample-ciment.002}

\item\pf{elcolors}\\
Dieser Stil benutzt die Farben der Drei-Farben-Theorie nach Thomas Young,
nämlich Schattierungen in rot, blau und gelb.\\

\includegraphics[angle={-90},width=5cm]{powerdot-styleexample-ciment.001}
\quad
\includegraphics[angle={-90},width=5cm]{powerdot-styleexample-ciment.002}

\item\pf{aggie}\\
Jack Stalnaker entwarf diesen Stil für \pf{HA-prosper}und hat ihn dann
für \pf{powerdot} konvertiert. Verwendet wird dunkles Rot und ein helles Braun.\\

\includegraphics[angle={-90},width=5cm]{powerdot-styleexample-aggie.001}
\quad
\includegraphics[angle={-90},width=5cm]{powerdot-styleexample-aggie.002}

\item\pf{husky}\\
Auch dieser Stil stammt von Jack Stalnaker und zeigt markante Rotschrift vor
einem sonnenartigen, hellgrauen Hintergrund.\\

\includegraphics[angle={-90},width=5cm]{powerdot-styleexample-husky.001}
\quad
\includegraphics[angle={-90},width=5cm]{powerdot-styleexample-husky.002}

\item\pf{sailor}\\
Dieser Stil ist von Mael Hill\'ereau beigetragen und bietet fünf verschiedene
Layouts: \texttt{Sea} (the default), \texttt{River}, \texttt{Wine}, 
\texttt{Chocolate} und \texttt{Cocktail}. Abgebildet ist ein Beispiel des
Layouts \texttt{Sea}.\\

\includegraphics[angle={-90},width=5cm]{powerdot-styleexample-sailor.001}
\quad
\includegraphics[angle={-90},width=5cm]{powerdot-styleexample-sailor.002}

\item\pf{upen}\\
Ein tiefes Blau als Hintergrund und gelber Text zeichnen diesen Stil aus.
Er stammt von Piskala Upendran.\\

\includegraphics[angle={-90},width=5cm]{powerdot-styleexample-husky.001}
\quad
\includegraphics[angle={-90},width=5cm]{powerdot-styleexample-husky.002}

\item\pf{bframe}\\
Der \pf{bframe} Stil ist ebenfalls von Piskala Upendran und hat blaue,
abgerundete Boxen in welche der weiße Text eingefügt wird.\\

\includegraphics[angle={-90},width=5cm]{powerdot-styleexample-bframe.001}
\quad
\includegraphics[angle={-90},width=5cm]{powerdot-styleexample-bframe.002}

\item\pf{horatio}\\
Der Stil \pf{horatio} wurde von Michael Lundholm beigesteuert und ist
eher zurückhaltend, in blauer und weißer Farbgebung.\\

\includegraphics[angle={-90},width=5cm]{powerdot-styleexample-horatio.001}
\quad
\includegraphics[angle={-90},width=5cm]{powerdot-styleexample-horatio.002}

\item\pf{paintings}\\
Dies ist ein einfacherer Stil ohne Inhaltsverzeichnis. Er wurde von
Thomas Koepsell entworfen und ermöglicht 10 verschiedene Layouts.
Die dabei jeweils verwendeten Farben lehnen an berühmte Gemälde 
an.\footnote{Der Stil benutzt einen Farbton, \texttt{pdcolor7}, welcher
nicht in den Layouts verwendet wird, aber dennoch wie die anderen
Farben von entsprechenden Gemälden stammt. Der Farbton kann
beispielsweise benutzt werden um einen Text hervor zu heben.}
Welche Gemälde als Vorlage dienten, kann man in der style file, 
der den Stil definierenden Datei nachlesen. Die zum Stil verfügbaren 
Layouts lauten: \texttt{Syndics} (Standard-Layout), \texttt{Skater}, 
\texttt{GoldenGate}, \texttt{Lamentation},
\texttt{HolyWood}, \texttt{Europa}, \texttt{Moitessier},
\texttt{MayThird}, \texttt{PearlEarring} und \texttt{Charon}. Hier
 ist ein Beispiel in \texttt{Syndics}.\\

\includegraphics[angle={-90},width=5cm]{powerdot-styleexample-paintings.001}
\quad
\includegraphics[angle={-90},width=5cm]{powerdot-styleexample-paintings.002}

\item\pf{klope}\\
Der Stil \pf{klope} führt ein horizontales Inhaltsverzeichnis aus, 
das nur die als \texttt{section} deklarierten Gliederungspunkte listet.
Folgende Layouts stehen zur Verfügung: \texttt{Spring}, \texttt{PastelFlower}, 
\texttt{BlueWater} und \texttt{BlackWhite}. Das Layout \texttt{Spring},
welches hier als Beispiel dient, entspricht dem Standard-Layout.\\

\includegraphics[angle={-90},width=5cm]{powerdot-styleexample-klope.001}
\quad
\includegraphics[angle={-90},width=5cm]{powerdot-styleexample-klope.002}

\item\pf{jefka}\\
Dieser Stil bietet vier Layouts: \texttt{brown} (Standard),
\texttt{seagreen}, \texttt{blue} und \texttt{white}. Die Beispielfolien
entsprechen dem Layout \texttt{brown}.\\

\includegraphics[angle={-90},width=5cm]{powerdot-styleexample-jefka.001}
\quad
\includegraphics[angle={-90},width=5cm]{powerdot-styleexample-jefka.002}

\item\pf{pazik}\\
Dieser Stil ist in zwei Layouts verwendbar: \texttt{brown} oder auch
\texttt{red}, wie in der Beispielabbildung.\\

\includegraphics[angle={-90},width=5cm]{powerdot-styleexample-pazik.001}
\quad
\includegraphics[angle={-90},width=5cm]{powerdot-styleexample-pazik.002}

\end{description}

\section{Kompilieren der Präsentation}\label{sec:compiling}
\subsection{Anforderungen}\label{sec:dependencies}
In Tabelle \ref{tab:Anforderungen} ist eine Liste von Dateipaketen
(packages) die \pf{powerdot} verwendet um spezifische Aufgaben
auszuführen. Bedingungen für die jeweiligen Pakete sind in dieser
Tabelle jedoch nicht mit aufgeführt. Die Bezeichnung `benötigt'
meint, dass das verwendete Paket \emph{mindestens} so aktuell sein sollte,
wie die gelistete Version. Die Bezeichnung `getestet' meint, dass
\pf{powerdot} mit dieser Paketversion getestet wurde, aber es auch mit
älteren als der gelisteten Version funktionieren könnte. Tritt bei der
Programmumwandlung ein Fehler auf, empfiehlt es sich, zur Behebung als
erstes die Anforderungen an die benötigten Paketversionen zu prüfen. Um zu
erfahren welche Version eines Pakets aktuell verwendet wird, hilft es den
Befehl \slash{listfiles} in die erste Zeile des \LaTeX Quelltextes zu setzen,
die |.log| Datei zu öffnen und die Dateiliste zu lesen (siehe \LaTeX 
Handbuch für mehr Informationen). Nötige Paket-Updates zur Aktualisierung
sind bei CTAN \cite{CTAN} erhältlich.
\begin{table}[htb]
\centering
\begin{tabular}{c|c|c|c}
Paket/Datei & Version & Datum & benötigt/getestet\\\hline
\pf{xkeyval} \cite{xkeyval} & 2.5c & 2005/07/10 & benötigt\\
\texttt{pstricks.sty} \cite{PSTricksWeb,PSTricks} & 0.2l & 2004/05/12 & benötigt\\
\pf{xcolor} \cite{xcolor} & 1.11 & 2004/05/09 & benötigt\\
\pf{enumitem} \cite{enumitem} & 1.0 & 2004/07/19 & benötigt\\
\pf{article} class & 1.4f & 2004/02/16 & tested\\
\pf{geometry} \cite{geometry} & 3.2 & 2002/07/08 & tested\\
\pf{hyperref} \cite{hyperref} & 6.74m & 2003/11/30 & tested\\
\pf{graphicx} \cite{graphics} & 1.0f & 1999/02/16 & tested\\
\pf{verbatim} & 1.5q & 2003/08/22 & tested
\end{tabular}
\caption{Anforderungen}\label{tab:Anforderungen}
\end{table}

\subsection{Erzeugen und Darstellen der Output-Datei}\label{sec:creation}
Der Quelltext der Präsentation wird von \LaTeX kompiliert. Die dabei
erstellte DVI kann mit dem MiK\TeX's DVI Viewer YAP\footnote{Mit Ausnahme
der Verwendung von \pf{pstricks-add}, welches das DVI-Koordinatensystem
verfälscht.} angezeigt werden. Leider werden von xdvi und kdvi (kile)
nicht alle PostScript-Besonderheiten unterstützt. Daher werden diese die
Präsentation inkorrekt anzeigen. Falls der verwendete DVI Viewer dies
jedoch dennoch unterstützt, sollte sicher gestellt werden, dass die DVI
Anzeigeeinstellungen den Einstellungen der Präsentation angepasst sind.
Im Falle der Verwendung von \texttt{screen} sollte in den DVI Anzeigeeinstellungen
das Briefpapier- bzw. Dokumentformat gewählt werden. Falls der DVI Viewer
Benutzereinstellungen für das Seitenformat zulässt, ist die Einstellung
von 21 mal 28 cm zu verwenden.

Es sei darauf hingewiesen, dass bestimmte Projekte die mir PostScript oder
PDF Programmen erstellt wurden, mit dem DVI Viewer nicht kompatibel sind.
Beispielsweise werden im Gegensatz zur PostScript-Ausgabe nicht alle Elemente
angezeigt (wie bei \slash{pause}, siehe Kapitel~\ref{sec:overlays}) oder es gibt
fehlende Verlinkungen im Inhaltsverzeichnis.

Wenn man ein PostScript-Dokument erzeugen möchte, startet man die DVI mit
dvips \emph{ohne eine Einstellung bezüglich des Formats oder der Seitengröße}.
\pf{Powerdot} wird dann die nötige Information in der DVI-Datei ergänzen, 
die dvips und ps2pdf (ghostscript) hilft, ein angemessenes Dokument zu
erzeugen. Sollte dies aus bestimmten Gründen nicht funktionieren oder man
möchte das Seitenformat selbst spezifizieren, sollte man die Option
\texttt{nopsheader}
verwenden, welche im Abschnitt ~\ref{sec:setup} beschrieben wird. Das
PostScript-Dokument könnte schließlich dazu dienen, mit Unterstützung des
\texttt{psnup} 
Hilfsprogramms mehrfache Folien auf einer Seite zu verwenden.

Um eine PDF Datei für eine Präsentation zu erstellen, kompiliert man die mit
dvips erstellte PS-Datei mit ps2pdf zum PDF-Dokument. Auch hier \emph{muss keine
Angabe für Format oder Seitengröße angegeben werden}. Treten dabei Probleme
auf, ist wieder die Option \texttt{nopsheader} hilfreich, um selber 
Formateinstellungen vornehmen zu können.

\section{Einen eigenen Folien-Stil erstellen}\label{sec:writestyle}
\subsection{Generelle Informationen}

Es ist nicht schwierig \pf{powerdot} Stilvorlagen zu bearbeiten oder selbst
zu erstellen. Wenn man einen Stil modifizieren oder einen neuen entwerfen
will, muss man zunächst die Datei in \TeX\ finden, die als Basis verwendet
werden soll. Diese Stildateien sind mit \texttt{powerdot-<style\_name>.sty} benannt.
Die zu bearbeitende Datei ist vorher zu kopieren und um zu benennen um die
Lizenz\footnote{Die \LaTeX\ Projekt-Lizenz fordert die Umbenennung von
modifizierten Dateien. Siehe auch \url{http://www.latex-project.org/lppl}.}
und Benennungskonflikte zu umgehen. Der neu benannte Stil muss dann dem
lokalen \TeX\-System ergänzt werden, um ihn verwenden zu können (für mehr
Informationen siehe \LaTeX\ Distribution).

Nachdem dies beachtet wurde, kann der neue Stil entworfen werden. Dazu ist
es äußerst hilfreich die Stildateien zu studieren (bspw.
\texttt{powerdot-default.sty}),
während man dieses Kapitel liest. So kann man sich ein gutes Beispiel
zum Inhalt des folgenden Textes heran ziehen. 

Ein Stil hat verschiedene Komponenten, welche nun beschrieben werden sollen.
\begin{description}
\item\textbf{Identifikation und Pakete}\\
Dies identifiziert und lädt alle benötigten Pakete in der Präambel einer
Präsentation. Der Standardstil \pf{default} beinhaltet so etwas wie:
\begin{Verbatim}[frame=single,fontsize=\small]
 \NeedsTeXFormat{LaTeX2e}[1995/12/01]
 \ProvidesPackage{powerdot-default}[2005/10/09 v1.2 default style 
 							    (HA)]
 \RequirePackage{pifont}
\end{Verbatim}
Mehr Informationen über diese Befehle findet man im \LaTeX\ -Handbuch
\cite{companion}.
\item\textbf{Layouts oder Farbwahl}\\
Dieser Abschnitt enthält die Definition der Layouts oder Farben, die in einem
Stil verwendet werden. \pf{Powerdot} verwendet \pf{xcolor} (über \pf{pstricks}).
Daher eignet sich, für mehr Informationen für Farben, die \pf{xcolor}
Dokumentation. Genauere Ausführungen über Layouts zudem in 
Kapitel~\ref{sec:defpals}. 
\item\textbf{Definition von Vorlagen (Templates)}\\
Hiermit beschäftigen sich die Kapitel~\ref{sec:deftemps}
bis~\ref{sec:defbg}.
\item\textbf{Benutzereinstellungen}\\
Alles was Teil des Stils sein soll, kann hier eingeschlossen sein. Der 
\pf{default} Stil beinhaltet beispielsweise die Definition für die Symbole
in Listen, wie \texttt{itemize} sowie einige Bereitstellungen für Listen generell
(umgesetzt mit \slash{pdsetup}, siehe Kapitel~\ref{sec:pdsetup}). In diesem Teil
der *.sty-Datei können also Benutzereinstellungen vorgenommen werden, wie
in Kapitel~\ref{sec:specialtemps} weiter beschrieben.
\item\textbf{Bereitstellung der Schriftart}\\
Dies setzt das Schriftdesign fest, welches durch Laden von Paketen umgesetzt
werden kann, wie beispielsweise \pf{helvet}.
\end{description}

\subsection{Layouts definieren}\label{sec:defpals}
Die ausdrückliche Definition von Vorlagen wird in Kapitel~\ref{sec:deftemps}
thematisiert. Hier sei zunächst nur verdeutlicht, dass die Vorlage (Template) 
die Schablone für den Stil einer Folie ist und somit ihr formales Design 
ausmacht. Layouts hingegen sind Farbgruppen in denen ein Stil ausgeführt 
werden kann. Ein Layout ändert demnach nicht das gesamte Design eines Stils.

\marginpar{\texttt{$\backslash$pddefinepalettes}}
Der folgende Befehl dient der Definition des Layouts eines eigenen Stils.
\begin{Verbatim}[frame=single,fontsize=\small,fillcolor=\color{yellow}]
 \pddefinepalettes{<name>}{<cmds>}...
\end{Verbatim}
Dieser Makro (ein Programmcodeteil) funktioniert mit \emph{jeder} geraden Zahl
obligatorischer Argumente von mindestens zwei. Für jeden <\texttt{name}> eines
Layouts ist eine Gruppe von <\texttt{commands}> zur Definition möglich. Die
<\texttt{commands}> können beispielsweise die Farben \texttt{pdcolor1},
\texttt{pdcolor2}, usw.
festlegen. Die Festlegung für \texttt{pdcolor1} entspricht dabei der Textfarbe. Die
gefärbten Layouts können dann auf den Stil einer Folie, die Schablone,
angewendet werden (siehe Kapitel~\ref{sec:deftemps}).
Zugriff auf die Layouts ist mit dem Key \texttt{palette} für den Befehl
\slash{pdsetup}
möglich (siehe Kapitel~\ref{sec:pdsetup}). Wird kein Layout spezifiziert,
wird das Design gemäß dem ersten Kompilieren der Präsentation ausfallen.
Hier nun ein Beispiel zur Definition zweier Layouts.
\begin{Verbatim}[frame=single,fontsize=\small]
 \pddefinepalettes{reds}{
   \definecolor{pdcolor1}{rgb}{1,0,0}
   \definecolor{pdcolor2}{rgb}{1,.1,0}
   \definecolor{pdcolor3}{rgb}{1,.2,0}
 }{greens}{
   \definecolor{pdcolor1}{rgb}{0,1,0}
   \definecolor{pdcolor2}{rgb}{.1,1,0}
   \definecolor{pdcolor3}{rgb}{.2,1,0}
 }
\end{Verbatim}
In diesem Beispiel dient das Layout \texttt{reds} als Standardfarbgebung für den Stil.
Mehr Informationen über \slash{definecolor} bietet die Dokumentation zum Paket
\pf{xcolor} \cite{xcolor}.

Die Verwendung der Farbennamen \texttt{pdcolor2}, \texttt{pdcolor3}, usw. ist nicht
zwingend nötig. Diese Farben sind jedoch definiert. \pf{Powerdot} verwendet
sie beispielsweise beim Feature \texttt{randomdots} (siehe Kapitel~\ref{sec:glopts}).
Die Flexibilität ermöglicht weitere Festlegungen für Vorlagen von Folien
und deren Layouts. Ein Beispiel wie die Möglichkeiten weiter auszuschöpfen
sind, bietet ein Einblick in den Stil \pf{klope}.

\subsection{Definition von Vorlagen (Templates)}\label{sec:deftemps}
Die Vorlage umfasst eine Gruppe von Festlegungen für Elemente der Folie 
als auch Festlegungen des Benutzers, welche den visuellen Charakter der 
Präsentation bestimmen. Ein Stil kann mehrere Formelemente enthalten.
 
\begin{Verbatim}[frame=single,fontsize=\small,fillcolor=\color{yellow}]
 \pddefinetemplate[<basis>]{<name>}{<options>}{<commands>}
\end{Verbatim}

\marginpar{\texttt{$\backslash$pddefinetemplate}}
Dies definiert die Umgebung <\texttt{name}>, welche eine Vorlage, bestimmt durch
die Charakteristika  <\texttt{basis}>,  <\texttt{options}> und
<\texttt{commands}> erstellt.
Diese Elemente werden in den nächsten Kapiteln genauer erläutert.

Sollen verschiedene Vorlagen erstellt werden, die sich nur gering voneinander
unterscheiden, lohnt es eine  <\texttt{basis}> Schablone zu definieren, von welcher
aus die anderen Varianten erstellt werden. Alle  <\texttt{options}> und
<\texttt{commands}>
für die neue Vorlage  <\texttt{name}> werden der bestehenden Liste von
<\texttt{options}> 
und  <\texttt{commands}> der  <\texttt{basis}> Schablone angefügt.

Die Vorlage sollte mit \emph{einem passenden Namen} benannt, die Neubenennung 
bestehender Schablonen oder Umgebungen jedoch vermieden werden.
\texttt{Blackslide}, 
\texttt{note} und \texttt{emptyslide} definiert \pf{powerdot} intern, womit eine Verwendung
dieser Namen in der Regel vermieden werden sollte. Außerdem besteht jeder
Designstil aus den Vorlagen \texttt{slide} und \texttt{titleslide}. Die
\texttt{titleslide} Umgebung
wird im Standard dazu verwendet, die Titelfolie zu erzeugen während |slide|
(standardmäßig) für Unterkapitel verwendet wird. Titel und Unterkapitel 
verwenden die <\texttt{options}> auf besondere Weise, was detaillierter in 
Kapitel~\ref{sec:specialtemps} erläutert ist.

\subsection{Steuerung des Setup}

\marginpar{\textsl{option}\\\texttt{ifsetup}}
Nachfolgend sind die <\texttt{options}> (Keys bzw. Parameter) beschrieben.
Mit dem Key \texttt{ifsetup} kann kontrolliert werden, wie die Optionen auf 
verschiedene Setups (Programmeinrichtungen) angelegt sind. Jeder verwendete 
Key  der vor der ersten \texttt{ifsetup} Meldung in <\texttt{options}> auftritt, wird bei 
jedem möglichen Setup befolgt. Die dem Key zugewiesenen Parameter werden 
umgesetzt. Ist der \texttt{ifsetup} Key jedoch einmal verwendet, erfolgt die Anwendung 
der untergeordneten Keys nur auf das deklarierte Setup im \texttt{ifsetup} Key. 
Die Parameter untergeordneter Keys werden demnach nur für das zugewiesene 
Setup beachtet. Dabei kann der \texttt{ifsetup} Key mehrere Male verwendet werden.

Mit möglichen Setups sind die zulässigen Werte für die Optionen gemeint, 
wie passende Werte für die Parameter \texttt{mode}, \texttt{paper},
\texttt{orient}, und \texttt{display}
(siehe Kapitel~\ref{sec:classopts}). Wenn ein Wert oder Werte für einen 
dieser vier Keys nicht in einer \texttt{ifsetup} Zuweisung spezifiziert ist, werden 
alle untergeordneten Key-Deklarationen zu einem beliebigen Layout diesen Typs 
angewendet. Man betrachte folgenden Quelltext als Beispiel.

\begin{Verbatim}[frame=single,fontsize=\small,numbers=left]
  ...
 textpos={.2\slidewidth,.3\slideheight},
 ifsetup={portrait,screen},
 textpos={.3\slidewidth,.2\slideheight}
 ...
 ifsetup=landscape,
 ...
 ifsetup,
 ...
\end{Verbatim}

Angenommen im Beispiel wäre keine \texttt{ifsetup} Deklaration vor der ersten
\texttt{textpos} Deklaration, würde der Befehl \texttt{textpos} auf jedes mögliche 
Setup
angewendet werden. Für das screen-Format in Hochformat (portrait) jedoch,
wird die nächste \texttt{textpos} Deklaration befolgt. Das heißt alle Befehle die bis
zum nächsten \texttt{ifsetup} (Zeile 6) ausgeführt werden, werden für Hochformat
(portrait) umgesetzt. Alle Keys nach diesem \texttt{ifsetup} gelten für Querformat
(landscape), wobei \emph{paper, mode und display nicht spezifiziert sind}.
Wenn man nach einer Spezifikation von Keys zu den Optionen zurück möchte,
die auf alle Setups angewendet werden, deklariert man \texttt{ifsetup} ohne weitere
Parameter (Zeile 8). Alle nachfolgenden Deklarationen werden nun wieder für
jedes mögliche Setup befolgt.

Der folgende Befehl ist eine unabhängige Anwendung des zuvor beschriebenen
Mechanismus. Er erlaubt die Kontrolle des Setups außerhalb des Arguments 
<\texttt{options}> des \slash{pddefinetemplate} Befehls.
\begin{Verbatim}[frame=single,fontsize=\small,fillcolor=\color{yellow}]
 \pdifsetup{<desired>}{<true>}{<false>}
\end{Verbatim}

\marginpar{\texttt{$\backslash$pdifsetup}}
Dieser Makro (Programmcodeteil) entscheidet <\texttt{true}> wenn die
Programmeinrichtung des Benutzers, sei Setup, mit dem verlangten
<\texttt{desired}>
Setup überein stimmt. In allen anderen Fällen gilt <\texttt{false}>. Wurde
beispielsweise landscape (Querformat) gewählt, dann gilt bei
\begin{Verbatim}[frame=single,fontsize=\small]
 \pdifsetup{landscape}{yes}{no}
\end{Verbatim}
"'yes"'. Wurde jedoch statt dessen portrait (Hochformat) gewählt, würde nein,
bzw. "`no"' gelten.

Dieser Makro kann genutzt werden um die Setupanforderungen zu überprüfen und 
um zum Beispiel ein Error zu generieren, wenn eine bestimmte 
Programmeinrichtung vom Stil nicht unterstützt wird. \pf{Powerdot} bietet 
eine vorverfasste Error-Nachricht, welche in den ersten Zeilen des 
Quelltextes der Stildatei verwendet werden kann.
\begin{Verbatim}[frame=single,fontsize=\small,fillcolor=\color{yellow}]
 \pd@noportrait
\end{Verbatim}

\marginpar{\texttt{$\backslash$pd@noportrait}}
Dieser Makro generiert ein Error wenn der Benutzer Hochformat abfragt. An 
dieser Stelle sei darauf hingewiesen, dass ein eventueller Handout-Modus nur 
in Hochformat möglich ist. Dieser Makro berücksichtigt dies jedoch und 
generiert bei Abfrage eines Handouts kein Error.

\subsection{Hauptkomponenten}\label{sec:maincomps}

Die <\texttt{options}> kontrollieren verschiedene Key-Komponenten einer Folie.
Jede Komponente hat verschiedene Einzelteile mit jew. Eigenschaften. Ein Key
der für das <\texttt{options}> Argument verwendet werden kann ist die Bezeichnung
der Komponente. Zum Beispiel benannt nach ihrer Eigenschaft, die kontrolliert
werden soll.

Die Komponenten\texttt{itle}, \texttt{text}, \texttt{toc}, \texttt{stoc}
und \texttt{ntoc} haben die Einzelteile
bzw. Eigenschaften \texttt{hook}, \texttt{pos}, \texttt{width} und
\texttt{font}. Zudem hat die Komponente
\texttt{text} die Eigenschaft \texttt{height}, die Komponenten \texttt{lf}
und \texttt{rf} die 
Eigenschaften \texttt{hook}, \texttt{pos}, \texttt{temp} und \texttt{font}. 
Eigenschaften 
für gültige
Keys sind daher \texttt{titlefont}, \texttt{tocpos} und \texttt{lftemp}. Diese 
Komponenten 
und ihre Einzelteile sollen nun erläutert werden.

Vom <\texttt{options}> Argument in \slash{pddefinetemplate} kontrollierte Komponenten 
sind:
\begin{description}
\marginpar{\textsl{option}\\\texttt{title-}}
\item[\fcolorbox{black}{gray!30}{title-}]
Betrifft den Titel der Folie.

\marginpar{\textsl{option}\\\texttt{text-}} 
\item[\fcolorbox{black}{gray!30}{text-}]
Betrifft die Box für den Haupttext auf der Folie.

\marginpar{\textsl{option}\\\texttt{toc-}} 
\item[\fcolorbox{black}{gray!30}{toc-}]

Betrifft das vollständige Inhaltsverzeichnis 
einschließlich der Unterkapitel und Folien.

\marginpar{\textsl{option}\\\texttt{stoc-}}
\item[\fcolorbox{black}{gray!30}{stoc-}]

Dieses Inhaltsverzeichnis umfasst nur die 
Unterkapitel "`sections"' (siehe auch \texttt{ntoc}).

\marginpar{\textsl{option}\\\texttt{ntoc-}} 
\item[\fcolorbox{black}{gray!30}{ntoc-}]

Dies ist ein Inhaltsverzeichnis, welches nur 
die Einträge des aktuellen
Kapitels anzeigt. Zusammen mit \texttt{stoc} kann es als ein gesplittetes
Inhaltsverzeichnis verwendet werden. Beispielsweise mit Überblick über die
Kapitel im einen und mit den Unterpunkten des jeweiligen Kapitels im anderen
Verzeichnis. In einer einzelnen Folienvorlage ist \texttt{toc} nützlicher, eine
Kombination von \texttt{stoc} und \texttt{ntoc} oder überhaupt kein Inhaltsverzeichnis.

\marginpar{\textsl{option}\\\texttt{lf-}} 
\item[\fcolorbox{black}{gray!30}{lf-}]

Betrifft die linke Fußnote bzw. Eingabefeld.

\marginpar{\textsl{option}\\\texttt{rf-}} 
\item[\fcolorbox{black}{gray!30}{rf-}]

Betrifft die rechte Fußnote bzw. Eingabefeld.
\end{description}

Angemerkt sei, dass alle Komponenten die bisher beschrieben wurden, mit
\slash{rput} von \pf{pstricks} \cite{PSTricksWeb,PSTricks} eingebracht werden.
Die Dokumentation zu \pf{pstricks} gibt mehr Informationen zu diesem Befehl.
Zudem sollte zur Kenntnis genommen werden, dass alle Komponenten (außer
\texttt{lf} und \texttt{rf}) ihren Inhalt in eine \texttt{minipage} Umgebung setzen.

Nun werden alle Einzelteile bzw. Eigenschaften der zuvor genannten 
Komponenten mit ihrer Bedeutung gelistet. Es ist daran zu denken, 
dass Keys aus der Kombination einer Komponente und einer Eigenschaft bestehen.

\begin{description}
\marginpar{\textsl{option}\\\texttt{-hook}}
\item[\fcolorbox{black}{gray!30}{-hook}]

Diese Option definiert den \slash{rput} Anker, welcher bei der Positionierung
eines Item verwendet wird. Dieser kann sein: \texttt{tl}, \texttt{t},
\texttt{tr}, \texttt{r}, \texttt{Br}, \texttt{br},
\texttt{b}, \texttt{bl}, \texttt{Bl}, \texttt{l}, \texttt{B} und \texttt{c}. Die 
\pf{pstricks} Dokumentation bietet hier zu weitere Informationen.

\marginpar{\textsl{option}\\\texttt{-pos}}
\item[\fcolorbox{black}{gray!30}{-pos}]

Dies definiert die Position des \texttt{hook}. Die linke, untere Ecke des "`Papiers"'
entspricht dem Punkt |{0,0}| und die rechte, obere Ecke dem Punkt
\texttt{\{\slash{slidewidth},\slash{slideheight}\}}. Soll die Box für den Haupttext 
20\% vom linken
Rand und 30\% vom oberen Rand entfernt sein, ergibt sich folgender Key:
\begin{Verbatim}[frame=single,fontsize=\small]
 textpos={.2\slidewidth,.7\slideheight}
\end{Verbatim}
Wenn die Position einer Komponente nicht spezifiziert wurde, wird die
Komponente nicht auf der Folie platziert. Dies ermöglicht es Folien zu 
entwerfen, die in einem nüchtern Stil ohne Fußnoten oder Inhaltsverzeichnis 
gehalten sind.

\marginpar{\textsl{option}\\\texttt{-width}}
\item[\fcolorbox{black}{gray!30}{-width}]

Bezieht sich auf die Breite der Komponente. Alle Komponenten, die \pf{powerdot} positioniert,
werden in eine \texttt{minipage} Umgebung gesetzt. Die Eigenschaft
\texttt{width} bestimmt
die Breite des Einsatzfeldes \texttt{minipage}. Zum Beispiel:
\begin{Verbatim}[frame=single,fontsize=\small]
 textwidth=.7\slidewidth
\end{Verbatim}
Für \texttt{lf} und \texttt{rf} Komponenten gibt es diese Eigenschaft nicht.

\marginpar{\textsl{option}\\\texttt{-height}}
\item[\fcolorbox{black}{gray!30}{-height}]

Diese Option ist nur für die Komponente \texttt{text} verfügbar. Oder anders: Für
diese Eigenschaft gibt es nur eine Verwendung, nämlich den Key
\texttt{textheight}.
Damit kann die Höhe der \texttt{minipage} spezifiziert werden, welche als
Haupttextfenster dient. Diese Höhe dient jedoch nur für die vertikale
Angleichung von Material auf der Folie, wie beispielsweise Fußnoten. Nicht
aber für die Länge oder den automatischen Folienumsprung durch \pf{powerdot}
bei vollen Textfenstern. Der vorbestimmte Wert ist \slash{slideheight}.

\marginpar{\textsl{option}\\\texttt{-font}}
\item[\fcolorbox{black}{gray!30}{-font}]

Dies wird im Quelltext kurz vor den betroffenen Text gesetzt, dessen 
Schriftsatz definiert werden soll. Mit dem Befehl können so Abweichungen 
des Textes in Schriftart und Farbe vorgenommen werden. Dies kann als 
Schriftsatzdeklaration umgesetzt werden, wie \slash{large}slash{\bfseries}, aber auch
mit anderen Inhalten wie \slash{color}\{\texttt{red}\} oder
\slash{raggedright}.

\marginpar{\textsl{option}\\\texttt{-temp}}
\item[\fcolorbox{black}{gray!30}{-temp}]

Diese Eigenschaft ist nur für Fußnoten (\texttt{lf} und \texttt{rf}) verfügbar und 
kann 
verwendet werden um deren Schablone in der Vorlage zu ändern. Beispielsweise
kann Inhalt durch den Benutzer hinzugefügt werden. Die Standard-Deklaration
bei \pf{powerdot} ist folgende:
\begin{Verbatim}[frame=single,fontsize=\small]
 rftemp=\pd@@rf\ifx\pd@@rf\@empty
   \else\ifx\theslide\@empty\else~--~\fi\fi\theslide
\end{Verbatim}
\slash{pd@@rf} wird hier den Inhalt des rechten Eingabefeldes umfassen, definiert 
durch den Benutzer, entgegen dem \slash{pdsetup} Befehl. Ähnlich beinhaltet 
\slash{pd@@rf} den Inhalt des linken Eingabefeldes. Der obige Befehl kontrolliert
ob das Eingabefeld und \slash{theslide} Inhalte haben. Wenn dem so ist, wird
\texttt{$\sim$--$\sim$} eingesetzt, um sie zu unterscheiden.

\marginpar{\textsl{option}\\\texttt{-orient}}
\item[\fcolorbox{black}{gray!30}{-orient}]

Diese Eigenschaft ist nur für die Komponenten \texttt{toc}, \texttt{stoc}
und \texttt{ntoc}
verfügbar. Es stehen die Alternativen \texttt{h} oder \texttt{v} zur Auswahl, um die
horizontale oder vertikale Ausrichtung des Inhaltsverzeichnisses zu
bestimmen. Die Voreinstellung entspricht \texttt{v}. Bezüglich dem Anlegen des
Inhaltsverzeichnisses liefert Kapitel~\ref{sec:slidetoc} weitere Informationen.
\end{description}

\subsection{Das Folieninhaltsverzeichnis }\label{sec:slidetoc} 

Das kleine Inhaltsverzeichnis auf den Folien kann durch vier Makros und mehrere 
Optionen gesteuert werden.

\marginpar{\texttt{$\backslash$pd@tocslide}}

\marginpar{\texttt{$\backslash$pd@tocsection}}

Diese Makros nehmen ein Argument an. Beim Erstellen des Inhaltsverzeichnisses 
durchläuft 
\pf{powerdot} den Inhalt mittels  \slash{pd@tocslide} oder
\slash{pd@tocsection}, je nach dem welcher Eingabetyp gerade erstellt wird.
Sie können beispielsweise 
\begin{Verbatim}[frame=single,fontsize=\small]
 \def\pd@tocslide#1{$\bullet$\ #1}
 \def\pd@tocsection#1{#1}
\end{Verbatim}
eingeben, wodurch alle normalen Eingaben (nicht die Abschnitte) mit einem 
"`Bullet"' präfigiert werden. Diese beiden Makros sind standardmäßig so definiert, 
dass sie sich genau auf ihre jeweiligen Argumente übertragen.

\marginpar{\texttt{$\backslash$pd@tocsisplay}}

\marginpar{\texttt{$\backslash$pd@tochighlight}}

Diese zwei Makros nehmen ebenfalls ein Argument. Nachdem die Eingabe mit dem 
Befehl  \slash{pd@tocslide} oder \slash{pd@tocsection} bearbeitet wurde, setzt 
\pf{power\-dot}
das Erstellen der Eingabe mit Durchlaufen von \slash{pd@tocdisplay} fort,
wenn die Eingabe nur angezeigt werden muss oder von \slash{pd@tochighlight},
wenn die Eingabe hervorgehoben werden muss. 
Diese Makros sind überwiegend beteiligt und betreuen das Erstellen des Inhalts in 
angemessener Schriftart und Farbe in einer \texttt{minipage}. Des Weiteren setzt
\slash{pd@tochighlight} eine Box um die Einheit. 

Beachten Sie, dass beispielsweise beide, zum einem die Eingabe des gesonderten
Inhaltsverzeichnis, genauso wie das Inhaltsverzeichnis als Ganzes in einer 
\texttt{minipage} Umgebung von diesen Makros gesetzt sind, in dem Fall, dass das 
Inhaltsverzeichnis vertikal angelegt ist. Die \texttt{-width}-Bestandteile 
determinieren dann die Breite des Inhaltsverzeichnisses und zusammen mit 
\texttt{tocsecindent} und
\texttt{tocslideindent} (siehe unten) die Breite der individuellen Eingaben. Wenn 
es horizontal ist, sind nur die gesonderten Eingaben und nicht das 
Inhaltsverzeichnis an sich in der \texttt{minipage}.
 Die \texttt{-width}-Bestandteile
determinieren nur die Breite der individuellen Eingaben (zusammen mit 
\texttt{tocsecindent} und \texttt{tocslideindent}).

Mehrere Aspekte des Prozesses des Anlegens des Inhaltsverzeichnisses können durch 
Schlüssel gesteuert werden, die im Befehl 
\slash{pddefinetemplate} abrufbar sind, die dann unten beschrieben werden. Falls 
diese Schlüssel nicht genug Spielraum bereitstellen um tun zu können, was Sie 
möchten, müssen Sie vielleicht einen Blick auf die zwei Makros in der Quelle werfen 
und sich entscheiden diese in ihrem Stil neu zu schreiben, bis sie zu ihren 
Bedürfnissen passen. Ein Beispiel finden Sie im 
\pf{fyma} Stil.

\begin{description}
\marginpar{\textsl{option}\\ \texttt{tocfrsep}}
\item[\fcolorbox{black}{gray!30}{tocfrsep}]

Diese Länge ist der Abstand zwischen der Box, die den Inhalt umgibt, die von der
\texttt{minipage} kreiert wurde und der hervorgehobenen Rahmenbox, kreiert durch 
\slash{pd@tochighlight}. Voreingestellt: \texttt{0.5mm}.

\marginpar{\textsl{option}\\\texttt{tocsecsep}}
\item[\fcolorbox{black}{gray!30}{tocsecsep}]

Dieser Abstand ist vor einem Abschnitt eingefügt (ausgenommen es ist das erste 
Element in dem Inhaltsverzeichnis). Voreingestellt: \texttt{2ex}.
Beachten Sie, dass wenn die Orientierung des Inhaltsverzeichnisses auf vertikal 
gesetzt ist, die Länge eine vertikale Auslassung kreiert, anderenfalls kreiert es
eine horizontale Auslassung. 

\marginpar{\textsl{option}\\\texttt{tocslidesep}}
\item[\fcolorbox{black}{gray!30}{tocslidesep}]

Der Abstand ist vor anderen Eingaben eingefügt (ausgenommen es ist das erste Element
in dem Inhaltsverzeichnis). Voreingestellt: \texttt{0ex}. Wie
\texttt{tocsecsep} ist der Effekt der Länge abhängig von der Ausrichtung des
Inhaltsverzeichnisses.

\marginpar{\textsl{option}\\\texttt{tocsecindent}}
\item[\fcolorbox{black}{gray!30}{tocsecindent}]

Ein horizontales Leerfeld links von der Abschnitteingabe. Voreingestellt:
\texttt{0pt}.

\marginpar{\textsl{option}\\\texttt{tocslideindent}}
\item[\fcolorbox{black}{gray!30}{tocslideindent}]

Ein horizontales Leerfeld links von der Folieneingabe. Die horizontale Auslassung 
wird nicht links von der Folieneingabe eingefügt, die vor dem ersten Abschnitt 
erscheint. Voreingestellt: \texttt{0pt}.

\marginpar{\textsl{option}\\\texttt{tocsecm}}
\item[\fcolorbox{black}{gray!30}{tocsecm}]

Dies wird vor dem Schrift setzen eines Abschnitts eingefügt. Es kann zum Markieren 
eines Abschnitts verwendet werden, zum Beispiel mit einer Linie im  \pf{default}
Stil. Voreingestellt ist: empty.

\marginpar{\textsl{option}\\\texttt{toccolor}}
\item[\fcolorbox{black}{gray!30}{toccolor}]

Dies ist die Schriftfarbe, benutzt für nicht hervorgehobene Elemente 
im Inhaltsverzeichnis. Voreingestellt: \texttt{black}.

\marginpar{\textsl{option}\\\texttt{tochltcolor}}
\item[\fcolorbox{black}{gray!30}{tochltcolor}]

Dies ist die Schriftfarbe, benutzt für hervorgehobene Elemente im
Inhaltsverzeichnis. Voreingestellt: \texttt{white}.

\marginpar{\textsl{option}\\\texttt{tochlcolor}}
\item[\fcolorbox{black}{gray!30}{tochlcolor}]

Die ist die Farbe, benutzt für den Rahmen hinter den hervorgehobenen Elementen. 
Voreingestellt: \texttt{black}.
\end{description}

\subsection{Sonstige Optionen}\label{sec:miscoptions}

Es gibt einige Optionen, die aus den Rahmen der vorherigen Abschnitte fallen. Diese 
werden im Folgenden besprochen.

\begin{description}

\marginpar{\textsl{option}\\\texttt{iacolor}}
\item[\fcolorbox{black}{gray!30}{iacolor}]

Die Option \texttt{iacolor} können Sie benutzen, um die Farbe, die für inaktive 
Symbole genutzt wird, zu spezifizieren. Sie wird beispielsweise durch
\slash{onslide}, \slash{pause}
(siehe Abschnitt~\ref{sec:overlays}) und \slash{tableofcontents} (siehe
Abschnitt~\ref{sec:tableofcontents}). Wenn \pf{xcolor} von
\pf{powerdot} verwendet wird, können Sie hierbei spezielle Darstellungsarten wählen, 
wie 
\begin{Verbatim}[frame=single,fontsize=\small]
 iacolor=black!20
\end{Verbatim}
Der voreingestellte Wert für diesen Schlüssel ist  \texttt{lightgray}.

Die folgenden Optionen steuern die Digitaluhr (siehe
Abschnitt~\ref{sec:classopts}). Die Uhr ist ein gestaltbares Textfeld mit einem 
dynamischen Inhalt, was durch javaskript über \pf{hyperref} Textfelder gesteuert 
wird. 
Einige Optionen arbeiten ähnlich, wie zum Beispiel für den Titelbaustein, aber es 
gibt ebenfalls spezielle Optionen. 

\marginpar{\textsl{option}\\\texttt{clockhook//clockpos}}
\item[\fcolorbox{black}{gray!30}{clockhook} \quad
\fcolorbox{black}{gray!30}{clockpos}]

Diese arbeiten auf die gleiche Weise wie die  \texttt{-hook} und
\texttt{-pos} Bestandteile, 
die in Abschnitt~\ref{sec:maincomps}diskutiert wurden. Der voreingestellte Wert der 
\texttt{clockhook} ist \texttt{tr}.

\marginpar{\textsl{option}\\\texttt{clockwidth\\clockheight}}
\item[\fcolorbox{black}{gray!30}{clockwidth} \quad
\fcolorbox{black}{gray!30}{clockheight}]

Diese steuern die Breite und die Höhe des Textfeldes, das die Uhr beinhaltet. Die 
voreingestellten Werte kommen von \pf{hyperref} und haben ein Maß von
\texttt{3cm} beziehungsweise von \slash{baselineskip}.

\marginpar{\textsl{option}\\\texttt{clockcharsize}}
\item[\fcolorbox{black}{gray!30}{clockcharsize}]

Die Größe der Ziffern auf der Uhr. Voreingestellt ist \texttt{14pt}.

\marginpar{\textsl{option}\\\texttt{clockalign}}
\item[\fcolorbox{black}{gray!30}{clockalign}]

Die Ausrichtung der Uhr im Textfeld. \texttt{0} ist links ausgerichtet,
\texttt{1} ist zentriert und  \texttt{2} ist rechts ausgerichtet. Voreingestellt 
ist \texttt{2}.

\marginpar{\textsl{option}\\\texttt{clockcolor}}
\item[\fcolorbox{black}{gray!30}{clockcolor}]

Dies legt die Schriftfarbe der Uhr fest. Der Wert muss eine bestimmte Farbe sein. 
Der voreingestellte Wert ist \texttt{black}.
\end{description}

\subsection{Voreingestellte Templates}

Unten werden die voreingestellten Einstellungen der Schlüssel beschrieben. Diese 
können benutzt werden, wenn Sie keinen anderen Input für diese Schlüssel in eine 
bestimmte Schablone liefern. Wenn die voreingestellten Werte Ihre Bedürfnisse
erfüllen, brauchen Sie diese nicht noch einmal in Ihrem eigenen Stil spezifizieren.
\begin{Verbatim}[frame=single,fontsize=\small]
 titlehook=Bl,titlepos=,titlewidth=\slidewidth,
 titlefont=\raggedright,texthook=tl,textpos=,
 textwidth=\slidewidth,textfont=\raggedright,
 textheight=\slideheight,
 tochook=tl,tocpos=,tocwidth=.2\slidewidth,
 tocfont=\tiny\raggedright,
 stochook=tl,stocpos=,stocwidth=.2\slidewidth,
 stocfont=\tiny\raggedright,
 ntochook=tl,ntocpos=,ntocwidth=.2\slidewidth,
 ntocfont=\tiny\raggedright,
 tocorient=v,stocorient=v,ntocorient=v,
 tocfrsep=.5mm,tocsecsep=2ex,tocslidesep=0ex,
 tocsecm=,toctcolor=black,tochlcolor=black,tochltcolor=white,
 tocsecindent=0pt,tocslideindent=0pt,
 lfhook=Bl,lfpos=,lffont=\scriptsize,lftemp=\pd@@lf,
 rfhook=Br,rfpos=,rffont=\scriptsize,rftemp=\pd@@rf\ifx\pd@@rf\@empty
   \else\ifx\theslide\@empty\else~--~\fi\fi\theslide,
 iacolor=lightgray,
 clockhook=tr,clockpos=,clockwidth=3cm,clockheight=\baselineskip,
 clockcharsize=14pt,clockalign=2,clockcolor=black
\end{Verbatim}

\subsection{Der Hintergrund}\label{sec:defbg}

Nur ein Argument von dem Makro \slash{pddefinetemplate} ist noch unbesprochen. Die ist 
das  <\textsl{Befehls}> (<\textsl{commands}>) Argument. 
Dieses Argument kann jeden Code einbinden, den Sie ausführen möchten,  
\textit{nachdem} die Optionen gesetzt wurden und \textit{bevor} die Folienbausteine 
wie der Folientitel, Haupttext und Fußnoten erstellt wurden. Dieses Argument ist
konstruiert, um Deklarationen einzubinden, die den Hintergrund einer Folie erstellen 
und zum Beispiel  \pf{pstricks} benutzen. Aber es kann auch andere Befehle enthalten, 
die Sie zum Erstellen Ihrer Schablone brauchen. 

Es ist wichtig festzuhalten, dass diese Befehle nicht unbedingt \TeX\
Material kreieren, das Ihre Konstruktion der Folie zerstören könnte. Falls Sie das
Wort "`Hallo"' in der unteren linke Ecke der Folie platzieren möchten, 
schreiben Sie nicht "`Hallo"', legen Sie aber die Breite, Höhe und Tiefe gleich der 
Null, zum Beispiel mit der Benutzung von  \pf{pstricks}' \slash{rput}.
\begin{Verbatim}[frame=single,fontsize=\small]
 \rput[bl](0,0){Hello}
\end{Verbatim}

\subsection{Titelfolie, Titel und Abschnitte}\label{sec:specialtemps}

Wie zuvor erwähnt, muss der Stil, in dem Sie schreiben, definiert werden und damit 
zuletzt die Schablonen \texttt{slide} und \texttt{titleslide}. Letzteres behandelt 
einige der Schlüssel in einer speziellen Weise. 

Die Titelfolie (erstellt durch \slash{maketitle}) setzt den Titel mit den Autor/en 
und das Datum in die Haupttextbox. Dies bedeutet, dass Sie eine Position für die
Haupttextbox  (\texttt{textpos})liefern müssen. Die Haupttextschriftgröße (zusammen 
mit den Erklärungen in dem 
\texttt{textfont} Schlüssel) wird für Autor/en und Datum benutzt. Die Erklärung wird 
aber in \texttt{titlefont} für den Titel der Präsentation gebraucht. Dadurch formen 
Titel und Autor/en einen zusammenhängenden Block und es wird sichergestellt, dass 
lange Titel Autor/en nach unten verschieben, anstatt ihn zu überschreiben

\marginpar{\texttt{$\backslash$pd@slidetitle}}
Das Makro \slash{pd@slidetitle} wird verwendet, um den Folientitel auf die Folien zu setzen. 
Dieses Makro ist zum Beispiel mit \slash{pd@tocslide} vergleichbar. Es nimmt ein Argument an, 
das den Folientitel mit der richtigen Schriftart und Formation hat. Standardmäßig passt dieses
Makro den Inhalt für das Schriftsetzen an, aber Sie können dieses Makro umdefinieren und somit 
seinem früheren Input erstellen um die Schrift zu setzen. Ein Beispiel ist der 
\pf{fyma} Stil, der den Titel unterstreicht, nachdem er in eine 
|minipage|gesetzt wurde und der den Mehrfachlinientitel unterstützt. 

\marginpar{\texttt{$\backslash$pd@sectiontitle}}
Diese Makros haben Ähnlichkeit zu \... und setzen den Titel auf die Titelfolie bzw. den Titel 
auf die Abschnittsfolien. Standardmäßig bestehen diese ebenso aus ihrem Argument (was der 
Titel der Präsentation oder der Titel eines Abschnitts ist). Aber dies kann auch umdefiniert
werden um so den früheren Input zu erstellen und so die Schrift zu setzen, wie bei
\slash{pd@slidetitle}.


\marginpar{\textsl{option}\\\texttt{sectemp}}

\marginpar{\textsl{option}\\\texttt{widesectemp}}

Der Befehl \slash{section} benutzt (standardmäßig) die  \texttt{slide} Umgebung und setzt den 
Abschnittstitel in die Titelbox mit der Schriftart \texttt{titlefont}. Wenn Sie zum Beispiel 
diesen Standard ändern möchten und die  \texttt{slide} Umgebung, die
\texttt{sectionslide} Umgebung oder eine beliebige eigens kreierte Abschnittschablone auch für 
die Abschnitte nutzen möchten, ändern Sie die voreingestellte Schablone in Ihrem Stil mit 
\begin{Verbatim}[frame=single,fontsize=\small]
 \setkeys[pd]{section}{sectemp=sectionslide}
\end{Verbatim}
Die bedeutet, dass bei der Forderung des Benutzers nach
\texttt{template=slide} in dem Befehl 
\slash{section} die \texttt{sectionslide} Umgebung stillschweigend benutzt wird. Um 
Überraschungen zu vermeiden sollte \texttt{sectionslide} vorzugsweise auf
der \texttt{slide} Umgebung basieren. 

Eine ähnliche Option ist verfügbar in dem Fall, dass der Benutzer \\
\texttt{template=wideslide} fordert. Beispielsweise die Folgende:
\begin{Verbatim}[frame=single,fontsize=\small]
 \setkeys[pd]{section}{widesectemp=sectionwideslide}
\end{Verbatim}
Jedes Mal wenn der Benutzer ein \texttt{wideslide} anfordert, gebraucht für 
\slash{section}, wird stattdessen die  \texttt{sectionwideslide} Umgebung benutzt. Bei 
anderen Inputs für den Schablonenschlüssel erfolgt keine spezielle Bearbeitung. 

Beachten Sie, dass diese Schlüssel in den \texttt{section} Gruppenschlüsseln verfügbar sind 
und dass Sie diese nicht für den Befehl \slash{pddefinetemplate} verwenden können.

\subsection{Das Testen des Stils}\label{sec:styletest}

\pf{powerdot} hat eine Testdatei, die die meisten Stile testet. Die Testdatei kann 
angefertigt werden, indem \LaTeX\ über
\texttt{powerdot.dtx} läuft. Diese generiert\\
\texttt{powerdot-styletest.tex} , was Ihnen hilft die Eingaben zu kontrollieren. 
Wenden Sie sich an uns, wenn Sie Ihren Stil \pf{powerdot} beisteuern möchten. Siehe 
ebenfalls Abschnitt~\ref{sec:questions}.

\section{Die Benutzung von \LyX\ für Präsentationen}\label{sec:lyx}

\LyX\ \cite{LyXWeb} ist ein WYSIWYM (What You See Is What You Mean)
Dokumentprozessor basierend auf \LaTeX. Es unterstützt  \LaTeX\
Standardklassen, braucht aber spezielle Dateien, genannt Layout-Dateien, um nicht 
standardisierte Klassen, wie \pf{powerdot} zu unterstützen.

Um  \LyX\ für \pf{powerdot} Präsentationen zu verwenden, kopieren Sie die 
Layout-Datei  \texttt{powerdot.layout} in das \LyX\ Layout-Datenverzeichnis. Diese Datei finden Sie 
in Ihrem  \LaTeX\ Installationsverzeichnis unter dem Pfad:
\url{texmf/doc/latex/powerdot}. Falls Sie ihn hier nicht finden, können Sie ihn auch von CTAN 
herunterladen \url{CTAN:/macros/latex/contrib/powerdot}. Sobald dies getan ist, 
rekonfigurieren Sie \LyX\ (\texttt{Edit}$\rhd$\texttt{Reconfigure} und
starten Sie \LyX\ danach neu). Jetzt können Sie die  \pf{powerdot}
Dokumentenklasse wie eine beliebige andere unterstützte Klasse benutzen. Wählen Sie 
\texttt{Layout}$\rhd$\texttt{Document} und wählen Sie \texttt{powerdot
presentation} als Dokumentenklasse aus. Für mehr Informationen schauen Sie in die \LyX\
Dokumentation, die unter dem \texttt{Hilfemenü} abrufbar ist.  

\subsection{Wie das Layout benutzt wird}

Das \pf{powerdot} \LyX\ Layout bietet einige Umgebungen\footnote{Nicht
mit den \LaTeX\ Umgebungen verwechseln.}, die in \LyX\ verwendet werden können.
Manche dieser Umgebungen  (beispielsweise \texttt{Title} oder
\texttt{Itemize}) 
 sind normal nutzbar seit sie ebenfalls in Standarddokumentenklassen wie \pf{article} 
 existieren. Mehr Informationen über die Standardumgebungen sind in der \LyX\ Dokumentation 
 zu finden. 

Dieser Abschnitt will Ihnen erklären, wie die speziellen \pf{powerdot} Umgebungen
\texttt{Slide}, \texttt{WideSlide}, \texttt{EmptySlide} und \texttt{Note} benutzt werden. 
Diese Umgebungen entsprechen der  \pf{powerdot} Umgebung \texttt{slide},
\texttt{emptyslide}, \texttt{wideslide} und \texttt{note}.

Begonnen wird mit einem einfachen Beispiel. Der folgende  \LaTeX\ Code
\begin{Verbatim}[frame=single,fontsize=\small]
 \begin{slide}{Slide title}
   Slide content.
 \end{slide}
\end{Verbatim}
ist bei der Benutzung der folgenden \LyX\ Umgebungen erhältlich. Die rechte Spalte 
repräsentiert den Text eingegeben in das \LyX\ Fenster und die linke Spalte repräsentiert die 
Umgebung angewandt auf diesen Text.
\begin{Verbatim}[frame=single,fontsize=\small]
 Slide         Slide title
 Standard      Slide content.
 EndSlide
\end{Verbatim}
Einige Anmerkungen, bezogen auf dieses Beispiel.
\begin{itemize}
\item Sie können dieses Umgebungsmenü (unter dem Menübalken, oberere linke Ecke) verwenden, um 
die Umgebung angewandt auf diesen Text zu wechseln.
\item Der Folientitel sollte in der Zeile der  \texttt{Slide}
Umgebung geschrieben werden.
\item \texttt{EndSlide} beendet die Folie und lässt die Linie unbeschrieben.
\end{itemize}

In dem \LyX\ Fenster liegt die \texttt{Slide} Umgebung (der Folientitel) in magenta aus, 
der  \texttt{WideSlide} Stil in grün, der 
\texttt{EmptySlide} Stil in cyan und der \texttt{Note} Stil in rot und daher sind diese 
leicht identifizierbar. 

Hier ist ein anderes Beispiel.
\begin{Verbatim}[frame=single,fontsize=\small]
 \begin{slide}{First slide title}
   The first slide.
 \end{slide}
 \begin{note}{First note title}
   The first note, concerning slide 1.
 \end{note}
 \begin{slide}{Second slide title}
   The second slide.
 \end{slide}
\end{Verbatim}
Das lässt sich im \LyX\ folgendermaßen erstellen.
\begin{Verbatim}[frame=single,fontsize=\small]
 Slide         First slide title
 Standard      The first slide.
 Note          First note title
 Standard      The first note, concerning slide 1.
 Slide         Second slide title
 Standard      The second slide.
 EndSlide
\end{Verbatim}
Dieses Beispiel demonstriert, dass es oft genügt den 
\texttt{EndSlide} Stil nach der letzten Folie oder Notiz einzufügen. Nur wenn Sie bestimmtes 
Material nicht als Teil einer Folie wollen, müssen Sie die vorausgehende Folie manuell mit 
dem  \texttt{EndSlide} Stil beenden. Beispiel:
\begin{Verbatim}[frame=single,fontsize=\small]
 Slide         First slide title
 Standard      The first slide.
 EndSlide
 [ERT box with some material]
 Slide         Second slide title
 ...
\end{Verbatim}

Optionen können für Folienumgebungen mit der Benutzung von 
\texttt{Insert}$\rhd$\texttt{Short title} vor dem Folientitel übermittelt werden. Das folgende 
Beispiel gebraucht die  \texttt{direkte} Methode (siehe
Abschnitt~\ref{sec:verbatim}) im Kurztitel-Argument (begrenzt durch einen eckigen Bereich) um 
eine  \texttt{lstlisting} Umgebung (definiert vom
dem \pf{listings} Packet) binnen des Folieninhalts zu erstellen.
\begin{Verbatim}[frame=single,fontsize=\small]
 Slide         [method=direct]Example of LaTeX source code
 Standard      Here's the \HelloWorld command:
 [ERT box:
   \lstset{language=[LaTeX]TeX}
   \begin{lstlisting}
   \newcommand{\HelloWorld}{Hello World!}
   \end{lstlisting}
 ]
 EndSlide
\end{Verbatim}
Beachten Sie, dass Sie nicht verpflichtet sind eine \texttt{verbatim} Umgebung zu nutzen, 
um  \slash{HelloWord} in das \LyX\ Fenster zu schreiben, weil \LyX\
direkt einen wörtlichen Standard unterstützt.\footnote{\LyX\ übersetzt
spezielle Zeichen in ihren dazugehörigen \LaTeX\ Befehl. Zum
Beispiel das backslash Zeichen ist in 
\slash{textbackslash} übersetzt. Resultierend, ist die Schriftart nicht die gleiche wie in der
wahren wörtgetreuen Übersetzung und Sie könnten dies durch den 
\texttt{Layout}$\rhd$\texttt{Character} Dialog ändern.}. Folglich ist die Benutzung der Methoden 
der Folienaufbereitung  \texttt{direct} und \texttt{file} nicht notwendig, wenn Sie einen 
wörtlichen Standard gebrauchen, aber es ist notwendig, wenn Sie fortgeschrittene Dinge tun 
möchten, wie im obigen Beispiel.

\subsection{Unterstützung der Syntax}
Dieser Abschnitt listet Optionen, Befehle und Umgebungen auf, die durch das \LyX\ Interface 
direkt unterstützt werden, ohne eine ERT-Box zu benutzen (\TeX-mode).

Alle Optionsklassen (siehe Abschnitt~\ref{sec:classopts}) werden durch 
den  \texttt{Layout}$\rhd$\texttt{Document} Dialog unterstützt(\texttt{Layout} Ausschnitt).
Optionen für den Befehl \slash{pdsetup}(siehe Abschnitt~\ref{sec:setup})
sollten in dem Präambel  \texttt{Preamble} Ausschnitt des 
\texttt{Layout}$\rhd$\texttt{Document} Dialog spezifiziert werden. 

Tabelle drei \ref{tab:lyxcommands} listet die \pf{powerdot} Befehle auf, die
in \LyX unterstützt werden.
\begin{table}[htb]
\centering
\begin{tabular}{l|p{7.5cm}}
Befehl & Methode in \LyX\\\hline
\slash{title} & Benutzt \texttt{Title} Umgebung.\\
\slash{author} & Benutzt \texttt{Author} Umgebung.\\
\slash{date} & Benutzt \texttt{Date} Umgebung.\\
\slash{maketitle} & direkt von \LyX ausgeführt.\\
\slash{section} & Benutzt die \texttt{Section} Umgebung. Optionen für diesen
Befehl (siehe Abschnitt~\ref{sec:section}) können durch 
\texttt{Insert}$\rhd$\texttt{Short title} vor dem Abschnittstitel spezifiziert werden.\\
\slash{tableofcontents} & Sie benutzen \texttt{Insert}$\rhd$\texttt{Lists \&
TOC}$\rhd$\texttt{Table of contents} und brauchen eine ERT-Box, wenn sie ein optionales Argument 
benutzen möchten. Siehe unten.
\end{tabular}

\caption{Unterstützt \pf{powerdot} Befehle in \LyX}\label{tab:lyxcommands}
\end{table}

Table \ref{tab:lyxenvs} listet die \pf{powerdot}Umgebungen auf, die neben den vorher diskutierten
\texttt{slide}, \texttt{wideslide}, \texttt{note} und
\texttt{emptyslide} Umgebungen auch in  \LyX unterstützt werden.
\begin{table}[htb]
\centering
\begin{tabular}{l|p{8cm}}
Umgebung & Methode in \LyX\\\hline
\texttt{itemize} & Benutzt \texttt{Itemize} und \texttt{ItemizeType1}
Umgebungen. Letzteres wird eine Liste mit  \texttt{type=1} kreieren (siehe
Abschnitt~\ref{sec:lists}).\\
\texttt{enumerate} & Benutzt \texttt{Enumerate} und
\texttt{EnumerateType1} Umgebungen.\\
\texttt{thebibliography} & Benutzt \texttt{Bibliography} Umgebung.
\end{tabular}
\caption{Unterstützt \pf{powerdot} Umgebungen in \LyX}\label{tab:lyxenvs}
\end{table}
Table \ref{tab:lyxERT} listet Befehle auf, die nur bei Benutzung der ERT-Box(durch 
\texttt{Insert}$\unrhd$\texttt{TeX})durchgeführt werden können.
\begin{table}[ht]
\centering
\begin{tabular}{l|p{8cm}}
Befehl & Methode in \LyX\\\hline
\slash{and} & innerhalb der \texttt{Author} Umgebung.\\
\slash{pause} & \\
\slash{item} & eine ERT-Box ist nur erforderlich für das optionale Argument, nicht obligatorisch für 
Overlay-Spezifikationen.\\
\slash{onslide} & Und die Versionen \slash{onslide+} und \slash{onslide*}.\\
\slash{twocolumn} & \\
\slash{tableofcontents} & nur bei der Benutzung eines optionalen Arguments.
\end{tabular}
\caption{\pf{powerdot} Befehle, die eine ERT-Box in  \LyX benötigen}\label{tab:lyxERT}
\end{table}
Beachten Sie, dass Sie die Ablage gebrauchen dürfen, um häufige Befehle, wie 
\slash{pause} zu wiederholen. Schließlich listet Tabelle \ref{tab:lyxadd} zusätzliche Befehle und Umgebungen 
auf, die vom Layout unterstützt werden.
\begin{table}[htb]
\centering
\begin{tabular}{l|p{8cm}}
Env./Befehl & Methode in \LyX\\\hline
\texttt{quote} & Benutzt \texttt{Quote} Umgebung.\\
\texttt{quotation} & Benutzt \texttt{Quotation} Umgebung.\\
\texttt{verse} & Benutzt \texttt{Verse} Umgebung.\\
\slash{caption} & Benutzt \texttt{Caption} Umgebung innerhalb einer Standardpufferumgebung.
\end{tabular}
\caption{zusätzliche Umgebungen für \LyX}\label{tab:lyxadd}
\end{table}

\subsection{Programmübersetzung mit \LyX}
Zuerst stellen Sie sicher, dass Sie auch 
Abschnitt~\ref{sec:compiling} gelesen haben. Dann, um ein einwandfreies 
PostScript-Dokument oder eine PDF-Datei zu bekommen, müssen Sie Ihre  \LyX\ Dokumentbestandteile, 
abhängig davon 
welches Papier und welche Ausrichtung Sie bevorzugen, anpassen und auswählen. 
Wenn das \LyX\ geöffnet ist, gehen Sie zu \texttt{Layout}$\rhd$\texttt{Document} Dialog. In dem 
\texttt{Layout} Ausschnitt setzen Sie die \texttt{nopsheader},
\texttt{orient} und \texttt{paper} Schlüssel als Klassenoptionen (siehe
Abschnitt~\ref{sec:classopts} für eine Beschreibung). Gehen Sie dann zum 
\texttt{Paper} Ausschnitt und wählen Sie die dementsprechende Papiergröße und Ausrichtung (Sie
können \texttt{letter} Papiergröße auswählen, im Fall dass Sie
\texttt{paper=screen} in
der Klassenoption setzen). Schließlich gehen Sie zum  \texttt{View} (oder
\texttt{File}$\rhd$\texttt{Export}) Menü und wählen Sie Ihren Output 
(PostScript or PDF).

\subsection{Erweiterung des Layouts}

Wenn Sie einen individuell gefertigten Stil  (siehe Abschnitt~\ref{sec:writestyle}),
welcher individuelle Schablonen definiert, kreiert haben, müssen Sie die Layout-Datei\footnote{Die LPPL 
schreibt vor, eine Datei umzubennen, wenn Sie diese modifizieren, um 
Verwechselungen zu vermeiden.} erweitern, sodass diese Schablonen ebenfalls von 
\LyX unterstützt werden. Die Erklärung unten unterstellt, dass Sie eine
Schablone (genannt \texttt{sunnyslide}) definiert haben.

Um diese neue Schablone in \LyX zu unterstützen, müssen Sie den folgenden Befehl benutzen.
\begin{Verbatim}[frame=single,fontsize=\small,fillcolor=\color{yellow}]
 \{pddefinelyxtemplate}<cs>{<template>}
\end{Verbatim}

\marginpar{\texttt{$\backslash$pddefinelyxtemplate}}

Dies wird die Befehlssequenz <\textsl{cs}> derart definieren, dass Sie eine Folie mit der Schablone  
<\textsl{template}> kreieren (die mit der Benutzung von
\slash{pddefinetemplate} definiert wurde). Diese neue Befehlssequenz kann wie folgt in der Layout-Datei 
gebraucht werden.
\begin{Verbatim}[frame=single,fontsize=\small]
 # SunnySlide environment definition
 Style SunnySlide
   CopyStyle     Slide
   LatexName     lyxend\lyxsunnyslide
   Font
     Color       Yellow
   EndFont
   Preamble
     \pddefinelyxtemplate\lyxsunnyslide{sunnyslide}
   EndPreamble
 End
\end{Verbatim}
Beachten Sie, dass das \texttt{LatexName} Feld mit \texttt{lyxend} beginnen muss. Die Definition 
der \LyX\  Schablone wurde zwischen 
\texttt{Preamble} und \texttt{EndPreamble} eingefügt, was gewährleistet, dass die neue \LyX\
Umgebung in jeder Präsentation funktionieren wird. Nachdem die Layout-Datei
identifiziert wurde, 
vergessen Sie nicht \LyX wieder zu starten. Für mehr Informationen über das Erstellen einer \LyX\ 
Umgebung schauen Sie in die Dokumentation für \LyX\ im \texttt{Hilfemenü}.

\section{Fragen }\label{sec:questions}
\subsection{Häufig gestellte Fragen}\label{sec:FAQ}

Dieser Abschnitt ist häufig gestellten Fragen gewidmet. Lesen Sie aufmerksam, Ihre Probleme könnten in 
diesem Abschnitt gelöst werden. 
\begin{description}
\item[\textbf{Frage 1}]
Hat \pf{powerdot} Beispieldateien? Wo kann ich diese finden? 
\item[\textbf{Antwort 1}]
\pf{powerdot} hat einige Beispiele, die sich in ihrem Pfad bei der \LaTeX\ Installation befinden. Genauer:
\url{texmf/doc/latex/powerdot}. Wenn Sie diese hier nicht finden können, laden Sie sie von CTAN herunter 
\url{CTAN:/macros/latex/contrib/powerdot}
\cite{CTAN}.
\item[\textbf{F 2}]
Ich bekomme Fehler oder unerwartete Outputs, wenn die simpelste Programmübersetzung läuft. 
\item[\textbf{A 2}]
 Haben Sie Abschnitt~\ref{sec:compiling} gelesen?
\item[\textbf{F 3}]
Ich habe einen Tippfehler in dem Foliencode gemacht, ließ die Datei durchlaufen, berichtigte den 
Tippfehler und ließ die Datei erneut durchlaufen. Aber nun bekam ich einen Fehler, der sich nicht 
entfernen ließ. 
\item[\textbf{A 3}]
Entfernen Sie die \texttt{.bm} und \texttt{.toc} Dateien und versuchen Sie es erneut.
\item[\textbf{F 4}]
\slash{pause} funktioniert nicht in der \texttt{align} Umgebung\footnote{Es gibt einige andere Umgebungen,
die ähnlich funktionieren. Ein Beispiel ist die \texttt{split} Umgebung, aber dies (oft in dem 
\pf{amsmath} Packet) kann die gleichen Probleme mit \slash{pause}
verursachen.}.
\item[\textbf{A 4}]  
\texttt{align} macht einige komplizierte Dinge, die es unmöglichen machen
\slash{pause} zu benutzen. 
Benutzen Sie stattdessen \slash{onslide}. Siehe Abschnitt~\ref{sec:onslide}.
\item[\textbf{F 5}]
Meine \pf{pstricks} nodes treten auf allen Overlays auf. Außerdem scheint
\texttt{color} nicht mit  \slash{onslide} zu funktionieren.
\item[\textbf{A 5}]
Einige PostScript Tricks , wie nodes und color funktionieren nicht mit 
\slash{onslide}.  Benutzen Sie stattdessen  \slash{onslide*}. Siehe das
folgende Beispiel.

\begin{Verbatim}[frame=single,fontsize=\small]
 \documentclass{powerdot}
 \usepackage{pst-node}
 \begin{document}
 \begin{slide}{Color}
 \onslide*{2}{\cnode(0,-5pt){2pt}{A}}
 This is {\onslide*{2-}{\color{red}} red} text.
 \onslide*{2}{\cnode(0,-5pt){2pt}{B}}
 \onslide{2}{\ncline{A}{B}}
 \end{slide}
 \end{document}
\end{Verbatim}
\item[\textbf{F 6}]
Muss ich Stil-Dateien editieren, um diese ein bisschen zu verändern?
\item[\textbf{A 6}]
Nein, Sie müssen keine Stil-Dateien editieren. Sie können jeden Teil eines bestimmten Stils verändern, 
indem Sie die Befehle  \slash{pddefinetemplate} und
\slash{pddefinepalettes} anwenden. Hier ist ein Beispiel, das die rechte Fußnote aus dem \pf{default} 
Stil entfernt, die linke Fußnote in das Zentrum verschiebt und eine andere Farbskala hinzufügt. 
\begin{Verbatim}[frame=single,fontsize=\small]
 \documentclass{powerdot}
 \pddefinetemplate[slide]{slide}{
   lfhook=Bc,lfpos={.5\slidewidth,.04\slideheight},
   rfpos
 }{}
 \pddefinepalettes{mypalette}{
   \definecolor{pdcolor1}{rgb}{.27,.31,.44}
   \definecolor{pdcolor2}{rgb}{.85,.85,.92}
   \definecolor{pdcolor3}{rgb}{.8,.75,.98}
 }
 \pdsetup{
   lf=My presentation,
   palette=mypalette
 }
 \begin{document}
 \begin{slide}{Title}
 \end{slide}
 \end{document}
\end{Verbatim}
Siehe Abschnitt~\ref{sec:writestyle} für mehr Informationen über diese beiden Befehle. 
\item[\textbf{F 7}]
Kann ich bei diesem Projekt mitarbeiten?
\item[\textbf{A 7}]
Sicher. Wenn Sie Fehler\footnote{Stellen Sie sicher, dass Sie bestätigen können, dass
der Fehler wirklich von \pf{powerdot} verursacht wird und nicht von einem anderen Paket, 
dass Sie benutzen.} oder Tippfehler finden, senden Sie uns eine Nachricht zu der Verteilerliste 
(siehe Abschnitt~\ref{sec:mailinglist}). Wenn Sie einen eigenen Stil erstellt haben, der sich von den 
existierenden Stilen unterscheidet und Sie diesen in  \pf{powerdot}aufgenommen sehen wollen, informieren 
Sie uns bitte durch eine private E-Mail und wir werden Ihren Beitrag prüfen. Beachten Sie, dass 
aufgenommene Beiträge unter das allgemeine \pf{powerdot} Lizenz und Urheberrecht fallen. Aber Ihr Name
wird in die Dokumentation aufgenommen, wenn Sie einen Beitrag leisten. Dies
wird getan, um zu garantieren,
dass wir Dateien adaptieren falls Wartungsarbeiten nötig sind. 
\end{description}

Wenn Ihre Fragen an diesem Punkt nicht beantwortet wurden, stoßen Sie zu dem nächsten Abschnitt vor, um 
zu Lesen, wo Sie mehr Antworten finden. 

\subsection{Mailing-Liste}\label{sec:mailinglist}

\pf{powerdot} hat eine Mailing-Liste von \url{freelists.org} und hat die Website hier:
\begin{center}
\url{http://www.freelists.org/list/powerdot}
\end{center}
Da gibt es einen Link zu  `List Archive'. Bitte durchsuchen Sie das Archiv, bevor Sie ihre Frage stellen.
Ihr Problem wurde vielleicht schon in der Vergangenheit gelöst. 

Wenn dies nicht der Fall ist, benutzen Sie die Box auf der Seite, um Ihre
E-Mail-Adresse zu schreiben; wählen Sie die Aktion `Subscribe' und klicken Sie auf `Go!'. Folgen Sie dann 
den Anweisungen, die Sie per E-Mail erreicht haben. An diesem bestimmten Zeitpunkt können Sie sich zum 
ersten Mal mit einem autorisierten Code, der Ihnen per E-Mail zugesandt wurde, einloggen. Nachdem Sie 
sich eingeloggt haben, können Sie sich ein eigenes Passwort für zukünftige Arbeitssitzungen unter dem 
Button `Hauptmenü' erstellen. Die anderen Buttons versorgen Sie mit einigen Informationen und Optionen 
für Ihren Account. 

Wenn Sie alles eingestellt haben, können Sie der Liste schreiben, indem
Sie eine E-Mail an 
\begin{center}
\url{powerdot [at] freelists [dot] org}
\end{center}
senden.

Wenn Sie der Liste schreiben, behalten Sie die folgenden Kernpunkte im Hinterkopf. 
\begin{enumerate}
\item Wir sind Freiwillige!
\item Beziehen Sie Ihre Fragen auf \pf{powerdot}.
\item Liefern Sie ein  \emph{minimales} Beispiel, was Ihr Problem deutlich macht. 
\item Senden Sie an die Liste keine großen Dateien. 
\end{enumerate}

Wir hoffen, dass Sie Gefallen an diesem Service finden.

\section{Quelltextdokumentation}\label{sec:source}

Für den Fall, dass Sie die Paketdateien von der Quelle aus erneuen möchten oder Sie einen Blick auf die 
Quelltextbeschreibung werfen möchten, finden Sie 
\texttt{powerdot.dtx}, suchen Sie in der Datei nach \slash{OnlyDescription}, entfernen Sie dies und 
führen sie aus: 
\begin{Verbatim}[frame=single,fontsize=\small]
 latex powerdot.dtx
 latex powerdot.dtx
 bibtex powerdot
 makeindex -s gglo.ist -o powerdot.gls powerdot.glo
 makeindex -s gind.ist -o powerdot.ind powerdot.idx
 latex powerdot.dtx
 latex powerdot.dtx
\end{Verbatim}


\section{Literatur}
\begin{thebibliography}{999}
\bibitem{HA-prosper} \textsc{Hendri Adriaens.}
 \texttt{HA-prosper} package.\\ \url{CTAN:/macros/latex/contrib/HA-prosper}.
\bibitem{xkeyval} \textsc{Hendri Adriaens.} \texttt{xkeyval} package.\\
\url{CTAN:/macros/latex/contrib/xkeyval}.
\bibitem{random} \textsc{Donald Arseneau.} \texttt{random.tex}.
\url{CTAN:/macros/generic/misc}.
\bibitem{enumitem} \textsc{Javier Bezos.} \texttt{enumitem} package.\\
\url{CTAN:/macros/latex/contrib/enumitem}.
\bibitem{graphics} \textsc{David Carlisle.} \texttt{graphics bundle}.\\
\url{CTAN:/macros/latex/required/graphics}.
\bibitem{LyXWeb} \LyX{} \textsc{crew.} \LyX{} website. \url{http://www.lyx.org}.
\bibitem{CTAN} \textsc{CTAN crew.} The Comprehensive \TeX{} Archive Network.\\
\url{http://www.ctan.org}.
\bibitem{natbib} \textsc{Patrick W. Daly.} \texttt{natbib} package.\\
\url{CTAN:/macros/latex/contrib/natbib}.
\bibitem{prosper} \textsc{Fr\'{e}d\'{e}ric Goualard and Peter M\o ller
Neergaard.}\\ \texttt{prosper class}. 
\url{CTAN:/macros/latex/contrib/prosper}.
\bibitem{xcolor} \textsc{Uwe Kern.} \texttt{xcolor} package.
\url{CTAN:/macros/latex/contrib/xcolor}.
\bibitem{extsizes} \textsc{James Kilfiger and Wolfgang May.}
\texttt{extsizes bundle}.\\
\url{CTAN:/macros/latex/ contrib/extsizes}.
\bibitem{companion} \textsc{Frank Mittelbach and Michel Goossens.} The
\LaTeX{} Companion. Tools and 
Techniques for Computer Typesetting. Addison-Wesley, Boston, Massachusetts, 2 edition, 2004. 
With Johannes Braams, David Carlisle, and Chris Rowley.
\bibitem{hyperref} \textsc{Sebastian Rahtz and Heiko Overdiek.}
\texttt{hyperref} package.\\
\url{CTAN:/macros/latex/contrib/hyperref}.
\bibitem{geometry} \textsc{Hideo Umeki.} \texttt{geometry} package.\\
\url{CTAN:/macros/latex/contrib/geometry}.
\bibitem{PSTricksWeb} \textsc{Herbert Vo}{\ss}. \texttt{PSTricks website}. 
\url{http://pstricks.tug.org}.
\bibitem{PSTricks} \textsc{Timothy Van Zandt et al.} \texttt{PSTricks} package, v1.07,
2005/05/06. \url{CTAN: /graphics/pstricks}.

\end{thebibliography}

\section*{Danksagung}
Die Autoren danken Mael Hill\'ereau für das Beisteuern der \LyX -Layoutdatei und deren 
Beschreibung. Ferner all jenen, die Stile für \pf{Powerdot} bereitstellten 
(Abschnitt~\ref{sec:styles}) und darüberhinaus allen, die an diesem Paket
in der einen oder 
anderen Weise beteiligt waren.\\[1em]
\hspace*{\stretch{1}}
\begin{minipage}{.9\linewidth}
Ramon van den Akker, Pavel \v C\'i\v zek, Darren Dale, Hans Marius
Eikseth, Morten H\o gholm, Andr\'as Horv\'ath, Laurent Jacques, Akira
Kakuto, Uwe Kern, Kyanh, Theo Stewart, Don P. Story and Herbert Vo\ss.
\end{minipage}
\hspace*{\stretch{1}}\\[1em]
Wir hoffen, dass wir niemanden vergessen haben.

\end{document}

