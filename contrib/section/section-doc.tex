\documentclass[pagesize=auto, numbers=enddot]{scrartcl}

\usepackage{fixltx2e}
\usepackage{etex}
\usepackage{xspace}
\usepackage{lmodern}
\usepackage[T1]{fontenc}
\usepackage{textcomp}
\usepackage{array}
\usepackage{microtype}
\usepackage{hyperref}

\newcommand*{\mail}[1]{\href{mailto:#1}{\texttt{#1}}}
\newcommand*{\pkg}[1]{\textsf{#1}}
\newcommand*{\cls}[1]{\textsf{#1}}
\newcommand*{\cs}[1]{\texttt{\textbackslash#1}}
\makeatletter
\newcommand*{\cmd}[1]{\cs{\expandafter\@gobble\string#1}}
\makeatother
\newcommand*{\meta}[1]{\textlangle\textsl{#1}\textrangle}
\newcommand*{\marg}[1]{\texttt{\{}\meta{#1}\texttt{\}}}
\newcommand*{\oarg}[1]{\texttt{[}\meta{#1}\texttt{]}}

\addtokomafont{title}{\rmfamily}
\deffootnote{1em}{1em}{\textsuperscript{\thefootnotemark}}

\title{The \pkg{section} package\thanks{This manual corresponds to \pkg{section.sty}~v1.4, dated~10 Mar 99.}}
\author{Oliver Pretzel\thanks{\mail{o.pretzel@ic.ac.uk}}}
\date{10 Mar 99}


\begin{document}

\maketitle

\begin{abstract}
  \noindent
  Leslie Lamport in writing \LaTeX\ assumed that authors would write
  complete \texttt{.sty} (\texttt{.clo}) files to implement the design decisions for their
  publications. However, the overwhelming majority of \LaTeX\ publications use the
  standard styles supplied by Lamport. These implement a particular set of
  typographic conventions, which while consistent are not of universal validity,
  for instance many publishers and authors would like a full stop:
  `4. Introduction'
  rather than
  `4 Introduction'.

  This file modifies the standard classes so that other conventions can be
  more easily implemented. Careful use of the file will allow a greater
  variation in the appearance of \LaTeX\ documents.
\end{abstract}


\setcounter{section}{-1}
\section{Changes to section implementation.}

The file redefines the way \LaTeX\ deals with negative indents for section
headings, placing the number in the margin and the heading itself flush
with the margin. With negative indents it sets starred section headings
flush (as it does when \cmd{\thesection} is empty.
Standard \LaTeX\ indenting can be produced by typing
%
\begin{verbatim}
>> \let\ifind\iffalse
\end{verbatim}
%
This file's version can be reestablished by typing
%
\begin{verbatim}
>> \let\ifind\ifdim
\end{verbatim}

The file also changes the way \LaTeX\ reacts to empty numbers when sections have
non-negative indents, placing the heading where the number would go, rather
than further indented (as in standard \LaTeX).
This effect can be removed by redefining an internal macro \cmd{\@makesvsec} (see
below).

The indents, pre- and postskips for section headings are placed in registers
so that they are constant throughout a document and (just as the heading sizes)
do not react to surrounding size changes. You can change this back to ordinary
\LaTeX\ behaviour by \cmd{\def}ing them as macros rather than setting the register
values (in that case you should probably use \verb|\smaller\larger| from
\pkg{SMALLER.STY} for the \cmd{\...size} commands).

\emph{WARNING}\quad\LaTeX\ typesets headings that run on by manipulating \cmd{\everypar}. If you
make local changes to the heading style of such a section inside braces or a
group you must ensure that a non-empty paragraph occurs before the closing
brace. Otherwise some strange effects may occur, because the \cmd{\everypar} tokens
will be evaluated outside the group (this is a feature of \LaTeX, not of
\pkg{SECTION.STY}). If your group ends with an empty section of this type just
follow it with \cmd{\leavevmode} or an explicit space: \verb*|\ |.

A detailed description of the format of \LaTeX\ sectioning commands is given
below (in paragraph~\ref{sec:9}). This is partially taken from the file \texttt{LATEX.TEX}.
We first list the macros for local control which you can use without knowing
this structure.

The file sets vertical distances that \LaTeX\ expressed in terms of `em's and
`ex's in equivalent `\cmd{\baselineskip}'s, as `em' is a measure of the horizontal
size of a font and `ex' is not a true measure of the full height of a line.
Earlier vesions used `ex's.

The original values are placed as comments.


\section{Sizing commands for Parts, Chapters and Titles etc}

These are set to \cmd{\large} or whatever
%
\begin{flushleft}
  \begin{tabular}{@{}>{\texttt{>{}> }}lll@{}}
    \cmd{\chsize}:  & Chapter, Part Heading    & (\verb|\LARGE\large|)     \\
    \cmd{\hdsize}:  & Chapter, Part Title Text & (\verb|\Huge\huge\LARGE|) \\
    \cmd{\ttsize}:  & Title of Document        & (\verb|\Huge\huge\LARGE|) \\
    \cmd{\ausize}:  & Author                   & (\verb|\LARGE\Large|)     \\
    \cmd{\dasize}:  & Date.                    & (\verb|\Large\large|)     \\
    \cmd{\secsize}: & Section                  & (\verb|\Large\large|)     \\
    \cmd{\subsize}: & Subsection               & (\verb|\large\normalsize|)
  \end{tabular}
\end{flushleft}
%
\texttt{subsubsection} and below use \cmd{\normalsize}


\section{Layout commands for headings.}

\begin{itemize}
\item
  \cmd{\hdpos}: Positioning of heading (\cmd{\raggedright}) (\cmd{\centering} \cmd{\raggedleft})
  (see section~\ref{sec:9});

\item counter \texttt{hddpth} (default~0) \\
  Maximum section level for \cmd{\hdpos} (chapters and parts \emph{ALWAYS} use \texttt{hdpos}).
  To centre section headings and above:
\begin{verbatim}
\let\hdpos\centering\setcounter{hddepth}{1}
\end{verbatim}

\item \cmd{\fpind}: Indenting of first paragraph after \cmd{\Part} \cmd{\Chapter}
  and also to override setting via \meta{preskip} for sections.
  \cmd{\let} to \cmd{\@afterindenttrue} to indent, \cmd{\@afterindentfalse} for flush,
  \cmd{\relax}: flush after \cmd{\Part} \cmd{\Chapter}, negative \meta{preskip}s; indented otherwise
  (\LaTeX\ default behaviour).
\end{itemize}


\section{Fonts}

\begin{itemize}
\item \cmd{\ttfnt}: font for title entries.
\item \cmd{\hdfnt}: font for all other headings (Part, Chapter \dots\ subparagraph).
\item \cmd{\fefnt}: font for theorem headings and description labels (\texttt{fe} = feature).
\item \cmd{\thfnt}: font for theorem statements.
\item \cmd{\pgfnt}: font for page numbers.
\item \cmd{\mkfnt}: font for page headings (\texttt{mk} = mark).
\item \cmd{\mkcse}: case for page headings.
  These can also be used to add other initial commands such as spaces or
  symbols.
\end{itemize}


\section{Optional punctuation (standard|my preference)} %|

These marks are transferred via auxiliary files for tables of contents etc.
They should be \cmd{\def}'d if their current value is to be transferred, and \cmd{\let}
if the value in force when the file auxiliary is read should be used.

\begin{flushleft}
  \begin{tabular}{@{}>{\texttt{>{}> }}lll@{}}
    \cmd{\hddot} & (\verb+{}+|\verb+.+) & (after inline headings, theorems etc)                       \\ %|
    \cmd{\cpdot} & (\verb+:+|\verb+.+)  & (in figure/table captions - also apears in list of figures. \\ %|
    \cmd{\nmdot} & (\verb+{}+|\verb+.+) & (after section numbers etc)                                 \\ %|
  \end{tabular}
\end{flushleft}
%
If these are \cmd{\let} to a single letter they are not expanded in the
\texttt{.toc}\dots\ files. So to remove \cmd{\cpdot} from list of figures:
%
\begin{verbatim}
\let\cpdot\relax\listoffigures\let\cpdot:
\end{verbatim}


\section{Spacing parameters.}

The macros for chapters and parts have been altered so that all the
vertical spacing commands are in a single macro in each case.
%
\begin{itemize}
\item
  \cmd{\part} inserts initial vertical space directly (preceded by \verb+\hbox{}+ if it
  is to be immovable) and passes the space between Part n  and title
  in  \cmd{\@tempskipa} and the space below the heading in \cmd{\@tempskipb}.

\item
  Both \cmd{\chapter} and \cmd{\chapter*} call \cmd{\@makechapterhead}\marg{chaptertitle}.
  This passes \marg{chaptertitle} on to \cmd{\@makeschapterhead} together with the
  vertical spacing. Default setting is
%
\begin{verbatim}
\def\@makechapterhead#1{%
  \hbox{}\@makeschapterhead{#1}{50\p@}{20\p@}{40\p@}}
\end{verbatim}
%
  \verb+\hbox{}+ makes first skip immovable,
  the first length is space above the heading,
  The second is the space between Chapter nn and the chapter title
  The third is the space below the heading.
  For both Part and Chapter the middle space is removed if either the
  section name or title are missing.
  If both are missing, the first space is also removed

\item
  \begin{sloppypar}
    For sections and below the spaces are kept in registers:
    \cmd{\secpreskp} and \cmd{\secpsrskp} \cmd{\secindent} used for \cmd{\section}
    \cmd{\subpreskp} and \cmd{\subpstskp} \cmd{\subindent} used for \cmd{\subsection}/\cmd{\subsubsection}
    \cmd{\paragraph} uses \cmd{\subindent} \cmd{\subpreskp} \cmd{\parpstskp}
    \cmd{\subparagraph} uses \cmd{\parindent} \cmd{\subpreskp} \cmd{\parpstskp}
  \end{sloppypar}

\item
  \cmd{\secindent} \cmd{\subindent}\quad Heading indentation

\item
  \cmd{\secpreskp} \cmd{\subpreskp}\quad Space before

\item
  \cmd{\secpstskp} \cmd{\subpstskp} \cmd{\parpstskp}\quad Space after.
\end{itemize}


\section{Internal modifications}

Sections with negative indents have heading flush with margin.

If \cmd{\thesection} is \cmd{\empty}
%
\begin{flushleft}
  \verb+\let\thesection\empty+ or \verb+\def\thesection{}+
\end{flushleft}
%
then \texttt{section} does not step the section counter.
If the section produces empty output for a value then
the section will start flush at the margin or \cmd{\secindent} if it is positive
and there will be no \cmd{\nmdot}.
It will produce (flush) page headings and a table of contents entry.
Similarly if \cmd{\thechapter} or \cmd{\thepart} are defined to be empty they will
suppress the heading line `Chapter\dots'\ and not step the counter.
To restore standard \LaTeX\ behaviour temporarily define \cmd{\thesection} to be
\cmd{\empty}
%
\begin{verbatim}
\def\thesection{\empty}
\end{verbatim}
%
In that case you can adjust post number spacing in the heading.
%
\begin{verbatim}
\section[Heading]{\hskip-1em Heading}
\end{verbatim}
%
You can also redefine \cmd{\@makesecdot} see paragraph~\ref{sec:7}.

\emph{WARNING}\quad\cmd{\let}ting \cmd{\thesection} etc to an unexpandable token (like \cmd{\relax}) will
cause strange effects under \LaTeX\ because its value will not be expanded.

\cmd{\maketitle} suppresses page headings (\verb+\thispagestyle{empty}+) and only resets
\cmd{\@title} etc locally. So place in a group if you want to preserve these for
later reference (eg.\ in headers).


\section{New internal macros}
\label{sec:7}

Place any modifications in the section \texttt{REDEFINITIONS} of the main code
%
\begin{itemize}
\item \cmd{\@secdot} contains section number and \cmd{\nmdot} if the number is non-empty.

\item \cmd{\@makesecdot}\marg{NAME} defines \cmd{\@secdot}.
  To make \LaTeX\ step the section counter independently of the definition
  of \cmd{\thesection} and always print \cmd{\nmdot}:
\begin{verbatim}
\def\@makesecdot#1{\refstepcounter{#1}%
    \edef\@secdot{\csname the#1\endcsname}}
\end{verbatim}
  If you want \cmd{\nmdot} suppressed when \cmd{\thesection} produces no output:
\begin{verbatim}
\def\@makesecdot#1{\refstepcounter{#1}%
  \edef\@secdot{%
    \expandafter\@gobifnul\csname the#1\endcsname\nmdot}}
\end{verbatim}

\item \cmd{\@makesvsec}\marg{INDENT} defines the \cmd{\@svsec} macro used by \LaTeX\ to typeset the
  section number\dots Its parameter is the indent so that it can react
  differently to negative indents. (it does \emph{NOT} make the indent).
  To obtain standard \LaTeX\ behaviour:
\begin{verbatim}
\def\@makesvsec#1{\def\@svsec{\@secdot\quad}}
\end{verbatim}
  To make positioning commands apply to the heading ignoring the number:
\begin{verbatim}
\def\@makesvsec#1{\def\@svsec{\llap{@secdot\quad}}}
\end{verbatim}
  Do \emph{NOT} forget the \verb+#1+.

\item \cmd{\@contmark}\marg{NAME}\marg{LEVEL}\marg{HEADING} makes contents line and sectionmark.
\end{itemize}

Contents headlines are defined by
%
\begin{itemize}
\item \cmd{\@tocheadline}\quad
 This allows these lines to be adjusted similarly to the \cmd{\@dotted\-tocline}
 commands for other contents lines. The format of these is defined below (in
 paragraph~\ref{sec:11}).

\item \verb+@tocpagenum#1+ sets \verb+#1+ flush right in \cmd{\@tocheadline}. \\
  Redefine (\verb+\let\@tocpagenum\@gobble+) to remove page numbers.
\begin{verbatim}
\def\l@part{\@tocheadline{0}{2.25em}{3em}{\let\@tocpagenum\@gobble
      \centering\large\bf}}
\end{verbatim}
  Centres parts lines of contents but leaves chapters unchanged
\begin{verbatim}
\def\l@part{\@tocheadline{0}{2.25em}{3em}{\let\@tocpagenum\@gobble
      \centering\def\numberline##1{\hskip-\secindent
      \hbox to\secindent{##1\hfil}}\large\bf}}#
\end{verbatim}
  Same but centring ignores part number.
\begin{verbatim}
\def\l@part{\@tocheadline{0}{2.25em}{3em}{\let\@tocpagenum\@gobble
      \centering\def\numberline##1{\hfilneg
      \hbox to\z@{##1\hss}\hfil}\large\bf}}
\end{verbatim}
  Centres heading with number flush left\\
  Contents lines use \cmd{\secindent} as a temporary register

\item \cmd{\makeheading} (set to \cmd{\section*}, \cmd{\chapter*})
 Makes headings for contents, index etc.
 To put Contents itself in the contents list, redefine (locally):
\begin{verbatim}
\let\thesection\empty\let\makeheading\section
\end{verbatim}
\end{itemize}


\section{Additional macros}

\begin{itemize}
\item \cmd{\@ifempty}\marg{test}\marg{yes}\marg{no} like \cmd{\@ifundefined} but tests for empty
  (fails iff test starts with \verb+~+ of catcode~12 (other)).
\item \cmd{\@gobifnul}\marg{a}\marg{b} = \cmd{\@ifempty}\marg{a}\verb+{}{+\meta{a}\meta{b}\verb+}+.
\end{itemize}


\section{Description of sectioning values}
\label{sec:9}

\cmd{\@startsection}\marg{NAME}\marg{LEVEL}\marg{INDENT}\marg{PRESKIP}\marg{POSTSKIP} \\
\hphantom{\cmd{\@startsection}}\marg{STYLE}\meta{optional~\texttt{*}}\oarg{ALTHEADING}\marg{HEADING} \\
Generic command to start a section.
%
\begin{flushleft}
  \begin{tabular}{@{}p{0.2\linewidth}@{\makebox[0.03\linewidth]{:}}p{0.77\linewidth}@{}}
    \meta{NAME}       & e.\,g.,\ `\texttt{subsection}' must be the name of a section counter                                                          \\
    \meta{LEVEL}      & a number, denoting depth of section~-- section~=~1; number increases down the hierarchy.                                      \\
    \meta{INDENT}     & Indentation of heading from left margin                                                                                       \\
    \meta{PRESKIP}    & Absolute value~=~skip to leave above the heading. If negative, then paragraph indent of text following heading is suppressed. \\
    \meta{POSTSKIP}   & if positive, then skip to leave below heading, else negative of skip to leave to right of run-on heading.                     \\
    \meta{ALTHEADING} & for page headings and contents.                                                                                               \\
    \meta{STYLE}      & commands to set style
  \end{tabular}
\end{flushleft}
%
If `\texttt{*}' missing, then increments the counter.  If it is present, then
there should be no \meta{ALTHEADING} argument.
Uses the counter `\texttt{secnumdepth}' whose value is the highest section
level that is to be numbered and entered into \cmd{\@dottedtocline}.


\section{Treatment of Section Entries}

\begin{description}
\item[\meta{NAME} (fixed)]
  Determines the counter to use.

\item[\meta{LEVEL} (fixed)]
  If level is greater than \verb|\counter{secnumdepth}| (default~2) no
  indexentry or number

\item[\meta{INDENTS}:]
  \cmd{\secindent} (sections), \cmd{\subindent} (subsections, \dots\ ,paragraphs).
  Subparagraphs have \cmd{\parindent} fixed. These are dimensions.

  Recommended: $\pm\cmd{\parindent}$ or 0\,pt

  If indent is negative then \pkg{SECTION.STY} causes the section number to be set
  flush right that distance from margin. Heading or text follows at the margin
  itself.

  Unnumbered headings are set at the margin if indent is negative. If the
  indent is non-negative they are set flush left with the number position.

\item[\meta{PRESKIPS}]
  \cmd{\secpreskp} (sections), \cmd{\subpreskp} (subsections \dots\ subparagraphs)

\item[\meta{POSTSKIPS}]
  \cmd{\secpstskp} (sections), \cmd{\subpstskp} (subsections, subsubsections) \\
  \cmd{\parpstskp} (paragraphs, subparagraphs)

  Recommended: leave \meta{PRESKIPS}' abs values unchanged, They are set $-$ve to
  have flush first paragraphs. Use \cmd{\secpreskp}--\cmd{\secpreskp} to indent them. \\
  Recommended: set \meta{POSTSKIPS} to $-1$\,em or $-\cmd{\parindent}$ to make sections run on.

  By assigning these values to registers \pkg{SECTION.STY} makes them independent of
  the surrounding font and size. Set the registers if you want this, redefine
  them as macros if you want them to be font and size dependent.

\item[\meta{STYLE}]
  \cmd{\secsize} \cmd{\subsize} \cmd{\hdfnt} size and fonts.
  \cmd{\secsize} for sections,
  \cmd{\subsize} for subsections.
  Subsubsections and below have \cmd{\normalsize} (fixed).
  \cmd{\hdfnt} is universal for all headings.
\end{description}


\section{Contents entries.}
\label{sec:11}

\cmd{\@tocheadline}\marg{LEVEL}\marg{PRESKIP}\marg{INDENT}\marg{STYLE} \\
\hphantom{\cmd{\@tocheadline}}\marg{ENTRY}\marg{PAGENUM} \\
\cmd{\@dottedtocline}\marg{LEVEL}\marg{MARGIN}\marg{INDENT}\marg{ENTRY}\marg{PAGENUM}

\begin{flushleft}
  \begin{tabular}{@{}p{0.145\linewidth}@{\makebox[0.03\linewidth]{:}}p{0.825\linewidth}@{}}
    \meta{LEVEL}   & determines if entry made.                                                                                            \\
    \meta{PRESKIP} & vertical gap above.                                                                                                  \\
    \meta{INDENT}  & This must leave enough room for the chapter or section number \newline It is added to the margin for the next level. \\
    \meta{MARGIN}  & Left margin inherited from level above (zero for headlines).                                                         \\
    \meta{STYLE}   & Font sizing and setting.                                                                                             \\
  \end{tabular}
\end{flushleft}
%
\LaTeX\ assumes that the \meta{ENTRY} for sections be defined by a macro \cmd{\l@section}.
These are defined in the standard document styles but can be changed.
All entries of the \cmd{\l@section} commands are fixed. \pkg{SECTION.STY} does not
change the commands using \cmd{\@dottedtocline} in the style file. Those that do
not do so (\cmd{\l@part}, \cmd{\l@chapter}) are modified to use \cmd{\@tocheadline}.

\meta{INDENT} for sections in \cls{article} and chapters in \cls{report} is rather tight at
1.5\,em. If you increase it should also increase the \meta{MARGIN}S for lower
sections.


\section{The following internal \LaTeX\ macros are redefined:}

\begin{flushleft}
  \begin{tabular}{@{}>{\raggedright\arraybackslash}p{0.3\linewidth}p{0.67\linewidth}@{}}
    \cmd{\part} \cmd{\@part} \cmd{\@spart} \cmd{\@endpart} \cmd{\@chapter} \cmd{\@schapter} & (also \cs{fi} moved) \\
    \cmd{\@makechapterhead} \cmd{\@makeschapterhead} & functions of these have been significantly changed. \\
    \cmd{\ps@headings} \cmd{\ps@myheadings} \\
    \cmd{\section} \cmd{\subsection} \cmd{\subsubsection} \cmd{\paragraph} \cmd{\subparagraph} \\
    \cmd{\thebibliography} \cmd{\theindex} \\
    \cmd{\@begintheorem} \cmd{\@oparbegintheorem} \\
    \cmd{\@makecaption} & optional style features added \\
    \cmd{\@startsection} & \cmd{\fpind} and does not pass \verb|#4| to \cmd{\@sect}, \cmd{\@ssect} \\
    \cmd{\@sect}, \cmd{\@ssect} & ignored parameter removed \newline \cmd{\llap} \cmd{\@secdot} used to adjust spacing if \cmd{\p@section} defined used for number style. calls \cmd{\@makesecdot} \newline \cmd{\@sect} only: \cmd{\@makesvsec} and \cmd{\@contmark}, which is moved in \cmd{\@svsechd} so that \cmd{\unskip} in \cmd{\@xsect} works). \\
    \cmd{\maketitle} \cmd{\@maketitle} & suppress page number, reset \cmd{\@title}\dots\ locally \\
    \cmd{\numberline} & adds space after number, so overfull box if too close to heading. \\
  \end{tabular}
\end{flushleft}


\section*{Changes}

\begin{description}
\item[Ver 1.3 (3. Apr 98)]
  If $\cmd{\secindent} < 0$, then \cmd{\section*} produces a flush heading, not outdented.
  \cmd{\hdpos} takes level as parameter and is used for all sections, not just
  \cmd{\part} and \cmd{\chapter}.
  `ex' measurements replaced by \cmd{\baselineskip} equivalents.

\item[Ver 1.31]
  Bugfix: \cmd{\relax} added to \cmd{\@alph}, \cmd{\@Alph} to enable use with \cmd{\@ifempty} test.

\item[Ver 1.32]
  \begin{sloppypar}
    Bugfixes: \cmd{\hdpos} added to \cmd{\@ssect}, unused macro \cmd{\secpos} removed,
    \cmd{\nobreak} added after \cmd{\thechapter} in \cmd{\makechapterhead}
  \end{sloppypar}

\item[Ver 1.4]
  \cmd{\hdpos} made simple macro again. Counter \cmd{\hddepth} determines when it
  should be used. Functions of \cmd{\@makechapterhead}, \cmd{\@smakechapterhead}
  changed to make editing vertical spacing of chapters easier.
  \cmd{\part}\cmd{\@part}\cmd{\@spart} rewritten so that all spacing info in \cmd{\part}.
  \cmd{\part} calls \cmd{\@spart} which finishes off the heading.
\end{description}
%
Memory use $< 200$ words.

\end{document}
