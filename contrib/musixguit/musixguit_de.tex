%%%%%%%%%%%%%%%%%%%%%%%%%%%%%%%%%%%%%%%%%%%%%%%%%%%%%%%%%%%%%%%%%%%%%%%%%%%%%%%%%%%%%%%%%
% ------------------------------------------------------------------------------------- %
% - musixguit - musixguit_de.tex ------------------------------------------------------ %
% - musixtex and guitar --------------------------------------------------------------- %
% ------------------------------------------------------------------------------------- %
% - Clemens Niederberger -------------------------------------------------------------- %
% - 2011/03/07 ------------------------------------------------------------------------ %
% ------------------------------------------------------------------------------------- %
% - http://www.niederberger-berlin.net/2011/01/latex-und-noten-fuer-klassische-gitarre/ %
% - kontakt@niederberger-berlin.net --------------------------------------------------- %
% ------------------------------------------------------------------------------------- %
% Copyright 2011 Clemens Niederberger                                                   %
%                                                                                       %
% This work may be distributed and/or modified under the                                %
% conditions of the LaTeX Project Public License, either version 1.3                    %
% of this license or (at your option) any later version.                                %
% The latest version of this license is in                                              %
%   http://www.latex-project.org/lppl.txt                                               %
% and version 1.3 or later is part of all distributions of LaTeX                        %
% version 2005/12/01 or later.                                                          %
%                                                                                       %
% This work has the LPPL maintenance status `maintained'.                               %
%                                                                                       %
% The Current Maintainer of this work is Clemens Niederberger.                          %
%                                                                                       %
% This work consists of the files musixguit.sty and musixguit_de.tex                    %
% ------------------------------------------------------------------------------------- %
%%%%%%%%%%%%%%%%%%%%%%%%%%%%%%%%%%%%%%%%%%%%%%%%%%%%%%%%%%%%%%%%%%%%%%%%%%%%%%%%%%%%%%%%%
\documentclass[DIV10]{scrartcl}
%%%%%%%%%%%%%%%%%%%%%%%%%%%%%%%%%%%%%%%%%%%%%%%%%%%%%%%%%%%%%%%%%%%%%%%%%%%%%%%%%%%%%%%%%
% = PAKETE/EINSTELLUNGEN ============================================================== %
%%%%%%%%%%%%%%%%%%%%%%%%%%%%%%%%%%%%%%%%%%%%%%%%%%%%%%%%%%%%%%%%%%%%%%%%%%%%%%%%%%%%%%%%%
\usepackage[ngerman]{babel}                      % deutsche Sprache
\usepackage[latin1]{inputenc}                    % Zeichenkodierung Eingabe
\usepackage{etex}
\usepackage{xcolor,xspace}
  \definecolor{dunkelblau}{rgb}{0,0.33,0.62}
  \colorlet{musixcolor}{dunkelblau!50!black!50}
\usepackage[color=musixcolor]{chemexec}          % `chemexec' f"ur die Beispiel-Umgebung
\usepackage[symbol,perpage]{footmisc}            % Fussnoten
\usepackage{musixguit}                           %
\usepackage{url,listings,graphicx}               % Web-Adressen und Code-Listings

% = Kopfzeile ========================================================================= %
\usepackage{fancyhdr}
\pagestyle{fancy}
\fancyhead{}
%\headheight 13pt
\lhead{\rightmark}
\chead{{\ttfamily musixguit.tex} \MGversion}
\rhead{\leftmark}
\lfoot{\small\color{lightgray}-~Seite~\thepage~-}
\cfoot{}
\rfoot{}
% = "Uberschriften ==================================================================== %
\setkomafont{disposition}{\rmfamily\bfseries}    % Gewicht fett und Schriftart roman

% = Listings einstellen =============================================================== %
\lstset{
   language=[LaTeX]TeX,                          % Sprache festlegen
   basicstyle={\ttfamily\footnotesize},          % Grundstil
   extendedchars=true,
   numbers=left,                                 % Zeilennummern
   numberstyle=\tiny,                            % Gr"osse des Zeilennummern
   numberblanklines=true,                        % Leerzeilen nummerieren
   gobble=1,                                     % das erste Leerzeichen abschneiden
   xleftmargin=20pt,                             % Einr"uckung links
   breaklines=true,                              % Zeilenumbruch
   moredelim=[is][\Red]{|}{|}                    % Hervorhebung
}
\usepackage{hyperref}
  \hypersetup{colorlinks=true,                   % Setup der Hyperlinks und des pdf
              linkcolor=black,
              urlcolor=blue!70,
              citecolor=black,
              plainpages=false,
              bookmarksopen=true,
              bookmarksopenlevel=1,
              bookmarksnumbered=true,
              pdfstartview=FitH,
              pdfauthor={Clemens Niederberger},
              pdftitle={musixguit},
              pdfsubject={Notes for (classical) guitar with musixTeX},
              pdfkeywords={musixguit},
              pdfcreator={LaTeX}
  }

% = Befehle =========================================================================== %
\newcommand\Red{\color{red}}
\makeatletter
\newcommand\MG{{\color{musixcolor}musixguit}\xspace}
\newcommand\MGversion{\mg@version\xspace}
\newcommand\MGdate[1]{
 \ifthenelse{\equal{#1}{de}}{\mg@date@de}{}
 \ifthenelse{\equal{#1}{en}}{\mg@date@en}{}
}
\makeatother
\begin{document}
\title{{\Huge\MG} \MGversion}
  \author{Clemens Niederberger \and \path=kontakt@niederberger-berlin.net=\thanks{~~Bei Anregungen, Kritik und Fragen bitte obenstehende E-Mail-Adresse verwenden.}\\
   \path=http://www.niederberger-berlin.net/=}
  \date{\MGdate{de}}
\maketitle

\begin{abstract}
 Das \verb=musixguit= Paket stellt ein paar kleine Befehle zur Verf"ugung, um z.B. Songtexte mit Akkorden zu versehen.\\ Au\ss{}erdem werden einige Zusatzbefehle f"ur das \verb=musixtex= Paket definiert, die inbesondere dann n"utzlich sind, wenn man Noten f"ur Gitarre schreiben m"ochte.
\end{abstract}
\tableofcontents
\section{Ben\"otigte Pakete}
`musixguit' ben"otigt zwei Pakete, und zwar
\begin{itemize}
 \item \verb=setspace= und
 \item \verb=musixtex=.
\end{itemize}
Au\ss erdem m"ussen die \verb=musixtex=-Erweiterungen \verb=musixper.tex= und \verb=musixgui.tex= ver\-f"ug\-bar sein.

\section{Neue Befehle}
\subsection{Allgemein}
Folgende f"unf Befehle erleichtern das Setzen von Akkorden "uber Text:
\begin{itemize}
 \item \verb=\chord{#1}{#2}= setzt den Akkord \#1 "uber \#2: \chord{C}{Alle}.
 \item \verb=\B= schreibt ein \B.
 \item \verb=\K= schreibt ein \K.
 \item \verb=\T= kann verwendet werden, um den Beginn oder das Ende eines Taktes anzudeuten: "`Alle meine \T\ Entchen\ldots"'.
 \item Die Umgebung \verb=song= bringt die einzelnen Zeilen auf den erforderlichen Abstand, so dass die Akkorde bequem dazwischen passen.
\end{itemize}
\begin{beispiel}
Folgender Code\footnote{Hervorgehobener Code ist auch im Ergebnis jeweils farbig hervorgehoben.}
\begin{lstlisting}
 \begin{framed}
 \begin{|song|}\noindent
  You are my |\T|\ |\chord{F}{|sunshine|}|, my only \T\ sunshine.\\
  You make me \T\ \chord{B|\B|}{happy} when skies are \T\ \chord{F}{gray}.\\
  You never \T\ \chord{B\B}{know}, Dear, how much I \T\ \chord{F}{love} you.\\
  Please don't \T\ take my \chord{C$^7$}{sunshine} a-\T\chord{F}{way}.
 \end{song}
 \end{framed}
\end{lstlisting}
ergibt
\begin{framed}
\begin{song}\noindent
 You are my \Red\T\ \normalcolor\chord{\Red F\normalcolor}{sunshine}, my only \T\ sunshine.\\
 You make me \T\ \chord{B\Red\B\normalcolor}{happy} when skies are \T\ \chord{F}{gray}.\\
 You never \T\ \chord{B\B}{know}, Dear, how much I \T\ \chord{F}{love} you.\\
 Please don't \T\ take my \chord{C$^7$}{sunshine} a-\T\chord{F}{way}.
\end{song}
\end{framed}
\end{beispiel}
\newpage
\subsection{`musixtex'}
Folgende Befehle wurden f"ur die Verwendung mit `musixtex' definiert. Beachten Sie: wenn Sie das \verb=musixguit= Paket laden, m"ussen Sie das \verb=musixtex= Paket nicht mehr laden. Alle Befehle werden, wenn nicht anders angegeben, innerhalb \verb=\notes=\ldots\verb=\enotes= verwendet.
\subsubsection{Klassische Notation}
\begin{itemize}
 %\item \verb=\teil{#1}= Markiert den Teil \#1 eines St"uckes.
 \item \verb=\lage[#1]{#2}{#3}= Gibt die Lage \#2 des Fingersatzes an. \#2 muss eine arabische Ziffer sein. \#3 ist die Position der Angabe, \#1 eine zus"atzliche vertikale Verschiebung um \verb=#1\internote=. Letzteres kann in den seltenen F"allen n"utzlich sein, wenn die Notenh"ohe \verb=z= nicht ausreicht.
 \item \verb=\finger{#1}{#2}= Wird verwendet, um den Fingersatz \textit{"uber} oder \textit{unter} eine Note zu schreiben. Dabei ist \#2 die Notenh"ohe der Fingersatzangabe und \#1 der benannte Finger.
 \item \verb=\Finger{#1}{#2}= Wird verwendet, um den Fingersatz \textit{vor} eine Note zu schreiben. Dabei ist \#2 die Position der Note und \#1 der benannte Finger.
 \item \verb=\barree{#1}{#2}{#3}{#4}= Makro, um eine Barr\'e-Anweisung zu setzen. \#1 ist die Notenh"ohe, bei der die Barr\'e-Klammer beginnt, \#2 ist die vertikale L"ange der Klammer in \verb=#2\internote=, \#3 die horizontale L"ange in \verb=#3\noteskip= und \#4 die Lage des Barr\'e.
 \item \verb=\saite{#1}{#2}= Die zu spielende Saite angeben. \#1 gibt die Nummer der Saite an, \#2 die Notenh"ohe, auf der das Symbol erscheinen soll.
\end{itemize}
\subsubsection{Jazz-Notation}
\begin{itemize}
 \item \verb=\strike=, \verb=\strk= Diagonaler Strich, der eine Viertelnotenl"ange repr"asentiert, ohne den Rhythmus zu spezifizieren, wie es in Jazz-Notationen "ublich ist; in einer gr"o\ss{}eren (\verb=\strike=) und einer kleineren (\verb=\strk=) Variante.
 \item \verb=\pickd[#1]= Anweisung f"ur ein \textit{downward picking} (\tpickd) im Plektrenspiel. \#1 ist ein optionales Argument, um die Lage des Zeichens in der Notenlinie anzugeben. Die Default-Einstellung ist \verb=o=.
 \item \verb=\picku[#1]= Anweisung f"ur ein \textit{upward picking} (\tpicku). \#1 hat die gleiche Funktion und Default-Einstellung wie bei \verb=\pickd=.
 \item \verb=\tpickd=, \verb=\tpicku= Stellen die Picking-Symbole auch im Text dar und sind quasi nur neue Namen f"ur \verb=\downbow= bzw. \verb=\upbow=.
 \item \verb=\release[#1]= Das Zeichen (\raisebox{\depth}{\huge ,}) zeigt an, dass der Druck der linken Hand auf die Saiten direkt nach dem Schlag entspannt werden soll, ohne die Finger von den Saiten zu nehmen. \#1 hat die gleiche Funktion wie bei \verb=\pickd=. Die Default-Einstellung ist \verb=h=
 %\item \verb=\quart=
 %\item \verb=\eighx=
\end{itemize}

\subsubsection{Beispiele}
In diesem Abschnitt werden die Befehle im Einsatz demonstriert.
\begin{beispiel}
Folgender Code ist ein Beispiel f"ur \verb=\lage[#1]{#2}{#3}=, \verb=\saite{#1}{#2}= und\\
\verb=\Finger{#1}{#2}=:
\begin{lstlisting}
 \generalmeter{\meterfrac44}%
 \parindent 0pt
 \setclefsymbol 1\treblelowoct
 \systemnumbers
 \generalsignature{0}
 \startpiece
 \leftrepeat\barno=1 % 1
 \Notes|\lage2p|\ibu0c2|\Finger2c\saite5M|\qb0{cde}\tbu0\qb0f\en
 \Notes\ibu0g1\qb0g|\Finger{1~~}g|\qb0{^g}\finger1h\qb0h\tbu0\qb0i\en
 \bar % 2
 \Notes\ibl0j2\qb0{jkl}\tbl0\Finger1m\qb0 m\en
 \Notes\ibl0n0\Finger2n\qb0{n^no}\tbl0\qb0n\en
 \bar % 3
 \Notes\ibl0n{-2}\cna n\qb0n\Finger1m\qb0m\finger4l\qb0l\tbl0\qb0k\en
 \Notes\ibl0j{-1}\qb0{jih}\Finger{4~~}h\tbl0\qb0{_h}\en
 \bar % 4
 \Notes\ibu0g{-2}\Finger4g\qb0{gfe}\tbu0\qb0d\en
 \Notes\ibu0c{-1}\qb0{cba}\tbu0\qb0b\en
 \setrightrepeat
 \endpiece
\end{lstlisting}
Das Ergebnis\footnote{aus `A Modern Method for Guitar Volume 1' von William G. Leavitt} sieht so aus:\\
\generalmeter{\meterfrac44}%
\parindent 0pt
\startbarno=1
\setclefsymbol 1\treblelowoct
\systemnumbers
\generalsignature{0}
\startpiece
\leftrepeat\barno=1 % 1
\Notes\Red\lage2p\normalcolor\ibu0c2\Red\Finger2c\saite5M\normalcolor\qb0{cde}\tbu0\qb0f\en
\Notes\ibu0g1\qb0g\Red\Finger{1~~}g\normalcolor\qb0{^g}\Finger1h\qb0h\tbu0\qb0 i\en
\bar % 2
\Notes\ibl0j2\qb0{jkl}\tbl0\Finger1m\qb0 m\en
\Notes\ibl0n0\Finger2n \qb0{n^no}\tbl0\qb0n\en
\bar % 3
\Notes\ibl0n{-2}\cna n\qb0n\Finger1m\qb0m\Finger4l\qb0l\tbl0\qb0k\en
\Notes\ibl0j{-1}\qb0{jih}\Finger{4~~}h\tbl0\qb0{_h}\en
\bar % 4
\Notes\ibu0g{-2}\Finger4g\qb0{gfe}\tbu0\qb0d\en
\Notes\ibu0c{-1}\qb0{cba}\tbu0\qb0b\en
\setrightrepeat
\endpiece

\bsp Beispiel\footnote{Takte 35 und 36 der Sonate Nr. 14 Adagio sostenuto von L.v.Beethoven} f"ur \verb=\lage[#1]{#2}{#3}=:
\begin{lstlisting}
 \notes\lage4v\zwh L\ibu0l3\finger4m\qb0{=k}\finger1p\qb0{^n}\tbu0\finger3o\qb0{=m}\en
 \notes\lage7y\ibu0o3\finger1r\qb0p\finger3p\qb0n\tbu0\finger4t\qb0{=r}\en
 \notes|\lage[2]{12}z|\ibu0r3\finger1r\qb0p\finger2v\qb0{=t}\tbu0\finger4t\qb0r\en
 \notes\lage[4]{16}z\ibu0t3\finger1w\qb0{^u}\finger3v\qb0t\tbu0\finger4y\qb0w\en
 \bar
 \notes\lage[3]{13}z\zwh L\ibu0t{-3}\finger4w\qb0{^u}\finger3t\qb0r\tbu0\finger1v\qb0t\en
 \notes\lage{10}z\ibu0q{-3}\finger3r\qb0p\finger1t\qb0r\tbu0\lage7x\finger3p\qb0{^n}\en
 \notes\ibu0o{-3}\finger1r\qb0p\finger4o\qb0m\tbu0\finger3p\qb0n\en
 \notes\lage6v\ibu0l{-3}\finger2m\qb0k\finger1o\qb0m\tbu0\lage3v\finger2k\qb0i\en
\end{lstlisting}
\startbarno=35
\setclefsymbol 1\treblelowoct
\generalsignature{0}
\setstaffs1{1}
\startpiece
\notes\lage4v\zwh L\ibu0l3\finger4m\qb0{=k}\finger1p\qb0{^n}\tbu0\finger3o\qb0{=m}\en
\notes\lage7y\ibu0o3\finger1r\qb0p\finger3p\qb0n\tbu0\finger4t\qb0{=r}\en
\notes\Red\lage[2]{12}z\normalcolor\ibu0r3\finger1r\qb0p\finger2v\qb0{=t}\tbu0\finger4t\qb0r\en
\notes\lage[4]{16}z\ibu0t3\finger1w\qb0{^u}\finger3v\qb0t\tbu0\finger4y\qb0w\en
\bar
\notes\lage[3]{13}z\zwh L\ibu0t{-3}\finger4w\qb0{^u}\finger3t\qb0r\tbu0\finger1v\qb0t\en
\notes\lage{10}z\ibu0q{-3}\finger3r\qb0p\finger1t\qb0r\tbu0\lage7x\finger3p\qb0{^n}\en
\notes\ibu0o{-3}\finger1r\qb0p\finger4o\qb0m\tbu0\finger3p\qb0n\en
\notes\lage6v\ibu0l{-3}\finger2m\qb0k\finger1o\qb0m\tbu0\lage3v\finger2k\qb0i\en
\endpiece

\bsp In diesem Beispiel\footnote{Takte 3 und 4 der Sonate Nr. 14 Adagio sostenuto von L.v.Beethoven} wird die Verwendung von \verb=\barree{#1}{#2}{#3}{#4}= demonstriert.
\begin{lstlisting}
 \notes|\barree M{24}{5.75}1| \zhl M\ibu0f8\Finger3f\qb0f\Finger2h\qb0h\tbu0\qb0j\en
 \notes\ibu0f8\qb0f\qb0h\tbu0\qb0j\en
 \notes\barree d{19}{5.75}3\finger3N\zhl d\ibu0f8\qb0f\qb0{_i}\tbu0\qb0k\en
 \notes\ibu0f8\qb0f\qb0{_i}\tbu0\qb0k\en
 \bar% Takt 4
 \notes\Finger0L\zwh L\ibu0e8\Finger2e\qb0e\finger1d\qb0{^g}\tbu0\Finger4k\qb0k\en
 \notes\ibu0e8\qb0e\Finger3h\qb0h\tbu0\Finger1j\qb0j\en
 \notes\zwh L\ibu0e8\qb0e\qb0h\tbu0\finger0f\qb0{=i}\en
 \notes\ibu0d8\Finger0d\qb0d\Finger1g\qb0g\tbu0\qb0i\en
\end{lstlisting}
\systemnumbers
\generalmeter{\meterC}
\generalsignature{0}
\setstaffs1{1}
\startbarno=3
\setclefsymbol 1\treblelowoct
\startpiece
\notes{\Red\barree M{24}{5.75}1}\zhl M\ibu0f8\Finger3f\qb0f\Finger2h\qb0h\tbu0\qb0j\en
\notes\ibu0f8\qb0f\qb0h\tbu0\qb0j\en
\notes\barree d{19}{5.75}3\finger3N\zhl d\ibu0f8\qb0f\qb0{_i}\tbu0\qb0k\en
\notes\ibu0f8\qb0f\qb0{_i}\tbu0\qb0k\en
\bar% Takt 4
\notes\Finger0L\zwh L\ibu0e8\Finger2e\qb0e\finger1d\qb0{^g}\tbu0\Finger4k\qb0k\en
\notes\ibu0e8\qb0e\Finger3h\qb0h\tbu0\Finger1j\qb0j\en
\notes\zwh L\ibu0e8\qb0e\qb0h\tbu0\finger0f\qb0{=i}\en
\notes\ibu0d8\Finger0d\qb0d\Finger1g\qb0g\tbu0\qb0i\en
\endpiece

\bsp Folgendes Beispiel zeigt die Verwendung von \verb=\strike=, \verb=\strk= \verb=\pickd=, \verb=\picku= und \verb=\release=:
\begin{lstlisting}
 \parindent 0pt
 \generalmeter\meterC
 \nobarnumbers
 \startextract % 1
 \Notes\lage3x\Uptext{Cm$^7$}|\strike|\en
 \Notes\strike\en
 \Notes\strike\en
 \Notes\strike\en
 \bar % 2
 \notes\lage5x\Uptext{Dm$^{7\B5}$}|\strk|\en
 \notes\strk\en
 \notes\strk\en
 \notes\strk\en
 \bar % 3
 \notes\lage4x\en
 \notes\Uptext{G$^{7\B9}$}|\release\pickd|\roql i\en
 \notes\release\pickd\roql i\en
 \notes\zcharnote M{sim.}\pickd\roql i\en
 \notes\pickd\roql i\en
 \bar % 4
 \notes\lage3x\en
 \notes\Uptext{Cm$^7$}\ibl0i0\pickd\roqb0 i|\picku|\tbl0\roqb0 i\en
 \notes\pickd\roql i\en
 \notes\pickd\roql i\en
 \notes\qp\en
 \endextract
\end{lstlisting}
\generalmeter\meterC
\nobarnumbers
\smallmusicsize
\def\quart{\roql i}
\startextract % 1
\Notes\lage3x\Uptext{Cm$^7$}\Red\strike\normalcolor\en
\Notes\strike\en
\Notes\strike\en
\Notes\strike\en
\bar % 2
\notes\lage5x\Uptext{Dm$^{7\B5}$}\Red\strk\normalcolor\en
\notes\strk\en
\notes\strk\en
\notes\strk\en
\bar % 3
\notes\lage4x\en
\notes\Uptext{G$^{7\B9}$}\Red\release\pickd\normalcolor\quart\en
\notes\release\pickd\quart\en
\notes\zcharnote M{sim.}\pickd\quart\en
\notes\pickd\quart\en
\bar % 4
\notes\lage3x\en
\notes\Uptext{Cm$^7$}\ibl0i0\pickd\roqb0 i\Red\picku\normalcolor\tbl0\roqb0 i\en
\notes\pickd\quart\en
\notes\pickd\quart\en
\notes\qp\en
\endextract
\normalmusicsize
Beachten Sie, dass Sie f"ur die Perkussionssymbole im vierten Takt normalerweise \verb=musixper=-Erweiterung per \verb=\input= einbinden m"ussen. Das ist mit \verb=musixguit= nicht mehr n"otig.

\bsp Einen ausgiebigeren Einsatz der Picking-Symbole zeigt folgende Arpeggio-"Ubung\footnote{aus `A Modern Method for Guitar Volume 1' von William G. Leavitt}:
\begin{lstlisting}
 \generalmeter{\meterfrac34}
 \setclefsymbol 1\treblelowoct
 \setsign13
 \startpiece
 \leftrepeat\barno=1 % 1
 \notes\lage2r\en
 \notes\ibu0a9\pickd\finger4L\qb0a\pickd\qb0c\pickd\qb0e|\pickd[p]|\qb0h\pickd[r]\qb0j|\picku[t]|\tbu0\qb0 l\en
 \bar % 2
 \notes\ibu0l{-9}\pickd[t]\qb0m\picku[s]\qb0{kifd}\tbu0\qb0 b\en
 \bar % 3
 \notes\ibl0f7\pickd\qb0e\picku\finger{4s}i\qb0g\pickd\qb0i\pickd\qb0k\picku\qb0l\pickd[p]\tbl0\qb0n\en
 \bar % 4
 \notes\ibu0n{-9}\pickd[v]\qb0o\picku[u]\qb0l\qb0{jhe}\tbu0\qb0c\en
 \bar % 5
 \notes\ibu0a8\pickd\qb0a\pickd\qb0d\pickd\qb0f\pickd[p]\qb0h\pickd[r]\qb0k\pickd[t]\tbu0\qb0m\en
 \bar % 6
 \notes\ibu0a8\pickd\qb0a\pickd\qb0c\pickd\qb0e\pickd[p]\qb0h\pickd[r]\qb0j\picku[t]\tbu0\qb0l\en
 \bar % 7
 \notes\ibu0j{-6}\pickd[r]\qb0k\picku[q]\qb0{iged}\tbu0\qb0b\en
 \rightrepeat % 8
 \Notes\zcharnote e{\qp}\hlp a\en
 \notes\zh c\zh e\zhu h\en
 \Endpiece
 \begin{center}
  Also practice arpeggios with alt. |\tpickd|\ |\tpicku| .
 \end{center}
\end{lstlisting}
\generalmeter{\meterfrac34}
\setclefsymbol 1\treblelowoct
\setsign13
\startpiece
\leftrepeat\barno=1 % 1
\notes\lage2r\en
\notes\ibu0a9\pickd\finger4L\qb0a\pickd\qb0c\pickd\qb0e\Red\pickd[p]\normalcolor\qb0h\pickd[r]\qb0j\Red\picku[t]\normalcolor\tbu0\qb0 l\en
\bar % 2
\notes\ibu0l{-9}\pickd[t]\qb0m\picku[s]\qb0{kifd}\tbu0\qb0 b\en
\bar % 3
\notes\ibl0f7\pickd\qb0e\picku\finger{4s}i\qb0g\pickd\qb0i\pickd\qb0k\picku\qb0l\pickd[p]\tbl0\qb0n\en
\bar % 4
\notes\ibu0n{-9}\pickd[v]\qb0o\picku[u]\qb0l\qb0{jhe}\tbu0\qb0c\en
\bar % 5
\notes\ibu0a8\pickd\qb0a\pickd\qb0d\pickd\qb0f\pickd[p]\qb0h\pickd[r]\qb0k\pickd[t]\tbu0\qb0m\en
\bar % 6
\notes\ibu0a8\pickd\qb0a\pickd\qb0c\pickd\qb0e\pickd[p]\qb0h\pickd[r]\qb0j\picku[t]\tbu0\qb0l\en
\bar % 7
\notes\ibu0j{-6}\pickd[r]\qb0k\picku[q]\qb0{iged}\tbu0\qb0b\en
\rightrepeat % 8
\Notes\zcharnote e{\qp}\hlp a\en
\notes\zh c\zh e\zhu h\en
\Endpiece
\begin{center}
 Also practice arpeggios with alt. \Red\tpickd\ \tpicku\normalcolor .
\end{center}
\end{beispiel}

\section{Redundanz}
Durch Laden von \verb=musixguit= werden \verb=musixtex=, \verb=musixper= und \verb=musixgui= automatisch eingebunden und m"ussen nicht mehr separat eingebunden werden. So stehen die Akkord-Tabulaturen von \verb=musixgui= und die Perkussions-Notenformen von \verb=musixper= auch zur Verf"ugung.
\begin{beispiel}
In diesem Beispiel\footnote{Die ersten acht Takte von `Blue Bossa' von Kenny Dorham.} sehen Sie verschiedenen Funktionen vereint.
\begin{lstlisting}
 \generalmeter{\meterfrac44}
 \setsign{1}{-3}
 \nobarnumbers
 \def\txh{-6.5}
 \def\tx#1*{\zchar\txh{\lrlap{\kern3\Internote#1}}}
 \def\rtx#1*{\zchar\txh{\kern-3\Internote#1}}
 \stafftopmarg10\Interligne
 \raiseguitar{20}
 \nostartrule
 \def\quart{\zcharnote{a}{|\roql d|}}
 \def\eight{\zcharnote{a}{\rocl d}}
 \def\eighx{\zcharnote{a}{|\xcl| d}}
 \startpiece
 \addspace{.5\afterruleskip}%
 \NOtes\uptext{Bossa}\qa g\en
 \doublebar % 1
 \NOtes|\guitar {Cm$^7$}{2}x-----\gbarre1\gdot33\gdot52|{\tinynotesize|\pickd[a]||\quart|}\qlp n\en
 \notes{\tinynotesize\pickd[a]|\release[N]\eight|}\en
 \Notes{\tinynotesize\picku[a]\eight}\ca m{\tinynotesize\pickd[a]\eight}\ca l\en
 \Notes{\tinynotesize\picku[a]\release[N]\quart}\qa k\en
 \Notes{\tinynotesize\picku[a]|\eighx|}\isluru0j\ca j\en
 \bar % 2
 \NOtes\zcharnote{M}{sim.}\tslur0j\hlp j\en
 \NOtes\qa i\en
 \bar % 3
 \NOTes\guitar {Fm$^7$}{}x----x\gdot24\gdot34\gdot43\gdot54\ha h\en
 \Notes\qlp n\en
 \Notes\isluru0m\ca m\en
 \bar % 4
 \NOTEs\tslur0m\wh m\en
 \bar % 5
 \NOtes\guitar {Dm$^{7|\B|5}$}{2}x----x\gdot23\gdot34\gdot43\gdot54\qlp m\en
 \Notes\ca{lk}\qa j\en
 \Notes\isluru0i\ca i\en
 \bar % 6
 \NOTEs\guitar {G$^{7\B9}$}{2}x----x\gdot23\gdot34\gdot42\gdot54\tslur0i\hlp i\en
 \Notes\qa h\en
 \bar % 7
 \NOTes\guitar {Cm$^7$}{2}x-----\gbarre1\gdot33\gdot52\ha g\en
 \Notes\qlp m\isluru0l\ca l\en
 \bar % 8
 \NOTEs\tslur0l\wh l\en
 \endpiece
\end{lstlisting}
\begin{music}
\hsize130mm
\tenrm
\parindent0pt
\generalmeter{\meterfrac44}
\setsign{1}{-3}
\nobarnumbers
\def\txh{-6.5}
\def\tx#1*{\zchar\txh{\lrlap{\kern3\Internote#1}}}
\def\rtx#1*{\zchar\txh{\kern-3\Internote#1}}
\stafftopmarg10\Interligne
\raiseguitar{20}
\nostartrule
\def\quart{\zcharnote{a}{\roql d}}
\def\eight{\zcharnote{a}{\rocl d}}
\def\eighx{\zcharnote{a}{\xcl d}}
\startpiece
\addspace{.5\afterruleskip}%
\notes\uptext{Bossa}\en
\NOtes\qa g\en
\doublebar % 1
\NOtes\Red\guitar {Cm$^7$}{2}x-----\gbarre1\gdot33\gdot52\normalcolor{\tinynotesize\Red\pickd[a]\quart\normalcolor}\qlp n\en
\notes{\tinynotesize\pickd[a]\Red\release[N]\normalcolor\eight}\en
\Notes{\tinynotesize\picku[a]\eight}\ca m{\tinynotesize\pickd[a]\eight}\ca l\en
\Notes{\tinynotesize\picku[a]\release[N]\quart}\qa k\en
\Notes{\tinynotesize\picku[a]\Red\eighx\normalcolor}\isluru0j\ca j\en
\bar % 2
\NOtes\zcharnote{M}{sim.}\tslur0j\hlp j\en
\NOtes\qa i\en
\bar % 3
\NOTes\guitar {Fm$^7$}{2}x----x\gdot24\gdot34\gdot43\gdot54\ha h\en
\Notes\qlp n\en
\Notes\isluru0m\ca m\en
\bar % 4
\NOTEs\tslur0m\wh m\en
\bar % 5
\NOtes\guitar {Dm$^{7\Red\B\normalcolor5}$}{2}x----x\gdot23\gdot34\gdot43\gdot54\qlp m\en
\Notes\ca{lk}\qa j\en
\Notes\isluru0i\ca i\en
\bar % 6
\NOTEs\guitar {G$^{7\B9}$}{2}x----x\gdot23\gdot34\gdot42\gdot54\tslur0i\hlp i\en
\Notes\qa h\en
\bar % 7
\NOTes\guitar {Cm$^7$}{2}x-----\gbarre1\gdot33\gdot52\ha g\en
\Notes\qlp m\isluru0l\ca l\en
\bar % 8
\NOTEs\tslur0l\wh l\en
\endpiece
\end{music}
\end{beispiel}

\end{document}