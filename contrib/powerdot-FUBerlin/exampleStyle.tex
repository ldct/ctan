\documentclass[style=BerlinFU]{powerdot}
\pagestyle{empty}
\setcounter{page}{6}
\usepackage[utf8]{inputenc}
\begin{document}
\title{Wissenschaftliches Schreiben}
\subtitle{Conference on Fabulous Presentations, 2009}
\author{Dipl. Frank.-Wiss. Jana Voß\inst{1} \quad Dr.-Ing. Herbert Voß\inst{2}}
\institute{\inst{1}BTO \quad \inst{2}ZEDAT}
\date{\today}
\titlegraphic{silberlaube2}
\fachbereich{ZEDAT}
\maketitle

\begin{slide}[toc=]{Inhaltsverzeichnis}
\tableofcontents
\end{slide}

\section{Eine Einführung}
\subsection{Die Vorlage}
\begin{slide}{Die \LaTeX-Vorlage für powerdot}
\begin{itemize}[type=1]
\item Die Vorlage folgt prinzipiell dem CD der FU
\item Im Gegensatz zu \texttt{PowerPoint} leichter erweiterbar
\item Bessere Steuerung von Inhaltsverzeichnissen
\end{itemize}
\end{slide}
\end{document}
