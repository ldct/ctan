\documentclass{article}
\usepackage{verbatim}
\begin{document}
%% This is an example file used with the nameauth package.
%% See README and nameauth.pdf for copyright info.
%%

\section{Test for Latex Engine}
%%%%%%%%%%%%%%%%%%%%%%%%%%%%%%%%%%%%%%%%%%%%%%%%%%%%%%%%%%%%%
%%     Test which LaTeX engine you are using
%%%%%%%%%%%%%%%%%%%%%%%%%%%%%%%%%%%%%%%%%%%%%%%%%%%%%%%%%%%%%
\begin{verbatim}
\ifdefined\Umathchar
  \usepackage{fontspec}
  \usepackage{polyglossia}
\else
  \usepackage[utf8]{inputenc}
  \usepackage[TS1,T1]{fontenc}
  \usepackage{babel}
\fi
% Below is optional; use only if your dvi viewer|\\
% crashes or becomes unresponsive with tikz.|\\
\usepackage{ifxetex}
\usepackage{ifluatex}
\usepackage{ifpdf}
\ifxetex
  \usepackage{tikz}
\else
  \ifpdf
    \usepackage{tikz}
  \fi
\fi
\end{verbatim}

\section{Use Different Latex Engines}
%%%%%%%%%%%%%%%%%%%%%%%%%%%%%%%%%%%%%%%%%%%%%%%%%%%%%%%%%%%%%
%%     Allow for multiple LaTeX engines
%%%%%%%%%%%%%%%%%%%%%%%%%%%%%%%%%%%%%%%%%%%%%%%%%%%%%%%%%%%%%
Requires \textsf{ifxetex}, \textsf{ifluatex}, and \textsf{ifpdf}.
\begin{verbatim}
\ifxetex
  xelatex %
\else
  \ifluatex
    \ifpdf
      lualatex in pdf mode %
    \else
      lualatex in dvi mode %
    \fi
  \else
    \ifpdf
      pdflatex %
    \else
      latex %
    \fi
  \fi
\fi
\end{verbatim}

%%%%%%%%%%%%%%%%%%%%%%%%%%%%%%%%%%%%%%%%%%%%%%%%%%%%%%%%%%%%%
%%     Print the first use of a name in text and margin
%%%%%%%%%%%%%%%%%%%%%%%%%%%%%%%%%%%%%%%%%%%%%%%%%%%%%%%%%%%%%
%%
%% Below we print the argument both in the text and in a margin paragraph
%% unless we are in internal vertical mode.
%%
%% This macro is suitable to replace \NamesFormat and \FrontNamesFormat.
%%
\section{First Use in Margin: $\epsilon$-\TeX}
\begin{verbatim}
\renewcommand*\NamesFormat[1]%
{%
  #1%
  \unless\ifinner
    \marginpar{\raggedleft\scriptsize #1}%
  \fi
}
\end{verbatim}

\section{First Use in Margin: Historic \TeX}
\begin{verbatim}
\renewcommand*\NamesFormat[1]%
{%
  #1%
  \ifinner\else
    \marginpar{\raggedleft\scriptsize #1}%
  \fi
}
\end{verbatim}

%%%%%%%%%%%%%%%%%%%%%%%%%%%%%%%%%%%%%%%%%%%%%%%%%%%%%%%%%%%%%
%%     Print a text tag after first name use
%%%%%%%%%%%%%%%%%%%%%%%%%%%%%%%%%%%%%%%%%%%%%%%%%%%%%%%%%%%%%
%%
%% Below we print the argument in the text and then we query the text tag
%% database to see if we can print a text tag after the argument.
%%
%% This macro is suitable to replace \NamesFormat and \FrontNamesFormat.
%%
\section{Text Tag With First Use: $\epsilon$-\TeX}
\begin{verbatim}
\newif\ifNoTag
\makeatletter
\renewcommand*\NamesFormat[1]{\begingroup%
  \protected@edef\temp{\endgroup\textbf{#1}%
  \unless\ifNoTag
    \if@nameauth@InName
      {\bfseries\noexpand\NameQueryInfo
      [\unexpanded\expandafter{\the\@nameauth@toksa}]
      {\unexpanded\expandafter{\the\@nameauth@toksb}}
      [\unexpanded\expandafter{\the\@nameauth@toksc}]}\fi
    \if@nameauth@InAKA
      \noexpand\NameQueryInfo
      [\unexpanded\expandafter{\the\@nameauth@toksa}]
      {\unexpanded\expandafter{\the\@nameauth@toksb}}
      [\unexpanded\expandafter{\the\@nameauth@toksc}]\fi
  \fi}\temp\global\NoTagfalse}
\makeatother
\end{verbatim}
\clearpage

\section{Text Tag With First Use: Historic \TeX}
\begin{verbatim}
\newif\ifNoTag
\makeatletter
\renewcommand*\NamesFormat[1]{%
  \let\ex\expandafter%
  #1%
  \if@nameauth@InName
    \ifNoTag
    \else
      \ex\ex\ex\ex\ex\ex\ex\NameQueryInfo\ex\ex\ex\ex\ex\ex\ex[%
        \ex\ex\ex\the\ex\ex\ex\@nameauth@toksa\ex\ex\ex]%
        \ex\ex\ex{\ex\the\ex\@nameauth@toksb\ex}\ex[\the\@nameauth@toksc]%
    \fi\fi
  \if@nameauth@InAKA
    \ifNoTag
    \else
      \ex\ex\ex\ex\ex\ex\ex\NameQueryInfo\ex\ex\ex\ex\ex\ex\ex[%
        \ex\ex\ex\the\ex\ex\ex\@nameauth@toksa\ex\ex\ex]%
        \ex\ex\ex{\ex\the\ex\@nameauth@toksb\ex}\ex[\the\@nameauth@toksc]%
    \fi\fi
  \global\NoTagfalse}
\makeatother
\end{verbatim}

%%%%%%%%%%%%%%%%%%%%%%%%%%%%%%%%%%%%%%%%%%%%%%%%%%%%%%%%%%%%%
%%     Surname index entries in an fbox
%%     First surname instances in text are in an fbox
%%%%%%%%%%%%%%%%%%%%%%%%%%%%%%%%%%%%%%%%%%%%%%%%%%%%%%%%%%%%%
%%
%% We create a macro \Fbox that prints its argument in a framed
%% box when \@nameauth@DoAlt is true, or it just prints its argument.
%%
%%
\section{Formatting and Capping: New Style}
\begin{verbatim}
\makeatletter
\newcommand*\Fbox[1]{%
  \if@nameauth@DoAlt
    \fbox{#1}\else#1\fi
}
\makeatother

\renewcommand*\MainNameHook[1]{\NameOnly\NameParser}

\let\FrontNameHook\MainNameHook
\end{verbatim}
\clearpage

%%%%%%%%%%%%%%%%%%%%%%%%%%%%%%%%%%%%%%%%%%%%%%%%%%%%%%%%%%%%%
%%     Use both the arguments passed to the hooks
%%     and \NameParser under different conditions
%%%%%%%%%%%%%%%%%%%%%%%%%%%%%%%%%%%%%%%%%%%%%%%%%%%%%%%%%%%%%
%%
%% We redefine the hooks to print a name in the text and in
%% a margin paragraph. We change some of the internal flags
%% to make \NameParser print the name differently.
%%
%%
\section{Putting \texttt{\textbackslash NameParser} on Display}
\begin{verbatim}
\makeatletter
\renewcommand*\NamesFormat[1]{%
  #1\unless\ifinner
    \marginpar{\small\raggedleft%
               \@nameauth@FullNametrue%
               \@nameauth@FirstNamefalse%
               \@nameauth@EastFNfalse%
               \NameParser}%
  \fi}
\renewcommand*\MainNameHook[1]{%
  \AltOff#1\unless\ifinner
    \marginpar{\small\raggedleft%
               \@nameauth@FullNamefalse%
               \@nameauth@FirstNamefalse%
               \@nameauth@EastFNfalse%
               \NameParser}%
  \fi}
\makeatother
\end{verbatim}
\clearpage

%%%%%%%%%%%%%%%%%%%%%%%%%%%%%%%%%%%%%%%%%%%%%%%%%%%%%%%%%%%%%
%%     Surname index entries in an fbox
%%     First surname instances in text are in an fbox
%%     We can capitalize name within that formatting
%%%%%%%%%%%%%%%%%%%%%%%%%%%%%%%%%%%%%%%%%%%%%%%%%%%%%%%%%%%%%
%%
%% Below we create a Boolean value \ifFbox and set it true.
%% This will trigger an fbox
%%
%% We then create a macro \Fbox that prints its argument in 
%% an fbox when \ifFbox is true, or makes no change otherwise.
%%
%% The \AltCaps macro only capitalizies its argument inside the
%% formatting hook \NamesFormat below.
%%
%% \Namesformat ignores its argument, sets \InHooktrue, then
%% calls the name parser used specifically in formatting hooks.
%% \MainNameHook toggles \Fboxfalse to suppress formatting.
%%
%%
\section{Formatting and Capping: Old Style}
\begin{verbatim}
\newif\ifFbox
\newif\ifFirstCap
\newif\ifInHook
\Fboxtrue

\renewcommand*\Fbox[1]{%
  \ifFbox\fbox{#1}\else#1\fi
}

\renewcommand*\AltCaps[1]{%
  \ifInHook
    \ifFirstCap\MakeUppercase{#1}\else#1\fi
  \else
    #1%
  \fi
}

\renewcommand\CapThis{\FirstCaptrue}

\renewcommand*\NamesFormat[1]
{%
  \InHooktrue\NameParser\InHookfalse%
  \global\FirstCapfalse%
}

\renewcommand*\MainNameHook[1]
{%
  \Fboxfalse\InHooktrue\NameParser\InHookfalse%
  \global\FirstCapfalse\Fboxtrue%
}

\let\FrontNamesFormat\Namesformat
\let\FrontNameHook\MainNameHook
\end{verbatim}
\clearpage

%%%%%%%%%%%%%%%%%%%%%%%%%%%%%%%%%%%%%%%%%%%%%%%%%%%%%%%%%%%%%
%%     Migrate new style to old style
%%%%%%%%%%%%%%%%%%%%%%%%%%%%%%%%%%%%%%%%%%%%%%%%%%%%%%%%%%%%%
%%
%% This is the full example abbreviated in the manual.
%% Below we create a Boolean value \ifCaps and set it true.
%% This will trigger small caps.
%%
%% We then redefine \textSC to use our flag instead of package internals.
%% We redefine \AltCaps, \CapThis, and \NamesFormat in the same manner
%% as above. We change \MainNameHook to use our Caps flag.
%%
%%
\section{Migrating New Style to Old Style}
\begin{verbatim}
\newif\ifCaps
\newif\ifFirstCap
\newif\ifInHook
\Capstrue

\renewcommand*\textSC[1]{%
  \ifCaps\textsc{#1}\else#1\fi
}

\renewcommand*\AltCaps[1]{%
  \ifInHook
    \ifFirstCap\MakeUppercase{#1}\else#1\fi
  \else
    #1%
  \fi
}

\renewcommand\CapThis{\FirstCaptrue}

\renewcommand*\NamesFormat[1]
{%
  \InHooktrue\NameParser\InHookfalse%
  \global\FirstCapfalse%
}

\renewcommand*\MainNameHook[1]
{%
  \Capsfalse\InHooktrue\NameParser\InHookfalse%
  \global\FirstCapfalse\Capstrue%
}

\let\FrontNamesFormat\Namesformat
\let\FrontNameHook\MainNameHook
\end{verbatim}
\end{document}