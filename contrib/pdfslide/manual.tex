\documentclass[a4paper]{article}
\usepackage{lucidabr}
\usepackage{amsmath}
\usepackage[contnav]{pdfslide}
\usepackage{pause}
\pagestyle{title}

\begin{document}

\orgname{Indian \TeX{} Users Group}
\orgurl{\protect\color{white}http://www.river-valley.com/tug/}

\title{\scalebox{1}[1.3]{Presentations with pdf\TeX}}
\author{\scalebox{1}[1.3]{C.~V.~Radhakrishnan}}
\address{River Valley Technologies\\
  {\realfootnotesize(Development Division of Focal Image (India) Pvt. Ltd.)}\\
              Software Technology Park,
              Trivandrum, India\\
              email: {\tt cvr@river-valley.com}}

\notes{This is for public consumption and for release to
       Comprehensive \TeX{} Archive Network} 

\overlay{d12.jpg}
\maketitle

\pagedissolve{Wipe /D 3 /Dm /V /M /O}
\overlay{metablue.pdf}
\sffamily\large
\color{section0}

\section{Objectives}
\begin{itemize}

\item to devise a method for easier technical presentation.\pause

\item to help the mix of mathematical formulae with text and graphics
which the present day \textsc{wysiwyg} tools fail to\linebreak accomplish.\pause

\item to exploit the platform independence of \TeX{} so that
presentation documents become portable.\pause

\item to offer the freedom and possibilities of using various
backgrounds and other embellishments that a user can imagine to have in
his presentation.

\end{itemize}

\headskip=48pt
\section{Methodology}
\pagedissolve{Glitter /D 3 /Di /V /M /O}
\overlay{metalgray.pdf}

\color{section2}
\begin{enumerate}

\item Make a \LaTeX{} document in the usual way.\pause

\item In the preamble load the slide package with the command\\
\color{red}\verb+         \usepackage{pdfslide}+\color{section2}

This shall be loaded as the last package.
You must have \textcolor{red}{hyperref.sty} version 6.60 or above
installed in your system.\pause

\item Run pdf\LaTeX{} over your document and you get the pdf output
which you can view with Acrobat Reader.
\end{enumerate}


\section{Backgrounds}
\overlay{bg.jpg}
\pagedissolve{Wipe /D 1 /Di /H /M /O}
\color{white}
\begin{itemize}
\item You can create your own background graphics with your preferred
packages and include them using the command\\
\color{yellow}
\verb+\overlay{+\textless\emph{filename}\textgreater\verb+}+%
\color{white}.\pause

\item pdf\TeX{} supports three graphic file formats \emph{viz.},
\textcolor{yellow}{\tt pdf, jpeg, png}. If you have \textcolor{yellow}{\tt eps} files
you can distill them with Acrobat Distiller or use Ghostscript to
convert them to pdf format.\pause

\item You can change backgrounds for each page, there is no limit.

\end{itemize}

\section{Page Transition}
\pagedissolve{Wipe /D 1 /Di /H /M /O}
\overlay{bg.jpg}
\color{white}
\begin{itemize}
\item You can exploit the page transition facilities in the Acrobat.
Specify your choice by using the command\\ \color{yellow}
\verb+\pagedissolve{+\textless\emph{option}\textgreater\verb+}+%
\color{white}\pause

\item A list of page dissolve options and keys are given in the user
manual of \textcolor{yellow}{\tt pdfslide.sty}.

\end{itemize}

\headskip30pt
\section{Some Math Equations}
\pagedissolve{Wipe /D 1 /Di /H /M /O}
\overlay{mpgraph.pdf}
\color{section1}

\begin{align}
\begin{split}
|I_l| &=\left|\int_\Omega gRu\,\Omega\right|\\ \pause
      &\le C_3\left[\int_\Omega \left(\int_{a}^x
        g(xi,t)\, d\xi\right)^2d \Omega\right]^{1/2}\\ \pause
      &\quad\times\left[\int_\Omega \left\{u^2_x+\frac{1}{k}
      \left(\int_{a}^x cu_t \, d\xi \right)^2\right\}
      c \Omega\right]^{1/2} \\ \pause
      &\le C_4\left|\left| f\left|\widetilde{S}^{-1,0}_{a,-}
       W_2(\Omega,\Gamma_1) \right|\right|
       \left||u|\overset{\circ}\to W_2^{\widetilde{A}}
       \Omega;\Gamma_r,T)\right|\right|.
       \end{split}
\end{align}

\headskip0pt
\section{Extra facilities}
\pagedissolve{Wipe /D 1 /Di /H /M /O}
\overlay{metalgray.pdf}
\color{black}\rmfamily
\realnormalsize
\subsection{Fonts}
All the font attributes have been redefined to make them larger than
the usual size. However, if you want to revert to the original size,
you will have to add the word \verb+real+ before the font size command,
i.e., for \verb+\normalsize+, use \verb+\realnormalsize+; for
\verb+\large+ it is \verb+\reallarge+ and so forth.

\subsection{Headings}
\verb+\section{...}+ may be used for first level heading for the
slides. If you need more skip before the heading so as to make the
whole matter vertically centered, you can change the dimension with the
command, \verb+\headskip={<+new dimension\verb+>}+. This command shall
be placed before the section heading and shall be reset at the end of
the current slide, if you do not want the current skip further.

\subsection{Post-processing}
The postprocessor \emph{viz.}, \verb+PPower4+ can be applied to the pdf
generated with this package, so that incremental additions to the pages
are possible. \verb+PPower4+ is available at
\href{ftp://ftp.dante.de/support/PPower4}{\textsc{ctan}}. You may need Java
Virtual Machine running in your system to work with \verb+PPower4+.

\subsection{Page Transition}

Following portion from the well known book, \emph{Web Publishing with
Acrobat/PDF} by Thomas Merz will largely help to know the options for
\verb+\pagedissolve+ function. 

\definecolor{gray9}{rgb}{1,.894,.769}
\subsubsection{Keys for page transitions}
\sffamily\itshape
\setlength\arrayrulewidth{0pt}
\def\dash{\noalign{\vskip1.5pt}\hline\noalign{\vskip1.5pt}}
$$\begin{tabular}{@{}p{1in}p{5in}@{}}
\rowcolor{section1}Key                           & Explanation\\\dash
\rowcolor{gray9} /Split          & Two lines sweep across the screen to reveal
                                   the new page similar to opening a curtain.\\\dash
\rowcolor{buttondisable} /Blinds & Similar to /Split, but with several lines
                                   resembling  ``venetian blinds''\\\dash
\rowcolor{gray9} /Box            & A box enlarges from the center of the old
                                   page to reveal the new one.\\\dash
\rowcolor{buttondisable} /Wipe   & A single line ``wipes'' across the old page
                                   to reveal the new  one.\\\dash
\rowcolor{gray9} /Dissolve       & The old page ``dissolves'' to reveal the
                                   new one.\\\dash
\rowcolor{buttondisable} /Glitter& Similar to /Dissolve, except the effect
                                   sweeps from one edge to another.\\\dash
\rowcolor{gray9} /R (Replace)    & The old page is simply replaced with
                                   the new one without any special effect.
                                   This is the default. 
\end{tabular}$$
\rmfamily\normalfont
For some of the transitions additional parameters may be specified. The
following code results in a split effect with the lines moving
horizontally (/H) from the inner parts of the page to the outer parts
(/O). The duration of the effect is two seconds (/D):
\begin{verbatim}
         /S /Split /D 2 /Dm /H /M /O 
\end{verbatim}
Given below are all supported parameters for page dissolve, along with
the kind of transition on which the parameters may be applied.

\subsubsection{Additional parameters for page transitions}
\sffamily\itshape
$$\begin{tabular}{@{}p{1in}p{5in}@{}}
\rowcolor{section1}Key           & Explanation\\ \dash
\rowcolor{gray9} /D              & Duration of the transition effect in
                                   seconds (applies to all effects)\\ \dash
\rowcolor{buttondisable} /Di\hfill\break
                (Direction)      & Direction of the movement (multiples of
                                   90$^\circ$ only). Values increase in a
                                   counterclockwise fashion, 0$^\circ$
                                   points to the  right (for /Wipe and
                                   /Glitter).\\\dash 
\rowcolor{gray9} /Dm\hfill\break
                   (Dimension)   & Possible values are /H or /V for a
                                   horizontal or  vertical effect,
                                   respectively (for /Split and
                                   /Blinds).\\\dash  
\rowcolor{buttondisable} /M \hfill\break
                  (Motion)       & Specifies whether the effect is
                                   performed from  the center out or
                                   the edges in. Possible values  are
                                   /I for in and /O for out (for /Split
                                   and /Box).  
\end{tabular}$$
\rmfamily\normalfont

\subsection{Navigation buttons}
There are two options available \emph{viz}., \verb+contnav+ and
\verb+ams+. For the first option to work smoothly,
you need to have the \verb+contnav+ fonts installed in your \TeX{}
system to have the navigation buttons in the side panel. This font is
part of the Con\TeX{}t macro package. Navigation buttons from
top to bottom correspond to first page, previous page, next page, last
page, go back, go forward and close file respectively. The button below
the pine tree indicate the current slide number and clicking at it will
open you the Acrobat window to go to a particular slide.

\TeX{}Live4 has \verb+contnav+ fonts installed, but will have to add
the following line to the \verb+hoekwater.map+
\begin{verbatim}
       contnav  ContextNavigation    <contnav.pfb
\end{verbatim}
so that it will be available to the pdf\TeX.

If \verb+contnav+ font is not available in your \TeX{} distribution,
you can choose the option \verb+ams+ so that AMS fonts will be used for
the navigation buttons.

\subsection{\tt pdfslide.cfg}
There is a configuration file, \verb+pdfslide.cfg+ in which you can
change all the font attributes for caption, title, author, address, notes,
all section level headings, change the side panel color, button color,
button disabled color, button background, button shadow color and  button
text (for languages other than English).

If you don't want \verb+pdfslide.sty+ to read
the configuration file, you can invoke \verb+nocfg+ as an option to the
package loading command.

\subsubsection{Acknowledgements}
Various users of \verb+pdfscreen.sty+ have asked me to write a slide
package and I drew considerable quantum of inspiration from all those
who send me kind words of compliments for the \verb+pdfscreen.sty+.
My special thanks are to Kaveh Bazargan of Focal Image Ltd., London and
Sebastian Rahtz of Oxford University Computing Services who were
instrumental in making this package a reality. 

\subsubsection{Bug Reports:}

Punch me with your bug reports and I am available at:
\begin{center}
\href{mailto:cvr@river-valley.com}{{\color{red}\tt
cvr@river-valley.com}}.
\end{center}
\end{document}
                 