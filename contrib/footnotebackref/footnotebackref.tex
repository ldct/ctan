\documentclass{article}
\usepackage{footnotebackref}

\begin{document}

\title{The \texttt{footnotebackref} package\\\large Back referencing from footnotes}%
\author{Holger Lange}%
\date{\today}%
\maketitle

\section{Introduction}
The \texttt{footnotebackref} package allows the creation of hyperlinks from the footnote at the bottom of the page back to the occurence of the footnote in the main text. The references and hyperlinks are created by the \texttt{hyperref} package.

\section{Usage and Options}
To use the package just type \verb+\usepackage{footnotebackref}+ in the preamble of the document.
There are two options to format the links:
\begin{quote}
    \begin{tabular}{lp{0.33\paperwidth}}
        \texttt{symbol}       & Hyperlinked symbol between the footnotenumber and the footnotetext (e.\,g. $\uparrow$, $\bigtriangleup$ or $\wedge$).\newline Default: \emph{no symbol}\\
        \texttt{numberlinked} & This boolean indicates whether the footnotenumber is also a hyperlink or not.\newline Default: true
    \end{tabular}
\end{quote}
The package options are specified as key value pairs (\texttt{optionname=value}). If they are omitted the default values are used.\footnote{The default behaviour is like this footnote: There is no additionally symbol and the footnotenumber is a hyperlink to the occurence of the footnote in the main text.}

\section{Example}
The example shows an included symbol as hyperlink.
\begin{verbatim}
\documentclass{article}
\usepackage[symbol=$\wedge$,numberlinked=false]{footnotebackref}
\begin{document}
Some text with a footnote\footnote{The first footnote.}\\
Text with the second footnote\footnote[4]{The second footnote.}
\end{document}
\end{verbatim}
\end{document}
