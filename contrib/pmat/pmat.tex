\documentclass[a4paper]{article}
\usepackage{pmat}

\begin{document}

\title{The \texttt{pmat} package to typeset partitioned matrices}
\author{Tom\'as Oliveira e Silva \\[4pt] Universidade de Aveiro \\
  Aveiro, Portugal \\[8pt] email: \texttt{tos@inesca.pt}}
\date{January 1, 1996}
\maketitle

This document is a very short description of the \texttt{pmat} package,
version $0.1$. The \texttt{pmat} package provides the macros \verb+\pmat+
and \verb+\endpmat+ to typeset partitioned matrices, and it works under
\TeX\ and under \LaTeX\ (any version). Under \LaTeX\ it is possible to replace
\verb+\pmat+ and \verb+\endpmat+ by \verb+\begin{pmat}+ and \verb+\end{pmat}+,
respectively.

The \verb+\pmat+ macro takes three arguments. The first one is a left
delimiter (the thing you put immediately after a \verb+\left+ command). 
The last one is a right delimiter (the thing you put immediately after a
\verb+\right+ command). As usual, a delimiter may be omitted by using a dot
(.). The middle argument specifies the dashed vertical lines that are to be
placed between columns of the matrix. This argument must contain exactly $n-1$
characters, where $n$ is the number of columns of the matrix. If a character
is a \verb+|+ then a dashed vertical line will be placed between the
appropriate columns. Otherwise, no dashed line will be placed between those
columns (we recommend the use of the character dot (.) in these cases) . The
format of the entries of the partitioned matrix follows the conventions of
plain \TeX, i.e., entries are separated by a \verb+&+ (just like in \LaTeX),
but lines are separated by a \verb+\cr+ (instead of the \verb+\\+ used in
\LaTeX). All entries are typeset in math mode (in \verb+\textstyle+). For
technical reasons, a \verb+\cr+ must also be placed at the end of the last
line. The placement of horizontal dashed lines is done with the command
\verb+\-+, which must be placed immediately after the command \verb+\cr+. For
example, the code
\begin{verbatim}
\[
  \begin{pmat}[{.|}]
    a_{11} & a_{12} & b_{11} \cr
    a_{21} & a_{22} & b_{21} \cr\-
    c_{11} & c_{12} & d_{11} \cr
  \end{pmat}
\]
\end{verbatim}
produces the matrix
\[
  \begin{pmat}[{.|}]
    a_{11} & a_{12} & b_{11} \cr
    a_{21} & a_{22} & b_{21} \cr\-
    c_{11} & c_{12} & d_{11} \cr
  \end{pmat}
\]

There are two macros that can be used to modify the shape of the crossings
between horizontal and vertical dash lines. The first one, \verb+\pmatcross+,
makes horizontal and vertical dash lines to cross. (It is the default.) The
other, \verb+\pmatnocross+, makes the lines not to cross. Under \LaTeX2e, the
first one can be made the default (it already is) by passing the option
\verb+cross+ to the package, and the second one can be made the default by
passing the option \verb+nocross+ to the package. The difference between the
two forms can be examined in the following example (before the second
\verb+\begin{pmat}+ we have put a \verb+\pmatnocross+)
\[
  \begin{pmat}[{.|}]
    a_{11} & a_{12} & b_{11} \cr
    a_{21} & a_{22} & b_{21} \cr\-
    c_{11} & c_{12} & d_{11} \cr
  \end{pmat}
  \qquad\qquad
  \pmatnocross\begin{pmat}[{.|}]
    a_{11} & a_{12} & b_{11} \cr
    a_{21} & a_{22} & b_{21} \cr\-
    c_{11} & c_{12} & d_{11} \cr
  \end{pmat}
\]

There are seven internal parameters that control the inner workings of the
\verb+pmat+ macros. To hide their names they can only be accessed or changed
through the macros \verb+\pmatget+ and \verb+\pmatset+. The \verb+\pmatget+
macro takes one argument, the number of the parameter (from 0 to 6). It
``returns'' the current value of that parameter. The \verb+\pmatset+ macro
takes two arguments, the number of the parameter, and its new value. All
parameters are dimensions, so do not forget to put a unit of length (such as
\verb+mm+) after its numerical value. The seven parameters are the following:
\begin{description}
  \item[0] This parameter controls the length of a dash ``unit''. All
    horizontal and vertical dash lines have a length that is an integer
    multiple of this parameter. Its default value is \verb+2.88pt+.
  \item[1] This parameter controls the thickness of the dash lines. Its
    default value is \verb+0.36pt+.
  \item[2] This parameter controls the minimum distance between two lines. Its
    default value is \verb+2pt+.
  \item[3] This parameter controls the minimum distance between two columns.
    Its default value is \verb+8pt+.
  \item[4] This parameter controls the minimum width of each entry of the
    matrix. Its default value is \verb+0pt+.
  \item[5] This parameter controls the minimum height of each entry of the
    matrix. Its default value is \verb+6pt+.
  \item[6] This parameter controls the minimum depth of each entry of the
    matrix. Its default value is \verb+1.5pt+.
\end{description}
Due to the way lengths are quantized, to avoid larger than strictly necessary
gaps between lines the values of $[5]+([1]+[2])/2$ and $[6]+([1]+[2])/2$
should be smaller but close to a multiple of $[0]$. (A number inside square
brackets refers to the corresponding parameter.) For example, the code
\begin{verbatim}
\[
  \pmatset{1}{0.72pt}
  \pmatset{0}{2\pmatget{1}}
  \pmatset{5}{0pt}
  \pmatset{6}{0pt}
  \begin{pmat}[{||}]
    A_{11} & A_{12} & A_{13} \cr\-
    A_{21} & A_{22} & A_{23} \cr\-
    A_{31} & A_{32} & A_{33} \cr
  \end{pmat}
\]
\end{verbatim}
produces the matrix
\[
  \pmatset{1}{0.72pt}
  \pmatset{0}{2\pmatget{1}}
  \pmatset{5}{0pt}
  \pmatset{6}{0pt}
  \begin{pmat}[{||}]
    A_{11} & A_{12} & A_{13} \cr\-
    A_{21} & A_{22} & A_{23} \cr\-
    A_{31} & A_{32} & A_{33} \cr
  \end{pmat}
\]

For a uniform look of the document all matrices, partitioned and not
partitioned, should be typeset using the \verb+pmat+ macros.

\bigskip
{\bfseries Possible future improvements:} Instead of quantizing the height and
depth of each entry separately, maybe their sum should be quantized. Perhaps
the minimum height and depth of each entry should depend upon the current font
size.

\bigskip
\hrule height 1pt
\[
  \begin{pmat}[{|}]
    \mathbf{A} & \mathbf{B} \cr\-
    \mathbf{C} & \mathbf{D} \cr
  \end{pmat} = \begin{pmat}[{.|}]
    -k_1 & -k_2\sqrt{1-k^2_1} & \sqrt{(1-k^2_1)(1-k^2_2)} \cr
    \sqrt{1-k^2_1} & -k_1k_2 & k_1\sqrt{1-k^2_2} \cr\-
    0 & \sqrt{1-k^2_2} & k_2 \cr
  \end{pmat}
\]


\end{document}
