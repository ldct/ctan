\documentclass[pagesize=auto]{scrartcl}

\usepackage{fixltx2e}
\usepackage{etex}
\usepackage{lmodern}
\usepackage[T1]{fontenc}
\usepackage{textcomp}
\usepackage{microtype}
\usepackage{hyperref}

\newcommand*{\mail}[1]{\href{mailto:#1}{\texttt{#1}}}
\newcommand*{\pkg}[1]{\textsf{#1}}

\areaset{14cm}{22.5cm}
\addtokomafont{title}{\rmfamily}

\title{The \pkg{dpfloat} package}
\subtitle{Formatting double-page floats in \LaTeX}
\author{Jim Fox, \mail{fox@washington.edu}}
\date{2006/10/05}


\begin{document}

\maketitle

\noindent
You sometimes want to format two, full-page figures or tables to appear
side-by-side in an opened book. The first figure must be on the left (even
page) and the second figure must appear on the right (odd page).

\LaTeX\ provides no way to assure that two consecutive figures will come out
this way. This double-page float package, \pkg{dpfloat.sty}, allows you to specify
that a float must appear on an even page, and thereby allows you to format a
double-page figure.

\bigskip

You format a double-page figure by formatting two consecutive, full-page
figures, specifying that the first must appear on a left page. The other
will follow on the right page.

It looks something like this:

\small
\noindent
\begin{verbatim}
\documentclass{...}

\usepackage{dpfloat}

\begin{document}

 ...

% a double-page figure

  \begin{figure}[p]% will be the left-side figure
    \begin{leftfullpage}
       This is the left side figure
    \end{leftfullpage}
  \end{figure}
  \begin{figure}[p]% will be the right-side figure
    \begin{fullpage}
       This is the right side figure
    \end{fullpage}
  \end{figure}

% end of the figure

 ...

\end{document}
\end{verbatim}

\end{document}
