\documentclass[compose]{exam-n}
\begin{document}

\begin{question}{30} \comment{by John Brown}
Show by considering the Newtonian rules of vector and velocity
addition that in Newtonian cosmology the cosmological principle
demands Hubble's Law $v_r\propto r$.\partmarks{10}

Prove that, in Euclidean geometry, the number $N(F)$  of objects
of identical luminosity $L$, and of space density  $n(r)$ at
distance $r$, observed with radiation flux $\ge F$  is (neglecting
other selection and redshift effects)
\begin{equation*}
N(F)=4\pi\int_0^{(\frac{L}{4\pi F})^{1/2}} n(r) r^2\ddd r.
\end{equation*}
\partmarks*{5}

Use this to show that for  $n=n_1=$constant at $r<r_1$ and
$n=n_2=$constant at $r>r_1$,
\begin{equation*}
N(F) = N_1\left(\frac{F}{F_1}\right)^{-3/2}\qquad \text{for
$F>F_1$},
\end{equation*}
and
\ifbigfont
  \begin{multline*}
  N(F) =
  N_1\left\{1+\frac{n_2}{n_1}\left[\left(\frac{F}{F_1}\right)^{-3/2}-1\right]
  \right\}\\\text{for $F<F_1$},
  \end{multline*}
\else
  \begin{equation*}
  N(F) =
  N_1\left\{1+\frac{n_2}{n_1}\left[\left(\frac{F}{F_1}\right)^{-3/2}-1\right]
  \right\} \qquad \text{for $F<F_1$},
  \end{equation*}
\fi
where $F_1=L/4\pi r_1^2$, $N_1=N(F_1)=\frac{4}{3}\pi r_1^3 n_1$.
\partmarks{9}

Reduce these two expressions to the result for a completely
uniform density universe with  $n_1=n_2=n_0$.
\partmarks{3}

% An itemized list followed by partmarks*
Sketch how $n(F)$  would look in universes which are
\begin{itemize}
\item flat,
\item open,
\item and closed.
\end{itemize}
\partmarks*{3}

\begin{solution}
A sufficiently heavy weight will reduce expressions to completely
uniform sheets of paper if it is placed on top of them.\partmarks3
In a flat universe, $n(F)$ will look like n(F).\partmarks*3
\end{solution}
\end{question}
\end{document}
