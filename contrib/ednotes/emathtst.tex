%% EMathTst.tex for demonstrating EdnMath0.sty. 
%% Uwe L"uck, http://www.contact-ednotes.sty.de.vu, 2005/01/16. 
\documentclass[12pt]{article}
\def\Anote#1#2{#1} \let\<\relax \let\>\relax 
% \usepackage{ednotes}
\usepackage[mathrefs]{lineno}

\begin{document}

\begin{linenumbers}
% \begin{NoNotesToMath} 
\linenumberdisplaymath 
\noindent 
\texttt{\string\ \unskip linelabel} and 
\texttt{\string\ \unskip Anote} in math mode? 
\begin{linenomath}
$$ 
% \begin{displaymath}
% \begin{equation}
x=\Anote{z}{$Z$}
% {\makeatletter \typeout{\meaning\@EN@lemmatag}}
\linelabel{Hm}
% \end{equation}
% \end{displaymath}
$$ 
\end{linenomath}
% \end{NoNotesToMath} 
$a=\Anote{x\<y\>z}{$b$}
% {\makeatletter \typeout{\meaning\@EN@lemmatag}}
\linelabel{Hmm}
$. 
Why not? 
\begin{NoNotesToMath}
Line No.~\ref{Hm}. 
\Anote{No \<math\> here.}{Went wrong earlier.}
% {\makeatletter \typeout{\meaning\@EN@lemmatag}}
\end{NoNotesToMath}
Line No.~\ref{Hmm}.
\end{linenumbers}

% \tracingonline=1 
% \showboxdepth=1
% \showboxbreadth=30
% \showlists

% % These really get lost: 
% \linenumbers 
% \mbox{\Anote{Boxed}{where?}.\linelabel{Hu}} 
% Line No.~\ref{Hu}.

\end{document}


