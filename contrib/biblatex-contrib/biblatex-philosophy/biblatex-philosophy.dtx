% \iffalse meta-comment
%<*internal>
\begingroup
\input docstrip.tex
\keepsilent
\preamble
______________________________________________________
The biblatex-philosophy package 
Copyright (C) 2009-2017 Ivan Valbusa 
All rights reserved

License information appended

\endpreamble
\postamble

Copyright 2009-2017 by Ivan Valbusa

This program is provided under the terms of the
LaTeX Project Public License distributed from CTAN
archives in directory macros/latex/base/lppl.txt.

Author: Ivan Valbusa
ivan dot valbusa at univr dot it

This work has the LPPL maintenance status "author-maintained".

\endpostamble
\askforoverwritefalse

\Msg{*** Generating the class file ***}
\generate{
  \file{philosophy-standard.bbx}{\from{biblatex-philosophy.dtx}{standard-bbx}}
  \file{philosophy-classic.bbx}{\from{biblatex-philosophy.dtx}{classic-bbx}}
  \file{philosophy-modern.bbx}{\from{biblatex-philosophy.dtx}{modern-bbx}}
  \file{philosophy-verbose.bbx}{\from{biblatex-philosophy.dtx}{verbose-bbx}}
  \file{philosophy-classic.cbx}{\from{biblatex-philosophy.dtx}{classic-cbx}}
  \file{philosophy-modern.cbx}{\from{biblatex-philosophy.dtx}{modern-cbx}}
  \file{philosophy-verbose.cbx}{\from{biblatex-philosophy.dtx}{verbose-cbx}}
  \file{italian-philosophy.lbx}{\from{biblatex-philosophy.dtx}{italian-lbx}}
  \file{english-philosophy.lbx}{\from{biblatex-philosophy.dtx}{english-lbx}}
  \file{spanish-philosophy.lbx}{\from{biblatex-philosophy.dtx}{spanish-lbx}}
  \file{french-philosophy.lbx}{\from{biblatex-philosophy.dtx}{french-lbx}}
  \file{biblatex-philosophy.bib}{\from{biblatex-philosophy.dtx}{bib}}
}

\Msg{***********************************************************}
\Msg{*}
\Msg{* To finish the installation you have to move the following}
\Msg{* files into a directory searched by TeX:}
\Msg{*}
\Msg{* \space\space philosophy-standard.bbx}
\Msg{* \space\space philosophy-classic.bbx}
\Msg{* \space\space philosophy-modern.bbx}
\Msg{* \space\space philosophy-verbose.bbx}
\Msg{* \space\space philosophy-classic.cbx}
\Msg{* \space\space philosophy-modern.cbx}
\Msg{* \space\space philosophy-verbose.cbx}
\Msg{* \space\space italian-philosophy.lbx}
\Msg{* \space\space english-philosophy.lbx}
\Msg{* \space\space spanish-philosophy.lbx}
\Msg{* \space\space french-philosophy.lbx}
\Msg{*}
\Msg{*}
\Msg{* To produce the documentation run}
\Msg{* the file ending with `.dtx' through (pdf)LaTeX. See the}
\Msg{* README file for more details.}
\Msg{*}
\Msg{* Happy TeXing}
\Msg{***********************************************************}
\endgroup
%</internal>
%
% Copyright (C) 2009-2017 by Ivan Valbusa 
% <ivan dot valbusa at univr dot it>
% -------------------------------------------------------
% 
% This work may be distributed and/or modified under the
% conditions of the LaTeX Project Public License, either version 1.3
% of this license or (at your option) any later version.
% The latest version of this license is in
%   http://www.latex-project.org/lppl.txt
% and version 1.3 or later is part of all distributions of LaTeX
% version 2005/12/01 or later.
%
% \fi
%
% \iffalse
%<*driver>
\ProvidesFile{biblatex-philosophy.dtx}
%</driver>
%<standard-bbx>\ProvidesFile{philosophy-standard.bbx}
%<classic-bbx>\ProvidesFile{philosophy-classic.bbx}
%<modern-bbx>\ProvidesFile{philosophy-modern.bbx}
%<verbose-bbx>\ProvidesFile{philosophy-verbose.bbx}
%<classic-cbx>\ProvidesFile{philosophy-classic.cbx}
%<modern-cbx>\ProvidesFile{philosophy-modern.cbx}
%<verbose-cbx>\ProvidesFile{philosophy-verbose.cbx}
%<italian-lbx>\ProvidesFile{italian-philosophy.lbx}
%<english-lbx>\ProvidesFile{english-philosophy.lbx}
%<spanish-lbx>\ProvidesFile{spanish-philosophy.lbx}
%<french-lbx>\ProvidesFile{french-philosophy.lbx}
%<*standard-bbx|classic-bbx|modern-bbx|verbose-bbx|classic-cbx|modern-cbx|verbose-cbx|italian-lbx|english-lbx|spanish-lbx|french-lbx>
[2017/04/13 v1.9.5 A set of styles for biblatex]
%</standard-bbx|classic-bbx|modern-bbx|verbose-bbx|classic-cbx|modern-cbx|verbose-cbx|italian-lbx|english-lbx|spanish-lbx|french-lbx>
%<*driver>
\documentclass[10pt]{ltxdoc}
\usepackage[T1]{fontenc}
\usepackage[utf8]{inputenc}
\usepackage[english]{babel}
% fonts and colors
\usepackage[osf,p,mono=false]{libertine}
\usepackage[libertine]{newtxmath}
\usepackage[svgnames]{xcolor}
\definecolor{philA}{rgb}{0.5,0,0}
\usepackage{metalogo}
\usepackage{guit}
\usepackage[final]{microtype}
% doc facilities
\let\cs\relax
\let\cmd\relax
\usepackage{ltxdockit}
\usepackage{btxdockit}
% bibliography
\usepackage{csquotes}
\usepackage[style=philosophy-modern,annotation=true]{biblatex}
\addbibresource{biblatex-philosophy.bib}
\addbibresource{biblatex-examples.bib}
\DeclareBibliographyCategory{biblatex}
\addtocategory{biblatex}{set,stdmodel,matuz:doody,vizedom:related,britannica,gaonkar,jaffe,westfahl:frontier,cms,ctan,jcg,yoon,worman,wilde,westfahl:space,weinberg,wassenberg,vazques-de-parga,springer,spiegelberg,sorace,sigfridsson,shore,sarfraz,salam,reese,pines,piccato,padhye,nussbaum,nietzsche:ksa,nietzsche:ksa1,nietzsche:historie,moraux,moore,moore:related,massa,maron,markey,malinowski,loh,laufenberg,kullback,kullback:reprint,kullback:related,kowalik,knuth:ct,knuth:ct:a,knuth:ct:b,knuth:ct:c,knuth:ct:d,knuth:ct:e,knuth:ct:related,kastenholz,kant:kpv,kant:ku,itzhaki,hyman,murray,iliad,herrmann,hammond,companion,gonzalez,glashow,gillies,gerhardt,vangennep,vangennep:trans,vangennep:related,geer,gaonkar:in,doody,cotton,coleridge,cicero,chiu,brandt,bertram,baez/article,baez/online,averroes/bland,averroes/hannes,averroes/hercz,augustine,aristotle:anima,aristotle:physics,aristotle:poetics,aristotle:rhetoric,angenendt,almendro,aksin}

% layout
\setcounter{tocdepth}{3}
\usepackage{geometry}
\geometry{paperwidth=17cm,paperheight=24cm,margin=1.4cm,top=2cm,bottom=2cm,headheight=15pt,ignoreall,heightrounded}
\usepackage{sectsty}
\allsectionsfont{\sffamily}
\usepackage{fancyhdr}
  \fancyhf{}
  \fancyhead[L]{© 2009--\the\year\quad Ivan Valbusa}
  \fancyfoot[L]{\textsf{biblatex-philosophy} 
    \fileversion{} -- \filedate}
  \renewcommand{\footrulewidth}{0.4pt}
  \fancyhead[R]{\thepage}
  \pagestyle{fancy}
\usepackage[framemethod=TikZ]{mdframed}
  \mdfsetup{roundcorner=3pt,linecolor=olive}
  \usetikzlibrary{shadows}
% New commands
\def\suftesi{\textsf{suftesi}}
\newcommand{\argstyle}{\itshape}
\DeclareRobustCommand*{\ar}[1]{\texttt{\char`\{}%
  \textrm{\argstyle#1}\texttt{\char`\}}}
\DeclareRobustCommand*{\oar}[1]{\texttt{[}%
  \textrm{\argstyle#1}\texttt{]}}
\DeclareRobustCommand*{\meta}[1]{%
  $\langle${\argstyle\rmfamily#1\kern0.12em}$\rangle$}
\DeclareRobustCommand*{\arm}[1]{\ar{\meta{\argstyle#1}}}
\DeclareRobustCommand*{\oarm}[1]{\oar{\meta{\argstyle#1}}}
\newcommand{\emphasize}[1]{\textcolor{teal}{#1}}
% environments
\newenvironment{ttquote}
{\begin{mdframed}
    \ttfamily\microtypesetup{activate=false}}
  {\end{mdframed}}
\newenvironment{latexcode}
{\begin{mdframed}}
  {\end{mdframed}}
\newenvironment{bibexample}
{\begin{mdframed}[backgroundcolor=philA!10,linecolor=white]}
  {\end{mdframed}}
\newenvironment{bibexamplelist}
{\begin{NoHyper}\begin{mdframed}[backgroundcolor=philA!10,linecolor=white]
    \list{}{\setlength{\itemindent}{-.5cm}\setlength{\leftmargin}{.5cm}\setlength{\itemsep}{0pt}}}
  {\endlist\end{mdframed}\end{NoHyper}
}

\makeatletter
\renewenvironment*{optionlist}
{\list{}{%
    \setlength{\labelwidth}{2.5cm}%
    \setlength{\labelsep}{\z@}%
    \setlength{\leftmargin}{2.5cm}%
    \renewcommand*{\makelabel}[1]{\hss\optionlistfont##1}}%
  \ltd@optionlist}
{\endlist}
\renewenvironment*{fieldlist}[1][3cm]
{\list{}{%
    \setlength{\labelwidth}{#1}%
    \setlength{\labelsep}{\marglistsep}%
    \setlength{\leftmargin}{2.5cm}%
    \renewcommand*{\makelabel}[1]{\hss\marglistfont##1}}%
  \def\fielditem##1##2{%
    \item[##1]%
    \ltd@pdfbookmark{##1}{##1}%
    field (##2)\par\nobreak
    \vspace{\itemsep}}%
  \def\listitem##1##2{%
    \item[##1]%
    \ltd@pdfbookmark{##1}{##1}%
    list (##2)\par\nobreak
    \vspace{\itemsep}}}
{\endlist}
\renewenvironment*{ltxsyntax}[1][3cm]
{\list{}{%
    \setlength{\labelwidth}{3cm}%
    \setlength{\labelsep}{0pt}%
    \setlength{\leftmargin}{#1}%
    \renewcommand*{\makelabel}[1]{%
      \hss\ltxsyntaxfont\ltxsyntaxlabelfont##1}}%
  \let\csitem\ltd@csitem
  \let\cmditem\ltd@cmditem
  \let\envitem\ltd@envitem
  \let\lenitem\ltd@csitem
  \let\boolitem\ltd@boolitem
  \let\cntitem\ltd@item
  \let\optitem\ltd@item}
{\endlist}
{\endlist}
\renewenvironment*{valuelist}[1][]
{\list{}{%
    \ifblank{#1}
    {\setlength{\labelwidth}{5em}}
    {\setlength{\labelwidth}{#1}}%
    \setlength{\labelsep}{1em}%
    \setlength{\leftmargin}{\labelwidth}%
    \addtolength{\leftmargin}{\labelsep}%
    \setlength{\itemsep}{0pt}%
    \renewcommand*{\makelabel}[1]{\valuelistfont##1\hss}}}
{\endlist}
\newcommand*{\valuelistfont}{%
  \color{olive}\sffamily\displayverbfont}
\renewcommand*{\optionlistfont}{%
  \color{philA}\sffamily\displayverbfont}
\renewcommand*{\ltxsyntaxlabelfont}{%
  \color{philA}\sffamily\displayverbfont}
\renewcommand*{\marglistfont}{%
  \color{philA}\sffamily\displayverbfont}
% Table of contents
\renewcommand\tableofcontents{%
  \setlength{\columnsep}{1cm}
  {\centering      
    \section*{\contentsname}%
    \@mkboth{\contentsname}{\contentsname}}   
  \thispagestyle{empty}
  \begin{multicols}{2}
    \@starttoc{toc}%
  \end{multicols}}
  \makeatother
  
  \usepackage{hyperref}
  \hypersetup{%
    pdftitle={User's Guide to \textsf{biblatex philosophy}},
    pdfsubject={Bibliography styles for (Italian) 
      users of biblatex},
    pdfauthor={Ivan Valbusa},
    pdfkeywords={bibliography},
  }   
  
  \EnableCrossrefs         
  \CodelineIndex
  \RecordChanges

\begin{document}
  \DocInput{biblatex-philosophy.dtx}
\end{document}
%
%</driver>
% \fi
%
% \CheckSum{4309}
%
% \CharacterTable
%  {Upper-case    \A\B\C\D\E\F\G\H\I\J\K\L\M\N\O\P\Q\R\S\T\U\V\W\X\Y\Z
%   Lower-case    \a\b\c\d\e\f\g\h\i\j\k\l\m\n\o\p\q\r\s\t\u\v\w\x\y\z
%   Digits        \0\1\2\3\4\5\6\7\8\9
%   Exclamation   \!     Double quote  \"     Hash (number) \#
%   Dollar        \$     Percent       \%     Ampersand     \&
%   Acute accent  \'     Left paren    \(     Right paren   \)
%   Asterisk      \*     Plus          \+     Comma         \,
%   Minus         \-     Point         \.     Solidus       \/
%   Colon         \:     Semicolon     \;     Less than     \<
%   Equals        \=     Greater than  \>     Question mark \?
%   Commercial at \@     Left bracket  \[     Backslash     \\
%   Right bracket \]     Circumflex    \^     Underscore    \_
%   Grave accent  \`     Left brace    \{     Vertical bar  \|
%   Right brace   \}     Tilde         \~}
%
% \changes{v1.9.5}{2017/04/13}{Styles completely revised. Provided support for the \opt{mergedate} default option. \opt{latinemph} option defined globally. New values for \opt{scauthors} option. Support for the \bibtype{set} entries for \sty{modern} style. \opt{classical} option removed for Spanish. Provided experimental French localization module. Improved \opt{annotation} option. Updated documentation.}
% \changes{v1.9.4}{2017/03/21}{Maintenance release. Fixed some bugs in modern style.}
% \changes{v1.9.3}{2017/03/17}{Maintenance release. Reset \file{philosophy-verbose.cbx} to version 1.9. Moved \texttt{labelname} format from \file{philosophy-standard.bbx} to \file{philosophy-classic.cbx}.}
% \changes{v1.9.2}{2017/03/14}{Support for \texttt{multivolume} related type. The \texttt{origed} string is substituted with \texttt{origpubas} (redefined for Italiana and Spanish). New string \texttt{opcited}. Deleted \texttt{cited} string. New multi-value option \opt{scauthors} substitutes \opt{scauthorcite} and \opt{scauthorbib} options. Updated documentation.}
% \changes{v1.9.1}{2017/02/16}{Redefined macros for the eechanism. Support for the \bibtype{set} entries for \sty{classic} style. Support for the \opt{origpubin} and  \opt{origpubas} default related types. Improved  \texttt{.lbx} files.  Updated documentation.}
% \changes{v1.9}{2016/11/26}{Redefined \opt{ibidem} and \opt{loccit} strings in file \file{english-philosophy.lbx} according to the Chicago Manual of Style.}
% \changes{v1.8}{2016/06/16}{Maintenance release. Corrected an incompatibility with \opt{scauthors} option.}
% \changes{v1.7}{2016/06/10}{Maintenance release. Updated documentation.}
% \changes{v1.6}{2016/05/22}{Removed compatibility with legacy Bib\TeX{} backend.}
% \changes{v1.5}{2016/05/18}{Improved compatibility with legacy Bib\TeX{} backend.}
% \changes{v1.4}{2016/03/10}{Maintenance release. Updated style for working with \sty{biblatex} v.3.4.}
% \changes{v1.3}{2015/10/09}{Maintenance release. Corrected a spurious space in article entries.}
% \changes{v1.2}{2015/09/19}{Maintenance release. New value \opt{superscript} for \opt{editionformat} option.}
% \changes{v1.1}{2015/06/13}{Maintenance release. Updated documentation.}
% \changes{v1.0}{2015/03/31}{Corrected a bug in \opt{volumeformat} and \opt{volnumformat} options. Change value \opt{romanupp} to \opt{Roman} in \opt{volume format} and \opt{edition format}. Corrected bug in \opt{related format} options: now the related block is not preceded by semicolon when using values \opt{parens} and \opt{brackets} styles. §updated documentation}
% \changes{v0.9i}{2015/03/14}{Corrected a bug with \opt{shorthandintro} option.}
% \changes{v0.9h}{2015/01/14}{New option \opt{lowscauthors}. Corrected some bugs. Updated documentation.}
% \changes{v0.9g}{2014/12/12}{Added localization module for spanish. Corrected a bug in \bibtype{inbook} and \bibtype{incollection} entries when using \bibfield{crossref} field. Updated documentation.}
% \changes{v0.9f}{2014/03/28}{Updated documentation. Corrected a bug in the \opt{volnumformat} option.}
% \changes{v0.9e}{2014/02/12}{Maintenance release. Updated documentation.}
% \changes{v0.9d}{2013/11/13}{Maintenance release. Corrected some bugs.}
% \changes{v0.9c}{2013/10/15}{Maintenance release. Corrected some bugs.}
% \changes{v0.9b}{2013/08/30}{Updated bibliography drivers to correct a bug when using the \bibfield{related} mechanism.}
% \changes{v0.9a}{2013/07/04}{Maintenance release. Corrected some bugs.}
% \changes{v0.8f}{2013/06/20}{New option \opt{nodate} for \cmd{printbibliography} command}
% \changes{v0.8e}{2013/04/13}{Improved \opt{relatedformat} option for cascading entries. Implemented cross-referencing mechanism for \bibtype{inproceedings} entries. Improved \cmd{ccite} command. Changed the values for the \bibfield{entrysubtype} from \bibfield{classical} to \bibfield{classic}. Added \opt{nodate} package option.}
% \changes{v0.8d}{2013/03/30}{Fixed some bugs related to \cmd{DeclareDriverSourcemap}, \cmd{ccite}, and \bibtype{review} entry type}
% \changes{v0.8c}{2013/03/27}{Removed \sty{biber.conf} configuration file. Added internal \file{biber} settings with \cmd{DeclareDriverSourcemap} command. Added \bibfield{trans-} field alias}
% \changes{v0.8b}{2013/03/22}{Added \sty{biber.conf} configuration file}
% \changes{v0.8a}{2013/04/18}{New \bibfield{entrysubtype} field for citing classical texts. New \bibfield{related} field mechanism. New \opt{relatedformat} option. New English documentation. Fixed some bugs}
% \changes{v0.7c}{2011/05/16}{Fixed some bugs. New \opt{origfieldtype} option. Added \bibfield{nodate} bibliography string. Updated documentation}
% \changes{v0.7b}{2010-04-23}{Removed package option \opt{romanvol}. Added package option \opt{volumeformat}. Added package option \opt{editionformat}. Activated the option \opt{singletitle} for style \sty{philosophy-verbose}. Added bibliography driver \bibtype{review}. Added bibliography string \opt{cit}. Added bibliography string \opt{reviewof}. Updated documentation}
% \changes{v0.7a}{2010-04-03}{Added command \opt{volumfont}. Added command \opt{footcitet}. Updated documentation}
% \changes{v0.7}{2010-03-30}{Removed package option \opt{colonloc}. Removed package options \opt{origparens} \opt{origbrackets}. Added package options \opt{origfieldsformat}, \opt{publocformat}, \opt{commacit}, \opt{inbeforejournal}, \opt{romanvol}, \opt{volnumformat}. Added command \opt{volnumpunct}. Added citation commands \cmd{sdcite}, \cmd{ccite}. Updated documentation}
% \changes{v0.6}{2010-03-07}{Added localization file \file{italian-philosophy.lbx}. Added file \file{philosophy-standard.bbx}. New documentation file \file{biblatex-philosophy}. Removed files \file{philosophy-authoryear-doc.tex}, \file{philosophy-verbose-doc.tex}.}
% \changes{v0.5}{}{Added new bibliography style \texttt{philosophy-verbose}. Added localization file \file{italian-philosophy.lbx}. Added package options \opt{origparens}, \opt{origbrackets}, \opt{latinemph}. Changed package options \opt{scauthors}, \opt{scauthorscite}, \opt{scauthorsbib}. Renamed file \file{biblatex-philosophy-doc.tex} to \texttt{philosophy-authoryear-doc.tex}. Added file \file{philosophy-verbose-doc.tex}. Updated documentation}
% \changes{v0.4}{}{Improved compatibility for \sty{biblatex} version 0.9}
%
% \GetFileInfo{biblatex-philosophy.dtx}
%
% \DoNotIndex{\newcommand,\newenvironment,\def,\begin,\vskip,\ }
% \DoNotIndex{\DeclareOption,\ExecuteOptions,\RequirePackage}
% \DoNotIndex{\@@end,\@empty,\@ifclassloaded,\@nameuse,\@nil}
% \DoNotIndex{\@undefined,\\,\`,\addtocounter,\advance,\bfseries}
% \DoNotIndex{\centering,\closeout,\define@key,\documentclass}
% \DoNotIndex{\edef,\else,\end,\endinput,\endtitlepage,\expandafter}
% \DoNotIndex{\extracolsep,\fi,\fill,\fontsize,\g@addto@macro,\toks}
% \DoNotIndex{\hrule,\hspace,\if,\if@twoside,\ifcase,\ifdefined}
% \DoNotIndex{\iffalse,\IfFileExists,\ifnum,\ifx,\immediate,\setcounter}
% \DoNotIndex{\jobname,\let,\long,\MakeUppercase,\MessageBreak}
% \DoNotIndex{\newcount,\newif,\newpage,\newtoks,\newwrite,\next}
% \DoNotIndex{\noexpand,\nofiles,\normalfont,\normalsize,\null}
% \DoNotIndex{\openout,\or,\styage,\styageError,\styageWarning}
% \DoNotIndex{\styageWarningNoLine,\paperheight,\paperwidth,\par}
% \DoNotIndex{\parbox,\parindent,\relax,\scshape,\selectfont,\setkeys}
% \DoNotIndex{\sffamily,\space,\stretch,\string,\textheight,\textwidth}
% \DoNotIndex{\the,\thispagestyle,\unexpanded,\unless,\unskip,\upshape}
% \DoNotIndex{\usepackage,\vbox,\vfill,\vspace,\write,\z@}
% \DoNotIndex{\CurrentOption,\AtEndDocument,\@ne,\c@page,\m@ne}
% \DoNotIndex{\@firstofone,\@gobble,\@makeother,\begingroup,\endgroup}
% \DoNotIndex{\eTeXversion,\hbox,\hsize,\includegraphics,\newlinechar}
% \DoNotIndex{\titlepage,\vss,\vtop,\xdef,\@gobbletwo,\color,\dimexpr}
% \DoNotIndex{\huge,\large,\makebox,\ProcessOptions,\renewcommand}
%
%
% \thispagestyle{empty}
% \begin{tikzpicture}[overlay,remember picture]
% \draw[circular drop shadow,draw=none,fill=white] (current page.center) circle (6cm);
% \node[align=center,anchor=center] at (current page.center) {%
% Ivan Valbusa\\[1cm]
% 
%\Huge\color{gray}\bfseries \parbox{10cm}{\centering The \\{{\color{philA}\fontsize{30}{32}\textsf{biblatex-philosophy}}\\ bundle}}\\[1cm]
% \color{black!60!Goldenrod}
%
%\normalsize \fileversion{} -- \filedate
% 
%};
% \end{tikzpicture}
%
% \clearpage 
%
% \thispagestyle{empty}
% \vspace*{\fill}
%
%\noindent Copyright \copyright\ 2009-\the\year{} Ivan Valbusa.
% \bigskip
%
%\noindent This package is author-maintained. 
%Permission is granted to copy, distribute and/or modify this software under the 
%terms of the LaTeX Project Public License, version 1.3c ora later (\url{http://latex-project.org/lppl}). This software is provided ''as is'', without warranty of any kind, either expressed or implied, including, but not limited to, the implied warranties of merchantability and fitness for a particular purpose.
%
% \bigskip
% 
%\noindent If you have any questions, feedback or requests please email me at \texttt{ivan dot valbusa at univr dot it}. If you need specific features not already implemented, remember to attach the example files.
%
% \clearpage
%
% \begin{center}
% {\color{philA}\bfseries 
%   \Huge User's Guide to\\ \textsf{biblatex-philosophy}\\[3mm]}
% {\large\itshape Bibliography styles for (Italian) users of~\textsf{biblatex}}
%\vspace{.8cm}
%
% \normalsize\fileversion{} -- \filedate
% \vspace{.7cm}
%
% Ivan Valbusa\\[2mm]
%
%  \small Dipartimento di Filologia, Letteratura e Linguistica\\
%  \small Università degli Studi di Verona\\
%  \footnotesize\texttt{ivan dot valbusa at univr dot it}
% \vspace{1cm}
% \end{center}
%
%\begin{abstract}
%\noindent This package provides a small collection of bibliography and citation styles for use with Philipp Lehman's \sty{biblatex} package. These styles try to be language-indipendent but their prime aim is to match the needs of the Italian writers, particularly those concerned in the humanities. They offer useful features to compose detailed bibliographic entries including the translation data of foreign texts, annotations etc. Many options allow you to change the style defaults. Only the Italian, English and Spanish localization is available for now but you can use the styles with all the languages adding simple redefinitions. 
%  \end{abstract} 
%
% \tableofcontents
%
%
%\section*{A brief history}
%
%The firs step toward the creation of the \sty{philosophy-modern} style  was the request of Lorenzo Pantieri in the \GuIT{} Forum: \url{http://www.guit.sssup.it/phpbb/viewtopic.php?t=6472} (See the discussion on \url{http://www.guit.sssup.it/phpbb/viewtopic.php?t=6717.}) Now this is the bibliography style of \citetitle{pantieri:artelatex}, the most popular Italian guide to \LaTeX{} \parencite{pantieri:artelatex}. 
%\nocite{ctan,guit:sito}
%
% \smallskip
%
% {\itshape\noindent  I would like to thank all those who took part in the debate on {\fontfamily{cmr}\upshape\selectfont\GuIT{}} Web site and the authors of the styles which inspired \sty{biblatex-philosophy}, specifically: Dominik \textcite{wassenhoven:dw}, James \textcite{clawson:mla} and  Sander \textcite{glibof:historian}. Last but not least, a special thank to Philipp \textcite{lehman:biblatex} for his fundamental package and to the actual developers, Philip Kime\index{Kime, Philip}, Audrey Boruvka\index{Boruvka, Audrey} and Joseph Wright\index{Wright, Joseph}.
% }
%
%\section{Use}
%
% 
%The styles can be loaded as usual, but to ensure language-specific quotation marks you need \sty{babel} or \sty{polyglossia} and \sty{csquotes} \parencite[see][]{babel,polyglossia,csquotes}. Biber in place of Bib\TeX{} is also required as backend bibliography processor \parencite{kime:biber}. The example below shows a typical code for an Italian document. Replace \meta{style} with \sty{classic}, \sty{modern} or \sty{verbose}, and \meta{bibfile} with the name of your bibliography file (``.bib'' must be declared). For other languages you can choose to use or not the Italian-style quotation marks provided by \sty{csquotes}.
%\begin{ttquote}
%\cmd{usepackage}\oar{italian}\ar{babel}\\
%\cmd{usepackage}\oar{style=italian}\ar{csquotes}\\
%\cmd{usepackage}\oar{style=philosophy-\meta{style}}\ar{biblatex}\\
%\mbox{}\quad \cmd{addbibresource}\ar{\meta{bibfile}.bib}
%\end{ttquote}
%To uniform the style of quotation marks in multilingual bibliographies typeset using the \opt{autolang=other} package option, you can use the \cmd{DeclareQuoteAlias} command. For example:
%\begin{ttquote}
%\cmd{DeclareQuoteAlias}\ar{italian}\ar{german}
%\end{ttquote}
%
%
%\section{The styles}\label{sec:introduction}
% This package provides three different styles: a verbose style and two author-year styles. The first simple and trivial characteristic of these style is that they use commas instead of dots to separate the parts of the entry, according to the most common Italian tradition. But they do much more, of course. The other features, some of which are style-dependent, are described in the next sections and can be easily examined looking at the examples at the end of this documentation or typesetting the example files in the \texttt{texmf-dist/doc/latex/biblatex-philosophy/examples.zip} \TeX Live folder. 
%
% Note that \sty{biblatex} adopts by default a very rational criterion for the ordering of the list of namens in multi-authors/editors entries. Only for the first author/editor the surname precedes the name while the other authors/editors are typeset in the form ``Name Surname'' (e.g. ``Eco, Umberto and Gianni Vattimo''). The Italian (academic) writers often see this feature like a sort of inconsistency. Actually it is inconsistent to typeset all the authors in the form ``Surname, Name'' when this is useless.
% 
%\subsection[\sty{philosophy-classic}]{The \sty{philosophy-classic} style}
%
%The \sty{classic} style is a standard author-year style associated to a compact citation scheme which allows to cite multiple entries of the same author and/or published in the same year, omitting some redundant informations:
%\begin{bibexample}
%Knuth (1984, 1986a,b,c,d)
%\end{bibexample}
% A \sty{classic} bibliography is shown below. You can change indentation, horizontal and vertical space between entries and  between blocks or groups of entries. The dash can be replaced by the author's label via the \opt{dashed=false} option and you can have brackets in place of parentheses as well. See the \sty{biblatex} documentation  and section \ref{sec:lengths}.
%\begin{bibexamplelist}
%\item Donald E. Knuth (1984-1986),  \emph{Computers \& Typesetting}, 5 vols., Addison-Wesley.
%\item ---
%(1984)  \emph{Computers \& Typesetting}, vol. A: \emph{The \TeX book}, Addison-Wesley.
%\item ---
%(1986a)  \emph{Computers \& Typesetting}, vol. B: \emph{\TeX: The Program}, Addison-Wesley.
% \item ---
%(1986b)  \emph{Computers \& Typesetting}, vol. C: \emph{The METAFONTbook}, Addison-Wesley.
%  \item ---
%(1986c)  \emph{Computers \& Typesetting}, vol. D: \emph{METAFONT: The Program}, Addison-Wesley.
%  \item ---
%(1986d)  \emph{Computers \& Typesetting}, vol. E: \emph{Computer Modern Typefaces}, Addison-Wesley.
%  \end{bibexamplelist}
%
%
%\subsection[\sty{philosophy-modern}]{The \sty{philosophy-modern} style} \label{esempio-modern}
%
%The \sty{modern} style uses the ``classic'' citation scheme but produces a fancy bibliography divided into blocks, which is particularly suited for bibliographies with many entries for the same author. You can change the distance between year and title and, of course, all the common features with the ``classic'' style. Here is an example of a \sty{modern} bibliography: 
%  \begin{bibexample}
%    \list{}{
%    \setlength{\labelwidth}{2cm}
%    \setlength{\leftmargin}{2cm}
%    \setlength{\itemsep}{0em}
%  }
%  \item \hskip-2cm Knuth, Donald E. 
%  \item[1984/1986]
%    \emph{Computers \& Typesetting}, 5 vols., Addison-Wesley.
%  \item[1984]
%    \emph{Computers \& Typesetting}, vol. A: \emph{The \TeX book}, Addison-Wesley.
%  \item[1986a]
%    \emph{Computers \& Typesetting}, vol. B: \emph{\TeX: The Program}, Addison-Wesley.
%   \item[1986b]
%    \emph{Computers \& Typesetting}, vol. C: \emph{The METAFONTbook}, Addison-Wesley.
%    \item[1986c]
%    \emph{Computers \& Typesetting}, vol. D: \emph{METAFONT: The Program}, Addison-Wesley.
%\item \hskip-2cm Nietzsche, Friedrich
% \item[1988a] \emph{Sämtliche Werke. Kritische Studienausgabe}, ed. by Giorgio Colli and Mazzino Montinari,
%2nd ed., 15 vols., Deutscher Taschenbuch-Verlag and Walter de Gruyter, München,
%Berlin, and New York.
%\item[1988b] \emph{Sämtliche Werke. Kritische Studienausgabe}, vol. 1: \emph{Die Geburt der Tragödie. Unzeitgemäße
%Betrachtungen I–IV. Nachgelassene Schriften 1870–1973}, ed. by Giorgio Colli and Mazzino
%Montinari, 2nd ed., Deutscher Taschenbuch-Verlag and Walter de Gruyter, München,
%Berlin, and New York.
%\item[1988c] ``Unzeitgemässe Betrachtungen. Zweites Stück. Vom Nutzen und Nachtheil der Historie
%für das Leben'', in \emph{Sämtliche Werke. Kritische Studienausgabe}, vol. 1: \emph{Die Geburt der
%Tragödie. Unzeitgemäße Betrachtungen I–IV. Nachgelassene Schriften 1870–1973}, ed. by
%Giorgio Colli and Mazzino Montinari, Deutscher Taschenbuch-Verlag and Walter de
%Gruyter, München, Berlin, and New York, pp. 243-334.
%\item \hskip-2cm Van Gennep, Arnold
%\item[1909a] \emph{Les rites de passage}, Nourry, Paris.
%\item[1909b] \emph{Les rites de passage}, Nourry, Paris; trans. by Monika B. Vizedom and 
% Gabrielle L. Caffee as \emph{The Rites of Passage} (University of Chicago Press, 1960).
%\item[1960] The Rites of Passage, trans. from the French by Monika B. Vizedom and Gabrielle 
% L. Caffee, University of Chicago Press.
%\endlist
%    \end{bibexample}
%  
%
%\subsection[\sty{philosophy-verbose}]{The \sty{philosophy-verbose} style}\label{sec:verbose}
%
%This style is aimed for citations given in the footnotes and follows the most popular scheme used in the Italian humanities. It prints a full citation similar to a bibliography entry when an item is cited for the first time, and a short citation afterwards, using the title (possibly shortened in the \bibfield{shorttitle} field), followed by the string ``cit.''.
%Citing the same entry two times, in the second one the string ``Ivi'' (``Ibid.'' for English documents) is used; citing the same place of the previous citation you will have ``Ibidem'' (``Ibid.'' for English documents):
%\smallskip
%
% \textbf{Italian}
%    \begin{bibexample}
%        \textsuperscript{1} \fullcite{Valbusa:2007}, p. 43\\
%        \textsuperscript{2} \emph{Ivi}, p. 26. \\
%        \textsuperscript{3} \emph{Ibidem}. \\
%        \textsuperscript{4} \fullcite{heidegger:sz}.\\
%        \textsuperscript{5} Valbusa, \emph{Psicologia e sistema} cit., p. 35.
%    \end{bibexample}
%\smallskip
%
% \textbf{English}
%    \begin{bibexample}
%        \textsuperscript{1} \fullcite{Poincare:1968-ORIG}, p. 43\\
%        \textsuperscript{2} \emph{Ibid.}, p. 26. \\
%        \textsuperscript{3} \emph{Ibid.}. \\
%        \textsuperscript{4} \fullcite{heidegger:sz}.\\
%        \textsuperscript{5} Poincaré, \emph{La science et l’hypothèse} cit., p. 35.
%    \end{bibexample}
%
%When there is only one entry for the same author, with the \opt{singletitle=true} option 
%the string ``op. cit.'' is used instead of the (short) title followed by ``cit.'':%
%\begin{bibexample}
%\textsuperscript{6} Descartes, op. cit., p. 35.
%\end{bibexample}
%
%All the scholarly abbreviations (\emph{latinitates}) but ``cit.'' are printed by default in normal font. With the \opt{latinemph} option (section \ref{sec:options-verbose}) you can get them in italic shape.
%
% A \sty{verbose} bibliography is similar to a \sty{classic} bibliography but with the year placed at the end of the entry:
%\begin{bibexamplelist}
%\item Donald E. Knuth,  \emph{Computers \& Typesetting}, 5 vols., Addison-Wesleys, 1984-1986.
%\item ---
%  \emph{Computers \& Typesetting}, vol. A: \emph{The \TeX book}, Addison-Wesley, 1984.
%\item ---
%  \emph{Computers \& Typesetting}, vol. B: \emph{\TeX: The Program}, Addison-Wesley, 1986.
% \item ---
%  \emph{Computers \& Typesetting}, vol. C: \emph{The METAFONTbook}, Addison-Wesley, 1986.
%  \item ---
%  \emph{Computers \& Typesetting}, vol. D: \emph{METAFONT: The Program}, Addison-Wesley, 1986.
%  \item ---
%  \emph{Computers \& Typesetting}, vol. E: \emph{Computer Modern Typefaces}, Addison- Wesley, 1986.
%  \end{bibexamplelist}
%
% \section{Specialities}\label{sec:specialities}
%
%\subsection{Related entries}\label{sec:related}
%
%The \sty{philosophy} styles use the mechanism provided by the \bibfield{related} field to typeset complex entries comprising both the original publication data and the translation data \parencite[see][]{Poincare:1968-ORIG}. The related entry is preceded by the \texttt{translationas} string which defaults to ``trad.~it.'', ``trans.~as'', ``trad.~es.'', ``trad.'' for Italian, English, Spanish and French documents, respectively. If you want to change it, use the \bibfield{relatedstring} field, like in \textcite{popper-logik} which shows, among others, an entry with cascading relations. 
%
%  \begin{bibexamplelist}
%\item \fullcite{Poincare:1968-ORIG}.
%  \end{bibexamplelist}
%
%\begin{latexcode}
%\begin{verbatim}
%@book{Poincare:1968-ORIG,
%  author    = {Jules-Henri Poincaré},
%  title     = {La science et l'hypothèse},
%  publisher = {Flammarion},
%  location  = {Paris},
%  date      = {1968},
%  related   = {Poincare:1968-ITA}}
%
%@book{Poincare:1968-ITA,
%  author    = {Jules-Henri Poincaré},
%  editor    = {Corrado Sinigaglia},
%  title     = {La scienza e l'ipotesi},
%  publisher = {Bompiani},
%  location  = {Milano}}
%\end{verbatim}
%
%  \end{latexcode}
%\begin{bibexamplelist}
%\item \fullcite{popper-logik}.
%\end{bibexamplelist}
%\begin{latexcode}
%\begin{verbatim}
%@book{popper-logik,
%  title = {Logik der Forschung},
%  publisher = {Springer},
%  author = {Karl R. Popper},
%  date = {1934},
%  location = {Wien},
%  related = {popper-logik:ing}}
%
%@book{popper-logik:ing,
%  title = {The Logic of Scientific Discovery},
%  publisher = {Hutchinson},
%  author = {Karl R. Popper},
%  edition = {3},
%  date = {1959},
%  location = {London},
%  related = {popper-logik:ita},
%  relatedstring={it\adddotspace trans\adddot}}
%
%@book{popper-logik:ita,
%  title = {Logica della scoperta scientifica},
%  publisher = {Einaudi},
%  author = {Karl R. Popper},
%  edition = {3},
%  date = {1998},
%  location = {Torino}}
%\end{verbatim}
%\end{latexcode}
%
%
%\subsection{Crossreferences}\label{sec:crossref}
%
%
%The \sty{philosophy} styles allow you to manage 
%entries referring to other entries via the \bibfield{crossref} fields. This is very useful when you have to cite two or more \bibtype{incollection} of the same \bibtype{collection} \parencite[see][]{corrocher:2009,federspil:2009}. In this way the \bibtype{collection} is printed in the bibliography and is cross-referenced inside the \bibtype{incollection}, using the corresponding author-year label (the mechanism is the same for \bibtype{inbook} items).
%\begin{bibexamplelist}
%\item Corrocher, Roberto (2009) “Riflessioni sull’uomo di fronte a nuove sfide”, in Giaretta et al. (2009), pp. 27-42.
%\item Federspil, Giovanni and Roberto Vettor (2009), “Medicina: un unico metodo e una sola argomentazione?”, in Giaretta et al. (2009), pp. 43-74.
%\item Giaretta, Pierdaniele, Antonio Moretto, Gian Franco Gensini, and Marco Trabucchi (2009) (eds.), \emph{Filosofia delle medicina. Metodo, modelli, cura ed errori}, 2 vols., il Mulino, Bologna.
%\end{bibexamplelist}
%\begin{latexcode}
%\begin{verbatim}
%@collection{Filmed:2009,
%  title = {Filosofia delle medicina},
%  booktitle = {Filosofia delle medicina},
%  subtitle = {Metodo, modelli, cura ed errori},
%  editor = {Pierdaniele Giaretta and Antonio Moretto 
%  and Gian Franco Gensini
%  and Marco Trabucchi},
%  volumes = {2},
%  publisher = {il Mulino},
%  location = {Bologna},
%  date = {2009}}
%
%@incollection{corrocher:2009,
%  author = {Roberto Corrocher},
%  title = {Riflessioni sull'uomo di fronte a nuove sfide},
%  pages = {27-42},
%  crossref = {Filmed:2009}}
%
%@incollection{federspil:2009,
%  author = {Giovanni Federspil and Roberto Vettor},
%  title = {Medicina: un unico metodo e una sola argomentazione?},
%  pages = {43-74},
%  crossref = {Filmed:2009}}
%\end{verbatim}
%\end{latexcode}
%When you have to cite \emph{only one} \bibtype{incollection} of a single \bibtype{collection} you have three choices:
% \smallskip
%
% \noindent 1. use the \bibfield{crossref} field  \parencite[see][]{Termini:2007}. In this case all the  \bibtype{collection} data are automatically printed inside 
% the \bibtype{incollection} entry:
%\begin{bibexamplelist}
%\item \fullcite{Termini:2007}
%\end{bibexamplelist}
%\begin{latexcode}
%\begin{verbatim}
%@incollection{Termini:2007,
%  author = {Settimo Termini},
%  title = {Vita morte e miracoli di Alan Mathison Turing},
%  crossref = {Bartocci:2007}}
%
%@collection{Bartocci:2007,
%  title = {Vite matematiche}
%  booktitle = {Vite matematiche},
%  booksubtitle = {Protagonisti del '900 da Hilbert a Wiles},
%  editor = {Claudio Bartocci and Renato Betti and Angelo Guerraggio and 
%  Roberto Lucchetti},
%  publisher = {Springer-Verlag Italia},
%  location = {Milano},
%  date = {2007}}  
%\end{verbatim}
%\end{latexcode}
%  2.  put the \bibtype{collection} data in the fields of the \bibtype{incollection} entry \parencite[see][]{Valbusa:2007}. In this case the \bibtype{incollection} is self-contained:
% \begin{bibexamplelist}
%\item \fullcite{Valbusa:2007}
% \end{bibexamplelist}  
%\begin{latexcode}
%\begin{verbatim}
%@incollection{Valbusa:2007,
%  author = {Ivan Valbusa},
%  title = {Psicologia e sistema in Alsted e in Wolff},
%  booktitle = {Christian Wolff tra psicologia empirica e 
%  psicologia razionale},
%  publisher = {Georg Olms Verlag},
%  editor = {Ferdinando Luigi Marcolungo},
%  location = {Hildesheim and Zürich and London},
%  date = {2007}}
%\end{verbatim}
%\end{latexcode}
%  3. put the \bibtype{collection} data in the fields of the \bibtype{incollection} entry and put the \bibtype{collection} label in the \bibfield{xref} field of the \bibtype{incollection} \parencite[see][]{kant:kpv:xref,kant:ku:xref}\label{kant}:
% \begin{bibexamplelist}
% \item Immanuel Kant (1968a), \emph{Kants Werke. Akademie Textausgabe}, 9 vols., Walter de Gruyter, Berlin
% \item --- (1968b), \emph{Kritik der praktischen Vernunft}, in Kant (1968a), vol. 5, pp. 1-163
% \item --- (1968c), \emph{Kritik der Urtheilskraft}, in Kant (1968a), vol. 5, pp. 165-485
% \end{bibexamplelist}  
%\begin{latexcode}
%\begin{verbatim}
%@bookinbook{kant:kpv:xref,
%  author = {Kant, Immanuel},
%  title = {Kritik der praktischen Vernunft},
%  shorttitle = {Kritik der praktischen Vernunft},
%  volume = {5},
%  pages = {1-163},
%  date = {1968},
%  xref = {kant:werke}}
%
%@bookinbook{kant:ku:xref,
%  author = {Kant, Immanuel},
%  title = {Kritik der Urtheilskraft},
%  volume = {5},
%  pages = {165-485},
%  date = {1968},
%  xref = {kant:werke}}
%
%@mvbook{kant:werke,
%  author = {Kant, Immanuel},
%  title = {Kants Werke. Akademie Textausgabe},
%  maintitle = {Kants Werke. Akademie Textausgabe},
%  booktitle = {Kants Werke. Akademie Textausgabe},
%  volumes = {9},
%  publisher = {Walter de Gruyter},
%  location = {Berlin},
%  date = {1968}}    
%\end{verbatim}
%\end{latexcode}
%
%With the \sty{verbose} style, when citing \bibtype{incollection}s entries, the data of the \bibtype{collection} are printed entirely in the first citation and shortened afterwards. Anyway in the final bibliography the \bibtype{incollection} is always complete of all the informations about the corresponding \bibtype{collection}.
%
%
%\subsection{Classical works}\label{sec:classical-works}
%
%
% The treatment of the classical works and other writings with uncertain or omitted date is not particularly difficult if you use the \sty{verbose} style, but with the \sty{classic} and \sty{modern} styles some difficulties inevitably impose clear choices.
%
%If a critical edition exists you should cite it directly, such as  \textcite{heidegger:sz}. If you do not like this anachronistic label you may use the \bibfield{shorthand} field, such as \cite{kant:kpv:xref}. 
% Note that a ``shorthand intro'' is automatically printed when the entry is cited for the first time (see p. \pageref{kant}) and omitted afterwards.  To turn off this feature load the  option \opt{shorthandintro=false}. Of course in this case you will need a list of shorthands. If you do not like these solutions you can use the \bibfield{entrysubtype} or the  \cmd{sdcite} command in order to get an author-title citation, such as \cite{aristotle:ethics} (see sections \ref{sec:newfields} and \ref{sec:commands:citations}). 
%
%\section{New fields}\label{sec:newfields}
%
%\begin{fieldlist}
%
%\fielditem{nameaddon}{literal} 
%
%An addon to be printed immediately after the author name in the bibliography. 
%It is useful for those author known with alias, Latinized names, etc. For example \textcite{comenio:oo}:
%\begin{bibexamplelist}%
%\item Komensky, Jan Amos \emphasize{[Comenius]} 
% (1969), \emph{Opera Omnia}, Praga.
%\end{bibexamplelist}
%\begin{latexcode}
%\begin{verbatim}
%@mvbook{comenio:oo,
%  author = {Jan Amos Komensky},
%  nameaddon = {Comenius},
%  title = {Opera Omnia}
%  location = {Praga},
%  date = {1969}}
%\end{verbatim}
%\end{latexcode}
%
% \fielditem{entrysubtype}{literal\makebox[0pt][l]{\hspace*{2cm}[\sty{philosophy-classic} and \sty{philosophy-modern} only]}}
% With the \opt{classic} value the citation commands will produce an 
% author-title label. This is useful for citing works from classical antiquity, such as \cite{aristotle:ethics}.
%\begin{latexcode}
%\begin{verbatim}
%@book{aristotle:ethics,
%  entrysubtype = {classic},
%  author       = {Aristotle},
%  title        = {Nichomachean Ethics},
%  ...
% }
%\end{verbatim}
%\end{latexcode}
%In the bibliography the entry is printed with the author-year label, but with the \opt{skipbib} option in the \opt{options} field you can exclude it from the bibliography.
%
%
%\fielditem{library}{literal} 
%
%This field is printed at the end of the entry, in a new period. It is aimed for secondary informations such as the location of the texts, historical notes, etc. For example \textcite{heidegger:sz}:
%\begin{bibexamplelist}
%\item Heidegger, Martin (2001), \emph{Sein und Zeit}, 18th ed., Max Niemeyer Verlag, Tübingen. \emphasize{Originally published in 1927 on the \emph{Jahrbuch für Philosophie und phänomenologische Forschung (vol. VIII)}, directed by H. Husserl}.
%\end{bibexamplelist}
%\begin{latexcode}
%\begin{verbatim}
%@book{heidegger:sz,
%  author = {Martin Heidegger},
%  title = {Sein und Zeit},
%  edition = {18},
%  publisher = {Max Niemeyer Verlag},
%  location = {Tübingen},
%  date = {2001},
%  library = {Originally published in 1927 on the \emph{Jahrbuch 
%  für Philosophie und phänomenologische Forschung (vol. VIII)}, 
%  directed by H. Husserl}}
%\end{verbatim}
%\end{latexcode}
%
%\fielditem{annotation/annote}{literal} 
%
%This field is printed in a new paragraph at the very end of the entry. It requires the \opt{annotation} option. The default font can be changed  redefining the \cmd{annotationfont} command (section \ref{sec:customization}):
%\begin{bibexamplelist}
%\item \fullcite{lehman:biblatex}.\\[.5ex]
%{\footnotesize \emphasize{This package provides advanced bibliographic facilities 
%  for use with \LaTeX. The package is a complete 
%  reimplementation of the bibliographic facilities provided 
%  by \LaTeX. The biblatex package works with the “backend” 
%  (program) \texttt{biber}, which is used to process Bib\TeX 
%  format data files and them performs all sorting, label 
%  generation (and a great deal more).}\par}
%\end{bibexamplelist}
%\begin{latexcode}
%\begin{verbatim}
%@online{lehman:biblatex,
%  author = {Philipp Lehman},
%  title = {The \texttt{biblatex} Package},
%  subtitle = {Programmable Bibliographies and Citations},
%  version = {3.7},
%  date = {2016-11-16},
%  annote = {This package provides advanced bibliographic facilities 
%  for use with \LaTeX. The package is a complete reimplementation 
%  of the bibliographic facilities provided by \LaTeX. The biblatex 
%  package works with the “backend” (program) \texttt{biber}, which 
%  is used to process Bib\TeX{} format data files and them performs
%  all sorting, label generation (and a great deal more).}}
%\end{verbatim}
%\end{latexcode}
%\end{fieldlist}
%
%
%\section{New citation commands}\label{sec:commands:citations}
%
%
%
%\begin{ltxsyntax}
%
%\cmditem{sdcite}{key}\hfill[\sty{philosophy-classic} and \sty{philosophy-modern} only]
%
%Uses an author-title label instead of an author-year label. 
%It is useful for some classical or undated works. Anyway you should prefer the \opt{entrysubtype=classic} field. 
% Writing \verb!\sdcite[15]{guzman:sd}! you will obtain: \sdcite{guzman:sd}.
%
%\cmditem{footcitet}{key} \hfill[\sty{philosophy-classic} and \sty{philosophy-modern} only]
%
%\begin{minipage}{.9\textwidth}
%Same as \cmd{footcite},\footcite[12-13]{corrocher:2009} but with the \cmd{textcite} style.\footcitet[12-13]{corrocher:2009}
%\end{minipage}
%
%\cmditem{ccite}{key}\hfill[\sty{philosophy-verbose} only]
%
%The same as  \cmd{cite}, but omits the author's (editor's) name (defined only for the \sty{verbose} style). Here is an example:
%\begin{latexcode}
%\begin{verbatim}
%This topic is discussed in \cite{Rossi:2007} and in the recent 
%\ccite{Rossi:2008}.
%\end{verbatim}
%\end{latexcode}
%\begin{bibexample}
%This topic is discussed in P. Rossi, \emph{History of Types}, La TeXnica, Verona 2007 and in the recent \emph{Types of History}, Typographica, Milano 2008.
%\end{bibexample}
%\end{ltxsyntax}
%
%
%
%
%\section{New options}\label{sec:options}
%
%
%
%\subsection{Global}
%
%\begin{optionlist}
%\optitem[semicolon]{relatedformat}{\opt{semicolon}, \opt{parens}, \opt{brackets}}
%\begin{valuelist}
%\item[semicolon] 
%The ``related'' entry is preceded by a semicolon.
%      \begin{bibexample}%
%        Poincaré, Jules-Henri (1968), \emph{La science et l'hypothèse}, Paris,
%         Flammarion\emphasize{; trad. it. \emph{La scienza e l'ipotesi}, 
%         Bompiani, 2003}.
%      \end{bibexample}
%\item[parens] 
%Puts the ``related'' entry in parentheses.
%        \begin{bibexample}%
%        Poincaré, Jules-Henri (1968), \emph{La science et l'hypothèse}, Paris, 
%        Flammarion \emphasize{(trad. it. \emph{La scienza e l'ipotesi}, 
%        Bompiani, 2003)}.
%        \end{bibexample}
%\item[brackets] 
%Same as the previous option but with brackets.
%        \begin{bibexample}%
%        Poincaré, Jules-Henri (1968), \emph{La science et l'hypothèse}, Paris, 
%        Flammarion \emphasize{[trad. it. \emph{La scienza e l'ipotesi}, 
%        Bompiani, 2003]}.
%        \end{bibexample}
%\end{valuelist}
%
%
%\optitem[publocyear]{publocformat}{\opt{publocyear}, \opt{locpubyear}, \opt{loccolonpub}}
%  This option provides three styles for typesetting the ``publisher/location/date'' block. 
%  It is active also for the related entry and for the \bibfield{orig-}fields (section \ref{sec:origfields}).
%        \begin{valuelist}
%        \item[publocyear] Oxford University Press, Oxford 2007
%            
%        \item[locpubyear] Oxford, Oxford University Press, 2007
%            
%        \item[loccolonpub] Oxford: Oxford University Press, 2007 
%        \end{valuelist}
%
%\optitem[plain]{volnumformat}{\opt{strings}, \opt{parens}, \opt{plain}}
%This option provides three styles for typesetting the ``volume/number'' block in  \bibtype{article} entries.
%\begin{valuelist}
%        \item[plain]
%              \ldots \emph{Journal Title}, 5, 8, \ldots 
%        \item[strings] 
%              \ldots \emph{Journal Title}, vol.~5, n.~8, \ldots
%        \item[parens]
%              \ldots \emph{Journal Title} (5, 8), \ldots 
%        \end{valuelist}
%
%\optitem[arabic]{volumeformat}{\opt{arabic}, \opt{roman}, \opt{romansc}, \opt{Roman}}
%This option provides three styles for typesetting the \bibfield{volume} field:
%        \begin{valuelist}
%        \item[arabic]
%                \ldots \emph{Book Title}, vol. 12, \ldots 
%          \item[roman] 
%                \ldots \emph{Book Title}, vol. xii, \ldots 
%          \item[romansc]
%                \ldots \emph{Book Title}, vol. \textsc{xii}, \ldots 
%          \item[Roman]
%                \ldots \emph{Book Title}, vol. XII, \ldots 
%          \end{valuelist}
%
%^^AWith the command
%^^A\begin{latexcode}
%^^A\begin{verbatim}
%^^A\DeclareFieldFormat{volume}{\bibstring{volume}~#1}
%^^A\end{verbatim}
%^^A\end{latexcode}
%^^Ayou can restore the default format for all but \bibtype{article} entries.
%
%\optitem[arabic]{editionformat}{\opt{arabic}, \opt{roman}, \opt{romansc}, \opt{Roman}, \opt{superscript}}
%This option provides three styles for typesetting the \bibfield{edition} field:
%        \begin{valuelist}
%        \item[arabic] 
%              \ldots \emph{Book Title}, 3\textsuperscript{a} ed., \ldots 
%          \item[roman] 
%              \ldots \emph{Book Title}, iii ed., \ldots 
%          \item[romansc] 
%              \ldots \emph{Book Title}, \textsc{iii} ed., \ldots 
%          \item[Roman] 
%              \ldots \emph{Book Title}, III ed., \ldots 
%          \item[superscript] [only for \sty{philosophy-verbose}]\\
%              \ldots \emph{Book Title}, Publisher, Location 2010\textsuperscript{3}. 
%          \end{valuelist}
%
%\optitem[false]{scauthors}{\opt{bib}, \opt{cite}, \opt{bibcite}, \opt{citefn}, \opt{bibcitefn}, \opt{all}}
% Prints some (or all) names in small caps shape.
% If you want \emph{all} the names (translator, commentator, etc.) in small caps, 
% you have to redefine the \cmd{mkbibname*} commands (see \sty{biblatex} documentation for details).
%        \begin{valuelist}
%        \item[bib] Small caps only for the names at the beginning of the entry in the bibliography.
%        \item[cite] Small caps only for the names at the beginning of the entry in the citations.
%        \item[bibcite] Small caps only for the names at the beginning of the entry both in bibliography and citations.
%        \item[citefn] Small caps only for the names at the beginning of the entry in the citations inside footnotes.
%        \item[bibcitefn] Small caps only for the names at the beginning of the entry both in bibliography and citations inside footnotes.
%        \item[all] Small caps for \emph{all} the names both in bibliography and citations.
%        \end{valuelist}
%
%\boolitem[false]{lowscauthors}
% Prints the initials of the names in lowercase small capitals.
% For example you will obtain \textsc{donald e. knuth} instead of \textsc{Donald E. Knuth}. 
%
%\boolitem[true]{shorthandintro}
%
%Prints a language-specific expression such as ``henceforth cited as \meta{shorthand}'' to introduce shorthands on the first citation. 
%      \begin{bibexample}
%      Kant (1968a \emphasize{[henceforth cited as KpV]})
%      \end{bibexample}
%You can overwrite the default expression using the \bibfield{shorthandintro}. Note that the alternative expression must include the shorthand.
% Obviously, if you do not use an intro to the shorthands
% you will need a list of shorthands (\cmd{printshorthand} command).
%
%\boolitem[false]{inbeforejournal}
%Prints the string ``in'' before the \bibfield{journaltitle} in the  \bibtype{article} entries.
%
%\boolitem[false]{annotation}
% Shows the \bibfield{annotation} field, only in the bibliography% (see also section \ref{sec:newfields}). 
% This option can be given globally or on a per-bibliography basis:
%\begin{latexcode}
%\begin{verbatim}
%\printbibliography[annotation=true]
%\end{verbatim}
%\end{latexcode}
%
%\boolitem[true]{library}
%Shows the \bibfield{library} field, both in the bibliography and in the citations% (see also section \ref{sec:newfields}). 
%
%\boolitem[false]{classical}
%
% [Only for Italian documents] It requires \sty{babel} or \sty{polyglossia}.
% If \opt{true} it doubles the last consonant of the abbreviations such as ``p.'', ``vol.'', ``col.'' etc. when used in the plural form. For example you will have ``p.'' for ``page'' and ``pp.'' for ``pages''. This habit is very common in Italian writings even if it remains useless. 
%^^A A similar setting is adopted for the default English and Spanish localization modules which, for example, use ``cols.'' for ``columns'' and ``pp'' for ``pages''. 
% 
% \end{optionlist}
%
%\subsection{Style-specific}
%
%
%
%\subsubsection[For \sty{classic} and 
%  \sty{modern}]{Options for \sty{philosophy-classic} and 
%  \sty{philosophy-modern}}\label{sec:options-classic-modern}
%
%
%
% \begin{optionlist}
%
%\boolitem[false]{latinemph}
%Prints the \emph{latinitas} ``et al.'' (\emph{et alii}) in italic shape.
%
%
%\boolitem[false]{square}
%
%Uses brackets instead of parentheses in the citations and in the author-year label used in the bibliography.
%
%\boolitem[true]{nodate} Prints the \texttt{nodate} string when \bibfield{year} or \bibfield{date} is missing. Yo can set this option globally in the package options or in the optional argument of \cmd{printbibliography}.
%
%
%\boolitem[false]{yearleft}
%
% [\sty{philosophy-modern} only]
%
%Prints the  date flushed left in the bibliography.
%
%
%\boolitem[true]{restoreclassic}
%
% [\sty{philosophy-modern}  and \sty{philosophy-classic} only]
%
%This option can be given in the optional argument of \cmd{printbibliography}.
%It restores the \sty{classic} style in a document typeset using  the  \sty{modern} style. It is useful to compose a ``Web List'' like that at the end of this document. For example:
%\begin{latexcode}
%\begin{verbatim}
%\printbibliography[restoreclassic,type=online]
%\end{verbatim}
%\end{latexcode}
%
% \end{optionlist}
%
%\subsubsection[For \sty{verbose}]{Options for \sty{philosophy-verbose}}\label{sec:options-verbose}
%
% \begin{optionlist}
%
%\boolitem[false]{latinemph}
%Prints the \emph{latinitates} ``{ivi}'', ``{ibidem}'' and ``et al.'' in italic shape.
%
%\boolitem[false]{commacit}
%Adds a comma at the end of the \bibfield{shorttitle} field when this is followed by the string \emph{cit.}: 
%``Descartes, \emph{Discours de la méthode}\emphasize{,} cit.''.
%\end{optionlist}
%
%  
%
%
%\section{Customizations}\label{sec:customization}
% Here we introduce the new commands and lenghts provided by \sty{biblatex-philosophy}. The \sty{biblatex} package offers other commands, lenghts and options to modify many aspects of citations and bibliography. See the \sty{biblatex} documentation for details.
%\subsection{Fonts and punctuation}
%\begin{ltxsyntax}
%\cmditem{annotationfont} \hfill default: \cmd{footnotesize}
%
%The font of the \bibfield{annotation} field. It can be redefined with:
%\begin{latexcode}
%\begin{verbatim}
%\renewcommand*{\annotationfont}{\normalsize\sffamily}
%\end{verbatim}
%\end{latexcode}
%
%\cmditem{libraryfont} \hfill default: \cmd{normalfont}
%
%The font of the \bibfield{library} field. It can be redefined with:
%
%\begin{latexcode}
%\begin{verbatim}
%\renewcommand*{\libraryfont}{\sffamily}
%\end{verbatim}
%\end{latexcode}
%\end{ltxsyntax}
%
%\begin{ltxsyntax}
%\cmditem{volnumpunct} \hfill default: \cmd{addcomma}\cmd{space}
% 
%The separator between \bibfield{volume} and \bibfield{number} in \bibtype{article} entries.
%It can be redefined with:
%\begin{latexcode}
%\begin{verbatim}
%\renewcommand*{\volnumpunct}{/}
%\end{verbatim}
%\end{latexcode}
%Combining this with the \opt{volnumformat} and \opt{volumeformat} options 
% you can get other styles for volume and number. For example:\\
%  \begin{bibexample}
%  \ldots\ \emph{Journal Title}, 5/8, \ldots \\
%  \ldots\ \emph{Journal Title}, \textsc{V}/8, \ldots \\
%  \ldots\ \emph{Journal Title} (5/8), \ldots \\
%  \ldots\ \emph{Journal Title} (\textsc{V}/8), \ldots
%  \end{bibexample}
%
%\cmditem{editorstrgdelim} \hfill default: \cmd{addspace}
%
% The separator to be printed after the \texttt{editorstrg}, \texttt{authorstrg} and \texttt{translatorstrg} strings, 
% which are enclosed in parentheses by default. If you want omit the parentheses you should also change it as follows:
%\begin{latexcode}
%\begin{verbatim}
%\renewcommand*{\editorstrgdelim}{\addcomma\space}
%\DeclareFieldFormat{editortype}{#1}% no parentheses
%\end{verbatim}
%\end{latexcode}
%\end{ltxsyntax}
%
%\subsection{Lengths}\label{sec:lengths}
%
%
% These lengths are (re)defined only for the \sty{modern} style. It  introduces two new lengths: 
%
%\begin{ltxsyntax}
%\lenitem{postnamesep}  
%
%The space between author (or editor) and the first entry relating to him.
%
%\lenitem{yeartitle} 
%
%The space between year and title. 
%\end{ltxsyntax}
%
%It also redefines the following \sty{biblatex} lengths:
%\begin{ltxsyntax}
%\lenitem{bibnamesep} The vertical space between two blocks of authors.
%\lenitem{bibitemsep} The vertical space between the individual entries in the bibliography.
%
%\lenitem{bibhang} The hanging indentation of the bibliography.
%
%\end{ltxsyntax}
%
%These are the default values for the lengths used by the \sty{modern} style.
% You can change them according to your specific needs.
%\begin{latexcode}
%\begin{verbatim}
%\setlength{\yeartitle}{0.8em}
%\setlength{\postnamesep}{0.5ex plus 2pt minus 1pt}
%\setlength{\bibitemsep}{\postnamesep}
%\setlength{\bibnamesep}{1.5ex plus 2pt minus 1pt}
%\setlength{\bibhang}{4\parindent}
%\end{verbatim}
%\end{latexcode}
%  
%
% \subsection{Date and page ranges}
%
% These style redefines the \cmd{bibrangedash} and \cmd{bibdaterangesep} commands in order to get a simple hyphen (-) instead of an en dash (--) in the page and date ranges. If you prefer the en dash use the following code:
%\begin{latexcode}
%\begin{verbatim}
%\DefineBibliographyExtras{<language>}{%
%  \protected\def\bibrangedash{%
%    \textendash\penalty\hyphenpenalty}%
%  \protected\def\bibdaterangesep{\bibrangedash}}%
%\end{verbatim}
%\end{latexcode}
% For a consistent result you probably have to do this for all the languages loaded by \sty{babel} or \sty{polyglossia}.
%
%\subsection[Languages]{Using the styles with other languages}\label{sec:languages}
%
% The languages currently supported by this package  are Italian, English, Spanish and French. In order to use the styles with different languages, you have first of all to declare the new \opt{opcited} string introduced by \sty{biblatex-philosophy}. You can then test the styles and if the default strings provided in the localization module does not match your needs you can re-define them. 
%
%Here is a sample code for using the styles in German documents. Note that we first declare the new string \opt{opcited}, then we define it and inherit the German default strings from \file{german.lbx}. The other strings (\opt{translationas}, \opt{ibidem}, \opt{loccit}, \dots) may be re-defined if the default ones are not satisfying. For example you may prefer ``deut. \"Ubers'' to the default ``\"Ubers unter dem Titel''. % Another approach is to use the \cmd{DeclareLanguageMapping} command. See the documentation of the \sty{biblatex} package for details \parencite{lehman:biblatex}.
%\begin{latexcode}
%\begin{verbatim}
%\NewBibliographyString{opcited}
%\DefineBibliographyStrings{german}{%
%  inherit       = {german},
%  opcited       = {op\adddotspace cit\adddot},
%  translationas = {deut\adddotspace \"Ubers\adddot},
%  ...other strings...
%}
%\end{verbatim}
%\end{latexcode}
%
% The French default localization module redefines, among others, the \cmd{mkbibnamefamily} command in order to get the family name in small caps shape. We do not like this approach because an author could use a localization module without adhering to the typographical standards which should be indipendent from the linguistic standards. For this reason we have reset it to the default definition.  If you prefer the \file{french.lbx} choice use this code:
%\begin{latexcode}
%\begin{verbatim}
%\DefineBibliographyExtras{french}{%
%  \protected\def\mkbibnamefamily#1{%
%    \textsc{\textnohyphenation{#1}}}%
%  \protected\def\bibrangedash{%
%    \textendash\penalty\hyphenpenalty}}%
%\end{verbatim}
%\end{latexcode}
%
%
% 
%\section{Backward compatibility}
%
%Previous versions of the styles provided a different mechanism to manage entries comprising both the original publication data and the translation data. This feature is now deprecated and it is still supported only for backward compatibility. This mechanism uses some special fields and provides specific options.
%
%\subsection{Deprecated fields}\label{sec:origfields}
%
%The following fields can hold the translation or the original edition data. They are precede by the string ``trans.'' or ``orig. ed.'', respectively according to the \opt{origfields=trans} (default)  or \opt{origfields=origed} option (see below). Note that the \bibfield{origdate/transdate} field is needed in order to print these fields. Contrarily they will be ignored.

%
%\begin{fieldlist}
%
%\fielditem{origtitle}{literal}\mbox{}\\[-9ex] 
%\fielditem{transtitle}{literal}  
%
%The title of the translation/original edition.
%
%\fielditem{origpublisher}{list}\mbox{}\\[-9ex] 
%\fielditem{transpublisher}{list}  
%
%The publisher of the translation/original edition.
%
%\fielditem{origlocation}{list}\mbox{}\\[-9ex] 
%\fielditem{translocation}{list}  
%
%The location of the translation/original edition.
%
%\fielditem{origdate}{range}\mbox{}\\[-9ex] 
%\fielditem{transdate}{range} 
 
%
%The publication date of the translation/original edition.
%^^A\begin{bibexample}
%^^ARobert Bringhurst (1992), \emph{The Elements of Typographic Style}, Hartley \& Marks Publisher Inc., Vancouver, Canada; trad. it. \emphasize{\emph{Gli elementi dello stile tipografico}, Sylvestre Bonnard, Milano 2009.}
%^^A\end{bibexample}
%
%\fielditem{reprinttitle}{literal} 
% The title of a reprint of the work.
%\fielditem{usera}{literal}\mbox{}\\[-9ex] 
%\fielditem{origbooktitle}{literal}\mbox{}\\[-9ex]
%\fielditem{transbooktitle}{literal} 
%
%The title of the \bibtype{collection}/\bibtype{book}/\bibtype{mvbook} in which the translation/original edition  of an \bibtype{article} (\bibtype{inbook} or \bibtype{incollection}) is published. 
%
% The field is printed after the \bibfield{origtitle/transtitle}.
%^^A\vbox{
%^^A    \begin{bibexample}
%^^A     Moore, George Edward (1903), «The refutation of idealism», \emph{Mind}, N.S., 12 (mag. 1903), p. 433-453; trad. it. «La confutazione dell'idealismo», in \emphasize{\emph{Il Neoempirismo}}, a cura di Alberto Pasquinelli, UTET, Torino 1969, p. 35-61.
%^^A    \end{bibexample}}
%
%\fielditem{userb}{literal}\mbox{}\\[-9ex] 
%\fielditem{orignote}{literal}\mbox{}\\[-9ex]
%\fielditem{transnote}{literal} 
%
%This field is printed after the \bibfield{origtitle/transtitle}. It is meat for secondary informations about the translation/original edition, such as the name of editors, translators, etc.:
%^^A    \begin{bibexample}
%^^A       Mach, E. (1883), \emph{Die Mechanik in ihrer Entwickelung historisch-kritisch dargestellt}; trad. it. \emph{La meccanica nel suo sviluppo storico-critico}, \emphasize{traduzione, introduzione e note di Alfonsina D'Elia}, Bollati Boringhieri, Torino 1977.
%^^A    \end{bibexample}
%    
%\fielditem{userc}{literal}\mbox{}\\[-9ex]
%\fielditem{origpages}{literal}\mbox{}\\[-9ex]
%\fielditem{transpages}{literal}
%
%This field is printed at the end of the entry, after the  \bibfield{origdate/transdate} field. It is meant for the page range of the translation/original edition or other useful informations.
%In the first case string ``p.'' is omitted.
%\end{fieldlist}
%
%
%
%\subsection{Deprecated options}
%
%\begin{optionlist}
%\optitem[trans]{origfields}{\opt{trans}, \opt{none}, \opt{edorig}}
%\begin{valuelist}
%\item[true] Prints the \bibfield{orig-} fields.
%\item[none] Omits the \bibfield{orig-} fields.
%\item[origed] This option cites the translation data first and adds the original publication data at the end of the entry, preceded by the string ``orig.~ed'' (or ``ed.~orig.'' for Italian  documents).
%\end{valuelist}
%
%\boolitem[true]{origed}
%
%Same as the previous but can be set on a per-entry basis in the \opt{options} field.
%
%\optitem[semicolon]{origfieldsformat}{\opt{semicolon}, \opt{parens}, \opt{brackets}}
% Deprecated. Use the \opt{relatedformat} option instead.
%\begin{valuelist}
%\item[semicolon] 
%The translation/original publication data are preceded by a semicolon.
%\item[parens] 
%Puts the translation or the original publication data  in parentheses.
%\item[brackets] 
%Same as the previous option but with brackets instead of parentheses.
%\end{valuelist}
%\boolitem[false]{scauthorsbib}
%Same as \opt{scauthors=bib}
%\boolitem[false]{scauthorscite}
%Same as \opt{scauthors=cite}
%\boolitem[false]{scauthors}
%Same as \opt{scauthors=bibcite}
%\end{optionlist}
%
%^^A\section{Known issues}
%^^A
%^^A The \cmd{textcite} command has an unexpected behaviour in managing entries with a shorthand defined. Note that \cmd{cite} and \cmd{parencite} prints correctly the shorthand, while \cmd{textcite} prints the name plus the shorthand. Anyway this is a bug (or a feature) of \file{authoryear-comp.cbx}.
%^^A \begin{description}
%^^A\item[\cmd{cite}\ar{kant:kpv}] \cite{kant:kpv}
%^^A\item[\cmd{parecite}\ar{kant:kpv}] \parencite{kant:kpv}
%^^A\item[\cmd{texcite}\ar{kant:kpv}] \textcite{kant:kpv}
%^^A\end{description}
%^^A\smallskip
%^^A
%^^A Some spurious spaces could come out in entries without author/editor label such as \cite{jcv}. Note the space preceding ``35''.
%^^A 
%
% \defbibnote{restoreclassic}{\sffamily\small This is the primary bibliography of this document and is typeset in \sty{classic} style (through the \opt{restoreclassic} option) even if the bibliography style of the document is \sty{philosophy-modern}. This is particularly useful for typesetting bibliographies in which there are only one entry for an author, such as the Web lists, as shown below.}
%
% \defbibnote{weblist}{\sffamily\small Here we have a list of Web sites typeset in the \sty{classic} style through the \opt{restoreclassic} option. Only the \bibtype{online} entries are printed.}
%
%\defbibnote{biblatex-examples}{\sffamily\small The source of this bibliography, typeset in the \sty{modern} style, is the \file{biblatex-examples.bib} database, distributed with the \sty{biblatex} package. It is provided for checking all the standard features. This list could highlight some bugs.}
%
%\defbibnote{philosophy-examples}{\sffamily\small The source of this bibliography, typeset in the \sty{modern} style, is the \file{biblatex-philosophy.bib} database, distributed with the \sty{biblatex-philosophy} package. It is provided for checking all the style-specific features. This list should not highlight any bugs.}
%
%
%\printbibliography[heading=bibintoc,keyword=primaria,restoreclassic,prenote=restoreclassic]
%
%\printbibheading[heading=bibintoc,title=Examples]
%
%\printbibliography[heading=subbibliography,title=A Web List,keyword=web,restoreclassic,prenote=weblist]
%
%\printbibliography[heading=subbibliography,title=Philosophy examples,prenote=philosophy-examples,keyword=esempio]
%
%\nocite{set,stdmodel,matuz:doody,vizedom:related,britannica,gaonkar,jaffe,westfahl:frontier,cms,ctan,jcg,yoon,worman,wilde,westfahl:space,weinberg,wassenberg,vazques-de-parga,springer,spiegelberg,sorace,sigfridsson,shore,sarfraz,salam,reese,pines,piccato,padhye,nussbaum,nietzsche:ksa,nietzsche:ksa1,nietzsche:historie,moraux,moore,moore:related,massa,maron,markey,malinowski,loh,laufenberg,kullback,kullback:reprint,kullback:related,kowalik,knuth:ct,knuth:ct:a,knuth:ct:b,knuth:ct:c,knuth:ct:d,knuth:ct:e,knuth:ct:related,kastenholz,kant:kpv,kant:ku,itzhaki,hyman,murray,iliad,herrmann,hammond,companion,gonzalez,glashow,gillies,gerhardt,vangennep,vangennep:trans,vangennep:related,geer,gaonkar:in,doody,cotton,coleridge,cicero,chiu,brandt,bertram,baez/article,baez/online,averroes/bland,averroes/hannes,averroes/hercz,augustine,aristotle:anima,aristotle:physics,aristotle:poetics,aristotle:rhetoric,angenendt,almendro,aksin}
%
%\printbibliography[heading=subbibliography,title=Biblatex examples,prenote=biblatex-examples,category=biblatex]
%
%
% \StopEventually{\PrintChanges\PrintIndex}
%
%
% \iffalse
%<*standard-bbx>
% \fi
%
% \section{The Code}
%
% \subsection{\file{philosophy-standard.bbx}}
% \subsubsection{Initial settings}
%
% Biber is the default bibliography processor for \sty{biblatex}.
% The \sty{philosophy} styles could work without Biber (excluding the experimental \bibtype{jurisprudence} driver) 
% but it is required because it offers many useful functionalities. The \opt{backend=bibtex} or \opt{backend=bibtex8} options 
% produce an error message.
%    \begin{macrocode}
\RequireBiber[3]
%    \end{macrocode}
% The styles are base on \sty{standard} \sty{biblatex} default style.
%    \begin{macrocode}
\RequireBibliographyStyle{standard}
%    \end{macrocode}
% A command to get an error message if you use an unknown value 
% for an option. 
%    \begin{macrocode}
\def\optionerror#1{%
  \ClassError{biblatex-philosophy}
  {\MessageBreak**** Unknown value for '#1' option}
  {\MessageBreak**** Unknown value for '#1' option}}
%    \end{macrocode}
% The \sty{philosophy} styles
% redefine some localized strings for Italian, English, Spanish and French in specific 
% localization modules. So we declare and map them to the associated languages.
%    \begin{macrocode}
\DeclareLanguageMapping{italian}{italian-philosophy}
\DeclareLanguageMapping{english}{english-philosophy}
\DeclareLanguageMapping{spanish}{spanish-philosophy}
\DeclareLanguageMapping{french}{french-philosophy}
%    \end{macrocode}
% The default value for the boolean options is |true|.
% This means that giving the options without the value is just like giving
% \opt{option=true.}
%    \begin{macrocode}
\newtoggle{bbx:annotation}
\newtoggle{bbx:library}
\newtoggle{bbx:inbeforejournal}
\newtoggle{bbx:classical}
\newtoggle{bbx:lowscauthors}
\newtoggle{cbx:shorthandintro}
\newtoggle{cbx:scauthorscite}
\newtoggle{bbx:scauthorsbib}
\newtoggle{cbx:scauthorscitefn}
\newtoggle{cbx:latinemph}

\DeclareBibliographyOption{annotation}[true]{%
  \settoggle{bbx:annotation}{#1}}
\DeclareBibliographyOption{library}[true]{%
  \settoggle{bbx:library}{#1}}
\DeclareBibliographyOption{inbeforejournal}[true]{%
  \settoggle{bbx:inbeforejournal}{#1}}
\DeclareBibliographyOption{classical}[true]{%
  \settoggle{bbx:classical}{#1}}
\DeclareBibliographyOption{lowscauthors}[true]{%
  \settoggle{bbx:lowscauthors}{#1}}
\DeclareBibliographyOption{shorthandintro}[true]{%
  \settoggle{cbx:shorthandintro}{#1}}
\DeclareBibliographyOption{latinemph}[true]{%
  \settoggle{cbx:latinemph}{#1}}
%    \end{macrocode}
% Also the multi-value options have a default value, which is declared
% in the optional bracketed argument of the \cmd{DeclareBibliographyOption} commands below. 
% For example, the new \opt{scauthors} option is now multi-value and defaults to \opt{all}. 
% So \opt{scauthors=all} is the same of \opt{scauthors}. In this way this option
% works exactly like the old \opt{scauthors} boolean option that for this reason has been erased.
%    \begin{macrocode}
\newcommand{\bbx@publocformat}{}
\newcommand{\bbx@volnumformat}{}
\newcommand{\bbx@relatedformat}{}
\newcommand{\bbx@editionformat}{}
\newcommand{\bbx@volumeformat}{}
\newcommand{\bbx@scauthors}{}
\DeclareBibliographyOption{publocformat}[publocyear]{%
  \renewcommand{\bbx@publocformat}{#1}}
\DeclareBibliographyOption{volnumformat}[plain]{%
  \renewcommand{\bbx@volnumformat}{#1}}
\DeclareBibliographyOption{origfieldsformat}[semicolon]{%
  \renewcommand{\bbx@relatedformat}{#1}}
\DeclareBibliographyOption{relatedformat}[semicolon]{%
  \renewcommand{\bbx@relatedformat}{#1}}
\DeclareBibliographyOption{origfields}[true]{%
  \renewcommand{\bbx@origfields}{#1}}
\DeclareBibliographyOption{editionformat}[arabic]{%
  \renewcommand{\bbx@editionformat}{#1}}
\DeclareBibliographyOption{volumeformat}[arabic]{%
  \renewcommand{\bbx@volumeformat}{#1}}
\DeclareBibliographyOption{scauthors}[all]{%
  \renewcommand{\bbx@scauthors}{#1}}
%    \end{macrocode}
% These options are defined for backwards compatibility. The \opt{origed} is now useless and it is substituted by the `related' mechanism. The \opt{scauthorscite} and \opt{scauthorsbib} are substituted by \opt{scauthors=cite} and \opt{scauthors=bib}, respectively.
%    \begin{macrocode}
\newcommand{\bbx@origfields}{}
\DeclareEntryOption{origed}[true]{%
  \renewcommand{\bbx@origfields}{origed}}
\DeclareBibliographyOption{scauthorsbib}[true]{%
  \settoggle{bbx:scauthorsbib}{#1}}
\DeclareBibliographyOption{scauthorscite}[true]{%
  \settoggle{cbx:scauthorscite}{#1}}
%    \end{macrocode}
% And now one option to be used in the \cmd{printbibliography} and \cmd{printbiblist} commands. 
%    \begin{macrocode}
\define@key{blx@bib1}{annotation}[]{}
\define@key{blx@bib2}{annotation}[true]{\settoggle{bbx:annotation}{#1}}
\define@key{blx@biblist1}{annotation}[]{}
\define@key{blx@biblist2}{annotation}[true]{\settoggle{bbx:annotation}{#1}}
%    \end{macrocode}
% Now we can execute the default options.
%    \begin{macrocode}
\ExecuteBibliographyOptions{%
  publocformat     = publocyear,  
  volnumformat     = plain,
  origfieldsformat = semicolon,
  relatedformat    = semicolon,
  editionformat    = arabic,
  volumeformat     = arabic,
  scauthors        = false,
  editionformat    = arabic,
  volumeformat     = arabic,
  shorthandintro   = true,
  library          = true,
  annotation       = false,
  latinemph        = false,
  classical        = false,
  inbeforejournal  = false,
  lowscauthors     = false,
  useprefix        = true,
  maxcitenames     = 2,
  mincitenames     = 1,
  maxbibnames      = 999,
  minbibnames      = 999}
%    \end{macrocode}
% Changing the penalty of the urls will prevent 
% many overfull boxes:
%    \begin{macrocode}
\setcounter{biburlnumpenalty}{9000}
\setcounter{biburlucpenalty}{9000}
\setcounter{biburllcpenalty}{9000}
%    \end{macrocode}
% These counters control the list of names
% in the cross-referenced entries:
%    \begin{macrocode}
\newcounter{maxnamesincross}
\newcounter{minnamesincross}
%    \end{macrocode}
% The \opt{scauthors} and \opt{lowscauthors} options are based on tests that require to be executed inside a command,
% a macro or \cmd{AtBeginDocument} and similar hooks. Otherwise they would produce an error message.
%    \begin{macrocode}
\AtBeginDocument{%
\ifdefstring{\bbx@scauthors}{bibcite}
{%
  \toggletrue{bbx:scauthorsbib}%
  \toggletrue{cbx:scauthorscite}%
}%
{}%
\ifdefstring{\bbx@scauthors}{bib}
{%
  \toggletrue{bbx:scauthorsbib}%
}%
{}%
\ifdefstring{\bbx@scauthors}{cite}
{%
  \toggletrue{cbx:scauthorscite}%
}%
{}%
\ifdefstring{\bbx@scauthors}{citefn}
{%
  \toggletrue{cbx:scauthorscitefn}%
}%
{}%
\ifdefstring{\bbx@scauthors}{bibcitefn}
{%
  \toggletrue{bbx:scauthorsbib}%
  \toggletrue{cbx:scauthorscitefn}%
}%
{}%
\ifdefstring{\bbx@scauthors}{all}
{%
  \usebibmacro{scswitch}
}%
{}%
}
%    \end{macrocode}
% With the \opt{scauthors=cite} option all the citations are printed in small caps.
% Anyway we do not like small caps in the citations inside the bibliography so we deactivate this option at the beginning of the bibliography.
%    \begin{macrocode}
\AtBeginBibliography{%
  \togglefalse{cbx:scauthorscite}%
  \togglefalse{cbx:shorthandintro}%
}
%    \end{macrocode}
% The \bibfield{annotation} field and the shorthand intro are omitted in the list of shorthands.
%    \begin{macrocode}
\AtBeginShorthands{%
  \togglefalse{bbx:annotation}%
  \togglefalse{cbx:shorthandintro}%
}
%    \end{macrocode}
% The \bibfield{annotation} field is omitted in every citations.
%    \begin{macrocode}
\AtEveryCite{%
  \togglefalse{bbx:annotation}%
}
%    \end{macrocode}
%
% \subsubsection{New commands}
%
% The \cmd{mkibid} command is provided for formatting the \emph{latinitates} ``et al.'', ``ivi'', ``ibidem''.
% Actually the command is introduce for formatting ``et al.'' considering that it is already defined by \sty{verbose-trad2.cbx}
% which uses it for ``ivi'' and ``ibidem''.
%    \begin{macrocode}
\providecommand*{\mkibid}[1]{\iftoggle{cbx:latinemph}{\emph{#1}}{#1}}
%    \end{macrocode}
% We (re)define some internal commands for the punctuation. The new \cmd{volnumpunct} command is provided 
% to separate volume and number in \bibtype{article} entries. 
%    \begin{macrocode}
\newcommand*{\volnumpunct}{\addcomma\space}
\renewcommand*{\newunitpunct}{\addcomma\space}
\renewcommand*{\subtitlepunct}{\addperiod\space}
\renewcommand*{\intitlepunct}{\nopunct\addspace}
\renewcommand*{\relatedpunct}{\addsemicolon\space}
%    \end{macrocode}
% The \cmd{editorstrgdelim} is introduced to customize the
% delimiter to be printed before the  |editorstrg|, |authorstrg| and |translatorstrg| strings. These strings are enclosed in parentheses by default: (eds.), (trans.), etc. Redefining the delimiter we can omit the parentheses end reset to the default authoryear style: eds., trans., etc. This requires to change the \bibfield{editortype} field format too.
%    \begin{macrocode}
\DeclareDelimFormat{editorstrgdelim}{\addspace}
%    \end{macrocode}
% New internal commands assure pure parentheses/brackets for some specific fields
% when using the \opt{square} option.
%    \begin{macrocode}
\newrobustcmd*{\mkpureparens}[1]{%
  \begingroup
    \blx@blxinit
    \blx@setsfcodes
    \bibleftparen#1\bibrightparen%
  \endgroup}
\newrobustcmd*{\mkpurebrackets}[1]{%
  \begingroup
    \blx@blxinit
    \blx@setsfcodes
    \bibleftbracket#1\bibrightbracket%
  \endgroup}
%    \end{macrocode}
% Some commands for changing the font of the \bibfield{annotation}, \bibfield{library} and \bibfield{edition} fields.
%    \begin{macrocode}
\newcommand*{\annotationfont}{\footnotesize}
\newcommand*{\libraryfont}{}
\newcommand*{\editionfont}{%
  \ifdefstring{\bbx@editionformat}{Roman}
    {\uppercase}%
    {\ifdefstring{\bbx@editionformat}{romansc}
      {\scshape}%
      {\relax}}}%
\newrobustcmd*{\edfnt}[1]{%
  \begingroup
  \expandafter\editionfont%
  \expandafter{\romannumeral#1}%
  \endgroup}
%    \end{macrocode}
% A command to select lowercase small caps.
%    \begin{macrocode}
\newrobustcmd*{\mkbibsc}[1]{%
  \iftoggle{bbx:lowscauthors}{%
    \textsc{\MakeLowercase{#1}}}%
    {\textsc{#1}}}
%    \end{macrocode}
%
% \subsubsection{Names format}
%
% First we define a macro to be used in the \cmd{DeclareNameFormat} specifications. The macro simply maps the \cmd{mkbibname*} commands to the new \cmd{mkbibsc} command defined above. 
%    \begin{macrocode}
\newbibmacro*{scswitch}{%
  \let\mkbibnamefamily\mkbibsc%
  \let\mkbibnamegiven\mkbibsc%
  \let\mkbibnameprefix\mkbibsc%
  \let\mkbibnamesuffix\mkbibsc}
%    \end{macrocode}
% In the following codes note that the font switching is declared inside \texttt{sortname} or \texttt{labelname}
% because the \opt{scauthors=bib} or \opt{scauthors=cite} option must be active 
% only for the names at the beginning of the entry which are formatted by |sortname| or |labelname|.
%    \begin{macrocode}
\DeclareNameFormat{sortname}{%
  \iftoggle{bbx:scauthorsbib}{\usebibmacro{scswitch}}{}%
  \nameparts{#1}%
  \ifnumequal{\value{listcount}}{1}
  {\ifgiveninits
    {\usebibmacro{name:family-given}
      {\namepartfamily}
      {\namepartgiveni}
      {\namepartprefix}
      {\namepartsuffix}}
    {\usebibmacro{name:family-given}
      {\namepartfamily}
      {\namepartgiven}
      {\namepartprefix}
      {\namepartsuffix}}%
    \ifboolexpe{%
      test {\ifdefvoid\namepartgiven}
      and
      test {\ifdefvoid\namepartprefix}}
    {}
    {\usebibmacro{name:revsdelim}}}
  {\ifgiveninits
    {\usebibmacro{name:given-family}
      {\namepartfamily}
      {\namepartgiveni}
      {\namepartprefix}
      {\namepartsuffix}}
    {\usebibmacro{name:given-family}
      {\namepartfamily}
      {\namepartgiven}
      {\namepartprefix}
      {\namepartsuffix}}}%
  \usebibmacro{name:andothers}}%
\DeclareNameFormat{labelname}{%
  \iftoggle{cbx:scauthorscite}{\usebibmacro{scswitch}}{}%  
  \iftoggle{cbx:scauthorscitefn}{\iffootnote{\usebibmacro{scswitch}}{}}{}%
  \bibhyperref{\nameparts{#1}%
  \ifcase\value{uniquename}%
    \usebibmacro{name:family}%
      {\namepartfamily}%
      {\namepartgiven}%
      {\namepartprefix}%
      {\namepartsuffix}%
  \or
    \ifuseprefix
      {\usebibmacro{name:given-family}%
        {\namepartfamily}%
        {\namepartgiveni}%
        {\namepartprefix}%
        {\namepartsuffixi}}%
      {\usebibmacro{name:given-family}%
        {\namepartfamily}%
        {\namepartgiveni}%
        {\namepartprefixi}%
        {\namepartsuffixi}}%
  \or
    \usebibmacro{name:given-family}%
      {\namepartfamily}%
      {\namepartgiven}%
      {\namepartprefix}%
      {\namepartsuffix}%
  \fi
  \usebibmacro{name:andothers}}}%
%    \end{macrocode}
% The \texttt{scdefauld} name format is used in the \texttt{cite:full} macro below to controll the small caps in the first citation of an antry (that is a full citation).
%    \begin{macrocode}
\DeclareNameFormat{scdefault}{%
\usebibmacro{scswitch}%
  \nameparts{#1}%
  \ifgiveninits
    {\usebibmacro{name:given-family}%
      {\namepartfamily}%
      {\namepartgiveni}%
      {\namepartprefix}%
      {\namepartsuffix}}%
    {\usebibmacro{name:given-family}%
      {\namepartfamily}%
      {\namepartgiven}%
      {\namepartprefix}%
      {\namepartsuffix}}%
  \usebibmacro{name:andothers}}%
%    \end{macrocode}
% The \cmd{fullcite} command employs the bibliography driver to print the entry
% so it has to be redefined in order to use the |scdefault| name format
% with \opt{scauthor=cite} or \opt{scauthor=full} options.
%    \begin{macrocode}
\DeclareCiteCommand{\fullcite}
  {\usebibmacro{prenote}}
  {\setkeys{blx@bib2}{restoreclassic}%
  \usedriver
    {\iftoggle{cbx:scauthorscite}%
      {\DeclareNameAlias{sortname}{scdefault}}%
      {\DeclareNameAlias{sortname}{default}}}%
  {\thefield{entrytype}}}
  {\multicitedelim}
  {\usebibmacro{postnote}}
%    \end{macrocode}
% \subsubsection{Fields format}
%    \begin{macrocode}
\DeclareFieldFormat[bookinbook,thesis]{title}{\mkbibemph{#1}}
\DeclareFieldFormat[review]{title}{\bibcplstring{reviewof}\addspace#1}
\DeclareFieldFormat[review]{citetitle}{\bibcplstring{reviewof}\addspace#1}
\DeclareFieldFormat[inreference,article]{title}{\mkbibquote{#1}}
\DeclareFieldFormat[bookinbook,thesis]{citetitle}{\mkbibemph{#1}}
\DeclareFieldFormat{origtitle}{\mkbibemph{#1}}
\DeclareFieldFormat[article]{origtitle}{\mkbibquote{#1}}
\DeclareFieldFormat{usera}{\mkbibemph{#1}}
\DeclareFieldFormat[bookinbook,inbook]{usera}{\mkbibemph{#1}}
\DeclareFieldFormat[incollection]{usera}{\mkbibquote{#1}}
\DeclareFieldFormat{userc}{\mkpageprefix[bookpagination]{#1}}
\DeclareFieldFormat{url}{\url{#1}}
\DeclareFieldFormat{annotation}{\annotationfont #1}
\DeclareFieldFormat{library}{\libraryfont #1}
\DeclareFieldFormat{pureparens}{\mkpureparens{#1}}
\DeclareFieldFormat{editortype}{\mkpureparens{#1}}
\DeclareFieldAlias{authortype}{editortype}
\DeclareFieldFormat{backrefparens}{\mkpureparens{#1}}
\DeclareFieldFormat*{number}{%
  \ifdefstring{\bbx@volnumformat}{strings}{%
    \bibstring{number}~#1}{#1}}
\DeclareFieldFormat*{series}{%
  \ifinteger{#1}
    {\mkbibordseries{#1}~\bibstring{jourser}}
    {\ifbibstring{#1}{\bibstring{#1}}{#1}}}
\DeclareFieldFormat{edition}{%
  \ifinteger{#1}{%
    \ifdefstring{\bbx@editionformat}{arabic}
    {\mkbibordedition{#1}~\bibstring{edition}}
    {\ifdefstring{\bbx@editionformat}{Roman}%
      {\RN{#1}~\bibstring{edition}}%
      {\ifdefstring{\bbx@editionformat}{romansc}%
        {\textsc{\Rn{#1}}~\bibstring{edition}}%
        {\ifdefstring{\bbx@editionformat}{roman}%
          {\Rn{#1}~\bibstring{edition}}
          {\ifdefstring{\bbx@editionformat}{superscript}%
            {\mkbibsuperscript{#1}}%
            {\optionerror{editionformat}}}}}}}{#1}}%\isdot??  
\DeclareFieldFormat{volume}{%
  \bibstring{volume}~%
    \ifinteger{#1}{%
      \ifdefstring{\bbx@volumeformat}{arabic}%
      {#1}%
      {\ifdefstring{\bbx@volumeformat}{Roman}%
        {\RN{#1}}%
        {\ifdefstring{\bbx@volumeformat}{romansc}%
          {\textsc{\Rn{#1}}}%
          {\ifdefstring{\bbx@volumeformat}{roman}%
            {\Rn{#1}}%
            {\optionerror{volumeformat}}}}}}{#1}}
\DeclareFieldFormat[article]{volume}{%
  \ifdefstring{\bbx@volnumformat}{strings}
    {\bibstring{volume}~}%
    {}%
      \ifinteger{#1}{%
        \ifdefstring{\bbx@volumeformat}{arabic}%
         {#1}%
         {\ifdefstring{\bbx@volumeformat}{Roman}%
           {\RN{#1}}%
           {\ifdefstring{\bbx@volumeformat}{romansc}%
             {\textsc{\Rn{#1}}}%
             {\ifdefstring{\bbx@volumeformat}{roman}%
               {\Rn{#1}}x%
               {\optionerror{volumeformat}}}}}}{#1}}
\DeclareFieldFormat{related}{%
  \ifdefstring{\bbx@relatedformat}{parens}%
    {\mkpureparens{#1}}%
    {\ifdefstring{\bbx@relatedformat}{brackets}%
      {\mkpurebrackets{#1}}%
      {\ifdefstring{\bbx@relatedformat}{semicolon}%
        {#1}%
        {\optionerror{relatedformat}}}}}%
\DeclareFieldAlias{related:origpubin}{related}
\DeclareFieldAlias{related:origpubas}{related}
\DeclareFieldFormat{relatedstring:default}{#1\addspace}%\addspace needed
%    \end{macrocode}
% \subsubsection{New macros}
% Experimental in version 1.9.4. The \texttt{translatorstrg} and \texttt{translator+othersstrg} macros do not use the
% \texttt{editortype} format so we add it for consistency with \texttt{editorstrg} and \texttt{editor+othersstrg}
% from \file{biblatex.def}. The idea behind this feature is that in this way you can change the format of the editor, translator, etc.
% following the year label simply with \cmd{DeclareFieldFormat}.
%    \begin{macrocode}
\renewbibmacro*{translatorstrg}{%
  \printtext[editortype]{%
    \ifboolexpr{
      test {\ifnumgreater{\value{translator}}{1}}
      or
      test {\ifandothers{translator}}
    }
    {\bibstring{translators}}
    {\bibstring{translator}}}}
\renewbibmacro*{translator+othersstrg}{%
  \ifboolexpr{
    test {\ifnumgreater{\value{translator}}{1}}
    or
    test {\ifandothers{translator}}
  }
  {\def\abx@tempa{translators}}
  {\def\abx@tempa{translator}}%
  \ifnamesequal{translator}{commentator}
  {\appto\abx@tempa{co}%
    \clearname{commentator}}
  {\ifnamesequal{translator}{annotator}
    {\appto\abx@tempa{an}%
      \clearname{annotator}}
    {}}%
  \ifnamesequal{translator}{introduction}
  {\appto\abx@tempa{in}%
    \clearname{introduction}}
  {\ifnamesequal{translator}{foreword}
    {\appto\abx@tempa{fo}%
      \clearname{foreword}}
    {\ifnamesequal{translator}{afterword}
      {\appto\abx@tempa{af}%
        \clearname{afterword}}
      {}}}%
  \printtext[editortype]{\bibstring{\abx@tempa}}}
%    \end{macrocode}
% The default macros for indexing include the \bibfield{indextitle} field (which defaults to \bibfield{title}).
% This involves getting an index with names and titles together. So we redefine the following two macros in order to get a simple index of names.
%    \begin{macrocode}
\renewbibmacro*{citeindex}{%
  \ifciteindex
    {\indexnames{labelname}}
    {}}
\renewbibmacro*{bibindex}{%
  \ifbibindex
    {\indexnames{labelname}}
    {}}
%    \end{macrocode}
% Here we (re)define different macros used to print various fields.
%    \begin{macrocode}
\newbibmacro*{volnumdefault}{%
  \printfield{volume}%
    \setunit*{\volnumpunct}%
    \printfield{number}}

\newbibmacro*{volnumparens}{%
  \nopunct%
  \printtext[pureparens]{%
    \printfield{volume}%
      \setunit*{\volnumpunct}%
    \printfield{number}}}

\newbibmacro*{volnumstrings}{%
  \iffieldundef{volume}{}{%
    \printfield{volume}\setunit*{\volnumpunct}}%
  \iffieldundef{number}{}{%
   \printfield{number}}}

\renewbibmacro*{volume+number+eid}{%
     \ifdefstring{\bbx@volnumformat}{strings}
      {\usebibmacro{volnumstrings}}%
      {\ifdefstring{\bbx@volnumformat}{parens}
        {\usebibmacro{volnumparens}}%
        {\ifdefstring{\bbx@volnumformat}{plain}
          {\usebibmacro{volnumdefault}}%
          {\optionerror{volnumformat}}}}%
  \setunit{\addcomma\space}%
  \printfield{eid}}

% TO be removed if implemented in biblatex.def.
% Code proposed by @moewew
\renewbibmacro*{journal}{%
  \ifboolexpr{
    test {\iffieldundef{journaltitle}}
    and
    test {\iffieldundef{journalsubtitle}}
  }
    {}
    {\printtext[journaltitle]{%
       \printfield[titlecase]{journaltitle}%
       \setunit{\subtitlepunct}%
       \printfield[titlecase]{journalsubtitle}}}}

\renewbibmacro*{periodical}{%
  \ifboolexpr{
    test {\iffieldundef{title}}
    and
    test {\iffieldundef{subtitle}}
  }
    {}
    {\printtext[title]{%
       \printfield[titlecase]{title}%
       \setunit{\subtitlepunct}%
       \printfield[titlecase]{subtitle}}}}

\renewbibmacro*{issue}{%
  \ifboolexpr{
    test {\iffieldundef{issuetitle}}
    and
    test {\iffieldundef{issuesubtitle}}
  }
    {}
    {\printtext[issuetitle]{%
       \printfield[titlecase]{issuetitle}%
       \setunit{\subtitlepunct}%
\printfield[titlecase]{issuesubtitle}}}}

%\renewbibmacro*{journal}{%
%  \iffieldundef{journaltitle}
%    {}%
%    {\printtext[journaltitle]{%
%       \printfield[titlecase]{journaltitle}%
%       \midsentence%
%       \setunit{\subtitlepunct}%
%       \printfield[titlecase]{journalsubtitle}}}}
%
%\renewbibmacro*{periodical}{%
%  \iffieldundef{title}
%    {}%
%    {\printtext[title]{%
%       \printfield[titlecase]{title}%
%       \midsentence%
%       \setunit{\subtitlepunct}%
%       \printfield[titlecase]{subtitle}}}}

\renewbibmacro*{journal+issuetitle}{%
  \usebibmacro{journal}%
  \setunit*{\addspace}%
  \iffieldundef{series}
    {}%
    {\newunit%
     \printfield{series}\setunit{\addspace}\midsentence}%
     \newunit%
  \usebibmacro{volume+number+eid}%
  \setunit{\addspace}%
  \usebibmacro{issue+date}%
  \setunit{\addcolon\space}%
  \usebibmacro{issue}%
  \newunit}

\renewbibmacro*{title+issuetitle}{%
  \usebibmacro{periodical}%
  \setunit*{\addspace}%
  \iffieldundef{series}
    {}%
    {\newunit
     \printfield{series}%
     \setunit{\addspace}\midsentence}%
  \usebibmacro{volume+number+eid}%
  \setunit{\addspace}%
  \usebibmacro{issue+date}%
  \setunit{\addcolon\space}%
  \usebibmacro{issue}%
  \newunit}

\renewbibmacro*{series+number}{%
  \printfield{series}%
 \setunit*{\addcomma\space}%
  \printfield{number}%
  \newunit}
  
\renewbibmacro*{issue+date}{%
  \printtext[pureparens]{%
    \iffieldundef{issue}
      {\usebibmacro{date}}
      {\printfield{issue}%
       \setunit*{\addspace}%
       \usebibmacro{date}}}%
  \newunit}

\renewbibmacro*{event+venue+date}{%
  \printfield{eventtitle}%
  \ifboolexpr{%
    test {\iffieldundef{venue}}
    and
    test {\iffieldundef{eventyear}}
  }%
    {}%
    {\setunit*{\addspace}%
     \printtext{%
       \printfield{venue}%
       \setunit*{\addcomma\space}%
       \printeventdate}}%
  \newunit}
 
\renewbibmacro*{publisher+location+date}{%
\ifdefstring{\bbx@publocformat}{loccolonpub}
  {\usebibmacro{loccolonpub}}
    {\ifdefstring{\bbx@publocformat}{locpubyear}
    {\usebibmacro{locpubyear}}
      {\ifdefstring{\bbx@publocformat}{publocyear}%
      {\usebibmacro{publocyear}}{\optionerror{publocformat}}}}}

\renewbibmacro*{institution+location+date}{%
\ifdefstring{\bbx@publocformat}{loccolonpub}
  {\usebibmacro{inloccolonpub}}
    {\ifdefstring{\bbx@publocformat}{locpubyear}
    {\usebibmacro{inlocpubyear}}
      {\ifdefstring{\bbx@publocformat}{publocyear}%
      {\usebibmacro{inpublocyear}}{\optionerror{publocformat}}}}}

\renewbibmacro*{organization+location+date}{%
\ifdefstring{\bbx@publocformat}{loccolonpub}
  {\usebibmacro{orgloccolonpub}}
    {\ifdefstring{\bbx@publocformat}{locpubyear}
    {\usebibmacro{orglocpubyear}}
      {\ifdefstring{\bbx@publocformat}{publocyear}%
      {\usebibmacro{orgpublocyear}}{\optionerror{publocformat}}}}}

\newbibmacro*{publocyear}{%
  \iflistundef{publisher}%
   {}%
{\printlist{publisher}}
  \setunit*{\addcomma\space}%
  \printlist{location}%
  \usebibmacro{relateddate}%
\newunit}

\newbibmacro*{inpublocyear}{%
  \iflistundef{institution}%
   {}%
{\printlist{institution}}
  \setunit*{\addcomma\space}%
  \printlist{location}%
  \usebibmacro{relateddate}%
\newunit}

\newbibmacro*{orgpublocyear}{%
  \iflistundef{organization}%
   {}%
{\printlist{organization}}
  \setunit*{\addcomma\space}%
  \printlist{location}%
  \usebibmacro{relateddate}%
\newunit}

\newbibmacro*{loccolonpub}{%
  \printlist{location}%
  \iflistundef{publisher}%
   {\setunit*{\addspace}}
     {\setunit*{\addcolon\space}}%
  \printlist{publisher}%
  \usebibmacro{commarelateddate}%
\newunit}

\newbibmacro*{inloccolonpub}{%
  \printlist{location}%
  \iflistundef{institution}
    {\setunit*{\addspace}}
    {\setunit*{\addcolon\space}}%
  \printlist{institution}%
  \usebibmacro{commarelateddate}%
  \newunit}
  
\newbibmacro*{orgloccolonpub}{%
  \printlist{location}%
  \iflistundef{organization}
    {\setunit*{\addspace}}
    {\setunit*{\addcolon\space}}%
  \printlist{organization}%
  \usebibmacro{commarelateddate}%
  \newunit}
  
\newbibmacro*{locpubyear}{%
  \printlist{location}%
  \iflistundef{publisher}%
   {\setunit*{\addspace}}
     {\setunit*{\addcomma\space}}%
  \printlist{publisher}%
  \usebibmacro{commarelateddate}%
\newunit}

\newbibmacro*{inlocpubyear}{%
  \printlist{location}%
  \iflistundef{institution}
    {\setunit*{\addspace}}
    {\setunit*{\addcomma\space}}%
  \printlist{institution}%
  \usebibmacro{commarelateddate}%
  \newunit}
  
\newbibmacro*{orglocpubyear}{%
  \printlist{location}%
  \iflistundef{organization}
    {\setunit*{\addspace}}
    {\setunit*{\addcomma\space}}%
  \printlist{organization}%
  \usebibmacro{commarelateddate}%
  \newunit}

\renewbibmacro*{addendum+pubstate}{%
  \printfield{addendum}%
  \newunit\newblock
  \printfield{pubstate}%
  \ifdefstring{\bbx@origfields}{none}{}{%
   \newunit\newblock
  \usebibmacro{origdata:book}}%
  \newunit\newblock
  \usebibmacro{library}}

\newbibmacro*{addendum+pubstate:article-inbook-incoll}{%
  \printfield{addendum}%
  \newunit\newblock
  \printfield{pubstate}%
  \ifdefstring{\bbx@origfields}{none}{}{%
   \newunit\newblock
  \usebibmacro{origdata:article-inbook}}%
  \newunit\newblock
  \usebibmacro{library}}

\newbibmacro*{library}{%
  \iftoggle{bbx:library}{%
    \iffieldundef{library}%
      {}%
      {\setunit{\addperiod\space}%
      {\printfield{library}}}}%
    {}}

\renewbibmacro*{pageref}{%
  \iflistundef{pageref}
    {}%
    {\setunit{\addperiod\space}%
     \printtext[backrefparens]{%
       \ifnumgreater{\value{pageref}}{1}
         {\bibcpstring{backrefpages}\ppspace}%
         {\bibcpstring{backrefpage}\ppspace}%
     \printlist[pageref][-\value{listtotal}]{pageref}\adddot}\nopunct}}%

\renewbibmacro*{finentry}{%
  \iftoggle{bbx:annotation}%
    {\iffieldundef{annotation}%
      {\finentry}%
      {\setunit{\addperiod\par\nobreak\vspace*{.5ex}}%
      \printtext[annotation]{\printfield{annotation}\finentry\par}}}%
    {\finentry}}
%    \end{macrocode}
% \subsubsection{Related entries}
%    \begin{macrocode}
\newbibmacro*{relateddate}{%
  \setunit*{\addspace}%
  \printdate}
\newbibmacro*{commarelateddate}{%
  \setunit*{\addcomma\space}%
  \printdate}
%    \end{macrocode}
% A trick to delete the author/editor list for related
% entries and \cmd{ccite} command:
%    \begin{macrocode}
\newbibmacro*{related:clearauthors}{%
  \renewbibmacro*{author/translator+others}{\usebibmacro{bbx:savehash}}%
  \renewbibmacro*{author/editor+others/translator+others}{\usebibmacro{bbx:savehash}}%
  \renewbibmacro*{editor+others}{\usebibmacro{bbx:savehash}}%
  \renewbibmacro*{author/translator+others}{\usebibmacro{bbx:savehash}}%
  \renewbibmacro*{author/editor}{\usebibmacro{bbx:savehash}}%
  \renewbibmacro*{author}{\usebibmacro{bbx:savehash}}%
  \renewbibmacro*{editor}{\usebibmacro{bbx:savehash}}%
  \renewcommand*{\labelnamepunct}{}}%
\DeclareCiteCommand{\relatedcite}
  {}%
  {\usedriver
     {\DeclareNameAlias{sortname}{default}%
  \usebibmacro{related:clearauthors}%
  \renewbibmacro*{relateddate}{%
    \setunit*{\addspace}\printdate}%
  \renewbibmacro*{commarelateddate}{%
    \setunit*{\addcomma\space}\printdate}}%
     {\thefield{entrytype}}}%
  {}%
  {}%
\renewbibmacro*{related:default}[1]{%
    \togglefalse{bbx:annotation}%
  \ifboolexpr{
      test {\iffieldundef{relatedtype}}
      and
      test {\iffieldundef{relatedstring}}
  }
  {\printtext{\bibstring{translationas}}}{}%
   \printtext{\addspace}%    
   \relatedcite{\thefield{related}}}
%    \end{macrocode}
% We redefine the \texttt{begrelatedloop} macro to avoid nested parentheses in cascading related entries:
%    \begin{macrocode}               
\renewbibmacro*{begrelatedloop}{%
  \renewrobustcmd*{\mkpureparens}{\relatedpunct}%
  \renewrobustcmd*{\mkpurebrackets}{\relatedpunct}}
%    \end{macrocode}
% This macro tests the value of the \opt{relatedformat} option. If it sets to \opt{semicolon} the macro adds \cmd{relatedpunct} (i.e. a semicolon plus a space), otherwise it adds a simple space.
%    \begin{macrocode}
\newbibmacro*{phil:related}{%
  \iftoggle{bbx:related}
    {\iffieldequalstr{relatedtype}{multivolume}%
      {\setunit{\addperiod}}%
      {\ifdefstring{\bbx@relatedformat}{semicolon}%
        {\setunit{\relatedpunct}}%
        {\setunit{\addspace}}}%
    \usebibmacro{related:init}%
    \usebibmacro{related}}{}}
%    \end{macrocode}
% The below macros will be used in the \bibtype{inbook}, \bibtype{incollection} and \bibtype{inproceedings} drivers.
%    \begin{macrocode}
\renewbibmacro*{bybookauthor}{%
  \ifnamesequal{author}{bookauthor}%
  {}%
  {\printnames[default]{bookauthor}}}

\newbibmacro*{xrefdata}{%
  \iffieldundef{volume}
  {}%
  {\printfield{volume}%
    \printfield{part}%
    \setunit{\addcolon\space}%
    \printfield{booktitle}}%
  \newunit\newblock
  \usebibmacro{byeditor+others}%
  \newunit\newblock
  \printfield{edition}%
  \newunit
  \printfield{volumes}%
  \newunit\newblock
  \usebibmacro{series+number}%
  \newunit\newblock
  \printfield{note}%
  \newunit\newblock
  \usebibmacro{publisher+location+date}%
  \newunit\newblock
  \usebibmacro{chapter+pages}%
  \newunit\newblock
  \iftoggle{bbx:isbn}
  {\printfield{isbn}}
  {}%
  \newunit\newblock
  \usebibmacro{doi+eprint+url}%
  \newunit\newblock
  \usebibmacro{addendum+pubstate:article-inbook-incoll}%
  \newblock
  \usebibmacro{phil:related}%
  \newunit\newblock
  \usebibmacro{pageref}%
  \usebibmacro{finentry}}

\newbibmacro*{crossrefdata}{%
  \iffieldundef{maintitle}
  {\printfield{volume}%
    \printfield{part}}
  {}%
  \newunit\newblock
  \usebibmacro{chapter+pages}%
  \newunit\newblock
  \iftoggle{bbx:isbn}
  {\printfield{isbn}}
  {}%
  \newunit\newblock
  \usebibmacro{doi+eprint+url}%
  \newblock
  \usebibmacro{phil:related}%
  \newunit\newblock
  \usebibmacro{pageref}%
  \usebibmacro{finentry}}
%    \end{macrocode}
% \paragraph{Backward compatibility}
% The \opt{orig*} macros are deprecated. The same feature is now 
% supported using the \opt{related} field. 
%    \begin{macrocode}
\newbibmacro*{origpublisher+origlocation+origdate}{%
  \ifdefstring{\bbx@publocformat}{loccolonpub}
    {\usebibmacro{origloccolonpub}\setunit{\bibpagespunct}%
    \printfield{userc}}{\ifdefstring{\bbx@publocformat}{locpubyear}
    {\usebibmacro{origlocpubyear}\setunit{\bibpagespunct}%
    \printfield{userc}}
    {\usebibmacro{origpublocyear}\setunit{\bibpagespunct}%
    \printfield{userc}}}}

\newbibmacro*{origpublocyear}{%
  \iflistundef{origpublisher}%
  {\printlist{origlocation}%
    \setunit*{\addspace}%
    \printorigdate}%
  {\printlist{origpublisher}%
    \setunit*{\addcomma\space}%
    \printlist{origlocation}%
    \setunit*{\addspace}%
    \printorigdate}%
  \newunit}

\newbibmacro*{origloccolonpub}{%
  \iflistundef{origlocation}{}%
  {\printlist{origlocation}}%
  \iflistundef{origpublisher}%
  {\setunit*{\addspace}%
    \printorigdate}%
  {\setunit*{\addcolon\space}%
    \printlist{origpublisher}%
    \setunit*{\addcomma\space}%
    \printorigdate}%
  \newunit}

\newbibmacro*{origlocpubyear}{%
  \iflistundef{origlocation}{}%
  {\printlist{origlocation}}%
  \iflistundef{origpublisher}%
  {\setunit*{\addspace}%
    \printorigdate}%
  {\setunit*{\addcomma\space}%
    \printlist{origpublisher}%
    \setunit*{\addcomma\space}%
    \printorigdate}%
  \newunit}

\newbibmacro*{reprinttitle}{%
  \iffieldundef{reprinttitle}{}{%
    \iffieldsequal{reprinttitle}{title}{}{%
      \printfield[title]{reprinttitle}%
      \setunit{\addcomma\space}}}%
  \iffieldundef{userb}{}{%
    \printfield{userb}}}%

\newbibmacro*{transorigstring}{%
  \iffieldundef{reprinttitle}%
  {\printtext{\ifdefstring{\bbx@origfields}{origed}
      {\bibstring{origpubas}}%
      {\bibstring{translationas}}}\nopunct}%
  {\printtext{\bibstring{reprint}}}\nopunct}

\newbibmacro*{origtitle:book}{%
  \iffieldundef{origtitle}{}{
    \printfield[origtitle]{origtitle}%
    \setunit{\addcomma\space}}
  \iffieldundef{userb}{}{%
    \printfield{userb}}}%

\newbibmacro*{origtitledata:book}{%
  \usebibmacro{transorigstring}%
  \iffieldundef{reprinttitle}%
  {\usebibmacro{origtitle:book}}%
  {\usebibmacro{reprinttitle}}%
  \newunit\newblock
  \usebibmacro{origpublisher+origlocation+origdate}}

\newbibmacro*{origdata:book}{%
  \ifboolexpr{%
    test {\iffieldundef{origtitle}}
    and
    test {\iffieldundef{origlocation}}
    and
    test {\iffieldundef{origpublisher}}
    and
    test {\iffieldundef{origyear}}
  }%
  {}%
  {\ifdefstring{\bbx@relatedformat}{parens}
    {\nopunct\printtext[pureparens]{\usebibmacro{origtitledata:book}}}
    {\ifdefstring{\bbx@relatedformat}{brackets}
      {\nopunct\printtext[brackets]{\usebibmacro{origtitledata:book}}}
      {\setunit{\addsemicolon\space}%
        \printtext{\usebibmacro{origtitledata:book}}}}}}

\newbibmacro*{origtitle:article-inbook-incoll}{%
  \iffieldundef{origtitle}{}{%
    \printfield[origtitle]{origtitle}%
    \setunit{\addcomma\space}}%
  \iffieldundef{usera}{}{%
    \usebibmacro{in:}%
    \printfield{usera}%
    \setunit{\addcomma\space}}%
  \iffieldundef{userb}{}{%
    \printfield{userb}%
    \newunit}}

\newbibmacro*{origtitledata:article-inbook-incoll}{%
  \usebibmacro{transorigstring}%
  \setunit{\addspace}%
  \usebibmacro{origtitle:article-inbook-incoll}%
  \usebibmacro{origpublisher+origlocation+origdate}}

\newbibmacro*{origdata:article-inbook}{%
  \iflistundef{origlocation}{}{%
    \ifdefstring{\bbx@relatedformat}{parens}
    {\nopunct\printtext[pureparens]{%
        \usebibmacro{origtitledata:article-inbook-incoll}}}%
    {\ifdefstring{\bbx@relatedformat}{brackets}
      {\nopunct\printtext[brackets]{%
          \usebibmacro{origtitledata:article-inbook-incoll}}}%
      {\setunit{\addsemicolon\space}%
        \printtext{%
          \usebibmacro{origtitledata:article-inbook-incoll}}}}}}
%    \end{macrocode}
% \subsubsection{Bibliography drivers}
%    \begin{macrocode}
 \DeclareBibliographyDriver{article}{%
  \usebibmacro{bibindex}%
  \usebibmacro{begentry}%
  \usebibmacro{author/translator+others}%
  \setunit{\labelnamepunct}\newblock
  \usebibmacro{title}%
  \newunit
  \printlist{language}%
  \newunit\newblock
  \usebibmacro{byauthor}%
  \newunit\newblock
  \usebibmacro{bytranslator+others}%
  \newunit\newblock
  \printfield{version}%
  \newunit\newblock
  \iftoggle{bbx:inbeforejournal}{\usebibmacro{in:}}{}%
  \usebibmacro{journal+issuetitle}%
  \newunit\newblock
  \usebibmacro{byeditor+others}%
  \newunit\newblock
  \usebibmacro{note+pages}%
  \newunit\newblock
  \iftoggle{bbx:isbn}
    {\printfield{issn}}
    {}%
  \newunit\newblock
  \usebibmacro{doi+eprint+url}%
  \newunit\newblock
  \usebibmacro{addendum+pubstate:article-inbook-incoll}%
  \newblock
  \usebibmacro{phil:related}%
  \newunit\newblock  
  \usebibmacro{pageref}%
  \usebibmacro{finentry}}
  
\DeclareBibliographyDriver{book}{%
  \usebibmacro{bibindex}%
  \usebibmacro{begentry}%
  \usebibmacro{author/editor+others/translator+others}%
  \setunit{\labelnamepunct}\newblock
  \usebibmacro{maintitle+title}%
  \newunit
  \printlist{language}%
  \newunit\newblock
  \usebibmacro{byauthor}%
  \newunit\newblock
  \usebibmacro{byeditor+others}%
  \newunit\newblock
  \printfield{edition}%
  \newunit
  \printfield{volumes}%
  \newunit\newblock
  \usebibmacro{series+number}%
  \newunit\newblock
  \printfield{note}%
  \newunit\newblock
  \usebibmacro{publisher+location+date}%
  \newunit
  \iffieldundef{maintitle}
    {\printfield{volume}%
     \printfield{part}}
    {}%
  \newunit\newblock
  \usebibmacro{chapter+pages}%
  \newunit
  \printfield{pagetotal}%
  \newunit\newblock
  \iftoggle{bbx:isbn}
    {\printfield{isbn}}
    {}%
  \newunit\newblock
  \usebibmacro{doi+eprint+url}%
  \newunit\newblock
  \usebibmacro{addendum+pubstate}%
  \newblock
  \usebibmacro{phil:related}%
  \newunit\newblock
  \usebibmacro{pageref}%
  \usebibmacro{finentry}}
  
\DeclareBibliographyDriver{booklet}{%
  \usebibmacro{bibindex}%
  \usebibmacro{begentry}%
  \usebibmacro{author/editor+others/translator+others}%
  \setunit{\labelnamepunct}\newblock
  \usebibmacro{title}%
  \newunit
  \printlist{language}%
  \newunit\newblock
  \usebibmacro{byauthor}%
  \newunit\newblock
  \usebibmacro{byeditor+others}%
  \newunit\newblock
  \printfield{howpublished}%
  \newunit\newblock
  \printfield{type}%
  \newunit\newblock
  \printfield{note}%
  \newunit\newblock
  \usebibmacro{location+date}%
  \newunit\newblock
  \usebibmacro{chapter+pages}%
  \newunit
  \printfield{pagetotal}%
  \newunit\newblock
  \usebibmacro{doi+eprint+url}%
  \newunit\newblock
  \usebibmacro{addendum+pubstate}%
  \newblock
  \usebibmacro{phil:related}%
  \newunit\newblock
  \usebibmacro{pageref}%
  \usebibmacro{finentry}}

\DeclareBibliographyDriver{collection}{%
  \usebibmacro{bibindex}%
  \usebibmacro{begentry}%
  \usebibmacro{editor+others}%
  \setunit{\labelnamepunct}\newblock
  \usebibmacro{maintitle+title}%
  \newunit
  \printlist{language}%
  \newunit\newblock
  \usebibmacro{byeditor+others}%
  \newunit\newblock
  \printfield{edition}%
  \newunit
  \iffieldundef{maintitle}
    {\printfield{volume}%
     \printfield{part}}
    {}%
  \newunit
  \printfield{volumes}%
  \newunit\newblock
  \usebibmacro{series+number}%
  \newunit\newblock
  \printfield{note}%
  \newunit\newblock
  \usebibmacro{publisher+location+date}%
  \newunit\newblock
  \usebibmacro{chapter+pages}%
  \newunit
  \printfield{pagetotal}%
  \newunit\newblock
  \iftoggle{bbx:isbn}
    {\printfield{isbn}}
    {}%
  \newunit\newblock
  \usebibmacro{doi+eprint+url}%
  \newunit\newblock
  \usebibmacro{addendum+pubstate}%
  \newblock
  \usebibmacro{phil:related}%  
  \newunit\newblock
  \usebibmacro{pageref}%
  \usebibmacro{finentry}}

\DeclareBibliographyDriver{inbook}{%
  \usebibmacro{bibindex}%
  \usebibmacro{begentry}%
  \usebibmacro{author/translator+others}%
  \setunit{\labelnamepunct}\newblock
  \usebibmacro{title}%
  \newunit
  \printlist{language}%
  \newunit\newblock
  \usebibmacro{byauthor}%
  \newunit\newblock
  \usebibmacro{in:}%
  \iffieldundef{xref}%
   {\iffieldundef{crossref}{\usebibmacro{inbook:full}}%
   {\bbx@crossref@inbook{\thefield{crossref}}%
   \newunit\newblock
   \usebibmacro{crossrefdata}}}%
   {\bbx@crossref@inbook{\thefield{xref}}%
   \newunit\newblock
   \usebibmacro{xrefdata}}}%

\newbibmacro*{inbook:full}{%
  \usebibmacro{bybookauthor}%
  \newunit\newblock
  \usebibmacro{maintitle+booktitle}%
  \newunit\newblock
  \usebibmacro{byeditor+others}%
  \newunit\newblock
  \printfield{edition}%
  \newunit
  \printfield{volumes}%
  \newunit\newblock
  \usebibmacro{series+number}%
  \newunit\newblock
  \printfield{note}%
  \newunit\newblock
  \usebibmacro{publisher+location+date}%
  \newunit
  \iffieldundef{maintitle}
    {\printfield{volume}%
     \printfield{part}}
    {}%
  \newunit\newblock
  \usebibmacro{chapter+pages}%
  \newunit\newblock
  \iftoggle{bbx:isbn}
    {\printfield{isbn}}
    {}%
  \newunit\newblock
  \usebibmacro{doi+eprint+url}%
  \newunit\newblock
  \usebibmacro{addendum+pubstate:article-inbook-incoll}%
  \newblock
  \usebibmacro{phil:related}%
  \newunit\newblock
  \usebibmacro{pageref}%
  \usebibmacro{finentry}}

\DeclareBibliographyDriver{incollection}{%
  \usebibmacro{bibindex}%
  \usebibmacro{begentry}%
  \usebibmacro{author/translator+others}%
  \setunit{\labelnamepunct}\newblock
  \usebibmacro{title}%
  \newunit
  \printlist{language}%
  \newunit\newblock
  \usebibmacro{byauthor}%
  \newunit\newblock
  \usebibmacro{in:}%
  \iffieldundef{xref}%
   {\iffieldundef{crossref}{\usebibmacro{incollection:full}}%
   {\bbx@crossref@incollection{\thefield{crossref}}%
   \newunit\newblock
   \usebibmacro{crossrefdata}}}%
   {\bbx@crossref@incollection{\thefield{xref}}%
   \newunit\newblock
   \usebibmacro{xrefdata}}}

\newbibmacro*{incollection:full}{%
  \usebibmacro{maintitle+booktitle}%
  \newunit\newblock
  \usebibmacro{byeditor+others}%
  \newunit\newblock
  \printfield{edition}%
  \newunit
  \printfield{volumes}%
  \newunit\newblock
  \usebibmacro{series+number}%
  \newunit\newblock
  \printfield{note}%
  \newunit\newblock
  \usebibmacro{publisher+location+date}%
  \newunit
  \iffieldundef{maintitle}
    {\printfield{volume}%
     \printfield{part}}
    {}%
  \newunit\newblock
  \usebibmacro{chapter+pages}%
  \newunit\newblock
  \iftoggle{bbx:isbn}
    {\printfield{isbn}}
    {}%
  \newunit\newblock
  \usebibmacro{doi+eprint+url}%
  \newunit\newblock
  \usebibmacro{addendum+pubstate:article-inbook-incoll}%
  \newblock
  \usebibmacro{phil:related}%
  \newunit\newblock
  \usebibmacro{pageref}%
  \usebibmacro{finentry}}

\DeclareBibliographyDriver{inproceedings}{%
  \usebibmacro{bibindex}%
  \usebibmacro{begentry}%
  \usebibmacro{author/translator+others}%
  \setunit{\labelnamepunct}\newblock
  \usebibmacro{title}%
  \newunit
  \printlist{language}%
  \newunit\newblock
  \usebibmacro{byauthor}%
  \newunit\newblock
  \usebibmacro{in:}%
  \iffieldundef{xref}%
   {\iffieldundef{crossref}{\usebibmacro{inproceedings:full}}%
   {\bbx@crossref@incollection{\thefield{crossref}}%
   \newunit\newblock
   \usebibmacro{crossrefdata}}}%
   {\bbx@crossref@incollection{\thefield{xref}}%
   \newunit\newblock
   \usebibmacro{xrefdata}}}

\newbibmacro*{inproceedings:full}{%
  \usebibmacro{maintitle+booktitle}%
  \newunit\newblock
  \usebibmacro{event+venue+date}%
  \newunit\newblock
  \usebibmacro{byeditor+others}%
  \newunit
  \printfield{volumes}%
  \newunit\newblock
  \usebibmacro{series+number}%
  \newunit\newblock
  \printfield{note}%
  \newunit\newblock
  \printlist{organization}%
  \newunit
  \usebibmacro{publisher+location+date}%
  \newunit
  \iffieldundef{maintitle}
    {\printfield{volume}%
     \printfield{part}}
    {}%
  \newunit\newblock
  \usebibmacro{chapter+pages}%
  \newunit\newblock
  \iftoggle{bbx:isbn}
    {\printfield{isbn}}
    {}%
  \newunit\newblock
  \usebibmacro{doi+eprint+url}%
  \newunit\newblock
  \usebibmacro{addendum+pubstate:article-inbook-incoll}%
  \newblock
  \usebibmacro{phil:related}%
  \newunit\newblock
  \usebibmacro{pageref}%
  \usebibmacro{finentry}}

\DeclareBibliographyDriver{manual}{%
  \usebibmacro{bibindex}%
  \usebibmacro{begentry}%
  \usebibmacro{author/editor}%
  \setunit{\labelnamepunct}\newblock
  \usebibmacro{title}%
  \newunit
  \printlist{language}%
  \newunit\newblock
  \usebibmacro{byauthor}%
  \newunit\newblock
  \usebibmacro{byeditor}%
  \newunit\newblock
  \printfield{edition}%
  \newunit\newblock
  \usebibmacro{series+number}%
  \newunit\newblock
  \printfield{type}%
  \newunit
  \printfield{version}%
  \newunit
  \printfield{note}%
  \newunit\newblock
  \printlist{organization}%
  \newunit
  \usebibmacro{publisher+location+date}%
  \newunit\newblock
  \usebibmacro{chapter+pages}%
  \newunit
  \printfield{pagetotal}%
  \newunit\newblock
  \iftoggle{bbx:isbn}
    {\printfield{isbn}}
    {}%
  \newunit\newblock
  \usebibmacro{doi+eprint+url}%
  \newunit\newblock
  \usebibmacro{addendum+pubstate}%
  \newblock
  \usebibmacro{phil:related}%
  \newunit\newblock
  \usebibmacro{pageref}%
  \usebibmacro{finentry}}

\DeclareBibliographyDriver{misc}{%
  \usebibmacro{bibindex}%
  \usebibmacro{begentry}%
  \usebibmacro{author/editor+others/translator+others}%
  \setunit{\labelnamepunct}\newblock
  \usebibmacro{title}%
  \newunit
  \printlist{language}%
  \newunit\newblock
  \usebibmacro{byauthor}%
  \newunit\newblock
  \usebibmacro{byeditor+others}%
  \newunit\newblock
  \printfield{howpublished}%
  \newunit\newblock
  \printfield{type}%
  \newunit
  \printfield{version}%
  \newunit
  \printfield{note}%
  \newunit\newblock
  \usebibmacro{organization+location+date}%
  \newunit\newblock
  \usebibmacro{doi+eprint+url}%
  \newunit\newblock
  \usebibmacro{addendum+pubstate}%
  \newblock
  \usebibmacro{phil:related}%
  \newunit\newblock
  \usebibmacro{pageref}%
  \usebibmacro{finentry}}

\DeclareBibliographyDriver{online}{%
  \usebibmacro{bibindex}%
  \usebibmacro{begentry}%
  \usebibmacro{author/editor+others/translator+others}%
  \setunit{\labelnamepunct}\newblock
  \usebibmacro{title}%
  \newunit
  \printlist{language}%
  \newunit\newblock
  \usebibmacro{byauthor}%
  \newunit\newblock
  \usebibmacro{byeditor+others}%
  \newunit\newblock
  \printfield{version}%
  \newunit
  \printfield{note}%
  \newunit\newblock
  \printlist{organization}%
  \newunit\newblock
  \usebibmacro{date}%
  \newunit\newblock
  \iftoggle{bbx:eprint}
    {\usebibmacro{eprint}}
    {}%
  \newunit\newblock
  \usebibmacro{url+urldate}%
  \newunit\newblock
  \usebibmacro{addendum+pubstate}%
  \newblock
  \usebibmacro{phil:related}%
  \newunit\newblock
  \usebibmacro{pageref}%
  \usebibmacro{finentry}}

\DeclareBibliographyDriver{patent}{%
  \usebibmacro{bibindex}%
  \usebibmacro{begentry}%
  \usebibmacro{author}%
  \setunit{\labelnamepunct}\newblock
  \usebibmacro{title}%
  \newunit
  \printlist{language}%
  \newunit\newblock
  \usebibmacro{byauthor}%
  \newunit\newblock
  \printfield{type}%
  \setunit*{\addspace}%
  \printfield{number}%
  \iflistundef{location}
    {}
    {\setunit*{\addspace}%
     \printtext[parens]{%
       \printlist[][-\value{listtotal}]{location}}}%
  \newunit\newblock
  \usebibmacro{byholder}%
  \newunit\newblock
  \printfield{note}%
  \newunit\newblock
  \usebibmacro{date}%
  \newunit\newblock
  \usebibmacro{doi+eprint+url}%
  \newunit\newblock
  \usebibmacro{addendum+pubstate}%
  \newblock
  \usebibmacro{phil:related}%
  \newunit\newblock
  \usebibmacro{pageref}%
  \usebibmacro{finentry}}

\DeclareBibliographyDriver{periodical}{%
  \usebibmacro{bibindex}%
  \usebibmacro{begentry}%
  \usebibmacro{editor}%
  \setunit{\labelnamepunct}\newblock
  \usebibmacro{title+issuetitle}%
  \newunit
  \printlist{language}%
  \newunit\newblock
  \usebibmacro{byeditor}%
  \newunit\newblock
  \printfield{note}%
  \newunit\newblock
  \iftoggle{bbx:isbn}
    {\printfield{issn}}
    {}%
  \newunit\newblock
  \usebibmacro{doi+eprint+url}%
  \newunit\newblock
  \usebibmacro{addendum+pubstate}%
  \newblock
  \usebibmacro{phil:related}%
  \newunit\newblock
  \usebibmacro{pageref}%
  \usebibmacro{finentry}}

\DeclareBibliographyDriver{proceedings}{%
  \usebibmacro{bibindex}%
  \usebibmacro{begentry}%
  \usebibmacro{editor+others}%
  \setunit{\labelnamepunct}\newblock
  \usebibmacro{maintitle+title}%
  \newunit
  \printlist{language}%
  \newunit\newblock
  \usebibmacro{event+venue+date}%
  \newunit\newblock
  \usebibmacro{byeditor+others}%
  \newunit\newblock
  \iffieldundef{maintitle}
    {\printfield{volume}%
     \printfield{part}}
    {}%
  \newunit
  \printfield{volumes}%
  \newunit\newblock
  \usebibmacro{series+number}%
  \newunit\newblock
  \printfield{note}%
  \newunit\newblock
  \printlist{organization}%
  \newunit
  \usebibmacro{publisher+location+date}%
  \newunit\newblock
  \usebibmacro{chapter+pages}%
  \newunit
  \printfield{pagetotal}%
  \newunit\newblock
  \iftoggle{bbx:isbn}
    {\printfield{isbn}}
    {}%
  \newunit\newblock
  \usebibmacro{doi+eprint+url}%
  \newunit\newblock
  \usebibmacro{addendum+pubstate}%
  \newblock
  \usebibmacro{phil:related}%
  \newunit\newblock
  \usebibmacro{pageref}%
  \usebibmacro{finentry}}

\DeclareBibliographyDriver{report}{%
  \usebibmacro{bibindex}%
  \usebibmacro{begentry}%
  \usebibmacro{author}%
  \setunit{\labelnamepunct}\newblock
  \usebibmacro{title}%
  \newunit
  \printlist{language}%
  \newunit\newblock
  \usebibmacro{byauthor}%
  \newunit\newblock
  \printfield{type}%
  \setunit*{\addspace}%
  \printfield{number}%
  \newunit\newblock
  \printfield{version}%
  \newunit
  \printfield{note}%
  \newunit\newblock
  \usebibmacro{institution+location+date}%
  \newunit\newblock
  \usebibmacro{chapter+pages}%
  \newunit
  \printfield{pagetotal}%
  \newunit\newblock
  \iftoggle{bbx:isbn}
    {\printfield{isrn}}
    {}%
  \newunit\newblock
  \usebibmacro{doi+eprint+url}%
  \newunit\newblock
  \usebibmacro{addendum+pubstate}%
  \newblock
  \usebibmacro{phil:related}%
  \newunit\newblock
  \usebibmacro{pageref}%
  \usebibmacro{finentry}}

\DeclareBibliographyDriver{thesis}{%
  \usebibmacro{bibindex}%
  \usebibmacro{begentry}%
  \usebibmacro{author}%
  \setunit{\labelnamepunct}\newblock
  \usebibmacro{title}%
  \newunit
  \printlist{language}%
  \newunit\newblock
  \usebibmacro{byauthor}%
  \newunit\newblock
  \printfield{note}%
  \newunit\newblock
  \printfield{type}%
  \newunit
  \usebibmacro{institution+location+date}%
  \newunit\newblock
  \usebibmacro{chapter+pages}%
  \newunit
  \printfield{pagetotal}%
  \newunit\newblock
  \iftoggle{bbx:isbn}
    {\printfield{isbn}}
    {}%
  \newunit\newblock
  \usebibmacro{doi+eprint+url}%
  \newunit\newblock
  \usebibmacro{addendum+pubstate}%
  \newblock
  \usebibmacro{phil:related}%
  \newunit\newblock
  \usebibmacro{pageref}%
  \usebibmacro{finentry}}

\DeclareBibliographyDriver{unpublished}{%
  \usebibmacro{bibindex}%
  \usebibmacro{begentry}%
  \usebibmacro{author}%
  \setunit{\labelnamepunct}\newblock
  \usebibmacro{title}%
  \newunit
  \printlist{language}%
  \newunit\newblock
  \usebibmacro{byauthor}%
  \newunit\newblock
  \printfield{howpublished}%
  \newunit\newblock
  \printfield{note}%
  \newunit\newblock
  \usebibmacro{location+date}%
  \newunit\newblock
  \iftoggle{bbx:url}
    {\usebibmacro{url+urldate}}
    {}%
  \newunit\newblock
  \usebibmacro{addendum+pubstate}%
  \newblock
  \usebibmacro{phil:related}%
  \newunit\newblock
  \usebibmacro{pageref}%
  \usebibmacro{finentry}}
%    \end{macrocode}
% In the \bibtype{set} entry type we restore the \sty{classic} style
% from the second entry onward, using the |entrysetcount| counter.
%    \begin{macrocode}
\DeclareBibliographyDriver{set}{%
  \savefield{annotation}{\@phil@nnote}%
  \clearfield{annotation}%
  \entryset{\ifnumgreater{\thefield{entrysetcount}}{1}%
    {\setkeys{blx@bib2}{restoreclassic}}{}}{}%
  \newunit\newblock
  \restorefield{annotation}{\@phil@nnote}%
  \usebibmacro{pageref}%
  \usebibmacro{finentry}}
%    \end{macrocode}
% \paragraph{Experimental drivers for jurisprudence}
% This feature is available for now only for Italian documents.
%    \begin{macrocode}
\DeclareFieldFormat[jurisdiction]{volume}{\RN{#1}}
\DeclareFieldFormat[jurisdiction]{number}{\bibsstring{number}~{#1}}
\DeclareFieldFormat[jurisdiction]{nameaddon}{%
  \ifinteger{#1}{\bibcpsstring{section}~\RN{#1}}{#1}}
\DeclareFieldFormat[jurisdiction]{pages}{%
  \iffieldundef{bookpagination}{#1}{\mkpageprefix[bookpagination]{#1}}}
\DeclareFieldFormat[jurisdiction]{title}{%
 \iffieldequalstr{entrysubtype}{international}{\emph{#1}}{#1}}
\DeclareFieldFormat[jurisdiction]{notacomm}{nt\adddotspace#1}%    \end{macrocode}
% A new macro to manage authors of \bibtype{jurisdiction} entries.
%    \begin{macrocode}
\newbibmacro*{juris:author}{%
%    \end{macrocode}
% Use the default name format: ``name surname''
%    \begin{macrocode}
\DeclareNameAlias{sortname}{default}%
  \ifboolexpr{%
    test \ifuseauthor
    and
    not test {\ifnameundef{author}}
  }%
    {%
    \iffieldequalstr{type}{conclusions}{%
    \printtext{Conclusioni dell'Avv\adddotspace generale}%
    \setunit{\addspace}}{}%
\printnames{author}%
     \iffieldundef{authortype}
       {}
       {\setunit{\addcomma\space}%
  \usebibmacro{authorstrg}}}
    {}%
\iffieldequalstr{type}{conclusions}{%
    \setunit{\addspace}%
  \printtext{presentate il}%
    \setunit{\addspace}%
  }{%
    \setunit{\addcomma\space}%
  \iffieldundef{nameaddon}{}{%
\printtext{\printfield{nameaddon}%
\setunit{\addcomma\space}}}%
  }%
        \printeventdate}

\newbibmacro*{addendum+pubstate:juris}{%
  \printfield{usera}%notacomm
  \newunit\newblock
  \printfield{addendum}%
  \newunit\newblock
  \printfield{pubstate}%
  \ifdefstring{\bbx@origfields}{none}{}{%
   \newunit\newblock
  \usebibmacro{origdata:article-inbook}}%
  \newunit\newblock
  \usebibmacro{library}}
    
\DeclareBibliographyDriver{jurisdiction}{%
  \usebibmacro{bibindex}%
  \usebibmacro{begentry}%
  \usebibmacro{juris:author}%
  \setunit{\labelnamepunct}\newblock
  \usebibmacro{title}%
  \newunit\newblock
  \usebibmacro{series+number}%
  \printlist{language}%
  \newunit\newblock
  \usebibmacro{byauthor}%
  \newunit\newblock
  \iffieldundef{booktitle}{}{%
  \usebibmacro{in:}%
  \usebibmacro{maintitle+booktitle}%
  \newunit\newblock
\printtext{%
   \printfield{labelyear}%
   \printfield{extrayear}}
     \usebibmacro{byeditor+others}%
  \newunit\newblock
  \printfield{edition}%
  \newunit
%  \iffieldundef{maintitle}
    {\printfield{volume}%
     \printfield{part}}
    {}%
  \newunit
  \printfield{volumes}%
  \newunit\newblock
  \printfield{note}%
  \newunit\newblock
  \usebibmacro{publisher+location+date}%
  \newunit\newblock
  \usebibmacro{chapter+pages}%
  \newunit\newblock
  \iftoggle{bbx:isbn}
    {\printfield{isbn}}
    {}%
  \newunit\newblock}
  \usebibmacro{doi+eprint+url}%
  \newunit\newblock
  \usebibmacro{addendum+pubstate:juris}%
  \newblock
  \usebibmacro{phil:related}%
  \newunit\newblock
  \usebibmacro{pageref}%
  \usebibmacro{finentry}}
%    \end{macrocode}
% Define new fields for \bibtype{jurisdiction} entry types and
% \opt{orig-} fields mechanism:
%    \begin{macrocode}
\DeclareStyleSourcemap{
\maps[datatype=bibtex]{
    \map{
     \step[fieldsource=court         , fieldtarget=author]
     \step[fieldsource=notacomm      , fieldtarget=usera]
     \step[fieldsource=section       , fieldtarget=nameaddon]
     \step[fieldsource=transdate     , fieldtarget=origdate]
     \step[fieldsource=transtitle    , fieldtarget=origtitle]
     \step[fieldsource=tranpublisher , fieldtarget=origpublisher]
     \step[fieldsource=translocation , fieldtarget=origlocation]
     \step[fieldsource=transbooktitle, fieldtarget=usera]
     \step[fieldsource=transnote     , fieldtarget=userb]
     \step[fieldsource=transpages    , fieldtarget=userc]
     \step[fieldsource=origbooktitle , fieldtarget=usera]
     \step[fieldsource=orignote      , fieldtarget=userb]
     \step[fieldsource=origpages     , fieldtarget=userc]
      }
    }
  }
\DeclareDataInheritance{*}{*}{\noinherit{annotation}}

\DeclareFieldAlias[jurisdiction]{usera}[jurisdiction]{notacomm}
\DeclareFieldAlias[jurisdiction]{nameadddon}[jurisdiction]{section}
\DeclareFieldAlias[jurisdiction]{author}[jurisdiction]{court}
%    \end{macrocode}
% \iffalse
%</standard-bbx>
% \fi
%
% \subsection{\file{philosophy-verbose.bbx}}
%
% \iffalse
%<*verbose-bbx>
% \fi
%
% \subsubsection{Initial settings}
%
%    \begin{macrocode}
\RequireBibliographyStyle{authortitle}
\RequireBibliographyStyle{philosophy-standard}
%    \end{macrocode}
% In the list of shorthands we always use the shorthand
% for the cross-referenced entries: 
%    \begin{macrocode}
\AtBeginShorthands{%
\DeclareCiteCommand{\bbx@crossref@inbook}%
  {}%
  {\iffieldundef{shorthand}%
    {\usebibmacro{inbook:full}}%
    {\usebibmacro{cite:shorthand}}}%
  {}%
  {}%
\DeclareCiteCommand{\bbx@crossref@incollection}%
  {}%
  {\iffieldundef{shorthand}%
    {\usebibmacro{incollection:full}}%
    {\usebibmacro{cite:shorthand}}}%  
  {}%
  {}%
}
%    \end{macrocode}
% \subsubsection{Authors and editors}
%    \begin{macrocode}
\renewbibmacro*{author}{%
  \ifboolexpr{%
    test \ifuseauthor
    and
    not test {\ifnameundef{author}}
  }%
    {\usebibmacro{bbx:dashcheck}%
       {\bibnamedash}%
       {\printnames{author}%
        \iffieldundef{nameaddon}{}%
        {\setunit{\addspace}%
        \printtext[brackets]{\printfield{nameaddon}}}%MOD
        \setunit{\printdelim{editorstrgdelim}}%
        \usebibmacro{bbx:savehash}}%
        \usebibmacro{authorstrg}}%
    {\global\undef\bbx@lasthash}}%

\renewbibmacro*{bbx:editor}[1]{%
  \ifboolexpr{%
    test \ifuseeditor
    and
    not test {\ifnameundef{editor}}
  }%
    {\usebibmacro{bbx:dashcheck}%
       {\bibnamedash}%
       {\printnames{editor}%
\setunit{\printdelim{editorstrgdelim}}%MOD
        \usebibmacro{bbx:savehash}}%
     \usebibmacro{#1}%
     \clearname{editor}}%
    {\global\undef\bbx@lasthash}}%

\renewbibmacro*{bbx:translator}[1]{%
  \ifboolexpr{%
    test \ifusetranslator
    and
    not test {\ifnameundef{translator}}
  }%
    {\usebibmacro{bbx:dashcheck}%
       {\bibnamedash}%
       {\printnames{translator}%
\setunit{\printdelim{editorstrgdelim}}%MOD
        \usebibmacro{bbx:savehash}}%
     \usebibmacro{#1}%
     \clearname{translator}}%
    {\global\undef\bbx@lasthash}}%

\newbibmacro*{nodash:author}{%
  \ifboolexpr{%
    test \ifuseauthor
    and
    not test {\ifnameundef{author}}
  }%
    {\printnames{author}%
        \iffieldundef{nameaddon}{}%
        {\setunit{\addspace}%
        \printtext[brackets]{\printfield{nameaddon}}}%
\setunit{\addcomma\space}%
     \usebibmacro{authorstrg}}%
    {\global\undef\bbx@lasthash}}

\newbibmacro*{nodash:editor+others}{%
  \usebibmacro{nodash:bbx:editor}{editor+othersstrg}}%
\newbibmacro*{nodash:bbx:editor}[1]{%
  \ifboolexpr{%
    test \ifuseeditor
    and
    not test {\ifnameundef{editor}}
  }%
    {\printnames{editor}%
\setunit{\addspace}%
     \usebibmacro{#1}%
     \clearname{editor}}%
    {\global\undef\bbx@lasthash}}

\newbibmacro*{nodash:author/editor+others/translator+others}{%
  \ifboolexpr{
    test \ifuseauthor
    and
    not test {\ifnameundef{author}}
  }
    {\usebibmacro{nodash:author}}
    {\ifboolexpr{
       test \ifuseeditor
       and
       not test {\ifnameundef{editor}}
     }
       {\usebibmacro{nodash:editor+others}}
       {\usebibmacro{translator+others}}}}

\renewbibmacro*{relateddate}{%
  \setunit*{\addspace}%
  \printdate\ifdefstring{\bbx@editionformat}{superscript}%
    {\printfield{edition}}{}}
\renewbibmacro*{commarelateddate}{%
  \setunit*{\addcomma\space}%
  \printdate\ifdefstring{\bbx@editionformat}{superscript}%
    {\printfield{edition}}{}}
%    \end{macrocode}
% \subsubsection{Crossreferences}
%    \begin{macrocode}
\DeclareCiteCommand{\bbx@crossref@inbook}%
  {}%
  {\ifciteseen{%
   \ifthenelse{\value{listtotal}=2}%
      {\printnames[][-\value{maxnamesincross}]{labelname}}%
      {\printnames[][-\value{minnamesincross}]{labelname}}%
  \setunit*{\addcomma\space}\printtext{%
    \printfield[citetitle]{labeltitle}%
      \iftoggle{cbx:commacit}{\setunit{\addcomma\space}}%
   {\setunit{\addspace\midsentence}}%
       \bibstring{opcit}}}%
    {\DeclareNameAlias{sortname}{default}%
    \usebibmacro{usedriver:book}}}%
  {}%
  {}%
\DeclareCiteCommand{\bbx@crossref@incollection}%
  {}%
  {\ifciteseen{%
\ifthenelse{\value{listtotal}=2}%
      {\printnames[][-\value{maxnamesincross}]{labelname}}%
      {\printnames[][-\value{minnamesincross}]{labelname}}%
      \setunit{\addspace}%
      \usebibmacro{editorstrg}%
  \setunit*{\addcomma\space}\printtext{%
    \printfield[citetitle]{labeltitle}%
       \iftoggle{cbx:commacit}{\setunit{\addcomma\space}}%
    {\setunit{\addspace\midsentence}}%
        \bibstring{opcit}}}%
    {\DeclareNameAlias{sortname}{default}%
    \usebibmacro{usedriver:collection}}}%
  {}%
  {}%
%    \end{macrocode}
% \subsubsection{Bibliography drivers}
%    \begin{macrocode}    
\DeclareBibliographyDriver{book}{%
  \usebibmacro{bibindex}%
  \usebibmacro{begentry}%
  \usebibmacro{author/editor+others/translator+others}%
  \setunit{\labelnamepunct}\newblock
  \usebibmacro{maintitle+title}%
  \newunit
  \printlist{language}%
  \newunit\newblock
  \usebibmacro{byauthor}%
  \newunit\newblock
  \usebibmacro{byeditor+others}%
  \newunit\newblock
  \ifdefstring{\bbx@editionformat}{superscript}{}%
  {\printfield{edition}%
  \newunit}%
  \printfield{volumes}%
  \newunit\newblock
  \usebibmacro{series+number}%
  \newunit\newblock
  \printfield{note}%
  \newunit\newblock
  \usebibmacro{publisher+location+date}%
  \newunit
  \iffieldundef{maintitle}
    {\printfield{volume}%
     \printfield{part}}
    {}%
  \newunit\newblock
  \usebibmacro{chapter+pages}%
  \newunit
  \printfield{pagetotal}%
  \newunit\newblock
  \iftoggle{bbx:isbn}
    {\printfield{isbn}}
    {}%
  \newunit\newblock
  \usebibmacro{doi+eprint+url}%
  \newunit\newblock
  \usebibmacro{addendum+pubstate}%
  \newblock
  \usebibmacro{phil:related}%
  \newunit\newblock
  \usebibmacro{pageref}%
  \usebibmacro{finentry}}

\DeclareBibliographyDriver{collection}{%
  \usebibmacro{bibindex}%
  \usebibmacro{begentry}%
  \usebibmacro{editor+others}%
  \setunit{\labelnamepunct}\newblock
  \usebibmacro{maintitle+title}%
  \newunit
  \printlist{language}%
  \newunit\newblock
  \usebibmacro{byeditor+others}%
  \newunit\newblock
  \ifdefstring{\bbx@editionformat}{superscript}{}%
  {\printfield{edition}%
  \newunit}%
  \iffieldundef{maintitle}
    {\printfield{volume}%
     \printfield{part}}
    {}%
  \newunit
  \printfield{volumes}%
  \newunit\newblock
  \usebibmacro{series+number}%
  \newunit\newblock
  \printfield{note}%
  \newunit\newblock
  \usebibmacro{publisher+location+date}%
  \newunit\newblock
  \usebibmacro{chapter+pages}%
  \newunit
  \printfield{pagetotal}%
  \newunit\newblock
  \iftoggle{bbx:isbn}
    {\printfield{isbn}}
    {}%
  \newunit\newblock
  \usebibmacro{doi+eprint+url}%
  \newunit\newblock
  \usebibmacro{addendum+pubstate}%
  \newblock
  \usebibmacro{phil:related}%
  \newunit\newblock
  \usebibmacro{pageref}%
  \usebibmacro{finentry}}

\renewbibmacro*{inbook:full}{%
  \usebibmacro{bybookauthor}%
  \newunit\newblock
  \usebibmacro{maintitle+booktitle}%
  \newunit\newblock
  \usebibmacro{byeditor+others}%
  \newunit\newblock
  \ifdefstring{\bbx@editionformat}{superscript}{}%
  {\printfield{edition}%
  \newunit}%
  \printfield{volumes}%
  \newunit\newblock
  \usebibmacro{series+number}%
  \newunit\newblock
  \printfield{note}%
  \newunit\newblock
  \usebibmacro{publisher+location+date}%
  \newunit
  \iffieldundef{maintitle}
    {\printfield{volume}%
     \printfield{part}}
    {}%
  \newunit\newblock
  \usebibmacro{chapter+pages}%
  \newunit\newblock
  \iftoggle{bbx:isbn}
    {\printfield{isbn}}
    {}%
  \newunit\newblock
  \usebibmacro{doi+eprint+url}%
  \newunit\newblock
  \usebibmacro{addendum+pubstate:article-inbook-incoll}%
  \newblock
  \usebibmacro{phil:related}%
  \newunit\newblock
  \usebibmacro{pageref}%
  \usebibmacro{finentry}}

  \renewbibmacro*{incollection:full}{%
  \usebibmacro{maintitle+booktitle}%
  \newunit\newblock
  \usebibmacro{byeditor+others}%
  \newunit\newblock
  \ifdefstring{\bbx@editionformat}{superscript}{}%
  {\printfield{edition}%
  \newunit}%
  \printfield{volumes}%
  \newunit\newblock
  \usebibmacro{series+number}%
  \newunit\newblock
  \printfield{note}%
  \newunit\newblock
  \usebibmacro{publisher+location+date}%
  \newunit
  \iffieldundef{maintitle}
    {\printfield{volume}%
     \printfield{part}}
    {}%
  \newunit\newblock
  \usebibmacro{chapter+pages}%
  \newunit\newblock
  \iftoggle{bbx:isbn}
    {\printfield{isbn}}
    {}%
  \newunit\newblock
  \usebibmacro{doi+eprint+url}%
  \newunit\newblock
  \usebibmacro{addendum+pubstate:article-inbook-incoll}%
  \newblock
  \usebibmacro{phil:related}%
  \newunit\newblock
  \usebibmacro{pageref}%
  \usebibmacro{finentry}}

\DeclareBibliographyDriver{manual}{%
  \usebibmacro{bibindex}%
  \usebibmacro{begentry}%
  \usebibmacro{author/editor}%
  \setunit{\labelnamepunct}\newblock
  \usebibmacro{title}%
  \newunit
  \printlist{language}%
  \newunit\newblock
  \usebibmacro{byauthor}%
  \newunit\newblock
  \usebibmacro{byeditor}%
  \newunit\newblock
  \ifdefstring{\bbx@editionformat}{superscript}{}%
  {\printfield{edition}%
  \newunit\newblock}%
  \usebibmacro{series+number}%
  \newunit\newblock
  \printfield{type}%
  \newunit
  \printfield{version}%
  \newunit
  \printfield{note}%
  \newunit\newblock
  \printlist{organization}%
  \newunit
  \usebibmacro{publisher+location+date}%
  \newunit\newblock
  \usebibmacro{chapter+pages}%
  \newunit
  \printfield{pagetotal}%
  \newunit\newblock
  \iftoggle{bbx:isbn}
    {\printfield{isbn}}
    {}%
  \newunit\newblock
  \usebibmacro{doi+eprint+url}%
  \newunit\newblock
  \usebibmacro{addendum+pubstate}%
  \newblock
  \usebibmacro{phil:related}%
  \newunit\newblock
  \usebibmacro{pageref}%
  \usebibmacro{finentry}}
      
\newbibmacro*{usedriver:book}{%
  \usebibmacro{bibindex}%
  \usebibmacro{begentry}%
  \usebibmacro{nodash:author/editor+others/translator+others}%
  \setunit{\labelnamepunct}\newblock
  \usebibmacro{maintitle+title}%
  \newunit
  \printlist{language}%
  \newunit\newblock
  \usebibmacro{byauthor}%
  \newunit\newblock
  \usebibmacro{byeditor+others}%
  \newunit\newblock
  \ifdefstring{\bbx@editionformat}{superscript}{}%
  {\printfield{edition}%
  \newunit}%
  \printfield{volumes}%
  \newunit\newblock
  \usebibmacro{series+number}%
  \newunit\newblock
  \printfield{note}%
  \newunit\newblock
  \usebibmacro{publisher+location+date}%
  \newunit
  \iffieldundef{maintitle}
    {\printfield{volume}%
     \printfield{part}}
    {}%
  \newunit\newblock
  \usebibmacro{chapter+pages}%
  \newunit
  \printfield{pagetotal}%
  \newunit\newblock
  \iftoggle{bbx:isbn}
    {\printfield{isbn}}
    {}%
  \newunit\newblock
  \usebibmacro{doi+eprint+url}%
  \newunit\newblock
  \usebibmacro{addendum+pubstate}%
  \newblock
  \usebibmacro{phil:related}%
  \newunit\newblock
  \usebibmacro{pageref}%
  }
\newbibmacro*{usedriver:collection}{%
  \usebibmacro{bibindex}%
  \usebibmacro{begentry}%
  \usebibmacro{nodash:editor+others}%
  \setunit{\labelnamepunct}\newblock
  \usebibmacro{maintitle+title}%
  \newunit
  \printlist{language}%
  \newunit\newblock
  \usebibmacro{byeditor+others}%
  \newunit\newblock
  \ifdefstring{\bbx@editionformat}{superscript}{}%
  {\printfield{edition}%
  \newunit}%
  \iffieldundef{maintitle}
    {\printfield{volume}%
     \printfield{part}}
    {}%
  \newunit
  \printfield{volumes}%
  \newunit\newblock
  \usebibmacro{series+number}%
  \newunit\newblock
  \printfield{note}%
  \newunit\newblock
  \usebibmacro{publisher+location+date}%
  \newunit\newblock
  \usebibmacro{chapter+pages}%
  \newunit
  \printfield{pagetotal}%
  \newunit\newblock
  \iftoggle{bbx:isbn}
    {\printfield{isbn}}
    {}%
  \newunit\newblock
  \usebibmacro{doi+eprint+url}%
  \newunit\newblock
  \usebibmacro{addendum+pubstate}%
  \newblock
  \usebibmacro{phil:related}%
  \newunit\newblock
  \usebibmacro{pageref}%
  }
%    \end{macrocode}
% \iffalse
%</verbose-bbx>
% \fi
%
% \subsection{\file{philosophy-classic.bbx}}
%
% \iffalse
%<*classic-bbx>
% \fi
% \subsubsection{Initial settings}
%    \begin{macrocode}
\RequireBibliographyStyle{authoryear}
\RequireBibliographyStyle{philosophy-standard}

\newtoggle{bbx:square}
\newtoggle{bbx:nodate}

\DeclareBibliographyOption{square}[true]{%
  \settoggle{bbx:square}{#1}}
\DeclareBibliographyOption{nodate}[true]{%
  \settoggle{bbx:nodate}{#1}}
%    \end{macrocode}
% We define the \opt{nodate} option also to be used in the
% optional argument of \cmd{printbibliography}:
%    \begin{macrocode}
\define@key{blx@bib1}{nodate}[]{}%
\define@key{blx@bib2}{nodate}[true]{%
  \ifstrequal{#1}{false}{\togglefalse{bbx:nodate}}{}}%
%    \end{macrocode}
% The \opt{mergedate} option from \file{authoryear.bbx} must be completely redefined.
% We actually revise only the \texttt{date+extrayear} macro and all the \texttt{issue+date} macros 
% except for that one in the \cmd{bbx@opt@mergedate@maximum}.
% The test \cmd{ifboolexpr} is required to make \texttt{bbx:nodate} macro work properly and 
% the \cmd{postsepyear} command is used to surround the date label with a box of fixed width.
%    \begin{macrocode}
\def\bbx@opt@mergedate@maximum{%
  \renewbibmacro*{date+extrayear}{%
    \ifboolexpr{%
      test {\iffieldundef{date}}
      and
      test {\iffieldundef{year}}
    }%
    {\usebibmacro{bbx:nodate}}
    {\postsepyear{%
          \printfield{issue}%
           \setunit*{\addspace}%
            \iffieldsequal{year}{labelyear}
             {\printlabeldateextra}%
             {\printfield{labelyear}%
             \printfield{extrayear}}%
          }}}%
  \renewbibmacro*{date}{}%
  \renewbibmacro*{issue+date}{}}

% merge date with date label
\def\bbx@opt@mergedate@compact{%
  \renewbibmacro*{date+extrayear}{%
    \ifboolexpr{%
      test {\iffieldundef{date}}
      and
      test {\iffieldundef{year}}
    }%
    {\usebibmacro{bbx:nodate}}
    {\postsepyear{%
           \iffieldsequal{year}{labelyear}
           {\printlabeldateextra}%
           {\printfield{labelyear}%
            \printfield{extrayear}}%
           }}}%
  \renewbibmacro*{date}{}%
  \renewbibmacro*{issue+date}{%
    \iffieldundef{issue}
    {}
      {\ifdefstring{\bbx@volnumformat}{parens}%
      {\printtext{%
        \printfield{issue}%
        \printdate}}%
      {\printtext[pureparens]{%
        \printfield{issue}}}}%
      \newunit}}

% merge year-only date with date label
\def\bbx@opt@mergedate@basic{%
  \renewbibmacro*{date+extrayear}{%
    \ifboolexpr{%
      test {\iffieldundef{date}}
      and
      test {\iffieldundef{year}}
    }%
    {\usebibmacro{bbx:nodate}}
    {\postsepyear{%
         \printfield{labelyear}%
         \printfield{extrayear}%
        }}}%
  \renewbibmacro*{date}{%
    \iffieldundef{month}
    {}
    {\printdate}}%
  \renewbibmacro*{issue+date}{%
    \ifboolexpr{
      test {\iffieldundef{issue}}
      and
      test {\iffieldundef{month}}
    }
    {}
      {\ifdefstring{\bbx@volnumformat}{parens}%
      {\printtext{%
        \printfield{issue}\setunit*{\addspace}%
        \printdate}}%
      {\printtext[pureparens]{%
        \printfield{issue}\setunit*{\addspace}%
        \printdate}}}%
    \newunit}}

% merge year-only date with year-only date label
\def\bbx@opt@mergedate@minimum{%
  \renewbibmacro*{date+extrayear}{%
    \ifboolexpr{%
      test {\iffieldundef{date}}
      and
      test {\iffieldundef{year}}
    }%
    {\usebibmacro{bbx:nodate}}
    {\postsepyear{%
         \printfield{labelyear}%
         \printfield{extrayear}%
        }}}%
  \renewbibmacro*{date}{%
    \ifboolexpr{
      test {\iffieldundef{month}}
      and
      test {\iffieldundef{extrayear}}
    }
    {}
    {\printdate}}%
  \renewbibmacro*{issue+date}{%
    \ifboolexpr{
      test {\iffieldundef{issue}}
      and
      test {\iffieldundef{month}}
      and
      test {\iffieldundef{extrayear}}
    }
    {}
      {\ifdefstring{\bbx@volnumformat}{parens}%
      {\printtext{%
        \printfield{issue}\setunit*{\addspace}%
        \printdate}}%
      {\printtext[pureparens]{%
        \printfield{issue}\setunit*{\addspace}%
        \printdate}}}%
    \newunit}}

% don't merge date/issue with date label
\def\bbx@opt@mergedate@false{%
  \renewbibmacro*{date+extrayear}{%
    \ifboolexpr{%
      test {\iffieldundef{date}}
      and
      test {\iffieldundef{year}}
    }%
    {\usebibmacro{bbx:nodate}}
    {\postsepyear{%
                \printfield{labelyear}%
                \printfield{extrayear}%
            }}}%
  \renewbibmacro*{date}{\printdate}%
  \renewbibmacro*{issue+date}{%
      {\ifdefstring{\bbx@volnumformat}{parens}%
      {\printtext{%
        \printfield{issue}\setunit*{\addspace}%
        \printdate}}%
      {\printtext[pureparens]{%
        \printfield{issue}\setunit*{\addspace}%
        \printdate}}}}}
%    \end{macrocode}
% Now we can execute all the style-specific options previously defined.
% We also define other default options according to the style features.
%    \begin{macrocode}
\ExecuteBibliographyOptions{%
  nodate      = true,
  mergedate   = basic,    
  uniquename  = false,
  pagetracker = true,
  singletitle = false,
  square      = false,
  dashed      = true,
}
%    \end{macrocode}
% The \cmd{postsepyear} is introduced here for convenience.
% It will be significantly redefined in \sty{philosophy-modern.bbx} below.
%    \begin{macrocode}
\newcommand*{\postsepyear}[1]{%
  \printtext[parens]{#1}}
\newbibmacro*{bbx:nodate}{%
    \iftoggle{bbx:nodate}{%
      \postsepyear{\midsentence\bibstring{nodate}}{}}}
%    \end{macrocode}
% The \opt{classic} and \opt{modern} styles 
% redefine the \texttt{relateddate} and \texttt{commarelateddate} macros
% because the date has to be printed after the name of the author/editor.
% In the list of shorthands we need a standard entry,
% with the date at the end and no date after the name of the author/editor.
% So we overwrite these macros locally.
%    \begin{macrocode}
\AtBeginShorthands{%
\renewcommand{\labelnamepunct}{\addcomma\space}%
\renewbibmacro*{relateddate}{%
  \setunit*{\addspace}%
  \printdate}%
\renewbibmacro*{commarelateddate}{%
  \setunit*{\addcomma\space}%
  \printdate}%
%    \end{macrocode}
% In the list of shorthands the author-date format is useless
% but the cross-referenced entries still require this format.
% So we first save the \texttt{date+extrayear} then we redefine it 
% so that it print nothing and finally we restore it in the definition of \cmd{bbx@crossref@inbook}
% command. The redefinition of \cmd{postsepyear} is also required here because the next codes are inherited
% by the \sty{modern} style which globally define \cmd{postsepyear}.
%    \begin{macrocode}
\savebibmacro{date+extrayear}
\renewbibmacro*{date+extrayear}{}
\DeclareCiteCommand{\bbx@crossref@inbook}
  {\renewcommand*{\postsepyear}{\printtext[parens]}%
   \restorebibmacro{date+extrayear}}%
  {\iffieldundef{shorthand}{%
  \usebibmacro{citeindex}%
  \ifuseeditor{%
    \ifthenelse{\value{listtotal}=2}%
      {\printnames[][-\value{maxnamesincross}]{labelname}}%
      {\printnames[][-\value{minnamesincross}]{labelname}}}%
     {\usebibmacro{labeltitle}}%
    \setunit*{\addspace}%
    \usebibmacro{date+extrayear}}%
    {\usebibmacro{cite:shorthand}}}%
  {}%
  {}%
\DeclareCiteCommand{\bbx@crossref@incollection}%
  {\renewcommand*{\postsepyear}{\printtext[parens]}%
   \restorebibmacro{date+extrayear}}%
  {\iffieldundef{shorthand}{%
  \usebibmacro{citeindex}%
  \ifuseeditor{%
    \ifthenelse{\value{listtotal}=2}%
      {\printnames[][-\value{maxnamesincross}]{labelname}}%
      {\printnames[][-\value{minnamesincross}]{labelname}}}%
     {\usebibmacro{labeltitle}}%
    \setunit*{\addspace}%
    \usebibmacro{date+extrayear}}%
    {\usebibmacro{cite:shorthand}}}%
  {}%
  {}%
  }%
%    \end{macrocode}
% The \opt{editionformat=superscript} is not available for \sty{classic} and \sty{modern} styles
% so if used it produces an error message. 
%    \begin{macrocode}
\AtBeginDocument{%
  \ifdefstring{\bbx@editionformat}{superscript}%
   {\ClassError{biblatex-philosophy}
   {\MessageBreak**** Option 'editionformat=superscript' 
    available only for philosophy-verbose style}
   {\MessageBreak**** Option 'editionformat=superscript' 
    available only for philosophy-verbose style}}{}
  \iftoggle{bbx:square}
    {\renewcommand{\bibopenparen}{\bibopenbracket}%
     \renewcommand{\bibcloseparen}{\bibclosebracket}}%
    {}%
  \setcounter{maxnamesincross}{\value{maxnames}}%
  \setcounter{minnamesincross}{\value{minnames}}%
}%
\AtEveryBibitem{%
  \iffieldequalstr{entrysubtype}{classic}{%
    \togglefalse{bbx:nodate}}%
}%
%    \end{macrocode}
% \subsubsection{New macros}
% We redefine the \texttt{relateddate} bibliography macro
% to delete the date at the end of the entry.
%    \begin{macrocode}
\renewbibmacro*{relateddate}{}
\renewbibmacro*{commarelateddate}{}
%    \end{macrocode}
% \subsubsection{Authors and editors} In the |author| macro add the \bibfield{nameaddon} test which
% prints the \bibfield{nameaddon} field (if defined) inside brackets. Moreover we use the new |editorstrgdelim| delimiter
% previously defined in \file{philosophy-standard.bbx} which defaults to \cmd{addspace}. In the |editor| macro we modify only the line which uses the |editorstrgdelim| delimiter. In the |translator| macro we modify also the line with |#1| (this is missing in the code provided by \file{authoryear.bbx}).
%    \begin{macrocode}
\renewbibmacro*{author}{%
  \ifboolexpr{
    test \ifuseauthor
    and
    not test {\ifnameundef{author}}
  }
    {\usebibmacro{bbx:dashcheck}
       {\bibnamedash}
       {\usebibmacro{bbx:savehash}%
        \printnames{author}%
        \iffieldundef{nameaddon}{}%
        {\setunit{\addspace}%
        \printtext[brackets]{\printfield{nameaddon}}}%*
        \iffieldundef{authortype}
          {\setunit{\printdelim{nameyeardelim}}}
          {\setunit{\printdelim{editorstrgdelim}}}}%*
     \iffieldundef{authortype}
       {}
       {\usebibmacro{authorstrg}%
        \setunit{\printdelim{nameyeardelim}}}}%
    {\global\undef\bbx@lasthash
     \usebibmacro{labeltitle}%
     \setunit*{\printdelim{nonameyeardelim}}}%
  \usebibmacro{date+extrayear}}

\renewbibmacro*{bbx:editor}[1]{%
  \ifboolexpr{
    test \ifuseeditor
    and
    not test {\ifnameundef{editor}}
  }
    {\usebibmacro{bbx:dashcheck}
       {\bibnamedash}
       {\printnames{editor}%
        \setunit{\printdelim{editorstrgdelim}}%MOD
        \usebibmacro{bbx:savehash}}%
     \usebibmacro{#1}%
     \clearname{editor}%
     \setunit{\printdelim{nameyeardelim}}}%
    {\global\undef\bbx@lasthash
     \usebibmacro{labeltitle}%
     \setunit*{\printdelim{nonameyeardelim}}}%
  \usebibmacro{date+extrayear}}

\renewbibmacro*{bbx:translator}[1]{%
  \ifboolexpr{
    test \ifusetranslator
    and
    not test {\ifnameundef{translator}}
  }
    {\usebibmacro{bbx:dashcheck}
       {\bibnamedash}
       {\printnames{translator}%
        \setunit{\printdelim{editorstrgdelim}}%MOD
        \usebibmacro{bbx:savehash}}%
     \usebibmacro{#1}%MOD
     \clearname{translator}%
     \setunit{\printdelim{nameyeardelim}}}%
    {\global\undef\bbx@lasthash
     \usebibmacro{labeltitle}%
     \setunit*{\printdelim{nonameyeardelim}}}%
  \usebibmacro{date+extrayear}}
%    \end{macrocode}
% When the \bibtype{incollection}s entries have no author, editor or translator the title is used in place of the label. As the title is printed inside quotes by default, the closing quotes end on a new line when using the \sty{modern} style. This is strange and, at least for me, unexpected. To avoid it we add \cmd{blx@postpunct}.
%    \begin{macrocode}
\renewbibmacro*{labeltitle}{%
  \iffieldundef{label}
    {\iffieldundef{shorttitle}
       {\printfield{title}%
        \clearfield{title}}
       {\printfield[title]{shorttitle}}\blx@postpunct}
    {\printfield{label}}}
%    \end{macrocode}
% \subsubsection{Crossreferences}
%    \begin{macrocode}
\DeclareCiteCommand{\bbx@crossref@inbook}%
  {}%
  {\usebibmacro{citeindex}%
  \ifuseeditor{%
    \ifthenelse{\value{listtotal}=2}%
      {\printnames[][-\value{maxnamesincross}]{labelname}}%
      {\printnames[][-\value{minnamesincross}]{labelname}}}%
     {\usebibmacro{labeltitle}}%
    \setunit*{\addspace}%
    \printtext[bibhyperref]{\usebibmacro{date+extrayear}}}%
  {}%
  {}%
\DeclareCiteCommand{\bbx@crossref@incollection}%
  {}%
  {\usebibmacro{citeindex}%
  \ifuseeditor{%
    \ifthenelse{\value{listtotal}=2}%
      {\printnames[][-\value{maxnamesincross}]{labelname}}%
      {\printnames[][-\value{minnamesincross}]{labelname}}}%
     {\usebibmacro{labeltitle}}%
    \setunit*{\addspace}%
    \printtext[bibhyperref]{\usebibmacro{date+extrayear}}}%
  {}%
  {}%
%    \end{macrocode}
% \iffalse
%</classic-bbx>
% \fi
%
% \subsection{\file{philosophy-modern.bbx}}
%
% \iffalse
%<*modern-bbx>
% \fi
% \subsubsection{Initial settings}
%    \begin{macrocode}
\RequireBibliographyStyle{philosophy-classic}
%    \end{macrocode}
% The \sty{modern} style has only one specific option (\opt{yearleft})
% which is turned off by default. The other compatible option is \opt{nodate} and is inherited from
% \sty{philosophy-classic.bbx}. 
%    \begin{macrocode}
\newtoggle{bbx:yearleft}
\DeclareBibliographyOption{yearleft}[true]{%
  \settoggle{bbx:yearleft}{#1}}
%    \end{macrocode}
% We define here the \opt{restoreclassic} option for the \cmd{printbibliography} and  \cmd{printbiblist} commands.
%    \begin{macrocode}
\define@key{blx@biblist1}{restoreclassic}[]{}
\define@key{blx@biblist2}{restoreclassic}[true]{\setkeys{blx@bib2}{restoreclassic}}%
\define@key{blx@bib1}{restoreclassic}[]{}
\define@key{blx@bib2}{restoreclassic}[true]{%
\ifstrequal{#1}{true}{%
\setlength{\bibhang}{\parindent}%
\renewcommand{\labelnamepunct}{\newunitpunct}%
\renewcommand*{\postsepyear}[1]{\printtext[parens]{##1}}%
\renewbibmacro*{author}{%
  \ifboolexpr{
    test \ifuseauthor
    and
    not test {\ifnameundef{author}}
  }
    {\usebibmacro{bbx:dashcheck}
       {\bibnamedash}%
       {\usebibmacro{bbx:savehash}%
        \printnames{author}%
        \iffieldundef{nameaddon}{}%
        {\setunit{\addspace}%
        \printtext[brackets]{\printfield{nameaddon}}}%*
        \iffieldundef{authortype}
          {\setunit{\printdelim{nameyeardelim}}}%
          {\setunit{\printdelim{editorstrgdelim}}}}%*
     \iffieldundef{authortype}
       {}%
       {\usebibmacro{authorstrg}%
        \setunit{\printdelim{nameyeardelim}}}}%
    {\global\undef\bbx@lasthash
     \usebibmacro{labeltitle}%
     \setunit*{\printdelim{nonameyeardelim}}}%
  \usebibmacro{date+extrayear}}%
\renewbibmacro*{bbx:editor}[1]{%
  \ifboolexpr{
    test \ifuseeditor
    and
    not test {\ifnameundef{editor}}
  }
    {\usebibmacro{bbx:dashcheck}
       {\bibnamedash}%
       {\printnames{editor}%
        \setunit{\printdelim{editorstrgdelim}}%
        \usebibmacro{bbx:savehash}}%
     \usebibmacro{##1}%
     \clearname{editor}%
     \setunit{\printdelim{nameyeardelim}}}%
    {\global\undef\bbx@lasthash
     \usebibmacro{labeltitle}%
     \setunit*{\printdelim{nonameyeardelim}}}%
  \usebibmacro{date+extrayear}}%
\renewbibmacro*{bbx:translator}[1]{%
  \ifboolexpr{
    test \ifusetranslator
    and
    not test {\ifnameundef{translator}}
  }
    {\usebibmacro{bbx:dashcheck}
       {\bibnamedash}%
       {\printnames{translator}%
        \setunit{\printdelim{editorstrgdelim}}%
        \usebibmacro{bbx:savehash}}%
     \usebibmacro{##1}%
     \clearname{translator}%
     \setunit{\printdelim{nameyeardelim}}}%
    {\global\undef\bbx@lasthash
     \usebibmacro{labeltitle}%
     \setunit*{\printdelim{nonameyeardelim}}}%
  \usebibmacro{date+extrayear}}%
}{}}%
%    \end{macrocode}
% Execute default options.
%    \begin{macrocode}
\ExecuteBibliographyOptions{yearleft=false}
%    \end{macrocode}
% The separator to be printed after the name is omitted in the \sty{modern} style.
%    \begin{macrocode}
\renewcommand{\labelnamepunct}{}
%    \end{macrocode}
% We declare and set two new lengths: \cmd{yeartitle} and \cmd{postnamesep}.
%    \begin{macrocode}
\newlength{\yeartitle}
\newlength{\postnamesep}
\setlength{\yeartitle}{0.8em}
\setlength{\postnamesep}{0.5ex plus 2pt minus 1pt}
%    \end{macrocode}
% These three standard lengths are redefined according to the desired characteristics.
%    \begin{macrocode}
\setlength{\bibitemsep}{\postnamesep}
\setlength{\bibnamesep}{1.5ex plus 2pt minus 1pt}
\setlength{\bibhang}{4\parindent}
%    \end{macrocode}
% In the list of shorthands we in fact restore the classic style
% resetting \cmd{postsep} and \cmd{labelnamepunct}.
%    \begin{macrocode}
\AtBeginShorthands{%
  \renewcommand{\postsep}{\addspace}%
  \renewcommand{\labelnamepunct}{\newunitpunct}}
\AtBeginBibliography{%
  \iftoggle{bbx:yearleft}{%
    \setlength{\yeartitle}{\fill}}{}}
%    \end{macrocode}
% The next two codes are the core of the \sty{modern} style. 
% \cmd{postsep} is the space to be printed after the name (author/editor\dots)
% and \cmd{postsepyear} sets the box that encloses the date label. \cmd{nopunct}
% is required to remove the potential punctuation in the field to be printed after the date label.
% This is useful for entries without an author or an editor such 
% as \bibtype{periodical} or \bibtype{online}.
%    \begin{macrocode}
\newcommand{\postsep}{%
  \null\par\nobreak\vskip\postnamesep%
    \hskip-\bibhang\ignorespaces}
\renewcommand*{\postsepyear}[1]{%
  \printtext{\makebox[\bibhang][r]{%
    #1\hskip\yeartitle}}\nopunct}
\renewbibmacro*{bbx:nodate}{%
  \postsepyear{%
    \iftoggle{bbx:nodate}{%
      \midsentence\bibstring{nodate}}{}}}
%    \end{macrocode}
% \subsubsection{Authors and editors}
%    \begin{macrocode}
\renewbibmacro*{author}{%
  \ifboolexpr{
    test \ifuseauthor
    and
    not test {\ifnameundef{author}}
  }
    {\usebibmacro{bbx:dashcheck}
       {}%
       {\usebibmacro{bbx:savehash}%
        \printnames{author}%
        \iffieldundef{nameaddon}{}%
        {\setunit{\addspace}%
        \printtext[brackets]{\printfield{nameaddon}}}%*
        \postsep}%
     \usebibmacro{date+extrayear}%
       \iffieldundef{authortype}
         {}%
         {\usebibmacro{authorstrg}%
         \printtext{\addcomma\space}}}%
    {\global\undef\bbx@lasthash
     \usebibmacro{labeltitle}%
       \postsep%
     \usebibmacro{date+extrayear}%
     }%
  }

\renewbibmacro*{bbx:editor}[1]{%
  \ifboolexpr{%
    test \ifuseeditor
    and
    not test {\ifnameundef{editor}}
  }%
  {\usebibmacro{bbx:dashcheck}%
    {}%
    {\printnames{editor}%
          \postsep%
    \usebibmacro{bbx:savehash}}%
    \usebibmacro{date+extrayear}%
    \usebibmacro{#1}%
    \clearname{editor}%
    \printtext{\addcomma\space}%
  }%
  {\global\undef\bbx@lasthash%
    \usebibmacro{labeltitle}%
      \postsep%
    \usebibmacro{date+extrayear}%
  }%
}%

\renewbibmacro*{bbx:translator}[1]{%
  \ifboolexpr{%
    test \ifusetranslator
    and
    not test {\ifnameundef{translator}}
  }%
  {\usebibmacro{bbx:dashcheck}%
    {}%
    {\printnames{translator}%
          \postsep%
    \usebibmacro{bbx:savehash}}%
    \usebibmacro{date+extrayear}%
    \usebibmacro{#1}%
    \clearname{translator}%
    \printtext{\addcomma\space}%
  }%
  {\global\undef\bbx@lasthash%
    \usebibmacro{labeltitle}%
      \postsep%
    \usebibmacro{date+extrayear}%
  }%
}%
%    \end{macrocode}
% \subsubsection{Crossreferences}
%    \begin{macrocode}
\DeclareCiteCommand{\bbx@crossref@inbook}%
  {\renewcommand*{\postsepyear}{\printtext[parens]}}%
  {\usebibmacro{citeindex}%
  \ifuseeditor{%
    \ifthenelse{\value{listtotal}=2}%
      {\printnames[][-\value{maxnamesincross}]{labelname}}%
      {\printnames[][-\value{minnamesincross}]{labelname}}}%
     {\usebibmacro{labeltitle}}%
    \setunit*{\addspace}%
    \printtext[bibhyperref]{\usebibmacro{date+extrayear}}}%
  {}%
  {}%
\DeclareCiteCommand{\bbx@crossref@incollection}%
  {\renewcommand*{\postsepyear}{\printtext[parens]}}%
  {\usebibmacro{citeindex}%
  \ifuseeditor{%
    \ifthenelse{\value{listtotal}=2}%
      {\printnames[][-\value{maxnamesincross}]{labelname}}%
      {\printnames[][-\value{minnamesincross}]{labelname}}}%
     {\usebibmacro{labeltitle}}%
    \setunit*{\addspace}%
    \printtext[bibhyperref]{\usebibmacro{date+extrayear}}}%   
  {}%
  {}%
%    \end{macrocode}
% \iffalse
%</modern-bbx>
% \fi
%
%
% \subsection{\file{philosophy-verbose.cbx}}
%
% \iffalse
%<*verbose-cbx>
% \fi
%
% \subsubsection{Initial settings}
%    \begin{macrocode}
\RequireCitationStyle{verbose-trad2}

\newtoggle{cbx:commacit}

\DeclareBibliographyOption{commacit}[true]{%
  \settoggle{cbx:commacit}{#1}}

\ExecuteBibliographyOptions{%
  idemtracker=false,
  loccittracker=strict,
  commacit=false}
%    \end{macrocode}
% The \bibfield{annotation} field is omitted in every citation:
%    \begin{macrocode}
\AtEveryCite{\togglefalse{bbx:annotation}}
%    \end{macrocode}
% \subsubsection{New macros}
% These two macros come from \file{verbose-trad1.cbx} without any changes:
%    \begin{macrocode}
\newbibmacro*{cite:opcit}{%
  \printtext[bibhyperlink]{\bibstring[\mkibid]{opcited}}}

\newbibmacro*{cite:loccit}{%
  \printtext{%
    \bibhyperlink{cite\csuse{cbx@lastcite@\thefield{entrykey}}}{%
      \bibstring[\mkibid]{loccit}}}%
  \global\toggletrue{cbx:loccit}}
%    \end{macrocode}
% The following macros come from \file{verbose-trad2.cbx} and has been redefined according to the desired features.
%    \begin{macrocode}
\renewbibmacro*{cite:ibid}{%
  \ifloccit
  {\usebibmacro{cite:loccit}}{%
    \printtext{%
      \bibhyperlink{cite\csuse{cbx@lastcite@\thefield{entrykey}}}{%
        \bibstring[\mkibid]{ibidem}}}}}%

\renewbibmacro*{cite:title}{%
  \ifsingletitle{\usebibmacro{cite:opcit}}{%
    \printtext[bibhyperlink]{%
      \printfield[citetitle]{labeltitle}%
      \iftoggle{cbx:commacit}{\setunit{\addcomma\space}}%
      {\setunit{\addspace\midsentence}}}%
    \bibstring{opcit}}}
%    \end{macrocode}
% \subsubsection{Citation commands}
% The |cite:full| macro employs the bibliography driver to print the entry
% so it has to be redefined in order to use the |scdefault| name format
% when \opt{scauthor=cite} or \opt{scauthor=full} options are active. The test
% for the \opt{shorthandintro} option allows for shorthand also in the first citation of an entry.
%    \begin{macrocode}       
\newbibmacro{cite:full:noshorthand}{%
    \usebibmacro{cite:full:citepages}%
    \global\toggletrue{cbx:fullcite}%
    \printtext[bibhypertarget]{%
      \usedriver
      {\iftoggle{cbx:scauthorscite}{%
      \DeclareNameAlias{sortname}{scdefault}}%
      {\DeclareNameAlias{sortname}{default}}}%
      {\thefield{entrytype}}}}

\renewbibmacro*{cite:full}{%
\iffieldundef{shorthand}
  {\usebibmacro{cite:full:noshorthand}}
  {\iftoggle{cbx:shorthandintro}
    {\usebibmacro{cite:full:noshorthand}%
     \usebibmacro{shorthandintro}}%
    {\usebibmacro{cite:shorthand}}}}

\renewbibmacro*{cite:idem}{%
  \iftoggle{cbx:scauthorscite}{%
    \bibstring[\mkbibsc]{idem\thefield{gender}}}{%
    \bibstring[\mkibid]{idem\thefield{gender}}}%
  \setunit{\nametitledelim}}
%    \end{macrocode}
% A new macro to be used in the new \cmd{ccite} command defined below.
%    \begin{macrocode}       
\newbibmacro*{ccite:cite}{%
  \usebibmacro{related:clearauthors}%
  \usebibmacro{cite:citepages}%
  \global\togglefalse{cbx:fullcite}%
  \global\togglefalse{cbx:loccit}%
  \bibhypertarget{cite\the\value{instcount}}{%
    \ifciteseen
    {\iffieldundef{shorthand}
      {\usebibmacro{cite:title}%
        \usebibmacro{cite:save}}
      {\usebibmacro{cite:shorthand}}}
    {\usebibmacro{cite:full}%
      \usebibmacro{cite:save}}}}
%    \end{macrocode}
% \subsubsection{Citation commands}
% This is the only new citation command introduced by the \sty{verbose} style. It is similar to \cmd{cite} but omits the author.
%    \begin{macrocode}       
\DeclareCiteCommand{\ccite}
  {\usebibmacro{prenote}}%
  {\usebibmacro{citeindex}%
    \usebibmacro{ccite:cite}}
  {\multicitedelim}
  {\usebibmacro{cite:postnote}}
%    \end{macrocode}
%
% \iffalse
%</verbose-cbx>
% \fi
%
% \subsection{\file{philosophy-classic.cbx}}
%
% \iffalse
%<*classic-cbx>
% \fi
%
% \subsubsection{Initial settings}
%
%    \begin{macrocode}
\RequireCitationStyle{authoryear-comp}

\ExecuteBibliographyOptions{citetracker}

\newcommand{\switchATAY}[2]{%
  \iffieldequalstr{entrysubtype}{classic}%
    {\usebibmacro{#1}}%
    {\usebibmacro{#2}}}
%    \end{macrocode}
% \subsubsection{New macros}
% The |cbx:testshorthand| macro provide a test for the
% \opt{shorthandintro} option. This is the same for both author-title and author-year styles. The shorthand intro is printed only if the \opt{shorthandintro} option is active and the entry has not been previously cited. Note that this macro is used only when the shorthand exists (see below).
%    \begin{macrocode} 
\newbibmacro*{cbx:testshorthand}[1]{%
\ifboolexpr{
  not test {\iftoggle{cbx:shorthandintro}}
    or
  test \ifciteseen}
  {\usebibmacro{cite:shorthand}}{\usebibmacro{#1}%
   \usebibmacro{shorthandintro}}}
%    \end{macrocode}
% \paragraph{Author-title macros}
% Import from \file{authortitle-comp.cbx} all the macros but |cite:shorthand| that has been loaded above.
%    \begin{macrocode}
\newbibmacro*{cite:init:AT}{%
  \ifnumless{\value{multicitecount}}{2}
    {\global\boolfalse{cbx:parens}%
     \global\undef\cbx@lasthash}%
    {\iffieldundef{prenote}%
       {}%
       {\global\undef\cbx@lasthash}}}

\newbibmacro*{cite:reinit:AT}{%
  \global\undef\cbx@lasthash}

\newbibmacro*{cite:AT:noshorthand}{%
\iffieldequals{namehash}{\cbx@lasthash}
       {\setunit{\compcitedelim}}
       {\ifnameundef{labelname}
          {}%
          {\printnames{labelname}%
           \setunit{\printdelim{nametitledelim}}}%
        \savefield{namehash}{\cbx@lasthash}}%
     \usebibmacro{cite:title:AT}}

\newbibmacro*{cite:AT}{%
  \iffieldundef{shorthand}
    {\usebibmacro{cite:AT:noshorthand}}
    {\usebibmacro{cbx:testshorthand}{cite:AT:noshorthand}%
     \usebibmacro{cite:reinit:AT}}%
  \setunit{\multicitedelim}}

\newbibmacro*{citetitle:AT}{%
  \iffieldundef{shorthand}
    {\usebibmacro{cite:title:AT}}%
    {\usebibmacro{cbx:testshorthand}{cite:title:AT}}%
  \setunit{\multicitedelim}}

\newbibmacro*{textcite:AT}{%
  \iffieldequals{namehash}{\cbx@lasthash}
    {\setunit{\compcitedelim}}
    {\ifnameundef{labelname}
       {}%
       {\printnames{labelname}%
        \setunit{%
          \global\booltrue{cbx:parens}%
          \printdelim{nametitledelim}\bibopenparen}}%
     \stepcounter{textcitecount}%
     \savefield{namehash}{\cbx@lasthash}}%
  \ifnumequal{\value{citecount}}{1}
    {\usebibmacro{prenote}}
    {}%
  \iffieldundef{shorthand}
    {\usebibmacro{cite:title:AT}}%
    {\usebibmacro{cbx:testshorthand}{cite:title:AT}}%
  \setunit{%
    \ifbool{cbx:parens}
      {\bibcloseparen\global\boolfalse{cbx:parens}}
      {}%
    \textcitedelim}}

\newbibmacro*{cite:title:AT}{%
  \printtext[bibhyperref]{\printfield[citetitle]{labeltitle}}}

\newbibmacro*{textcite:postnote:AT}{%
  \ifnameundef{labelname}
    {\setunit{%
       \global\booltrue{cbx:parens}%
       \extpostnotedelim\bibopenparen}}
    {\setunit{\postnotedelim}}%
  \printfield{postnote}%
  \ifthenelse{\value{multicitecount}=\value{multicitetotal}}
    {\setunit{}%
     \printtext{%
       \ifbool{cbx:parens}
         {\bibcloseparen\global\boolfalse{cbx:parens}}
         {}}}
    {\setunit{%
       \ifbool{cbx:parens}
         {\bibcloseparen\global\boolfalse{cbx:parens}}
         {}%
       \textcitedelim}}}
%    \end{macrocode}
% \paragraph{Author-year macros}
% Import from \file{authoryear-comp.cbx} all the common macros with \file{authortitle-comp}.
%    \begin{macrocode}
\newbibmacro*{cite:init:AY}{%
  \ifnumless{\value{multicitecount}}{2}
    {\global\boolfalse{cbx:parens}%
     \global\undef\cbx@lasthash
     \global\undef\cbx@lastyear}
    {\iffieldundef{prenote}
       {}
       {\global\undef\cbx@lasthash
        \global\undef\cbx@lastyear}}}

\newbibmacro*{cite:reinit:AY}{%
  \global\undef\cbx@lasthash
  \global\undef\cbx@lastyear}

\newbibmacro*{cite:AY:noshorthand}{%
\ifthenelse{\ifnameundef{labelname}\OR\iffieldundef{labelyear}}
       {\usebibmacro{cite:label}%
        \setunit{\printdelim{nonameyeardelim}}%
        \usebibmacro{cite:labelyear+extrayear}%
        \usebibmacro{cite:reinit}}
       {\iffieldequals{namehash}{\cbx@lasthash}
          {\ifthenelse{\iffieldequals{labelyear}{\cbx@lastyear}\AND
                       \(\value{multicitecount}=0\OR\iffieldundef{postnote}\)}
             {\setunit{\addcomma}%
              \usebibmacro{cite:extrayear}}
             {\setunit{\compcitedelim}%
              \usebibmacro{cite:labelyear+extrayear}%
              \savefield{labelyear}{\cbx@lastyear}}}
          {\printnames{labelname}%
           \setunit{\printdelim{nameyeardelim}}%
           \usebibmacro{cite:labelyear+extrayear}%
           \savefield{namehash}{\cbx@lasthash}%
           \savefield{labelyear}{\cbx@lastyear}}}}

\newbibmacro*{cite:AY}{%
  \iffieldundef{shorthand}
    {\usebibmacro{cite:AY:noshorthand}}%
    {\usebibmacro{cbx:testshorthand}{cite:AY:noshorthand}%
     \usebibmacro{cite:reinit}}%
  \setunit{\multicitedelim}}

\newbibmacro*{textcite:AY:noshorthand:A}{%
\ifthenelse{\iffieldequals{labelyear}{\cbx@lastyear}\AND
                    \(\value{multicitecount}=0\OR\iffieldundef{postnote}\)}
          {\setunit{\addcomma}%
           \usebibmacro{cite:extrayear}}
          {\setunit{\compcitedelim}%
           \usebibmacro{cite:labelyear+extrayear}%
           \savefield{labelyear}{\cbx@lastyear}}}

\newbibmacro*{textcite:AY:noshorthand:B}{%
\usebibmacro{cite:label}%
           \setunit{%
             \global\booltrue{cbx:parens}%
             \printdelim{nonameyeardelim}\bibopenparen}%
           \ifnumequal{\value{citecount}}{1}
             {\usebibmacro{prenote}}
             {}%
           \usebibmacro{cite:labelyear+extrayear}}

\newbibmacro*{textcite:AY:noshorthand:C}{%
\iffieldundef{labelyear}
             {\usebibmacro{cite:label}}
             {\usebibmacro{cite:labelyear+extrayear}}%
           \savefield{labelyear}{\cbx@lastyear}}
% EXPERIMENTAL. 
%\newbibmacro*{test:shorthand}{%
%  \ifboolexpr{
%    (
%        test {\ifnumgreater{\value{citetotal}}{1}}
%        or
%        test {\iffieldundef{shorthand}}
%    )
%    or
%    (
%        test {\ifnumequal{\value{citetotal}}{1}}
%        and
%        not test {\iffieldundef{shorthand}}
%        and
%        not test \ifciteseen
%        and 
%        test {\iftoggle{cbx:shorthandintro}}%phil
%    )
%   }{\printnames{labelname}%
%        \setunit{%
%          \global\booltrue{cbx:parens}%
%          \printdelim{nameyeardelim}\bibopenparen}}
%{}%
%}

\newbibmacro*{textcite:AY}{%
  \iffieldequals{namehash}{\cbx@lasthash}
    {\iffieldundef{shorthand}
       {\usebibmacro{textcite:AY:noshorthand:A}}
       {\setunit{\compcitedelim}%
\usebibmacro{cbx:testshorthand}{textcite:AY:noshorthand:A}%
        \global\undef\cbx@lastyear}}
    {\ifnameundef{labelname}%
       {\iffieldundef{shorthand}
          {\usebibmacro{textcite:AY:noshorthand:B}}
          {\usebibmacro{cbx:testshorthand}{textcite:AY:noshorthand:B}}}
       {\printnames{labelname}%
        \setunit{%
          \global\booltrue{cbx:parens}%
          \printdelim{nameyeardelim}\bibopenparen}%
        \ifnumequal{\value{citecount}}{1}
          {\usebibmacro{prenote}}
          {}%
        \iffieldundef{shorthand}
          {\usebibmacro{textcite:AY:noshorthand:C}}%
          {\usebibmacro{cbx:testshorthand}{textcite:AY:noshorthand:C}%
           \global\undef\cbx@lastyear}}%
     \stepcounter{textcitecount}%
     \savefield{namehash}{\cbx@lasthash}}%
  \setunit{%
    \ifbool{cbx:parens}
      {\bibcloseparen\global\boolfalse{cbx:parens}}
      {}%
    \textcitedelim}}

\newbibmacro*{textcite:postnote:AY}{%
  \usebibmacro{postnote}%
  \ifthenelse{\value{multicitecount}=\value{multicitetotal}}
    {\setunit{}%
     \printtext{%
       \ifbool{cbx:parens}
   {\bibcloseparen\global\boolfalse{cbx:parens}}
   {}}}
    {\setunit{%
       \ifbool{cbx:parens}
   {\bibcloseparen\global\boolfalse{cbx:parens}}
   {}%
       \multicitedelim}}}
%    \end{macrocode}
% If the field \bibfield{entrysubtype} equals to \texttt{classic}
% the citation commands will use the author-title macros. In this way it is as if you were using the citation commands of the \sty{authortitle-comp} style.
%    \begin{macrocode}
\renewbibmacro*{cite:init}{%
   \switchATAY{cite:init:AT}{cite:init:AY}}
\renewbibmacro*{cite:reinit}{%
   \switchATAY{cite:reinit:AT}{cite:reinit:AY}}
\renewbibmacro*{cite}{%
   \switchATAY{cite:AT}{cite:AY}}
\renewbibmacro*{textcite}{%
   \switchATAY{textcite:AT}{textcite:AY}}
\renewbibmacro*{textcite:postnote}{%
   \switchATAY{textcite:postnote:AT}{textcite:postnote:AY}}
%    \end{macrocode}
% \subsubsection{Citation commands}
% These are two common commands for \sty{authortitle-comp} and \sty{authoryear-comp} that require the \cmd{switchATAY} to be executed internally.
%    \begin{macrocode}
\DeclareCiteCommand*{\cite}
  {\usebibmacro{cite:init}%
   \usebibmacro{prenote}}
  {\usebibmacro{citeindex}%
    \switchATAY{citetitle:AT}{citeyear}}%
  {}
  {\usebibmacro{postnote}}

\DeclareCiteCommand*{\parencite}[\mkbibparens]
  {\usebibmacro{cite:init}%
   \usebibmacro{prenote}}
  {\usebibmacro{citeindex}%
    \switchATAY{citetitle:AT}{citeyear}}
  {}
  {\usebibmacro{postnote}}
%    \end{macrocode}
% These citation commands are from \file{biblatex.def}. Maybe they should not be redefined. Do we really need years and titles hyperrefered?
%    \begin{macrocode}
\DeclareCiteCommand{\citetitle}
  {\boolfalse{citetracker}%
   \boolfalse{pagetracker}%
   \usebibmacro{prenote}}
  {\ifciteindex
     {\indexfield{indextitle}}
     {}%
   \printtext[bibhyperref]{\printfield[citetitle]{labeltitle}}}
  {\multicitedelim}
  {\usebibmacro{postnote}}

\DeclareCiteCommand*{\citetitle}
  {\boolfalse{citetracker}%
   \boolfalse{pagetracker}%
   \usebibmacro{prenote}}
  {\ifciteindex
     {\indexfield{indextitle}}
     {}%
   \printtext[bibhyperref]{\printfield[citetitle]{title}}}
  {\multicitedelim}
  {\usebibmacro{postnote}}

\DeclareCiteCommand{\citeyear}
  {\boolfalse{citetracker}%
   \boolfalse{pagetracker}%
   \usebibmacro{prenote}}
  {\printtext[bibhyperref]{\printfield{year}}}
  {\multicitedelim}
  {\usebibmacro{postnote}}

\DeclareCiteCommand*{\citeyear}
  {\boolfalse{citetracker}%
   \boolfalse{pagetracker}%
   \usebibmacro{prenote}}
  {\printtext[bibhyperref]{\printfield{year}\printfield{extrayear}}}
  {\multicitedelim}
  {\usebibmacro{postnote}}

\DeclareCiteCommand{\citedate}
  {\boolfalse{citetracker}%
   \boolfalse{pagetracker}%
   \usebibmacro{prenote}}
  {\printtext[bibhyperref]{\printdate}}
  {\multicitedelim}
  {\usebibmacro{postnote}}

\DeclareCiteCommand*{\citedate}
  {\boolfalse{citetracker}%
   \boolfalse{pagetracker}%
   \usebibmacro{prenote}}
  {\printtext[bibhyperref]{\printdateextra}}
  {\multicitedelim}
  {\usebibmacro{postnote}}
%    \end{macrocode}
% This is the only new command provided by the style:
%    \begin{macrocode}
\DeclareCiteCommand{\footcitet}[\mkbibfootnote]
  {\usebibmacro{cite:init}}
  {\usebibmacro{citeindex}%
   \usebibmacro{textcite}}
  {}%
  {\usebibmacro{textcite:postnote}}
%    \end{macrocode}
% This next command is now deprecated because it is substituted by the \bibfield{entrysubtype=classic} mechanism.
%    \begin{macrocode}
\DeclareCiteCommand{\sdcite}
  {\boolfalse{citetracker}%
   \boolfalse{pagetracker}%
   \usebibmacro{prenote}}
  {\indexnames{labelname}%
   \printtext[bibhyperref]{\printnames{labelname}}%
   \setunit{\addcomma\space}%
   \indexfield{indextitle}%
   \printtext[bibhyperref]{\printfield[citetitle]{labeltitle}}}
  {\multicitedelim}
  {\usebibmacro{postnote}}
%    \end{macrocode}
% \iffalse
%</classic-cbx>
% \fi
%
% \subsection{\file{philosophy-modern.cbx}}
%
% \iffalse
%<*modern-cbx>
% \fi
%
% The \sty{modern} style uses the \sty{classic} citation scheme:
%    \begin{macrocode}
\RequireCitationStyle{philosophy-classic}
%    \end{macrocode}
%
% \iffalse
%</modern-cbx>
% \fi
%
%
% \subsection{\file{italian-philosophy.lbx}}
%
% \iffalse
%<*italian-lbx>
% \fi
% The custom localization module of these style inherits the standard \file{italian.lbx} module. There is only one new string: \texttt{opcited}. 
% The other strings are redefined according to the particular features of the style.       
%
%    \begin{macrocode}
\InheritBibliographyExtras{italian}

\DeclareBibliographyExtras{%
%    \end{macrocode}
% We prefer the hyphen dash (-) to the en dash (--) for page and date ranges. 
%    \begin{macrocode}  
  \protected\def\bibrangedash{-\penalty\hyphenpenalty}%
  \protected\def\bibdaterangesep{\bibrangedash}%
}%
%    \end{macrocode}
% The \texttt{opcit} string used by the \sty{verbose-trad2} style works like the Italian `cit.' and it is already defined in the \file{italian.lbx} file with ``cit.'. The string ``cit.' is added to a truncated entry (usually after the short title) to mark that it has been previously cited. Additionally in the Italian bibliographies there is also the special string ``op. cit.'' which stands for ``the only entry'' of an author. For example, if ``Eco, \emph{Il nome della rosa}'' is the only entry of Eco cited in the paper, from the second occurence it will be abbreviated with ``Eco, \emph{op. cit.}''. So we need a new string, \texttt{opcited}, in order to get ``op. cit.'' for these cases:
%    \begin{macrocode}  
\NewBibliographyString{opcited}
%    \end{macrocode}
% First of all we inherit the italian localization module and then we define the new string \texttt{opcited} and the other strings as well.
%    \begin{macrocode}       
\DeclareBibliographyStrings{%   
  inherit            = {italian},
  opcited            = {{op\adddotspace cit\adddot}{op\adddotspace cit\adddot}},
%    \end{macrocode}
% Redefined strings:
%    \begin{macrocode}       
  ibidem             = {{ivi}{ivi}},
  loccit             = {{ibidem}{ibidem}},
  editor             = {{a cura di}{a cura di}},
  editors            = {{a cura di}{a cura di}},
  backrefpage        = {{citato a pagina}{citato a \bibsstring{page}\adddot}},
  backrefpages       = {{citato alle pagine}{citato alle \bibsstring{pages}\adddot}},  
  nodate             = {{senza data}{s\adddot d\adddot}},
  volumes            = {{volumi}{\iftoggle{bbx:classical}{voll\adddot}{vol\adddot}}},
  pages              = {{pagine}{\iftoggle{bbx:classical}{pp\adddot}{p\adddot}}},
  columns            = {{colonne}{\iftoggle{bbx:classical}{coll\adddot}{col\adddot}}},
  lines              = {{righe}{\iftoggle{bbx:classical}{rr\adddot}{r\adddot}}},
  verses             = {{versi}{\iftoggle{bbx:classical}{vv\adddot}{v\adddot}}}, 
  paragraphs         = {{paragrafi}{\iftoggle{bbx:classical}{parr\adddot}{par\adddot}}},
  byreviser          = {{revisione di}{rev\adddotspace di}},
  translationof      = {{traduzione di}{trad\adddotspace di}},
  translationas      = {{traduzione italiana}{trad\adddotspace it\adddot}},
  reviewof           = {{recensione di}{rec\adddotspace di}},
  origpubas          = {{ed\adddotspace orig\adddot}{ed\adddotspace orig\adddot}},
  astitle            = {{come}{come}}, 
  bypublisher        = {{\addcomma\space}{\addcomma\space}},
  section            = {{sezione}{sez\adddot}},
  sections           = {{sezioni}{\iftoggle{bbx:classical}{sezz\adddot}{sez\adddot}}},
  withcommentator    = {{commenti di}{commenti di}},
  withannotator      = {{annotazioni di}{annotazioni di}},
  withintroduction   = {{introduzione di}{introduzione di}},
  withforeword       = {{prefazione di}{prefazione di}},
  withafterword      = {{postfazione di}{postfazione di}},
%    \end{macrocode}
% The |endothers| and |andmore| strings must be printed in italic shape
% when using the \opt{latinemph} option so we add \cmd{mkibid}. Adding it to the wrapper
% of the \cmd{bibstring} command in the |name:andothers| and |list:andothers| macros
% is a wrong choice because some languages (for example German) uses non Latin expressions for those strings.
%    \begin{macrocode}
  andothers          = {{\mkibid{et\addabbrvspace al\adddot}}{\mkibid{et\addabbrvspace al\adddot}}},
  andmore            = {{\mkibid{et\addabbrvspace al\adddot}}{\mkibid{et\addabbrvspace al\adddot}}},
%    \end{macrocode}
% The followings strings are not yet defined in \file{italian.lbx} file:
%    \begin{macrocode}       
 reviser            = {{revisore}{rev\adddot}},% FIXME: missing
 revisers           = {{revisori}{rev\adddot}},% FIXME: missing
 founder            = {{fondatore}{fond\adddot}},% FIXME: missing
 founders           = {{fondatori}{fond\adddot}},% FIXME: missing
 continuator        = {{continuatore}{cont\adddot}},% FIXME: missing
 continuators       = {{continuatori}{cont\adddot}},% FIXME: missing
  editortr         = {{curatore e traduttore}% FIXME: missing
    {cur\adddotspace e trad\adddot}},
  editorstr        = {{curatori e traduttori}% FIXME: missing
    {cur\adddotspace e trad\adddot}},
  editorco         = {{curatore e commentatore}% FIXME: missing
    {cur\adddotspace e comm\adddot}},
  editorsco        = {{curatori e commentatori}% FIXME: missing
    {cur\adddotspace e comm\adddot}},
  editoran         = {{curatore e annotatore}% FIXME: missing
    {cur\adddotspace e annot\adddot}},
  editorsan        = {{curatori e annotatori}% FIXME: missing
    {cur\adddotspace e annot\adddot}},
  editorin         = {{curatore e introduzione}% FIXME: missing
    {cur\adddotspace e introd\adddot}},
  editorsin        = {{curatori e introduzione}% FIXME: missing
    {cur\adddotspace e introd\adddot}},
  editorfo         = {{curatore e prefazione}% FIXME: missing
    {cur\adddotspace e pref\adddot}},
  editorsfo        = {{curatori e prefazione}% FIXME: missing
    {cur\adddotspace e pref\adddot}},
  editoraf         = {{curatore e postfazione}% FIXME: missing
    {cur\adddotspace e postf\adddot}},
  editorsaf        = {{curatori e postfazione}% FIXME: missing
    {cur\adddotspace e postf\adddot}},
  editortrco       = {{curatore, traduttore\finalandcomma\ e commentatore}% FIXME: missing
    {ed.,\addabbrvspace trad\adddot\finalandcomma\ e comm\adddot}},
  editorstrco      = {{curatori, traduttori\finalandcomma\ e commentatori}% FIXME: missing
    {eds.,\addabbrvspace trad\adddot\finalandcomma\ e comm\adddot}},
  editortran       = {{curatore, traduttore\finalandcomma\ e annotatore}% FIXME: missing
    {ed.,\addabbrvspace trad\adddot\finalandcomma\ e annot\adddot}},
  editorstran      = {{curatori, traduttori\finalandcomma\ e annotatori}% FIXME: missing
    {eds.,\addabbrvspace trad\adddot\finalandcomma\ e annot\adddot}},
  editortrin       = {{curatore, traduttore\finalandcomma\ e introduzione}% FIXME: missing
    {ed.,\addabbrvspace trad\adddot\finalandcomma\ e introd\adddot}},
  editorstrin      = {{curatori, traduttori\finalandcomma\ e introduzione}% FIXME: missing
    {eds.,\addabbrvspace trad\adddot\finalandcomma\ e introd\adddot}},
  editortrfo       = {{curatore, traduttore\finalandcomma\ e prefazione}% FIXME: missing
    {ed.,\addabbrvspace trad\adddot\finalandcomma\ e pref\adddot}},
  editorstrfo      = {{curatori, traduttori\finalandcomma\ e prefazione}% FIXME: missing
    {eds.,\addabbrvspace trad\adddot\finalandcomma\ e pref\adddot}},
  editortraf       = {{curatore, traduttore\finalandcomma\ e postfazione}% FIXME: missing
    {ed.,\addabbrvspace trad\adddot\finalandcomma\ e postf\adddot}},
  editorstraf      = {{curatori, traduttori\finalandcomma\ e postfazione}% FIXME: missing
    {eds.,\addabbrvspace trad\adddot\finalandcomma\ e postf\adddot}},
  editorcoin       = {{curatore, commentatore\finalandcomma\ e introduzione}% FIXME: missing
    {ed.,\addabbrvspace comm\adddot\finalandcomma\ e introd\adddot}},
  editorscoin      = {{curatori, commentatori\finalandcomma\ e introduzione}% FIXME: missing
    {eds.,\addabbrvspace comm\adddot\finalandcomma\ e introd\adddot}},
  editorcofo       = {{curatore, commentatore\finalandcomma\ e prefazione}% FIXME: missing
    {ed.,\addabbrvspace comm\adddot\finalandcomma\ e pref\adddot}},
  editorscofo      = {{curatori, commentatori\finalandcomma\ e prefazione}% FIXME: missing
    {eds.,\addabbrvspace comm\adddot\finalandcomma\ e pref\adddot}},
  editorcoaf       = {{curatore, commentatore\finalandcomma\ e postfazione}% FIXME: missing
    {ed.,\addabbrvspace comm\adddot\finalandcomma\ e postf\adddot}},
  editorscoaf      = {{curatori, commentatori\finalandcomma\ e postfazione}% FIXME: missing
    {eds.,\addabbrvspace comm\adddot\finalandcomma\ e postf\adddot}},
  editoranin       = {{curatore, annotatore\finalandcomma\ e introduzione}% FIXME: missing
    {ed.,\addabbrvspace annot\adddot\finalandcomma\ e introd\adddot}},
  editorsanin      = {{curatori, annotatori\finalandcomma\ e introduzione}% FIXME: missing
    {eds.,\addabbrvspace annot\adddot\finalandcomma\ e introd\adddot}},
  editoranfo       = {{curatore, annotatore\finalandcomma\ e prefazione}% FIXME: missing
    {ed.,\addabbrvspace annot\adddot\finalandcomma\ e pref\adddot}},
  editorsanfo      = {{curatori, annotatori\finalandcomma\ e prefazione}% FIXME: missing
    {eds.,\addabbrvspace annot\adddot\finalandcomma\ e pref\adddot}},
  editoranaf       = {{curatore, annotatore\finalandcomma\ e postfazione}% FIXME: missing
    {ed.,\addabbrvspace annot\adddot\finalandcomma\ e postf\adddot}},
  editorsanaf      = {{curatori, annotatori\finalandcomma\ e postfazione}% FIXME: missing
    {eds.,\addabbrvspace annot\adddot\finalandcomma\ e postf\adddot}},
  editortrcoin     = {{curatore, traduttore, commentatore\finalandcomma\ e introduzione}% FIXME: missing
    {cur.,\addabbrvspace trad., comm\adddot\finalandcomma\ e introd\adddot}},
  editorstrcoin    = {{curatori, traduttori, commentatori\finalandcomma\ e introduzione}% FIXME: missing
    {cur.,\addabbrvspace trad., comm\adddot\finalandcomma\ e introd\adddot}},
  editortrcofo     = {{curatore, traduttore, commentatore\finalandcomma\ e prefazione}% FIXME: missing
    {cur.,\addabbrvspace trad., comm\adddot\finalandcomma\ e pref\adddot}},
  editorstrcofo    = {{curatori, traduttori, commentatori\finalandcomma\ e prefazione}% FIXME: missing
    {cur.,\addabbrvspace trad., comm\adddot\finalandcomma\ e pref\adddot}},
  editortrcoaf     = {{curatore, traduttore, commentatore\finalandcomma\ e postfazione}% FIXME: missing
    {cur.,\addabbrvspace trad., comm\adddot\finalandcomma\ e postf\adddot}},
  editorstrcoaf    = {{curatori, traduttori, commentatori\finalandcomma\ e postfazione}% FIXME: missing
    {cur.,\addabbrvspace trad., comm\adddot\finalandcomma\ e postf\adddot}},
  editortranin     = {{curatore, traduttore, annotatore\finalandcomma\ e introduzione}% FIXME: missing
    {cur.,\addabbrvspace trad., annot\adddot\finalandcomma\ e introd\adddot}},
  editorstranin    = {{curatori, traduttori, annotatori\finalandcomma\ e introduzione}% FIXME: missing
    {cur.,\addabbrvspace trad., annot\adddot\finalandcomma\ e introd\adddot}},
  editortranfo     = {{curatore, traduttore, annotatore\finalandcomma\ e prefazione}% FIXME: missing
    {cur.,\addabbrvspace trad., annot\adddot\finalandcomma\ e pref\adddot}},
  editorstranfo    = {{curatori, traduttori, annotatori\finalandcomma\ e prefazione}% FIXME: missing
    {cur.,\addabbrvspace trad., annot\adddot\finalandcomma\ e pref\adddot}},
  editortranaf     = {{curatore, traduttore, annotatore\finalandcomma\ e postfazione}% FIXME: missing
    {cur.,\addabbrvspace trad., annot\adddot\finalandcomma\ e postf\adddot}},
  editorstranaf    = {{curatori, traduttori, annotatori\finalandcomma\ e postfazione}% FIXME: missing
    {cur.,\addabbrvspace trad., annot\adddot\finalandcomma\ e postf\adddot}},
  translatorco     = {{traduttore e commentatore}% FIXME: missing
    {trad\adddot\ e comm\adddot}},
  translatorsco    = {{traduttori e commentatori}% FIXME: missing
    {trad\adddot\ e comm\adddot}},
  translatoran     = {{traduttore e annotatore}% FIXME: missing
    {trad\adddot\ e annot\adddot}},
  translatorsan    = {{traduttori e annotatori}% FIXME: missing
    {trad\adddot\ e annot\adddot}},
  translatorin     = {{traduzione e introduzione}% FIXME: missing
    {trad\adddot\ e introd\adddot}},
  translatorsin    = {{traduzione e introduzione}% FIXME: missing
    {trad\adddot\ e introd\adddot}},
  translatorfo     = {{traduzione e prefazione}% FIXME: missing
    {trad\adddot\ e pref\adddot}},
  translatorsfo    = {{traduzione e prefazione}% FIXME: missing
    {trad\adddot\ e pref\adddot}},
  translatoraf     = {{traduzione e postfazione}% FIXME: missing
    {trad\adddot\ e postf\adddot}},
  translatorsaf    = {{traduzione e postfazione}% FIXME: missing
    {trad\adddot\ e postf\adddot}},
  translatorcoin   = {{traduzione, commenti\finalandcomma\ e introduzione}% FIXME: missing
    {trans., comm\adddot\finalandcomma\ e introd\adddot}},
  translatorscoin  = {{traduzione, commenti\finalandcomma\ e introduzione}% FIXME: missing
    {trans., comm\adddot\finalandcomma\ e introd\adddot}},
  translatorcofo   = {{traduzione, commenti\finalandcomma\ e prefazione}% FIXME: missing
    {trans., comm\adddot\finalandcomma\ e pref\adddot}},
  translatorscofo  = {{traduzione, commenti\finalandcomma\ e prefazione}% FIXME: missing
    {trans., comm\adddot\finalandcomma\ e pref\adddot}},
  translatorcoaf   = {{traduzione, commenti\finalandcomma\ e postfazione}% FIXME: missing
    {trans., comm\adddot\finalandcomma\ e postf\adddot}},
  translatorscoaf  = {{traduzione, commenti\finalandcomma\ e postfazione}% FIXME: missing
    {trans., comm\adddot\finalandcomma\ e postf\adddot}},
  translatoranin   = {{traduzione, annotazioni\finalandcomma\ e introduzione}% FIXME: missing
    {trans., annot\adddot\finalandcomma\ e introd\adddot}},
  translatorsanin  = {{traduzione, annotazioni\finalandcomma\ e introduzione}% FIXME: missing
    {trans., annot\adddot\finalandcomma\ e introd\adddot}},
  translatoranfo   = {{traduzione, annotazioni\finalandcomma\ e prefazione}% FIXME: missing
    {trans., annot\adddot\finalandcomma\ e pref\adddot}},
  translatorsanfo  = {{traduzione, annotazioni\finalandcomma\ e prefazione}% FIXME: missing
    {trans., annot\adddot\finalandcomma\ e pref\adddot}},
  translatoranaf   = {{traduzione, annotazioni\finalandcomma\ e postfazione}% FIXME: missing
    {trans., annot\adddot\finalandcomma\ e postf\adddot}},
  translatorsanaf  = {{traduzione, annotazioni\finalandcomma\ e postfazione}% FIXME: missing
    {trans., annot\adddot\finalandcomma\ e postf\adddot}},
%byreviser        = {{rivisto da}{riv\adddotspace da}},% FIXME: missing: UNSURE
 byreviewer       = {{recensito da}{rec\adddotspace da}},% FIXME: missing: UNSURE
 byfounder        = {{fondato da}{fond\adddotspace da}},% FIXME: missing
 bycontinuator    = {{continuato da}{cont\adddotspace da}},% FIXME: missing: UNSURE
 bycollaborator   = {{in collaborazione con}{in coll\adddotspace con}},% FIXME: missing: UNSURE
 book             = {{libro}{lib\adddot}},% FIXME: missing: UNSURE
 part             = {{parte}{pt\adddot}},% FIXME: missing: UNSURE
 issue            = {{uscita}{uscita}},% FIXME: missing: UNSURE
 reprintas        = {{ristampato come}{rist\adddotspace come}},% FIXME: missing
 reprintfrom      = {{ristampato da}{rist\adddotspace da}},% FIXME: missing
% translationof    = {{traduzione di}{trad\adddotspace di}},% FIXME: missing
% translationas    = {{tradotto come}{trad\adddotspace come}},% FIXME: missing
 translationfrom  = {{tradotto da}{trad\adddotspace da}},% FIXME: missing
% reviewof         = {{recensione di}{rec\adddotspace di}},% FIXME: missing
% origpubas        = {{edizione originale}{ed\adddotspace orig\adddotspace}},% FIXME: missing: UNSURE
 origpubin        = {{originalmente pubblicato in}{orig\adddotspace pub\adddotspace in}},% FIXME: missing: UNSURE
% astitle          = {{come}{come}},% FIXME: missing
% bypublisher      = {{publicato da}{pubb\adddotspace da}},% FIXME: missing: UNSURE
% section          = {{sezione}{\S}},% FIXME: missing
% sections         = {{sezioni}{\S\S}},% FIXME: missing
 candthesis       = {{candidato}{cand\adddot}},% FIXME: missing
 software         = {{software}{software}},% FIXME: missing
 datacd           = {{CD-ROM}{CD-ROM}},% FIXME: missing
 audiocd          = {{audio CD}{audio CD}},% FIXME: missing
 urlfrom          = {{disponibile su}{disponibile su}},% FIXME: missing: UNSURE
 inpreparation    = {{in preparazione}{in preparazione}},% FIXME: missing
% submitted        = {{}{}},% FIXME: missing
 forthcoming      = {{in uscita}{in uscita}},% FIXME: missing
 prepublished     = {{pre-pubblicato}{pre-pubblicato}},% FIXME: missing: UNSURE
 thiscite         = {{specialmente}{spec\adddot}},% FIXME: missing: UNSURE
 langcatalan      = {{catalano}{catalano}},% FIXME: missing
 langcroatian     = {{croato}{croato}},% FIXME: missing
 langczech        = {{ceco}{ceco}},% FIXME: missing
 langestonian     = {{estone}{estone}},% FIXME: missing
 langfinnish      = {{finnico}{finnico}},% FIXME: missing
 langpolish       = {{polacco}{polacco}},% FIXME: missing
 langrussian      = {{russo}{russo}},% FIXME: missing
 langslovene      = {{sloveno}{sloveno}},% FIXME: missing
 fromcatalan      = {{dal catalano}{dal catalano}},% FIXME: missing
 fromcroatian     = {{dal croato}{dal croato}},% FIXME: missing
 fromczech        = {{dal ceco}{dal ceco}},% FIXME: missing
 fromestonian     = {{dall'estone}{dall'estone}},% FIXME: missing
 fromfinnish      = {{dal finnico}{dal finnico}},% FIXME: missing
 frompolish       = {{dal polacco}{dal polacco}},% FIXME: missing
 fromrussian      = {{dal russo}{dal russo}},% FIXME: missing
 fromslovene      = {{dallo sloveno}{dallo sloveno}},% FIXME: missing
 circa            = {{circa}{ca\adddot}},% FIXME: missing
 spring           = {{primavera}{prim\adddot}},% FIXME: missing
 summer           = {{estate}{est\adddot}},% FIXME: missing
 autumn           = {{autunno}{aut\adddot}},% FIXME: missing
 winter           = {{inverno}{inv\adddot}},% FIXME: missing
}
%    \end{macrocode}
% \iffalse
%</italian-lbx>
% \fi
%
% \subsection{\file{english-philosophy.lbx}}
%
% \iffalse
%<*english-lbx>
% \fi
%    \begin{macrocode}
\InheritBibliographyExtras{english}

\DeclareBibliographyExtras{%
\protected\def\bibrangedash{-\penalty\hyphenpenalty}%
\protected\def\bibdaterangesep{\bibrangedash}%
}%

\NewBibliographyString{opcited}

\DeclareBibliographyStrings{%
  inherit            = {english},
%    \end{macrocode}
% New string:
%    \begin{macrocode}     
  opcited             = {{op\adddotspace cit\adddot}{op\adddotspace cit\adddot}},
%    \end{macrocode}
% Redefined strings:
%    \begin{macrocode}     
  opcit              = {{cit\adddot}{cit\adddot}},
  ibidem             = {{ibid\adddot}{ibid\adddot}},
  loccit             = {{ibid\adddot}{ibid\adddot}},
  translationas      = {{trans\adddot}{trans\adddot}},
  withcommentator    = {{commentary by}{comment\adddot\ by}},
  withannotator      = {{annotations by}{annots\adddot\ by}},
  withintroduction   = {{introduction by}{intro\adddot\ by}},
  withforeword       = {{foreword by}{forew\adddot\ by}},
  withafterword      = {{afterword by}{afterw\adddot\ by}}
%    \end{macrocode}
% See the note for the Italian localization module.
%    \begin{macrocode}
  andothers          = {{\mkibid{et\addabbrvspace al\adddot}}{\mkibid{et\addabbrvspace al\adddot}}},
  andmore            = {{\mkibid{et\addabbrvspace al\adddot}}{\mkibid{et\addabbrvspace al\adddot}}},
  }
%    \end{macrocode}
% \iffalse
%</english-lbx>
% \fi
% \subsection{\file{spanish-philosophy.lbx}}
% Thanks to Eduardo Villegas for these translations.
% \iffalse
%<*spanish-lbx>
% \fi
%    \begin{macrocode}
% 
% Thanks to Eduardo Villegas for these translations.
% 
\InheritBibliographyExtras{spanish}

\DeclareBibliographyExtras{%
\protected\def\bibrangedash{-\penalty\hyphenpenalty}%
\protected\def\bibdaterangesep{\bibrangedash}%
}%

\NewBibliographyString{opcited}

\DeclareBibliographyStrings{%
  inherit            = {spanish},
%    \end{macrocode}
% New string:
%    \begin{macrocode}     
  opcited            = {{\'{o}p\adddotspace cit\adddot}{\'{o}p\adddot cit\adddot}},
%    \end{macrocode}
% Redefined strings:
%    \begin{macrocode}    
  opcit              = {{cit\adddot}{cit\adddot}},
  ibidem             = {{ivi}{ivi}},
  loccit             = {{ibidem}{ibidem}},
  langspanish        = {{espa\~{n}ol}{es\adddot}}, 
  editor             = {{ed\adddot}{ed\adddot}},
  editors            = {{ed\adddot}{eds\adddot}},
  byreviser          = {{revisi\'{o}n de}{rev\adddotspace de}},
  reviewof           = {{rese\~{n}a de}{rese\~{n}a de}},%%FIXME
  backrefpage        = {{citado en la p\'{a}gina}{citado en la \bibsstring{page}\adddot}},%%FIXME
  backrefpages       = {{citado en las p\'{a}ginas}{citado en las p\'ags\adddot}},%%FIXME
  withcommentator    = {{comentarios de}{com\adddotspace de}},
  withannotator      = {{notas de}{notas de}},
  withintroduction   = {{introducci\'{o}n de}{intr\adddotspace de}},
  withforeword       = {{prefacio de}{pref\adddotspace de}},
  withafterword      = {{postfacio de}{postfacio de}},
  translationof      = {{traducci\'{o}n al espa\~{n}ol de}{trad\adddotspace de}},
  translationas      = {{traducido al espa\~{n}ol como}{trad\adddotspace es\adddot}},
  origpubas          = {{edici\'{o}n original}{ed\adddot orig\adddot}},  
  section            = {{secci\'{o}n}{sec\adddot}},
  sections           = {{secciones}{\iftoggle{bbx:classical}{secs\adddot}{sec\adddot}}}
%    \end{macrocode}
% Here we redefine only the |andmore| string because the |andothers| string is 
% a non Latin expression in \file{spanish.lbx} (``y col.'').
%    \begin{macrocode}
  andmore            = {{\mkibid{et\addabbrvspace al\adddot}}{\mkibid{et\addabbrvspace al\adddot}}},
}
%    \end{macrocode}
% \iffalse
%</spanish-lbx>
% \fi
%
% \subsection{\file{french-philosophy.lbx}}
% !EXEPRIMENTAL! This file needs a revision!
% \iffalse
%<*french-lbx>
% \fi
%    \begin{macrocode}
\InheritBibliographyExtras{french}

\DeclareBibliographyExtras{%
%    \end{macrocode}
% The \file{french.lbx} localization module redefines \cmd{mkbibnamefamily} in order to get the family name in small caps shape. We do not like this approach because an author could use a localization module without adhering to the typographical standards which should be indipendent from the linguistic standards. For this reason we prefer to reset it to the default definition.
%    \begin{macrocode}     
\protected\def\mkbibnamefamily#1{#1}%
\protected\def\bibrangedash{-\penalty\hyphenpenalty}%
\protected\def\bibdaterangesep{\bibrangedash}%
}%

\NewBibliographyString{opcited}

\DeclareBibliographyStrings{%
  inherit            = {french},
%    \end{macrocode}
% New string:
%    \begin{macrocode}     
  opcited             = {{op\adddotspace cit\adddot}{op\adddotspace cit\adddot}},
%    \end{macrocode}
% Redefined strings:
%    \begin{macrocode}     
  opcit              = {{cit\adddot}{cit\adddot}},%%FIXME
  ibidem             = {{ibid\adddot}{ibid\adddot}},%%FIXME
  loccit             = {{ibid\adddot}{ibid\adddot}},%%FIXME
  translationas      = {{trad\adddot}{trad\adddot}},%%FIXME
  withcommentator    = {{commentaires \smartof}{comment\adddotspace\smartof}},%%FIXME
  withannotator      = {{annotations \smartof}{annot\adddotspace\smartof}},%%FIXME
  withintroduction   = {{introduction \smartof}{introd\adddotspace\smartof}},%%FIXME
  withforeword       = {{pr\'eface \smartof}{pr\'ef\adddotspace\smartof}},%%FIXME
  withafterword      = {{postface \smartof}{postf\adddotspace\smartof}}%%FIXME
%    \end{macrocode}
% See the note for the Italian localization module.
%    \begin{macrocode}
  andothers          = {{\mkibid{et\addabbrvspace al\adddot}}{\mkibid{et\addabbrvspace al\adddot}}},
  andmore            = {{\mkibid{et\addabbrvspace al\adddot}}{\mkibid{et\addabbrvspace al\adddot}}},
  }
%    \end{macrocode}
% \iffalse
%</french-lbx>
% \fi
%
% \iffalse
%<*bib>

@INCOLLECTION{Termini:2007,
  author = {Settimo Termini},
  title = {Vita morte e miracoli di Alan Mathison Turing},
  crossref = {Bartocci:2007},
  date-added = {2009-10-22 14:36:21 +0200},
  date-modified = {2013-03-15 16:53:58 +0100},
  keywords= {esempio},
  hyphenation = {italian},
  annotation = {This entry includes all the informations of the parent \texttt{collection} linked through the \texttt{crossref} field}
}

@BOOK{aristotle:ethics,
  title = {Nichomachean Ethics},
  publisher = {J. Hall \& Son},
  author = {Aristotle},
  annotation = {A \texttt{book} entry with a \texttt{translator} field. Note also the \texttt{entrysubtype} field which is set to \texttt{classic}},
  booktitle = {Nichomachean Ethics},
  date = {1872},
  entrysubtype = {classic},
  hyphenation = {american},
  keywords = {esempio},
  location = {London},
  shorttitle = {Nich\adddotspace Ethics},
  translator = {F. A. Paley, F. M.}
}

@ONLINE{clawson:mla,
  annote = {\texttt{Biblatex-mla} provides support to Biblatex , BibTeX, and LaTeX for citations and Works. Cited lists in the style established by the Modern Language Association (MLA). For commands and options to change package defaults, see § 3.1 and § 3.2, respectively, below. MLA style, a common standard for writers in the humanities, is outlined in the MLA
    \emph{Style Manual}, in its 3\textsubscript{rd} edition, and the \emph{MLA Handbook for Writers of Research Papers}, now
    in its 8\textsubscript{th} edition. \texttt{Biblatex-mla} follows the style outlined in the latter of these. It also
    follows the logic of the MLA when citing similar material repeatedly, trimming unnecessary
    information from citations where necessary. \texttt{Biblatex-mla} is compatible with \texttt{Biblatex}’s
    support for \texttt{hyperref} and \texttt{tex4ht} , and the main word in each citation (either the author’s
    name, the title, or the page number) serves as a link to the particular entry in the Works
    Cited.},
  author = {James Clawson},
  date = {2010},
  date-added = {2012-04-18 11:51:47 +0200},
  date-modified = {2013-03-13 22:02:08 +0100},
  keywords = {primaria},
  subtitle = {MLA Style Using biblatex},
  title = {\texttt{biblatex-mla}},
  url = {http://mirrors.ctan.org/macros/latex/contrib/biblatex-contrib/biblatex-mla/doc/biblatex-mla.pdf},
  version = {0.95}
}

@ONLINE{glibof:historian,
  annote = {The files \texttt{historian.bbx}, \texttt{historian.cbx}, and \texttt{historian.lbx} im-
    plement a bibliography and citation style for use with Philipp Lehman’s
    \texttt{biblatex} package. Historian follows the conventions of \emph{The Chicago
    Manual of Style}, as presented in Turabian’s \emph{Manual for Writers}. The
    style is designed for use by historians who need to generate detailed
    footnotes not only for ordinary books and articles, but also reprint edi-
    tions, correspondence, archives and archival documents, online sources,
    book reviews, unpublished manuscripts, and conference presentations.},
  author = {Sander Gliboff},
  date = {2010},
  date-added = {2012-04-18 11:49:37 +0200},
  date-modified = {2013-03-13 23:38:14 +0100},
  hyphenation = {english},
  keywords = {primaria},
  subtitle = {A Footnotes-and-Bibliography Style, Following Turabian/Chicago Guidelines:
  For Use with the Biblatex System of Programmable Bibliographies and
  Citations},
  title = {User's Guide to \texttt{Historian}},
  url = {http://mirrors.ctan.org/macros/latex/contrib/biblatex-contrib/biblatex-historian/historian.pdf},
  version = {0.4}
}

@BOOK{guzman:sd,
  title = {Problemática logico-lingüística de la comunicacíon social con el
  pueblo Aymara},
  year = {s.d.},
  author = {Guzmán de Rojas, Iván},
  note = {mimeo},
  addendum = {Con los auspicios del Centro internacional de Investigacíones para
  el Desarrollo de Canada},
  annote = {A work without a pubblication date. The string \texttt{nodate} (localized)
  is automatically printed},
  date-added = {2012-04-17 23:48:05 +0200},
  date-modified = {2013-03-15 16:40:38 +0100},
  keywords = {esempio},
  shorttitle = {Problemática}
}

@BOOK{heidegger:sz,
  title = {Sein und Zeit},
  publisher = {Max Niemeyer Verlag},
  author = {Martin Heidegger},
  edition = {18},
  annote = {Note the \texttt{library} field, use for some details about the first
  edition},
  booktitle = {Sein und Zeit},
  date = {2001},
  date-added = {2012-04-15 18:03:59 +0200},
  date-modified = {2013-03-15 16:41:30 +0100},
  keywords = {esempio,volumesingolo},
  library = {Originally published in 1927 on the \emph{Jahrbuch für Philosophie 
  und phänomenologische Forschung (vol. VIII)}, directed by H. Husserl},
  location = {Tübingen}
}

@BOOKINBOOK{kant:kpv:xref,
  author = {Kant, Immanuel},
  date = {1968},
  date-added = {2013-02-25 22:44:19 +0100},
  date-modified = {2013-03-15 19:27:33 +0100},
  keywords = {esempio},
  pages = {1-163},
  shorthand = {KpV},
  shorttitle = {Kritik der praktischen Vernunft},
  title = {Kritik der praktischen Vernunft},
  volume = {5},
  xref = {kant:werke},
  annotation = {A single volume from the critical edition of Kant's (\emph{Kants Werke}). Note the \texttt{xref} field.}
}

@BOOKINBOOK{kant:ku:xref,
  author = {Kant, Immanuel},
  date = {1968},
  date-added = {2013-02-28 09:42:34 +0100},
  date-modified = {2013-03-15 19:27:38 +0100},
  hyphenation = {german},
  keywords = {esempio},
  pages = {165-485},
  title = {Kritik der Urtheilskraft},
  volume = {5},
  xref = {kant:werke},
  annotation = {A single volume from the critical edition of Kant's (\emph{Kants Werke}). Note the \texttt{xref} field.}
}

@mvbook{kant:werke,
  author = {Kant, Immanuel},
  booktitle = {Kants Werke. Akademie Textausgabe},
  date = {1968},
  date-added = {2010-03-06 00:26:39 +0100},
  date-modified = {2013-03-15 14:01:45 +0100},
  hyphenation = {german},
  keywords = {esempio},
  location = {Berlin},
  maintitle = {Kants Werke. Akademie Textausgabe},
  publisher = {Walter de Gruyter},
  shorthand = {KW},
  title = {Kants Werke. Akademie Textausgabe},
  volumes = {9}
}

@mvbook{comenio:oo,
  annote = {This author is known with his Latin name, given in the \texttt{nameaddon}
  field},
  author = {Jan Amos Komensky},
  booktitle = {Opera Omnia},
  date = {1969},
  date-added = {2012-04-19 20:20:03 +0200},
  date-modified = {2013-03-15 16:32:24 +0100},
  keywords = {esempio},
  location = {Praga},
  nameaddon = {Comenius},
  title = {Opera Omnia}
}

@ONLINE{kime:biber,
  annotation = {Biber is conceptually a Bib\TeX replacement for Biblatex. It is written in Perl
    with the aim of providing a customised and sophisticated data preparation backend
    for Biblatex. You do not need to install Perl to use Biber—binaries are provided
    for many operating systems via the main \TeX distributions (\TeX Live, Mac\TeX,
    MiK\TeX) and also via download from SourceForge. Functionally, Biber offers a
    superset of Bib\TeX’s capabilities but is tightly coupled with Biblatex and cannot
    be used as a stand-alone tool with standard .bst styles. Biber’s primary role is to
    support Biblatex by performing the following tasks:
    Parsing data from datasources; 
    Processing cross-references, entry sets, related entries; 
    Generating data for name, name list and name/year disambiguation; 
    Structural validation according to Biblatex data model; 
    Sorting reference lists; 
    Outputting data to a .bbl for Biblatex to consume},
  author = {Philip Kime and François Charette},
  title = {\texttt{biber}},
  subtitle = {A backend bibliography processor for biblatex},
  date = {2016-05-12},
  hyphenation = {english},
  keywords = {primaria},
  url = {http://biblatex-biber.sourceforge.net},
  version = {2.7}
}

@ONLINE{lehman:biblatex,
  annote = {This package provides advanced bibliographic facilities for use with LaTeX. The
    package is a complete reimplementation of the bibliographic facilities provided by
    LaTeX. The biblatex package works with the “backend” (program) biber, which
    is used to process BibTeX format data files and them performs all sorting, label
    generation (and a great deal more). Formatting of the bibliography is entirely con-
    trolled by TeX macros. Good working knowledge in LaTeX should be sufficient to
    design new bibliography and citation styles. This package also supports subdivided
    bibliographies, multiple bibliographies within one document, and separate lists of
    bibliographic information such as abbreviations of various fields. Bibliographies may
    be subdivided into parts and/or segmented by topics. Just like the bibliography styles,
    all citation commands may be freely defined. Features such as full Unicode support
    for bibliography data, customisable sorting, multiple bibliographies with different
    sorting, customisable labels and dynamic data modification are available},
  author = {Philipp Lehman},
  nameaddon = {with Philip Kime, Audrey Boruvka and Joseph Wright},
  date = {2016-11-16},
  hyphenation = {english},
  keywords = {primaria},
  subtitle = {Programmable Bibliographies and Citations},
  title = {The \texttt{biblatex} Package},
  url = {http://mirrors.ctan.org/macros/latex/contrib/biblatex/doc/biblatex.pdf},
  version = {3.7}
}

@ONLINE{babel,
  annote = {This manual describes babel, a package that makes
    use of the capabilities of TEX version 3 and, to some
    extent, xetex and luatex, to provide an environment
    in which documents can be typeset in a language
    other than US English, or in more than one language
    or script. However, no attempt has been done to take full
    advantage of the features provided by the latter,
    which would require a completely new core (as for
    example polyglossia or as part of \LaTeX 3)},
  author = {Johannes L. Braams},
  date = {2016-04-23},
  hyphenation = {english},
  keywords = {primaria},
  title = {Babel},
  url = {http://mirrors.ctan.org/macros/latex/contrib/babel/babel.pdf},
  version = {3.9r}
}

@ONLINE{polyglossia,
  annote = {Polyglossia is a package for facilitating multilingual typesetting with Xe\LaTeX{} and
    (at an early stage) Lua\LaTeX. Basically, it can be used as an alternative to babel
    for performing the following tasks automatically:
    1. Loading the appropriate hyphenation patterns.
    2. Setting the script and language tags of the current font (if possible and
    available), via the package fontspec.
    3. Switching to a font assigned by the user to a particular script or language.
    4. Adjusting some typographical conventions according to the current lan-
    guage (such as afterindent, frenchindent, spaces before or after punctu-
    ation marks, etc.).
    5. Redefining all document strings (like “chapter”, “figure”, “bibliography”).
    6. Adapting the formatting of dates (for non-Gregorian calendars via external
    packages bundled with polyglossia: currently the Hebrew, Islamic and
    Farsi calendars are supported).
    7. For languages that have their own numbering system, modifying the
    formatting of numbers appropriately (this also includes redefining the al-
    phabetic sequence for non-Latin alphabets).
    8. Ensuring proper directionality if the document contains languages that are
    written from right to left (via the package bidi, available separately).},
  author = {François Charette},
  date = {2015-03-25},
  hyphenation = {english},
  keywords = {primaria},
  title = {Polyglossia: An Alternative to Babel for Xe\LaTeX{} and Lua\LaTeX},
  url = {http://mirrors.ctan.org/macros/latex/contrib/babel/babel.pdf},
  version = {1.42.4}
}

@ONLINE{csquotes,
  annote = {This package provides advanced facilities for inline and display quotations. It is
    designed for a wide range of tasks ranging from the most simple applications to
    the more complex demands of formal quotations. The facilities include commands,
    environments, and user-definable ‘smart quotes’ which dynamically adjust to their
    context. Quotation marks are switched automatically if quotations are nested and
    can adjust to the current language. There are additional features designed to cope
    with the more specific demands of academic writing. All quote styles as well as the
    optional active quotes are freely configurable},
  author = {Philipp Lehman and Joseph Wright},
  date = {2017-02-03},
  hyphenation = {english},
  keywords = {primaria},
  subtitle = {Context Sensitive Quotation Facilities},
  title = {The \texttt{csquotes} Package},
  url = {http://mirrors.ctan.org/macros/latex/contrib/csquotes/csquotes.pdf},
  version = {5.2a}
}

@BOOK{pantieri:artelatex,
  title = {L'arte di scrivere con \LaTeX},
  author = {Lorenzo Pantieri and Tommaso Gordini},
  annote = {Lo scopo di questo lavoro, rivolto sia a chi muove i primi passiin \LaTeX{} sia a quanti già lo conoscono, è di offrire ai suoi utenti italiani le conoscenze essenziali per poterlo usare con successo. I concetti fondamentali della materia, raccolti da svariati manuali, vengono presentati nel modo più chiaro e organico possibile; nel contempo si fornisce un vasto campionario di esempi e si analizzano alcuni tipici problemi che potrebbero presentarsi nella redazione di una pubblicazione scientifica o professionale in italiano, indicando per ciascuno le soluzioni per noi migliori.},
  booktitle = {L'arte di scrivere con \LaTeX},
  date = {2011},
  date-added = {2010-03-06 00:26:39 +0100},
  date-modified = {2013-03-13 22:04:30 +0100},
  foreword = {Enrico Gregorio},
  hyphenation = {italian},
  keywords = {primaria},
  subtitle = {Un'introduzione a \LaTeX},
  url = {http://www.lorenzopantieri.net/LaTeX_files/ArteLaTeX.pdf}
}

@BOOK{Poincare:1968-ITA,
  title = {La scienza e l'ipotesi},
  publisher = {Bompiani},
  author = {Jules-Henri Poincaré},
  editor = {Corrado Sinigaglia},
  note = {testo greco a fronte},
  booktitle = {La science et l'hypothèse},
  date = {2003},
  date-modified = {2013-03-15 19:07:03 +0100},
  hyphenation = {italian},
  keywords = {esempio},
  location = {Milano}
}

@BOOK{Poincare:1968-ORIG,
  title = {La science et l'hypothèse},
  publisher = {Flammarion},
  author = {Jules-Henri Poincaré},
  annote = {A book entry followed by its translation, cross-referenced in the
  \texttt{related} field},
  booktitle = {La science et l'hypothèse},
  date = {1968},
  date-added = {2010-03-05 16:18:11 +0100},
  date-modified = {2013-03-15 19:07:49 +0100},
  keywords = {esempio},
  location = {Paris},
  related = {Poincare:1968-ITA}
}

@BOOK{popper-logik,
  title = {Logik der Forschung},
  publisher = {Springer},
  author = {Karl R. Popper},
  annote = {A book entry followed by two differents translations, cross-referenced
  in the \texttt{related} (biber 1.6 required)},
  booktitle = {Logik der Forschung},
  date = {1934},
  date-added = {2013-03-01 17:50:26 +0100},
  date-modified = {2013-03-15 16:46:51 +0100},
  keywords = {esempio},
  location = {Wien},
  related = {popper-logik:ing},
  timestamp = {2012.04.25}
}

@BOOK{popper-logik:ing,
  title = {The Logic of Scientific Discovery},
  publisher = {Hutchinson},
  author = {Karl R. Popper},
  edition = {3},
  booktitle = {The Logic of Scientific Discovery},
  date = {1959},
  date-added = {2013-03-01 17:55:40 +0100},
  date-modified = {2013-03-15 16:48:11 +0100},
  keywords = {esempio,popper},
  location = {London},
  related = {popper-logik:ita},
  relatedstring={it. trans.},
  timestamp = {2012.04.25}
}

@BOOK{popper-logik:ita,
  title = {Logica della scoperta scientifica},
  publisher = {Einaudi},
  author = {Karl R. Popper},
  edition = {3},
  date = {1998},
  date-added = {2013-03-01 17:50:46 +0100},
  date-modified = {2013-03-15 16:48:21 +0100},
  hyphenation = {italian},
  keywords = {esempio,popper},
  location = {Torino},
  timestamp = {2012.04.25}
}

@INCOLLECTION{Valbusa:2007,
  author = {Ivan Valbusa},
  title = {Psicologia e sistema in Alsted e in Wolff},
  booktitle = {Christian Wolff tra psicologia empirica e psicologia razionale},
  publisher = {Georg Olms Verlag},
  editor = {Ferdinando Luigi Marcolungo},
  date = {2007},
  date-added = {2009-09-27 23:17:07 +0200},
  date-modified = {2009-10-13 13:16:37 +0200},
  hyphenation = {italian},
  keywords = {esempio},
  location = {Hildesheim and Zürich and London}
}

@ONLINE{wassenhoven:dw,
  annote = {A small collection of styles for the biblatex package.
  It was designed for citations in the Humanities and offers some features 
  that are not provided by the standard biblatex styles. biblatex-dw 
  is dependent on biblatex – version 1.7 needs at least version 3.3
  of biblatex and was tested with biblatex version 3.6 and biber version 2.6.},
  author = {Dominik Waßenhoven},
  date = {2011},
  date-added = {2013-03-13 21:58:04 +0100},
  date-modified = {2013-03-13 22:05:36 +0100},
  keywords = {primaria},
  title = {\texttt{biblatex-dw}},
  url = {http://mirrors.ctan.org/macros/latex/contrib/biblatex-contrib/biblatex-dw/doc/biblatex-dw.pdf},
  version = {1.4}
}

@COLLECTION{Bartocci:2007,
  annote = {A collection with four authors},
  booksubtitle = {Protagonisti del '900 da Hilbert a Wiles},
  booktitle = {Vite matematiche},
  date = {2007},
  editor = {Claudio Bartocci and Renato Betti and Angelo Guerraggio and Roberto
  Lucchetti},
  hyphenation = {italian},
  location = {Milano},
  publisher = {Springer-Verlag Italia},
  title = {Vite matematiche}
}


@COLLECTION{Filmed:2009,
  annote = {A collection with four editors. The list is automatically truncated
  in the citations},
  booktitle = {Filosofia delle medicina},
  date = {2009},
  date-added = {2009-09-27 23:05:08 +0200},
  date-modified = {2013-03-15 16:37:44 +0100},
  editor = {Pierdaniele Giaretta and Antonio Moretto and Gian Franco Gensini
  and Marco Trabucchi},
  hyphenation = {italian},
  keywords = {esempio},
  location = {Bologna},
  publisher = {il Mulino},
  subtitle = {Metodo, modelli, cura ed errori},
  title = {Filosofia delle medicina},
  volumes = {2}
}

@INCOLLECTION{corrocher:2009,
  author = {Roberto Corrocher},
  title = {Riflessioni sull'uomo di fronte a nuove sfide},
  pages = {27-42},
  annote = {An \texttt{@incollection} entry. The \texttt{@collection} is automatically
    printed in the bibliography because another \texttt{@incollection}
    has been cited},
  crossref = {Filmed:2009},
  date-added = {2010-03-06 00:26:39 +0100},
  date-modified = {2013-03-15 16:33:22 +0100},
  hyphenation = {italian},
  keywords = {esempio},
  read = {0}
}

@INCOLLECTION{federspil:2009,
  author = {Giovanni Federspil and Roberto Vettor},
  title = {Medicina: un unico metodo e una sola argomentazione?},
  pages = {43-74},
  annote = {An \texttt{@incollection} entry. The \texttt{@collection} is automatically
    printed in the bibliography because another \texttt{@incollection}
    has been cited},
  crossref = {Filmed:2009},
  date-added = {2010-03-06 00:26:39 +0100},
  date-modified = {2013-03-15 16:37:35 +0100},
  hyphenation = {italian},
  keywords = {esempio}
}

@ONLINE{ctan,
  bdsk-url-1 = {http://www.ctan.org},
  date = {2006},
  date-added = {2011-06-02 17:33:32 +0200},
  date-modified = {2013-03-15 16:35:26 +0100},
  hyphenation = {american},
  keywords = {web},
  label = {CTAN},
  subtitle = {The Comprehensive TeX Archive Network},
  title = {CTAN},
  url = {http://www.ctan.org},
  urldate = {2006-10-01}
}

@ONLINE{guit:sito,
  bdsk-url-1 = {http://www.guitex.org},
  date = {2012},
  date-added = {2012-04-21 16:52:51 +0200},
  date-modified = {2013-03-15 16:40:18 +0100},
  hyphenation = {italian},
  keywords = {web},
  title = {GuIT. Gruppo degli Utilizzatori Italiani di \TeX},
  url = {http://www.guitex.org},
  urldate = {2012-01-15}
}
%</bib>
% \fi
% \clearpage
% \Finale
