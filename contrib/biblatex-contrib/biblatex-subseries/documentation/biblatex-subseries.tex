\documentclass{ltxdockit}[2011/03/25]
\usepackage{btxdockit}
\usepackage{fontspec}
\usepackage[mono=false]{libertine}
\usepackage{microtype}
\usepackage[american]{babel}
\usepackage[strict]{csquotes}
\setmonofont[Scale=MatchLowercase]{DejaVu Sans Mono}
\usepackage{shortvrb}
\usepackage{pifont}
\usepackage{minted}
% Usefull commands
\newcommand{\biblatex}{biblatex\xspace}
\pretocmd{\bibfield}{\sloppy}{}{}
\pretocmd{\bibtype}{\sloppy}{}{}
\newcommand{\namebibstyle}[1]{\texttt{#1}}
\newrobustcmd*{\LF}{%
  \textcolor{spot}{\margnotefont Label field}}
\newrobustcmd*{\LFMark}{%
  \leavevmode\marginpar{\LF}}
% Meta-datas
\titlepage{%
	title={Description of subseries with biblatex},
	subtitle={New data fields},
	email={maieul <at> maieul <dot> net},
	author={Maïeul Rouquette},
	revision={1.2.0},
	date={24/02/2017},
	url={https://git.framasoft.org/maieul/biblatex-subseries}}

% biblatex
\usepackage[bibstyle=subseries,citestyle=verbose]{biblatex}
\addbibresource{biblatex-subseries-example.bib}

\begin{document}

\printtitlepage
\tableofcontents
\section{Introduction}

Some publishers organize book series with subseries. In this case, two numbers are associated with one volume: the number inside the series and the number inside the subseries. That is the case, for example, of the series \emph{\foreignlanguage{latin}{Corpus Scriptorium Christianorum Orientalium}} published by Peeters.\footnote{\url{http://www.peeters-leuven.be/search_serie_book.asp?nr=94}.}
Sometime, there is also subsubseries, and number for subsubseries.

This package provides new fields to manage such system.
\section{Basic use}

\subsection{The .bib file}
You can use six additional fields:
\begin{fieldlist}

\fielditem{shortsubseries}{literal\LFMark} The shorthand form of the subseries. Not used by default styles.
\fielditem{subnumber}{literal} The number of the book inside the subseries.
\fielditem{subseries}{literal} The subseries.

\fielditem{shortsubsubseries}{literal\LFMark} The shorthand form of the subsubseries. Not used by default styles.
\fielditem{subsubnumber}{literal} The number of the book inside the subsubseries.
\fielditem{subsubseries}{literal} The subsubseries.

\end{fieldlist}


\inputminted{tex}{biblatex-subseries-example.bib}


\subsection{Loading of the \biblatex package}

When loading the \biblatex package, use the \namebibstyle{subseries}  bibstyle, which is based on the \namebibstyle{verbose} bibstyle:

\begin{minted}{latex}
\usepackage[citestyle=verbose,bibstyle=subseries]{biblatex}
\end{minted}

\subsection{Example of result}

Here, an example of result with the default setting:
\begin{quotation}
\cite{CSCO545}
\end{quotation}

\section{Customization}
You can redefine \cmd{subseriesseparator} to change the separator between series and subseries. You can also redefine \cmd{subsubseriesseparator} to change the separator between subseries and subsubseries. By default, it is the same as \cmd{subseriesseparator}.

If you want to customize the \verb+series+number+ bibmacro, do not forget to call the \verb|subseries+subnumber| bibmacro inside. 

You can also redefine this macro, and redefine the \verb+subsubseries+subsubnumber+ bibmacro which is called by the \verb|subseries+subnumber| bibmacro.


\section{Use with other packages which provide new fields}

To use this package with the features of others which provide new fields, like \verb+biblatex-manuscripts-philology+, you must use the package \verb+biblatex-multiple-dm+.

\section{Credits}

This package was created for Maïeul Rouquette's phd dissertation\footnote{\url{http://apocryphes.hypothese.org}.} in 2015. It is licensed on the \emph{\LaTeX\ Project Public License}\footnote{\url{http://latex-project.org/lppl/lppl-1-3c.html}.}.


All issues can be submitted, in French or English, in the Framagit issues page.\footnote{\url{https://git.framasoft.org/maieul/biblatex-subseries/issues}.}

\section{Change history}

\begin{changelog}

\begin{release}{1.2.0}{2017-02-24}
  \item Add \bibfield{subsubseries}, \bibfield{subsubnumber} and \bibfield{shortsubsubseries} fields.
\end{release}

\begin{release}{1.1.0}{2016-06-07}
  \item Add error message to know more quickly break compatibility with new releases of biblatex.
\end{release}

\begin{release}{1.0.0a}{2015-05-14}
\item Fix typo in handbook.
\end{release}

\begin{release}{1.0.0}{2015-05-12}
\item First public release.
\end{release}
\end{changelog}
\end{document}
