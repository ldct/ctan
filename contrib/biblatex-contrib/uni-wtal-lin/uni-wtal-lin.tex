%!TEX TS-program = xelatex
%!TEX encoding = UTF-8 Unicode
\documentclass[12pt,notitlepage,parskip]{scrartcl}
\usepackage{xltxtra}
\usepackage{fontspec}
\usepackage{xunicode}
\setmainfont{Linux Libertine}
\usepackage[german]{babel}
\usepackage[babel]{csquotes}
\usepackage[
  backend=biber,
  style=uni-wtal-lin,
  %edsuper=false
  ]{biblatex}
 
\bibliography{uni-wtal-lin.bib}
\defbibheading{style}{\section*{Quellen für die Style-Regeln}}
\defbibheading{lin}{\section*{Beispielliteratur der Germanistikbroschüre}}
\defbibheading{weitere}{\section*{Weitere Beispielliteratur}}


\begin{document}
Dies ist eine Beispiel- und Dokumentations-Datei für den bib\LaTeX-Zitierstil
\texttt{uni-wtal-lin}.

\textit{The following information is mainly interesting for students and docents
of the} Bergische Universität Wuppertal \textit{or for people who are
interested in how the style works. All others: just have fun using the style.
:)}

\section{uni-wtal-lin}
Auch wenn ich diesem Stil den sehr spezifischen Namen \texttt{uni-wtal-lin}
gegeben habe – eben weil ich ihn speziell für die Anwendung in der
germanistischen Linguistik der Uni Wuppertal geschrieben habe –, so
ist er durchaus für viele andere Zitierbedürfnisse geeignet.

Dieses PDF-Dokument soll einerseits – zusammen mit der \texttt{tex}- und der
\texttt{bib}-Datei – beispielhaft zeigen, wie die entsprechenden Daten in die
bib\LaTeX-Felder eingegeben werden müssen, um die gewünschte Ausgabe zu
erhalten; für diesen Vergleich kann der Leser einfach die Ausgabe im
Literaturverzeichnis auf den letzten Seiten dieses Dokuments mit der
\texttt{bib}-Datei vergleichen.\\
Andererseits soll dieses Dokument jedoch auch dazu dienen, zu beschreiben,
welche Quellenart welchen Regeln unterliegt. Aus diesem Grunde habe ich dies auf
den folgenden Seiten detailliert ausgeführt.\\
Zur reinen Anwendung des Zitierstils sind die folgenden Informationen jedoch
nicht wichtig.\footnote{Der Hintergrund dieser detaillierten Aufstellung ist vor
allem, den Studierenden und Dozierenden der BUW einen Überblick über die
angewandten Regeln dieses Zitierstils zu verschaffen und meine Entscheidungen
bei Zweifelsfällen zu begründen.}

\subsection{Umsetzung}
Wie auch \texttt{uni-wtal-ger}\footnote{Mein literaturwissenschaftlicher
Zitierstil für die Germanistik in Wuppertal, zu finden in CTAN sowie unter:
\url{http://www.dahlmann.net/?Informatives/LaTeX}} ist dieser Zitierstil
angelehnt an die Germanistikbroschüre der Bergischen Universität Wuppertal.
\parencite[Vgl.][43–45]{germanistik}\footnote{Ab der 5. aktual. Aufl. gibt es
einige Änderungen bei den Bibliographieregeln. Diese Änderungen sind hier
berücksichtigt und in der Changelog genauer beschrieben. Achtung:
Diese Version des Stils unterscheidet sich somit erheblich von der
Vorgängerversion 0.1. Falls Sie diese verwendet haben und weiterhin verwenden
möchten, so kann man sie auf meiner o.a. Homepage downloaden und manuell
einbinden, indem man die
\texttt{bbx}- und die \texttt{cbx}-Datei ins Projektverzeichnis kopiert.}

Diese Version des Zitierstils folgt also möglichst genau der 5. aktualisierten
Auflage von 2012\footnote{Zwar ist es natürlich vor allem wichtig, eine
angebrachte Zitierweise konsequent zu nutzen, und die 5. Auflage betont nun auch
konkret, dass ein einheitliches System wichtig sei und bietet zum Teil auch
Alternativvorschläge (wodurch die Broschüre im Vergleich zu den vorherigen
Ausgaben – zumindest im linguistischen Teil – nicht mehr als absolute
Pflichtvorgabe erscheint) – jedoch ist dieser Zitierstil für eben jene Anwendung
geschaffen: für möglichst genaue Literaturangaben gemäß der Germanistikbroschüre
der BUW, wenn man diese benötigt. Bei Alternativmöglichkeiten und Zweifelsfällen
habe ich diesen Stil nach meinem Ermessen konfiguriert. Dazu mehr weiter unten.}
und wurde von Herrn Prof. Dr.
Horst Lohnstein, der ab dieser Auflage für die Broschüre inhaltlich
mitverantwortlich ist, durchgesehen und überprüft – wofür ich sehr dankbar bin.

Der Stil basiert technisch auf dem bib\LaTeX-Stil \texttt{authoryear}, wobei
alle Zitate im Text der Harvard-Kurzzitierweise entsprechen, angelehnt an die
Vorschläge der Broschüre.
Das Literaturverzeichnis wird, soweit interpretierbar, angelehnt an die
Beispiele der Broschüre erzeugt. Für die optionale Erzeugung einer
hochgestellten Auflagennummer, die ab der 5. aktualisierten Auflage der
Broschüre ab 2012 empfohlen wird, habe ich mir die Funktion der
\texttt{edsuper}-Option zu Nutzen gemacht, die Dominik Waßenhovens Stil
\texttt{authortitle-dw} nutzt.\footnote{Zu dieser Option, die standardmäßig
aktiviert ist, vgl. Kapitel 1.2.3 dieser Übersicht.}

Wie weiter oben schon angedeutet, sind einige Vorgaben der Broschüre leider (für
die Umsetzung mit \LaTeX) zu unscharf oder nicht in einem Beispiel oder anhand einer
konkreten Regel aufgeführt.
In diesen Fällen orientiert sich der Stil an den Empfehlungen des
\textit{Unified style sheet for linguistics}. Bei manchen Zweifelsfällen habe
ich darüber hinaus nach eigenem Ermessen entschieden.
Dadurch ergibt sich folgende Dominanzfolge in der Priorität der Regeln:

(Eigenes Ermessen)\footnote{Nur in Notfällen, um die Einheitlichkeit zu
bewahren.} \textgreater\ Germanistik-Broschüre BUW \textgreater\ 
\textit{Unified style sheet for linguistics}.

Somit sollte mit \texttt{uni-wtal-lin} ein vollständig mit den Richtlinien der
germanistischen Linguistik der Uni Wuppertal kompatibler Zitierstil vorliegen.

Im Folgenden möchte ich nun kurz die oben genannten Zweifelsfälle thematisieren
und dabei zeigen, wie ich mich jeweils bei der Umsetzung entschieden
habe.\footnote{Aufgrund der Tatsache, dass die Beispiele der
Germanistik-Broschüre jedoch nicht alle Fälle abdecken, habe ich weitere Quellen
hinzugefügt, die unter anderem dem literaturwissenschaftlichen Teil der
Broschüre entnommen sind. Hiermit sollte der Wuppertaler gleichzeitig einen
übersichtlichen Vergleich zwischen den beiden Zitierstilen erhalten. Anhand
der Daten aus der Literaturwissenschaft kann man außerdem sehen, was passiert, wenn
kein \texttt{publisher} vorhanden ist, den die Literaturwissenschaft ja nicht
nutzt. (Das Feld \texttt{shorttitle} wurde bei den Daten jedoch entfernt, da
es mit diesem Stil nicht kompatibel ist.)}

\subsection{Zweifelsfälle und Alternativen der Broschüre}
\subsubsection{Hgg. vs. eds.}
Die Beispiele der Broschüre enthalten sowohl \texttt{(Hgg.)} als auch
\texttt{(eds.)} Dies könnte man so interpretieren, dass deutsche Herausgeber mit
\texttt{(Hgg.)} und englische mut \texttt{(eds.)} bezeichnet werden sollen.
Dies ist mithilfe von bib\LaTeX\ jedoch kaum umsetzbar. Der Einheitlichkeit
halber wird daher \texttt{(Hgg.)} erzeugt. Ich habe mich hier für die deutsche
Variante entschieden, da ich davon ausgehe, dass die meisten Studierenden der
BUW ihre Hausarbeiten auf Deutsch verfassen werden. Dies kann jedoch einfach
geändert werden, indem die Sprache in der Präambel auf \texttt{english} gesetzt
wird. Der Zitierstil deckt also beide Sprachen ab.

\subsubsection{Aufl. vs. edn.}
Sprachzweifel gibt es meines Erachtens auch hinsichtlich der Auflage.
Das einzige Beispiel war bis zur 4. Auflage „second revised edition 2002“ bei
\textcite{schatz2}.\footnote{In der 5. Auflage fehlt die Auflage bei
Beibehaltung dieses Beispiels.}
Ich gehe davon aus, dass dies an dieser Stelle auf Englisch formuliert wurde, da
auch die Quelle englischsprachig ist. Das Wort für „Auflage“ orientiert sich
ebenfalls an der eingestellten Sprache, sofern im entsprechenden Feld nur ein
Zahlenwert angegeben wird. Ansonsten kann der Text in \texttt{edition} natürlich
frei editiert werden – was ab dieser Version auch die Position und die Art der
Auflagenwiedergabe beeinflusst. Dazu mehr in Kapitel 1.2.3.

\subsubsection{Position der Auflage}
Die Position der Auflage wird – sofern das Feld \texttt{edition} anstatt/neben
einer Zahl auch Text enthält – mit einem Komma als vorangehendem Delimiter nach
Verlagsort und Verlag erzeugt.\footnote{Das oben erwähnte Beispiel der 4.
Auflage ist in der Broschüre quasi der einzige Anhaltspunkt zu der Frage, wo
eine längere Auflagenbeschreibung – also z.B. „11., korr. und aktual. Aufl.“
erscheinen sollte. Es befindet sich dort ganz am Ende der Literaturangabe.
Anderen Prioritäten und Gepflogenheiten zufolge müssen danach jedoch noch Daten
wie z.B. die Seitenzahl oder die Reihe folgen.} Enthält das Feld
\texttt{edition} jedoch nur eine Zahl, so wird die Auflage ab dieser Version
hochgestellt nach der Jahreszahl am Anfang der Literaturangabe erzeugt.\\
Diese Zweiteilung der Auflagenposition ist meines Erachtens notwendig, da man
ansonsten nie erweiterte – und oft durchaus wichtige – Auflagenbezeichnungen
nutzen könnte.

Möchte man auf das Hochstellen der Auflage verzichten, so kann man dies
einfach in der Präamel mithilfe der Option \texttt{edsuper=false} tun. In diesem
Fall wird die Auflage grundsätzlich am Ende der Literaturangabe erzeugt.

Eine hochgestellte Auflage enthält die Literaturangabe \textcite{greule}. Viele
andere, wie z.B. \textcite{nt} oder \textcite{sterne1}, enthalten nachgestellte
Auflagen.

\subsubsection{Heftnummer}
Die Heftnummer (\texttt{number}) einer Zeitschrift wird ab dieser Version mit
einem Doppelpunkt vom Jahrgang (\texttt{volume}) getrennt \parencite[vgl. die
Literaturangabe][]{unselbst-zeitsch2}.

\subsubsection{Zeitungen}
Aufgrund von fehlenden Informationen werden Zeitungen (die bei der
Literaturwissenschaft ja einzeln aufgeführt sind) genauso wie Artikel in
Zeitschriften behandelt \parencite[vgl.][]{unselbst-zeitung}.

\subsubsection{note}
Einträge wie „Unpublished Ph.D. dissertation“ bei \textcite{swallow2} oder
„Euphoria, California, 25–27 February 2004“ bei \textcite{house} werden als
\texttt{note} behandelt.\footnote{\texttt{note} wird nicht mehr mit einem
nachfolgenden Komma erzwungen wie in Version 0.1. Hier greift nun wieder die
Standardeinstellung von \texttt{authoryear}.}
Für entsprechende Informationen beim Literaturtypus \texttt{@Unpublished} kann
alternativ jedoch auch \texttt{pubstate} genutzt werden.
(Vgl. neben \textcite{swallow2} und \textcite{house} auch \textcite{ferra-ma}.)

\subsubsection{Mehrbändige Publikationen}
Wie soll bei Titeln aus einer mehrbändigen Veröffentlichung vorgegangen
werden? Die Broschüre liefert kein Beispiel. Abhilfe schafft hier das
\textit{Unified style sheet for linguistics}: In einer Beispielreferenz
erscheint \texttt{volume} nach dem \texttt{title}. Die Gesamtbände werden
dort nicht angegeben.
Folglich erzeugt \texttt{uni-wtal-lin} den Band an eben dieser Stelle und
ignoriert die standardmäßig vorhandere Gesamtzahl der Bände (\texttt{volumes}).
Dies kann man z.B. bei \textcite{sterne1} sehen, der \texttt{volumes} als Feld
führt, das aber ignoriert wird.

Wie ein einzelner \texttt{booktitle} einer \texttt{volume} verarbeitet werden
soll, ist in beiden Regelvorschlägen nicht angegeben. Mein Ansatz, der bei
\textcite{unselbst-sammel-ders} gesichtet werden kann, sollte jedoch eine
elegante Lösung darstellen.

\subsubsection{Drei und mehr Autoren}
Bei drei Autoren wird in der Broschüre ein „und“ zwischen dem zweiten und
dritten Autor erzeugt. Da insgesamt zwei Autoren jedoch per Schrägstrich
getrennt werden, ist dies problematisch für \LaTeX, da beides über die Funktion
\texttt{finalnamedelim} bedient werden müsste. Entweder muss der letzte
Delimiter ein „und“ sein – dann ist er es jedoch auch bei zwei Namen – oder man
bleibt immer beim Schrägstrich. Letzteres habe ich hier als Standard gewählt.\\
Bei Bedarf kann die Option in der \texttt{bbx}-Datei angepasst werden.

Wie bei mehr als drei Autoren oder Herausgebern im Literaturverzeichnis
verfahren werden soll, ist nicht ersichtlich. Hier habe ich mich an der ZS
orientiert. Dort ist es üblich, dass explizit alle(!) Autoren mit vollem Namen
genannt werden sollten. Aus diesem Grunde wird der Stil standardmäßig mit der
Option \texttt{maxbibnames=99} geladen.

\subsubsection{Lexika}
Ein Beispiel für Lexikoneinträge besteht ebenfalls nicht. Gemäß
\textit{Unified style sheet} werden diese – wie allgemein üblich – an die
vorderste Stelle gesetzt. Nötig hierzu ist die Option (im entsprechenden
Datensatz in der \texttt{bib}-Datei!) \texttt{options = {useeditor=false}}.

\subsubsection{Onlinequellen}
Für Onlinequellen existiert in der Broschüre inzwischen ein Beispiel (bzw. ein
Standardschema); das \textit{Unified style sheet} fordert jedoch, URLs an die
Stelle von \texttt{publisher} oder \texttt{journal} zu setzen und dabei das
grundlegende Format der Quellenart unverändert zu lassen. Neben dem von
bib\LaTeX\ unterstützten einzelnen Feld \texttt{@Online} ist es bei diesem Style
daher auch möglich, URLs in Büchern, Artikeln und Sammelbänden anzugeben. Da
jedoch erstens im \textit{Unified style sheet} selbst an einer Stelle dennoch
sowohl das \texttt{journal} als auch die URL angegeben sind und es zweitens in
der Praxis auch vorkommt, dass zu einer Onlinequelle auch Ort und Verlag
existieren – nämlich bei digitalisierten Büchern, wie es z.B.
bei \textcite{sievers} der Fall ist –, wird in \texttt{uni-wtal-lin} die URL
fast ganz ans Ende der Literaturangabe gesetzt; nur die Reihe sowie, falls
verwendet, \texttt{addendum} und \texttt{pubstate} folgen noch.\footnote{Der URL
geht nun ein führendes \texttt{URL:} voraus, da die Broschüre dies – entgegen
dem \textit{Unified style sheet} – inzwischen so vorsieht.}\\
Die Schriftart bleibt, trotz der etwas klobigen Darstellungsweise, in Monospace
formatiert, damit man die Zeichen besser unterscheiden kann.

\subsection{Harvard-Kurzzitierweise}
Bei der Harvard-Kurzzitierweise wird gemäß den Vorschlägen in der Broschüre
ein Doppelpunkt zwischen Jahr und Seitenzahl gesetzt.

Schon \textcite[34]{swallow2} bemerkte, dass \ldots

Mehrfachzitate werden mit Komma getrennt.\footnote{Noch gemäß dem Beispiel
der 4. Auflage.} \parencites{schatz1}{schatz2}

Bei Literaturangaben mit zwei Autoren wird nun gemäß der
Delimiter-Konfiguration ein Schrägstrich verwendet: So schreiben
\textcite{swallow1}: [\ldots]

Bei Literaturangaben mit mehr als zwei Autoren wird „et al.“ verwendet. Hierzu
lädt der Stil automatisch die Option \texttt{maxnames=2}.
\parencite[Vgl.][21]{haupt}


\section{Anmerkungen}
Ich hoffe, mit diesem Zitierstil einen nützlichen Beitrag geleistet zu haben –
und wie ich hoffe nicht nur für Wuppertaler Linguisten. Ich bin sehr an Kritik,
Anregungen und natürlich Bugreports interessiert.

Wenn ich richtig recherchiert habe, so scheint zu den Vorgaben des
\textit{Unified style sheet for linguistics} bislang auch nur eine
bib\TeX-Datei, jedoch keine bib\LaTeX-Version zu existieren. Sollte jemand eine
gefunden haben oder aber Bedarf daran haben, würde ich mich über eine Nachricht
freuen.

Ich bedanke mich an dieser Stelle auch für das tolle Feedback, das ich bislang
für \texttt{uni-wtal-lin} und für \texttt{uni-wtal-ger} erhalten habe.
Mein besonderer Dank gilt Herrn Prof. Dr. Horst Lohnstein, der diesen Stil
ausgiebig getestet und mich in meiner weiteren Arbeit an diesem Projekt
motiviert hat.

– Carsten A. Dahlmann (\texttt{Ace@Dahlmann.net})

\newpage
\printbibliography[heading=style,keyword=style]
\printbibliography[heading=lin,keyword=lin]
\printbibliography[heading=weitere,keyword=weitere]
\nocite{*}

\end{document}