%!TEX TS-program = xelatex
%!TEX encoding = UTF-8 Unicode

% biblatex-dw 
% Copyright (c) Dominik Waßenhoven <domwass(at)web.de>, 2016
%
% Diese Datei enthält die deutschsprachige Dokumentation von biblatex-dw

\documentclass[ngerman]{scrartcl}
%!TEX encoding = UTF-8 Unicode

% biblatex-dw 
% Copyright (c) Dominik Waßenhoven <domwass(at)web.de>, 2016
%
% This file is the preamble for the documentation of 
% biblatex-dw (both  the English and the German version)

%%%%% biblatex-dw Version %%%%% version of biblatex-dw %%%%%
\newcommand{\biblatexdwversion}{1.7}
\newcommand{\biblatexdwdate}{\printdate{2016-12-06}}
\newcommand{\mindestanforderung}{3.3}% minimum biblatex version
\newcommand{\testversion}{3.6}% tested biblatex version
\newcommand{\biberversion}{2.6}% tested biber version
\newcommand{\screenversion}{}
\newcommand{\TOC}{}
\newcommand{\lizenz}{}
	
%%%%% Kodierung %%%%% Encoding %%%%%
\usepackage{fontspec,xltxtra,xunicode}

%%%%% Inhaltsverzeichnis %%%%% Table of Contents %%%%%
\setcounter{tocdepth}{2}

%%%%% Schriftarten %%%%% Fonts %%%%%
\usepackage[osf]{libertine}
\defaultfontfeatures{Mapping=tex-text}
\setmonofont[Scale=MatchLowercase]{Bitstream Vera Sans Mono}
\setkomafont{sectioning}{\sffamily}%     Überschriften
\renewcommand{\headfont}{\normalfont\itshape}% Kolumnentitel

%%%%% Fußnoten %%%%% Footnotes %%%%%
\deffootnote%
  {2em}% Einzug des Fußnotentextes; bei dreistelligen Fußnoten evtl. vergrößern
  {1em}% zusätzlicher Absatzeinzug in der Fußnote
  {%
  \makebox[2em]% Raum für Fußnotenzeichen: ebenso groß wie Einzug des FN-Textes
    [r]% Ausrichtung des Fußnotenzeichens: [r]echts, [l]inks
    {\libertineLF% keine Mediävalziffern als Fußnotenmarke
    \thefootnotemark%
    \hspace{1em}% Abstand zw. FN-Zeichen und FN-Text
    }%
  }
\renewcommand{\footnoterule}{}% keine Zeile zw. Text und Fußnoten

%%%%% Kopf- und Fußzeilen %%%%% page header and footer %%%%%
\usepackage{scrpage2}
\pagestyle{scrheadings}
\clearscrheadfoot

%%%%% Farben %%%%% Colors %%%%%
\usepackage{xcolor}
\definecolor{dkblue}{rgb}{0 0.1 0.5}%			dark blue
\definecolor{dkred}{rgb}{0.85 0 0}%				dark red
\definecolor{pyellow}{rgb}{1 0.97 0.75}%	pale yellow

%%%%% Kompakte Listen %%%%% Compact lists %%%%%
\usepackage{enumitem}
\setlist{noitemsep}
\setitemize{leftmargin=*,nolistsep}
\setenumerate{leftmargin=*,nolistsep}

%%%%% Datumsformat %%%%% Date format %%%%%
\usepackage{isodate}
\numdate[arabic]% 15. 9. 2008
\isotwodigitdayfalse% führende Nullen weglassen

%%%%% Spracheinstellungen %%%%% Language settings %%%%%
\usepackage{babel}
\newcommand{\versionname}{Version}
\iflanguage{english}{\renewcommand{\versionname}{version}}{}
\iflanguage{german}{\monthyearsepgerman{\,}{\,}}{}
\iflanguage{ngerman}{\monthyearsepgerman{\,}{\,}}{}
\iflanguage{english}{\isodate}{}

%%%%% Verschiedene Pakete %%%%% Miscellaneous packages
\usepackage{microtype}%		optischer Randausgleich
\usepackage{dtk-logos}%		Logos wie \BibTeX etc.
\usepackage{xspace}
\usepackage{textcomp}%		Text-Companion-Symbole
\usepackage{manfnt}%			für das Achtung-Symbol
\usepackage{marginnote}
\reversemarginpar

%%%%% Auslassungspunkte %%%%% dots %%%%%
\let\ldotsOld\ldots
\renewcommand{\ldots}{\ldotsOld\unkern}

%%%%% Zwischenraum vor und nach \slash %%%%% Spacing around \slash %%%%%
\let\slashOld\slash
\renewcommand{\slash}{\kern.05em\slashOld\kern.05em}

%%%%% Anführungszeichen %%%%% quotation marks %%%%%
\usepackage[babel,german=guillemets]{csquotes}

%%%%% Listings %%%%% listings %%%%%
\usepackage{listings}
\lstset{%
	frame=none,
%	backgroundcolor=\color{pyellow},
	language=[LaTeX]TeX,
	basicstyle=\ttfamily\small,
	commentstyle=\color{red},
	keywordstyle=, % LaTeX-Befehle werden nicht fett dargestellt
	numbers=none,%left/right
%	numberstyle=\tiny\lnstyle,
%	numbersep=5pt,
%	numberblanklines=false,
	breaklines=true,
%	caption=\lstname,
	xleftmargin=5pt,
	xrightmargin=5pt,
	escapeinside={(*}{*)},
	belowskip=\medskipamount,
	prebreak=\mbox{$\hookleftarrow$}% übernommen vom scrguide (KOMA-Script)
}

%%%%% biblatex %%%%% biblatex %%%%%
\usepackage[%
  bibencoding=utf8,
  style=authortitle-dw,
%  bernhard=true,%
  journalnumber=date
]{biblatex}
\addbibresource{examples/examples-dw.bib}

%%%%% Hyperref %%%%% Hyperref %%%%%
\usepackage{hyperref}
\hypersetup{%
	colorlinks=true,%
	linkcolor=dkblue,% Links
	citecolor=dkblue,% Links zu Literaturangaben
	urlcolor=dkblue,% Links ins Internet
	pdftitle={biblatex-dw},%
	pdfsubject={Dokumentation des LaTeX-Pakets biblatex-dw},%
	pdfauthor={Dominik Waßenhoven},%
	pdfstartview=FitH,%
	bookmarksopen=true,%
	bookmarksopenlevel=2,%
	pdfprintscaling=None,%
}

%%%%% Kurzbefehle %%%%% shortening commands %%%%%
\newcommand{\cmd}[1]{\texttt{\textbackslash #1}}
\newcommand{\option}[1]{\textcolor{dkblue}{#1}}
\newcommand{\wert}[1]{\textcolor{dkblue}{\enquote*{#1}}}
\newcommand{\paket}[1]{\textsf{#1}}
\newcommand{\xbx}[1]{\enquote{#1}}

\makeatletter
\newcommand{\tmp@beschreibung}{}
\DeclareRobustCommand\beschreibung[2][]{% 
  \bgroup 
    \def\tmp@beschreibung{#1}% 
    \ifx\tmp@beschreibung\@empty 
      % leerer Fall 
      \label{#2}%
	    \marginpar{\footnotesize\sffamily\textcolor{dkblue}{#2}}%
    \else 
      % unleerer Fall 
      \label{#1}%
	    \marginpar{\footnotesize\sffamily\textcolor{dkblue}{#2}}%
    \fi 
  \egroup 
}
\newcommand{\tmp@beschreibungcmd}{}
\DeclareRobustCommand\beschreibungcmd[2][]{% 
  \bgroup 
    \def\tmp@beschreibungcmd{#1}% 
    \ifx\tmp@beschreibungcmd\@empty 
      % leerer Fall 
      \label{#2}%
	    \marginpar{\footnotesize\sffamily\textcolor{dkblue}{\textbackslash #2}}%
    \else 
      % unleerer Fall 
      \label{#1}%
	    \marginpar{\footnotesize\sffamily\textcolor{dkblue}{\textbackslash #2}}%
    \fi 
  \egroup 
}
\makeatother

\newcommand{\bl}{\paket{biblatex}}
\newcommand{\bldw}{\paket{biblatex-dw}}

% \optlist[nur bei xy]{Option}{Wert} -> Option (Wert)
% \optset[nur bei xy]{Option}{Wert} -> Option=Wert
% \opt{Option}
% \optnur[nur bei xy]{Option}
% \befehl{Befehlsname}{Definition}{Beschreibung}
% \befehlleer{Befehlsname}{Beschreibung}
\newcommand{\optlist}[3][]{\item[\option{#2}](#3) \emph{#1}\hfill%
  {\footnotesize\sffamily\seite{\pageref{#2}}\\}}
\newcommand{\optset}[3][]{\item[\option{#2=#3}] \emph{#1}\\}
\newcommand{\opt}[1]{\item[\option{#1}]~\hfill%
  {\footnotesize\sffamily\seite{\pageref{#1}}\\}}
\newcommand{\optnur}[2][]{\item[\option{#2}]\emph{#1}\hfill%
  {\footnotesize\sffamily\seite{\pageref{#2}}\\}}  
\newcommand{\befehl}[3]{\item[\option{\cmd{#1}}]\texttt{#2}\\{#3}}
\newcommand{\befehlmitverweis}[2]{\item[\option{\cmd{#1}}]~\hfill%
  {\footnotesize\sffamily\seite{\pageref{#1}}}\\{#2}}
\newcommand{\befehlleer}[2]{\item[\option{\cmd{#1}}]\emph{(\iflanguage{english}{empty}{leer})}\\{#2}}
\newcommand{\biblstring}[3]{%
  \item[\option{#1}]\texttt{#2}\hspace{0.4em}\textbullet\hspace{0.4em}\texttt{#3}}
\newcommand{\feldformat}[3]{\item[{\option{\cmd{DeclareFieldFormat\{{#1}\}\{}\cmd{#2\}}}}]~\\#3}
\newcommand{\eintragstyp}[3]{\item[{\texttt{@#1}}]#2 {\small(in \bl{}:
  \iflanguage{ngerman}{wie}{same as} \texttt{@#3})}\hfill%
  {\footnotesize\sffamily\seite{\pageref{@#1}}}}

\newcommand{\achtung}{\marginnote{\footnotesize\dbend}}

\newcounter{beispiel}
\newcommand{\beispiel}{%
  \iflanguage{english}{Example}{Beispiel}}
\newcommand{\seite}[1]{%
  \iflanguage{english}{page~#1}{Seite~#1}}  

\newcommand{\Mindestanforderung}{%
  \iflanguage{english}%
	  {\textcolor{dkred}{needs at least version~\mindestanforderung{} of \bl{}}}%
		{\textcolor{dkred}{benötigt mindestens Version~\mindestanforderung{} von \bl{}}}}

\newcommand{\Testversion}{%
  \iflanguage{english}%
	  {~and was tested with \bl{} version~\testversion{} and \paket{biber} version~\biberversion{}}%
		{~und wurde mit Version~\testversion{} von \bl{} sowie Version~\biberversion{} von \paket{biber} getestet}}

%%%%% Titelei %%%%% title page %%%%%
\author{Dominik Waßenhoven}
\title{biblatex-dw}
\date{Version~\biblatexdwversion, \biblatexdwdate}

%%%%% Worttrennungen %%%%% Hyphenation %%%%%
\usepackage[htt]{hyphenat}
\hyphenation{
  Stan-dard-ein-stel-lun-gen
}

\nonfrenchspacing% 	Präambel
%%!TEX encoding = UTF-8 Unicode
% biblatex-dw 
% Copyright (c) Dominik Waßenhoven <domwass(at)web.de>, 2016
%
% This file configures the documentation 
% of biblatex-dw to be printed

%%%%% KOMA-Script Optionen %%%%% KOMA-Script options %%%%%
\KOMAoptions{
	paper=a4
	,headinclude=true
%	,fontsize=12pt,DIV=calc
	,fontsize=11pt,DIV=9
}

%%%%% Mindestanforderung %%%%% Minimum requirement
\renewcommand{\Mindestanforderung}{%
  \iflanguage{english}%
	  {needs at least version~\mindestanforderung{} of \bl{}}%
		{mindestens Version~\mindestanforderung{} von \bl{}}}
		
%%%%% Kopf- und Fußzeilen %%%%% page header and footer %%%%%
\ihead{%
  \iflanguage{english}%
    {biblatex-dw documentation}%
    {Dokumentation von biblatex-dw}}
\ohead{\versionname~\biblatexdwversion, \biblatexdwdate}% Kopf rechts
\cfoot{\pagemark}

%%%%% Lizenzhinweis %%%%% note on licence %%%%%
\renewcommand{\lizenz}{%
  \vfill
  \begin{quote}
  \small
  \itshape
  \iflanguage{english}{%
    This manual is part of the \bldw\ package. It may be distributed and\slash
    or modified under the conditions of the \enquote{\LaTeX\ Project Public
    License}. For more details, please have a look at the
    \enquote{\texttt{README}} file.
   }{%
    Dieses Handbuch ist Teil des Pakets \bldw. Es darf nach den Bedingungen
    der \enquote{\LaTeX\ Project Public Licence} verteilt und\slash oder
    verändert werden. Für weitere Informationen schauen Sie bitte in die Datei
    \enquote{\texttt{LIESMICH}}.
   }%
   \end{quote}
   \clearpage}

%%%%% Inhaltsverzeichnis %%%%% Table of Contents %%%%%
\let\TOC\tableofcontents

%%%%% Keine farbigen Links %%%%% No coloured links %%%%%
\hypersetup{%
	colorlinks=false,%
	%linkcolor=dkblue,% Links
	%citecolor=dkblue,% Links zu Literaturangaben
	%urlcolor=dkblue,% Links ins Internet
	%pdftitle={biblatex-dw},%
	%pdfsubject={Dokumentation des LaTeX-Pakets biblatex-dw},%
	%pdfauthor={Dominik Waßenhoven},%
	%pdfstartview=FitH,%
%	bookmarksopen=true,%
%	bookmarksopenlevel=0,%
	%pdfprintscaling=None,%
}
%  		Druckversion
%!TEX encoding = UTF-8 Unicode
% biblatex-dw 
% Copyright (c) Dominik Waßenhoven <domwass(at)web.de>, 2016
%
% This file configures the documentation 
% of biblatex-dw to be viewed on a screen

%%%%% Satzspiegel %%%%% type area %%%%%
% übernommen aus dem scrguide von Markus Kohm
% adopted from Markus Kohm's scrguide
\usepackage{geometry}
\geometry{papersize={140mm,210mm},%
      includehead,includemp,reversemp,marginparwidth=2em,%
      vmargin={2mm,4mm},hmargin=2mm}%
      
%%%%% Kopf- und Fußzeilen %%%%% page header and footer %%%%%
% zu großen Teilen übernommen aus dem scrguide von Markus Kohm
% for the most part adopted from Markus Kohm's scrguide
\makeatletter
    \setlength{\@tempdimc}{\oddsidemargin}%
    \addtolength{\@tempdimc}{1in}%
    \setheadwidth[-\@tempdimc]{paper}%
\makeatother
\ohead{\smash{%
  \rule[-\dp\strutbox]{0pt}{\headheight}\pagemark\hspace{2mm}}}%
\ihead[%
  {\smash{\colorbox{pyellow}{%
    \makebox[\dimexpr\linewidth-2\fboxsep\relax][l]{%
      \rule[-\dp\strutbox]{0pt}{\headheight}%
      \makebox[2em][r]%
        biblatex-dw, \versionname~\biblatexdwversion, \biblatexdwdate}}}}]%
  {\smash{\colorbox{pyellow}{%
     \makebox[\dimexpr\linewidth-2\fboxsep\relax][l]{%
       \rule[-\dp\strutbox]{0pt}{\headheight}%
       \makebox[2em][r]%
         biblatex-dw, \versionname~\biblatexdwversion, \biblatexdwdate}}}}%

%%%%% Hinweis zur Bildschirmversion %%%%% note on screen version %%%%%
\renewcommand{\screenversion}{%
  \begin{quote}
  \small
  \color{dkred}
  \sffamily
  \iflanguage{english}{%
    This is the screen version of the \bldw{} documentation.
    If you would like to have a printable version, please have a look
    at the \enquote{\texttt{README}} file.
   }{%
    Dies ist die Bildschirmversion der Dokumentation von \bldw{}.
    Wenn Sie eine Druckversion haben möchten, schauen Sie bitte in die
    Datei \enquote{\texttt{LIESMICH}}.
   }%
   \end{quote}}
    
%%%%% Lizenzhinweis %%%%% note on licence %%%%%
\renewcommand{\lizenz}{%
  \vfill
  \begin{quote}
  \small
  \itshape
  \iflanguage{english}{%
    This manual is part of the \bldw\ package. It may be distributed and\slash
    or modified under the conditions of the \enquote{\LaTeX\ Project Public
    License}. For more details, please have a look at the
    \enquote{\texttt{README}} file.
   }{%
    Dieses Handbuch ist Teil des Pakets \bldw. Es darf nach den Bedingungen
    der \enquote{\LaTeX\ Project Public Licence} verteilt und\slash oder
    verändert werden. Für weitere Informationen schauen Sie bitte in die Datei
    \enquote{\texttt{LIESMICH}}.
   }%
   \end{quote}}
     
%%%%% Kein Inhaltsverzeichnis %%%%% No Table of Contents %%%%%
\renewcommand{\TOC}{\clearpage}% 		Bildschirmversion

%%%%% %%%%% %%%%% %%%%% %%%%% %%%%% 
%%%%%  Anfang des Dokuments   %%%%%
%%%%% %%%%% %%%%% %%%%% %%%%% %%%%% 
\begin{document}
\maketitle
\thispagestyle{empty}

\abstract{\noindent \bldw{} ist eine kleine Sammlung von Zitierstilen
 für das Paket \bl{} von Philipp Lehman. Sie ist gedacht zum 
 Zitieren und Bibliographieren im geisteswissenschaftlichen Bereich 
 und bietet dafür einige Funktionen, die über die Standardfunktionen 
 von \bl{} hinausgehen. \bldw{} baut vollständig auf 
 \bl{} auf~-- die Version~\biblatexdwversion{} 
 \Mindestanforderung{}\Testversion. Bitte achten Sie auch auf die 
 Mindestanforderungen von \bl{} selbst.}

\lizenz
\screenversion
\TOC

\section{Einleitung}
\subsection{Installation}
\bldw{} ist in den Distributionen MiK\TeX{}\footnote{Webseite: \url{http://www.miktex.org}.} 
und \TeX{}~Live\footnote{Webseite: \url{http://www.tug.org/texlive}.} enthalten und
kann mit deren Paketmanagern bequem installiert werden. Wenn Sie stattdessen eine
manuelle Installation durchführen, so gehen Sie wie folgt vor:
Entpacken Sie die Datei \texttt{biblatex-dw.tds.zip} in den \texttt{\$LOCALTEXMF}"=Ordner ihres 
Systems.\footnote{Falls Sie nicht wissen, was das ist, können Sie sich unter 
\url{http://projekte.dante.de/DanteFAQ/TDS} bzw. 
\url{http://mirror.ctan.org/tds/tds.html} informieren.} Aktualisieren Sie 
anschließend die \emph{filename database} ihrer \TeX"=Distribution. 
Weitere Informationen:
\begin{itemize}
  \small
	\item \url{http://projekte.dante.de/DanteFAQ/PaketInstallation}
	\item \url{http://projekte.dante.de/DanteFAQ/Verschiedenes#67}
\end{itemize}
 
\subsection{Benutzung}
Die hier angebotenen Zitierstile werden wie die Standard-Stile
beim Laden des Pakets \bl{} eingebunden:

\begin{lstlisting} 
\usepackage[style=authortitle-dw]{biblatex}
\end{lstlisting}
bzw.
\begin{lstlisting} 
\usepackage[style=footnote-dw]{biblatex}
\end{lstlisting}
Die Stile sind so konstruiert, dass sie stark miteinander verschränkt sind.
Das bedeutet, dass die Kombination eines \bldw"=Stiles mit einem anderen Stil
nicht unbedingt möglich sein wird.
Für einen ersten Überblick über die Stile stehen die Beispiele \enquote{de-authortitle-dw} 
und \enquote{de-footnote-dw} im Ordner \texttt{examples} zur Verfügung.

\subsection{Globale Optionen und Eintragsoptionen}
Die Optionen, die \bl{} zur Verfügung stellt, können auch mit \bldw{} verwendet
werden. Die von den \bldw"=Stilen zusätzlich bereitgestellten Optionen
werden im Folgenden erläutert. Dabei ist grundsätzlich zu unterscheiden
zwischen globalen Optionen und Eintragsoptionen: Globale Optionen gelten für alle
Literaturverweise im Dokument; sie werden entweder als Option beim Laden von \bl{}
oder in einer Konfigurationsdatei (\texttt{biblatex.cfg}) gesetzt. Eintragsoptionen
werden für jeden einzelnen Eintrag im Feld \texttt{options} gesetzt. Manche der
Eintragsoptionen überschreiben für den jeweiligen Eintrag eventuell gesetzte globale Optionen.

\subsection{Fragen und Antworten (FAQ)}
In der deutschen \TeX"=FAQ, die auf den Internetseiten der Deutschen Anwendervereinigung
\TeX\ e.\,V. (\DANTE) eingesehen werden kann, sind auch einige häufig gestellte Fragen 
zu \bl{} und \bldw{} beantwortet, abzurufen unter folgender Adresse:\\
\href{http://projekte.dante.de/DanteFAQ/LiteraturverzeichnisMitBiblatex}%
     {http://projekte.dante.de/DanteFAQ/LiteraturverzeichnisMitBiblatex}

Außerdem habe ich für \enquote{\DTK}, die Mitgliederzeitschrift
von \DANTE, einen einführenden Artikel zu \bl{} geschrieben, den ich auch als PDF
zum Herunterladen anbiete:\\
\href{http://biblatex.dominik-wassenhoven.de/dtk.shtml}%
     {http://biblatex.dominik-wassenhoven.de/dtk.shtml}

\subsection{Entwicklung}
\bldw{} wird quelloffen entwickelt und bei \href{http://sourceforge.net}{sourceforge.net} gehostet. Der Code (auch der jeweils aktuellen, noch nicht veröffentlichten Version) steht zum Download zur Verfügung.\footnote{\url{http://sourceforge.net/p/biblatex-dw/code}.} Auf sourceforge.net gibt es auch die Möglichkeit, Fehler zu melden und Verbesserungsvorschläge zu machen. Wenn Sie einen Fehler finden, erstellen Sie bitte einen Bug-Report (am besten inklusive Minimalbeispiel).\footnote{\url{http://sourceforge.net/p/biblatex-dw/tickets/milestone/Bugs}.} Für Verbesserungsvorschläge verwenden Sie ein Feature-Request.\footnote{\url{http://sourceforge.net/p/biblatex-dw/tickets/milestone/Features}.}

\section{Der Stil \xbx{authortitle-dw}}

Dieser Stil basiert auf dem Standardstil \xbx{authortitle}.
Neben einigen Änderungen in der Zeichensetzung gibt es folgende Unterschiede:

\subsection{Bibliographie}
\begin{itemize}
	\item Die\beschreibung{namefont}
	      Schriftart der Autoren und Herausgeber kann mit den Optionen% 
	      \beschreibung{firstnamefont}
				\option{namefont} und \option{firstnamefont} auf \wert{normal}, 
	      \wert{smallcaps}
	      (Kapitälchen), \wert{italic} (kursive Schrift) oder \wert{bold}
	      (fetter Schnitt) eingestellt werden.
	      Wenn die Option \option{useprefix=true} gesetzt ist, wirkt sich
	      \option{namefont} auch auf das Namenspräfix aus (also \enquote{von},
	      \enquote{de} etc.).
	      Mit \option{useprefix=false} (der Standardeinstellung) hängt das 
	      Namenspräfix von der Option \option{firstnamefont} ab, die in jedem 
	      Fall
	      das Namenssuffix (den \enquote{Junior}"=Teil) beeinflusst.
	\item Wird\beschreibung{oldauthor}
	      die Option \option{namefont} benutzt, aber es sollen dennoch 
	      einige Namen nicht in der gewählten Schriftform gesetzt werden (z.\,B.
	      mittelalterliche oder antike Autoren), kann im entsprechenden
	      Eintrag der bib-Datei das Feld \texttt{options\,=\,\{oldauthor=true\}} 
	      oder \texttt{options\,=\,\{oldauthor\}} gesetzt werden. Sollen diese
	      mit \texttt{oldauthor} gekennzeichneten Einträge in derselben Schrift
	      gesetzt werden wie die anderen Einträge, kann man die Eintragsoption
	      mit der globalen Option \option{oldauthor=false} überschreiben.
  \item Es\beschreibung{oldbookauthor} gibt auch die Eintragsoption \option{oldbookauthor},
        die dieselbe Funktionalität wie \option{oldauthor} für das Feld \texttt{bookauthor}
        bereitstellt. Dies ist nützlich für \texttt{@inbook}"=Einträge, die beispielsweise
        eine Einleitung zur Edition eines Werkes bieten, dessen Autor nicht in der normalen
        Schrift gesetzt werden soll. Diese Option lässt sich mit der globalen Option
        \option{oldauthor=false} ausschalten.
	\item Die\beschreibung{idemfont}
	      Schriftart von \enquote{Ders.}\slash\enquote{Dies.} (siehe unten) 
	      kann mit der Option \option{idemfont} auf \wert{normal}, 
	      \wert{smallcaps} (Kapitälchen), \wert{italic} (kursive Schrift) oder 
	      \wert{bold} (fetter Schnitt) eingestellt werden. Wird die Option nicht 
	      verwendet, so wird der Schriftschnitt der Option \option{namefont} 
	      benutzt. Das ist auch das Standardverhalten.
	\item Die\beschreibung{ibidemfont}
	      Schriftart von \enquote{ebenda}\slash\enquote{ebd.} (siehe unten) 
	      kann mit der Option \option{ibidemfont} auf \wert{normal}, 
	      \wert{smallcaps} (Kapitälchen), \wert{italic} (kursive Schrift) oder 
	      \wert{bold} (fetter Schnitt) eingestellt werden. Der Standard ist 
	      \wert{normal}.	      
  \item Die\beschreibung{acronyms}
	      Siglen (\texttt{shorthand} und \texttt{shortjournal}) können mit
        dem\beschreibung{acronym}
	      Befehl \cmd{mkbibacro} gesetzt werden (Standard für diesen Befehl:
        \textsc{smallcaps}). Dazu setzt man die globale Option \option{acronyms=true}
        \emph{und} im entsprechenden Eintrag der bib-Datei das Feld 
        \texttt{options = \{acronym=true\}}. Zur Anpassung des Befehls
        \cmd{mkbibacro} siehe auch den Abschnitt \enquote{\nameref{mkbibacro-anpassen}}
        (S.~\pageref{mkbibacro-anpassen}).
	\item Mit\beschreibung{idembib}
	      der Option \option{idembib} können aufeinanderfolgende Einträge 
	      desselben Autors\slash{}Herausgebers in der Bibliographie durch 
	      \enquote{Ders.} bzw. \enquote{Dies.} oder durch einen langen Strich
	      (---) ersetzt werden. Setzt man die Option auf \wert{false}, werden
	      die Namen auch in aufeinanderfolgenden Einträgen gleicher Autoren\slash
	      Herausgeber ausgegeben. Mit \option{idembib=true} wird die Ersetzung 
	      eingeschaltet. Das Format lässt sich dann durch die Option\beschreibung{idembibformat}
	      \option{idembibformat} einstellen; sie kann die Werte \wert{idem} 
	      (\enquote{Ders.}\slash\enquote{Dies.}) und \wert{dash} (---) annehmen. 
	      Das Geschlecht, das bei der Verwendung von \wert{idem} in manchen 
	      Sprachen 
	      wichtig ist, wird im Feld  \texttt{gender} festgelegt (siehe dazu die 
	      \bl"=Dokumentation). Der Standard für \option{idembib} ist 
	      \wert{true}, 
	      der Standard für \option{idembibformat} ist \wert{idem}.
	\item Wenn\beschreibung{edbyidem}
	      Autor und Herausgeber bei \texttt{@inbook}"~, \texttt{@incollection}"~ oder 
	      \texttt{@inreference}"=Einträgen dieselben sind, werden ihre Namen nicht 
	      wiederholt, sondern durch \enquote{hg.\,v.\,ders.}, 
	      \enquote{hg.\,v.\,dems.} etc. ersetzt. Das Geschlecht wird im Feld
	      \texttt{gender} festgelegt (siehe dazu die \bl{}"=Dokumentation).
	      Diese Funktion wird durch die Option \option{edbyidem} kontrolliert,
	      die die Werte \wert{true} oder \wert{false} annehmen kann; 
	      der Standard ist \wert{true}.
	\item Die\beschreibung{editorstring}
	      Option \option{editorstring} kann die Werte \wert{parens}, 
	      \wert{brackets} und \wert{normal} annehmen; der Standard ist
	      \wert{parens}. Diese Option setzt den Ausdruck \enquote{Herausgeber}
	      (abgekürzt \enquote{Hrsg.}) in runde Klammern (parens) oder in eckige 
	      Klammern (brackets). Wenn die Option den Wert \wert{normal} hat, wird 
	      der Ausdruck \enquote{Hrsg.} nicht von Klammern umgeben. Stattdessen
	      wird er an den Namen des Herausgebers und ein anschließendes Komma
	      angehängt.
	      Bei Verwendung von \option{usetranslator=true} trifft die Einstellung
	      von \option{editorstring} auch auf den Ausdruck \enquote{Übersetzer}
	      (abgekürzt \enquote{Übers.}) zu.
	\item Die\beschreibung{editorstringfont} Option \option{editorstringfont}
        bestimmt die Schriftart, die für den Ausdruck \enquote{Herausgeber}
	      (abgekürzt \enquote{Hrsg.}) benutzt wird. Mit \wert{normal} wird die
        Standardschrift des Dokuments benutzt, \wert{namefont} übernimmt die
        Einstellung der Option \option{namefont}. 
        Bei Verwendung von \option{usetranslator=true} trifft die Einstellung
	      von \option{editorstringfont} auch auf den Ausdruck \enquote{Übersetzer}
	      (abgekürzt \enquote{Übers.}) zu. Der Standard für diese Option
        ist \wert{normal}.
	\item Mit\beschreibung{pseudoauthor}
	      der Eintragsoption \option{pseudoauthor} kann man Autorennamen
	      in eckige Klammern einschließen oder ganz unterdrücken. Das ist beispielsweise nützlich für
				Editionen von Werken, deren Autoren nicht genannt werden, aber bekannt
				sind. Wenn die globale Option \option{pseudoauthor} auf \wert{true} gesetzt 
				(und die Eintragsoption \option{pseudoauthor} benutzt) wird, dann wird der 
				Autor ausgegeben. Dabei wird er von den Befehlen \cmd{bibleftpseudo} und \cmd{bibrightpseudo}
				eingerahmt. Diese Befehle sind standardmäßig leer. Wenn man den Autor beispielsweise in eckigen
				Klammern haben möchte, muss man die Befehle umdefinieren:
				\begin{lstlisting}
\renewcommand*{\bibleftpseudo}{\bibleftbracket}
\renewcommand*{\bibrightpseudo}{\bibrightbracket}
				\end{lstlisting}
				Wenn die globale Option \option{pseudoauthor} auf \wert{false} gesetzt wird, werden Autoren
				von Einträge mit der Eintragsoption \option{pseudoauthor} gar nicht ausgegeben. Der Standard für
				die globale Option ist \wert{true}.
  \item In\beschreibung{nopublisher}
	      der Standardeinstellung wird der Verlag (\texttt{publisher}) nicht
        ausgegeben, sondern nur Ort (\texttt{location}) und Jahr 
        (\texttt{date}). Will man den Verlag ausgeben, muss man die Option 
        \option{nopublisher=false} setzen.
  \item Mit\beschreibung{nolocation}
	      \option{nolocation=true} kann man auch die Ausgabe des Ortes 
        unterdrücken. In diesem Fall wird auch der Verlag nicht ausgegeben 
        (unabhängig von der
        Einstellung von \option{nopublisher}). Der Standard ist \wert{false}.
  \item Die Felder \texttt{doi}, \texttt{eprint}, \texttt{isbn}, \texttt{isrn},
	      \texttt{issn}\beschreibung{pagetotal} und \texttt{pagetotal} werden mit den Standardeinstellungen 
        nicht ausgegeben. Sie können aber mit den Optionen \option{doi=true},
        \option{eprint=true}, \option{isbn=true} (gilt auch für \texttt{isrn}
				und \texttt{issn}) bzw. \option{pagetotal=true} berücksichtigt werden.
  \item Mit\beschreibung{origfields}
	      der Option \option{origfields} kann man entscheiden, ob man die 
        Felder \texttt{origlocation}, \texttt{origpublisher} und 
        \texttt{origdate} ausgeben lassen möchte oder nicht. Der Standard ist 
        \wert{true}. Wenn die Option genutzt wird und das Feld 
        \texttt{origlocation} gesetzt ist, werden die \enquote{orig}"=Felder 
        ausgegeben. In diesem Fall werden dann die Felder \texttt{location}, 
        \texttt{publisher} und \texttt{date} in Klammern angefügt, eingeleitet 
        durch den \emph{bibliography string} \texttt{reprint} (\enquote{Nachdr.} oder 
        \enquote{Nachdruck}). Dabei werden die Felder \texttt{publisher} und 
        \texttt{origpublisher} nur ausgegeben,
        wenn die Option \option{nopublisher=false} eingestellt ist.
        \achtung Beachten Sie, dass die Edition eines Werks sich immer auf die
        ursprüngliche Ausgabe bezieht, da Nachdrucke normalerweise nicht in 
        mehreren Auf"|lagen erscheinen, sondern eine spezifische Auf"|lage 
        erneut veröffentlichen. Wenn die Option \option{edsuper} benutzt wird,
        erscheint die Auf"|lage also als hochgestellte Zahl vor 
        \texttt{origdate}.
  \item Mit\beschreibung{origfieldsformat}
	      der Option \option{origfieldsformat}, die die Werte \wert{parens},
        \wert{brackets} und \wert{punct} annehmen kann, lässt sich einstellen,
        wie die Angaben zum Nachdruck (bei \option{origfields=true}) ausgegeben
        werden. Mit \wert{parens} oder \wert{brackets} werden sie in runden 
        bzw. eckigen Klammern gesetzt. Der Standard ist \wert{punct}; dabei 
        werden die Angaben durch das Zeichen \cmd{origfieldspunct} 
        eingeleitet, das auf ein Komma voreingestellt ist.
  \item Die Zeichensetzung vor dem Titelzusatz (\texttt{titleaddon},
        \texttt{booktitleaddon} und \texttt{maintitleaddon}) wird durch den 
        neuen Befehl \cmd{titleaddonpunct} gesteuert. Der Standard ist ein 
        Punkt.
  \item Wird\beschreibung{edsuper}
	      die Option \option{edsuper} auf \wert{true} gesetzt, erscheint
        die Auf"|lage (\texttt{edition}) als hochgestellte Zahl (nicht als 
        Ordnungszahl) direkt vor dem Jahr. Der Standardwert für diese Option 
        ist \wert{false}.\\
        \achtung Beachten Sie, dass die Option \option{edsuper} nur dann 
        funktioniert,
        wenn im Feld \texttt{edition} \emph{nur} eine Zahl steht. Angaben wie 
        \enquote{5., aktualisierte und ergänzte Aufl.} werden wie sonst auch
        normal ausgegeben. Gleichzeitig wird in diesem Fall eine Warnung 
        ausgegeben.
        Will man von der Option \option{edsuper} Gebrauch machen, wird deshalb 
        dringend geraten, in das Feld \texttt{edition} weiterhin nur Zahlwerte 
        einzugeben und für ausführliche Angaben zu Auf"|lagen das Feld 
        \texttt{note} zu verwenden.
	\item Mit\beschreibung{editionstring} \option{editionstring=true} wird
        der \emph{bibliography string} \enquote{edition} zur Edition (Feld
        \texttt{edition}) automatisch hinzugefügt, auch wenn es sich dabei 
        nicht um eine Zahl handelt. Man kann also beispielsweise in der bib-Datei
	      \begin{lstlisting}
edition = {2., aktualisierte}
	      \end{lstlisting}
        schreiben und erhält \enquote{2., aktualisierte Aufl.}
        Setzt man die Option auf \wert{false}, wird nur dann der \emph{bibliography string}
        hinzugefügt, wenn in \texttt{edition} nur eine Zahl angegeben wird
        (\bl"=Standard"=Verhalten). 
        Der Standard für diese Option ist \wert{false}.
	\item Mit\beschreibung{shortjournal}
	      der Option \option{shortjournal=true} wird statt \texttt{journaltitle}
	      das Feld \texttt{shortjournal} verwendet. Das ist sehr nützlich für
	      Zeitschriftensiglen, die man bei Bedarf einsetzen kann.
  \item Fehlt bei Zeitschriften die Bandangabe (\texttt{volume}), wird die 
        Jahreszahl \emph{nicht} in Klammern gesetzt: 
        \enquote{Zeitschriftenname 2008}. Falls dagegen das Feld \texttt{month}
        gesetzt ist (oder das Feld \texttt{date} den entsprechenden Inhalt aufweist, z.\,B.
				\texttt{2008-03}), wird das Datum vom Zeitschriftennamen durch ein 
        zusätzliches Komma getrennt.
	\item Mit\beschreibung{journalnumber}
	      der Option \option{journalnumber} lässt sich die Position der
	      Heftnummer (\texttt{number}) bei Zeitschriften variieren. Mit dem 
	      Wert \wert{afteryear} wird sie nach dem Jahr
	      ausgegeben, eingeleitet durch den Befehl \cmd{journumstring} (siehe 
	      unten), also \enquote{Zeitschriftenname 28 (2008), Nr.~2}. Setzt man
	      die Option auf \wert{standard}, wird das Standardverhalten 
	      wiederhergestellt, wobei das Zeichen zwischen \texttt{volume} und 
	      \texttt{number} mit dem neuen Befehl \cmd{jourvolnumsep}
	      (Standard: \cmd{adddot}) einstellbar ist. Zusätzlich gibt es noch
	      die Möglichkeit, \option{journalnumber} auf \wert{date} zu setzen.
	      Damit wird gewährleistet, dass das Datum auch dann ausgegeben wird,
	      wenn das Feld \texttt{issue} definiert ist (das ist in den 
	      Standard"=Stilen nicht der Fall). Außerdem wird die Heftnummer vor
	      dem Datum ausgegeben, wenn zumindest Jahr (\texttt{year}) und Monat
	      (\texttt{month}) angegeben sind (\texttt{date = \{YYYY-MM\}} ist natürlich
				ebenfalls möglich). Wenn nur das Jahr gegeben ist,
	      wird die Heftnummer nach dem Jahr ausgegeben. Für nähere Einzelheiten
	      zur Option \option{journalnumber=date} siehe 
	      Abschnitt~\ref{journalnumberdate} auf 
	      Seite~\pageref{journalnumberdate}. Der Standard für
	      \option{journalnumber} ist \wert{true}.
	\item Mit dem neuen Befehl \cmd{journumstring} wird die Heftnummer von
	      Zeitschriften eingeleitet. Der Standard ist \wert{, Nr.~}. Der 
	      Befehl kann leicht an die eigenen Bedürfnisse angepasst werden, 
	      z.\,B.:
				\begin{lstlisting}
\renewcommand*{\journumstring}{\addspace}
				\end{lstlisting}
	\item Mit dem neuen Befehl \cmd{jourvolstring} wird die Bandnummer von
	      Zeitschriften eingeleitet. Der Standard ist ein Leerzeichen. Der 
	      Befehl kann leicht an die eigenen Bedürfnisse angepasst werden, 
	      z.\,B.:
				\begin{lstlisting}
\renewcommand*{\jourvolstring}{%
  \addspace Jg\adddot\space}
				\end{lstlisting}
  \item Die\beschreibung{series}
	      Option \option{series} bestimmt die Position der Reihe (Feld
        \texttt{series}), möglich sind die Werte \wert{afteryear}, \wert{beforeedition} und \wert{standard}. Das betrifft die Typen \texttt{@book}, 
        \texttt{@inbook}, \texttt{@collection}, \texttt{@incollection}, 
        \texttt{@proceedings}, \texttt{@inproceedings} und \texttt{@manual}. Bei \wert{afteryear} wird die Reihe nach dem Jahr ausgegeben, bei \wert{beforeedition} vor der Auf"|lage.
        Der Standardwert für diese Option ist \wert{standard}, die Reihe wird dann vor dem Ort ausgegeben.
	\item Die\beschreibung{seriesformat} Option \option{seriesformat} kann
        die Werte \wert{standard} und \wert{parens} annehmen. Mit
        \wert{parens} werden Reihe und Nummer eines Werkes (\texttt{series} und 
        \texttt{number}) von Klammern umgeben, ansonsten ohne Klammern (wie es
        in den Standard"=Stilen der Fall ist). Der Standard für diese Option ist
        \wert{parens}.
  \item Der Befehl \cmd{seriespunct} bestimmt die Zeichensetzung vor dem 
        Reihentitel (Feld \texttt{series}). Bei \option{seriesformat=parens}
        wird das Zeichen innerhalb der Klammer gesetzt. 
        Es kann z.\,B. auf \wert{=\cmd{addspace}} eingestellt 
        werden, was in manchen Fächern üblich ist. Standardmäßig ist dieser 
        Befehl leer, d.\,h. der Reihentitel wird direkt nach der öffnenden 
        runden Klammer gesetzt.
  \item Der Befehl \cmd{sernumstring} bestimmt die Zeichensetzung zwischen dem 
        Reihentitel (Feld \texttt{series}) und der dazugehörigen Nummer (Feld 
        \texttt{number}). Der Standardwert ist \wert{\cmd{addspace}}. Er kann 
        leicht angepasst werden, z.\,B.:
				\begin{lstlisting}
\renewcommand*{\sernumstring}{%
  \addcomma\space\bibstring{volume}\addspace}
				\end{lstlisting}
        Das würde \enquote{(Reihentitel, Bd. N)} ergeben.
  \item Mit\beschreibung{shorthandinbib} der Option \option{shorthandinbib}
	      lassen sich die Siglen (\texttt{shorthand}) im Literaturverzeichnis
				ausgeben. Mit \option{shorthandinbib=true} werden sie direkt vor dem
				entsprechenden Eintrag in eckigen Klammern ausgegeben. Das Erscheinungsbild
				wird durch das Feldformat \enquote{shorthandinbib} festgelegt. Will man
				die Klammern weglassen, definiert man das Feldformat folgendermaßen:
				\begin{lstlisting}
\DeclareFieldFormat{shorthandinbib}{#1}
				\end{lstlisting}
				Da in diesem Fall nur ein Leerzeichen zwischen Sigle und Literatureintrag
				steht, sollte man auch die Zeichensetzung nach der Sigle umdefinieren. Dafür 
				ist der Befehl \cmd{shorthandinbibpunct} zuständig, dem man etwa ein
				Gleichheitszeichen zuweisen könnte; dabei sollte man aber \cmd{nopunct}
				verwenden, um überflüssige Zeichen nach der Sigle zu vermeiden:
				\begin{lstlisting}
\renewcommand*{\shorthandinbibpunct}{%
  \addspace=\nopunct\addspace}
				\end{lstlisting}
	\item Setzt\beschreibung{annotation}
	      man die Option \option{annotation} auf
	      \wert{true}, wird das Feld \texttt{annotation} in 
	      \textit{\small kleiner kursiver} Schrift am Ende des Eintrags
	      ausgegeben. Der Standardwert für diese Option ist \wert{false}.
	      Das Erscheinungsbild kann mit dem folgenden Befehl angepasst werden:
				\begin{lstlisting}
\renewcommand{\annotationfont}{\small\itshape}
				\end{lstlisting}
	\item Setzt\beschreibung{library}
	      man die Option \option{library} auf
	      \wert{true}, wird das Feld \texttt{library} in 
	      {\small\sffamily  kleiner serifenloser} Schrift am Ende des Eintrags
	      ausgegeben. Der Standardwert für diese Option ist \wert{false}.
	      Das Erscheinungsbild kann mit dem folgenden Befehl angepasst werden:
				\begin{lstlisting}
\renewcommand{\libraryfont}{\small\sffamily}
				\end{lstlisting}
	\item Werden sowohl \option{annotation} als auch \option{library} auf
	      \wert{true} gesetzt, so wird der Inhalt des Feldes \texttt{annotation}
	      vor dem Inhalt des \texttt{library}-Feldes ausgegeben.
	\item In\beschreibung{@inreference}
        \bl{} ist der Eintragstyp \texttt{@inreference} ein Alias für
	      \texttt{@incollection}. In \bldw{} kann dieser Eintragstyp für
	      Artikel in Nachschlagewerken verwendet werden. Die Ausgabe ist ähnlich
	      wie bei \texttt{@incollection}, aber es gibt ein paar Unterschiede:
	      \begin{itemize}
          \item Der Titel wird in Anführungszeichen gesetzt.
          \item Der \textit{bibliography string} \enquote{inrefstring} (Standard:
                \enquote{Artikel}\slash\enquote{Art.}) wird vor dem Titel ausgegeben.
          \item Der Ort (Feld \texttt{location}) wird nicht ausgegeben.
          \item Gibt es eine Bandangabe (Feld \texttt{volume}), so sieht die Ausgabe
                z.\,B. folgendermaßen aus: \enquote{in: Lexikon 2 (1990), S. 120.}
        \end{itemize}
        Der Eintragstyp \texttt{@reference} bleibt ein Alias zu \texttt{@collection}.
        Damit kann also sowohl \texttt{@collection} als auch \texttt{@reference} für
        Nachschlagewerke (Lexika, Wörterbücher etc.) verwendet werden.
  \item Mit\beschreibung{inreference} der Option \option{inreference=full} werden
  			\texttt{@inreference}"=Einträge in Literaturverweisen immer vollständig ausgegeben,
				dafür aber nicht in die Bibliographie aufgenommen. Beachten Sie, dass bei Benutzung
				der Option \option{xref=false} der korrespondierende \texttt{@reference}"=Eintrag
				manuell zitiert werden muss (z.\,B. mit \cmd{nocite}), wenn er in der Bibliographie
				erscheinen soll. Die Option \option{inreference} befolgt die Einstellungen sowohl 
				von \option{ibidtracker} als auch \option{idemtracker}. Der Standardwert für diese
				Option ist \wert{normal}. 
	\item In\beschreibung{@review}
        \bl{} ist der Eintragstyp \texttt{@review} ein Alias für
	      \texttt{@article}. In \bldw{} kann dieser Eintragstyp für
        Rezensionen verwendet werden. Die Ausgabe ist ähnlich wie bei 
        \texttt{@article}, mit folgenden Unterschieden:
        \begin{itemize}
          \item Der Titel wird in Anführungszeichen gesetzt.
          \item Der \emph{bibliography string} \enquote{reviewof} (Standard:
                \enquote{Rezension zu}\slash\enquote{Rez. zu}) wird vor dem
                Titel ausgegeben.
          \item Statt einer manuellen Angabe des rezensierten Werkes im \texttt{title}"=Feld
                des \texttt{@review}"=Eintrags kann man auch einen Verweis im Feld
                \texttt{xref} machen. Der Eintrag mit dem dort angegebenen \BibTeX"=Key
                wird dann zitiert. Diese Vorgehensweise hat den Vorteil, das Einstellungen
                wie \option{namefont} oder \option{firstfull} auch für das rezensierte Werk 
                gelten.
        \end{itemize}
\end{itemize}

\subsection{Zitate im Text}
\begin{itemize}
	\item Folgen zwei unterschiedliche Zitate desselben Autors\slash
	      Heraus\-gebers direkt aufeinander, wird der Name im zweiten Zitat durch
	      \enquote{Ders.} oder \enquote{Dies.} ersetzt, sofern das Zitat nicht
	      das erste der aktuellen Seite ist. Das Geschlecht wird im Feld
	      \texttt{gender} festgelegt (siehe dazu die \bl"=Dokumentation).
	      Diese Funktion wird durch die \bl"=Option \option{idemtracker}
	      kontrolliert, die auf \wert{constrict} gesetzt ist. Wer sie abschalten
	      möchte, verwendet \option{idemtracker=false}. Für weitere Informationen
	      zu dieser Option wird auf die \bl"=Dokumentation verwiesen.
	\item Direkt aufeinanderfolgende Zitate werden durch \enquote{ebd.} ersetzt,
	      sofern das Zitat nicht das erste der aktuellen Seite ist. 
        Dieses Verhalten lässt sich mit der \bl{}"=Option \option{ibidtracker} 
        unterdrücken. In diesem Fall bleibt die \enquote{idem}-Funktionalität
	      erhalten, solange nicht die Option \option{idemtracker} auf \wert{false}
	      gesetzt wird.
	\item Die\beschreibung{shorthandibid}
	      Option \option{shorthandibid} kontrolliert, ob direkt 
	      aufeinanderfolgende Zitate mit einer Sigle (\texttt{shorthand})
	      ebenfalls durch \enquote{ebd.} ersetzt werden sollen oder nicht.
	      Mögliche Werte für diese Option sind \wert{true} und \wert{false}, 
	      der Standard ist \wert{true}. Diese Option hat keinerlei Effekt,
	      wenn die Option \option{ibidtracker} auf \wert{false} gesetzt wird.\\
	      Diese Funktion kann auch für jeden Eintrag separat eingestellt werden
	      mit \texttt{options\,=\,\{short""hand""ibid=""true\}} oder 
	      \texttt{options\,=\,\{short""hand""ibid""=false\}}. Die globale
	      Einstellung wird dann für den entsprechenden Eintrag ignoriert.	      
  \item Mit\beschreibung{addyear} der Option \option{addyear} wird das Jahr
  			der Publikation nach dem Titel angefügt. Das Jahr erscheint dabei in
			  Klammern. Der Standard für diese Option ist \wert{false}.
	\item Wenn\beschreibung{omiteditor}
	      man die Option \option{useeditor} auf \wert{false} setzt, kann man 
				mit \option{omiteditor=true} erreichen, dass der Herausgeber in 
				Literaturverweisen unterdrückt wird. Dasselbe gilt auch für Kurzverweise
				in der Bibliographie, sofern man \option{xref=true} verwendet.
				Mit \option{useeditor=true} hat die Option \option{omiteditor} keine
				Auswirkung. Der Standard für diese Option ist \wert{false}.
	\item Mit\beschreibung[edstringincitations]{edstring\-incitations}
	      der Option \option{edstringincitations=true} werden Herausgeber
	      (\texttt{edi\-tor}) und Übersetzer (\texttt{translator}) bei jedem
	      Literaturverweis mit den entsprechenden Kürzeln versehen, nicht nur
	      bei Vollzitaten. Das Erscheinungsbild richtet sich dabei nach der
	      Einstellung der Option \option{editorstring}. Der Standardwert für
	      diese Option ist \wert{true}.
	\item Wenn der Befehl \cmd{textcite} mit einem Eintrag benutzt wird, der
	      weder einen Autor (\texttt{author}) noch einen Herausgeber (\texttt{editor})
	      hat, gibt \bl{} eine Warnung aus und setzt den \BibTeX{}-Key des
	      Eintrags in fetter Schrift in den Text.
	\item Mit\beschreibung{firstfull}
	      der Option \option{firstfull} kann man für das erste Zitat die
	      volle Literaturangabe ausgeben lassen. Der Standard für diese Option ist
	      \wert{false}.
	\item Wenn\beschreibung{citedas}
	      es eine Sigle (\texttt{shorthand}) gibt, wird bei \option{firstfull=true}
	      dem Erstzitat der Zusatz \enquote{im Folgenden zit. als~\ldots} angefügt.
	      Dies kann durch die Option \option{citedas} beeinflusst werden.
	      Sie kann die Werte \wert{true} oder \wert{false} annehmen; 
	      der Standard ist \wert{true}. 
	      Diese Funktion kann auch für jeden Eintrag separat eingestellt werden mit
	      \texttt{options\,=\,\{citedas=true\}} oder 
	      \texttt{options\,=\,\{citedas=false\}}. Die globale Einstellung
	      wird dann für den entsprechenden Eintrag ignoriert.
  \item Mit\beschreibung{citepages}
	      der Option \option{citepages} lassen sich die Seitenangaben in Vollzitaten 
				unterdrücken, wenn das Feld \texttt{pages} vorhanden ist:
				\option{citepages=permit} erlaubt Duplikate, d.\,h. sowohl \texttt{pages}
				als auch \texttt{postnote} werden ausgegeben;
				mit \option{citepages=suppress} wird das \texttt{pages}"=Feld des Vollzitats
				in jedem Fall unterdrückt, es wird also nur die \texttt{postnote} ausgegeben (falls vorhanden);
				\option{citepages=omit} unterdrückt die Ausgabe von \texttt{pages} nur, wenn
				in der \texttt{postnote} eine Seitenzahl angegeben ist;
				\option{citepages=separate} gibt das \texttt{pages}"=Feld immer aus, trennt
				die \texttt{postnote} aber mit dem Begriff \enquote{hier} ab, falls es sich 
				um eine Seitenzahl handelt. Dazu wird der \emph{bibliography string} \enquote{thiscite}
				verwendet, der sich natürlich umdefinieren lässt. 		
				Der Standard für diese Option ist \wert{separate}.
				Dazu ein Beispiel, das die folgenden Zitierbefehle verwendet:
				\begin{lstlisting}
\cite{BibTeX-Key}
\cite[eine Anmerkung]{BibTeX-Key}
\cite[125]{BibTeX-Key}
				\end{lstlisting}
				%
				\option{citepages=permit}:
				\begin{quote}
				Autor: Titel, in: Buch, S.\,100--150.

				Autor: Titel, in: Buch, S.\,100--150, eine Anmerkung.

				Autor: Titel, in: Buch, S.\,100--150, S.\,125.
				\end{quote}
				%
				\option{citepages=suppress}:
				\begin{quote}
				Autor: Titel, in: Buch.

				Autor: Titel, in: Buch, eine Anmerkung.

				Autor: Titel, in: Buch, S.\,125.
				\end{quote}
				%
				\option{citepages=omit}:
				\begin{quote}
				Autor: Titel, in: Buch, S.\,100--150.

				Autor: Titel, in: Buch, S.\,100--150, eine Anmerkung.

				Autor: Titel, in: Buch, S.\,125.
				\end{quote}
				%
				\option{citepages=separate}:
				\begin{quote}
				Autor: Titel, in: Buch, S.\,100--150.

				Autor: Titel, in: Buch, S.\,100--150, eine Anmerkung.

				Autor: Titel, in: Buch, S.\,100--150, hier S.\,125.
				\end{quote}
	\item Die\beschreibung{citeauthor} Option \option{citeauthor} bestimmt das Format für den 
	      Befehl \cmd{citeauthor}; sie kann die Werte \wert{namefont}, \wert{namefontfoot} und 
	      \wert{normalfont} annehmen. Mit \option{citeauthor=namefont} wird das Format
	      benutzt, das mit der Option \option{namefont} eingestellt wurde. Dies ist die 
	      Standardeinstellung. Mit \option{citeauthor=normalfont} wird immer die normale 
	      Schrift für \cmd{citeauthor} benutzt, unabhängig von der Einstellung der Option 
	      \option{namefont}. Mit \option{citeauthor=namefontfoot} wird das 
	      \option{namefont}"=Format benutzt, wenn der \cmd{citeauthor}"=Befehl in einer 
	      Fußnote steht, ansonsten wird die normale Schrift benutzt.
  \item Mit\beschreibung{citeauthorname}
	      der Option \option{citeauthorname} hat man die Möglichkeit, die Namensform
	      beim Zitieren eines Autors\slash Herausgebers mit dem Befehl \cmd{citeauthor}
        oder mit dem Befehl \cmd{textcite} zu steuern. \option{citeauthorname=firstfull}
        bewirkt, dass beim ersten Zitieren der volle Name ausgegeben wird und bei allen 
        weiteren Zitaten nur der Nachname. Das funktioniert auch bei unterschiedlichen
        Literaturverweisen desselben Autors, so dass auch dann der volle Name nur beim 
        ersten Verweis ausgegeben wird. Dabei ist es egal, ob man nur \cmd{citeauthor}, 
        nur \cmd{textcite} oder beide Befehle gemischt einsetzt. Mit 
        \option{citeauthorname=full} wird immer der volle Name ausgegeben, mit 
        \option{citeauthorname=normal} wird immer nur der Nachname ausgegeben. 
        Der Standard für diese Option ist \wert{normal}.
	\item Mit\beschreibung{singletitle}
	      der \bl"=Option \option{singletitle=true} wird der Titel eines Werkes
	      nur dann ausgegeben, wenn mehr als ein Werk desselben Autors vorhanden ist.
				Für weitere Informationen sehen Sie bitte in der \bl"=Dokumentation nach.
\end{itemize}

\subsection[Sigelverzeichnis]{Sigelverzeichnis \emph{(List of Shorthands)}}
\begin{itemize}
	\item Das\beschreibung{terselos}
	      Sigelverzeichnis enthält in der Standardeinstellung lediglich 
	      Autor (oder Herausgeber), Titel und gegebenenfalls Buchtitel bzw. bei 
	      mehrbändigen Werken den Gesamttitel. Diese Angaben sollten ausreichen, 
	      um die vollständigen Daten in der Bibliographie zu finden. Dieses
	      Verhalten wird durch die Option \option{terselos} gesteuert, die
	      die Werte \wert{true} und \wert{false} annehmen kann; der Standard
	      ist \wert{true}.
	\item Mit\beschreibung{shorthandwidth}
	      der Option \option{shorthandwidth} kann man die Breite der Label
	      im Sigelverzeichnis angeben. Das ist vor allem nützlich, wenn man sehr
	      lange Sigel hat. Die Option kann alle gängigen Längenwerte annehmen,
	      also beispielsweise \wert{40pt} oder \wert{3em}. \achtung Wenn man die 
	      Option \option{shorthandwidth} benutzt, wird der Abstand nach dem Label
	      reduziert und gleichzeitig ein Doppelpunkt nach dem Label angefügt. 
	      Das Trennzeichen lässt sich durch den Befehl \cmd{shorthandpunct} 
	      umdefinieren, der Abstand wird durch die neue Länge \cmd{shorthandsep}
	      festgelegt. Die Standardwerte (sobald \option{shorthandwidth} benutzt
	      wird) sind:
				\begin{lstlisting}
\renewcommand{\shorthandpunct}{\addcolon}
\setlength{\shorthandsep}{3pt plus 0.5pt minus 0.5pt}
				\end{lstlisting}
\end{itemize}

\section{Der Stil \xbx{footnote-dw}}

Dieser Stil ähnelt dem Stil \xbx{verbose-inote}. Er basiert auf
\xbx{authortitle-dw}, so dass auch alle Optionen, die von \xbx{authortitle-dw} definiert 
werden, mit \xbx{footnote-dw} benutzt werden können, mit Ausnahme der Optionen 
\option{addyear}, \option{firstfull} und \option{inreference}. Daneben gibt es folgende 
Unterschiede:
\medskip
\begin{itemize}
	\item Zitate sind \emph{nur} innerhalb von Fußnoten möglich. Zitate
	      außerhalb von Fußnoten werden automatisch zu einem \cmd{footcite}
	      geändert. Einzige Ausnahme ist \cmd{textcite}, das im Text den
	      Namen und ein Zitat in der Fußnote ausgibt; verwendet man 
	      \cmd{textcite} in einer Fußnote, wird der Name ausgegeben und
	      das Zitat in Klammern dahinter gesetzt.
	\item Das erste Zitat eines Eintrags gibt die volle Literaturangabe aus, 
	      alle weiteren Zitate desselben Eintrags werden nur durch den
	      Autor (\texttt{author}) und den Kurztitel (\texttt{shorttitle}) 
	      repräsentiert, bzw. durch den Titel (\texttt{title}), falls kein
	      Kurztitel definiert wurde. Ergänzt wird die Angabe durch den
	      Zusatz \enquote{wie Anm.~\enquote{N}}, wobei \enquote{N} für
	      die Nummer der Fußnote steht, in der das Werk zuerst zitiert
	      wurde.
	\item Die\beschreibung{pageref}
	      Option \option{pageref}, die es auch in den \bl"=Stilen 
	      \xbx{verbose-note} und \xbx{ver\-bose-inote} gibt, ist ebenfalls
	      verfügbar. Setzt man sie auf \wert{true}, wird die Seite, auf der
	      das erste Zitat erscheint, zur Fußnotennummer, die auf das erste
	      Zitat verweist, hinzugefügt, sofern es sich nicht auf derselben
	      Seite befindet. Der Standard ist \wert{false}.
	\item Wird \cmd{parencite} außerhalb einer Fußnote benutzt, werden keine
	      Klammern gesetzt, sondern stattdessen der Befehl \cmd{footcite} 
	      ausgeführt. Innerhalb einer Fußnote funktioniert \cmd{parencite}
	      wie erwartet. Der Zusatz \enquote{wie Anm.~\ldots} wird dann von
	      eckigen Klammern eingeschlossen, nicht von runden Klammern.     
	\item Sind die Optionen \option{annotation} und \option{library} (oder eine
	      von ihnen) auf \wert{true} gesetzt, werden die Anmerkungen 
	      (\texttt{annotation}) und Bibliotheksinformationen (\texttt{library})
	      nur in der Bibliographie ausgegeben (falls eine vorhanden ist), nicht
	      aber im Erstzitat und in der Sigelliste.
\end{itemize}

\section{Die Querverweis-Funktionalität}
\label{xreffunctionality}
\subsection{Funktionsweise}
Mit\beschreibung{xref} der Querverweis"=Funktionalität von \bldw{} können unselbständige 
Schriften auf einen \enquote{Eltern"=Eintrag} verweisen.
Dazu wird in der bib-Datei ein Eltern"=Eintrag des Typs \texttt{@book}, \texttt{@collection}
oder \texttt{@proceedings} angelegt. Jeder \enquote{Kind"=Eintrag}, der sich auf diesen
Eltern"=Eintrag bezieht, verweist mit dem Feld \texttt{xref} auf den \BibTeX"=key des
Eltern"=Eintrags. Das funktioniert für Einträge der Typen \texttt{@inbook}, 
\texttt{@incollection} und \texttt{@inproceedings}.

Zur Veranschaulichung ein kleines Beispiel: 
\begin{lstlisting}
@collection{parent,
  editor = {(*\emph{Herausgeber}*)},
  title = {(*\emph{Buchtitel}*)},
  location = {(*\emph{Ort}*)},
  date = {2008}
}
@incollection{child,
  author = {(*\emph{Autor}*)},
  title = {(*\emph{Titel des Beitrags}*)},
  xref = {parent}% Verweis
}
\end{lstlisting}
Wenn nun der \texttt{@incollection}"=Eintrag zitiert und die Option \option{xref} auf
\wert{true} gesetzt wird, werden automatisch 
Daten aus dem Eintrag mit dem \BibTeX"=key \texttt{parent} übernommen. Dabei wird das Feld
\texttt{shorthand} genutzt, sofern es vorhanden ist. Ansonsten werden die Felder 
\texttt{author}\slash\texttt{editor} und \texttt{title} (bzw. \texttt{shorttitle},
falls vorhanden) ausgegeben. Damit wird der Leser auf den entsprechenden Eltern"=Eintrag
im Literaturverzeichnis verwiesen und hat alle relevanten Daten, ohne dass sie
mehrfach aufgeführt werden. 

Bei mehreren Kind"=Einträgen wäre es möglich, die Daten nicht mehrfach in die 
\BibTeX-Datei einzugeben (was auch die Möglichkeit von Tippfehlern reduzieren könnte). 
Allerdings sollte man sich bewusst sein, dass man dann \emph{immer} auf die
Verweistechnik angewiesen ist. Sollte man einmal ein Dokument benötigen, in dem
immer vollständige Einträge vorhanden sein sollen, würden die entsprechenden Daten
(\texttt{editor}, \texttt{booktitle} etc.) fehlen. Deshalb ist es besser, auch
bei \texttt{@incollection}, \texttt{@inbook} und \texttt{@inproceedings} immer
die vollständigen Daten einzugeben und auf den Eltern"=Eintrag selbst mit dem
\texttt{xref}"=Feld zu verweisen. 

Das \texttt{xref}"=Feld beachtet die Paketoption \option{mincrossrefs}, 
die in der Standardeinstellung auf \wert{2} steht. Das heißt, wenn aus einem
Sammelband nur ein einziger Beitrag zitiert wird, würde der Sammelband selbst nicht in das
Literaturverzeichnis aufgenommen werden (es sei denn, er würde selbst explizit 
zitiert) und die Literaturangabe des Beitrags wäre nicht vollständig. Aus diesem Grund
wird bei \option{xref=true} die Option \option{mincrossrefs} auf \wert{1} gesetzt.

In der Standardeinstellung (oder mit \option{xref=false}) wird diese Verweistechnik 
nicht genutzt. Das \texttt{xref}-Feld kommt in diesem Fall nur insoweit zum Tragen, dass ein
Eltern"=Eintrag nur dann als eigenständiger Eintrag im Literaturverzeichnis aufgeführt 
wird, wenn mindestens zwei seiner Kind"=Einträge zitiert wurden (\option{mincrossrefs=2}).
Der Wert für \option{mincrossrefs} lässt sich natürlich individuell anpassen.

\achtung{}Der Verweis funktioniert nur mit dem \texttt{xref}"=Feld. Mit dem Feld 
\texttt{crossref} arbeitet diese Verweistechnik \emph{nicht} zusammen! Anders als 
bei der Benutzung des Feldes \texttt{crossref} in herkömmlichem \BibTeX{} 
werden die fehlenden Felder nämlich nicht einfach in den Kind"=Eintrag importiert. 
Stattdessen wird ein spezieller Zitierbefehl ausgeführt, der die entsprechenden 
Daten des Eltern"=Eintrags liefert.

\subsection{Besonderheiten}
\subsubsection{Verwendung mit \xbx{footnote-dw}}
Die Querverweis"=Funktionalität arbeitet auch mit dem Stil \xbx{footnote-dw} zusammen. 
Wenn man die Option \option{xref=true} verwendet, wird beim ersten Zitat des
Eltern"=Eintrags ein \cmd{label} gesetzt. Dabei spielt es keine Rolle, ob der 
Eltern"=Eintrag selbst oder durch einen Kind"=Eintrag zitiert wurde. Sobald ein
(zweiter) Kind"=Eintrag zitiert wird, wird ein Kurzzitat des Eltern"=Eintrags mit einem Verweis 
auf die Fußnote des ersten Zitats, in dem der Eltern"=Eintrag vorkam, ausgegeben.

\subsubsection{Mehrbändige \enquote{Eltern}}
Wenn ein Eltern"=Eintrag ein mehrbändiges Werk ist, bezieht sich der 
Kind"=Eintrag normalerweise auf einen bestimmten Band des Eltern"=Eintrags und
nicht auf den gesamten Eintrag. Um das zu berücksichtigen, wird zunächst
überprüft, ob im Kind-Eintrag das Feld \texttt{volume} vorhanden ist. Wenn dies
der Fall ist, wird überprüft, ob im Eltern"=Eintrag das Feld \texttt{volume}
ebenfalls vorhanden ist. Nur wenn das nicht der Fall ist, wird die Bandangabe
des Kind"=Eintrags unmittelbar vor der Seitenangabe ausgegeben.%
\footnote{Der Grund, warum auf das bloße Vorhandensein
des \texttt{volume}"=Feldes und nicht auf die Übereinstimmung geprüft wird, ist folgender:
Sobald das Feld \texttt{volume} des Eltern"=Eintrags vorhanden ist, handelt es sich
um einen Einzelband eines mehrbändigen Werkes. Ist nun beim Kind"=Eintrag ein anderer
Band im \texttt{volume}"=Feld angegeben, wird es sich um einen Fehler handeln. Mir ist
jedenfalls kein Szenario eingefallen, bei dem ein Kind"=Eintrag eine andere Bandangabe
erforderte als der dazugehörige Eltern"=Eintrag.} In diesem Fall wird gleichzeitig 
geprüft, ob die Angaben im Feld \texttt{date} übereinstimmen. Ist dies nicht der Fall
(wenn etwa das mehrbändige Werk über mehrere Jahre erschienen ist), wird zusätzlich
zur Bandangabe auch das Erscheinungsjahr ausgegeben. Außerdem wird dann schließlich
auch noch überprüft, ob die Angaben im Feld \texttt{location} (oder \texttt{address})
übereinstimmen. Tun sie das nicht (etwa wenn das mehrbändige Werk an unterschiedlichen
Orten erschienen ist, der konkrete Band, auf den sich der Kind"=Eintrag bezieht, aber nur
in einem Ort), wird vor dem Erscheinungsjahr auch noch der Erscheinungsort ausgegeben.

\section{Die Optionen im Überblick}

\subsection{Globale Optionen}
Globale Optionen gelten für alle Literaturverweise im Dokument; sie werden entweder 
als Option beim Laden von \bl{} oder in einer Konfigurationsdatei (\texttt{biblatex.cfg}) 
gesetzt. Der Wert in Klammern gibt den Standard an.
	\optlist{acronyms}{false}
	  Nur wenn man die Option auf \wert{true} setzt, wird die Eintragsoption
	  \option{acronym} beachtet.
	\optlist[nur \xbx{authortitle-dw}]{addyear}{false}
	  Setzt man die Option auf \wert{true}, wird in Literaturverweisen nach dem Titel
	  das Jahr in Klammern hinzugefügt.
	\optlist{annotation}{false}
	  Das Feld \texttt{annotation} wird am Ende des Eintrags in der 
	  Bibliographie ausgegeben.
	\optlist{citeauthor}{namefont}
	  Legt den Schriftschnitt der Autorennamen bei Verwendung des Befehls \cmd{citeauthor}
	  fest. Mögliche Werte sind \wert{namefont}, \wert{normalfont} und 
	  \wert{namefontfoot}.
	\optlist{citeauthorname}{normal}
	  Legt fest, ob bei \cmd{citeauthor} oder \cmd{textcite} nur der Nachname oder der volle 
	  Name ausgegeben wird. Mögliche Werte sind \wert{normal} (nur Nachname), \wert{full}
	  (immer voller Name) und \wert{firstfull} (beim ersten Zitat voller Name, danach nur 
	  Nachname).
	\optlist{citedas}{true}
	  Bei Einträgen mit \texttt{shorthand} wird dem Erstzitat (bei
	  \xbx{authortitle-dw} nur mit der Option \option{firstfull}) ein
	  \enquote{im Folgenden zit. als~\ldots} angefügt.
	\optlist{citepages}{separate}
	  Legt fest, ob bei einem Voll- oder Erstzitat (bei \xbx{authortitle-dw} nur mit der 
	  Option \option{firstfull}) eines Eintrags mit \texttt{pages}"=Feld die Seitenangaben 
	  ausgegeben werden oder nicht.
	\optlist{edbyidem}{true}
	  \enquote{hg.\,v.\,dems.} statt \enquote{hg.\,v. \emph{Herausgeber}}.
	\optlist{editionstring}{false}
    Setzt den Ausdruck \enquote{Aufl.} immer hinter die Edition, unabhängig vom Inhalt
    des Feldes \texttt{edition}.
	\optlist{editorstring}{parens}
	  Setzt den Ausdruck \enquote{Hrsg.} (und bei \option{usetranslator=true}
	  auch \enquote{Übers.}) in runde Klammern (\wert{parens}) oder
	  in eckige Klammern (\wert{brackets}). Mit dem Wert \wert{normal} wird der
	  Ausdruck \enquote{Hrsg.} nach dem Namen des Herausgebers und einem
	  angehängten Komma angefügt.
	\optlist{editorstringfont}{normal}
    Schriftart, die für den Ausdruck \enquote{Hrsg.} (und bei \option{usetranslator=true}
	  auch \enquote{Übers.}) benutzt wird: normale Schrift (\wert{normal}) oder
    die Schrift, die mit der Option \option{namefont} gewählt wurde (\wert{namefont}).
	\optlist{edstringincitations}{true}
	  Setzt im Zitat den Ausdruck \enquote{Hrsg.} (und bei \option{usetranslator=true}
	  auch \enquote{Übers.}) hinter den Herausgeber bzw. Übersetzer.
  \optlist{edsuper}{false}
    Die Auf"|lage (\texttt{edition}) wird als hochgestellte Zahl direkt vor
    dem Jahr ausgegeben.
  \optlist[nur \xbx{authortitle-dw}]{firstfull}{false}
    Beim ersten Zitat wird die volle Literaturangabe ausgegeben.
	\optlist{firstnamefont}{normal}
	  Legt den Schriftschnitt der Vornamen von Autoren und Editoren fest, ebenso 
	  von Namensaffixen und (mit der Option \option{useprefix=false}) 
	  Namenspräfixen.
	  Mögliche Werte sind \wert{smallcaps} (Kapitälchen), \wert{italic} (kursiv),
	  \wert{bold} (fett) und \wert{normal} (Voreinstellung~-- das heißt,
	  der normale Schriftschnitt wird verwendet).
	\optlist{ibidemfont}{normal}
	  Legt den Schriftschnitt von \enquote{ebenda}\slash\enquote{ebd.} fest.
	  Mögliche Werte sind \wert{smallcaps} (Kapitälchen), \wert{italic} (kursiv),
	  \wert{bold} (fett) und \wert{normal} (Voreinstellung~-- das heißt,
	  der normale Schriftschnitt wird verwendet).
	\optlist{idembib}{true}
	  \enquote{Ders.}\slash\enquote{Dies.} oder \enquote{---} statt Namen bei 
	  aufeinanderfolgenden Einträgen derselben Autoren\slash Herausgeber in der 
	  Bibliographie.
	\optlist{idembibformat}{idem}
	  Nur bei Benutzung von \option{idembib=true}: Mit \wert{idem} werden die 
	  Namen durch \enquote{Ders.}\slash\enquote{Dies.} ersetzt, mit \wert{dash}
	  durch einen langen Strich (\enquote{---}).
	\optnur[kein Standard gesetzt]{idemfont}
	  Legt den Schriftschnitt von \enquote{Ders.}\slash\enquote{Dies.} fest.
	  Mögliche Werte sind \wert{smallcaps} (Kapitälchen), \wert{italic} (kursiv),
	  \wert{bold} (fett) und \wert{normal}. Wird die Option nicht gesetzt, so 
	  wird der Schriftschnitt von der Option \option{namefont} übernommen (das 
	  ist auch die Voreinstellung).
	\optlist[nur \xbx{authortitle-dw}]{inreference}{normal}
	  Mit \wert{inreference=full} werden \texttt{@inreference}"=Einträge in Literaturverweisen
	  vollständig ausgegeben, aber nicht in die Bibliographie aufgenommen.
	\optlist{journalnumber}{standard}
	  Position der Heftnummer (\texttt{number}) einer Zeitschrift: bei
	  \wert{standard} wie in den Standard"=Stilen, bei \wert{afteryear} nach dem
	  Jahr (\texttt{year}\slash \texttt{date}), eingeleitet durch den \emph{bibliography string} \enquote{number}
	  (\enquote{Nr.}), und bei \wert{date} in Abhängigkeit von den Datumsangaben
	  (siehe Abschnitt~\ref{journalnumberdate} auf
	  Seite~\pageref{journalnumberdate}).
	\optlist{library}{false}
	  Das Feld \texttt{library} wird am Ende des Eintrags in der Bibliographie 
	  ausgegeben.
	\optlist{namefont}{normal}
	  Legt den Schriftschnitt der Nachnamen von Autoren und Editoren fest, 
	  ebenso von Namenspräfixen (mit der Option \option{useprefix=true}).
	  Mögliche Werte sind \wert{smallcaps} (Kapitälchen), \wert{italic} (kursiv),
	  \wert{bold} (fett) und \wert{normal} (Voreinstellung~-- das heißt,
	  der normale Schriftschnitt wird verwendet).
	\optlist{nopublisher}{true}
	  Der Verlag wird nicht ausgegeben.
	\optlist{nolocation}{false}
	  Setzt man die Option auf \wert{true}, wird der Ort nicht ausgegeben. 
	  Außerdem wird dann auch der Verlag nicht ausgegeben,
	  selbst wenn \option{nopublisher} auf \wert{false} gesetzt wird.
	\optlist{oldauthor}{true}
	  Setzt man die Option auf \wert{false}, werden die Eintragsoptionen
	  \option{oldauthor} und \option{oldbookauthor} ignoriert.
	\optlist{omiteditor}{false}
	  Unterdrückt die Ausgabe des Herausgebers in Literaturverweisen.
	\optlist{origfields}{true}
	  Mit \option{origfields=true} werden die Felder \texttt{origlocation} und 
	  \texttt{origdate} (sowie \texttt{origpublisher}, wenn
	  \option{nopublisher=false} eingestellt ist) ausgegeben.
	\optlist{origfieldsformat}{punct}
	  Mit dieser Option kann man einstellen, wie die Angaben zum Nachdruck (mit 
	  \option{origfields=true}) ausgegeben werden sollen: in runden
	  (\wert{parens}) oder eckigen (\wert{brackets}) Klammern, oder eingeleitet 
	  durch \cmd{origfieldspunct} (Voreinstellung ist ein Komma).
	\optlist[nur \xbx{footnote-dw}]{pageref}{false}
	  Zusätzlich zur Fußnotennummer wird auch auf die Seitenzahl des ersten Zitats
	  verwiesen.
	\optlist{pagetotal}{false}
	  Das Feld \texttt{pagetotal} wird ausgegeben oder die Ausgabe unterdrückt.
	\optlist{pseudoauthor}{true}
	  Setzt man die Option auf \wert{false}, werden Autoren von Einträgen mit der Eintragsoption
		\option{pseudoauthor} \emph{nicht} ausgegeben.
	\optlist{series}{standard}
	  Position der Reihe (\texttt{series}) eines Werkes: bei \wert{standard} wie
	  in den Standard"=Stilen, bei \wert{afteryear} nach dem Jahr und bei \wert{beforeedition} vor der Auf"|lage
	  (\texttt{year}\slash \texttt{date}).
	\optlist{seriesformat}{parens}
    Format der Reihe (\texttt{series}) eines Werkes: bei \wert{standard} wie in
    den Standard"=Stilen, bei \wert{parens} in Klammern.
	\optlist{shorthandibid}{true}
	  Direkte Folgezitate von Einträgen mit \texttt{shorthand} werden durch 
	  \enquote{ebd.} ersetzt.
	\optlist{shorthandinbib}{false}
	  Setzt man die Option auf \wert{true}, werden die Siglen (\texttt{shorthand})
		im Literaturverzeichnis direkt vor dem Eintrag ausgegeben.
	\optnur[kein Standard gesetzt]{shorthandwidth}
	  Legt die Breite der Label im Sigelverzeichnis \emph{(List of Shorthands)} 
	  fest. Gleichzeitig werden nach jedem Label \cmd{shorthandsep}
	  (standardmäßig 3pt) und \cmd{shorthandpunct} (standardmäßig ein
	  Doppelpunkt) ausgeführt.
	\optlist{shortjournal}{false}
    Mit \option{shortjournal=true} wird das Feld \texttt{shortjournal} statt des 
    \texttt{journaltitle} ausgewertet. Falls \texttt{shortjournal} nicht 
    vorhanden ist, wird \texttt{journaltitle} (und ggf. \texttt{journalsubtitle})
    verwendet.
	\optlist{singletitle}{false}
    Setzt man die Option auf \wert{true}, wird der Titel in Literaturverweisen
		unterdrückt, falls nicht mehr als ein Werk desselben Autors vorhanden ist.
		Vollzitate sind davon ausgenommen.
	\optlist{terselos}{true}
	  Es wird ein knapperes Sigelverzeichnis \emph{(List of Shorthands)} 
	  ausgegeben.
	\optlist{xref}{false}
	  Die Querverweis"=Funktionalität wird verwendet und die Option \option{mincrossrefs}
		auf den Wert \wert{1} gesetzt. Genaueres dazu findet sich 
	  im Abschnitt~\ref{xreffunctionality} auf Seite~\pageref{xreffunctionality}.

\subsection{Optionen für einzelne Einträge}
Eintragsoptionen werden für jeden einzelnen Eintrag im Feld \texttt{options} gesetzt. 
Sie überschreiben für den jeweiligen Eintrag eventuell gesetzte globale Optionen.
	\opt{acronym}
	  Die Siglen (\texttt{shorthand}, bei \option{shortjournal=true} auch 
	  \texttt{shortjournal}) werden mit dem Befehl \cmd{mkbibacro} gesetzt,
	  sofern die globale Option \option{acronyms} aktiviert ist.
	\opt{citedas}
	  Bei Einträgen mit \texttt{shorthand} nach dem Erstzitat (bei 
	  \xbx{authortitle-dw} nur mit der Option \option{firstfull}) das
	  angefügte \enquote{im Folgenden zit. als~\ldots} erzwingen (\wert{true})
	  bzw. unterdrücken (\wert{false}).
	\opt{oldauthor}
	  Der Autor (Feld \texttt{author}) wird nicht in dem mit \option{namefont} gewählten Schriftschnitt 
	  gesetzt, sofern die globale Option \option{oldauthor} aktiviert ist.
	\opt{oldbookauthor}
	  Der Buchautor (Feld \texttt{bookauthor}) wird nicht in dem mit \option{namefont} gewählten Schriftschnitt 
	  gesetzt, sofern die globale Option \option{oldauthor} aktiviert ist.
	\opt{pseudoauthor}
	  Der Autor wird von \cmd{bibleftpseudo} und \cmd{bibrightpseudo} umgeben, 
		sofern die globale Option \option{pseudoauthor} auf \wert{true} gesetzt ist.
		Bei globaler Option \option{pseudoauthor=false} wird der Autor von Einträgen
		mit der Eintragsoption \option{pseudoauthor=true} gar nicht ausgegeben.
	\opt{shorthandibid}
	  Bei Einträgen mit Sigle (\texttt{shorthand}) wird~-- unabhängig von der 
	  globalen Option \option{shorthandibid}~-- bei direkt aufeinanderfolgenden 
	  Wiederholungszitaten die Sigle durch \enquote{ebd.} ersetzt (\wert{true})
	  bzw. nicht ersetzt (\wert{false}).

\subsection{\texorpdfstring{\bl"=}{biblatex-}Optionen}
Im Folgenden werden Optionen aufgeführt, die von \bl{} zur Verfügung gestellt und von \bldw{} lediglich auf einen bestimmten Wert voreingestellt werden. Nähere Informationen zu diesen Optionen sind der \bl"=Dokumentation zu entnehmen.

	\optset{autocite}{footnote}
	  Für den Befehl \cmd{autocite} wird ein \cmd{footcite} ausgeführt.
  \optset{citetracker}{true}
    Der \emph{citation tracker}, der überprüft, ob ein Werk bereits zitiert
    wurde, wird global aktiviert.
	\optset{doi}{false}
	  Das Feld \texttt{doi} wird ausgegeben oder die Ausgabe unterdrückt.
	\optset{eprint}{false}
	  Das Feld \texttt{eprint} wird ausgegeben oder die Ausgabe unterdrückt.
	\optset{ibidtracker}{constrict}
	  Bei direkten Folgezitaten desselben Werks wird \enquote{ebd.} ausgegeben.
	  Dabei werden Text und Fußnoten unabhängig voneinander behandelt.
	\optset{idemtracker}{constrict}
	  Bei direkten Folgezitaten desselben Autors wird dessen Name durch 
	  \enquote{ders.} ersetzt. Dabei werden Text und Fußnoten unabhängig
	  voneinander behandelt.
	\optset{isbn}{false}
	  Die Felder \texttt{isbn}, \texttt{isrn} und \texttt{issn} werden
		ausgegeben oder die Ausgabe unterdrückt.
  \optset{loccittracker}{false}
    Der \emph{\enquote{loccit} tracker}, der überprüft, ob die Stelle eines
    Werks dieselbe ist wie die des zuletzt von diesem Autor zitierten Werks,
    wird ausgeschaltet.  
  \optset{opcittracker}{false}
    Der \emph{\enquote{opcit} tracker}, der überprüft, ob das Werk dasselbe
    ist wie das letzte von diesem Autor zitierte Werk, wird ausgeschaltet.
  \optset{pagetracker}{true}
    Der \emph{page tracker} wird eingeschaltet; bei einseitigen Dokumenten
    prüft er auf einzelne Seiten, bei zweiseitigen Dokumenten auf 
    Doppelseiten. Die internen Tests \cmd{iffirstonpage} und \cmd{ifsamepage} 
    machen Gebrauch von dieser Einstellung.
%  \optset[footnote-dw]{uniquename}{false}
%    Es werden immer Vor- und Nachnamen ausgegeben.

\subsection{Die Option \option{journalnumber=date}}
\label{journalnumberdate}
Anstatt viele Worte zu verlieren, um zu beschreiben, wie die Option 
funktioniert, werden im Folgenden einfach entsprechende Beispiele
gezeigt. Dabei erscheint zunächst immer der \BibTeX"=Eintrag und darunter die
entsprechende Ausgabe mit \option{journalnumber=date}. Die Beispiele wurden
von Bernhard Tempel zusammengestellt.

\begin{lstlisting}
@ARTICLE{Fingiert:1939,
  author = {Anonym},
  title = {Gegen Mißbrauch der Genußgifte},
  journal = {Hannoverscher Kurier},
  volume = {91},
  number = {65},
  issue = {Morgen-Ausg\adddot},
  pages = {2},
  date = {1939-03-06}}
\end{lstlisting}
\fullcite{Fingiert:1939}

\begin{lstlisting}
@ARTICLE{Fingiert:1939a,
  author = {Anonym},
  title = {Gegen Mißbrauch der Genußgifte},
  journal = {Hannoverscher Kurier},
  volume = {91},
  number = {65},
  issue = {Morgen-Ausg\adddot},
  pages = {2},
  date = {1939-03}}
\end{lstlisting}
\fullcite{Fingiert:1939a}

\begin{lstlisting}
@ARTICLE{Gerstmann:2007a,
  author = {Gerstmann, Günter},
  title = {Gerhart Hauptmann-Aktivitäten in Hohenhaus},
  journal = {Schlesischer Kulturspiegel},
  date = {2007},
  volume = {42},
  number = {1},
  pages = {13},
  issue = {Januar--März}}
\end{lstlisting}
\fullcite{Gerstmann:2007a}

\begin{lstlisting}
@ARTICLE{GMG:1939,
  author = {Anonym},
  title = {Gegen Mißbrauch der Genußgifte},
  journal = {Hannoverscher Kurier},
  volume = {91},
  number = {65},
  pages = {2},
  date = {1939-03-06}}
\end{lstlisting}
\fullcite{GMG:1939}

\begin{lstlisting}
@ARTICLE{Guilford:1950,
  author = {Guilford, J[oy] P[aul]},
  title = {Creativity},
  journal = {The American Psychologist},
  date = {1950-09},
  volume = {5},
  number = {9},
  pages = {444--454}}
\end{lstlisting}
\fullcite{Guilford:1950}

\begin{lstlisting}
@ARTICLE{Page:1997,
  author = {Page, Penny Booth},
  title = {E.\,M. Jellinek and the evolution of alcohol studies},
  subtitle = {A critical essay},
  journal = {Addiction},
  date = {1997},
  volume = {92},
  number = {12},
  pages = {1619-1637}}
\end{lstlisting}
\fullcite{Page:1997}

\begin{lstlisting}
@ARTICLE{Fingiert:1939b,
  author = {Anonym},
  title = {Gegen Mißbrauch der Genußgifte},
  journal = {Hannoverscher Kurier},
  number = {65},
  issue = {Morgen-Ausg\adddot},
  pages = {2},
  date = {1939-03-06}}
\end{lstlisting}
\fullcite{Fingiert:1939b}

\begin{lstlisting}
@ARTICLE{Fingiert:1939c,
  author = {Anonym},
  title = {Gegen Mißbrauch der Genußgifte},
  journal = {Hannoverscher Kurier},
  volume = {91},
  issue = {Morgen-Ausg\adddot},
  pages = {2},
  date = {1939-03}}
\end{lstlisting}
\fullcite{Fingiert:1939c}


\begin{lstlisting}
@ARTICLE{Ewers:1906,
  author = {Ewers, Hanns Heinz},
  title = {Rausch und Kunst},
  journal = {Blaubuch},
  date = {1906},
  volume = {1},
  pages = {1726-1730},
  issue = {4. Quartal},
}
\end{lstlisting}
\fullcite{Ewers:1906}

\begin{lstlisting}
@ARTICLE{Fingiert:1939d,
  author = {Anonym},
  title = {Gegen Mißbrauch der Genußgifte},
  journal = {Hannoverscher Kurier},
  volume = {91},
  pages = {2},
  date = {1939-03-13}}
\end{lstlisting}
\fullcite{Fingiert:1939d}

\begin{lstlisting}
@ARTICLE{Fingiert:1939e,
  author = {Anonym},
  title = {Gegen Mißbrauch der Genußgifte},
  journal = {Hannoverscher Kurier},
  volume = {91},
  pages = {2},
  date = {1939-03}}
\end{lstlisting}
\fullcite{Fingiert:1939e}

\begin{lstlisting}
@ARTICLE{Landolt:2000,
  author = {Landolt, H. P. and Borbély, A. A.},
  title = {Alkohol und Schlafstörungen},
  journal = {Therapeutische Umschau},
  date = {2000},
  volume = {57},
  pages = {241-245},
}
\end{lstlisting}
\fullcite{Landolt:2000}

\begin{lstlisting}
@ARTICLE{Chapiro:1930,
  author = {Chapiro, Joseph},
  title = {Das neueste Werk Gerhart Hauptmanns},
  subtitle = {\enquote{Die Spitzhacke}},
  journal = {Neue Freie Presse},
  number = {23773},
  pages = {1-3},
  issue = {Morgenblatt},
  date = {1930-11-19},
}
\end{lstlisting}
\fullcite{Chapiro:1930}

\begin{lstlisting}
@ARTICLE{Fingiert:1939f,
  author = {Anonym},
  title = {Gegen Mißbrauch der Genußgifte},
  journal = {Hannoverscher Kurier},
  number = {65},
  pages = {2},
  date = {1939-03}}
\end{lstlisting}
\fullcite{Fingiert:1939f}

\begin{lstlisting}
@ARTICLE{Barski:2007,
  author = {Barski, Jacek and Mahnken, Gerhard},
  title = {Museumsverbund Gerhart Hauptmann},
  subtitle = {Ein deutsch-polnisches Kulturprojekt mit Weitblick},
  journal = {Kulturpolitische Mitteilungen},
  date = {2007},
  number = {119},
  pages = {62},
  issue = {IV},
}
\end{lstlisting}
\fullcite{Barski:2007}

\begin{lstlisting}
@ARTICLE{Essig:2005,
  author = {Essig, Rolf-Bernhard},
  title = {Mit liebender Schafsgeduld},
  subtitle = {Erhart Kästner im Dienste Gerhart Hauptmanns},
  journal = {Süddeutsche Zeitung},
  number = {237},
  pages = {16},
  date = {2005-10-14},
}
\end{lstlisting}
\fullcite{Essig:2005}

\begin{lstlisting}
@ARTICLE{Kluwe:2007,
  author = {Kluwe, Sandra},
  title = {Furor poeticus},
  subtitle = {Ansätze zu einer neurophysiologisch fundierten Theorie der literarischen Kreativität am Beispiel der Produktionsästhetik Rilkes und Kafkas},
  journal = {literaturkritik.de},
  date = {2007-02},
  number = {2},
  url = {http://literaturkritik.de/public/rezension.php?rez_id=10438},
}
\end{lstlisting}
\fullcite{Kluwe:2007}

\begin{lstlisting}
@ARTICLE{Burckhardt:2006,
  author = {Burckhardt, Barbara},
  title = {Frauen sind einfach klüger, starke Frauen},
  subtitle = {Michael Thalheimers \enquote{Rose	Bernd} am Hamburger Thalia Theater und Schirin Khodadadians Kasseler Räuber},
  journal = {Theater heute},
  date = {2006},
  number = {5},
  pages = {14-18},
}
\end{lstlisting}
\fullcite{Burckhardt:2006}

\begin{lstlisting}
@ARTICLE{Ossietzky:1922,
  author = {Ossietzky, Carl von},
  title = {Moritz Heimann \enquote{Armand Carrel} Staatstheater},
  journal = {Berliner Volks-Zeitung},
  date = {1922-03-30},
  issue = {Abend-Ausg\adddot}
}
\end{lstlisting}
\fullcite{Ossietzky:1922}

\begin{lstlisting}
@ARTICLE{Fingiert:1939g,
  author = {Anonym},
  title = {Gegen Mißbrauch der Genußgifte},
  journal = {Hannoverscher Kurier},
  issue = {Abend-Ausgabe},
  pages = {2},
  date = {1939-03}}
\end{lstlisting}
\fullcite{Fingiert:1939g}

\begin{lstlisting}
@ARTICLE{Weiss:1960,
  author = {Weiss, Grigorij},
  title = {Auf der Suche nach der versunkenen Glocke},
  subtitle = {Johannes R. Becher bei Gerhart Hauptmann},
  journal = {Sinn und Form},
  date = {1960},
  pages = {363--385},
  issue = {Zweites Sonderheft Johannes R. Becher},
}
\end{lstlisting}
\fullcite{Weiss:1960}

\begin{lstlisting}
@ARTICLE{Hofer:2006,
  author = {Hofer, Hermann},
  title = {Der Schrei der Verwundeten},
 subtitle = {Erschütternd: Gerhart Hauptmanns \enquote{Rose Bernd} am Hamburger Thalia Theater},
  journal = {Lübecker Nachrichten},
  date = {2006-03-14}}
\end{lstlisting}
\fullcite{Hofer:2006}

\begin{lstlisting}
@ARTICLE{Kammerhoff:2006,
  author = {Kammerhoff, Heiko},
  title = {Rose Bernd},
  journal = {Szene Hamburg},
  date = {2006-04}}
\end{lstlisting}
\fullcite{Kammerhoff:2006}

\begin{lstlisting}
@ARTICLE{Fingiert:1939h,
  author = {Anonym},
  title = {Gegen Mißbrauch der Genußgifte},
  journal = {Hannoverscher Kurier},
  pages = {2},
  date = {1939}}
\end{lstlisting}
\fullcite{Fingiert:1939h}

\section{Befehle, Begriffe, Eintragstypen, Feldformate}
\subsection{Zusätzliche Befehle}
Im Folgenden werden die von \bldw{} zur Verfügung gestellten Befehle mit ihrer Standarddefinition aufgelistet. Diese Befehle können mit \cmd{renewcommand*} angepasst werden.

%	\befehl{}{}{}
% \befehlleer{}{}
	\befehl{annotationfont}{\cmd{small}\cmd{itshape}}{Schrift des Feldes
	  \texttt{annotation}.}
	\befehl{bibfinalnamedelim}{%
	  \cmd{ifnum}\cmd{value}\{liststop\}\textgreater 2\%\\
	  \hspace*{8.1em}\cmd{finalandcomma}\cmd{fi}\%\\
	  \hspace*{8.1em}\cmd{addspace}\cmd{bibstring\{and\}}\cmd{space}}{Begrenzer
	  zwischen dem vorletzten und letzten Namen bei Literaturangaben in der
	  Bibliographie. Vergleiche \cmd{finalnamedelim} von \bl.}
	\befehlleer{bibleftpseudo}{Zeichen vor dem Autor bei Benutzung von 
	  \option{pseudoauthor=true}.}
	\befehl{bibmultinamedelim}{\cmd{addcomma}\cmd{space}}{Begrenzer
	  zwischen Namen bei Literaturangaben in der Bibliographie. Vergleiche
	  \cmd{multinamedelim} von \bl.}
	\befehl{bibrevsdnamedelim}{\cmd{addspace}}{Zusätzliches Zeichen zwischen dem 
	  ersten und zweiten Namen bei Literaturangaben in der Bibliographie beim
	  Schema \enquote{Nachname, Vorname, Vorname Nachname}. Das Komma ist damit 
	  nicht gemeint! Vergleiche \cmd{revsdnamedelim} von \bl.}
	\befehlleer{bibrightpseudo}{Zeichen nach dem Autor bei Benutzung von 
	  \option{pseudoauthor=true}.}
	\befehl{citefinalnamedelim}{\cmd{slash}}{Begrenzer zwischen dem vorletzten 
	  und letzten Namen in Zitaten. Vergleiche \cmd{finalnamedelim} von \bl.}
	\befehl{citemultinamedelim}{\cmd{slash}}{Begrenzer zwischen Namen in Zitaten.
	  Vergleiche \cmd{multinamedelim} von \bl.}
	\befehlleer{citerevsdnamedelim}{Zusätzliches Zeichen zwischen
	  dem ersten und zweiten Namen in Zitaten beim Schema \enquote{Nachname,
	  Vorname, Vorname Nachname}. Das Komma ist damit nicht gemeint! Vergleiche
	  \cmd{revsdnamedelim} von \bl.}
	\befehl{journumstring}{\cmd{addcomma}\cmd{space}\cmd{bibstring\{number\}}%
	  \cmd{addnbspace}}{Zeichen\slash Begriff vor der Heftnummer 
	  (\texttt{number}) einer Zeitschrift.}
	\befehl{jourvolnumsep}{\cmd{adddot}}{Zeichen zwischen Band und Heftnummer
	  einer Zeitschrift (bei \option{journumafteryear=false}).}
	\befehl{jourvolstring}{\cmd{addspace}}{Zeichen\slash Begriff vor dem
	  Jahrgangsband (\texttt{volume}) einer Zeitschrift.}
	\befehl{libraryfont}{\cmd{small}\cmd{sffamily}}{Schrift des Feldes
	  \texttt{library}.}
	\befehl{locationdatepunct}{\cmd{addspace}}{Zeichen zwischen Ort (\texttt{location})
	  und Jahr (\texttt{year}\slash \texttt{date}) bei \option{nopublisher=true} oder
		wenn kein Verlag angegeben wurde.}
	\befehl{locationpublisherpunct}{\cmd{addcolon}\cmd{space}}{Zeichen zwischen Ort (\texttt{location})
	  und Verlag (\texttt{publisher}) bei \option{nopublisher=false}.}
  \befehl{origfieldspunct}{\cmd{addcomma}\cmd{space}}{Zeichen vor dem
    Nachdruck bei \option{origfields=true} und
    \option{origfieldsformat=punct}.}
	\befehl{publisherdatepunct}{\cmd{addcomma}\cmd{space}}{Zeichen zwischen Verlag (\texttt{publisher})
	  und Jahr (\texttt{year}\slash \texttt{date}) bei \option{nopublisher=false}.}
	\befehlleer{seriespunct}{Zeichen vor dem Reihentitel (\texttt{series}),
	  innerhalb der Klammer.}
	\befehl{sernumstring}{\cmd{addspace}}{Zeichen\slash Begriff zwischen dem
	  Reihentitel (\texttt{series}) und der Nummer (\texttt{number}).}
	\befehl{shorthandinbibpunct}{\cmd{addspace}}{Zeichen nach einer Sigle im Literaturverzeichnis, wenn
	  \option{shorthandinbib} benutzt wird.}
	\befehl{shorthandpunct}{\cmd{addcolon}}{Zeichen nach einer Sigle im Sigelverzeichnis, wenn
	  \option{shorthandwidth} benutzt wird.}
  \befehl{shorthandsep}{3pt plus 0.5pt minus 0.5pt}{Länge zwischen Sigle und
    Siglenbeschreibung, wenn \option{shorthandwidth} benutzt wird.}
  \befehl{textcitesdelim}{\cmd{addspace}\cmd{bibstring\{and\}}\cmd{space}}{Trenner zwischen
    mehreren Autoren bei Verwendung von \cmd{textcites}.}
  \befehl{titleaddonpunct}{\cmd{addperiod}\cmd{space}}{Zeichen vor dem
	  Titelzusatz (\texttt{titleaddon}, \texttt{booktitleaddon},
	  \texttt{maintitleaddon}).}
  \befehl{titleyeardelim}{\cmd{addspace}}{Zeichen zwischen Titel
	  (\texttt{title}) und Jahr (\texttt{year}) bei Nutzung von \option{add\-year=true}.}

\subsection{Geänderte Befehle}
In dieser Liste werden die Befehle aufgeführt, die von \bl{} zur Verfügung
gestellt und von \bldw{} umdefiniert werden. Diese Befehle können mit 
\cmd{renewcommand*} angepasst werden.
  \befehl{labelnamepunct}{\cmd{addcolon}\cmd{space}}{Zeichen nach Namen im
    Literaturverzeichnis.}
  \befehl{nametitledelim}{\cmd{addcolon}\cmd{space}}{Zeichen zwischen Name und Titel in 
    Zitaten.}
  \befehl{newunitpunct}{\cmd{addcomma}\cmd{space}}{Zeichen zwischen einzelnen Elementen
	  im Literaturverzeichnis.}
  \befehl{subtitlepunct}{\cmd{addperiod}\cmd{space}}{Zeichen zwischen Titel und Untertitel.}

\subsection{Zusätzliche Begriffe \emph{(bibliography strings)}}
In dieser Liste werden die von \bldw{} zusätzlich definierten Begriffe 
aufgeführt. Es gibt sie jeweils in einer langen und einer abgekürzten Form. 
Welche Form verwendet wird, hängt von der \bl"=Option \option{abbreviate} ab. 
        
\begin{labeling}{mmmmmmm}
  \biblstring{idemdat}{demselben}{dems\cmd{adddot}}
  \biblstring{idemdatsf}{derselben}{ders\cmd{adddot}}
  \biblstring{idemdatsm}{demselben}{dems\cmd{adddot}}
  \biblstring{idemdatsn}{demselben}{dems\cmd{adddot}}
  \biblstring{idemdatpf}{denselben}{dens\cmd{adddot}}
  \biblstring{idemdatpm}{denselben}{dens\cmd{adddot}}
  \biblstring{idemdatpn}{denselben}{dens\cmd{adddot}}
  \biblstring{idemdatpp}{denselben}{dens\cmd{adddot}}
  \biblstring{inrefstring}{Artikel\cmd{addspace}}{Art\cmd{adddotspace}}
  \biblstring{review}{Rezension\cmd{addspace} zu}{Rez\cmd{adddotspace} zu}
\end{labeling}

Die \emph{bibliography strings} lassen sich folgendermaßen umdefinieren, wobei 
nicht zwischen einer langen und einer kurzen Form unterschieden werden kann:
\begin{lstlisting}
\DefineBibliographyStrings{german}{%
  idemdat = {idem},
  idemdatsf = {eadem}}
\end{lstlisting}

\subsection{Geänderte Begriffe \emph{(bibliography strings)}}
In dieser Liste werden die Begriffe aufgeführt, die von \bl{} zur Verfügung
gestellt und von \bldw{} umdefiniert werden. Ob die lange oder die abgekürzte
Variante verwendet wird, hängt von der \bl"=Option \option{abbreviate} ab. 

\begin{labeling}{mmmmmmm}
%  \biblstring{}{}{}
  \biblstring{seenote}{wie Anmerkung}{wie Anm\cmd{adddot}}
  \biblstring{reprint}{Nachdruck}{Ndr\cmd{adddot}}
  \biblstring{reprintof}{Nachdruck von}{\\Ndr\cmd{adddot}\cmd{addabthinspace} v\cmd{adddot}}
  \biblstring{reprintas}{Nachdruck unter dem Titel}{\\Ndr\cmd{adddotspace} u\cmd{adddot}\cmd{addabthinspace}\\%
                         d\cmd{adddot}\textbackslash{} Titel}
  \biblstring{byauthor}{von}{v\cmd{adddot}}
\end{labeling}
Auch die übrigen zusammengesetzten Abkürzungen (wie das in der obigen Liste aufgeführte \enquote{Ndr.\,v.}) 
verwenden den Befehl \cmd{add\-ab\-thin\-space}, also einen kurzen Abstand zwischen
den einzelnen Teilen der Abkürzung. Die genaue Bedeutung dieses Befehls wird
in der \bl"=Dokumentation erklärt.

\subsection{Eintragstypen}
Die folgenden Eintragstypen werden anders als in \bl{} verwendet.
  \eintragstyp{inreference}{Lexikon-Artikel}{incollection}
  \eintragstyp{review}{Rezension~-- auf das rezensierte Werk kann mit \texttt{xref}
    verwiesen werden}{article}
  
\subsection{Zusätzliche Feldformate}
In dieser Liste werden die von \bldw{} zusätzlich definierten Feldformate 
aufgeführt.
        
\feldformat{shorthandinbib}{mkbibbrackets\{\#1\}}{Formatierung der Siglen
  bei \option{shorthandinbib=true}.}

\section{Weitere Hinweise}
\label{weitere-anpassungen}
Die folgenden Hinweise verstehen sich als Anregung für fortgeschrittene Anwender
und beschreiben weitere Anpassungen, die von den Optionen des Pakets \bldw{}
nicht abgedeckt werden.

\subsection{Trennzeichen zwischen Namen}
Anders als \bl{} unterscheidet \bldw{} zwischen den Trennzeichen, die in 
Zitaten verwendet werden, und denen in der Bibliographie. Bei \bl{} gibt es 
lediglich \cmd{multinamedelim} (zwischen mehreren Autoren), 
\cmd{finalnamedelim} (vor dem letzten Autor) und \cmd{revsdnamedelim}
(zusätzliches Zeichen bei \enquote{Nachname, Vorname\textbar\ und Vorname2 
Nachname2}: dort wo \textbar\ steht, würde \cmd{revsdnamedelim} eingefügt
werden).

\bldw{} hingegen hat \cmd{bibmultinamedelim}, \cmd{bibfinalnamedelim} und
\cmd{bibrevsdnamedelim} für die Bibliographie sowie \cmd{citemultinamedelim}, 
\cmd{citefinalnamedelim} und \cmd{citerevsdnamedelim} für die Literaturverweise
im Text. Außerdem werden \cmd{multinamedelim}, \cmd{finalnamedelim} und
\cmd{revsdnamedelim} im Sigelverzeichnis \emph{(List of Shorthands)} verwendet.
Damit lassen sich unterschiedliche Darstellungsweisen festlegen. Die
Standarddefinitionen sehen folgendermaßen aus:

\begin{lstlisting}[commentstyle=]
\newcommand*{\multinamedelim}{\addcomma\space}
\newcommand*{\finalnamedelim}{%
  \ifnum\value{liststop}>2 \finalandcomma\fi
  \addspace\bibstring{and}\space}
\newcommand*{\revsdnamedelim}{}

\newcommand*{\bibmultinamedelim}{\addcomma\space}
\newcommand*{\bibfinalnamedelim}{%
  \ifnum\value{liststop}>2 \finalandcomma\fi
  \addspace\bibstring{and}\space}%
\newcommand*{\bibrevsdnamedelim}{\addspace}

\newcommand*{\citemultinamedelim}{\slash}
\newcommand*{\citefinalnamedelim}{\slash}
\newcommand*{\citerevsdnamedelim}{}
\end{lstlisting}
Damit wird bei mehreren Autoren im Text ein Schrägstrich (/) ausgegeben, im
Literaturverzeichnis und im Sigelverzeichnis dagegen Kommata bzw. (vor dem
letzten Namen) ein \enquote{und}. Die Definitionen für das Literatur- und das 
Sigelverzeichnis stimmen also mit den Standarddefinitionen von \bl{} überein.
Mit \cmd{renewcommand*} lassen sich diese Befehle nach eigenen Bedürfnissen
anpassen.

\subsection{Darstellung von Siglen (\cmd{mkbibacro})}
\label{mkbibacro-anpassen}
Wenn eine Sigle (\texttt{shorthand}) ein Akronym ist (z.\,B. \enquote{LMA} für
das \emph{Lexikon des Mittelalters}), kann man dem Eintrag 
\texttt{options\,=\,\{acronym= true\}} hinzufügen und die globale Option
\option{acronyms=true} nutzen. Dann wird die Sigle mit dem Befehl \cmd{mkbibacro}
gesetzt. Dasselbe gilt für Abkürzungen von Zeitschriften (z.\,B. \enquote{HZ} für
die \emph{Historische Zeitschrift}); hier benutzt man die Felder
\texttt{shortjournal\,=\,\{HZ\}} und \texttt{options\,=\,\{acronym=true\}}.

In \bl{} werden Akronyme (also Abkürzungen) wie z.\,B. \enquote{\textsc{url}} in 
Kapitälchen gesetzt. Dazu wird der Befehl \cmd{mkbibacro} benutzt, der standardmäßig
folgendermaßen definiert ist:
\begin{lstlisting}
\newcommand*{\mkbibacro}[1]{%
  \ifcsundef{\f@encoding/\f@family/\f@series/sc}
    {#1}
    {\textsc{\MakeLowercase{#1}}}}
\end{lstlisting}
Das bedeutet: Falls Kapitälchen in der verwendeten Schrift vorhanden sind, werden
Akronyme in Kapitälchen gesetzt, ansonsten normal. 

Hat man in einer Schrift keine Kapitälchen, kann man für
Akronyme Großbuchstaben verwenden, die leicht gesperrt werden und etwas kleiner
sind als gewöhnliche Großbuchstaben. Die Sperrung lässt sich (bei Verwendung von 
\paket{pdftex} bzw. \paket{pdf"|latex}) mit dem Paket \paket{microtype} einstellen. Die 
Verkleinerung lässt sich mit dem Paket \paket{scalefnt} erreichen. Somit könnte man
den Befehl \cmd{mkbibacro} folgendermaßen anpassen:
\begin{lstlisting}
\usepackage{scalefnt}
\usepackage{microtype}
\renewcommand{\mkbibacro}[1]{%
  \textls[55]{\scalefont{0.95}#1}\isdot}
\end{lstlisting}
Die Werte für \cmd{textls} und \cmd{scalefont} lassen sich natürlich den eigenen 
Vorstellungen oder Anforderungen anpassen.

\end{document}
