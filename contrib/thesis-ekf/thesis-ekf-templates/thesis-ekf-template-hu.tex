\documentclass[colorlinks]{thesis-ekf}
\usepackage[T1]{fontenc}
\usepackage[utf8]{inputenc}
\PassOptionsToPackage{defaults=hu-min}{magyar.ldf}
\usepackage[magyar]{babel}
\usepackage{graphicx,amsmath,amssymb,amsthm}
\graphicspath{{./figures/}} % A képfájlokat a [figures] mappába kell tenni!
\footnotestyle{rule=fourth}

\newtheorem{tetel}{Tétel}[chapter]
\newtheorem{lemma}[tetel]{Lemma}
\theoremstyle{definition}
\newtheorem{definicio}[tetel]{Definíció}
\newtheorem{feladat}[tetel]{Feladat}
\theoremstyle{remark}
\newtheorem{megjegyzes}[tetel]{Megjegyzés}
\newtheorem*{megoldas}{Megoldás}

\begin{document}
\logo{\includegraphics[width=9cm]{eke-logo}}
\institute{Matematikai és Informatikai Intézet}
\title{A szakdolgozat címe}
\author{Szerző neve\\ szak}
\supervisor{Tanár neve\\ beosztás}
\city{Eger}
\date{2017}
\maketitle
\tableofcontents

\chapter*{Bevezetés}
\chapter{Fejezet címe}
\section{Szakasz címe}
\subsection{Alszakasz címe}

\begin{thebibliography}{1}
\bibitem{cimke} \textsc{Szerző}: Cím, Kiadó, Hely, évszám.
\end{thebibliography}
\end{document}
