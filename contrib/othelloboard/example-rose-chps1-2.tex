\documentclass[a4paper,12pt]{book}

\setcounter{secnumdepth}{0} % Switches off numbering for sections a la original book.

%%% PAGE DIMENSIONS %%%
\usepackage{titlesec}
\titleformat{\chapter}[display]
{\normalfont\huge\bfseries}{\chaptertitlename\ \thechapter}{20pt}{\Huge}
% removes excessive "before" spacing (the second length argument) in chapter headings (default is 50pt)
\titlespacing*{\chapter}{0pt}{0pt}{40pt}

\usepackage[inner=2.5cm, outer=3.2cm, bottom=4cm]{geometry}
\usepackage{setspace}
%\usepackage[parfill]{parskip}    % Activate to begin paragraphs with an empty line rather than an indent
\addtolength{\belowcaptionskip}{-8pt} % Reduce some of the excessive space after figures

\usepackage[labelsep=newline, textfont=it, figurename=Diagram, justification=centering]{caption}
\newcommand{\jargon}[1]{\textbf{#1}}
\newcommand{\exercisenum}[1]{\vspace{8pt}\noindent\makebox[2.8cm][l]{\textbf{#1}}}
\newcommand{\puzzlenum}[1]{\vspace{8pt}\noindent\makebox[2.2cm][l]{\textbf{#1}}}
\newcommand{\glossitem}[1]{\vspace{8pt}\noindent\textbf{#1}\hspace{6pt}}
\renewcommand\thefigure{\arabic{chapter}-\arabic{figure}}

\usepackage{wrapfig}
%\usepackage{qtree}
%\usepackage{multirow}
%\usepackage{color}
%\usepackage{hyperref}
%\hypersetup{linktocpage}
%\definecolor{gray}{RGB}{211,211,211}

%%% OTHELLO DIAGRAMS %%%
\usepackage{othelloboard}
\let\Oldothellogrid\othellogrid
\renewcommand{\othellogrid}{\dotmarkings\Oldothellogrid}

% diagram sizes
\newcommand{\scalefactorthreeup}{0.64}
\newcommand{\scalefactortwoup}{0.64}
\newcommand{\scalefactorfourup}{0.48}

%\usepackage{appendix}

%%% TITLE %%%
% Title given on title page

%% DOCUMENT %%

\begin{document}
%\frontmatter

%\begin{titlepage}
%\begin{center}
%\vspace*{5cm}
%\Huge{\textbf{Othello}}\\[0.5cm]
%\LARGE{A Minute to Learn \ldots\ a Lifetime to Master}\\[1cm]
%\large{\textsc{Brian Rose}} 
%\end{center}
%\end{titlepage}

%\thispagestyle{empty}
%\null
%\vfill
%\begin{center}
%\noindent Copyright \copyright\ 2005 by Brian Rose\\

%\noindent\emph{Othello} and \emph{A Minute to Learn \ldots\ A lifetime to Master} are registered trademarks of Anjar Co., \copyright\ 1973, 2004 Anjar Co. All Rights Reserved.
%\end{center}

%\chapter{Acknowledgements}
%\tableofcontents

%\chapter{Introduction}

%\mainmatter

%\part{}
\chapter{Rules and notation}
Diagram 1-1 shows the standard notation for Othello. The columns are labeled
`a' through `h' from left to right, and the rows are labeled `1' through `8' from top to
bottom. In this book, squares will be referenced using a small letter followed by a
number, e.g., `a1' for the upper-left corner and `h8' for the lower-right corner. Certain
squares are assigned special letters, which will be capitalized, as shown in Diagram
1-2. This notation was developed by Othello's inventor, Goro Hasegawa, and remains
in use today. The B-squares are in the center of the edge, the C-squares are on
the edge next to the corner, and the A-squares lie between the B-squares and C-squares.
The X-squares are diagonally adjacent to the corners, with the `X' indicating
danger.
\begin{figure}[h]
\begin{center}
\begin{minipage}[t]{.32\textwidth}
\begin{othelloboard}{\scalefactorthreeup}
\annotationsfirstrow{a1}{b1}{c1}{d1}{e1}{f1}{g1}{h1}
\annotationssecondrow{a2}{b2}{c2}{d2}{e2}{f2}{g2}{h2}
\annotationsthirdrow{a3}{b3}{c3}{d3}{e3}{f3}{g3}{h3}
\annotationsfourthrow{a4}{b4}{c4}{d4}{e4}{f4}{g4}{h4}
\annotationsfifthrow{a5}{b5}{c5}{d5}{e5}{f5}{g5}{h5}
\annotationssixthrow{a6}{b6}{c6}{d6}{e6}{f6}{g6}{h6}
\annotationsseventhrow{a7}{b7}{c7}{d7}{e7}{f7}{g7}{h7}
\annotationseighthrow{a8}{b8}{c8}{d8}{e8}{f8}{g8}{h8}
\end{othelloboard}
\caption{}
\end{minipage}
\hfill
\begin{minipage}[t]{.32\textwidth}
\begin{othelloboard}{\scalefactorthreeup}
\annotationsfirstrow{}{C}{A}{B}{B}{A}{C}{}
\annotationssecondrow{C}{X}{}{}{}{}{X}{C}
\annotationsthirdrow{A}{}{}{}{}{}{}{A}
\annotationsfourthrow{B}{}{}{}{}{}{}{B}
\annotationsfifthrow{B}{}{}{}{}{}{}{B}
\annotationssixthrow{A}{}{}{}{}{}{}{A}
\annotationsseventhrow{C}{X}{}{}{}{}{X}{C}
\annotationseighthrow{}{C}{A}{B}{B}{A}{C}{}
\end{othelloboard}
\caption{Square names}
\end{minipage}
\hfill
\begin{minipage}[t]{.32\textwidth}
\begin{othelloboard}{\scalefactorthreeup}
\othelloarrayfourthrow{0}{0}{0}{1}{2}{0}{0}{0}
\othelloarrayfifthrow{0}{0}{0}{2}{1}{0}{0}{0}
\end{othelloboard}
\caption{Black to move}
\end{minipage}
\end{center}
\end{figure}

Black and White, written with capital letters, will refer to the players, while
lowercase letters (black and white) will refer to the color of the discs. For example:
``at the end of the game there were more black discs than white discs, so Black won
and White lost''. Black and White are referred to as ``he'', although they could of
course be ``she'', as many women play Othello, including Carol Jacobs, who won the
U.S. Othello Championship twice in a row.

Compass directions (north, south, east, west) are sometimes used to refer to
areas of the board (top, bottom, right, and left, respectively).

\section{Rules of the game}

\begin{enumerate}
\item The game begins with black discs on d5 and e4, and white discs on d4 and e5, as
shown in Diagram 1-3.

\item Players alternate taking turns, with Black moving first.

\item A legal move consists of placing a new disc on an empty square, and flipping
one or more of the opponent's discs.

\item Any of the opponent's pieces which are `sandwiched' between the disc just placed
on the board and a disc of the same color already on the board should be flipped.
Sandwiches can be formed vertically, horizontally, or diagonally. To form a sandwich,
all of the squares between the new disc and the disc of the same color
already on the board must be occupied by the opponent's pieces, with no blank
squares in between.

\item Pieces may be flipped in several directions on the same move. Any pieces which
are caught in a sandwich must be flipped; the player moving does not have the
right to choose to not flip a disc.

\item A new disc can not be played unless at least one of the opponent's discs is
flipped. If a player has no legal moves, that is, if no matter where the player
places a new disc he could not flip at least one disc, that player passes his turn,
and his opponent continues to make consecutive moves until a legal move becomes
available to that player.

\item If a player has at least one legal move available, he must make a move and may
not pass his turn.

\item The game continues until the board is completely filled or neither player has a
legal move.
\end{enumerate}

\section{Scoring}

Scoring is done at the end of game. The usual way to determine the score is to
simply count the number of discs of each color, e.g., if there are 34 black discs and 30
white discs, then Black wins 34--30. If both players have the same number of discs,
then the game is a draw.

In tournament play, if one player captures all of his opponents discs, the game is
usually scored as a 64--0 victory for that player, regardless of the number of discs on
the board. Further, in certain tournaments, such as the World Championship, empty
squares are awarded to the winner. For example, if at the end of the game there are 32
black discs and 29 white discs, with 3 empty squares, the score is recorded as a 35--29
victory for Black.

\section{Examples}

Diagrams 1-4 through 1-9 show a sequence of moves at the start of the game to
demonstrate the rules. In Diagram 1-4, Black makes the first move of the game to
f5, sandwiching the white disc on e5 between this new disc and the black disc on
d5. In the diagram, the numeral 1 on the disc on f5 indicates that this is where the
first move is played. The diamond-shaped black disc on e5 indicates that this disc
was white before the move, and was flipped as the result of Black's move. Below
the diagram, the phrase `White to move' indicates that White will make the next
move in the game. In Diagram 1-5, White plays to f6, sandwiching the disc on e5
diagonally using the existing white disc on d4. In Diagram 1-7, White plays to f4,
flipping discs in two directions. The black disc on f5 is sandwiched between the
new disc on f4 and the white disc on f6, while the black disc on e4 is sandwiched
between f4 and d4. In Diagram 1-9, White plays to c5, sandwiching the black discs
on d5 and e5 using the existing white disc on f5.
\begin{figure}[h]
\begin{center}
\begin{minipage}[t]{.32\textwidth}
\begin{othelloboard}{\scalefactorthreeup}
\dotmarkings
\othelloarrayfourthrow{0}{0}{0}{1}{2}{0}{0}{0}
\othelloarrayfifthrow{0}{0}{0}{2}{4}{2}{0}{0}
\annotationsfifthrow{}{}{}{}{}{1}{}{}
\end{othelloboard}
\caption{White to move}
\end{minipage}
\hfill
\begin{minipage}[t]{.32\textwidth}
\begin{othelloboard}{\scalefactorthreeup}
\othelloarrayfourthrow{0}{0}{0}{1}{2}{0}{0}{0}
\othelloarrayfifthrow{0}{0}{0}{2}{3}{2}{0}{0}
\othelloarraysixthrow{0}{0}{0}{0}{0}{1}{0}{0}
\annotationssixthrow{}{}{}{}{}{2}{}{}
\end{othelloboard}
\caption{Black to move}
\end{minipage}
\hfill
\begin{minipage}[t]{.32\textwidth}
\begin{othelloboard}{\scalefactorthreeup}
\othelloarrayfourthrow{0}{0}{0}{1}{2}{0}{0}{0}
\othelloarrayfifthrow{0}{0}{0}{2}{4}{2}{0}{0}
\othelloarraysixthrow{0}{0}{0}{0}{2}{1}{0}{0}
\annotationssixthrow{}{}{}{}{3}{}{}{}
\end{othelloboard}
\caption{White to move}
\end{minipage}
\end{center}
\end{figure}
 
\begin{figure}[h]
\begin{center}
\begin{minipage}[t]{.32\textwidth}
\begin{othelloboard}{\scalefactorthreeup}
\dotmarkings
\othelloarrayfourthrow{0}{0}{0}{1}{3}{1}{0}{0}
\othelloarrayfifthrow{0}{0}{0}{2}{2}{3}{0}{0}
\othelloarraysixthrow{0}{0}{0}{0}{2}{1}{0}{0}
\annotationsfourthrow{}{}{}{}{}{4}{}{}
\end{othelloboard}
\caption{Black to move}
\end{minipage}
\hfill
\begin{minipage}[t]{.32\textwidth}
\begin{othelloboard}{\scalefactorthreeup}
\othelloarraythirdrow{0}{0}{0}{0}{2}{0}{0}{0}
\othelloarrayfourthrow{0}{0}{0}{1}{4}{1}{0}{0}
\othelloarrayfifthrow{0}{0}{0}{2}{2}{1}{0}{0}
\othelloarraysixthrow{0}{0}{0}{0}{2}{1}{0}{0}
\annotationsthirdrow{}{}{}{}{5}{}{}{}
\end{othelloboard}
\caption{White to move}
\end{minipage}
\hfill
\begin{minipage}[t]{.32\textwidth}
\begin{othelloboard}{\scalefactorthreeup}
\othelloarraythirdrow{0}{0}{0}{0}{2}{0}{0}{0}
\othelloarrayfourthrow{0}{0}{0}{1}{2}{1}{0}{0}
\othelloarrayfifthrow{0}{0}{1}{3}{3}{1}{0}{0}
\othelloarraysixthrow{0}{0}{0}{0}{2}{1}{0}{0}
\annotationsfifthrow{}{}{6}{}{}{}{}{}
\end{othelloboard}
\caption{Black to move}
\end{minipage}
\end{center}
\end{figure}

Suppose that in the position shown in Diagram 1-10, Black moves to f8. Diagram
1-11 shows the correct position after this move. The white disc on e6 is completely
surrounded by black discs, but Black does not get to flip this disc, as it was not
sandwiched by the move to f8. Diagram 1-12 shows a position in which White does
not have a legal move. White passes, and Black moves again.

\begin{figure}[h!]
\begin{center}
\begin{minipage}[t]{.32\textwidth}
\begin{othelloboard}{\scalefactorthreeup}
\dotmarkings
\othelloarrayfirstrow{0}{0}{0}{2}{2}{2}{0}{0}
\othelloarraysecondrow	{0}{0}{0}{2}{2}{1}{0}{0}
\othelloarraythirdrow{0}{0}{0}{2}{2}{1}{0}{0}
\othelloarrayfourthrow{0}{0}{0}{2}{2}{1}{0}{0}
\othelloarrayfifthrow{0}{0}{0}{2}{2}{1}{0}{0}
\othelloarraysixthrow{0}{0}{0}{2}{1}{1}{0}{0}
\othelloarrayseventhrow{0}{0}{0}{2}{2}{1}{0}{0}
\othelloarrayeighthrow	{0}{0}{0}{0}{2}{0}{0}{0}
\end{othelloboard}
\caption{Black to move}
\end{minipage}
%\hfill
\begin{minipage}[t]{.32\textwidth}
\begin{othelloboard}{\scalefactorthreeup}
\dotmarkings
\othelloarrayfirstrow{0}{0}{0}{2}{2}{2}{0}{0}
\othelloarraysecondrow	{0}{0}{0}{2}{2}{4}{0}{0}
\othelloarraythirdrow{0}{0}{0}{2}{2}{4}{0}{0}
\othelloarrayfourthrow{0}{0}{0}{2}{2}{4}{0}{0}
\othelloarrayfifthrow{0}{0}{0}{2}{2}{4}{0}{0}
\othelloarraysixthrow{0}{0}{0}{2}{1}{4}{0}{0}
\othelloarrayseventhrow{0}{0}{0}{2}{2}{4}{0}{0}
\othelloarrayeighthrow	{0}{0}{0}{0}{2}{2}{0}{0}
\annotationseighthrow{}{}{}{}{}{1}{}{}
\end{othelloboard}
\caption{White to move}
\end{minipage}
\begin{minipage}[t]{.32\textwidth}
\begin{othelloboard}{\scalefactorthreeup}
\dotmarkings
\othelloarrayfirstrow{0}{0}{1}{2}{2}{2}{2}{2}
\othelloarraysecondrow	{0}{0}{1}{2}{2}{1}{2}{1}
\othelloarraythirdrow{0}{0}{1}{2}{2}{2}{2}{2}
\othelloarrayfourthrow{0}{0}{0}{1}{2}{2}{2}{2}
\othelloarrayfifthrow{0}{0}{1}{2}{2}{2}{2}{2}
\othelloarraysixthrow{2}{2}{2}{2}{2}{2}{2}{2}
\othelloarrayseventhrow{2}{2}{2}{2}{2}{2}{2}{2}
\othelloarrayeighthrow{2}{2}{2}{2}{2}{2}{2}{2}
\end{othelloboard}
\caption{White passes}
\end{minipage}
\end{center}
\end{figure}

\section{Playing through a transcript}

Diagram 1-13 shows an example of a transcript of a complete game. The numbers
indicate the order in which the moves were made, but not which pieces were
flipped. To replay the game, place a black disc on the square marked 1 (f5 in this
case), and flip pieces according to the normal rules of the game (e5 should be flipped
to black in this case). Continue by playing a move for White on the square marked 2,
a move for Black on the square marked 3, etc. Diagram 1-14 shows the position
created after move 30, while Diagram 1-15 shows the final position. Partial transcripts
are sometimes used to indicate a sequence of moves (see Diagram 2-9 for an
example).

\begin{figure}[h!]
\begin{center}
\begin{minipage}[t]{.32\textwidth}
\begin{othelloboard}{\scalefactorthreeup}
\dotmarkings
\othelloarrayfirstrow		{2}{1}{1}{2}{2}{1}{2}{1}
\othelloarraysecondrow	{1}{1}{2}{1}{1}{1}{2}{1}
\othelloarraythirdrow		{2}{1}{2}{1}{2}{2}{1}{2}
\othelloarrayfourthrow	{2}{1}{2}{1}{2}{1}{2}{1}
\othelloarrayfifthrow		{2}{1}{1}{2}{1}{2}{1}{2}
\othelloarraysixthrow	{1}{2}{2}{1}{2}{1}{2}{2}
\othelloarrayseventhrow	{2}{1}{2}{2}{1}{1}{2}{1}
\othelloarrayeighthrow	{1}{2}{2}{1}{1}{2}{2}{1}
\annotationsfirstrow	{49}	{44}	{38}	{39}	{33}	{40}	{59}	{60}
\annotationssecondrow	{46}	{48}	{31}	{42}	{10}	{12}	{47}	{52}
\annotationsthirdrow	{29}	{20}	{27}	{22}	{5}	{11}	{36}	{51}
\annotationsfourthrow	{45}	{34}	{7}	{}	{}	{4}	{13}	{32}
\annotationsfifthrow	{43}	{18}	{6}	{}	{}	{1}	{14}	{15}
\annotationssixthrow	{30}	{21}	{9}	{16}	{3}	{2}	{23}	{35}
\annotationsseventhrow{55}	{56}	{53}	{17}	{8}	{28}	{37}	{54}
\annotationseighthrow	{58}	{57}	{25}	{24}	{26}	{19}	{41}	{50}
\end{othelloboard}
\caption{Transcript}
\end{minipage}
\hfill
\begin{minipage}[t]{.32\textwidth}
\begin{othelloboard}{\scalefactorthreeup}
\dotmarkings
\othelloarrayfirstrow{0}{0}{0}{0}{0}{0}{0}{0}
\othelloarraysecondrow	{0}{0}{0}{0}{1}{1}{0}{0}
\othelloarraythirdrow{2}{2}{2}{1}{1}{1}{0}{0}
\othelloarrayfourthrow{0}{0}{2}{2}{2}{1}{2}{0}
\othelloarrayfifthrow{0}{1}{2}{1}{2}{1}{2}{2}
\othelloarraysixthrow{1}{3}{1}{1}{1}{1}{2}{0}
\othelloarrayseventhrow{0}{0}{0}{1}{1}{1}{0}{0}
\othelloarrayeighthrow	{0}{0}{2}{1}{1}{2}{0}{0}
\annotationssixthrow{30}{}{}{}{}{}{}{}
\end{othelloboard}
\caption{After move 30}
\end{minipage}
\hfill
\begin{minipage}[t]{.32\textwidth}
\begin{othelloboard}{\scalefactorthreeup}
\dotmarkings
\othelloarrayfirstrow{2}{2}{2}{2}{2}{2}{2}{1}
\othelloarraysecondrow	{2}{2}{1}{1}{1}{2}{1}{1}
\othelloarraythirdrow{2}{2}{2}{1}{2}{1}{2}{1}
\othelloarrayfourthrow{2}{2}{2}{2}{1}{1}{2}{1}
\othelloarrayfifthrow{2}{2}{2}{1}{2}{1}{2}{1}
\othelloarraysixthrow{2}{2}{1}{2}{2}{2}{1}{1}
\othelloarrayseventhrow{2}{1}{2}{1}{1}{1}{1}{1}
\othelloarrayeighthrow{1}{1}{1}{1}{1}{1}{1}{1}
\end{othelloboard}
\caption{Final position}
\end{minipage}
\end{center}
\end{figure}

\chapter{Corners and stable discs}
Perhaps the most basic strategy in Othello is to take the corners. By the rules of
play, it is impossible to flip a disc in a corner, so that if you are able to take a corner,
that disc will be yours for the rest of the game. In Diagram 2-1, the disc on h8 must be
white at the end of the game: even if Black later moves to both g8 and h7, he can not
capture the disc on h8. Moreover, once you have a corner, it is often possible to build
a large number of discs that are protected by the corner and can never be flipped.
Such discs are called \jargon{stable discs}.

\begin{figure}[h!]
\begin{center}
\begin{minipage}[t]{.32\textwidth}
\begin{othelloboard}{\scalefactorthreeup}
\dotmarkings
\drawboardfromstring{---------------------------OX------OOX-----OOO--------O--------O}
\end{othelloboard}
\caption{}
\end{minipage}
\hfill
\begin{minipage}[t]{.32\textwidth}
\begin{othelloboard}{\scalefactorthreeup}
\dotmarkings
\drawboardfromstring{--------------------O------OO------OOX-----OXX-----XOOO--OOOOOOO}
\end{othelloboard}
\caption{}
\end{minipage}
\hfill
\begin{minipage}[t]{.32\textwidth}
\begin{othelloboard}{\scalefactorthreeup}
\dotmarkings
\drawboardfromstring{---------------O---XXX-O---XXXXO---XXXOO---XXOOO--OOOOOO-OOOOOOO}
\end{othelloboard}
\caption{}
\end{minipage}
\end{center}
\end{figure}

In Diagram 2-2, the discs on the bottom row are stable discs, and in Diagram 2-3, 
all 21 white discs are stable discs. If this is not obvious to you, then take some time
now to convince yourself. Set up the positions on a board, then try to flip the stable
discs by placing black discs wherever you like. There is simply no way for Black to
get ``behind'' these discs to surround and flip them. The possibility of building up
stable discs usually makes corners very valuable, especially early in the game.

If taking corners is that good, then it should be obvious that you usually do not
want to give any to your opponent! Given the rules of the game, the only way for your
opponent to take a corner is if you play in one of the squares next to a corner, i.e., the
C-squares or X-squares. The X-squares are particularly dangerous, and a move to an
X-square early in the game is almost certain to give up the adjacent corner. For example, in Diagram 2-4, White has just moved to the X-square at g7. Although Black
can not take the h8 corner immediately, if he can establish even one disc on the c3-f6
diagonal, then Black will be able take the corner.

\begin{figure}[h]
\begin{center}
\begin{minipage}[t]{.35\textwidth}
\begin{othelloboard}{\scalefactortwoup}
\dotmarkings
\drawboardfromstring{-----------O-O----OOOOO---OOOOX---OOOOX--XXOXO----XOO3O---XXXXX-}
\posannotation{g7}{28}
\end{othelloboard}
\caption{Black to move}
\end{minipage}
\hspace{24pt}
%\hfill
\begin{minipage}[t]{.35\textwidth}
\begin{othelloboard}{\scalefactortwoup}
\dotmarkings
\drawboardfromstring{-----------O-O----OOOOO---OOOOX--X4444X--XXOXO----XOOOO---XXXXX-}
\posannotation{b5}{1}

\end{othelloboard}
\caption{White to move}
\end{minipage}
\end{center}
\end{figure}

One possibility is for Black to play b5, capturing the disc on e5, as shown in
Diagram 2-5. No matter where White plays, he will not be able to recapture the e5
disc, and Black will be able to take the h8 corner on his next turn. Once black has the
corner, all of his discs on row 8 become stable discs, and later in the game he is likely
to be able to create stable discs on the right edge as well. In general, the earlier in the
game a corner is taken the more valuable it is, as the potential for building up stable
discs around the corner is greater. In most cases, moving to an X-square early in the
game will prove to be a fatal error, although later in the book we will examine some
exceptional circumstances under which early X-square moves are useful.

\begin{figure}[h]
\begin{center}
\begin{minipage}[t]{.32\textwidth}
\begin{othelloboard}{\scalefactorthreeup}
\dotmarkings
\drawboardfromstring{---------------X--XXXO4O--XXX4O----X4OO---XXXO-------O----------}
\posannotation{h2}{1}
\end{othelloboard}
\caption{White to move}
\end{minipage}
\hfill
\begin{minipage}[t]{.32\textwidth}
\begin{othelloboard}{\scalefactorthreeup}
\dotmarkings
\drawboardfromstring{-----------XOO-X--XXOO4---XOX4XX--OXXXX----OXX------------------}
\posannotation{h2}{1}
\end{othelloboard}
\caption{White to move}
\end{minipage}
\hfill
\begin{minipage}[t]{.32\textwidth}
\begin{othelloboard}{\scalefactorthreeup}
\dotmarkings
\drawboardfromstring{-------------X-X----OX4----OOXO----OOXOO---OXX------OX----------}
\posannotation{h2}{1}
\end{othelloboard}
\caption{White to move}
\end{minipage}
\end{center}
\end{figure}

While moves to the X-square will usually allow the opponent to take the adjacent
corner, for C-squares the degree of danger depends largely on the rest of the
squares on the same edge. For example, in Diagrams 2-6, 2-7, and 2-8, Black will
quickly lose the h1 corner. In Diagram 2-6, White simply takes the corner on the next
move. In Diagram 2-7, White can play h3; Black has no way of capturing the h3 disc,
and White will be able to play h1 on his next turn. Can you see the way that White can capture the h1 corner in Diagram 2-8?

Starting from Diagram 2-8, White should play h3, gaining access to the h1 corner.
Even if Black captures the h3 disc by playing h4, as in Diagram 2-9, White still
has access to the corner, as shown in Diagram 2-10. As these diagrams suggest, C-squares are often the most dangerous when the adjacent A-square is empty, allowing
the opponent to attack the corner by playing into the A-square. We will see many
more examples like this in later chapters.

\begin{figure}[h]
\begin{center}
\begin{minipage}[t]{.35\textwidth}
\begin{othelloboard}{\scalefactortwoup}
\dotmarkings
\drawboardfromstring{-------------X-X----OXXO---OOXOX---OOXOO---OXX------OX----------}
\posannotation{h3}{1}
\posannotation{h4}{2}
\end{othelloboard}
\caption{}
\end{minipage}
\hspace{24pt}
%\hfill
\begin{minipage}[t]{.35\textwidth}
\begin{othelloboard}{\scalefactortwoup}
\dotmarkings
\drawboardfromstring{-------------X-X----OOXX---OOXXX---OOXXO---OXX------OX----------}
\end{othelloboard}
\caption{White to move}
\end{minipage}
\end{center}
\end{figure}

While there are many circumstances under which C-squares are bad moves, they are quite often perfectly good moves, and frequently they involve no danger of giving up a corner despite being adjacent to it. Diagrams 2-11, 2-12, and 2-13 all show examples where Black has a good C-square move at h2. In Diagram 2-11, h2 builds on Black's stable discs, and offers no prospect of white taking the h1 corner. In Diagram 2-12, Black must play h2 to prevent White from capturing the h8 corner. Once he does so, he is in no immediate danger of losing a corner. In Diagram 2-13, black can play h2 and later play another C-square at h7, all with no danger of losing a corner. As these diagrams suggest, the best time to take a C-square is often when you have pieces of your own color in the other squares along the edge.
\addtolength{\belowcaptionskip}{-8pt}
\begin{figure}[h]
\begin{center}
\begin{minipage}[t]{.32\textwidth}
\begin{othelloboard}{.6}
\dotmarkings
\othelloarrayfirstrow		{0}{0}{0}{0}{0}{0}{0}{0}
\othelloarraysecondrow	{0}{0}{0}{1}{0}{1}{0}{0}
\othelloarraythirdrow	{0}{0}{1}{1}{1}{1}{1}{2}
\othelloarrayfourthrow	{0}{1}{1}{1}{1}{1}{2}{2}
\othelloarrayfifthrow		{1}{1}{1}{2}{2}{2}{2}{2}
\othelloarraysixthrow	{0}{2}{2}{1}{2}{2}{1}{2}
\othelloarrayseventhrow	{0}{0}{2}{2}{2}{1}{2}{2}
\othelloarrayeighthrow	{0}{0}{2}{2}{2}{2}{2}{2}
\end{othelloboard}
\caption{Black to move}
\end{minipage}
\hfill
\begin{minipage}[t]{.32\textwidth}
\begin{othelloboard}{.6}
\dotmarkings
\othelloarrayfirstrow		{0}{0}{0}{0}{0}{0}{0}{0}
\othelloarraysecondrow	{0}{0}{0}{0}{0}{1}{0}{0}
\othelloarraythirdrow	{0}{0}{0}{1}{2}{1}{1}{1}
\othelloarrayfourthrow	{0}{0}{0}{2}{2}{2}{1}{2}
\othelloarrayfifthrow		{0}{0}{2}{2}{2}{2}{1}{2}
\othelloarraysixthrow	{0}{0}{2}{2}{2}{1}{2}{2}
\othelloarrayseventhrow	{0}{0}{0}{0}{0}{0}{0}{2}
\othelloarrayeighthrow	{0}{0}{0}{0}{0}{0}{0}{0}
\end{othelloboard}
\caption{Black to move}
\end{minipage}
\hfill
\begin{minipage}[t]{.32\textwidth}
\begin{othelloboard}{.6}
\dotmarkings
\othelloarrayfirstrow		{0}{0}{0}{0}{0}{0}{0}{0}
\othelloarraysecondrow	{0}{0}{0}{0}{0}{0}{0}{0}
\othelloarraythirdrow	{0}{0}{2}{2}{2}{1}{1}{1}
\othelloarrayfourthrow	{0}{0}{2}{2}{2}{2}{2}{2}
\othelloarrayfifthrow		{0}{0}{0}{2}{1}{1}{1}{2}
\othelloarraysixthrow	{0}{0}{0}{1}{1}{1}{1}{2}
\othelloarrayseventhrow	{0}{0}{0}{0}{1}{1}{0}{2}
\othelloarrayeighthrow	{0}{0}{0}{0}{0}{0}{0}{0}
\end{othelloboard}
\caption{Black to move}
\end{minipage}
\end{center}
\end{figure}

\clearpage
\section*{Exercises}
\addtolength{\belowcaptionskip}{8pt}
\captionsetup{figurename=Exercise}\setcounter{figure}{0}
In each diagram, find the best move. Answers begin on page \pageref{answersc2}.

\begin{figure}[h]
\begin{center}
\begin{minipage}[t]{.32\textwidth}
\begin{othelloboard}{\scalefactorthreeup}
\dotmarkings
\othelloarrayfirstrow		{0}{0}{0}{2}{0}{0}{2}{0}
\othelloarraysecondrow	{0}{0}{2}{2}{2}{2}{2}{0}
\othelloarraythirdrow	{2}{2}{2}{2}{2}{2}{2}{1}
\othelloarrayfourthrow	{2}{2}{2}{2}{2}{1}{2}{0}
\othelloarrayfifthrow		{2}{2}{2}{2}{1}{2}{2}{0}
\othelloarraysixthrow	{2}{2}{2}{2}{2}{2}{2}{0}
\othelloarrayseventhrow	{2}{0}{0}{0}{0}{0}{0}{0}
\othelloarrayeighthrow	{0}{0}{0}{0}{0}{0}{0}{0}
\end{othelloboard}
\caption{White to move}
\end{minipage}
\hfill
\begin{minipage}[t]{.32\textwidth}
\begin{othelloboard}{\scalefactorthreeup}
\dotmarkings
\othelloarrayfirstrow		{0}{1}{1}{1}{1}{1}{1}{0}
\othelloarraysecondrow	{0}{0}{1}{1}{1}{1}{1}{1}
\othelloarraythirdrow	{2}{1}{1}{1}{1}{1}{1}{1}
\othelloarrayfourthrow	{2}{2}{1}{1}{1}{1}{1}{1}
\othelloarrayfifthrow		{2}{2}{1}{1}{2}{1}{1}{1}
\othelloarraysixthrow	{2}{2}{1}{1}{1}{1}{1}{1}
\othelloarrayseventhrow	{2}{0}{2}{2}{1}{2}{0}{1}
\othelloarrayeighthrow	{0}{2}{2}{2}{2}{2}{2}{0}
\end{othelloboard}
\caption{Black to move}
\end{minipage}
\hfill
\begin{minipage}[t]{.32\textwidth}
\begin{othelloboard}{\scalefactorthreeup}
\dotmarkings
\othelloarrayfirstrow		{0}{0}{2}{2}{2}{2}{0}{0}
\othelloarraysecondrow	{0}{0}{2}{2}{2}{2}{0}{0}
\othelloarraythirdrow	{0}{0}{2}{2}{2}{2}{0}{0}
\othelloarrayfourthrow	{0}{2}{2}{1}{2}{2}{0}{0}
\othelloarrayfifthrow		{0}{2}{1}{2}{1}{2}{0}{0}
\othelloarraysixthrow	{0}{2}{2}{2}{2}{2}{0}{0}
\othelloarrayseventhrow	{0}{1}{1}{1}{1}{1}{0}{0}
\othelloarrayeighthrow	{1}{1}{1}{1}{1}{1}{1}{0}
\end{othelloboard}
\caption{White to move}
\end{minipage}
\end{center}
\end{figure}

\begin{figure}[h]
\begin{center}
\begin{minipage}[t]{.32\textwidth}
\begin{othelloboard}{\scalefactorthreeup}
\dotmarkings
\othelloarrayfirstrow		{0}{0}{0}{0}{0}{0}{0}{0}
\othelloarraysecondrow	{1}{1}{1}{1}{1}{1}{1}{1}
\othelloarraythirdrow	{1}{1}{1}{1}{1}{1}{1}{1}
\othelloarrayfourthrow	{1}{1}{1}{1}{1}{1}{1}{1}
\othelloarrayfifthrow		{1}{1}{1}{1}{1}{1}{1}{1}
\othelloarraysixthrow	{1}{1}{1}{1}{1}{1}{1}{1}
\othelloarrayseventhrow	{1}{1}{1}{1}{1}{1}{1}{1}
\othelloarrayeighthrow	{2}{2}{2}{2}{2}{2}{1}{0}
\end{othelloboard}
\caption{White to move}
\end{minipage}
\hfill
\begin{minipage}[t]{.32\textwidth}
\begin{othelloboard}{\scalefactorthreeup}
\dotmarkings
\othelloarrayfirstrow		{0}{1}{1}{1}{0}{1}{0}{0}
\othelloarraysecondrow	{0}{0}{2}{1}{1}{1}{0}{0}
\othelloarraythirdrow	{0}{0}{1}{2}{2}{0}{0}{0}
\othelloarrayfourthrow	{0}{0}{1}{1}{2}{2}{2}{0}
\othelloarrayfifthrow		{0}{0}{1}{1}{2}{2}{2}{0}
\othelloarraysixthrow	{0}{0}{1}{1}{2}{2}{0}{0}
\othelloarrayseventhrow	{0}{0}{0}{1}{2}{2}{0}{0}
\othelloarrayeighthrow	{0}{0}{0}{0}{0}{0}{0}{0}
\end{othelloboard}
\caption{Black to move}
\end{minipage}
\hfill
\begin{minipage}[t]{.32\textwidth}
\begin{othelloboard}{\scalefactorthreeup}
\dotmarkings
\othelloarrayfirstrow		{0}{0}{0}{0}{2}{0}{0}{0}
\othelloarraysecondrow	{0}{0}{0}{2}{2}{1}{0}{2}
\othelloarraythirdrow	{0}{0}{2}{2}{2}{1}{2}{2}
\othelloarrayfourthrow	{0}{0}{2}{2}{2}{2}{2}{0}
\othelloarrayfifthrow		{0}{0}{0}{2}{2}{2}{1}{0}
\othelloarraysixthrow	{0}{0}{0}{1}{2}{0}{0}{1}
\end{othelloboard}
\caption{White to move}
\end{minipage}
\end{center}
\end{figure}

\chapter{Frontier discs and walls}
\captionsetup{figurename=Diagram}

In chapter 2, we learned about the value of corners, and the danger of moving to
X-squares and C-squares. While knowing this alone might be enough to let you win
against a complete novice, it will not get you far against more seasoned competition.
In games between players that are both aware of the strategies presented in chapter 2,
neither player will voluntarily make the sort of bad X-square and C-squares moves
that give up corners for no reason. If you want your opponent to make these moves,
then you will have to force him to do so. That is, you want to create a situation where
the \emph{only} moves available to your opponent are bad moves. How to go about doing this
is the subject of this chapter, and indeed most of the rest of the book.

\begin{figure}[h]
\begin{center}
\begin{minipage}[t]{.32\textwidth}
\begin{othelloboard}{\scalefactorthreeup}
\dotmarkings
\othelloarraythirdrow	{0}{2}{2}{2}{2}{2}{1}{1}
\othelloarrayfourthrow	{2}{2}{2}{2}{2}{2}{1}{0}
\othelloarrayfifthrow		{0}{1}{2}{2}{1}{1}{1}{2}
\othelloarraysixthrow	{1}{1}{1}{1}{1}{1}{1}{0}
\othelloarrayseventhrow	{0}{0}{0}{0}{1}{0}{0}{0}
\end{othelloboard}
\caption{White to move}
\end{minipage}
\hfill
\begin{minipage}[t]{.32\textwidth}
\begin{othelloboard}{\scalefactorthreeup}
\dotmarkings
\othelloarraythirdrow		{1}{3}{3}{3}{3}{3}{1}{1}
\othelloarrayfourthrow		{2}{3}{2}{2}{2}{2}{1}{0}
\othelloarrayfifthrow		{0}{1}{3}{2}{1}{1}{1}{2}
\othelloarraysixthrow		{1}{1}{1}{1}{1}{1}{1}{0}
\othelloarrayseventhrow	{0}{0}{0}{0}{1}{0}{0}{0}
\posannotation{a3}{1}
\end{othelloboard}
\caption{Black to move}
\end{minipage}
\hfill
\begin{minipage}[t]{.32\textwidth}
\begin{othelloboard}{\scalefactorthreeup}
\dotmarkings
\othelloarraysecondrow	{2}{0}{0}{0}{0}{0}{0}{0}
\othelloarraythirdrow		{4}{4}{1}{1}{1}{1}{1}{1}
\othelloarrayfourthrow		{2}{1}{2}{2}{2}{2}{1}{0}
\othelloarrayfifthrow		{0}{1}{1}{2}{1}{1}{1}{2}
\othelloarraysixthrow		{1}{1}{1}{1}{1}{1}{1}{0}
\othelloarrayseventhrow	{0}{0}{0}{0}{1}{0}{0}{0}
\posannotation{a2}{2}
\end{othelloboard}
\caption{White to move}
\end{minipage}
\end{center}
\end{figure}

Diagram 3-1 shows the sort of position that often arises in games between an
expert (Black) and a novice (White). Many novices choose their moves mainly on the
basis of the number of discs that are flipped, with the more discs flipped the better.
After all, the object of the game is to end up with as many pieces as possible, so it
seems logical to want to take a lot of pieces at every point during the game. Following
this logic, the novice chooses to play a3, flipping 7 discs, as shown in Diagram 3-2.
The problem with this move becomes apparent after Black replies with a2, resulting
in the position shown in Diagram 3-3.

In Diagram 3-3, White's only legal option is the b2 X-square, which White is
obliged to play whether he wants to or not (Diagram 3-4). This immediately surrenders
the a1 corner (Diagram 3-5), and Black will eventually gain many \ldots

%% Answers %%

\chapter*{Answers to Exercises}
\label{answersc2}
\subsection*{Chapter 2}
\exercisenum{Exercise 2-1} White should play e1, capturing the disc at e4, which will provide access to the h2 corner.

\exercisenum{Exercise 2-2} Black should play a2. Although this is a C-square, there is no
danger of White gaining access to the a1 corner. Further, this move flips the
disc at d5, which will allow Black to take the h1 corner.

\exercisenum{Exercise 2-3} White should play a7, using the a8 corner to build more stable
discs.

\exercisenum{Exercise 2-4} This is an extreme example of building up stable discs. The
correct sequence of moves is shown in the diagram.

\begin{center}
\begin{othelloboard}{.8}
\othelloarrayfirstrow			{2}{2}{2}{2}{2}{2}{2}{2}
\othelloarraysecondrow	{1}{1}{1}{1}{1}{1}{1}{1}
\othelloarraythirdrow		{1}{1}{1}{1}{1}{1}{1}{1}
\othelloarrayfourthrow		{1}{1}{1}{1}{1}{1}{1}{1}
\othelloarrayfifthrow		{1}{1}{1}{1}{1}{1}{1}{1}
\othelloarraysixthrow		{1}{1}{1}{1}{1}{1}{1}{1}
\othelloarrayseventhrow	{1}{1}{1}{1}{1}{1}{1}{1}
\othelloarrayeighthrow		{2}{2}{2}{2}{2}{2}{1}{2}
\annotationsfirstrow			{1}{2}{3}{4}{5}{6}{8}{9}
\posannotation{h8}{7}
\end{othelloboard}
\end{center}

\exercisenum{Exercise 2-5} Black should fill in the hole at e1. Since White can not capture
this disc, Black will be able to take the a1 corner.

\exercisenum{Exercise 2-6} White should play h4, threatening to take the h1 corner. If Black
tries to prevent this by playing h5, the white disc at h6 still allows White to take
the corner.


\end{document}