%
% This is the file webeqtst.tex that is distributed with the AeB Bundle
%
\documentclass{article}
\usepackage{amsmath}
\usepackage{graphicx}
\usepackage[tight,designi]{web}  % dvipsone, dvips, pdftex, dvipdfm
\usepackage{exerquiz}

\usepackage[def=lmacs_aeb,js=lmacs_aeb]{lmacs}

\begin{document}

\maketitle

\tableofcontents


\section{Introduction}

The \textsf{lmacs} is designed to clean up the preamble of a source file.
For this file, we have
\begin{verbatim}
    \usepackage[def=lmacs_aeb,js=lmacs_aeb]{lmacs}
\end{verbatim}
The preamble definitions are in the file \texttt{lmacs\_aeb.def} and a
document JavaScript is imported with the file \texttt{lmacs\_aeb.js}.

\medskip\noindent We'll test the JavaScript first, press this button:
\pushButton[\CA{Press Me}\A{\JS{%
    makeAlert("Hooray for the lmacs package!")
}}]{alertBtn}{}{11bp}

\medskip\noindent The next section is taken from the file
\texttt{webeqtst.tex}. The problem environment is defined in the file
\texttt{lmacs\_aeb.def}, other definitions and customizations can be found
in that file.

\medskip\noindent Though I am using the \textsf{web} and \textsf{exerquiz} package, lmacs
does not require them; \textsf{lmacs} is a general purpose package for inputting
local definitions.

\section{Online Exercises}

A well-designed sequences of exercises can be of aid to the
student.  The \texttt{exercise} environment makes it easy to
produce electronic exercises.  By using the \texttt{forpaper}
option, you can also make a paper version of your exercises.

\begin{exercise}
Evaluate the integral \(\displaystyle\int x^2 e^{2x}\,dx\).
\begin{solution}
We evaluate by \texttt{integration by parts}:\normalsize
\begin{alignat*}{2}
 \int x^2 e^{2x}\,dx &
   = \tfrac12 x^2 e^{2x} - \int x e^{2x}\,dx &&\quad
           \text{$u=x^2$, $dv=e^{2x}\,dx$}\\&
   = \tfrac12 x^2 e^{2x} -
           \Bigl[\tfrac12 x e^{2x}-\int \tfrac12 e^{2x}\,dx\Bigr] &&\quad
           \text{integration by parts}\\&
   = \tfrac12 x^2 e^{2x} - \tfrac12 x e^{2x} + \tfrac12\int e^{2x}\,dx &&\quad
           \text{$u=x^2$, $dv=e^{2x}\,dx$}\\&
   = \tfrac12 x^2 e^{2x} - \tfrac12 x e^{2x} + \tfrac14 e^{2x} &&\quad
           \text{integration by parts}\\&
   = \tfrac14(2x^2-2x+1)e^{2x} &&\quad
           \text{simplify!}
\end{alignat*}
\end{solution}
\end{exercise}

In the preamble of this document, we defined a \texttt{problem}
environment with its own counter.  Here is an example of it.

\begin{problem}
Is $F(t)=\sin(t)$ an antiderivative of $f(x)=\cos(x)$?  Explain
your reasoning.
\begin{solution}
The answer is yes. The definition states that $F$ is an
antiderivative of $f$ if $F'(x)=f(x)$.  Note that
$$
       F(t)=\sin(t) \implies F'(t) = \cos(t)
$$
hence, $F(x) = \cos(x) = f(x)$.
\end{solution}
\end{problem}

\begin{problem}
Is $F(t)=\sin(t)$ an antiderivative of $f(x)=\cos(x)$?  Explain
your reasoning.
\begin{solution}
The answer is yes. The definition states that $F$ is an
antiderivative of $f$ if $F'(x)=f(x)$.  Note that
$$
       F(t)=\sin(t) \implies F'(t) = \cos(t)
$$
hence, $F(x) = \cos(x) = f(x)$.
\end{solution}
\end{problem}

\noindent By modifying the \texttt{exercise} environment, you can
also create an \texttt{example} environment.  The one defined in
the preamble of this document has no associated counter.

\begin{example}
Give an example of a set that is \textit{clopen}.
\begin{solution}
The real number line is both closed and open in the usual topology of the
real line.%
\end{solution}
\end{example}

There is a \texttt*-option with the \texttt{exercise} environment,
using it signals the presence of a multiple part exercise
question. The following exercise illustrates this option.

\begin{exercise}*\label{ex:parts}
Suppose a particle is moving along the $s$-axis, and that its position
at any time $t$ is given by $s=t^2 - 5t + 1$.
\begin{parts}
\item[h]\label{item:part} Find the velocity, $v$, of the particle at any time
$t$.
\begin{solution}
Velocity is the rate of change of position with respect to time. In
symbols:
$$
                    v = \frac{ds}{dt}
$$
For our problem, we have
$$
        v = \frac{ds}{dt} =\frac d{dt}(t^2 - 5t + 1) = 2t-5.
$$
The velocity at time $t$ is given by $\boxed{v=2t-5}$.
\end{solution}

\item Find the acceleration, $a$, of the particle at any time $t$.
\begin{solution}
Acceleration is the rate of change of velocity with respect to time.
Thus,
$$
                    a = \frac{dv}{dt}
$$
For our problem, we have
$$
        a = \frac{dv}{dt} =\frac d{dt}(2t-5)=2.
$$
The acceleration at time $t$ is constant: $\boxed{a=2}$.
\end{solution}
\end{parts}
\end{exercise}

References can be made to a particular part of an exercise; for example,
``see \hyperref[item:part]{Exercise~\ref*{ex:parts}(\ref*{item:part})}.''
Part (a) is in \textcolor{webblue}{blue}; the solutions for that part is
``hidden''.  This is a new option for the \texttt{exercise} environment.

There is now an option for listing multipart question in tabular form.
This problem style does not obey the \texttt{solutions\-after} option.

\begin{exercise}*
Simplify each of the following expressions in the complex number
system. \textit{Note}: $\bar z$ is the conjugate of $z$;
$\operatorname{Re} z$ is the real part of $z$ and
$\operatorname{Im} z$ is the imaginary part of $z$.
\begin{parts}[2]
\item $i^2$
\begin{solution} $i^2 = -1$ \end{solution}
&
\item $i^3$ \begin{solution} $i^3 = i i^2 = -i$\end{solution}
\\
\item $z+\bar z$
\begin{solution} $z+\bar z=\operatorname{Re} z$\end{solution}
&
\item[h] $1/z$
\begin{solution}
$\displaystyle\frac 1z=\frac 1z\frac{\bar z}{\bar z}=\frac z{z\bar z}=\frac z{|z|^2}$
\end{solution}
\end{parts}
\end{exercise}


\end{document}
